\begin{SCn}
	\scnheader{Подготовка специалистов в области Искусственного интеллекта}
	\scnexplanation{Итак, современного специалиста в области \textit{Искусственного интеллекта} необходимо учить:
		\begin{scnitemize}
			\item не только тому, как следует разрабатывать \textit{интеллектуальные компьютерные системы} с помощью имеющихся (существующих) \textit{методов} и \textit{средств}, т.е. с помощью имеющихся \textit{технологий};
			\item но и тому, как надо развивать (совершенствовать) имеющиеся \textit{технологии}.
		\end{scnitemize}
		\textit{Технология OSTIS} рассматривается нами не столько, как предлагаемая технология разработки \textit{интеллектуальных компьютерных систем}, а как предлагаемые \uline{принципы} построения технологии разработки \textit{интеллектуальных компьютерных систем} следующего поколения. Т.е. фактически мы предлагаем не саму технологию (\textit{Технологию OSTIS}), а участие в её создании и развитии, которое может привести даже к радикальным изменениям текущего состояния (текущей версии) этой \textit{технологии}. Это психологически снимает ощущение навязывания предлагаемой технологии и заменяет его на атмосферу партнерства, направленного на перманентную эволюцию указанной технологии. Такой подход создаст также условия для существенного повышения качества \textit{подготовки специалистов в области Искусственного интеллекта}, поскольку дает возможность осуществлять обучение путём непосредственного вовлечения студентов и магистрантов в реальные, практически значимые процессы разработки \textit{интеллектуальных компьютерных систем}, а также в процессы совершенствования (эволюции) соответствующих \textit{технологий}.
	}
	
	\scnheader{Методологические проблемы современного состояния работ в области Искусственного интеллекта}
	\scnrelfromvector{примечания}{\scnfileitem{Самые тяжелые кризисные ситуации в научно-технической сфере -- это те, которые носят фундаментальный и не совсем очевидный характер. Развитие кибернетики, информатики и искусственного интеллекта подтверждает это. За впечатляющими практическими и теоретическими достижениями незаметно возрастает огромный вал накладных расходов при разработке сложных больших систем -- возрастает дублирование, нестыковки, несогласованности.};
		\scnfileitem{Нет ничего более грустного, чем созерцать активную творческую деятельность большого количества умных людей, которые по инерции, не отдавая себе отчета, накапливают проблемы, препятствующие дальнейшему качественному развитию этой деятельности. Вместо того, чтобы "разгребать"{} эти проблемы на благо всем.};
		\scnfileitem{Как только мы начнем серьезно относиться к \uline{формальному} уточнению и согласованию всего многообразия понятий, используемых в области Искусственного интеллекта и различных его приложений, как только мы начнем \uline{реальную}(!) \uline{совместную} работу по общей комплексной формальной теории интеллектуальных компьютерных систем и по созданию комплексной технологии их проектирования, многие современные проблемы \textit{Искусственного интеллекта} начнут решаться. Нет ничего практичнее хорошей теории.};
		\scnfileitem{Основной лейтмотив развития технологий \textit{Искусственного интеллекта} -- это не только создание компьютерной технологии разработки сей совместной \textit{интеллектуальной компьютерной системы}, но и создание \uline{Метатехнологии} перманентной \uline{эволюции}(!) такой технологии. Иначе -- эклектика, усугубляющая современный кризис. Для создания эффективно и самостоятельно взаимодействующих \textit{интеллектуальных компьютерных систем} несущественных мелочей не бывает -- дьявол кроется в деталях и тонкостях. Важен не столько инжиниринг, сколько реинжиниринг интеллектуальных компьютерных систем и человеческой деятельности в целом.};
		\scnfileitem{Решение рассматриваемых кризисных проблем требует:
			\begin{scnitemize}
				\item Существенного фундаментального общесистемного переосмысления всего того, что мы творим;
				\item Осознания того, что кибернетика, информатика и искусственный интеллект -- это общая фундаментальная наука, требующая единого серьезного математического аппарата;
				\item Осознания того, что сейчас требуется не расширяемость многообразия точек зрения, а учиться их согласовывать, совершенствуя соответствующие методы.
		\end{scnitemize}};
		\scnfileitem{Нам необходимо переходить от автоматизации отдельных видов \textit{человеческой деятельности} к интегрированной автоматизации всего комплекса человеческой деятельности, к созданию и постоянной эволюции всей общечеловеческой \textit{экосистемы}, состоящей из самостоятельно взаимодействующих \textit{интеллектуальных компьютерных систем} как между собой, так и между людьми, автоматизацию деятельности которой они осуществляют. При этом надо помнить, что основные накладные расходы, основные проблемы, возникают на "стыках"{} при интеграции различных технических решений. Разработчик каждой подсистемы должен гарантировать отсутствие указанных накладных расходов.};
		\scnfileitem{Самое главное -- надо ориентироваться не на создание идеальной информационной \textit{экосистемы}, а на создание эффективной технологии, направленной на перманентную эволюцию(!) указанной экосистемы.};
		\scnfileitem{Уникальность современного кризиса в области кибернетики, информатики и искусственного интеллекта заключается в том, что, несмотря на глобальность этого кризиса, абсолютно реально создать локальный эпицентр по разрешению этого кризиса -- в частности, в Республике Беларусь. Для этого есть все предпосылки -- специальность \textit{Искусственный интеллект}, опыт разработки компьютерных систем, наука.}
	}
\end{SCn}
