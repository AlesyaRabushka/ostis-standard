\newpage

\scnheader{не следует путать}
\scnhaselementset{Алфавит символов, входящих в состав строковых идентификаторов\\
\scnaddlevel{1}
\scnexplanation{В рамках строковых идентификаторов sc-элементов рекомендуется использовать алфавит, состоящий из букв одного или нескольких естественных языков, дополненный при необходимости специальными символами (знаками препинания) из таблицы ASCII. Такое требование обусловлено удобством набора таких идентификаторов при помощи традиционной клавиатуры, а также удобством их программной обработки, и, в частности, визуализации.}
\scnsuperset{Алфавит символов, входящих в состав системных идентификаторов}
    \scnaddlevel{1}
        \scnexplanation{Алфавит символов, входящих в состав системных идентификаторов, включает в себя строчные и прописные буквы латинского алфавита, знак нижнего подчеркивания и знак дефиса. Такие ограничения обусловлены необходимостью использования некоторых системных идентификаторов sc-элементов непосредственно в коде программ sc-агентов, удобством их автоматической обработки (в том числе передачи по различным каналам связи), независимостью от используемой кодировки, независимостью от естественного языка, используемого при формировании основных идентификаторов sc-элементов и ряда других факторов.}
    \scnaddlevel{-1}
\scnaddlevel{-1}
;Алфавит символов, входящих в состав естественно-языковых файлов ostis-систем\\
\scnaddlevel{1}
\scnexplanation{В зависимости от типа естественно-языкового файла указанный алфавит может как совпадать с \textit{Алфавитом символов, входящих в состав строковых идентификаторов}, так и значительно расширять его, требуя в таком случае наличия специализированных средств для визуализации и редактирования файлов, использующих такой расширенный алфавит.}
\scnaddlevel{-1}
;Алфавит символов, входящих в состав sc.s-ограничителей и sc.s-разделителей, используемых при работе с обычным текстовым редактором\\
\scnaddlevel{1}
\scnexplanation{Для того, чтобы иметь возможность создавать sc.s-тексты (в частности, исходные тексты баз знаний) при помощи обычных текстовых редакторов, возникает необходимость иметь вариант изображения sc.s-ограничителей и sc.s-разделителей, при котором sc.s-ограничители и sc.s-разделители составляются только из специальных символов (знаков препинания), входящих в состав таблицы ASCII}
\scnaddlevel{-1}
;Алфавит символов, входящих в состав sc.s-ограничителей и sc.s-разделителей, используемых при визуализации и при работе со специализированным редактором sc.s-текстов\\
\scnaddlevel{1}
\scnexplanation{Указанный алфавит позволяет обеспечить компактность и наглядность визуализации sc.s-текстов (а следовательно и sc.n-текстов) и при необходимости может включать в себя любые символы, входящие в кодировки стандарта Unicode.}
\scnaddlevel{-1}
;Алфавит символов, входящих в состав исходных текстов баз знаний, записанных в SCs-коде\\
\scnaddlevel{1}
\scnexplanation{Указанный алфавит символов определяется тем, какие варианты алфавита выбраны для sc.s-ограничителей и sc.s-разделителей, строковых идентификаторов sc-элементов и  естественно-языковых файлов ostis-систем. В настоящее время используется следующая комбинация символов:
\begin{scnitemize}
\item при идентификации sc-элементов (и в sc.s-текстах, и в sc.g-текстах) используются только системные идентификаторы и соответствующим им алфавит\char58
\item в качестве \textit{Алфавита символов, входящих в состав естественно-языковых файлов ostis-систем} используется только \textit{Алфавит символов, входящих в состав строковых идентификаторов}\char58
\item используется \textit{Алфавит символов, входящих в состав sc.s-ограничителей и sc.s-разделителей,используемых при работе с обычным текстовым редактором}\char58
\item при необходимости в качестве содержимого файла (в том числе -- явно приписываемого идентификатора sc-элемента) может указываться содержимое внешнего файла, хранящегося на файловой системе (в том числе текстового файла, изображения, бинарного файла и т.д. Более подробно об этом см. \textit{Раздел. Правила работы с исходными текстами баз знаний ostis-систем})\char58
\end{scnitemize}}
\scnaddlevel{-1}
}
