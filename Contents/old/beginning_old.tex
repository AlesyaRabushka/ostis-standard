\scnheader{Документация Технологии OSTIS}
\scnidtf{Исходный текст базы знаний по Технологии OSTIS -- Open Sematic Technology for Intelligent Systems}

\scnaddlevel{1}
\scnsourcecommentpar{Здесь в квадратных скобках после разделителя ":=" указывается семантически эквивалентное, но неосновное имя описываемого объекта. В данном случае описываемым объектом является предлагаемая Документация Технологии OSTIS}
\scnaddlevel{-1}

\scniselement{научное издание}
\scniselement{монография}
\scntext{аннотация}{***}
\scnrelfromlist{автор}{Голенков В.В.;Гулякина Н.А.;Шункевич Д.В.}
\scnrelfrom{научный редактор}{***}
\scnrelfromlist{рецензент}{***;***}
\scnrelfrom{рекомендатор}{***}
\scnrelfrom{издательство}{***}
\scnrelfrom{финансовая поддержка}{***}
\scntext{УДК}{***}
\scntext{ББК}{***}
\scntext{ISBN}{***}
\scnrelfrom{технический редактор}{***}
\scnrelfrom{художественный редактор}{***}
\scnrelfrom{корректор}{***}
\scnrelfrom{верстка}{***}
\scntext{дата подписания в печать}{***}

\scnresetlevel
\scnsourcecommentpar{Здесь представлена декомпозиция Документации Технологии OSTIS на части с указанием последовательности этих частей}

\scnreltoset{конкатенация подразделов}{***}
\scnrelfrom{оглавление}{Оглавление Документации Технологии OSTIS}
\scnrelfrom{введение}{\nameref{chap_intro}}
\scnrelfrom{заключение}{\nameref{conclusion}}
\scnrelfrom{библиография}{\nameref{biblio}}
\scntext{аннотация}{Основной особенностью Технологии OSTIS является ориентация на использование компьютеров нового поколения, специально предназначенных для реализации семантически совместимых гибридных \textit{интеллектуальных компьютерных систем}. Само предлагаемое описание Технологии OSTIS оформлено в виде исходного текста базы знаний специальной интеллектуальной компьютерной метасистемы, которая построена по Технологии OSTIS и представляет собой постоянно совершенствуемый интеллектуальный портал научно-технических знаний, посвященный Технологии OSTIS.}

\scnsourcecommentpar{Обратите внимание на то, что в оглавлении номера разделов и номера их страниц оформлены как \uline{не\-транс\-ли\-ру\-е\-мые комментарии} к исходному тексту базы знаний, то есть комментарии, которые состав базы знаний не вводятся, в отличие от комментариев, которые вводятся в состав базы знаний, например, в виде содержимого файлов. Это обусловлено тем, что номера разделов и номера их страниц актуальны только для текущего состояния данного исходного текста базы знаний. База знаний эволюционирует независимо от вводимых в неё исходных текстов, которые в общем случае разрабатываются разными авторами и могут иметь разный объем. В результате такой эволюции появляются новые разделы базы знаний, некоторые разделы могут "переместиться".}
