\begin{SCn}

\scnsectionheader{Предметная область и онтология технологий искусственного интеллекта}

\scnstartsubstruct

\scnheader{Предметная область технологий искусственного интеллекта}
\scnsdmainclasssingle{технология искусственного интеллекта}
\scnsdclass{***}
\scnsdrelation{***}

\scnheader{технология искусственного интеллекта}
\scntext{текущее состояние}{До настоящего времени \textit{традиционные компьютерные технологии} и \textbf{\textit{технологии искусственного интеллекта}} развивались \textbf{\uline{независимо друг от друга}}.

Сейчас настало время \textbf{фундаментального переосмысления} опыта использования и эволюции \textit{традиционных компьютерных технологий} и их \textbf{интеграции} с \textit{технологиями искусственного интеллекта}. Это необходимо для устранения целого ряда недостатков современных компьютерных технологий.

Опыт использования компьютерных систем для автоматизации различных видов человеческой деятельности показывает, что автоматизация беспорядка приводит к еще большему беспорядку, а безграмотная автоматизация хуже ее отсутствия. При этом, если автоматизация требует применения методов и средств искусственного интеллекта, то последствия безграмотной автоматизации могут быть еще более разрушительны.

Это значит, что прежде, чем приступить к автоматизации какой-либо деятельности (и, тем более, с применением средств искусственного интеллекта), необходимо построить качественную формальную модель этой деятельности (т.е. достаточно детальное целостное ее описание, но без излишеств).}

\scnheader{технология искусственного интеллекта}
\scntext{направления эволюции}{Расширение областей применения компьютерных систем приводит к расширению многообразия автоматизируемых видов деятельности - управление предприятиями различного вида, управление организациями, управление сложными техническими системами, мультисенсорная интеграция и первичный анализ невербальной информации, распознавание, проектирование искусственных объектов различного вида, проектирование систем бизнес-процессов, направленных на воспроизводство спроектированных искусственных объектов, общение с пользователями (на естественных языках в текстовый и речевой форме, с помощью средств когнитивной графики), обучение пользователей, комплексное информационное обслуживание пользователей.

В свою очередь, расширение многообразия автоматизируемых видов деятельности приводит к расширению многообразия видов решаемых задач, видов методов и средств решения задач, видов используемой информации (видов знаний).

Так, например, повышение уровня автоматизации различных предприятий приводит к знание-ориентированной организации их деятельности, а в перспективе -- к знание-ори\-ен\-ти\-ро\-ван\-ной экономике. Это означает, что основой автоматизации предприятия становятся средства управления знаниями. 

Из этого, в свою очередь, следует, что в перспективных системах управления предприятиями необходимо переходить от баз данных, обеспечивающих представление достаточно простых (фактографических) видов знаний, к базам знаний, в состав которых могут входить знания самого различного вида.}

\scnheader{технология искусственного интеллекта}
\scnevolutiondirections{К числу современных наиболее активно развиваемых направлений развития теории интеллектуальных компьютерных систем и технологий искусственного интеллекта можно отнести:

\begin{scnitemize}
    \item управление знаниями и онтологический инжиниринг \cite{Gavrilova2016}, Semantic Web~\cite{W3C};
    \item формальные логики (четкие, нечеткие, дедуктивные, индуктивные, абдуктивные, дескриптивные, темпоральные, пространственные и т.д.);
    \item искусственные нейросети, байесовские сети, генетические алгоритмы (Machine learning в узком смысле);
    \item компьютерная лингвистика (Natural Language Processing, NLP), семантический анализ текстов естественного языка;
    \item speech processing, семантический анализ речевых сообщений;
    \item image processing – техническое зрение, семантический анализ изображений; 
    \item многоагентные системы, коллективы интеллектуальных систем \cite{Wooldridge2009, Tarasov2002, Yarushkina2007};
    \item гибридные интеллектуальные системы, синергетические интеллектуальные системы \cite{Kolesnikov2001}.
\end{scnitemize}
}

\scnheader{технология искусственного интеллекта}
\scnevolutionproblems{Несмотря на наличие серьезных \textbf{научных результатов} в области искусственного интеллекта, темпы \textbf{развития рынка интеллектуальных систем} не столь впечатляющи.

Причин тому несколько:
\begin{scnitemize}
    \item имеет место большой разрыв между научными исследованиями в области искусственного интеллекта и созданием качественных технологий разработки интеллектуальных систем. Научные исследования в области искусственного интеллекта в основном сконцентрированы на разработку новых методов решения интеллектуальных задач;
    \item указанные исследования разрозненны и не осознана необходимость их интеграции и создания общей формальной теории интеллектуальных систем, т. е. имеет место "вавилонское столпотворение"\ различных моделей, методов и средств, используемых в искусственном интеллекте при отсутствии осознания проблемы обеспечения их совместимости. Без решения этой проблемы не может быть создана ни общая теория интеллектуальных систем, ни, следовательно, комплексная технология разработки интеллектуальных систем, доступная инженерам и \textbf{экспертам};
    \item указанная интеграция моделей и методов искусственного интеллекта весьма сложна, т. к. носит междисциплинарный характер;
    \item интеллектуальные системы как объекты проектирования имеют значительно более высокий уровень сложности по сравнению со всеми техническими системами, которыми человечество имело дело;
    \item как следствие вышесказанного, имеет место большой разрыв между научными исследованиями и инженерной практикой в этой области. Заполнить этот разрыв можно только путем создания эволюционируемой технологии разработки интеллектуальных систем, развитие которой осуществляется путем активного сотрудничества ученых и инженеров;
    \item качество разработки прикладных интеллектуальных систем в большой степени зависит от взаимопонимания экспертов и инженеров знаний. Инженеры знаний, не владея тонкостями прикладной области, могут вносить серьезные ошибки в разрабатываемые базы знаний. Посредничество инженеров знаний между экспертами и разрабатываемой базой знаний существенно снижает качество разрабатываемых интеллектуальных систем. Для решения этой проблемы необходимо, чтобы \textit{язык представления знаний} в базе знаний был удобен не только \textit{интеллектуальной системе} и \textit{инженерам знаний}, \textbf{но и \textit{экспертам}}.
    
\end{scnitemize}

Текущее состояние технологий искусственного интеллекта можно охарактеризовать следующим образом:
\begin{scnitemize}
    \item Есть большой набор частных технологий искусственного интеллекта с соответствующими инструментальными средствами, но отсутствует общая теория интеллектуальных систем и, как следствие, отсутствует общая комплексная технология проектирования интеллектуальных систем  (см. конференции «Artificial General Intelligence»~\cite{AGI2018});
    \item Совместимость частных технологий искусственного интеллекта практически не осуществляется и более того, отсутствует осознание такой необходимости.
\end{scnitemize}


Развитие технологий искусственного интеллекта существенным образом затрудняется следующими социально-методологическими обстоятельствами:
\begin{scnitemize}
    \item Высокий социальный интерес к результатам работ в области искусственного интеллекта и большая сложность этой науки порождает поверхностность и нечистоплотность при разработке и рекламе различных приложений. Серьезная наука перемешивается с безответственным маркетингом, понятийной и терминологической неряшливостью и безграмотностью, вбрасыванием новых абсолютно ненужных эффектных терминов, запутывающих суть дела, но создающих иллюзию принципиальной новизны.
    \item Междисциплинарный характер исследований в области искусственного интеллекта существенно затрудняет эти исследования, т.к. работа на стыках научных дисциплин требует высокой культуры и квалификации.
\end{scnitemize}}

\scnaddlevel{1}
\scntext{предлагаемый подход к решению}{Для решения указанных выше проблем развития \textbf{\textit{технологий искусственного интеллекта}}:
\begin{scnitemize}
    \item Продолжая разрабатывать новые формальные модели решения интеллектуальных задач и совершенствовать существующие модели (логические, нейросетевые, продукционные), необходимо обеспечить совместимость этих моделей как между собой, так и с традиционными моделями решения задач, не попавших в число интеллектуальных задач. Другими словами, речь идет о разработке принципов организации гибридных интеллектуальных систем, обеспечивающих решение \textbf{комплексных задач}, требующих совместного использования и в непредсказуемых комбинациях самых различных видов знаний и самых различных моделей решения задач.
    \item Необходим переход от эклектичного построения сложных интеллектуальных систем, использующих различные виды знаний и различные виды моделей решения задач, к глубокой их интеграции, когда одинаковые модели представления и модели обработки знаний реализуется в разных системах и подсистемах одинаково. 
    \item Необходимо сократить дистанцию между современным уровнем теории интеллектуальных систем и практики их разработки. 
    \item Необходимо существенно повысить уровень согласованности действий лиц, участвующих в процессе постоянного совершенствования баз знаний.
    \item Надо, чтобы в решении этой проблемы совместимости интеллектуальных систем активно участвовали сами системы, а не только их разработчики. Системы должны сами заботиться о поддержке своей совместимости с другими системами в условиях активного изменения этих систем с помощью механизма автоматизированного согласования используемых понятий между интеллектуальными системами.
\end{scnitemize}}
\scnaddlevel{-1}
\scnendstruct

\end{SCn}