
\scsubsection{Введение в язык структурированного представления баз знаний ostis-систем}
\label{intro_scn}

\begin{SCn}

\scnsectionheader{\currentname}

\scnstartsubstruct

\bigskip
\scnfilelong{SCn-код, прежде всего, рассматривается нами как расширение SCs-кода, позволяющее перейти от чисто линейных текстов SCs-кода к форматированным и фактически двухмерным текстам, в которых появляется декомпозиция исходного линейного текста SCs-кода на строчки, размещенные по вертикали. При этом, каждая последующая строчка располагается под предыдущей, причем первый символ каждой строчки размещается строго под первым символом предыдущей строчки и над первым символом последующей строчки. Таким образом, начало всех строчек текста фиксировано, что дает возможность использовать это при форматировании sc.n-текста.

Фрагмент исходного текста базы знаний ostis-системы, представленный в виде sc.n-текста, представляет собой последовательность максимальных предложений различного вида.

Максимальным предложением будем называть предложение, не являющееся частью другого предложения. В отличие от этого, предложения, входящие в состав других предложений, будем называть встроенными предложениями, признаком которых является то, что они являются частью последовательности sc.s-предложений или sc.n-предложений, которая ограничена фигурными скобками.

Важнейшим видом предложений sc.n-текста (как максимальных, так и встроенных) являются предложения, построенные путем форматирования sc.s-предложений (такие предложения будем называть sc.n-предложениями).

В свою очередь, основным видом sc.n-предложений являются спецификации различных сущностей. В таких sc.n-предложениях, которые будем называть sc.n-предложениями-спецификациями, внешний идентификатор (имя) специфицируемой сущности (который может быть как атомарным, так и неатомарным) выделяется жирным курсивом.

Каждое максимальное sc.n-предложение-спецификация, как и любое другое максимальное предложение sc.n-текста, начинается с первого символа новой строчки.

Начальными символами строчек sc.n-предложений могут быть пробелы (пустые символы). Пробелами могут быть также и конечные символы строчек sc.n-предложения. Максимальное количество символов в строчках sc.n-предложений для каждого sc.n-текста фиксировано и определяется конкретным вариантом размещения sc.n-текста.

Перечислим основные правила преобразования sc.s-предложений в sc.n-предложения:
\begin{scnitemize}
\item sc.s-коннектор может размещаться на следующей строчке под предшествующим sc.s-идентификатором, начиная с того же символа следующей строчки, что и указанный sc.s-идентификатор;
\item если sc.s-идентификатор переносится на следующую строчку, то его продолжение на следующей строчке осуществляется с таким же отступом от начала строчки, с каким указанный sc.s-идентификатор начинается;
\item перечисление sc.s-идентификаторов, разделенных точкой с запятой, может осуществляться не "в строчку", а "в столбик" при размещении каждого следующего sc.s-идентификатора строго под предшествующим. При этом, разделительная точка с запятой может быть заменена "мячиком", который помещается \uline{перед} каждым перечисляемым sc.s-идентификатором;
\item Закрывающая фигурная или квадратная скобка может быть размещена строго \uline{под} соответствующей открывающей скобкой.
\item sc.s-идентификатор в sc.n-предложении может быть связан с другими sc.s-идентификаторами через несколько разных sc.s-коннекторов. При этом, каждый из этих sc.s-коннекторов размещается строго под предшествующим, но только после того, когда будет завершена запись всей, в общем случае разветвленной, цепочки sc.s-коннекторов и sc.s-идентификаторов, которая начинается с предшествующего sc.s-коннектора. В SCs-коде аналога таким предложениям с неограниченной возможностью описания “разветвленных” связей sc.s-идентификаторов нет. Следовательно, если в sc.s-тексте sc.s-идентификатор может быть инцидентен не более, чем двум sc.s-коннекторам (слева и справа от него), то в sc.n-тексте sc.s-идентификатор может быть дополнительно инцидентен неограниченному числу (причем, не обязательно одинаковых) sc.s-коннекторов, которые размещаются “по вертикали” строго под ним.
\end{scnitemize}

Более подробно и с примерами указанные отличия sc.n-предложений от sc.s-предложений рассмотрены в разделе \nameref{section_scn} \scnsourcecomment{раздел \ref{section_scn}}.

\textbf{\textit{SCn-код}} не только является расширением SCs-кода в направлении форматирования sc.s-предложений, но и является результатом объединения и SCs-кода, и указанной расширенной модификации SCs-кода, и SCg-кода. Последнее означает, что каждый текст SC.n-кода (sc.n-текст) в общем случае представляет собой последовательность предложений, каждое из которых может быть либо максимальным sc.s-предложением, либо максимальным sc.n-предложением, либо максимальным sc.g-предложением. Это дает возможность при построении исходного текста базы знаний ostis-системы недостатки одного из предлагаемых вариантов формального представления исходных текстов баз знаний (в виде sc.s-предложений, sc.n-предложений, sc.g-предложений) компенсировать достоинствами других вариантов. Каждое предложение sc.n-текста отделяется от следующего за ним предложения любого вида пустой строчкой (строчкой пробелов).

Каждое максимальное sc.g-предложение, входящее в состав sc.n-текста ограничивается (сверху и снизу) специальным ограничителем, который будем называть "усеченным" sc.n-контуром и который отличается от sc.n-контура тем, что верхняя и нижняя вертикальные линии в нем проводятся с начала строчки и до ее середины.

Кроме sc.n-ограничителя максимального sc.g-текста в sc.n-текстах для структурирования их атомарных разделов используется sc.n-разделитель содержательно целостных частей. Оформляется такой sc.n-разделитель в виде нетранслируемого комментария, содержащего заголовок соответствующей части атомарного раздела базы знаний, который оформляется жирным курсивом увеличенного размера с увеличенными расстояниями между символами.
}

\scnheader{sc.n-текст}
\scnidtf{текст SCn-кода}
\scnidtf{последовательность предложений SCn-кода}
\scnidtf{последовательность предложений SCn-кода, каждое из которых не является частью какого-либо другого предложения из \uline{этой} последовательности}

\scnheader{максимальное предложение sc.n-текста*}
\scniselement{бинарное ориентированное отношение}
\scnsubdividing{максимальное sc.s-предложение*;максимальное sc.n-предложение*;
максимальное sc.g-предложение*}
\scnnote{Понятие максимального предложения является относительным. О максимальности предложения можно говорить только по отношению к заданному исходному тексту базы знаний, в состав которого это предложение входит. Но, поскольку, для каждого предложения sc.n-текста всегда известен (задан) тот sc.n-текст, в состав которого это предложение входит, предложение sc.n-текста не может одновременно быть немаксимальным (встроенным) и максимальным. Это означает, что наряду с относительным понятием \textit{максимального предложения*} можно ввести и абсолютное понятие максимального предложения как второй домен указанного отношения.}

\scnheader{максимальное предложение sc.n-текста}
\scnrelto{второй домен}{максимальное предложение sc.n-текста*}
\scnreltoset{разбиение}{максимальное sc.s-предложение;максимальное sc.n-предложение;
максимальное sc.g-предложение}

\scnheader{sc.n-рамка}
\scnidtf{ограничитель изображения файла ostis-системы, используемый в sc.n-предложениях и отличающийся от sc.g-рамки необязательностью полного изображения вертикальных линий (только в начале и в конце) и тем, что вертикальные и горизонтальные линии в sc.n-рамке строятся из символов, составляющих Алфавит SCs-кода.}

\scnheader{sc.n-контур}
\scnidtf{используемый в sc.n-предложениях ограничитель, отличающийся отличающийся от sc.g-контура необязательностью полного изображения вертикальных линий, тем, что вертикальные и горизонтальные линии в sc.n-контуре строятся из символов, составляющих Алфавит SCs-кода, и тем, что sc.n-контур может ограничивать изображение информационной конструкции любого вида (ея-текста, SCg, SCs, SCn), трактуемой как изображение семантически эквивалентного sc-текста.}

\scnheader{"усеченный"\ sc.n-контур}
\scnsubdividing{sc.n-ограничитель максимального sc.g-текста в рамках sc.n-текста;ограничитель транслируемого фрагмента ея-текста}
\scnidtf{ограничитель фрагмента содержимого ея-файла, который предназначен не только для хранения в рамках этого файла, но и для трансляции в SC-код и последующего погружения в базу знаний ostis-систем.}

\scnendstruct

\scnsourcecommentpar{Это закрывающий ограничитель раздела \ref{intro_scn} \textit{\nameref{intro_scn}}}

\end{SCn}
