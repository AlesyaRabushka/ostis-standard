\begin{SCn}


\scseparatedfragment{Оглавление Стандарта OSTIS}

\begin{SCn}

\scnsectionheader{\currentname}
\scnidtf{Оглавление текущей версии Стандарта OSTIS}
\scnidtfexp{Иерархический перечень названий разделов, входящих в состав \textit{Стандарта OSTIS}, с дополнительной спецификацией некоторых разделов, указывающей альтернативные названия разделов, а также их авторов и редакторов}
\scnaddlevel{1}
\scnnote{Существенно подчеркнуть, что иерархия разделов \textit{Стандарта OSTIS} как и \textit{разделов} любой другой \textit{базы знаний} не означает то, что \textit{разделы} более низкого уровня иерархии входят в состав (являются частями) соответствующих разделов более высокого уровня. Связь между \textit{разделами} разных уровней иерархии означает то, что \textit{раздел} более низкого уровня иерархии является \textit{дочерним} разделом по отношению к соответствующему \textit{разделу} более высокого уровня, т.е. \textit{разделом}, который наследует свойства указанного \textit{раздела} более высокого уровня.}
\scnaddlevel{-1}
\scnnote{Описание логико-семантических связей каждого раздела \textit{Стандарта OSTIS} с другими разделами \textit{Стандарта OSTIS} приводится в рамках Титульной спецификации каждого раздела.}
\scnnote{Названия тех разделов, которые планируется написать в последующих изданиях \textit{Стандарта OSTIS} или разделов, которые \uline{в данном} издании \textit{Стандарта OSTIS} не печатаются, поскольку содержание их не изменилось по сравнению с предыдущим \uline{явно указываемым} изданием \textit{Стандарта OSTIS}, выделяются также \uline{жирным курсивом}, но для них страницы в рамках данного издания Стандарта OSTIS не указываются}
\scnrelfrom{примечание}{\scnstartsetlocal
	
	\bigskip
	\scnfilelong{\textit{Стандарт OSTIS}}
	\scnrelto{сокращение}{\scnfilelong{Стандарт \textit{Технологии OSTIS}}}
	\scnaddlevel{1}
	\scnrelto{сокращение}{\scnfilelong{Standart of Open Semantic Technology for Intelligent Systems}}
	\scnaddlevel{-1}
	\scnendstruct}
\scneqhierstruct


\end{SCn}

\normalsize 

\begingroup
\let\clearpage\relax
\tableofcontents
\endgroup

\begin{SCn}

\scnendhierstruct \scninlinesourcecommentpar{Завершили \textit{Оглавление Стандарта OSTIS}}

\end{SCn}

\newpage

\scseparatedfragment[\scnmonographychapter{Глава 7.3. Метасистема OSTIS и Стандарт OSTIS}]{Титульная спецификация Стандарта OSTIS}

\begin{SCn}

\scnsectionheader{\currentname}

\scnstartsubstruct

\scnrelto{титульная спецификация}{Стандарт OSTIS}
\scnaddlevel{1}
	\scnrelfrom{оглавление}{Оглавление Стандарта OSTIS}
	\scnrelfrom{общая структура}{Общая Структура Стандарта OSTIS}
	\scnrelfrom{система ключевых знаков}{Система ключевых знаков Стандарта OSTIS}
	\scnrelfrom{редакционная коллегия}{Редакционная коллегия Стандарта OSTIS}
	\scnrelfrom{авторский коллектив}{Авторский коллектив Стандарта OSTIS}
	\scnrelfrom{направления развития}{Направления развития Стандарта OSTIS}
	\scnrelfrom{правила построения}{Правила построения Стандарта OSTIS}
	\scnaddlevel{1}
		\scnidtf{правила построения*(Стандарт OSTIS)}
		\scnaddlevel{1}
			\scniselement{sc-выражение}
		\scnaddlevel{-1}
	\scnaddlevel{-1}
	\scnrelfrom{правила организации развития}{Правила организации развития Стандарта OSTIS}
	\scnaddlevel{1}	
		\scnrelfromset{декомпозиция}{Правила организации развития исходного текста Стандарта OSTIS;Правила организации развития Стандарта OSTIS на уровне его внутреннего представления в памяти Метасистемы IMS.ostis}
\scnaddlevel{-2}

\scnauthorcomment{Вычитать то, что дальше}		

\scnfilelong{В титульную спецификацию \textit{Стандарта OSTIS} должны быть включены ссылки на все разделы и фрагменты этих разделов, где описываются правила построения и оформления всех видов информационных конструкций, входящих в состав \textit{Стандарта OSTIS} (внешних идентификаторов знаков, входящих в состав \textit{Стандарта OSTIS}, спецификаций различного вида cущностей, описываемых в \textit{Стандарте OSTIS})\\
В \textit{баз знаний ostis-систем} задаются правила унифицированного построения (представления, оформления) следующих видов \textit{информационных конструкций}:
\begin{scnitemize}
	\item\textit{sc-идентификаторов} -- внешних идентификаторов \textit{sc-элементов} следующих классов:
	\begin{itemize}
		\item\textit{sc-элементов} (имеются в виду общие правила идентификации любых sc-элементов) -- смотрите в разделе ``\nameref{intro_idtf}''
		\item\textit{sc-переменных, sc-констант}
		\item знаков материальных сущностей
		\begin{itemize}
			\item знаков персон
			\item знаков библиографических источников 
		\end{itemize}
		\item знаков множеств 
		\begin{itemize}
			\item классов, понятий
			\begin{itemize}
				\item отношений
				\item параметров
				\item структур
				\begin{itemize}
					\item знаний
				\end{itemize}
			\end{itemize}
		\end{itemize}
		\item знаков файлов ostis-систем
		\item знаков sc-знаний баз знаний 
	\end{itemize}
	\item\textit{sc-конструкций}
	\item\textit{sc.g-конструкций}
	\item\textit{sc.s-конструкций}
	\item\textit{sc.n-конструкций}
	\item базовых правил \textit{sc-спецификаций:}
	\begin{itemize}
		\item понятий 
		\item разделов баз знаний (титульные спецификации разделов)
		\item файлов ostis-систем
		\item библ. источников
		\item предметных областей 	
	\end{itemize}
	\item специализированная \textit{sc-спецификация}
	\begin{itemize}
		\item информационная конструкция
		\begin{itemize}
			\item оглавление
			\item система ключевых знаков	
		\end{itemize}
	\end{itemize}
	\begin{itemize}
		\item понятий
		\begin{itemize}
			\item пояснение
			\item определения
			\item теоретико-множественная окрестность
			\item семейство утверждений	
		\end{itemize}
	\end{itemize}
	\begin{itemize}
		\item сегментов баз знаний (титульная спецификация)
		\item семейство разделов баз знаний	
	\end{itemize}
\end{scnitemize}}

\scnheader{Стандарт OSTIS-2021}
\scnidtf{Издание Документации Технологии OSTIS-2021}
\scnidtf{Первое издание (публикация) Внешнего представления Документации Технологии OSTIS в виде книги}
\scniselement{публикация}
\scnaddlevel{1}
\scnidtf{библиографический источник}
\scnaddlevel{-1}
\scniselement{официальная версия Стандарта OSTIS}
\scniselement{бумажное издание}
\scniselement{научное издание}
\scnrelfrom{рекомендация издания}{Совет БГУИР}
\scnrelfromset{рецензенты}{Курбацкий А.Н.; Дудкин А.А.}
\scnrelfrom{издательство}{Бестпринт}
\scniselement{\scnstartsetlocal\scnendstructlocal}
\scnaddlevel{1}
\scniselement{УДК}
\scnaddlevel{1}
\scniselement{параметр}
\scnaddlevel{-1}
\scnrelfrom{Индекс УДК}{004.8}
\scnaddlevel{-1}
\scnidtftext{ISBN}{978-985-7267-13-2}

\scnheader{Стандарт OSTIS-2022}
\scnidtf{Издание Документации Технологии OSTIS-2022}
\scnidtf{Второе издание (публикация) Внешнего представления Документации Технологии OSTIS в виде книги}
\scniselement{публикация}
\scniselement{официальная версия Стандарта OSTIS}
\scniselement{бумажное издание}
\scniselement{научное издание}

\bigskip
\scnendstruct \scninlinesourcecommentpar{Завершили Титульную спецификацию \textit{Стандарта OSTIS}}

\end{SCn}

\newpage

\scseparatedfragment{Титульная спецификация второго издания Стандарта OSTIS}

\begin{SCn}
	
\scnsectionheader{\currentname}
	
\scnstartsubstruct


\scnendstruct

\end{SCn}

\newpage

\scnsectionheader{Авторский коллектив}

\begin{SCn}
	
\scnheader{соавтор Стандарта OSTIS}
\scnidtf{\uline{член Авторского Коллектива Стандарта OSTIS}, направленного на развитие Технологии OSTIS и описание каждой текущей версии этой Технологии в виде соответствующего Стандарта}
\scnrelfromset{направления и принципы организации деятельности}{
	\scnfileitem{Отслеживать и читать новые публикации по тематике, рассматриваемой в Стандарте OSTIS.
		Близкими источниками для этого являются:
		\begin{scnitemize}
			\item выпуски журналов:
			\begin{scnitemizeii}
				\item "Онтология проектирования"
			\end{scnitemizeii}
			\item материалы конференций:
			\begin{scnitemizeii}
				\item КИИ, Консорциум W3C
			\end{scnitemizeii}
			\item публикации, ключевыми терминами которых являются:
			\begin{scnitemizeii}
				\item формальная онтология
				\item онтология верхнего уровня
				\item семантическая сеть
				\item граф знаний
				\item графовая база данных
				\item смысловое представление знаний
				\item конвергенция в ИИ
			\end{scnitemizeii}
			\item стандарты
		\end{scnitemize}
	};
\scnfileitem{Фиксировать результаты знакомства с новыми публикациями по тематике, близкой Стандарту 	OSTIS, в Библиографии OSTIS, а также в основном тексте Стандарта OSTIS в виде соответствующих 		ссылок, цитат, сравнительного анализа};
\scnfileitem{Отслеживать текущее состояние всего текста Стандарта OSTIS, формировать предложения, направленные на развитие Стандарта OSTIS и на повышение темпов этого развития. Активно участвовать в обсуждении проблем развития Технологии OSTIS};
\scnfileitem{Максимально возможным образом увязывать персональную работу над Стандартом OSTIS с другими формами деятельности - научной, учебной, прикладной};
\scnfileitem{Указывать авторство своих предложений по дополнению и/или корректировке текущего текста Стандарта OSTIS};
\scnfileitem{Участвовать в рецензировании и согласовании предложений, представленных другими авторами Стандарта OSTIS}
}


\scnheader{Организация работ по развитию Стандарта OSTIS}
\scneqtoset{
	\scnfileitem{Всем прочитать текст текущей версии Стандарта OSTIS и написать конструктивные замечания к доработке};
	\scnfileitem{Каждому уточнить свой персональный план участия в работе над второй версией (с учетом своих интересов и диссертационной тематики)}
}




\scnheader{План работ по подготовке \textit{Стандарта OSTIS-2022}}
\scneqtoset{
	\scnfileitem{Распределить разделы по авторам (для всех сотрудников)};
	\scnfileitem{Составить четкий план доработки каждого раздела};
	\scnfileitem{Организовать регулярные семинары по обсуждению развития каждого раздела}
}




\scnheader{Орг-План каждого члена авторского коллектива}
\scneqtoset{
	\scnfileitem{Прочитать весь текст текущей версии Стандарта OSTIS};
	\scnfileitem{Сформировать свое мнение о текущих недостатках и направлениях развития Стандарта (осознать свою ответственность как соавтора)};
	\scnfileitem{Из оглавления последующей версии Стандарта определить те разделы, в развитии которых вы готовы активно участвовать};
	\scnfileitem{По согласованию утвердить распределение авторов по разделам и определить \uline{иерархическую} структуру локальных рабочих коллективов}
		\newline
		\scnaddlevel{1}
		\scnnote{По каждому рабочему коллективу определить перечень вопросов, требующих обсуждения и согласования}
		\scnaddlevel{-1}
}
	
\end{SCn}

\newpage



\bigskip
\scnfragmentcaption

\scnheader{Пояснения к оглавлению Стандарта OSTIS и к некоторым разделам этого Стандарта}

\scnstartsubstruct

\scnheader{Спецификация второго издания Стандарта OSTIS}
\scnidtf{Спецификация второй официальной версии Стандарта OSTIS}
\scnidtf{Спецификация Стандарта OSTIS-2022}

\scnheader{Анализ методологических проблем современного состояния работ в области Искусственного интеллекта}
\scnidtf{Актуальность Технологии OSTIS}
\scnidtf{Современные требования, предъявляемые к деятельности в области Искусственного интеллекта  к интеллектуальным компьютерным системам следующего поколения -- конвергенция, глубокая ("бесшовная"{}) интеграция, высокий уровень обучаемости (гибкости, стратифицированности, рефлексивности), высокий уровень социализации (взаимопонимания, договороспособности, способности координировать свои действия с другими субъектами), стандартизация}

\scnheader{Введение в описание внутреннего языка ostis-систем}
\scnidtf{Введение в SC-code (Semantic Computer Code)}

\scnheader{Предметная область и онтология внешних идентификаторов знаков, входящих в информационные конструкции внутреннего языка ostis-систем}
\scnidtf{Предметная область и онтология sc-идентификаторов}

\scnheader{Введение в описание языка графического представления информационных конструкций, хранимых в памяти ostis-систем}
\scnidtf{Введение в SCg-code (Semantic Code graphical)}

\scnheader{Введение в описание языка линейного представления информационных конструкций, хранимых в памяти ostis-систем}
\scnidtf{Введение в SCs-code (Semantic Code string)}

\scnheader{Введение в описание языка форматирования линейного представления информационных конструкций, хранимых в памяти ostis-систем}
\scnidtf{Введение в SCn-code (Semantic Code natural)}

\scnheader{Предметная область и онтология кибернетических систем}
\scnidtf{Предпосылки создания компьютерных систем нового поколения}

\scnheader{Предметная область и онтология компьютерных систем}
\scnidtf{Этапы эволюции (повышения качества) компьютерных систем -- эволюции памяти, информации, хранимой в памяти, решателей задач, интерфейсов}

\scnheader{Предметная область и онтология интеллектуальных компьютерных систем}
\scnidtf{Этапы эволюции (повышения качества) интеллектуальных компьютерных систем и проблемы дальнейшей их эволюции}

\scnheader{Предметная область и онтология технологий автоматизации различных видов человеческой деятельности}
\scnidtf{Эволюция технологий проектирования, производства и эксплуатации компьютерных систем и предпосылки создания компьютерных технологий нового поколения}

\scnheader{Предметная область и онтология логико-семантических моделей компьютерных систем, основанных на смысловом представлении информации}
\scnidtf{Предлагаемый подход к построению интеллектуальных компьютерных систем следующего поколения}

\scnheader{Предметная область и онтология внутреннего языка ostis-систем}
\scnidtf{Предметная область и онтология SC-кода (Semantic Computer Code)}
\scnrelfrom{введение}{\textit{\nameref{intro_sc_code}}}

\scnheader{Предметная область и онтология  базовой денотационной семантики SC-кода}
\scniselement{\textit{предметная область и онтология верхнего уровня}}


\scnheader{Предметная область и онтология языка графического представления информационных конструкций, хранимых в памяти ostis-систем}
\scnidtf{Предметная область и онтология SCg-кода (Semantic Code graphical)}
\scnaddhind{1}
\scnrelfrom{введение}{\textit{\nameref{intro_scg}}}
\scnresetlevel

\scnheader{Предметная область и онтология языка линейного представления информационных конструкций, хранимых в памяти ostis-систем}
\scnidtf{Предметная область и онтология SCs-кода (Semantic Code string)}
\scnaddhind{1}
\scnrelfrom{введение}{\textit{\nameref{intro_scs}}}
\scnresetlevel

\scnheader{Предметная область и онтология языка форматирования линейного представления информационных конструкций, хранимых в памяти ostis-систем}
\scnidtf{Предметная область и онтология SCn-кода (Semantic Code natural)}
\scnaddhind{1}
\scnrelfrom{введение}{\textit{\nameref{intro_scn}}}
\scnresetlevel

\scnheader{Предметная область и онтология файлов, внешних информационных конструкций и внешних языков ostis-систем}
\scnrelto{дочерний раздел}{\nameref{intro_lang}}

\scnheader{Предметная область и онтология операционной семантики sc-языка вопросов}
\scnidtf{Предметная область информационно-поисковых действий и агентов, а также соответствующая онтология методов}

\scnheader{Предметная область и онтология операционной семантики логических sc-языков}
\scnidtf{Предметная область и онтология логических исчислений}
\scnidtf{Предметная область и онтология действий и агентов логического вывода, а также соответствующая онтология методов (правил) логического вывода}

\scnheader{Предметная область и онтология sc-языков программирования высокого уровня}
\scnidtf{Предметная область и онтология sc-языков программирования высокого и сверхвысокого уровня, ориентированных на обработку баз знаний ostis-систем}

\scnheader{Предметная область и онтология операционной семантики sc-моделей искусственных нейронных сетей}
\scnidtf{Предметная область и онтология процессов функционирования sc-моделей искусственных нейронных сетей при обработке баз знаний ostis-систем}

\scnheader{Логико-семантическая модель средств автоматизации управления взаимодействием разработчиков различных категорий в процессе проектирования базы знаний ostis-системы}
\scnidtf{Логико-семантическая модель средств автоматизации управления взаимодействием менеджеров, авторов, рецензентов, экспертов и редакторов в процессе проектирования базы знаний ostis-системы}

\scnheader{Предметная область и онтология встроенных ostis-систем поддержки эксплуатации соответствующих ostis-систем конечными пользователями}
\scnidtf{Интеллектуальные \textit{встроенные ostis-системы}, обучающие \textit{конечных пользователей} эффективной эксплуатаии тех \textit{ostis-систем}, в состав которых они входят}
\scnidtf{Предметная область и онтология методов и средств реализации целенаправленного и персонифицированного обучения пользователей каждой ostis-системы}

\scnheader{Предметная область и онтология Экосистемы OSTIS}
\scnidtf{Проект smart-общества}

\scnheader{Логико-семантическая модель Метасистемы IMS.ostis}
\scnrelfrom{примечание}{\scnstartsetlocal

	\bigskip
	\scnfilelong{IMS.ostis}
	\scnrelto{сокращение}{\scnfilelong{Метасистема IMS.ostis}}
	\scnaddlevel{1}
	\scnrelto{сокращение}{\scnfilelong{Intelligent MetaSystem of Open Semantic Technology for Intelligent Systems}}
	\scnaddlevel{-1}
	\scnendstruct}
\scnidtf{Логико-семантическая модель интеллектуального ostis-портала научно-технических знаний по Технологии OSTIS}

\scnendstruct

\end{SCn}

\newpage