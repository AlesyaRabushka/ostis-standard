\begin{SCn}

\scsuperchapter

\scnsectionheader{Стандарт OSTIS}
\label{super_char}
\scnstartsubstruct

\scnidtf{Документация Технологии OSTIS}
\scnidtf{Документация открытой технологии онтологического проектирования, производства и эксплуатации семантически совместимых гибридных интеллектуальных компьютерных систем}
\scnidtf{Описание \textit{Технологии OSTIS} (Open Semantic Technology for Intelligent Systems), представленное в виде раздела \textit{базы знаний ostis-системы} на внутреннем языке \textit{ostis-систем} и обладающее достаточной полнотой для использования этой технологии разработчиками \textit{интеллектуальных компьютерных систем}}
\scnidtf{Полное описание текущего состояния \textit{Технологии OSTIS}, представленное в виде раздела \textit{базы знаний}, построенной по \textit{Технологии OSTIS}}
\scnidtf{Основной раздел \textit{базы знаний} \scnbigspace \textit{Метасистемы IMS.ostis}, которая предназначена для комплексной поддержки онтологического проектирования семантически совместимых \textit{гибридных интеллектуальных компьютерных систем}}
\scniselement{раздел базы знаний}
    \scnaddlevel{1}
    \scnidtf{раздел внутреннего представления \textit{базы знаний ostis-системы} -- \textit{интеллектуальной компьютерной системы}, построенной по \textit{Технологии OSTIS}}
    \scnaddlevel{-1}
%\scnrelfromset{основные авторы}{Голенков В.В.;Гулякина Н.А.;Шункевич Д.В.}
%\scnrelfrom{научный редактор}{Голенков В.В.}
\scnrelfromset{рецензенты}{Курбацкий А.Н.;Дудкин А.А.}
\scnrelfrom{финансовая поддержка}{Intelligent Semantic Systems Ltd.}
\scnaddlevel{1}
\scnaddlevel{-1}
\scnreltovector{конкатенация подразделов}{Вводный раздел Документации Технологии OSTIS;Обоснование Технологии OSTIS;Предметная область и онтология Технологии OSTIS;Заключительная часть Документации Технологии OSTIS;Библиографическая часть Документации Технологии OSTIS}

\scnheader{Стандарт OSTIS}
\scntext{эпиграф}{From data science to knowledge science}
\scntext{аннотация}{В настоящее время информатика преодолевает важнейший этап своего развития --- переход от информатики данных (data science) к информатике знаний (knowledge science), где акцентируется внимание на \uline{семантических} аспектах представления и обработки \textit{знаний}.\\
Без фундаментального анализа такого перехода невозможно решить многие проблемы, связанные с управлением \textit{знаниями}, экономикой \textit{знаний}, с \textit{семантической совместимостью интеллектуальных компьютерных систем}.\\
Основной особенностью \textit{Технологии OSTIS} является ориентация на использование компьютеров нового поколения, специально предназначенных для  реализации семантически совместимых гибридных \textit{интеллектуальных компьютерных систем}. Предлагаемая \textit{Документация Технологии OSTIS} оформлена в виде \textit{раздела базы знаний} специальной интеллектуальной компьютерной \textit{Метасистемы IMS.ostis} (Intelligent MetaSystem for ostis-systems), которая построена по Технологии OSTIS и представляет собой постоянно совершенствуемый интеллектуальный \textit{портал научно-технических знаний}, который поддерживает перманентную эволюцию \textit{Документации Технологии OSTIS}, а также разработку различных \textit{ostis-систем} (интеллектуальных компьютерных систем, построенных по \textit{Технологии OSTIS}).}
\scnidtf{Процесс перманентной эволюции \textit{Стандарта OSTIS}, совмещенного (интегрированного) с комплексными учебно-методическим обеспечением подготовки специалистов в области Искусственного интеллекта и представленного в виде специального раздела базы знаний}
\scnnote{Подчеркнем, что \textit{Стандарт OSTIS} -- это не описание некоторого состояния \textit{Технологии OSTIS}, а \uline{динамическая} информационная модель процесса эволюции этой технологии}
\scnidtf{Стандарт Технологии OSTIS}
\scnidtf{Документация \textit{Технологии OSTIS}, полностью отражающая \uline{текущее} состояние \textit{Технологии OSTIS} и представленная соответствующим \textit{разделом базы знаний} специальной \textit{ostis-системы}, которая ориентирована на поддержку проектирования, производства, эксплуатации и эволюции (реинжиниринга) \textit{ostis-систем}, а также на поддержку эволюции самой \textit{Технологии OSTIS} и которая названа нами \textit{Метасистемой IMS.ostis}}
\scnidtf{Максимальный раздел \textit{Стандарта OSTIS}, т.е. раздел, в состав которого входят все остальные \textit{разделы} (подразделы) \textit{Стандарта OSTIS}}
\scnidtf{Раздел базы знаний, текущее состояние которого отражает текущее состояние (текущую версию) перманентно эволюционируемого Стандарта Комплексной Технологии OSTIS}
\scnidtf{Представленное в форме раздела базы знаний специальной ostis-системы (Метасистемы IMS.ostis) полное описание (спецификация, документация) текущего состояния Технологии OSTIS}
\scnidtf{Мы рассматриваем \textit{Стандарт OSTIS} (Документацию Стандарта Технологии OSTIS) как продукт \textit{научно-технической деятельности}, к которому предъявляются \uline{высокие требования} по полноте, согласованности, непротиворечивости, практической значимости разрабатываемой документации, описывающей текущее состояние \textit{Технологии OSTIS}}

\scnheader{официальная версия Стандарта OSTIS}
\scnidtf{официально издаваемая (публикуемая в бумажном и/или электронном виде) \textit{версия Стандарта OSTIS}}
\scnhaselement{Стандарт OSTIS-2021}

\scnheader{Стандарт OSTIS}
\scnrelfrom{авторский коллектив}{Авторский коллектив Стандарта OSTIS}
\scnaddlevel{1}
\scnnote{Работоспособность, квалификация и результативность \textit{Авторского коллектива Стандарта OSTIS} определяют темпы и качество эволюции \textit{Стандарта OSTIS}}
\scnnote{При подготовке к изданию каждой официально фиксируемой версии \textit{Стандарта OSTIS}, как правило, будут формироваться специальные группы из общего \textit{Авторского коллектива Стандарта OSTIS}, каждая из которых ориентируется на подготовку к изданию конкретной соответствующей версии \textit{Стандарта OSTIS}}
\scnaddlevel{-1}
\scnheader{автор Стандарта OSTIS}
\scnidtf{соавтор Стандарта OSTIS}
\scnidtf{член Авторского коллектива Стандарта OSTIS}
\scnnote{соавтором \textit{Стандарта OSTIS} может быть любой желающий, способный согласовывать свою персональную инициативную деятельность и свою точку зрения с другими соавторами \textit{Стандарта OSTIS}}

\scnheader{Стандарт OSTIS}
\scnrelfrom{редакционная коллегия}{Редакционная коллегия Стандарта OSTIS}
\scnaddlevel{1}
\scnidtf{редколлегия стандарта OSTIS}
\scnidtf{Редакционная коллегия, обеспечивающая развитие \textit{Стандарта OSTIS}, совмещенного (интегрированного) с комплексным учебно-методическим обеспечением \textit{подготовки специалистов в области Искусственного интеллекта}.}
\scntext{несёт ответственность за}{Корректность (непротиворечивость), системность, целостность, полноту всех разрабатываемых материалов \textit{Стандарта OSTIS} и, в том числе, \textit{учебно-методического обеспечения подготовки специалистов в области Искусственного интеллекта}.}
\scnidtf{Рабочий орган, обеспечивающий организацию коллективного творческого процесса по развитию \textit{Стандарта OSTIS}, совмещенного с учебно-методическим обеспечением подготовки соответствующих специалистов.}
\scnidtf{Редакционная коллегия, несущая ответственность за качество перманентно эволюционируемого \textit{Стандарта OSTIS}.}
\scnidtf{Редакционная коллегия, осуществляющая координацию деятельности авторов разработки очередной (следующей) версии \textit{Стандарта OSTIS} к следующей \textit{конференции OSTIS}.}
\scnnote{В частности, \textit{Редколлегия Стандарта OSTIS} может осуществлять распределение работ по построению следующей версии \textit{Стандарта OSTIS} с четкой привязкой ответственных авторов \textit{Стандарта OSTIS} к соответствующим разделам \textit{Стандарта OSTIS}.}
\scnidtf{Очень важная структура, определяющая научно-технический уровень, авторитет и репутацию всей \textit{Технологии OSTIS} в глазах международной научно-технической общественности.}
\scnnote{Каждый член \textit{Редакционной коллегии Стандарта OSTIS} должен быть достаточно активным членом \textit{Авторского коллектива Стандарта OSTIS}. Все члены \textit{Редакционной коллегии Стандарта OSTIS} должны иметь научную степень не ниже кандидата наук.}
\scneqtoset{Голенков В.В.;
Головко В.А.;
Гулякина Н.А.;
Краснопрошин В.В.;
Курбацкий А.Н.;
Шункевич Д.В.;
Азаров И.С.;
Захарьев В.;
Родченко В.;
Голубева О.В.
}
\scnaddlevel{-1}
\scnheader{Стандарт OSTIS}
\scnrelfrom{консорциум}{Консорциум OSTIS}
\scnaddlevel{1}
\scnrelfromlist{несёт ответственность за}{\scnfileitem{Распространение, внедрение и продвижение комплекса стандартов \textit{Технологии OSTIS} во все виды \textit{человеческой деятельности} в рамках глобального комплексного прикладного проекта \textit{Экосистемы OSTIS}.};
\scnfileitem{Взаимодействие с международными институтами и консорциумами, заинтересованными в стандартизации \textit{интеллектуальных компьютерных систем} и соответствующих технологий их \textit{проектирования}, \textit{производства}, \textit{эксплуатации} и \textit{реинжиниринга}.}}
\scnaddlevel{-1}
\scnheader{консорциум*}
\scnidtf{быть консорциумом для заданного инновационного продукта, которым, в частности, может быть стандарт некоторые перспективные технологии*}
\scnidtf{быть субъектом продвижение заданного инновационного продукта на международной арене*}
\scniselement{бинарное ориентированное отношение}

\scnheader{Стандарт OSTIS}
\scntext{аналоги}{Аналогами (в широком смысле) \textit{Стандарта OSTIS} можно считать:
\begin{scnitemize}
\item любую серьезную попытку систематизации результатов, полученных в области \textit{Искусственного интеллекта} к текущему моменту:
\begin{scnitemizeii}
\item учебник, достаточно полно отражающий текущее состояние \textit{Искусственного интеллекта};
\item справочник, содержащий достаточно полную информацию о текущем состоянии \textit{Искусственного интеллекта}. Примером такого справочника является трехтомный справочник по Искусственному интеллекту
\end{scnitemizeii}

\scnauthorcomment{Дополнить библиографию}

\item любую попытку перехода от частных формальных моделей различных компонентов интеллектуальных компьютерных систем общей (объединённой, интегрированной) формальной модели \textit{интеллектуальных компьютерных систем} в целом -- к общей теории \textit{интеллектуальных компьютерных систем};
\item любую попытку унификации технических решений, устранения "вредного"{} многообразия форм технических решений при разработке \textit{интеллектуальных компьютерных систем}
\item первые попытки разработки стандартов \textit{интеллектуальных компьютерных систем} и \textit{технологий Искусственного интеллекта}, которые чаще всего ограничиваются построением систем соответствующих понятий.
\end{scnitemize}
}
\scntext{сравнение с аналогами}{Перечислим основные особенности и достоинства \textit{Стандарта OSTIS} по сравнению с различного вида его аналогами:
\begin{scnitemize}
\item \textit{Стандарт OSTIS} -- это не просто систематизация современного состояния результатов в области \textit{Искусственного интеллекта}, это систематизация, представленная в виде общей комплексной \uline{формальной} модели \textit{интеллектуальных компьютерных систем} и комплексной \uline{формальной} модели \textit{технологии} их разработки. Более того, текст \textit{Стандарта OSTIS} представляет собой раздел \textit{базы знаний} специальной \textit{интеллектуальной метасистемы}, которая ориентирована:
\begin{scnitemizeii}
\item на поддержку эволюция \textit{Стандарта OSTIS};
\item на поддержку разработки \textit{интеллектуальных компьютерных систем} различного назначения;
\item на поддержку \textit{подготовки специалистов в области Искусственного интеллекта};
\end{scnitemizeii}
\item \textit{Стандарт OSTIS} -- это \uline{динамический} текст, перманентно отражающий все новые и новые научно-технические результаты, получаемые в области \textit{Искусственного интеллекта} в рамках \textit{Общей теории интеллектуальных компьютерных систем} и \textit{Общей комплексной технологии разработки интеллектуальных компьютерных систем}. Здесь важной является оперативность фиксации новых научно-технических результатов, т.е. минимизация отрезка времени между моментом получения новых результатов и моментом интеграции описания этих результатов в состав \textit{Стандарта OSTIS}. В перспективе авторы новых научно-технических результатов в области \textit{Искусственного интеллекта} будут заинтересованы лично публиковать (интегрировать) свои результаты в состав \textit{Стандарта OSTIS}, т.е. становиться соавторами \textit{Стандарта OSTIS}, чтобы обеспечить необходимую оперативность такой публикации и отсутствие искажений своих результатов. Динамичность \textit{Стандарта OSTIS} и достаточная оперативность интеграции в его состав новых научно-технических результатов в области \textit{Искусственного интеллекта} делает \textit{Стандарт OSTIS} всегда актуальным и никогда морально устаревшим;
\item В рамках стандарта OSTIS нет противопоставления между научно-технической информацией, добываемой в области Искусственного интеллекта, и учебно-методической информацией, используемой для подготовки и самоподготовки специалистов в области Искусственного интеллекта. информация о том, чему учить, должна быть "переплетена"{}, интегрирована с информацией о том, как учить.
\end{scnitemize}
}
\scnrelfromvector{ключевые знаки}{Технология OSTIS
	\scnaddlevel{1}
	\scnidtf{Open Semantic Technology for Intelligent Systems}
	\scnidtf{Открытая технология онтологического проектирования, производства и эксплуатации семантически совместимых гибридных интеллектуальных компьютерных систем}
	\scnaddlevel{-1}
;технология\\
	\scnaddlevel{1}
	\scnsuperset{открытая технология}
		\scnaddlevel{1}
		\scnidtf{технология, доступная не только для тех, кто желает ее использовать, но и для тех, кто желает участвовать в ее развитии -- для того, чтобы технология быстро развивалась, она должна иметь широкий круг своих пользователей и разработчиков}
		\scnaddlevel{-1}
	\scnsuperset{технология проектирования}
	\scnsuperset{технология производства}
	\scnsuperset{технология эксплуатации}
	\scnsuperset{онтологическая технология}
		\scnaddlevel{1}
		\scnidtf{технология выполнения соответствующего вида деятельности, в основе которой лежит иерархическая система формальных онтологий, обеспечивающая четкую стратификацию указанной деятельности и наследование свойств между различными уровнями детализации этой
деятельности}
		\scnsuperset{технология онтологического проектирования}
		\scnsuperset{технология онтологического производства}
		\scnsuperset{технология онтологической эксплуатации}
		\scnaddlevel{-1}
		\scnaddlevel{-1}
;интеллектуальная компьютерная система
			\scnaddlevel{1}
			\scnidtf{искусственная интеллектуальная система}
			\scnsubset{интеллектуальная система}
				\scnaddlevel{1}
				\scnidtf{интеллектуальная кибернетическая система}
				\scnsubset{кибернетическая система}
				\scnaddlevel{-1}
		\scnsuperset{гибридная интеллектуальная компьютерная система}
		\scnaddlevel{1}
		\scnidtf{интеллектуальная компьютерная система, в которой глубоко интегрированы различные виды знаний и различные модели решения задач}
		\scnaddlevel{-1}
	\scnaddlevel{-1}
;семантическая совместимость интеллектуальных компьютерных систем\scnsupergroupsign
	\scnaddlevel{1}
	\scnidtf{свойство, определяющее степень (уровень) семантической совместимости каждой пары интеллектуальных компьютерных систем\scnsupergroupsign}
	\scnaddlevel{-1}
;семейство семантически совместимых интеллектуальных компьютерных систем*
	\scnaddlevel{1}
	\scnidtf{семейство интеллектуальных компьютерных
систем, все пары которых имеют одинаковый уровень семантической совместимости (взаимопонимания)}
	\scnaddlevel{-1}
;семантическая унификация интеллектуальных компьютерных систем 
	\scnaddlevel{1}
	\scnidtf{процесс обеспечения высокого уровня семантической совместимости интеллектуальных компьютерных систем в ходе их проектирования, эксплуатации и реинжиниринга}
	\scnaddlevel{-1}
;ostis-система
	\scnaddlevel{1}
	\scnidtf{интеллектуальная компьютерная система, разработанная по Технологии OSTIS}
	\scnsubset{интеллектуальная компьютерная система}
	\scnaddlevel{-1}
;ostis-документация
	\scnaddlevel{1}
	\scnidtf{документация соответствующего объекта, представленная в виде раздела базы знаний некоторой ostis-системы}
	\scnhaselement{Документация Технологии OSTIS}
	\scnaddlevel{-1}
;публикация ostis-документации
	\scnaddlevel{1}
	\scnidtf{внешнее представление ostis-документации, доступное широкому кругу читателей}
	\scnsubset{внешнее представление фрагмента базы знаний ostis-системы}
	\scnhaselement{Публикация Документации Технологии OSTIS-2021}
		\scnaddlevel{1}
		\scnnote{Здесь имеется в виду Документации Технологии OSTIS версии 2021 года}
		\scnaddlevel{-1}
	\scnaddlevel{-1}
;публикация ostis-документации*
	\scnaddlevel{1}
	\scnidtf{быть публикацией заданной ostis-документации*}
	\scnidtfexp{Бинарное ориентированное \textit{отношение}, каждая \textit{пара} которого связывает знак некоторого \textit{раздела базы знаний} со знаком \textit{файла}, который является внешним представлением указанного раздела, а также является либо копией электронной публикации материалов этого раздела, либо оригинал-макетом бумажной публикации указанных материалов*}
	\scnaddlevel{-1}
;оглавление публикации ostis-документации\\
	\scnaddlevel{1}
	\scnhaselement{Оглавление Публикации Документации Технологии OSTIS-2021}
	\scnaddlevel{-1}
;оглавление публикации ostis-документации*
	\scnaddlevel{1}
	\scnidtf{быть оглавлением заданной публикации соответствующей ostis-документации*}
	\scnidtfexp{Бинарное ориентированное \textit{отношение}, каждая \textit{пара} которого связывает знак некоторого \textit{раздела базы знаний} либо знак \textit{файла}, содержащего некоторый \textit{документ}, с описанием иерархии \uline{всех} \textit{разделов}, входящих в состав указанного \textit{раздела базы знаний} либо указанного \textit{документа}*}
	\scnaddlevel{-1}
;база знаний ostis-системы
;раздел базы знаний ostis-системы
;конкатенация подразделов*\\
	\scnaddlevel{1}
	\scnexplanation{Бинарное ориентированное \textit{отношение}, каждая \textit{пара} которого связывает \textit{знак} некоторого \textit{раздела базы знаний} либо знак \textit{файла}, содержащего некоторый \textit{документ}, с упорядоченным множеством всех \uline{непосредственных} подразделов указанного \textit{раздела базы знаний} или указанного \textit{документа}*}
	\scnaddlevel{-1}
;Искусственный интеллект
	\scnaddlevel{1}	
	\scnidtf{Научно-техническая дисциплина, направленная на изучение интеллектуальных систем для построения искусственных интеллектуальных систем}
	\scniselement{научно-техническая дисциплина}
	\scnidtf{кибернетическая система, имеющая достаточно высокий уровень интеллекта}
	\scnsubset{кибернетическая система}
	\scnaddlevel{1}
	\scnidtf{система, в основе функционирования которой лежит обработка информации}
	\scnaddlevel{-1}
	\scnsuperset{интеллектуальная компьютерная система}
	\scnaddlevel{1}
	\scnidtf{искусственная интеллектуальная система}
	\scnidtf{интеллектуальная компьютерная система}
	\scnaddlevel{-1}
	\scnaddlevel{-1}	
;SC-код
	\scnaddlevel{1}
	\scnidtf{Semantic Computer Code}
	\scnidtf{Внутренний язык ostis-систем}
	\scnaddlevel{-1}
;sc-конструкция
	\scnaddlevel{1}
	\scnidtf{информационная конструкция, принадлежащая SC-коду}
	\scnaddlevel{-1}
;sc-элемент
	\scnaddlevel{1}
	\scnidtf{знак, входящий в в состав sc-конструкции}
	\scnaddlevel{-1}
;sc-идентификатор
	\scnaddlevel{1}
	\scnidtf{внешний идентификатор sc-элемента}
	\scnidtf{внешний идентификатор знака, входящего в текст Внутреннего языка ostis-системы}	
	\scnaddlevel{-1}
;внешний язык ostis-систем
	\scnaddlevel{1}
	\scnidtf{язык коммуникации ostis-систем с их пользователями и другими ostis-системами}
	\scnhaselement{SCg-код}
		\scnaddlevel{1}
		\scnidtf{Semantic Computer Code graphical}
		\scnidtf{Язык графического представления знаний ostis-систем}
		\scnaddlevel{-1}
	\scnhaselement{SCs-код}
		\scnaddlevel{1}
		\scnidtf{Semantic Computer Code string}
		\scnidtf{Язык линейного представления знаний ostis-систем}
		\scnaddlevel{-1}
	\scnhaselement{SCn-код}
		\scnaddlevel{1}
		\scnidtf{Semantic Computer Code natural}
		\scnidtf{Язык структурированного представления знаний ostis-систем}
		\scnaddlevel{-1}
	\scnaddlevel{-1}
;знание ostis-системы\\
	\scnaddlevel{1}
	\scnsuperset{внутреннее представление знания ostis-системы}
	\scnsuperset{внешнее представление знания ostis-системы}
	\scnaddlevel{-1}
;предметная область
	\scnaddlevel{1}
	\scnidtf{предметная область, представленная в памяти ostis-системы}
	\scnidtf{sc-модель предметной области}
	\scnaddlevel{-1}
;онтология
	\scnaddlevel{1}
	\scnidtf{формальная онтология, представленная в памяти ostis-системы}
	\scnidtf{sc-модель онтологии}
	\scnaddlevel{-1}
;предметная область и онтология
	\scnaddlevel{1}
	\scnidtf{объединение предметной области и онтологии}
	\scnaddlevel{-1}
;кибернетическая система\\
	\scnaddlevel{1}
	\scnsuperset{интеллектуальная система}
	\scnsuperset{компьютерная система}
		\scnaddlevel{1}
		\scnsuperset{интеллектуальная компьютерная система}
		\scnaddlevel{-1}
	\scnaddlevel{-1}
;решатель задач кибернетической системы
;интерфейс кибернетической системы
;база знаний
;традиционная компьютерная технология
	\scnaddlevel{1}
	\scnidtf{современная информационная технология}
	\scnaddlevel{-1}
;технология искусственного интеллекта
;логико-семантическая модель ostis-системы
	\scnaddlevel{1}
	\scnidtf{формальная логико-семантическая модель ostis-системы, представленная в SC-коде}
	\scnidtf{sc-модель ostis-системы}
	\scnidtf{результат проектирования ostis-системы}
	\scnidtf{стартовое состояние базы знаний проектируемой ostis-системы}
	\scnaddlevel{-1}
;семантическая сеть
;семантический язык\\
	\scnaddlevel{1}
	\scnidtf{язык, построенный на основе семантических сетей}
	\scnaddlevel{-1}
;семантическая модель базы знаний\\
	\scnaddlevel{1}
	\scnidtf{база знаний, представленная в SC-коде}
	\scnaddlevel{-1}
;семантическая модель решателя задач
;семантическая модель интерфейса компьютерной системы
;семантическая окрестность\\
	\scnaddlevel{1}
	\scnidtf{семантическая спецификация}
	\scnaddlevel{-1}
;логическая формула
;логическая онтология\\
	\scnaddlevel{1}
	\scnsubset{онтология}
	\scnaddlevel{-1}
;внешняя информационная конструкция ostis-системы\\
	\scnaddlevel{1}
	\scnidtf{информационная конструкция, записанная не в SC-коде}
	\scnaddlevel{-1}
;файл ostis-системы\\
;свойство\\
	\scnaddlevel{1}
	\scnidtf{параметр}
	\scnidtf{множество множеств сущностей, имеющих некоторое общее свойство}
	\scnsubset{класс классов}
	\scnaddlevel{-1}
;величина\\
	\scnaddlevel{1}
	\scnidtf{значение свойства}
	\scnaddlevel{-1}
;шкала
;структура\\
	\scnaddlevel{1}
	\scnidtf{фрагмент базы знаний ostis-системы}
	\scnaddlevel{-1}
;теоретико-множественная онтология\\
	\scnaddlevel{1}
	\scnsubset{онтология}
	\scnaddlevel{-1}
;материальная сущность
;пространственная сущность
;темпоральная сущность\\
	\scnaddlevel{1}
	\scnidtf{временная сущность}
	\scnidtf{временно существующая сущность}
	\scnaddlevel{-1}
;темпоральная сущность базы знаний ostis-системы\\
	\scnaddlevel{1}
	\scnsubset{темпоральная сущность}
	\scnaddlevel{-1}
;действие
;задача
;план
;протокол
;метод\\
	\scnaddlevel{1}
	\scnidtf{метод решения задач заданного класса}
	\scnaddlevel{-1}
;задачная онтология\\
	\scnaddlevel{1}
	\scnsubset{онтология}
	\scnidtf{онтология методов решения задач в рамках заданной предметной области}
	\scnaddlevel{-1}
;внутренний агент ostis-системы
;Базовый язык программирования ostis-систем\\
	\scnaddlevel{1}
	\scnidtf{Язык SCP}
	\scnidtf{Semantic Code Programming}
	\scnaddlevel{-1}
;денотационная семантика языка программирования
;операционная семантика языка программирования
;информационно-поисковый агент ostis-системы
;логическое исчисление
;логический агент ostis-системы
;сообщение
;интерфейсное действие ostis-системы
;агент пользовательского интерфейса ostis-системы
;знаковая конструкция
;естественный язык
;методика разработки ostis-систем
;средства разработки ostis-систем
;базовый интерпретатор логико-семантических моделей ostis-систем
;семантический ассоциативный компьютер
;Библиотека многократно используемых компонентов ostis-систем
;встраиваемая ostis-система
;понимание информации
;противоречие в базе знаний ostis-системы
;информационная дыра в базе знаний ostis-системы\\
	\scnaddlevel{1}
	\scnidtf{неполнота в базе знаний ostis-системы}
	\scnaddlevel{-1}
;разработчик баз знаний ostis-систем
;Экосистема OSTIS
;интеллектуальный портал научно-технических знаний
;интеллектуальная справочная система
;интеллектуальная help-система
;интеллектуальная корпоративная система
;интеллектуальная система в сфере образования\\
	\scnaddlevel{1}
	\scnsuperset{интеллектуальная обучающая система}
	\scnaddlevel{-1}
;интеллектуальная система автоматизации проектирования
;интеллектуальная система управления проектированием
;интеллектуальная система управления производством
;Метасистема IMS.ostis
}

\scnheader{следует отличать*}
\scnhaselementset{Стандарт OSTIS\\
	\scnaddlevel{1}
	\scnexplanation{как \uline{внутреннее} представление Стандарта OSTIS в памяти ostis-системы}
	\scnaddlevel{-1}
;Технология OSTIS\\
	\scnaddlevel{1}
	\scnexplanation{как объект. специфицируемый (описываемый) стандартом OSTIS}
	\scnaddlevel{-1}
;официальная версия Стандарта OSTIS\\
	\scnaddlevel{1}
	\scnhaselement{\scnkeyword{Стандарт OSTIS-2021}}
		\scnaddlevel{1}
		\scnexplanation{как \uline{внешнее} представление Стандарта OSTIS в виде исходного текста соответствующего раздела базы знаний}
		\scnaddlevel{-1}
	\scnaddlevel{-1}}


\scnheader{Публикация Документации Технологии OSTIS-2021}
\scnrelto{публикация ostis-документации}{Документация Технологии OSTIS}
\scnidtf{Предлагаемое Вашему вниманию издание внешнего представления \textit{раздела базы знаний} \scnbigspace \textit{Метасистемы IMS.ostis}, посвященного комплексному описанию \textit{Технологии OSTIS} и отражающего версию указанного раздела, соответствующую весеннему периоду 2021 года}

\scnidtf{Издание Документации Технологии OSTIS-2021}
\scnidtf{Первое издание (публикация) Внешнего представления Документации Технологии OSTIS в виде книги}
\scniselement{публикация}
\scnaddlevel{1}
\scnidtf{Официальная \textit{версия Стандарта OSTIS}, издаваемая непостредственно перед началом \textit{Конференции OSTIS-2021}}
\scnaddlevel{-1}

\bigskip

\scnaddlevel{1}
\scnheaderlocal{публикация ostis-документации*}
\scnidtfexp{Бинарное ориентированное \textit{отношение}, каждая \textit{пара} которого связывает знак некоторого \textit{раздела базы знаний} со знаком \textit{файла}, который является внешним представлением указанного раздела, а также является либо копией электронной публикации материалов этого раздела, либо оригинал-макетом бумажной публикации указанных материалов*}
\scnaddlevel{-1}

\scnheader{Стандарт OSTIS}
\scnrelto{формальная спецификация}{Технология OSTIS}
	\scnaddlevel{1}
	\scnidtf{Перманентно развиваемый в рамках открытого проекта комплекс моделей, методов и средств, ориентированных на онтологическое проектирование, производство, экплуатацию и реинжиниринг семантически совместимых гибридных интеллектуальных компьютерных систем, способных самостоятельно взаимодействовать друг с другом}
	\scnidtf{Технология разработки семантически совместимых и самостоятельно взаимодействующих интеллектуальных компьютерных систем}
	\scnexplanation{\textit{Технология OSTIS} -- это технология принципиально нового уровня, это обусловлено:
		\begin{scnitemize}
		\item высоким качеством интеллектуальных компьютерных систем (ostis-систем), разрабатываемых на ее основе -- их семантической совместимостью, способностью к самостоятельному взаимодействию, способностью адаптироваться к пользователям и способностью адаптировать (обучать) самих пользователей более эффективному взаимодействию с интеллектальными компьютерными системами;
		\item высоким качеством самой \textit{технологии} -- возможностью интегрировать самые различные \textit{виды знаний} и самые различные \textit{модели решения задач}, неразрывной связью процесса разработки интеллектуальных компьютерных систем и процесса повышения квалификации разработчиков.
		\end{scnitemize}}
	\scnaddlevel{-1}
\bigskip
\scnrelfromvector{ключевые знаки}{Технология OSTIS
	\scnaddlevel{1}
	\scnidtf{Open Semantic Technology for Intelligent Systems}
	\scnidtf{Открытая технология онтологического проектирования, производства и эксплуатации семантически совместимых гибридных интеллектуальных компьютерных систем}
	\scnaddlevel{-1}
;технология\\
	\scnaddlevel{1}
	\scnsuperset{открытая технология}
		\scnaddlevel{1}
		\scnidtf{технология, доступная не только для тех, кто желает ее использовать, но и для тех, кто желает участвовать в ее развитии -- для того, чтобы технология быстро развивалась, она должна иметь широкий круг своих пользователей и разработчиков}
		\scnaddlevel{-1}
	\scnsuperset{технология проектирования}
	\scnsuperset{технология производства}
	\scnsuperset{технология эксплуатации}
	\scnsuperset{онтологическая технология}
		\scnaddlevel{1}
		\scnidtf{технология выполнения соответствующего вида деятельности, в основе которой лежит иерархическая система формальных онтологий, обеспечивающая четкую стратификацию указанной деятельности и наследование свойств между различными уровнями детализации этой
деятельности}
		\scnsuperset{технология онтологического проектирования}
		\scnsuperset{технология онтологического производства}
		\scnsuperset{технология онтологической эксплуатации}
		\scnaddlevel{-1}
		\scnaddlevel{-1}
;интеллектуальная компьютерная система
			\scnaddlevel{1}
			\scnidtf{искусственная интеллектуальная система}
			\scnsubset{интеллектуальная система}
				\scnaddlevel{1}
				\scnidtf{интеллектуальная кибернетическая система}
				\scnsubset{кибернетическая система}
				\scnaddlevel{-1}
		\scnsuperset{гибридная интеллектуальная компьютерная система}
		\scnaddlevel{1}
		\scnidtf{интеллектуальная компьютерная система, в которой глубоко интегрированы различные виды знаний и различные модели решения задач}
		\scnaddlevel{-1}
	\scnaddlevel{-1}
;семантическая совместимость интеллектуальных компьютерных систем\scnsupergroupsign
	\scnaddlevel{1}
	\scnidtf{свойство, определяющее степень (уровень) семантической совместимости каждой пары интеллектуальных компьютерных систем\scnsupergroupsign}
	\scnaddlevel{-1}
;семейство семантически совместимых интеллектуальных компьютерных систем*
	\scnaddlevel{1}
	\scnidtf{семейство интеллектуальных компьютерных
систем, все пары которых имеют одинаковый уровень семантической совместимости (взаимопонимания)}
	\scnaddlevel{-1}
;семантическая унификация интеллектуальных компьютерных систем 
	\scnaddlevel{1}
	\scnidtf{процесс обеспечения высокого уровня семантической совместимости интеллектуальных компьютерных систем в ходе их проектирования, эксплуатации и реинжиниринга}
	\scnaddlevel{-1}
;ostis-система
	\scnaddlevel{1}
	\scnidtf{интеллектуальная компьютерная система, разработанная по Технологии OSTIS}
	\scnsubset{интеллектуальная компьютерная система}
	\scnaddlevel{-1}
;ostis-документация
	\scnaddlevel{1}
	\scnidtf{документация соответствующего объекта, представленная в виде раздела базы знаний некоторой ostis-системы}
	\scnhaselement{Документация Технологии OSTIS}
	\scnaddlevel{-1}
;публикация ostis-документации
	\scnaddlevel{1}
	\scnidtf{внешнее представление ostis-документации, доступное широкому кругу читателей}
	\scnsubset{внешнее представление фрагмента базы знаний ostis-системы}
	\scnhaselement{Публикация Документации Технологии OSTIS-2021}
		\scnaddlevel{1}
		\scnnote{Здесь имеется в виду Документации Технологии OSTIS версии 2021 года}
		\scnaddlevel{-1}
	\scnaddlevel{-1}
;публикация ostis-документации*
	\scnaddlevel{1}
	\scnidtf{быть публикацией заданной ostis-документации*}
	\scnidtfexp{Бинарное ориентированное \textit{отношение}, каждая \textit{пара} которого связывает знак некоторого \textit{раздела базы знаний} со знаком \textit{файла}, который является внешним представлением указанного раздела, а также является либо копией электронной публикации материалов этого раздела, либо оригинал-макетом бумажной публикации указанных материалов*}
	\scnaddlevel{-1}
;оглавление публикации ostis-документации\\
	\scnaddlevel{1}
	\scnhaselement{Оглавление Публикации Документации Технологии OSTIS-2021}
	\scnaddlevel{-1}
;оглавление публикации ostis-документации*
	\scnaddlevel{1}
	\scnidtf{быть оглавлением заданной публикации соответствующей ostis-документации*}
	\scnidtfexp{Бинарное ориентированное \textit{отношение}, каждая \textit{пара} которого связывает знак некоторого \textit{раздела базы знаний} либо знак \textit{файла}, содержащего некоторый \textit{документ}, с описанием иерархии \uline{всех} \textit{разделов}, входящих в состав указанного \textit{раздела базы знаний} либо указанного \textit{документа}*}
	\scnaddlevel{-1}
;база знаний ostis-системы
;раздел базы знаний ostis-системы
;конкатенация подразделов*\\
	\scnaddlevel{1}
	\scnexplanation{Бинарное ориентированное \textit{отношение}, каждая \textit{пара} которого связывает \textit{знак} некоторого \textit{раздела базы знаний} либо знак \textit{файла}, содержащего некоторый \textit{документ}, с упорядоченным множеством всех \uline{непосредственных} подразделов указанного \textit{раздела базы знаний} или указанного \textit{документа}*}
	\scnaddlevel{-1}
;Искусственный интеллект
	\scnaddlevel{1}	
	\scnidtf{Научно-техническая дисциплина, направленная на изучение интеллектуальных систем для построения искусственных интеллектуальных систем}
	\scniselement{научно-техническая дисциплина}
	\scnidtf{кибернетическая система, имеющая достаточно высокий уровень интеллекта}
	\scnsubset{кибернетическая система}
	\scnaddlevel{1}
	\scnidtf{система, в основе функционирования которой лежит обработка информации}
	\scnaddlevel{-1}
	\scnsuperset{интеллектуальная компьютерная система}
	\scnaddlevel{1}
	\scnidtf{искусственная интеллектуальная система}
	\scnidtf{интеллектуальная компьютерная система}
	\scnaddlevel{-1}
	\scnaddlevel{-1}	
;SC-код
	\scnaddlevel{1}
	\scnidtf{Semantic Computer Code}
	\scnidtf{Внутренний язык ostis-систем}
	\scnaddlevel{-1}
;sc-конструкция
	\scnaddlevel{1}
	\scnidtf{информационная конструкция, принадлежащая SC-коду}
	\scnaddlevel{-1}
;sc-элемент
	\scnaddlevel{1}
	\scnidtf{знак, входящий в в состав sc-конструкции}
	\scnaddlevel{-1}
;sc-идентификатор
	\scnaddlevel{1}
	\scnidtf{внешний идентификатор sc-элемента}
	\scnidtf{внешний идентификатор знака, входящего в текст Внутреннего языка ostis-системы}	
	\scnaddlevel{-1}
;внешний язык ostis-систем
	\scnaddlevel{1}
	\scnidtf{язык коммуникации ostis-систем с их пользователями и другими ostis-системами}
	\scnhaselement{SCg-код}
		\scnaddlevel{1}
		\scnidtf{Semantic Computer Code graphical}
		\scnidtf{Язык графического представления знаний ostis-систем}
		\scnaddlevel{-1}
	\scnhaselement{SCs-код}
		\scnaddlevel{1}
		\scnidtf{Semantic Computer Code string}
		\scnidtf{Язык линейного представления знаний ostis-систем}
		\scnaddlevel{-1}
	\scnhaselement{SCn-код}
		\scnaddlevel{1}
		\scnidtf{Semantic Computer Code natural}
		\scnidtf{Язык структурированного представления знаний ostis-систем}
		\scnaddlevel{-1}
	\scnaddlevel{-1}
;знание ostis-системы\\
	\scnaddlevel{1}
	\scnsuperset{внутреннее представление знания ostis-системы}
	\scnsuperset{внешнее представление знания ostis-системы}
	\scnaddlevel{-1}
;предметная область
	\scnaddlevel{1}
	\scnidtf{предметная область, представленная в памяти ostis-системы}
	\scnidtf{sc-модель предметной области}
	\scnaddlevel{-1}
;онтология
	\scnaddlevel{1}
	\scnidtf{формальная онтология, представленная в памяти ostis-системы}
	\scnidtf{sc-модель онтологии}
	\scnaddlevel{-1}
;предметная область и онтология
	\scnaddlevel{1}
	\scnidtf{объединение предметной области и онтологии}
	\scnaddlevel{-1}
;кибернетическая система\\
	\scnaddlevel{1}
	\scnsuperset{интеллектуальная система}
	\scnsuperset{компьютерная система}
		\scnaddlevel{1}
		\scnsuperset{интеллектуальная компьютерная система}
		\scnaddlevel{-1}
	\scnaddlevel{-1}
;решатель задач кибернетической системы
;интерфейс кибернетической системы
;база знаний
;традиционная компьютерная технология
	\scnaddlevel{1}
	\scnidtf{современная информационная технология}
	\scnaddlevel{-1}
;технология искусственного интеллекта
;логико-семантическая модель ostis-системы
	\scnaddlevel{1}
	\scnidtf{формальная логико-семантическая модель ostis-системы, представленная в SC-коде}
	\scnidtf{sc-модель ostis-системы}
	\scnidtf{результат проектирования ostis-системы}
	\scnidtf{стартовое состояние базы знаний проектируемой ostis-системы}
	\scnaddlevel{-1}
;семантическая сеть
;семантический язык\\
	\scnaddlevel{1}
	\scnidtf{язык, построенный на основе семантических сетей}
	\scnaddlevel{-1}
;семантическая модель базы знаний\\
	\scnaddlevel{1}
	\scnidtf{база знаний, представленная в SC-коде}
	\scnaddlevel{-1}
;семантическая модель решателя задач
;семантическая модель интерфейса компьютерной системы
;семантическая окрестность\\
	\scnaddlevel{1}
	\scnidtf{семантическая спецификация}
	\scnaddlevel{-1}
;логическая формула
;логическая онтология\\
	\scnaddlevel{1}
	\scnsubset{онтология}
	\scnaddlevel{-1}
;внешняя информационная конструкция ostis-системы\\
	\scnaddlevel{1}
	\scnidtf{информационная конструкция, записанная не в SC-коде}
	\scnaddlevel{-1}
;файл ostis-системы\\
;свойство\\
	\scnaddlevel{1}
	\scnidtf{параметр}
	\scnidtf{множество множеств сущностей, имеющих некоторое общее свойство}
	\scnsubset{класс классов}
	\scnaddlevel{-1}
;величина\\
	\scnaddlevel{1}
	\scnidtf{значение свойства}
	\scnaddlevel{-1}
;шкала
;структура\\
	\scnaddlevel{1}
	\scnidtf{фрагмент базы знаний ostis-системы}
	\scnaddlevel{-1}
;теоретико-множественная онтология\\
	\scnaddlevel{1}
	\scnsubset{онтология}
	\scnaddlevel{-1}
;материальная сущность
;пространственная сущность
;темпоральная сущность\\
	\scnaddlevel{1}
	\scnidtf{временная сущность}
	\scnidtf{временно существующая сущность}
	\scnaddlevel{-1}
;темпоральная сущность базы знаний ostis-системы\\
	\scnaddlevel{1}
	\scnsubset{темпоральная сущность}
	\scnaddlevel{-1}
;действие
;задача
;план
;протокол
;метод\\
	\scnaddlevel{1}
	\scnidtf{метод решения задач заданного класса}
	\scnaddlevel{-1}
;задачная онтология\\
	\scnaddlevel{1}
	\scnsubset{онтология}
	\scnidtf{онтология методов решения задач в рамках заданной предметной области}
	\scnaddlevel{-1}
;внутренний агент ostis-системы
;Базовый язык программирования ostis-систем\\
	\scnaddlevel{1}
	\scnidtf{Язык SCP}
	\scnidtf{Semantic Code Programming}
	\scnaddlevel{-1}
;денотационная семантика языка программирования
;операционная семантика языка программирования
;информационно-поисковый агент ostis-системы
;логическое исчисление
;логический агент ostis-системы
;сообщение
;интерфейсное действие ostis-системы
;агент пользовательского интерфейса ostis-системы
;знаковая конструкция
;естественный язык
;методика разработки ostis-систем
;средства разработки ostis-систем
;базовый интерпретатор логико-семантических моделей ostis-систем
;семантический ассоциативный компьютер
;Библиотека многократно используемых компонентов ostis-систем
;встраиваемая ostis-система
;понимание информации
;противоречие в базе знаний ostis-системы
;информационная дыра в базе знаний ostis-системы\\
	\scnaddlevel{1}
	\scnidtf{неполнота в базе знаний ostis-системы}
	\scnaddlevel{-1}
;разработчик баз знаний ostis-систем
;Экосистема OSTIS
;интеллектуальный портал научно-технических знаний
;интеллектуальная справочная система
;интеллектуальная help-система
;интеллектуальная корпоративная система
;интеллектуальная система в сфере образования\\
	\scnaddlevel{1}
	\scnsuperset{интеллектуальная обучающая система}
	\scnaddlevel{-1}
;интеллектуальная система автоматизации проектирования
;интеллектуальная система управления проектированием
;интеллектуальная система управления производством
;Метасистема IMS.ostis
}

\begin{SCn}
%Begin

\scnfragmentcaption

\scnheader{Стандарт OSTIS-2021}
\scnrelto{официальная версия}{Стандарт OSTIS}
\scnnote{Данная официально изданная версия \textit{Стандарта OSTIS}, которую Вы держите в руках, занимает особое место:
\begin{scnitemize}
\item Во-первых, это первый опыт издания (публикации) подобного документа, в рамках которого необходимо обеспечивать, с одной стороны, строгую формальность, а, с другой стороны, интуитивное и адекватное понимание формальных текстов со стороны читателей;
\item Во-вторых, данный текст является описанием условно выделенной первой версии \textit{Стандарта OSTIS} (\textit{Стандарта OSTIS-2021}), в рамках которого представлены далеко не все разделы \textit{Стандарта OSTIS}. Эти разделы будут представлены в последующих версиях \textit{Стандарта OSTIS} (в \textit{Стандарте OSTIS-2022}, в \textit{Стандарте OSTIS-2023} и т.д.);
\item Особенностью \textit{публикации} (издания) Стандарта OSTIS версии OSTIS-2021, как, впрочем, и всех последующих версий, является то, что она оформлена в виде \uline{внешнего представления} основной части \textit{базы знаний} специальной \textit{ostis-системы}, которая предназначена для комплексной поддержки проектирования \uline{семантически совместимых} \textit{ostis-систем}. Эту систему мы назвали \textbf{\textit{Метасистемой IMS.ostis}} (Intelligent MetaSystem for ostis-systems). Последовательность изложения материала во внешнем представлении \textit{базы знаний} не является единственно возможным маршрутом прочтения (просмотра) \textit{базы знаний}. Каждый читатель, войдя в \textbf{\textit{Метасистему IMS.ostis}}, может выбрать любой другой маршрут навигации по этой \textit{базе знаний}, задавая указанной метасистеме те \textit{вопросы}, которые в текущий момент его интересуют. Таким образом, читая предлагаемый вашему вниманию текст и одновременно работая с \textbf{\textit{Метасистемой IMS.ostis}}, можно значительно быстрее усвоить детали \textbf{\textit{Технологии OSTIS}} и значительно быстрее приступить к непосредственному использованию указанной технологии. Этому также способствует большое количество примеров семантических моделей различных фрагментов \textit{интеллектуальных компьютерных систем};
\item Основной семантический вид \textit{разделов баз знаний \textbf{ostis-систем}} -- это формальное представление различных \textbf{\textit{предметных областей}} вместе с соответствующими им \textbf{\textit{онтологиями}}. При этом явно указываются связи между этими \textbf{\textit{предметными областями} и \textit{онтологиями}}. Таким образом, \textit{база знаний} \textbf{\textit{Метасистемы IMS.ostis}}, как и любых других \textit{интеллектуальных компьютерных систем}, построенных по \textbf{\textit{Технологии OSTIS}}, представляет собой иерархическую систему связанных между собой формальных моделей \textit{предметных областей} и соответствующих им \textit{онтологий}. Соответственно этому структурирован и текст \textit{Стандарта OSTIS-2021};
\item В основе \textit{Технологии OSTIS} лежит предлагаемая нами унификация \textit{интеллектуальных компьютерных систем}, основанная, в свою очередь, на \textit{смысловом представлении знаний} в \textit{памяти интеллектуальных компьютерных систем}. Таким образом, данную \textit{публикацию} \textit{Стандарта OSTIS-2021} можно рассматривать как версию \textit{стандарта} семантических моделей \textit{интеллектуальных компьютерных систем}. Последующие \textit{публикации}, посвящённые детальному описанию различных компонентов \textit{Технологии OSTIS}, будут также оформляться как внешнее представление соответствующих \textit{разделов базы знаний} \scnbigspace \textit{Метасистемы IMS.ostis} и будут отражать следующие этапы развития  \textit{Технологии OSTIS}, следующие версии этой технологии, и, соответственно, следующие версии \textit{Метасистемы IMS.ostis};
\item Все основные положения \textit{Технологии OSTIS} рассматривались и обсуждались на ежегодных \textit{конференциях OSTIS}, которые стали важным стимулирующим фактором становления и развития \textit{Технологии OSTIS}. Мы благодарим всех активных участников этих конференций;
\item Важной задачей \textit{Стандарта OSTIS-2021} была выработка стилистики формализованного представления научно-технической информации, которая одновременно была бы понятна как человеку, так и интеллектуальной компьютерной системе. По сути это принципиально новый подход к оформлению научно-технических результатов, позволяющий:
\begin{scnitemizeii}
 \item существенно повысить уровень автоматизации анализа качества (корректности, целостности) научно-технической информации;
 \item интеллектуальным компьютерным системам непосредственно (без какой-либо дополнительной "ручной"{} доработки) использовать информацию (знания), содержащуюся в разработанных специалистами документах;
 \item существенно упростить согласование точек зрения различных специалистов, входящих в коллектив разработчиков той или  иной научно-технической документации.
\end{scnitemizeii}
\item Для выработки стилистики и формального представления научно-технической информации нам было важно привлечь к обсуждению и анализу материала \textit{Стандарта OSTIS-2021} как можно больше коллег, участвующих в развитии и применении \textit{Технологии OSTIS}. При этом некоторых коллег мы включили в число соавторов соответствующих разделов монографии.
Основной целью написания \textit{Стандарта OSTIS-2021} является создание технологических и организационных предпосылок к принципиально новому  подходу к организации \textit{научно-технической деятельности} в любой области и, в частности, в области создания и перманентного развития комплексной технологии проектирования и производства семантически совместимых интеллектуальных компьютерных систем (\textit{Технологии OSTIS}). Суть указанного подхода заключается  в глубокой конвергенции и интеграции результатов деятельности всех специалистов, участвующих в создании и развитии \textit{Технологии OSTIS}, путем организации коллективной разработки \textit{базы знаний}, являющейся формальным представлением полной \textit{Документации Технологии OSTIS}, отражающей текущее состояние этой технологии.
\end{scnitemize}
}
\scntext{благодарности}{На данном этапе к разработке и оформлению различных разделов \textit{Стандарта OSTIS-2021} текущей версии \textit{Технологии OSTIS} кроме основных авторов были привлечены студенты, магистранты, аспиранты и преподаватели кафедры интеллектуальных информационных технологий Белорусского государственного университета информатики и радиоэлектроники и кафедры интеллектуальных информационных технологий Брестского государственного технического университета, а также сотрудники ОАО «Савушкин продукт» и ООО «Интелиджент семантик системс». Так, например,
\begin{scnitemize}
\item соавторами Раздела ``\nameref{sec:sd_neuronetworks}''{} являются Головко В.А., Ковалёв М.В., Крощенко А.А., Михно Е.В.;
\item соавторами Раздела ``\nameref{sec:sd_ecosys_enterprise}''{} являются Таберко В.В., Иванюк Д.С., Касьяник В.В.;
\item соавторами Раздела ``\nameref{sec:sd_interfaces}''{} являются Садовский М.Е., Захарьев В.А.,  Никифоров С.А., Коршунов Р.А.;
\item соавторами Раздела ``\nameref{sec:sd_ann}''{} являются Головко В.А., Ковалев М.В.;
\end{scnitemize}

\scnauthorcomment{Добавить соавторов}

Благодарим также студентов кафедры Интеллектуальных информационных технологий Белорусского государственного университета информатики и радиоэлектроники за оказание технической помощи при подготовке текста к печати, сотрудников кафедры Интеллектуальных информационных технологий Брестского государственного технического университета, сотрудников ОАО <<Савушкин продукт>>, сотрудников кафедры ВМиП БГУИР, сотрудников кафедры ЭВМ БГУИР, выражаем также благодарность ООО <<Интелиджент семантик системс>> и его генеральному директору Т. Грюневальду за финансовую поддержку работ по развитию \textit{Технологии OSTIS}, также финансовую поддержку издания \textit{Документации Технологии OSTIS}.
}

\scnheader{Стандарт OSTIS-2021}
\scnidtf{Издание Документации Технологии OSTIS-2021}
\scnidtf{Первое издание (публикация) Внешнего представления Документации Технологии OSTIS в виде книги}
\scniselement{публикация}
	\scnaddlevel{1}
\scnidtf{библиографический источник}
\scnaddlevel{-1}
\scniselement{официальная версия Стандарта OSTIS}
\scniselement{бумажное издание}
\scniselement{научное издание}
\scnrelfrom{рекомендация издания}{***}
\scnrelfrom{издательство}{***}
\scniselementrole{УДК}{\scnfileshort{***}}
\scniselementrole{ББК}{\scnfileshort{***}}
\scniselementrole{ISBN}{\scnfileshort{***}}
\scnrelfrom{технический редактор}{***}
\scnrelfrom{художественный редактор}{***}
\scnrelfrom{корректор}{***}
\scnrelfrom{верстка}{***}
\scnrelfrom{дата подписания в печать}{***}
\scnrelfrom{тираж}{***}
\scnrelfrom{оглавление официальной версии Стандарта OSTIS}{Оглавление Стандарта OSTIS-2021}
\bigskip

%End
\end{SCn}



\begin{SCn}
\scnheader{Подготовка специалистов в области Искусственного интеллекта}
\scnnote{Массовая подготовка высококвалифицированных \textit{специалистов в области Искусственного интеллекта}, способных преодолеть современное кризисное состояние \textit{Искусственного интеллекта}, фактически и является самым главным фактом преодоления указанного кризиса.

Необходимым условием и эпицентром вывода \textit{Искусственного интеллекта} из кризисного состояния и повышения темпов эволюции технологий \textit{Искусственного интеллекта} является организация \textit{подготовки специалистов в области Искусственного интеллекта} на основе активного привлечения студентов, магистрантов и аспирантов к \uline{перманентному} процессу эволюции \textit{Технологии OSTIS}.

Очевидно, этому должно способствовать объединение соответствующей учебно-методической базы для разных кафедр, осуществляющих такую подготовку.

На современном этапе развития \textit{Искусственного интеллекта} требуется не просто подготовка специалистов в этой области -- а подготовка специалистов \uline{принципиально новой формации}, способных
\begin{scnitemize}
	\item рассматривать область \textit{Искусственного интеллекта} не просто как многообразие \textit{интеллектуальных компьютерных систем}, а как постоянно эволюционируемую \uline{\textit{Экосистему}} таких систем
	\item эффективно участвовать в решении как фундаментальных, системных, технологических проблем, так и практических, прикладных проблем эволюции указанной \textit{Экосистемы}
\end{scnitemize}
Все это требует существенного переосмысления организации учебного процесса и учебно-методического обеспечения и 

\scnfileshort{часто, например, совершенствование программных систем сводится к программным "заплаткам"{}. Через какое-то время мы имеем программу со множеством "заплаток"{}, как правило уже громоздкую и малоэффективную. В итоге -- иногда её проще выбросить и создать новую}

\scnaddlevel{-1}
\scnrelfrom{автор}{Курбацкий А. Н.}
\scnaddlevel{1}

Современная разработка каждой сложной программной системы требует построения \uline{качественной} формальной ("цифровой"{}) модели объекта управления, объекта автоматизации, причем \textit{семантически совместимой} с соответствующими моделями в смежных системах.

Здесь важна общая математическая культура и унификация такой формализации.

В настоящее время методологический подход к инженерной деятельности при разработке компьютерных систем часто выглядит следующим образом:

"Поставьте мне четкую инженерную задачу и я ее выполню. Но ответственность за ее постановку я с себя снимаю и не хочу учитывать критерий качества постановки задачи высшего уровня."{}

Для наукоемких проектов, реализуемых в рамках развивающихся технологий, это недопустимо.

Каждый инженер должен \uline{понимать}, что он делает и каковы истинные более глубокие критерии качества его результата.

Нужна принципиально новая психологическая установка.

Необходимо учитывать не только желание заказчика, но и общие принципы и стандарты разрабатываемых \textit{интеллектуальных компьютерных систем}.

В основе организации образовательной деятельности на современном этапе развития \textit{Искусственного интеллекта} лежит:
\begin{scnitemize}
	\item четкое формальное описание того, чему мы учим (каким знаниям и навыкам) -- в нашем случае это описание текущей версии \textit{Стандарта OSTIS} и направлений эволюции этого стандарта;
	\item уточнение того, что должен делать студент, магистрант и любой специалист для быстрого и качественного приобретения этих знаний и навыков.
\end{scnitemize}

Нужна \uline{комплексная} учебная программа по специальности \textit{Искусственный интеллект}, а не мозаика отдельных учебных дисциплин. И, соответственно этому необходимо \uline{комплексное} учебно-методическое пособие, достаточно полно отражающее текущее состояние теории и технологии проектирования \textit{интеллектуальных компьютерных систем}.

Использование проектного метода при подготовке специалистов в области Искусственного интеллекта предполагает составление систематизированного сборника упражнений и задач, в частности, направленных на эволюцию Технологии OSTIS и посильных для студентов специальности \textit{Искусственный интеллект}:
\begin{scnitemize}
	\item представление конкретных фрагментов различных предметных областей и онтологий;	
	\item представление конкретных специфицированных методов (пополнение библиотек используемых методов из разных предметных областей, например, из теории графов);
	\item спецификация библиографических источников (в контексте \textit{Базы знаний IMS.ostis});
	\item выявление синонимии, омонимии, противоречий;
	\item сравнительный анализ и обзор близких внешних публикаций.
\end{scnitemize}

Таким образом, фронт самостоятельных, весьма полезных и посильных для студентов работ весьма широк. Главное сформировать у студентов профессиональный интерес, познавательную активность, инициативность и самостоятельность.
}
\scntext{проектный метод}{Для того, чтобы научиться разрабатывать \textit{интеллектуальные компьютерные системы}, необходимо приобрести достаточно большой опыт участия и \uline{завершения} разработки реально востребованных \textit{интеллектуальных компьютерных систем}}
\scntext{проектный метод}{Для того, чтобы научиться разрабатывать и совершенствовать \textit{технологии искусственного интеллекта}, необходимо приобрести достаточно большой опыт успешного (!) участия в создании различных компонентов комплексной \textit{технологии Искусственного интеллекта}}
\end{SCn}


\newpage

\scnstructheader{Оглавление Стандарта OSTIS-2021}
\scnstartfile

%\vspace{-3\baselineskip}

\end{SCn}

%\DeactivateBG

\normalsize 

\begingroup
\let\clearpage\relax
\tableofcontents
\endgroup

\begin{SCn}
\scnendfile \scninlinesourcecommentpar{Завершили \textit{Оглавление Стандарта OSTIS-2021}}
\end{SCn}

\newpage
%\ActivateBG

\begin{SCn}

\scnnote{Подчеркнем, что в \textit{Оглавлении Публикации Документации Технологии OSTIS-2021} отсутствуют номера разделов.\\
 Это обусловлено тем, что количество и порядок разделов, а также номера их страниц актуальны только для текущего состояния данного внешнего текста базы знаний. База знаний эволюционирует независимо от вводимых в неё знаний, которые в общем случае разрабатываются разными авторами и могут иметь разный объем. В результате такой эволюции появляются новые разделы базы знаний, некоторые разделы могут "переместиться"\ и, соответственно, поменять приписываемый им номер.\\
Это обусловлено эволюцией самой базы знаний, а также тем, какой раздел базы знаний отображается в виде внешнего текста}

\bigskip
\scnaddlevel{1}
\scnheaderlocal{следует отличать*}
\scnhaselementset{Документация Технологии OSTIS;публикация Документации Технологии OSTIS;файл Документации Технологии OSTIS
\scnaddlevel{1}
    \scnidtf{файл соответствующей версии Документации Технологии OSTIS}
    \scnidtf{файл соответствующей версии стандарта Технологии OSTIS}
\scnaddlevel{-1}
}
\scnaddlevel{-1}

\begin{SCn}
	\scnheader{Стандарт OSTIS}
	\scnrelfromvector{общие принципы организации эволюционных работ}{\scnfileitem{Формирование работоспособного \textit{Авторского коллектива Стандарта OSTIS}}
		;\scnfileitem{Формирование \textit{Редакционной коллегии Стандарта OSTIS} для контроля целостности и качества \textit{Стандарта OSTIS} в \uline{каждый} момент времени}
		;\scnfileitem{Формирование \textit{Консорциума OSTIS} для международного продвижения \textit{Стандарта OSTIS}, для взаимодействия с международными структурами, занимающимися стандартизацией \textit{интеллектуальных компьютерных систем} и \textit{технологий Искусственного интеллекта}};
		\scnfileitem{Повышение качества \textit{подготовки специалистов в области	Искусственного интеллекта} в вузах РБ (БГУИР, БрГТУ, БГУ, БНТУ, ГрГТУ, ПГУ) путём: 
			\begin{scnitemize}
				\item тесного сотрудничества и унификации \textit{подготовки специалистов в области Искусственного интеллекта} в разных вузах; 
				\item интеграции учебно-методических материалов в состав \textit{Стандарта OSTIS};
				\item непосредственного подключения студентов, магистрантов и аспирантов к реальному процессу эволюции \textit{Стандарта OSTIS}, т.е. путём непосредственного включения студентов, магистрантов и аспирантов в состав \textit{Авторского коллектива Стандарта OSTIS} со всеми вытекающими отсюда возможностями, правами и обязанностями.
		\end{scnitemize}};
		\scnfileitem{Перед началом каждой (ежегодной) \textit{конференции OSTIS} осуществлять издание очередной \textit{официальной версии Стандарта OSTIS}, в которой отражаются основные изменения и дополнения \textit{Стандарта OSTIS}, внесенные в \textit{Стандарт OSTIS} за истёкший год после проведения предыдущей\textit{ конференции OSTIS}. При этом речь идет не только о содержательных (семантических) изменениях, но и об изменениях структуризации материала, изменениях в правилах и стиле оформления материала.}
		;\scnfileitem{Существенно повысить уровень конструктивности и полезности каждой \textit{конференции OSTIS} для ускорения темпов \textit{эволюции стандарта OSTIS}.\\ Каждая \textit{конференция OSTIS} должна быть посвящена:\begin{scnitemize}
				\item подведению итогов \textit{эволюции Стандарта OSTIS} за истёкший год;
				\item анализу текущего состояния \textit{Стандарта OSTIS};
				\item уточнению наиболее актуальных направлений эволюции \textit{Стандарта OSTIS} (в первую очередь -- на следующий год).
	\end{scnitemize}}}
	\scnaddlevel{1}
	\scnnote{Таким образом, \textit{конференции OSTIS} должны стать ежегодной площадкой для согласования и координации деятельности в направлении \textit{эволюции Стандарта OSTIS}, а также в направлении \textit{подготовки специалистов в области Искусственного интеллекта}.\\ 
		Координация деятельности необходима
		\begin{scnitemize}
			\item не только между различными кафедрами различных \textit{вузов}, осуществляющими \textit{подготовку специалистов в области Искусственного интеллекта}, 
			\item но и между различными членами и группами \textit{Авторского коллектива Стандарта OSTIS}, 
			\item a также между \textit{Авторским коллективом Стандарта OSTIS}, \textit{Редакционной коллегией Стандарта OSTIS} и \textit{Консорциумом OSTIS}.
	\end{scnitemize}}
	\scnaddlevel{-1}
	\scnrelfromvector{план издания официальных версий}{Стандарт OSTIS-2021 \\
		\scnaddlevel{1}
		\scnidtf{Официальная версия \textit{Стандарта OSTIS}, публикуемая (издаваемая) до начала проведения конференции OSTIS-2021 (16-18 сентября 2021 года)}\\ 
		\scniselement{текст, построенный на основе \textit{русскоязычной терминологии}}
		\scnaddlevel{1}
		\scnnote{При этом возможны некоторые англоязычные заимствования -- SC-код, sc-текст и др.}
		\scnaddlevel{-2}
		;Стандарта OSTIS-2022
		\scnaddlevel{1}
		\scnidtf{Официальная версия \textit{Стандарта OSTIS}, публикуемая до Конференции OSTIS-2022 (апрель 2022 года)}
		\scniselement{текст, построенный на основе \textit{русскоязычной терминологии}}
		\scnaddlevel{-1}
		;Стандарта OSTIS-2023
		\scnaddlevel{1}
		\scnidtf{\uline{Специальное} \uline{англоязычное} официальное издание версии \textit{Стандарта OSTIS}, публикуемое до \textit{Конференции OSTIS-2023} (апрель 2023 года) и ориентированное на широкий круг \uline{международной} научно-технической общественности}
		\scniselement{текст, построенный на основе \textit{англоязычной терминологии}}
		\scnnote{Данное англоязычное издание \textit{Стандарта OSTIS} рассматривается нами как повод, визитная карточка, приглашение к переговорам с зарубежными коллегами и организациями, занимающимися стандартизацией \textit{интеллектуальных компьютерных систем} и технологий}
		\scnnote{В перспективе по мере возникновения необходимости мы будем переиздавать (в расширенном и дополнительном варианте) на английском языке некоторые версии \textit{Стандарта OSTIS} для активизации различного рода международных переговоров.}
		\scnnote{При ежегодном переиздании версий \textit{Стандарта OSTIS} для "внутреннего пользования"{} -- для работы \textit{Редакционной коллеги Стандарта OSTIS} и \textit{Авторского коллектива Стандарта OSTIS} мы будем ориентироваться на \uline{интеграцию} использования как русскоязычной, так и англоязычной терминологии, что фактически означает включение в состав \textit{Стандарта OSTIS} русско-английского и англо-русского словарей}
		\scnaddlevel{-1}
		;Стандарт OSTIS-2023
		\scnaddlevel{1}
		\scnidtf{Официальная версия \textit{Стандарта OSTIS}, публикуемая до \textit{конференции OSTIS-2023} (апрель 2023 года) и использующая в равной степени как русскоязычную, так и англоязычную терминологию}
		\scniselement{текст, построенный на основе интеграции русскоязычной и англоязычной терминологии}
		\scnaddlevel{-1}
		;Стандарт OSTIS-2024 
		;Стандарт OSTIS-2025
		;...
	}
\end{SCn}
\end{SCn}
