\begin{SCn}

\scsuperchapter

\scnsectionheader{Документация Технологии OSTIS}
\label{super_char}
\scnstartsubstruct

\scnidtf{Документация технологии онтологического проектирования, производства и эксплуатации семантически совместимых гибридных интеллектуальных компьютерных систем}
\scnidtf{Полное описание текущего состояния \textit{Технологии OSTIS}, представленное в виде раздела \textit{базы знаний}, построенной по \textit{Технологии OSTIS}}
\scnidtf{Основной раздел \textit{базы знаний Метасистемы IMS.ostis}, которая предназначена для комплексной поддержки онтологического проектирования семантически совместимых \textit{гибридных интеллектуальных компьютерных систем}}
\scniselement{раздел базы знаний}
    \scnaddlevel{1}
    \scnidtf{раздел внутреннего представления \textit{базы знаний ostis-системы} -- \textit{интеллектуальной компьютерной системы}, построенной по \textit{Технологии OSTIS}}
    \scnaddlevel{-1}
\scnidtf{Описание \textit{Технологии OSTIS} (Open Semantic Technology for Intelligent Systems), представленное в виде раздела \textit{базы знаний ostis-системы}}
\scnidtf{Описание \textit{Технологии OSTIS}, представленное в виде \textit{раздела базы знаний} на внутреннем языке \textit{ostis-систем} и обладающее достаточной полнотой для использования этой технологии разработчиками \textit{интеллектуальных компьютерных систем}}
\scnrelfromset{основные авторы}{Голенков В.В.;Гулякина Н.А.;Шункевич Д.В.}
\scnrelfrom{научный редактор}{Голенков В.В.}
\scnrelfromset{рецензенты}{Курбацкий А.Н.;Дудкин А.А.}
\scnrelfrom{финансовая поддержка}{Intelligent Semantic Systems Ltd.}
\scnaddlevel{1}
\scnaddlevel{-1}
\scnreltovector{конкатенация разделов}{Вводная часть Документации Технологии OSTIS;Обоснование Технологии OSTIS;Предметная область и онтология Технологии OSTIS;Заключительная часть Документации Технологии OSTIS;Библиографическая часть Документации Технологии OSTIS}

\bigskip

\scnaddlevel{1}
\scnheaderlocal{конкатенация подразделов*}
\scnexplanation{Бинарное ориентированное \textit{отношение}, каждая \textit{пара} которого связывает \textit{знак} некоторого \textit{раздела базы знаний} либо знак \textit{файла}, содержащего некоторый \textit{документ}, с упорядоченным множеством всех \uline{непосредственных} подразделов указанного \textit{раздела базы знаний} или указанного \textit{документа}*}
	\scnaddlevel{1}
	\scnnote{Подчеркнем, что в указанное \textit{упорядоченное множество} подразделов заданного \textit{раздела базы знаний} или заданного \textit{документа} подразделы подразделов этого \textit{раздела базы знаний} (или этого \textit{документа}) \uline{не входят}.}
	\scnaddlevel{-1}
\scnaddlevel{-1}

\scnheader{Документация Технологии OSTIS}
\scntext{эпиграф}{From data science to knowledge science}
\scntext{аннотация}{В настоящее время информатика преодолевает важнейший этап своего развития --- переход от информатики данных (data science) к информатике знаний (knowledge science), где акцентируется внимание на \uline{семантических} аспектах представления и обработки \textit{знаний}.\\
Без фундаментального анализа такого перехода невозможно решить многие проблемы, связанные с управлением \textit{знаниями}, экономикой \textit{знаний}, с \textit{семантической совместимостью интеллектуальных компьютерных систем}.\\
Основной особенностью \textit{Технологии OSTIS} является ориентация на использование компьютеров нового поколения, специально предназначенных для  реализации семантически совместимых гибридных \textit{интеллектуальных компьютерных систем}. Предлагаемая \textit{Документация Технологии OSTIS} оформлена в виде \textit{раздела базы знаний} специальной интеллектуальной компьютерной \textit{Метасистемы IMS.ostis} (Intelligent MetaSystem for ostis-systems), которая построена по Технологии OSTIS и представляет собой постоянно совершенствуемый интеллектуальный \textit{портал научно-технических знаний}, который поддерживает перманентную эволюцию \textit{Документации Технологии OSTIS}, а также разработку различных \textit{ostis-систем} (интеллектуальных компьютерных систем, построенных по \textit{Технологии OSTIS}).}
\scnrelfrom{публикация ostis-документации}{Публикация Документации Технологии OSTIS-2021}
	\scnaddlevel{1}
	\scnidtf{Предлагаемое Вашему вниманию издание внешнего представления \textit{раздела базы знаний} \scnbigspace \textit{Метасистемы IMS.ostis}, посвященного комплексному описанию \textit{Технологии OSTIS} и отражающего версию указанного раздела, соответствующую весеннему периоду 2021 года}
	\scnaddlevel{-1}

\bigskip

\scnaddlevel{1}
\scnheaderlocal{публикация ostis-документации*}
\scnidtfexp{Бинарное ориентированное \textit{отношение}, каждая \textit{пара} которого связывает знак некоторого \textit{раздела базы знаний} со знаком \textit{файла}, который является внешним представлением указанного раздела, а также является либо копией электронной публикации материалов этого раздела, либо оригинал-макетом бумажной публикации указанных материалов* }
\scnaddlevel{-1}

\scnheader{Публикация Документации Технологии OSTIS-2021}
\scnidtf{Издание Документации Технологии OSTIS-2021}
\scnidtf{Первое издание (публикация) Внешнего представления Документации Технологии OSTIS в виде книги}
\scniselement{публикация}
	\scnaddlevel{1}
	\scnidtf{библиографический источник}
	\scnaddlevel{-1}
\scniselement{бумажное издание}
\scniselement{научное издание}
\scnrelfrom{рекомендация издания}{***}
\scnrelfrom{финансовая поддержка}{Intelligent Semantic Systems Ltd.}
\scnaddlevel{1}
\scnsourcecommentpar{Имеется в виду финансовая поддержка издания данной публикации}
\scnaddlevel{-1}
\scnrelfrom{издательство}{***}
\scniselementrole{УДК}{\scnfileshort{***}}
\scniselementrole{ББК}{\scnfileshort{***}}
\scniselementrole{ISBN}{\scnfileshort{***}}
\scnrelfrom{технический редактор}{***}
\scnrelfrom{художественный редактор}{***}
\scnrelfrom{корректор}{***}
\scnrelfrom{верстка}{***}
\scnrelfrom{дата подписаний в печать}{***}
\scnrelfrom{тираж}{***}
\scnrelfrom{оглавление публикации ostis-документации*}{Оглавление Публикации Документации Технологии OSTIS-2020}

\bigskip

\scnaddlevel{1}
\scnheaderlocal{оглавление публикации ostis-документации*}
\scnidtfexp{Бинарное ориентированное \textit{отношение}, каждая \textit{пара} которого связывает знак некоторого \textit{раздела базы знаний} либо знак \textit{файла}, содержащего некоторый \textit{документ}, с описанием иерархии \uline{всех} \textit{разделов}, входящих в состав указанного \textit{раздела базы знаний} либо указанного \textit{документа}* }

\scnheader{следует отличать*}
\scnhaselementset{конкатенация разделов*;оглавление*}
	\scnaddlevel{1}
	\scnexplanation{Отношение \textit{конкатенация разделов*} описывает декомпозицию заданного раздела или документа только на один шаг --- на непосредственные подразделы заданного раздела базы знаний или документа.}
	\scnaddlevel{-1}

\newpage

\scnstructheader{Оглавление Публикации Документации Технологии OSTIS-2020}
\scnstartfile

%\vspace{-3\baselineskip}

\end{SCn}

\DeactivateBG

\doublespacing
\normalsize 

\begingroup
\let\clearpage\relax
\tableofcontents
\endgroup

\singlespacing

\begin{SCn}
\scnendfile \scninlinesourcecommentpar{Завершили \textit{Оглавление Публикации Документации Технологии OSTIS-2020}}
\end{SCn}

\newpage
\ActivateBG

\begin{SCn}

\scnnote{Подчеркнем, что в \textit{Оглавлении Публикации Документации Технологии OSTIS-2020} номера разделов носят условный характер и \uline{не входят в состав базы знаний}.\\
Обратите внимание на то, что в оглавлении номера разделов и номера их страниц в состав базы знаний не входят. Это обусловлено тем, что номера разделов и номера их страниц актуальны только для текущего состояния данного внешнего текста базы знаний. База знаний эволюционирует независимо от вводимых в неё знаний, которые в общем случае разрабатываются разными авторами и могут иметь разный объем. В результате такой эволюции появляются новые разделы базы знаний, некоторые разделы могут "переместиться"\ и, соответственно, поменять приписываемый им номер.\\
Это обусловлено эволюцией самой базы знаний, а также тем, какой раздел базы знаний отображается в виде внешнего текста}

\scnheader{следует отличать*}
\scnhaselementset{Документация Технологии OSTIS;Публикация Документации Технологии OSTIS;Файл Документации Технологии OSTIS
\scnaddlevel{1}
    \scnidtf{Файл текущей версии Документации Технологии OSTIS}
    \scnidtf{Файл текущей версии стандарта Технологии OSTIS}
\scnaddlevel{-1}
}

\scnendstruct \scninlinesourcecommentpar{Завершили начало Документации Технологии OSTIS}

\end{SCn}

