\begin{SCn}

\scsuperchapter

\scnsectionheader{Документация Технологии OSTIS}
\label{super_char}
\scnstartsubstruct

\scsupersectionbeginning

\scnstartsubstruct

\scnheader{Документация Технологии OSTIS}
\scnidtf{Полное описание текущего состояния Технологии OSTIS, представленное в виде раздела базы знаний, построенной по Технологии OSTIS}
\scnidtf{Основной раздел базы знаний Метасистемы IMS.ostis, которая предназначена для комплексной поддержки онтологического проектирования семантически совместимых гибридных интеллектуальных компьютерных систем}
\scniselement{раздел базы знаний}
    \scnaddlevel{1}
    \scnidtf{раздел внутреннего представления базы знаний ostis-системы --- интеллектуальной компьютерной системы, построенной по Технологии OSTIS}
    \scnaddlevel{-1}
\scnidtf{Описание Технологии OSTIS (Open Semantic Technology for Intelligent Systems), представленное в виде раздела базы знаний ostis-системы (интеллектуальной компьютерной системы, построенной по Технологии OSTIS)}
\scnidtf{Описание Технологии OSTIS, представленное в виде раздела базы знаний на внутреннем языке ostis-систем (в SC-коде) и обладающее достаточной полнотой для использования этой технологии разработчиками интеллектуальных компьютерных систем}
\scnrelfromset{основные авторы}{Голенков В.В.;Гулякина Н.А.;Шункевич Д.В.}
\scnrelfrom{научный редактор}{Голенков В.В.}
\scnrelfromset{рецензенты}{***;***}
\scnrelfrom{финансовая поддержка}{ISS Corp}
\scnaddlevel{1}
\scnsourcecommentpar{Имеется в виду финансовая поддержка разработки указываемого раздела базы знаний}\\
\scnaddlevel{-1}
\scnreltovector{конкатенация разделов}{Вводная часть Документации Технологии OSTIS;Обоснование Технологии OSTIS;Предметная область и онтология Технологии OSTIS;Заключительная часть Документации Технологии OSTIS;Библиографическая часть Документации Технологии OSTIS}

\bigskip

\scnsourcecomment{Поясним смысл отношения ``конкатенация разделов*''~}
\scnheader{конкатенация разделов*}
\scnexplanation{Бинарное ориентированное отношение, каждая пара которого связывает знак некоторого раздела базы знаний либо знак файла, содержащего некоторый документ, с упорядоченным множеством всех \uline{непосредственных} подразделов указанного раздела базы знаний или указанного документа*}
	\scnaddlevel{1}
	\scnnote{Подчеркнем, что в указанное упорядоченное множество подразделов заданного раздела базы знаний или заданного документа подразделы подразделов этого раздела базы знаний (или этого документа) \uline{не входят}.}
	\scnaddlevel{-1}

\scnheader{Документация Технологии OSTIS}
\scntext{эпиграф}{From data science to knowledge science}
\scnrelfrom{предисловие}{Предисловие к Документации Технологии OSTIS}
\scnreltoset{благодарности}{***}
\scnrelfrom{введение}{Вводная часть Документации Технологии OSTIS}
\scnrelfrom{заключение}{Заключительная часть Документации Технологии OSTIS}
\scnrelfrom{библиография}{Библиографическая часть Документации Технологии OSTIS}
\scntext{аннотация}{В настоящее время информатика преодолевает важнейший этап своего развития --- переход от информатики данных (data science) к информатике знаний (knowledge science), где акцентируется внимание на \uline{семантических} аспектах представления и обработки знаний.\\
Без фундаментального анализа такого перехода невозможно решить многие проблемы, связанные с управлением знаниями, экономикой знаний, с семантической совместимостью интеллектуальных компьютерных систем.\\
Основной особенностью Технологии OSTIS является ориентация на использование компьютеров нового поколения, специально предназначенных для  реализации семантически совместимых гибридных \textit{интеллектуальных компьютерных систем}. Предлагаемая Документация Технологии OSTIS оформлена в виде раздела базы знаний специальной интеллектуальной компьютерной Метасистемы IMS.ostis (Intelligent MetaSystem for ostis-systems), которая построена по Технологии OSTIS и представляет собой постоянно совершенствуемый интеллектуальный портал научно-технических знаний, который поддерживает перманентную эволюцию Документации Технологии OSTIS, а также разработку различных ostis-систем (интеллектуальных компьютерных систем, построенных по Технологии OSTIS).}
\scnrelfromvector{ключевые сущности}{Технология OSTIS;ostis-система;IMS.ostis;SC-код}
	\scnaddlevel{1}
	\scnnote{Здесь перечислены sc-элементы, являющиеся знаками ключевых сущностей (в том числе понятий), описываемых в Документации Технологии OSTIS.\\
	При этом указанный перечень sc-элементов упорядочивается по алфавиту их основных внешних идентификаторов. Факт упорядочивания указывается угловыми скобками.}
	\scnaddlevel{-1}
\scnreltovector{используемые сокращения}{
    \scsabrreviation{б.з.}{база знаний}  ;IMS.ostis;\scsabrreviation{OSTIS}{Open Semantic Technology for Intelligent Systems};ostis-система;sc-текст;SC-код;SCg-код;SCs-код;SCn-код;sc.g-текст, sc.s-текст;sc.n-текст;ея-файл; ея-текст;\scsabrreviation{и.н.с.}{искусственная нейронная сеть};\scsabrreviation{и.к.с.}{интеллектуальная компьютерная система};\scsabrreviation{к.с.}{ компьютерная система};\scsabrreviation{и.с.}{ интеллектуальная система (не обязательно искусственная)};\scsabrreviation{SCL}{Semantic Code Logical};ostis-;sc-;и т.д.;и т.п.;\scsabrreviation{ея}{естественно-языковой};\scsabrreviation{фр-нт}{фрагмент}}
\scnrelfrom{публикация}{Публикация Документации Технологии OSTIS-2020}
	\scnaddlevel{1}
	\scnidtf{Предлагаемое Вашему вниманию издание внешнего представления раздела базы знаний Метасистемы IMS.ostis, посвященного комплексному описанию Технологии OSTIS и отражающего версию указанного раздела, соответствующую осеннему периоду 2020 года}
	\scnaddlevel{-1}

\bigskip

\scnsourcecomment{Поясним смысл отношения ``публикация*''~}
\scnheader{публикация*}
\scnexplanation{Бинарное ориентированное отношение, каждая пара которого связывает знак некоторого раздела базы знаний со знаком файла, который является внешним представлением указанного раздела, а также является либо копией электронной публикации материалов этого раздела, либо оригинал-макетом бумажной публикации указанных материалов*}

\scnheader{Публикация Документации Технологии OSTIS-2020}
\scnidtf{Издание Документации Технологии OSTIS-2020}
\scnidtf{ГоленковВ.В..ДокуменТехнOSTIS-2020кн}
	\scnaddlevel{1}
	\scniselement{стандартный для Технологии OSTIS вид идентификатора библиографического источника}
	\scnaddlevel{-1}
\scnidtf{Первое издание (публикация) Внешнего представления Документации Технологии OSTIS в виде книги}
\scniselement{публикация}
	\scnaddlevel{1}
	\scnidtf{библиографический источник}
	\scnaddlevel{-1}
\scniselement{бумажное издание}
\scniselement{научное издание}
\scniselement{монография}
\scnrelfrom{финансовая поддержка}{ISS}
\scnaddlevel{1}
\scnsourcecommentpar{Имеется в виду финансовая поддержка издания данной публикации}
\scnaddlevel{-1}
\scnrelfrom{издательство}{***}
\scniselementrole{УДК}{\scnfileshort{***}}
\scniselementrole{ББК}{\scnfileshort{***}}
\scniselementrole{ISBN}{\scnfileshort{***}}
\scnrelfrom{технический редактор}{***}
\scnrelfrom{художественный редактор}{***}
\scnrelfrom{корректор}{***}
\scnrelfrom{верстка}{***}
\scnrelfrom{дата подписаний в печать}{***}
\scnrelfrom{рекомендация издания}{***}
\scnrelfrom{тираж}{***}
\scnrelfrom{оглавление}{Оглавление Публикации Документации Технологии OSTIS-2020}

\bigskip

\scnsourcecomment{Поясним смысл отношения ``оглавление*''~}
\scnheader{оглавление}
\scnidtfexp{Бинарное ориентированное отношение, каждая пара которого связывает знак некоторого раздела базы знаний либо знак файла, содержащего некоторый документ, с описанием иерархии \uline{всех} разделов, входящих в состав указанного раздела базы знаний либо указанного документа*}

\scnheader{следует отличать*}
\scnhaselementset{конкатенация разделов*;оглавление*}
	\scnaddlevel{1}
	\scnexplanation{Отношение \textit{конкатенация разделов*} описывает декомпозицию заданного раздела или документа только на один шаг --- на непосредственные подразделы заданного раздела базы знаний или документа.}
	\scnaddlevel{-1}

\newpage

\scnstructheader{Оглавление Публикации Документации Технологии OSTIS-2020}
\scnstartfile

%\vspace{-3\baselineskip}

\end{SCn}

\DeactivateBG

\doublespacing
\normalsize 

\begingroup
\let\clearpage\relax
\tableofcontents
\endgroup

\singlespacing

\begin{SCn}
\scnendfile \scninlinesourcecommentpar{Завершили \textit{Оглавление Публикации Документации Технологии OSTIS-2020}}
\end{SCn}

\newpage
\ActivateBG

\begin{SCn}

\scnnote{Подчеркнем, что в \textit{Оглавлении Публикации Документации Технологии OSTIS-2020} номера разделов носят условный характер и \uline{не входят в состав базы знаний}.\\
Обратите внимание на то, что в оглавлении номера разделов и номера их страниц в состав базы знаний не входят. Это обусловлено тем, что номера разделов и номера их страниц актуальны только для текущего состояния данного внешнего текста базы знаний. База знаний эволюционирует независимо от вводимых в неё знаний, которые в общем случае разрабатываются разными авторами и могут иметь разный объем. В результате такой эволюции появляются новые разделы базы знаний, некоторые разделы могут "переместиться"\ и, соответственно, поменять приписываемый им номер.\\
Это обусловлено эволюцией самой базы знаний, а также тем, какой раздел базы знаний отображается в виде внешнего текста}

\scnheader{следует отличать*}
\scnhaselementset{Документация Технологии OSTIS;Публикация Документации Технологии OSTIS;Файл Документации Технологии OSTIS
\scnaddlevel{1}
    \scnidtf{Файл текущей версии Документации Технологии OSTIS}
    \scnidtf{Файл текущей версии стандарта Технологии OSTIS}
\scnaddlevel{-1}
}

\scnendstruct \scninlinesourcecommentpar{Завершили начало Документации Технологии OSTIS}

\end{SCn}

