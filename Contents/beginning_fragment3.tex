\begin{SCn}
	\scnheader{Стандарт OSTIS}
	\scnrelfromvector{общие принципы организации эволюционных работ}{\scnfileitem{Формирование работоспособного \textit{Авторского коллектива Стандарта OSTIS}}
		;\scnfileitem{Формирование \textit{Редакционной коллегии Стандарта OSTIS} для контроля целостности и качества \textit{Стандарта OSTIS} в \uline{каждый} момент времени}
		;\scnfileitem{Формирование \textit{Консорциума OSTIS} для международного продвижения \textit{Стандарта OSTIS}, для взаимодействия с международными структурами, занимающимися стандартизацией \textit{интеллектуальных компьютерных систем} и \textit{технологий Искусственного интеллекта}};
		\scnfileitem{Повышение качества \textit{подготовки специалистов в области	Искусственного интеллекта} в вузах РБ (БГУИР, БрГТУ, БГУ, БНТУ, ГрГТУ, ПГУ) путём: 
			\begin{scnitemize}
				\item тесного сотрудничества и унификации \textit{подготовки специалистов в области Искусственного интеллекта} в разных вузах; 
				\item интеграции учебно-методических материалов в состав \textit{Стандарта OSTIS};
				\item непосредственного подключения студентов, магистрантов и аспирантов к реальному процессу эволюции \textit{Стандарта OSTIS}, т.е. путём непосредственного включения студентов, магистрантов и аспирантов в состав \textit{Авторского коллектива Стандарта OSTIS} со всеми вытекающими отсюда возможностями, правами и обязанностями.
		\end{scnitemize}};
		\scnfileitem{Перед началом каждой (ежегодной) \textit{конференции OSTIS} осуществлять издание очередной \textit{официальной версии Стандарта OSTIS}, в которой отражаются основные изменения и дополнения \textit{Стандарта OSTIS}, внесенные в \textit{Стандарт OSTIS} за истёкший год после проведения предыдущей\textit{ конференции OSTIS}. При этом речь идет не только о содержательных (семантических) изменениях, но и об изменениях структуризации материала, изменениях в правилах и стиле оформления материала.}
		;\scnfileitem{Существенно повысить уровень конструктивности и полезности каждой \textit{конференции OSTIS} для ускорения темпов \textit{эволюции стандарта OSTIS}.\\ Каждая \textit{конференция OSTIS} должна быть посвящена:\begin{scnitemize}
				\item подведению итогов \textit{эволюции Стандарта OSTIS} за истёкший год;
				\item анализу текущего состояния \textit{Стандарта OSTIS};
				\item уточнению наиболее актуальных направлений эволюции \textit{Стандарта OSTIS} (в первую очередь -- на следующий год).
	\end{scnitemize}}}
	\scnaddlevel{1}
	\scnnote{Таким образом, \textit{конференции OSTIS} должны стать ежегодной площадкой для согласования и координации деятельности в направлении \textit{эволюции Стандарта OSTIS}, а также в направлении \textit{подготовки специалистов в области Искусственного интеллекта}.\\ 
		Координация деятельности необходима
		\begin{scnitemize}
			\item не только между различными кафедрами различных \textit{вузов}, осуществляющими \textit{подготовку специалистов в области Искусственного интеллекта}, 
			\item но и между различными членами и группами \textit{Авторского коллектива Стандарта OSTIS}, 
			\item a также между \textit{Авторским коллективом Стандарта OSTIS}, \textit{Редакционной коллегией Стандарта OSTIS} и \textit{Консорциумом OSTIS}.
	\end{scnitemize}}
	\scnaddlevel{-1}
	\scnrelfromvector{план издания официальных версий}{Стандарт OSTIS-2021 \\
		\scnaddlevel{1}
		\scnidtf{Официальная версия \textit{Стандарта OSTIS}, публикуемая (издаваемая) до начала проведения конференции OSTIS-2021 (16-18 сентября 2021 года)}\\ 
		\scniselement{текст, построенный на основе \textit{русскоязычной терминологии}}
		\scnaddlevel{1}
		\scnnote{При этом возможны некоторые англоязычные заимствования -- SC-код, sc-текст и др.}
		\scnaddlevel{-2}
		;Стандарта OSTIS-2022
		\scnaddlevel{1}
		\scnidtf{Официальная версия \textit{Стандарта OSTIS}, публикуемая до Конференции OSTIS-2022 (апрель 2022 года)}
		\scniselement{текст, построенный на основе \textit{русскоязычной терминологии}}
		\scnaddlevel{-1}
		;Стандарта OSTIS-2023
		\scnaddlevel{1}
		\scnidtf{\uline{Специальное} \uline{англоязычное} официальное издание версии \textit{Стандарта OSTIS}, публикуемое до \textit{Конференции OSTIS-2023} (апрель 2023 года) и ориентированное на широкий круг \uline{международной} научно-технической общественности}
		\scniselement{текст, построенный на основе \textit{англоязычной терминологии}}
		\scnnote{Данное англоязычное издание \textit{Стандарта OSTIS} рассматривается нами как повод, визитная карточка, приглашение к переговорам с зарубежными коллегами и организациями, занимающимися стандартизацией \textit{интеллектуальных компьютерных систем} и технологий}
		\scnnote{В перспективе по мере возникновения необходимости мы будем переиздавать (в расширенном и дополнительном варианте) на английском языке некоторые версии \textit{Стандарта OSTIS} для активизации различного рода международных переговоров.}
		\scnnote{При ежегодном переиздании версий \textit{Стандарта OSTIS} для "внутреннего пользования"{} -- для работы \textit{Редакционной коллеги Стандарта OSTIS} и \textit{Авторского коллектива Стандарта OSTIS} мы будем ориентироваться на \uline{интеграцию} использования как русскоязычной, так и англоязычной терминологии, что фактически означает включение в состав \textit{Стандарта OSTIS} русско-английского и англо-русского словарей}
		\scnaddlevel{-1}
		;Стандарт OSTIS-2023
		\scnaddlevel{1}
		\scnidtf{Официальная версия \textit{Стандарта OSTIS}, публикуемая до \textit{конференции OSTIS-2023} (апрель 2023 года) и использующая в равной степени как русскоязычную, так и англоязычную терминологию}
		\scniselement{текст, построенный на основе интеграции русскоязычной и англоязычной терминологии}
		\scnaddlevel{-1}
		;Стандарт OSTIS-2024 
		;Стандарт OSTIS-2025
		;...
	}
\end{SCn}