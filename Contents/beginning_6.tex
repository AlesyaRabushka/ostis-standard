\begin{SCn}
%Begin

\scnfragmentcaption

\scnheader{Стандарт OSTIS-2021}
\scnrelto{официальная версия}{Стандарт OSTIS}
\scnnote{Данная официально изданная версия \textit{Стандарта OSTIS}, которую Вы держите в руках, занимает особое место:
\begin{scnitemize}
\item Во-первых, это первый опыт издания (публикации) подобного документа, в рамках которого необходимо обеспечивать, с одной стороны, строгую формальность, а, с другой стороны, интуитивное и адекватное понимание формальных текстов со стороны читателей;
\item Во-вторых, данный текст является описанием условно выделенной первой версии \textit{Стандарта OSTIS} (\textit{Стандарта OSTIS-2021}), в рамках которого представлены далеко не все разделы \textit{Стандарта OSTIS}. Эти разделы будут представлены в последующих версиях \textit{Стандарта OSTIS} (в \textit{Стандарте OSTIS-2022}, в \textit{Стандарте OSTIS-2023} и т.д.);
\item Особенностью \textit{публикации} (издания) Стандарта OSTIS версии OSTIS-2021, как, впрочем, и всех последующих версий, является то, что она оформлена в виде \uline{внешнего представления} основной части \textit{базы знаний} специальной \textit{ostis-системы}, которая предназначена для комплексной поддержки проектирования \uline{семантически совместимых} \textit{ostis-систем}. Эту систему мы назвали \textbf{\textit{Метасистемой IMS.ostis}} (Intelligent MetaSystem for ostis-systems). Последовательность изложения материала во внешнем представлении \textit{базы знаний} не является единственно возможным маршрутом прочтения (просмотра) \textit{базы знаний}. Каждый читатель, войдя в \textbf{\textit{Метасистему IMS.ostis}}, может выбрать любой другой маршрут навигации по этой \textit{базе знаний}, задавая указанной метасистеме те \textit{вопросы}, которые в текущий момент его интересуют. Таким образом, читая предлагаемый вашему вниманию текст и одновременно работая с \textbf{\textit{Метасистемой IMS.ostis}}, можно значительно быстрее усвоить детали \textbf{\textit{Технологии OSTIS}} и значительно быстрее приступить к непосредственному использованию указанной технологии. Этому также способствует большое количество примеров семантических моделей различных фрагментов \textit{интеллектуальных компьютерных систем};
\item Основной семантический вид \textit{разделов баз знаний \textbf{ostis-систем}} -- это формальное представление различных \textbf{\textit{предметных областей}} вместе с соответствующими им \textbf{\textit{онтологиями}}. При этом явно указываются связи между этими \textbf{\textit{предметными областями} и \textit{онтологиями}}. Таким образом, \textit{база знаний} \textbf{\textit{Метасистемы IMS.ostis}}, как и любых других \textit{интеллектуальных компьютерных систем}, построенных по \textbf{\textit{Технологии OSTIS}}, представляет собой иерархическую систему связанных между собой формальных моделей \textit{предметных областей} и соответствующих им \textit{онтологий}. Соответственно этому структурирован и текст \textit{Стандарта OSTIS-2021};
\item В основе \textit{Технологии OSTIS} лежит предлагаемая нами унификация \textit{интеллектуальных компьютерных систем}, основанная, в свою очередь, на \textit{смысловом представлении знаний} в \textit{памяти интеллектуальных компьютерных систем}. Таким образом, данную \textit{публикацию} \textit{Стандарта OSTIS-2021} можно рассматривать как версию \textit{стандарта} семантических моделей \textit{интеллектуальных компьютерных систем}. Последующие \textit{публикации}, посвящённые детальному описанию различных компонентов \textit{Технологии OSTIS}, будут также оформляться как внешнее представление соответствующих \textit{разделов базы знаний} \scnbigspace \textit{Метасистемы IMS.ostis} и будут отражать следующие этапы развития  \textit{Технологии OSTIS}, следующие версии этой технологии, и, соответственно, следующие версии \textit{Метасистемы IMS.ostis};
\item Все основные положения \textit{Технологии OSTIS} рассматривались и обсуждались на ежегодных \textit{конференциях OSTIS}, которые стали важным стимулирующим фактором становления и развития \textit{Технологии OSTIS}. Мы благодарим всех активных участников этих конференций;
\item Важной задачей \textit{Стандарта OSTIS-2021} была выработка стилистики формализованного представления научно-технической информации, которая одновременно была бы понятна как человеку, так и интеллектуальной компьютерной системе. По сути это принципиально новый подход к оформлению научно-технических результатов, позволяющий:
\begin{scnitemizeii}
 \item существенно повысить уровень автоматизации анализа качества (корректности, целостности) научно-технической информации;
 \item интеллектуальным компьютерным системам непосредственно (без какой-либо дополнительной "ручной"{} доработки) использовать информацию (знания), содержащуюся в разработанных специалистами документах;
 \item существенно упростить согласование точек зрения различных специалистов, входящих в коллектив разработчиков той или  иной научно-технической документации.
\end{scnitemizeii}
\item Для выработки стилистики и формального представления научно-технической информации нам было важно привлечь к обсуждению и анализу материала \textit{Стандарта OSTIS-2021} как можно больше коллег, участвующих в развитии и применении \textit{Технологии OSTIS}. При этом некоторых коллег мы включили в число соавторов соответствующих разделов монографии.
Основной целью написания \textit{Стандарта OSTIS-2021} является создание технологических и организационных предпосылок к принципиально новому  подходу к организации \textit{научно-технической деятельности} в любой области и, в частности, в области создания и перманентного развития комплексной технологии проектирования и производства семантически совместимых интеллектуальных компьютерных систем (\textit{Технологии OSTIS}). Суть указанного подхода заключается  в глубокой конвергенции и интеграции результатов деятельности всех специалистов, участвующих в создании и развитии \textit{Технологии OSTIS}, путем организации коллективной разработки \textit{базы знаний}, являющейся формальным представлением полной \textit{Документации Технологии OSTIS}, отражающей текущее состояние этой технологии.
\end{scnitemize}
}
\scntext{благодарности}{На данном этапе к разработке и оформлению различных разделов \textit{Стандарта OSTIS-2021} текущей версии \textit{Технологии OSTIS} кроме основных авторов были привлечены студенты, магистранты, аспиранты и преподаватели кафедры интеллектуальных информационных технологий Белорусского государственного университета информатики и радиоэлектроники и кафедры интеллектуальных информационных технологий Брестского государственного технического университета, а также сотрудники ОАО «Савушкин продукт» и ООО «Интелиджент семантик системс». Так, например,
\begin{scnitemize}
\item соавторами Раздела ``\nameref{sec:sd_neuronetworks}''{} являются Головко В.А., Ковалёв М.В., Крощенко А.А., Михно Е.В.;
\item соавторами Раздела ``\nameref{sec:sd_ecosys_enterprise}''{} являются Таберко В.В., Иванюк Д.С., Касьяник В.В.;
\item соавторами Раздела ``\nameref{sec:sd_interfaces}''{} являются Садовский М.Е., Захарьев В.А.,  Никифоров С.А., Коршунов Р.А.;
\item соавторами Раздела ``\nameref{sec:sd_ann}''{} являются Головко В.А., Ковалев М.В.;
\end{scnitemize}

\scnauthorcomment{Добавить соавторов}

Благодарим также студентов кафедры Интеллектуальных информационных технологий Белорусского государственного университета информатики и радиоэлектроники за оказание технической помощи при подготовке текста к печати, сотрудников кафедры Интеллектуальных информационных технологий Брестского государственного технического университета, сотрудников ОАО <<Савушкин продукт>>, сотрудников кафедры ВМиП БГУИР, сотрудников кафедры ЭВМ БГУИР, выражаем также благодарность ООО <<Интелиджент семантик системс>> и его генеральному директору Т. Грюневальду за финансовую поддержку работ по развитию \textit{Технологии OSTIS}, также финансовую поддержку издания \textit{Документации Технологии OSTIS}.
}

\scnheader{Стандарт OSTIS-2021}
\scnidtf{Издание Документации Технологии OSTIS-2021}
\scnidtf{Первое издание (публикация) Внешнего представления Документации Технологии OSTIS в виде книги}
\scniselement{публикация}
	\scnaddlevel{1}
\scnidtf{библиографический источник}
\scnaddlevel{-1}
\scniselement{официальная версия Стандарта OSTIS}
\scniselement{бумажное издание}
\scniselement{научное издание}
\scnrelfrom{рекомендация издания}{***}
\scnrelfrom{издательство}{***}
\scniselementrole{УДК}{\scnfileshort{***}}
\scniselementrole{ББК}{\scnfileshort{***}}
\scniselementrole{ISBN}{\scnfileshort{***}}
\scnrelfrom{технический редактор}{***}
\scnrelfrom{художественный редактор}{***}
\scnrelfrom{корректор}{***}
\scnrelfrom{верстка}{***}
\scnrelfrom{дата подписания в печать}{***}
\scnrelfrom{тираж}{***}
\scnrelfrom{оглавление официальной версии Стандарта OSTIS}{Оглавление Стандарта OSTIS-2021}
\bigskip

%End
\end{SCn}

