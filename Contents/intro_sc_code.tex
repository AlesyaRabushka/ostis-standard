
\scsubsection{Введение в описание внутреннего языка ostis-систем}
\label{intro_sc_code}

\begin{SCn}

\scnsectionheader{\currentname}

\scnstartsubstruct

\scnheader{SC-код}
\scnidtf{Внутренний язык ostis-систем}
\scnidtf{Множество sc-текстов}
\scnidtf{sc-текст}
\scnidtf{Множество sc-конструкций}
\scnidtf{Язык унифицированного смыслового представления знаний в памяти интеллектуальных компьютерных систем}
\filemodetrue
\scnrelfromvector{принципы, лежащие в основе}{Знаки (обозначения) всех сущностей, описываемых в \textit{sc-текстах} (текстах SC-кода) представляются в виде синтаксически элементарных (атомарных) фрагментов \textit{sc-текстов} и, следовательно, не имеющих внутренней структуры, не состоящих из более простых фрагментов текста, как, например, имена (термины), которые представляют знаки описываемых сущностей в привычных языках и состоят из букв.;Имена (термины), естественно-языковые тексты и другие информационные конструкции, не являющиеся \textit{sc-текстами}, могут входить в состав \textit{sc-текста}, но только как файлы, описываемые (специфицируемые) \textit{sc-текстами}. Таким образом, в состав базы знаний \textit{интеллектуальной компьютерной системы}, построенной на основе \textit{SC-кода}, могут входить имена (термины), обозначающие некоторые описываемые сущности и представленные соответствующими файлами. Каждый sc-элемент будем называть внутренним обозначением некоторой сущности, а имя этой сущности, представленное соответствующим файлом, будем называть внешним идентификатором (внешним обозначением) этой сущности. При этом каждый именуемый (идентифицируемый) \textit{sc-элемент} связывается дугой, принадлежащей отношению "\textit{\textbf{быть внешним идентификатором*}}, с узлом, содержимым которого является файл идентификатора (в частности, имени), обозначающего ту же сущность, что и указанный выше \textit{sc-элемент}. Внешним обозначением может быть не только имя (термин), но и иероглиф, пиктограмма, озвученное имя, жест. Особо отметим, что внешние обозначения описываемых сущностей в интеллектуальной компьютерной системе, построенной на основе \textit{SC-кода}, используются только (1) для анализа информации, поступающей в эту систему из вне из различных источников, и ввода (понимания и погружения) этой информации в базу знаний, а также (2) для синтеза различных сообщений, адресуемых различным субъектам (в т.ч. пользователям).;Тексты \textit{SC-кода} (sc-тексты) имеют в общем случае нелинейную (графовую) структуру, поскольку (1) знак каждой описываемой сущности в ходит в состав sc-текста однократно и (2) каждый такой знак может быть инцидентен неограниченному числу других знаков, поскольку каждая описываемая сущность может быть связана неограниченным числом связей с другими описываемыми сущностями.;
База знаний, представленная текстом \textit{SC-кода}, является графовой структурой специального вида, алфавит элементов которой включает в себя множество узлов, множество ребер, множество дуг, множество базовых дуг -- дуг специально выделенного типа, обеспечивающих структуризацию баз знаний, а также множество специальных узлов, каждый из которых имеет содержимое, являющееся файлом, хранящимся в памяти интеллектуальной компьютерной системы. Структурная особенность данной графовой структуры заключается в том, что ее дуги и ребра могут связывать не только узел с узлом, но и узел с ребром или дугой, ребро или дугу с другим ребром или дугой.;
\uline{Все элементы} указанной выше графовой структуры (текста SC-кода), т.е. все ее узлы, ребра и дуги являются обозначениями различных сущностей. При этом ребро является обозначением бинарной неориентированной связки между двумя сущностями, каждая из которых либо представлена в рассматриваемой графовой структуре соответствующим знаком, либо является самим этим знаком. Дуга является обозначением бинарной ориентированной связки между двумя сущностями. Дуга специального вида (\textit{\textbf{базовая дуга}}) является знаком связи между узлом, обозначающим некоторое множество элементов рассматриваемой графовой структуры, и одним из элементов этой графовой структуры, который принадлежит указанному множеству. Узел, имеющий содержимое (узел, для которого содержимое существует, но может в текущий момент быть неизвестным) является знаком файла, который является содержимым этого узла. Узел, не являющийся знаком файла, может обозначать какой-либо материальный объект, первичный абстрактный объект(например, число, точку в некотором абстрактном пространстве), какую-либо бинарную связь, какое-либо множество (в частности, понятие, структуру, ситуацию, событие, процесс). При этом сущности, обозначаемые элементами рассматриваемой графовой структуры, могут быть постоянными (существующими всегда) и временными (сущностями, которым соответствует отрезок времени их существования). Кроме того, сущности, обозначаемые элементами рассматриваемой графовой структуры, могут быть константными (конкретными) сущностями и переменными (произвольными) сущностями. Каждому элементу рассматриваемой графовой структуры, являющемуся обозначением переменной сущности, ставится в соответствие область возможных значений этого обозначения. Область возможных значений каждого переменного ребра является подмножеством множества всевозможных константных ребер, область возможных значений каждой переменной дуги является подмножеством множества всевозможных константных дуг, область возможных значений каждого переменного узла является подмножеством множества всевозможных константных узлов.;
В рассматриваемой графовой структуре, являющейся представлением базы знаний в SC-коде, могут, но не должны существовать разные элементы графовой структуры, обозначающие одну и ту же сущность. Если пара таких элементов обнаруживается, то эти элементы склеиваются (отождествляются). Таким образом, синонимия внутренних обозначений в базе знаний интеллектуальной компьютерной системы, построенной на основе \textit{SC-кода,} запрещена. При этом синонимия внешних обозначений считается нормальным явлением. Формально это означает, что из некоторых элементов рассматриваемой графовой структуры выходит несколько дуг, принадлежащих отношению "\textit{\textbf{быть внешним идентификатором*}}". Из всех указанных дуг, принадлежащих отношению "\textit{\textbf{быть внешним идентификатором*}}" и выходящих из одного элемента рассматриваемой графовой структуры, обязательно выделяется одна (очень редко две) путем включения их в число дуг, принадлежащих отношению "\textit{\textbf{быть основным внешним идентификатором*}}". Это означает, что указываемый таким образом внешний идентификатор не является омонимичным, т.е. не может быть использован как внешний идентификатор, соответствующий другомуэлементу рассматриваемой графовой структуры.;
Кроме файлов, представляющих различные внешние обозначения (имена, иероглифы, пиктограммы), в памяти интеллектуальной компьютерной системе, построенной на основе \textit{SC-кода,} могут хранится файлы различных текстов (книг, статей, документов, примечаний, комментариев, пояснений, чертежей, рисунков, схем, фотографий, видео-материалов, аудио-материалов).;
\uline{Любую сущность}, требующую описания, можно обозначить в виде sc-элемента. Особо подчеркнем, что sc-элементы являются не просто обозначениями различных описываемых сущностей, а обозначениями, которые являются элементарными (атомарными) фрагментами знаковой конструкции, т.е. фрагментами, детализация структуры которых не требуется для "прочтения" и понимания этой знаковой конструкции.;
Текст \textit{\textbf{SC-кода}}, как и любая другая графовой структура, является абстрактным математическим объектом, не требующим детализации (уточнения) его кодирования в памяти компьютерной системы (например, в виде матрицы смежности, матрицы инцидентности, списковой структуры). Но такая детализация потребуется для технической реализации памяти, в которой хранятся и обрабатываются sc-тексты.;
Важнейшим дополнительным свойством \textit{\textbf{SC-кода}} является то,что он удобен не просто для внутреннего представления знаний в памяти интеллектуальной компьютерной системы, но также и для внутреннего представления информации в памяти компьютеров, специально предназначенных для интерпретации семантических моделей интеллектуальных компьютерных систем. Т.е., SC-код определяет синтаксические, семантические и функциональные принципы организации памяти компьютеров нового поколения, ориентированных на реализацию интеллектуальных компьютерных систем, -- принципы организации графодинамической ассоциативной семантической памяти.;
SC-код рассматривается нами как объединение нескольких его подъязыков, в число которых входит ядро SC-кода и его расширение, обеспечивающее ввод и вывод информации для ostis-системы на всевозможных внешних языках.
}
\filemodefalse

\scnsourcecomment{Завершили описание принципов SC-кода}

\scnheader{Описание Ядра SC-кода}
\scnstartsubstruct

\scnheader{Ядро SC-кода}
\scnrelfrom{алфавит}{Алфавит Ядра SC-кода}
\scnaddlevel{1}
\scnhaselement{sc-узел}
\scnhaselement{sc-ребро}
 \scnaddlevel{1}
 \scnidtf{обозначение бинарной неориентированной связи между sc-элементами}
 \scnaddlevel{-1}
\scnhaselement{sc-дуга}
\scnaddlevel{1}
 \scnidtf{обозначение бинарной ориентированной связи между sc-элементами}
 \scnaddlevel{-1}
\scnhaselement{базовая sc-дуга}
 \scnaddlevel{1}
 \scnidtf{sc-дуга константной позитивной стационарной принадлежности}
 \scnidtf{знак константной позитивной стационарной пары принадлежности}
 \scnaddlevel{-1}
\scnnote{Подчеркнем, что с помощью указанных типов sc-элементов можно описать любые связи между sc-элементами, трактуя эти связи как множества связываемых sc-элементов и используя некоторые sc-узлы как знаки этих множеств.}
\scnaddlevel{-1}

\scnendstruct

\scnheader{Описание Расширения Ядра SC-кода}
\scnstartsubstruct

\scnheader{SC-код}
\scnidtf{Расширение Ядра SC-кода}
\scnidtf{Результат введения в Ядро SC-кода sc-узлов, имеющих содержимое и обозначающих файлы, хранимые в памяти ostis-системы}
\scnnote{Все файлы, представляющие собой электронные образы инородных для SC-кода информационных конструкций, можно представить в SC-кода с помощью графовых структур, в которых sc-элементы обозначают буквы текстов или пиксели изображений. Но такой вариант кодирования внешних для ostis-системы информационных конструкций не дает возможности непосредственно использовать накопленный человечеством арсенал электронных информационных ресурсов.}
\scnnote{Важнейшим видом файлов ostis-систем являются внешние идентификаторы sc-элементов (в частности, имена sc-элементов), представляющие sc-элементы в текстах внешних языков (в том числе, текстах SCs-кода и SCn-кода)} 
\scnnote{Результатом просмотренного расширения \textit{Ядра SC-кода} является расширение \textit{Алфавита Ядра SC-кода}}

\scnheader{SC-код}
\scnrelfrom{алфавит}{Алфавит SC-кода}
\scnaddlevel{1}
\scnhaselement{sc-узел}
\scnhaselement{sc-ребро}
\scnhaselement{sc-дуга}
\scnhaselement{базовая sc-дуга}
\scnhaselement{файл ostis-системы}
\scnaddlevel{-1}

\scnheader{файл ostis-системы}
\scnidtf{sc-узел с содержимым}
\scnidtf{sc-узел, имеющий содержимое}
\scnidtf{sc-узел, обозначающий файл, хранимый в памяти ostis-системы}
\scnidtf{знак файла ostis-системы}
\scnreltoset{разбиение}{ея-файл ostis-системы\\
\scnaddlevel{1}
\scnidtf{естественно-языковой файл ostis-системы}
\scnaddlevel{-1};файл ostis-системы, являющийся текстом формального языка\\
\scnaddlevel{1}
\scnsuperset{sc.g-файл ostis-системы}
\scnsuperset{sc.s-файл ostis-системы}
\scnsuperset{sc.n-файл ostis-системы}
\scnaddlevel{-1};файл ostis-системы, отражающий процесс изменения sc.g-текста;графический файл ostis-системы;файл ostis-системы, являющийся изображением;видео-файл ostis-системы;аудио-файл ostis-системы}
\scnreltoset{разбиение}{файл-экземпляр ostis-системы
\scnaddlevel{1}
\scnidtf{файл, являющийся конкретным электронным документом или электронным образом конкретной внешней информационной конструкции}
\scnaddlevel{-1};файл-класс ostis-системы
\scnaddlevel{1}
\scnidtf{файл, являющийся знаком множества всевозможных экземпляров (копий) этого файла}
\scnaddlevel{-1}
}

\scnheader{SC-код}
\scnrelfrom{синтаксис}{Cинтаксис SC-кода} 
\scnaddlevel{1}
\scnexplanation{\textit{\textbf{Синтаксис}} \textit{\textbf{SC-кода}} задается
\begin{scnitemize}
\item типологией (алфавитом) sc-элементов (атомарных фрагментов текстов sc-кода);
\item правилами соединения (инцидентности) sc-элементов (например, sc-элементы каких типов не могут быть инцидентными друг другу);
\item типологией конфигураций sc-элементов (связки, классы, структуры), связями между конфигурациями sc-элементов (в частности, теоретико-множественными)
\end{scnitemize}
}
\scnaddlevel{-1}
\scnrelfrom{денотационная семантика}{Денотационная семантика SC-кода} 
\scnaddlevel{1}
\scnexplanation{\textit{\textbf{Денотационная семантика}} \textit{\textbf{SC-кода}} задается
\begin{scnitemize}
\item
 семантической интерпретацией sc-элементов и их конфигураций;
\item
 семантической интерпретацией инцидентности sc-элементов;
\item
 иерархической системой предметных областей;
\item
 структурой используемых понятий в каждой предметной области (исследуемые классы объектов, исследуемые отношения, исследуемые классы объектов отношений из смежных предметных областей, ключевые экземпляры исследуемых классов объектов);
\item
 онтологиями предметных областей.
\end{scnitemize}
}
\scnaddlevel{-1}
\scnnote{Следует особо подчеркнуть, что  унификация и максимально возможное упрощение  \textbf{\textit{синтаксиса}} и \textbf{\textit{денотационной семантики}} внутреннего языка интеллектуальных компьютерных систем необходимы потому, что подавляющий объем \textbf{\textit{знаний}}, хранимых в составе  базы знаний интеллектуальной компьютерной системы, представляют собой \textbf{\textit{метазнания}}, описывающими свойства других знаний. Более того, по указанной причине конструктивное (формальное) развитие теории интеллектуальных компьютерных систем невозможно без уточнения (унификации, стандартизации) и обеспечения семантической совместимости различных видов знаний, хранимых в базе знаний интеллектуальной компьютерной  системы.  Очевидно, что многообразие форм представления семантически эквивалентных знаний делает разработку общей теории  интеллектуальных компьютерных систем практически невозможной. К \textit{метазнаниям}, в частности, следует отнести и различного вида логические высказывания и всевозможного вида программы, описания методов (навыков). Обеспечивающих решение различных классов информационных задач.}

\scnendstruct~
\scnsourcecomment{Завершили сегмент "Описание расширения Ядра SC-кода"}

\scnendstruct~
\scnsourcecomment{Завершили раздел "\currentname"}

\end{SCn}
