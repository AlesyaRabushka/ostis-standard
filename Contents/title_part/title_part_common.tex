\scseparatedfragment[\scnmonographychapter{Глава 7.3. Метасистема OSTIS и Стандарт OSTIS}]{Титульная спецификация Стандарта OSTIS}

\begin{SCn}

\scnsectionheader{\currentname}

\scnstartsubstruct

\scnrelto{титульная спецификация}{Стандарт OSTIS}
\scnaddlevel{1}
	\scnrelfrom{оглавление}{Оглавление Стандарта OSTIS}
	\scnrelfrom{общая структура}{Общая Структура Стандарта OSTIS}
	\scnrelfrom{система ключевых знаков}{Система ключевых знаков Стандарта OSTIS}
	\scnrelfrom{редакционная коллегия}{Редакционная коллегия Стандарта OSTIS}
	\scnrelfrom{авторский коллектив}{Авторский коллектив Стандарта OSTIS}
	\scnrelfrom{направления развития}{Направления развития Стандарта OSTIS}
	\scnrelfrom{правила построения}{Правила построения Стандарта OSTIS}
	\scnaddlevel{1}
		\scnidtf{правила построения*(Стандарт OSTIS)}
		\scnaddlevel{1}
			\scniselement{sc-выражение}
		\scnaddlevel{-1}
	\scnaddlevel{-1}
	\scnrelfrom{правила организации развития}{Правила организации развития Стандарта OSTIS}
	\scnaddlevel{1}	
		\scnrelfromset{декомпозиция}{Правила организации развития исходного текста Стандарта OSTIS;Правила организации развития Стандарта OSTIS на уровне его внутреннего представления в памяти Метасистемы IMS.ostis}
\scnaddlevel{-2}

\scnauthorcomment{Вычитать то, что дальше}		

\scnfilelong{В титульную спецификацию \textit{Стандарта OSTIS} должны быть включены ссылки на все разделы и фрагменты этих разделов, где описываются правила построения и оформления всех видов информационных конструкций, входящих в состав \textit{Стандарта OSTIS} (внешних идентификаторов знаков, входящих в состав \textit{Стандарта OSTIS}, спецификаций различного вида cущностей, описываемых в \textit{Стандарте OSTIS})\\
В \textit{баз знаний ostis-систем} задаются правила унифицированного построения (представления, оформления) следующих видов \textit{информационных конструкций}:
\begin{scnitemize}
	\item\textit{sc-идентификаторов} -- внешних идентификаторов \textit{sc-элементов} следующих классов:
	\begin{itemize}
		\item\textit{sc-элементов} (имеются в виду общие правила идентификации любых sc-элементов) -- смотрите в разделе ``\nameref{intro_idtf}''
		\item\textit{sc-переменных, sc-констант}
		\item знаков материальных сущностей
		\begin{itemize}
			\item знаков персон
			\item знаков библиографических источников 
		\end{itemize}
		\item знаков множеств 
		\begin{itemize}
			\item классов, понятий
			\begin{itemize}
				\item отношений
				\item параметров
				\item структур
				\begin{itemize}
					\item знаний
				\end{itemize}
			\end{itemize}
		\end{itemize}
		\item знаков файлов ostis-систем
		\item знаков sc-знаний баз знаний 
	\end{itemize}
	\item\textit{sc-конструкций}
	\item\textit{sc.g-конструкций}
	\item\textit{sc.s-конструкций}
	\item\textit{sc.n-конструкций}
	\item базовых правил \textit{sc-спецификаций:}
	\begin{itemize}
		\item понятий 
		\item разделов баз знаний (титульные спецификации разделов)
		\item файлов ostis-систем
		\item библ. источников
		\item предметных областей 	
	\end{itemize}
	\item специализированная \textit{sc-спецификация}
	\begin{itemize}
		\item информационная конструкция
		\begin{itemize}
			\item оглавление
			\item система ключевых знаков	
		\end{itemize}
	\end{itemize}
	\begin{itemize}
		\item понятий
		\begin{itemize}
			\item пояснение
			\item определения
			\item теоретико-множественная окрестность
			\item семейство утверждений	
		\end{itemize}
	\end{itemize}
	\begin{itemize}
		\item сегментов баз знаний (титульная спецификация)
		\item семейство разделов баз знаний	
	\end{itemize}
\end{scnitemize}}

\scnheader{Стандарт OSTIS-2021}
\scnidtf{Издание Документации Технологии OSTIS-2021}
\scnidtf{Первое издание (публикация) Внешнего представления Документации Технологии OSTIS в виде книги}
\scniselement{публикация}
\scnaddlevel{1}
\scnidtf{библиографический источник}
\scnaddlevel{-1}
\scniselement{официальная версия Стандарта OSTIS}
\scniselement{бумажное издание}
\scniselement{научное издание}
\scnrelfrom{рекомендация издания}{Совет БГУИР}
\scnrelfromset{рецензенты}{Курбацкий А.Н.; Дудкин А.А.}
\scnrelfrom{издательство}{Бестпринт}
\scniselement{\scnstartsetlocal\scnendstructlocal}
\scnaddlevel{1}
\scniselement{УДК}
\scnaddlevel{1}
\scniselement{параметр}
\scnaddlevel{-1}
\scnrelfrom{Индекс УДК}{004.8}
\scnaddlevel{-1}
\scnidtftext{ISBN}{978-985-7267-13-2}

\scnheader{Стандарт OSTIS-2022}
\scnidtf{Издание Документации Технологии OSTIS-2022}
\scnidtf{Второе издание (публикация) Внешнего представления Документации Технологии OSTIS в виде книги}
\scniselement{публикация}
\scniselement{официальная версия Стандарта OSTIS}
\scniselement{бумажное издание}
\scniselement{научное издание}

\bigskip
\scnendstruct \scninlinesourcecommentpar{Завершили Титульную спецификацию \textit{Стандарта OSTIS}}

\end{SCn}