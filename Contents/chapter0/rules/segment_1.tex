\begin{SCn}

\scnsegmentheader{Формальная типология информационных конструкций, хранимых в памяти ostis-системы}

\scnstartsubstruct

\scnheader{информационная конструкция}
\scnsubdividing{знак\\
\scnaddlevel{1}
\scnidtf{семантически и структурно элементарный фрагмент информационной конструкции}
\scnidtf{элементарные атомарная информационная конструкция}
\scnidtf{информационная конструкция, состоящая из одного знака}
\scnaddlevel{-1}
;знаковая конструкция\\
\scnaddlevel{1}
\scnidtf{информационная конструкция, состоящая из \uline{нескольких} знаков} 
\scnaddlevel{-1}
}
\scnrelfrom{покрытие}{язык}
\scnaddlevel{1}
\scnidtf{множество (класс) \textit{информационных конструкций},
\begin{scnitemize}
		\item которому ставится в соответствие
		\begin{scnitemizeii}
			\item семейство \uline{синтаксических} правил построения \textit{информационных конструкций}, принадлежащих этому \textit{множеству}, а также
			\item семейство \uline{семантических} правил построения \textit{информационных конструкций} этого \textit{множества}
		\end{scnitemizeii}
		\item и которому \textit{принадлежат} не только синтаксически и семантически правильно построенные (корректные) \textit{информационные конструкции}, но и также неправильно построенные (некорректные, ошибочные) \textit{информационные конструкции}, которые, тем не менее, дают возможность их семантически интерпретировать (понимать) и, по крайней мере, локализовать допущенные в них ошибки.
\end{scnitemize}}
\scnaddlevel{1}
\scnnote{Очень важно обеспечить эффективный обмен \textit{информацией} не только с \uline{грамотными} \textit{носителями языков}, но также и с не очень грамотными, которые допускают большое количество языковых ошибок.}
\scnaddlevel{-1}
\scnnote{Поскольку теоретически некоторые \textit{информационные конструкции} могут принадлежать одновременно нескольким \textit{языкам} и поскольку понятие синтаксической и семантической корректности \textit{информационных конструкций} определяются правилами соответствующих \textit{языков} (их \textit{синтаксисом} и \textit{денотационной семантикой}), для уточнения понятий структурного уровня и корректности (правильности) \textit{информационных конструкций} вводится ряд \textit{ролевых отношений}, связывающих различные \textit{языки} с принадлежащими им \textit{информационными конструкциями}.}
\scnaddlevel{-1}

\newpage
\scnheader{знак\scnrolesign}
\scnidtf{быть \textit{знаком}, входящим в состав \textit{информационных конструкций} \uline{заданного} \textit{языка}\scnrolesign}

\scnheader{синтаксически выделяемый класс знаков\scnrolesign}
\scnidtf{быть синтаксические выделяемым \textit{классом знаков} в рамках заданного \textit{языка}\scnrolesign}
\scnhaselementlist{пример}{{\normalfont(}SC-код $\in$ sc-узел{\normalfont)};{\normalfont(}SC-код $\in$ sc-коннектор{\normalfont)}}
\scnsuperset{класс синтаксически эквивалентных знаков\scnrolesign}

\scnheader{класс синонимичных знаков\scnrolesign}
\scnidtf{быть множеством \textit{знаков} заданного \textit{языка}, обозначающих одно это уже описываемую \textit{сущность}\scnrolesign}

\scnheader{знаковая конструкция\scnrolesign}
\scnidtf{быть \textit{знаковой конструкцией} заданного \textit{языка}\scnrolesign}

\scnheader{текст\scnrolesign}
\scnidtf{быть синтаксически корректной \textit{информационной конструкцией} заданного \textit{языка}\scnrolesign}

\scnheader{синтаксически некорректная информационная конструкция\scnrolesign}
\scnidtf{быть синтаксически ошибочной \textit{информационной конструкцией} заданного \textit{языка}\scnrolesign}

\scnheader{семантически корректная информационная конструкция\scnrolesign}
\scnidtf{\textit{информационная конструкция} заданного \textit{языка}, не имеющая семантических ошибок\scnrolesign}

\scnheader{семантически некорректная информационная конструкция\scnrolesign}
\scnidtf{\textit{информационная конструкция} заданного \textit{языка}, имеющая семантические ошибки\scnrolesign}

\scnheader{знание\scnrolesign}
\scnidtf{\textit{текст} заданного \textit{языка}, не имеющий семантических ошибок и обладающий целостностью и ценностью\scnrolesign}

\scnheader{следует отличать*}
\scnhaselementset{текст\scnrolesign\\
\scnaddlevel{1}
\scniselement{ролевое отношение}
\scnaddlevel{-1}
;текст\\
\scnaddlevel{1}
\scnidtf{второй домен ролевого отношения \textit{текст}\scnrolesign}
\scnidtf{текст некоторого языка}
\scnaddlevel{-1}
;знание'\\
\scnaddlevel{1}
\scniselement{ролевое отношение}
\scnaddlevel{-1}
;знание\\
\scnaddlevel{1}
\scnidtf{второй домен ролевого отношения \textit{знание}\scnrolesign}
\scnaddlevel{-1}
;информационная конструкция\\
\scnaddlevel{1}
\scnsuperset{текст}
\scnsuperset{знание}
\scnaddlevel{-1}
}

\scnheader{структура}
\scnidtf{sc-конструкция}
\scnidtf{информационная конструкция SC-кода}
\scnidtf{Множество всевозможных структур}
\scnsubset{информационная конструкция}
\scnsuperset{\scnkeyword{sc-текст}} 
\scnaddlevel{1}
\scnidtf{Множество sc-текстов}
\scnidtf{текст \textit{SC-кода}}
\scnidtf{\textit{информационная конструкция}, удовлетворяющая синтаксическим правилам \textit{SC-кода}}
\scnidtf{синтаксически корректная для \textit{SC-кода} \textit{информационная конструкция}}
\scnidtf{связное множество \textit{sc-элементов}, удовлетворяющее синтаксическим правилам \textit{SC-кода}}
\scnidtf{синтаксически правильно построеная \textit{информационная конструкция} \textit{SC-кода}}
\scnsubset{текст}
\scnsuperset{\scnkeyword{sc-знание}}
\scnaddlevel{1}
\scnidtf{знание, представленное в \textit{SC-коде}}
\scnidtf{\textit{sc-текст}, удовлетворяющий определенным семантическим правилам \textit{SC-кода} и, в частности, определённым правилам семантической полноты}
\scnidtf{семантически корректный и целостный \textit{sc-текст} }
\scnsubset{знание}
\scnaddlevel{-2}

\scnheader{sc-знание}
\scnidtf{формализованное знание, хранимое в \textit{памяти ostis-системы} и непосредственно используемое (понимаемое) её \textit{решателем задач}}
\scnidtf{\textit{sc-текст}, имеющий \uline{однозначную} семантическую интерпретацию в рамках \textit{SC-пространства} (формализованного смыслового пространства)}

\scnnote{Не каждый \textit{sc-текст} является \textit{знанием}. В отличие от \textit{sc-текстов} в знаниях важна \uline{семантическая} целостность (полнота) \textit{информационных конструкций}. Так, например, \textit{логическая формула} со свободными переменными не является \textit{знанием}. Но полный текст \textit{высказывания}, включающий в себя все компоненты всех логических связок (вплоть до \textit{атомарных логических формул}) \textit{знанием} является. Второй пример: \textit{sc-текст}, в состав которого входит \textit{sc-коннектор} или \textit{связка}, но не входят все компоненты этого \textit{sc-коннектора} или \textit{связки},\scnbigspace \textit{знанием} не является.}

\scnnote{Семантическая целостность \textit{знания ostis-системы} означает, во-первых, то, что такое \textit{знание} представляет собой \textit{информационную конструкцию}, являющуюся высказыванием, то есть информационную конструкцию, имеющую истинностное значение, которое может быть подтверждено или опровергнуто, например, экспертом (рецензентом) \textit{базы знаний ostis-системы}. Во-вторых, \textit{знание ostis-системы} должно содержать достаточно полную и \uline{однозначную} спецификацию по возможности всех входящих в него неидентифицированных (неименованных) \textit{sc-элементов}. Это необходимо для того, чтобы внешнее представление знания \mbox{ostis-системы} можно было по возможности однозначно погрузить ("вставить") в \textit{базу знаний} \scnbigspace \textit{ostis-системы}.}

\scnheader{информация, представленная в памяти ostis-системы}
\scnidtf{\textit{информационная конструкция}, хранимая в памяти \textit{ostis-системы}}
\scnsubdividing{структура\\
\scnaddlevel{1}
\scnidtf{информация, представленная в виде конструкции \textit{SC-кода}}
\scnaddlevel{-1}
;файл ostis-системы}
\scnexplanation{В \textit{ostis-системе} используются две формы представления информации в памяти системы --
\begin{scnitemize}
	\item полностью формализованное представление, понятное для решателя задач \textit{ostis-системы} (конструкции \textit{SC-кода});
	\item инородное для \textit{SC-кода} представление, используемое исключительно для коммуникации \textit{ostis-систем} с внешними субъектами (файлы \textit{ostis-систем}).
\end{scnitemize}}

\scnheader{файл ostis-системы}
\scnidtf{файл, хранимый в sc-памяти}
\scnidtf{инородная для \textit{SC-кода} \textit{информационная конструкция}, хранимая в памяти \textit{ostis-системы} в виде содержимого sc-узла, обозначающего эту \textit{информационную конструкцию}}
\scnidtf{хранимая в sc-памяти "электронная"{} форма представления \textit{информационной конструкции}}


\scnsuperset{фрагмент знака, представленный файлом ostis-системы}
\scnaddlevel{1}
\scnsuperset{синтаксически элементарный фрагмент информационной конструкции, представленный файлом ostis-системы}
\scnaddlevel{-1}

\scnsuperset{знак, представленный файлом ostis-системы}
\scnaddlevel{1}
\scnsuperset{sc-идентификатор, представленный файлом ostis-системы}
\scnaddlevel{-1}

\scnsuperset{\textbf{знаковая конструкция, представленная файлом ostis-системы}}
\scnaddlevel{1}
\scnsuperset{текст, представленный файлом ostis-системы}
\scnaddlevel{1}
\scnsuperset{знание, представленное файлом ostis-системы}
\scnaddlevel{-1}
\scnaddlevel{-1}

\scnheader{знаковая конструкция, представленная файлом ostis-системы}
\scnsuperset{sc.s-файл ostis-системы}
\scnaddlevel{1}
\scnidtf{конструкция SCs-кода, хранимая в памяти ostis-системы в виде содержимого некоторого SC-узла}
\scnidtf{файл ostis-системы, являющийся sc.s-конструкцией}
\scnaddlevel{-1}
\scnsuperset{sc-идентификатор, представленный файлом ostis-системы}
\scnaddlevel{1}
\scnidtf{файл ostis-системы, являющийся sc-идентификатором}
\scnidtf{(sc-идентификатор $\cap$ файл ostis-системы)}
\scnreltoset{пересечение}{sc-идентификатор;файл ostis-системы}
\scnaddlevel{-1}
\scnsuperset{sc.g-файл ostis-системы}
\scnaddlevel{1}
\scnreltoset{пересечение}{sc.g-конструкция;файл ostis-системы}
\scnaddlevel{-1}
\scnsuperset{sc.n-файл ostis-системы}
\scnaddlevel{1}
\scnreltoset{пересечение}{sc.n-конструкция;файл ostis-системы}
\scnaddlevel{-1}

\scnsuperset{ея-файл ostis-системы}
\scnaddlevel{1}
\scnreltoset{пересечение}{ея-конструкция\\
	\scnaddlevel{1}
	\scnidtf{естественно-языковая конструкция}
	\scnidtf{конструкция естественного языка}
	\scnsuperset{ея-текст}
	\scnaddlevel{-1}
	;файл ostis-системы}
\scnidtf{конструкция естественного языка, представленная в виде файла ostis-системы}
\scnsuperset{Русский язык}
\scnaddlevel{1}
\scnidtf{конструкция Русского языка}
\scnaddlevel{-1}
\scnsuperset{Английский язык}
\scnidtf{естественно-языковая конструкция, являющаяся содержимм sc-узла, обозначающего эту конструкцию}
\scnaddlevel{-1}

\scnheader{файл ostis-системы}
\scnsubdividing{файл ostis-системы, предполагающий одномерную визуализацию хранимой информации
	;файл ostis-системы, предполагающий двухмерную визуализацию хранимой информации
	;файл ostis-системы, предполагающий трехмерную визуализацию хранимой информации}
\scnsubdividing{файл ostis-системы, представляющий статическую информацию
	;файл ostis-системы, представляющий динамическую информацию\\
	\scnaddlevel{1}
	\scnsuperset{видео-файл ostis-системы}
	\scnsuperset{аудио-файл ostis-системы}
	\scnaddlevel{-1}}
\scnsubdividing{файл-экземпляр ostis-системы\\
	\scnaddlevel{1}
	\scnidtf{\textit{sc-узел}, обозначающий конкретное вхождение \textit{информационной конструкции}, структура которой представлена содержимым этого \textit{sc-узла}}
	\scnaddlevel{-1}
	;файл-образец ostis-системы\\
	\scnaddlevel{1}
	\scnidtf{класс \textit{синтаксически эквивалентных} \textit{файлов-экземпляров} \textit{ostis-системы}}
	\scnidtf{множество всевозможных \textit{файлов-экземпляров ostis-системы}, которые являются \textit{синтаксически эквивалентными} копиями содержимого заданного \textit{sc-узла}}
	\scnaddlevel{-1}
}
\scnsubdividing{сформированный файл ostis-системы\\
	\scnaddlevel{1}
	\scnidtf{\textit{файл ostis-системы}, представленный \textit{sc-узлом}, имеющим сформированное (полностью построенное) содержимое}
	\scnaddlevel{-1}
	;несформированный файл ostis-системы\\
	\scnaddlevel{1}
	\scnidtf{\textit{файл ostis-системы}, представленный \textit{sc-узлом}, содержимое которого либо полностью отсутствует, либо сформировано \uline{частично}}
	\scnaddlevel{-1}}
\scnnote{\textit{файл ostis-системы} можеть быть \uline{электронной копией} и знаком (!) внешней \textit{информационной конструкции}, которая может быть:
	\begin{scnitemize}
		\item общедоступный в сети Internet информационным ресурсом;
		\item документом, опублиеованным на бумажном носители в виде какой-либо книги или статьи.
	\end{scnitemize}
	Кроме того, \textit{файл ostis-системы} может быть просто \uline{обозначением} указанной внешней \textit{информационной конструкции} (т.е. может быть \textit{sc-узлом}, обозначающим \textit{внешнюю информационную конструкцию}, но не имеющим содержимого). Такой \textit{sc-узел} используется формальной спецификации (средствами \textit{SC-кода}) соответствующего обозначаемого им информационного ресурса.
}

\scnheader{ея-файл ostis-системы}
\scnidtf{естественно-языковой \textit{файл ostis-системы}}
\scnidtfexp{\textit{файл ostis-системы}, представляющий собой \textit{sc-узел}, содержимым которого является "электронная"{} форма \textit{информационной конструкции} (чаще всего, \textit{текста}) одного из \textit{естественных языков}}
\scnidtf{структурно выделенный ея-текст, хранимый в памяти ostis-системы (в sc-памяти)}
\scnrelfromvector{правила оформления}{
	\scnfileitem{При оформлении текстов в естественно-языковых файлах (ея-файлах) \textit{ostis-систем} используются обычные разделители (точки в аббревиатурах и между предложениями\char59 круглые скобки, запятые, пробелы\char59), а также специальный символ {\scriptsize$\square$}, используемый в \textit{ея-файлах ostis-систем} как разделитель и целый ряд следующих ограничителей, позволяющих выделять некоторые фрагменты ея-текстов:
		\begin{scnitemize}
			\item подчеркивание выделяет логически важные фрагменты в предложениях\char59
			\item цитатные кавычки являются ограничителем цитат\char59
			\item прямые кавычки ограничивают иносказательные термины, метафоры\char59
			\item жирным курсивом стандартного размера выделяются идентификаторы \textit{sc-элементов} базы знаний, являющиеся ключевыми для заданного контекста\char59
			\item жирным курсивом стандартного размера выделяются также условные внешние идентификаторы (обозначения) условно вводимых \textit{sc-элементов}, например, условные обозначения \textit{sc-элементов} произвольного структурного типа ($\bm{e_i}$, $\bm{e_j}$, ...), условные обозначения \textit{sc-узлов} ($\bm{v_i}$, $\bm{v_j}$, ...), условные обозначения \textit{sc-коннекторов} ($\bm{c_i}$, $\bm{c_j}$, ...), \textit{sc-дуг}, \textit{связок}, не являющихся \textit{sc-коннекторами}, \textit{структур} и так далее\char59
			\item нежирным курсивом стандартного размера выделяются \textit{sc-идентификаторы} \textit{sc-элементов} \textit{базы знаний}, не являющиеся ключевыми для данного ея-текста.
	\end{scnitemize}}
	;\scnfileitem{Если в ея-тексте идентификаторы sc-элементов базы знаний (чаще всего -- \textit{простые sc-иден\-ти\-фи\-ка\-то\-ры}), выделенные курсивом одинаковой жирности, следуют друг за другом, то число пробелов между разными идентификаторами должно быть увеличено. Если при этом выделенный sc-идентификатор включает в себя другие sc-идентификаторы, на которые желательно отдельно сослаться, то такая ссылка оформляется в виде дополнительной фразы типа "Смотрите также ..."{}}
	;\scnfileitem{Текст \textit{ея-файла ostis-системы} может иметь \uline{любые} вставки не являющиеся естественно-языковыми текстами, в том числе, и фрагменты, являющиеся формальными внешними текстами представления знаний для ostis-систем (текстами SCg-кода, SCs-кода, SCn-кода). При этом указанные формальные фрагменты (вставки) могут быть как транслируемыми на внутренний язык ostis-системы (\mbox{SC-код}) и погружаемыми в состав ее базы знаний (т.е. фактически являться sc-текстами), так и нетранслируемыми формальными фрагментами, которые входят в состав базы знаний ostis-системы в виде содержимого соответствующих файлов. Все указанные выше "вставки"{} в ея-файл ostis-системы оформляются как ссылки на соответствующие sc-тексты или файлы ostis-системы. Каждая такая ссылка представляет собой \textit{sc-идентификатор} соответствующего sc-текста или файла ostis-системы и выделяется в ея-файле жирным курсивом стандартного размера со стандартным расстоянием между символами.
	Таким образом, если в естественно-языковой \textit{файл ostis-системы}, необходимо "вставить"{} информационную конструкцию иного рода (\textit{sc.g-текст}, рисунок, таблицу, изображение), то (1) указанная конструкция оформляется как отдельный файл (2) которому приписывается имя (название), построенное по установленным правилам, и (3) на который в указанном естественно-языковом файле делается ссылка.
	В естественно-языковых \textit{файлах ostis-систем} можно делать ссылки не только на другие \textit{файлы ostis-системы}, но и на \uline{именуемые} (!) фрагменты базы знаний, которые во внешнем представлении базы знаний оформляются в виде именуемых (идентифицируемых) \mbox{sc.n-контуров}.
	Файлы и sc-тексты, на которые делаются ссылки из ея-файла, во внешнем представлении (при визуализации) базы знаний размещаются после указанного ея-файла в порядке первого их упоминания в этом ея-файле, если, конечно, на эти файлы или sc-тексты не было ссылок из ранее представленных ея-файлов.}
	;\scnfileitem{В состав \textit{ея-файла ostis-системы} могут входить ссылки на любые идентифицированные (именованные) \textit{информационные конструкции} (Internet-ресурсы, библиографические источники, различные документы). Для этого достаточно указывать соответствующие \textit{sc-идентификаторы}.}
	;\scnfileitem{В ея-текстах \uline{все} основные sc-идентификаторы описываемых в базе знаний сущностей должны быть выделены жирным или нежирным курсивом и могут быть представлены в любом склонении и спряжении.}
	;\scnfileitem{В ея-текстах используются только основные идентификаторы (термины). Используемые синонимы явно указываются как неосновные идентификаторы.}
	;\scnfileitem{Основная часть (содержимого текста) \textit{ея-файла ostis-системы} оформляется стандартным печатным шрифтом.}
}
\scnaddlevel{1}
\scnsourcecommentpar{Завершили перечень правил оформления содержимого \textit{ея-файлов ostis-систем}}
\scnaddlevel{-1}

\scnheader{ея-файл ostis-системы}
\scnnote{Выделенные курсивом в \textit{ея-файле ostis-системы} \textit{sc-идентификаторы sc-элементов} могут являться \uline{ар\-гу\-мен\-та\-ми} различного вида \uline{запросов} к \textit{базе знаний ostis-системы} и, в первую очередь запросов типа "Что это такое"{}, предполагающих выделение из \uline{текущего} состояния \textit{базы знаний ostis-системы} семантической окрестности (спецификации) указываемого \textit{sc-элемента}, содержащей основную информацию о сущности, обозначаемой \textit{этим sc-элементом}.}

\bigskip

\scnendstruct \scnendsegmentcomment{Формальная типология информационных конструкций, хранимых в памяти ostis-системы}
\end{SCn}