\begin{SCn}

\scnsegmentheader{Смысловая типология и спецификация знаний, представленных в памяти ostis-систем}

\scnstartsubstruct

\scnheader{знание, представленное в памяти ostis-системы}
\scnsuperset{предметная область}
\scnaddlevel{1}
\scnidtf{sc-модель предметной области}
\scnaddlevel{-1}
\scnsuperset{sc-метазнание}
\scnaddlevel{1}
\scnidtf{\textit{sc-знание}, являющееся спецификацией некоторого другого \textit{знания, представленного в памяти \mbox{ostis-системы}}}
\scnsuperset{онтология предметной области}
\scnaddlevel{1}
\scnidtf{семантическая спецификация предметной области}
\scnaddlevel{-2}

\scnsuperset{sc-знание "дидактического"{} назначения}
\scnaddlevel{1}
\scnrelfromset{покрытие}{спецификация "дидактического"{} файла ostis-системы\\
\scnaddlevel{1}
\scnnote{Такая спецификация описывает семантические связи специфицируемого "дидактического"{} файла ostis-системы с базой знаний этой системы}
\scnaddlevel{-1};
sc-знание, описывающее различные сущности в сравнительном аспекте
}
\scnaddlevel{1}
\scntext{эпиграф}{Сущности познаются в связях между собой и в сравнении друг с другом}
\scnaddlevel{-2}

\scnsuperset{"дидактический"{} файл ostis-системы}
\scnaddlevel{1}
\scnnote{Для знания, представленного \textit{файлом ostis-системы}, в \textit{базе знаний} этой \textit{ostis-системы} может присутствовать \textit{семантически эквивалентное*} этому файлу знание, представленное \textit{текстом SC-кода}. Это является вариантом описания семантической интерпретации некоторых фрагментов \textit{базы знаний}. Кроме того, \textit{знание}, представленное \textit{файлом ostis-системы}, может быть предварительным этапом формализации этого \textit{знания}, предполагающим последующую трансляцию этого знания на \textit{SC-код}. Использование таких файлов является важнейшим механизмом коллективной разработки \textit{баз знаний} \scnbigspace \textit{ostis-систем}. Следует также заметить, что некоторые \textit{знания}, представленные \textit{файлами ostis-систем}, не требует трансляции на \textit{SC-код}, а носят \uline{вспомогательный} характер, позволяющий разработчикам и конечным пользователям \textit{ostis-систем} ускорить процесс установления семантического контакта (взаимопонимания) с \textit{ostis-системами}.}
\scnsuperset{пояснение, представленное файлом ostis-системы}
\scnsuperset{примечание, представленное файлом ostis-системы}
\scnsuperset{сравнение, представленное файлом ostis-системы}
\scnaddlevel{-1}

\scnheader{предметная область}
\scnexplanation{Для эффективной коллективной разработки и эксплуатации \textit{базы знаний ostis-системы} важна не просто её структуризация, а такая структуризация, которая носит максимально \uline{объективный} характер, имеющая четкую \uline{семантическую интерпретацию} (!) и позволяющая на основе \uline{семантических} (!) связей между структурно выделяемыми фрагментами \textit{базы знаний} легко определять (локализовывать) "местоположение"{} либо искомых \textit{знаний}, либо новых \textit{знаний}, вводимых в состав \textit{базы знаний}. Такая семантическая структуризация базы \textit{знаний}, формирование системы семантически связанных между собой "семантических полочек"{}, на которых размещаются конкретные \textit{знания}, удовлетворяющие четко заданным требованиям, существенно упрощает навигацию по \textit{базе знаний} и четко \uline{локализует} эволюцию \textit{знаний}, находящихся на каждой "семантической полочке"{}. В основе указанной семантической структуризации \textit{базы знаний} лежит иерархическая система \textit{предметных областей}. С содержательной точки зрения \textit{предметная область} представляет собой совокупность фактографических \textit{высказываний}, описывающих \uline{все} (!) элементы \uline{заданного} \textit{множества объектов исследования\scnrolesign} с помощью \uline{заданного} набора \textit{отношений} и \textit{параметров} (характеристик). Максимальный класс указанных \textit{объектов исследования}, некоторые специально выделяемые подклассы этого класса, а также указанные \textit{отношения} и \textit{параметры} задают систему \textit{понятий}, лежащих в основе \textit{предметной области}, которую будем называть \textit{схемой предметной области}. Подчеркнем то, что \textit{схема предметной области}, определяющая семантическую "систему координат"{} в рамках соответствующей \textit{предметной области}, в известной степени имеет \uline{субъективный} характер и является предметом \uline{соглашения} (!) (консенсуса) между специалистами в этой области. Подчеркнем также, что \textit{схема предметной области} может эволюционировать (может меняться набор понятий, может уточняться их денотационная семантика).\\
	Понятие \textit{предметной области} можно считать обобщением понятия \textit{алгебраической системы}, ориентированным на решение проблемы \uline{семантической} структуризации \textit{базы знаний}. В \textit{памяти ostis-системы} предметная область представляется обычно бесконечной \textit{структурой}, которая задается (1) неким множеством исследуемых \textit{сущностей}, (2) семейством \textit{отношений}, \textit{алгебраических операций} и \textit{параметров} (свойств), заданных на множестве исследуемых сущностей, каждое из которых либо рассматривается в рамках данной \textit{предметной области}, либо "наследуется"{} из другой \textit{предметной области} более высокого уровня. Каждая \textit{предметная область} в \textit{базе знаний} может быть представлена своим \textit{разделом}. При этом на множестве таких \textit{разделов} могут быть заданы не только рассмотренные выше "синтаксические"{} \textit{отношения}, но и целый ряд семантически интерпретируемых \textit{отношений}, например, отношение ``быть \textit{частной предметной областью*}''{}. Примером \textit{связки} этого \textit{отношения} является \textit{связка} между \textit{Предметной областью геометрических фигур} в Евклидовом пространстве и \textit{Предметной областью планарных фигур} Евклидова пространства. Таких \textit{отношений}, заданных на множестве \textit{предметных областей} и уточняющих характер соотношения между множествами исследуемых сущностей, а также рассматриваемыми или "наследуемыми"{} \textit{отношениями}, \textit{алгебраическими операциями} и \textit{параметрами}, существует достаточно много. Но, если к этому добавить анализ соотношения не только между самими \textit{предметными областями}, но и соотношения между представляющими их \textit{структурами} в рамках \textit{базы знаний} конкретной \textit{ostis-системы}, а также в рамках глобального смыслового пространства (\textit{SC-пространства}), то число семантически интерпретируемых \textit{отношений}, заданных на множестве \textit{предметных областей} существенно расширится. К числу важных \textit{отношений}, заданных на множестве \textit{предметных областей} относятся также \textit{отношения}, связывающие \textit{предметные области} с различными структурно выделяемыми фрагментами \textit{базы знаний}, которые \textit{предметными областями} не являются. Ключевым \textit{отношением} такого вида является \textit{Отношение, связывающее предметные области с соответствующими им онтологиями} (Отношение ``быть \textit{онтологией}*'').} 
\scnrelfromset{правила построения}{\scnfileitem{Понятия и соответствующие им термины в \textit{базе знаний} ostis-системы группируются по четко выделенным \textit{предметным областям}. При этом связи между \textit{предметными областями} \uline{явно} (!) указываются.};
\scnfileitem{Изложение материала в \textit{базе знаний ostis-системы} построено по принципу "сверху вниз"{}, переходя от общих \textit{предметных областей и онтологий} к частным для явного указания направления \uline{наследования свойств}.};
\scnfileitem{Во всех представляемых и описываемых \textit{предметных областях} необходимо явно указывать \uline{все} (!) \textit{понятия}, используемые в \textit{предметных областях} с указанием каждой роли этих \textit{понятий} в рамках \textit{предметной области}. К таким ролям относятся: \textit{максимальный класс объектов исследования\scnrolesign}; \textit{класс объектов исследования\scnrolesign}; \textit{исследуемое отношение\scnrolesign}, заданное на объектах исследования; \textit{вспомогательное понятие\scnrolesign}, используемое в описываемой предметной области, но исследуемое в других предметных областях).}}

\scnheader{онтология предметной области}
\scnexplanation{Все рассматриваемые в \textit{базе знаний ostis-системы} \textit{предметные области} должны быть формально специфицированы, т.е. для каждой из них должна быть построена и указана соответствующая ей \textit{онтология} и, в частности, должны быть указаны все известные связи с другими \textit{предметными областями} и с иными фрагментами \textit{базы знаний ostis-системы}. Можно выделить целый ряд частных онтологий, описывающих свойства соответствующих \textit{предметных областей} с разных "ракурсов"{}.}
\scnsuperset{схема предметной области}
\scnaddlevel{1}
\scnidtf{специфицикация внутренней структуры \textit{предметной области}}
\scnidtf{перечень всех \textit{понятий} и некоторых \textit{ключевых объектов исследования}, лежащих в основе специфицируемой \textit{предметной области}, с указанием их роли в рамках этой \textit{предметной области}}
\scnaddlevel{-1}
\scnsuperset{специфицикация внешних связей \textit{предметной области}}
\scnaddlevel{1}
\scnidtf{описание связей \textit{предметной области} с \textit{онтологиями} и с другими \textit{предметными областями}}
\scnaddlevel{-1}
\scnsuperset{онтология определений исследуемых понятий}
\scnsuperset{логическая онтология предметной области}
\scnaddlevel{1}
\scnidtf{онтология аксиом, теорем и текстов доказательств теорем}
\scnaddlevel{-1}
\scnsuperset{онтология классов задач предметной области и методов их решения}
\scnaddlevel{1}
\scnidtf{онтология задач, решаемых в рамках предметной области, объединенной с соответствующей онтологией}
\scnaddlevel{-1}
\scnsuperset{терминологическая онтология предметной области}
\scnsuperset{онтология эволюции предметной области и соответствующей ей онтологии}

\scnheader{спецификация "дидактического"{} файла}
\scnrelfromlist{используемое понятие}{иллюстрация;
иллюстрация*;
пояснение;
пояснение*;
примечание;
примечание*;
закономерность*\\
\scnaddlevel{1}
\scnidtf{закономерность, описывающая некоторое свойство экземпляров заданного понятия*}
\scnidtf{утверждение*}
\scnaddlevel{-1};
определение*;
однозначная спецификация*;
упражнение*\\
\scnaddlevel{1}
\scnidtf{упражнение, относящееся к заданной сущности или к заданному фрагменту базы знаний*}
\scnaddlevel{-1};
вопрос для самопроверки*
\scnaddlevel{1}
\scnidtf{вопрос, относящийся к заданной сущности или к заданному фрагменту базы знаний*}
\scnaddlevel{-1};
сравнение;
сравнение*
\scnaddlevel{1}
\scnidtf{быть сравнительным анализом двух заданных сущностей*}
\scnaddlevel{-1};
сходство*
\scnaddlevel{1}
\scnidtf{быть описанием сходства двух заданных сущностей*}
\scnaddlevel{-1};
отличие*
\scnaddlevel{1}
\scnidtf{быть описанием одного из различий двух заданных сущностей*}
\scnaddlevel{-1};
спецификация типичного примера*
\scnaddlevel{1}
\scnidtf{быть спецификацие типичного примера (экземпляра) заданного класса*}
\scnaddlevel{-1};
спецификация типичного подкласса заданного класса*;
пример использования экземпляра заданного класса*;
используемое понятие*
\scnaddlevel{1}
\scnidtf{понятие, используемое в заданной ифнормационной конструкции или в заданном классе информационных конструкций*}
\scnaddlevel{-1};
принцип, лежащий в основе*
\scnaddlevel{1}
\scnidtf{основное свойство (ключевое положение), характезирующее заданную сущность*}
\scnaddlevel{-1};
обоснование целесообразности создания*;
требование*
\scnaddlevel{1}
\scnidtf{требование, предъявляемое к заданной сущности*}
\scnaddlevel{-1};
правило построение*;
достоинство*;
недостаток*;
сравнительный анализ*;
логическое следствие*
\scnaddlevel{1}
\scnidtf{следовательно*}
\scnidtf{быть ориентированной парой двух знаний ostis-системы, второе из которых логически следует из первого*}
\scnidtf{что из этого логически следует*}
\scnaddlevel{-1};
логическое обоснование;
последствие*
\scnaddlevel{1}
\scnidtf{быть ориентированной парой двух знаний ostis-системы, первое из которых является описанием ситуации, являющейся причиной (предпосылкой, достаточным условием) возникновения ситуации, которая описывается вторым из указанных знаний ostis-системы*}
\scnidtf{причинно-следственная связь*}
\scnrelboth{обратное отношение}{причина*}
\scnaddlevel{1}
\scnidtf{предпосылка*}
\scnidtf{ситуация, являющаяся достаточным условием*}
\scnaddlevel{-2};
семантическая эквивалентность*\\
\scnaddlevel{1}
\scniselement{бинарное неориентированное отношение}
\scnidtf{\textit{бинарное неориентированное отношение}, каждая пара которого связывает \textit{\uline{знаки}} двух семантически эквивалентных \textit{информационных конструкций}, которые могут \textit{принадлежать} \uline{разным} \textit{языкам}}
\scnexplanation{\textit{семантическая эквивалентность*} двух \textit{информационных конструкций} означает то, что эти \textit{информационные конструкции}:
\begin{scnitemize}
	 \item оперируют (явно или неявно) \textit{знаками} \uline{одних и тех же} \textit{сущностей} и
	 \item описывают \uline{одни и те же} \textit{связи} между этими \textit{сущностями}
\end{scnitemize}}
\scnaddlevel{-1};
синтаксическая эквивалентность*;
семантическое включение*\\
}

\scnheader{sc-знание, описывающее различные сущности в сравнительном аспекте}
\scnrelfromlist{используемое понятие}{следует отличать*;
сходство*;
отличие*;
пример\scnrolesign;
ключевой знак\scnrolesign;
спецификация*
\scnaddlevel{1}
\scnidtf{быть спецификацией (описанием) заданной сущности*}
\scnaddlevel{-1};
аналог*
\scnaddlevel{1}
\scnidtf{быть аналогом заданной сущности*}
\scnaddlevel{-1};
изоморфность*;
изоморфизм*;
гомоморфность*;
гомоморфизм*
}

\bigskip

\scnendstruct \scnendsegmentcomment{Смысловая типология и спецификация знаний, представленных в памяти ostis-систем}

\end{SCn}