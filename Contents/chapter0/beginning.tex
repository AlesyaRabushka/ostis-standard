\begin{SCn}

\label{begin}
\scnheader{Раздел 0.2}

Методологические проблемы современного состояния работ в области Искусственного интеллекта

\scnstartsubstruct
Начало раздела "Методологические проблемы современного состояния работ в области Искусственного интеллекта"\\
\scnstartsubstruct
Методологические проблемы современного состояния работ в области Искусственного интеллекта\\

\scnrelfromvector{конкатенация сегментов}
{Структура деятельности в области Искусственного интеллекта}

\filemodetrue

\scnrelfromset{рассматриваемые вопросы}{
Каковы основные стратегические цели (сверхзадачи) научно-технической деятельности в области \textit{Искусственного интеллекта}; Какие проблемы являются на сегодняшний день актуальными для дальнейшего развития различных направлений \textit{Искусственного интеллекта} и для развития \textit{Искусственного интеллекта} в целом как общей (объединённой) \textit{научно-технической дисциплины}, а также для развития различных форм деятельности в этой области (научно-исследовательской деятельности создания технологий разработки интеллектуальных компьютерных систем, образовательной деятельности, бизнеса)?;
Какие проблемы являются на сегодняшний день актуальными для развития других \textit{научно-технических дисциплин} и являются ли эти проблемы аналогичными тем, которые актуальны для развития \textit{Искусственного интеллекта}?;
Какие можно предложить подходы к решению указанных выше проблем и как для этого можно использовать создаваемый сейчас новый технологический уклад в области \textit{Искусственного интеллекта} (следующий уровень технологий искусственного интеллекта)?;
Как будет выглядеть на основе следующего уровня \textit{технологий Искусственного интеллекта} комплексная автоматизация вех \textit{видов человеческой деятельности}, а также взаимодействие различных \textit{видов человеческой деятельности}, т.е. как будет выглядеть архитектура \textit{smart-общества}?;
Устраивает ли нас уровень семантической совместимости взаимопонимания между современными виртуальными компьютерными системами и что необходимо сделать для повышения этого уровня?; 
Устраивает ли нас уровень семантической совместимости взаимопонимания между современными интеллектуальными компьютерными системами их пользователями и что необходимо сделать для повышения этого уровня? 
}

\filemodetrue
\scnrelfromset{подвопрос}{
Недостатки современных интеллектуальных компьютерных систем; 
Недостатки современной технологии ИИ
;Каким требованиям должна удовлетворять качественная технология разработки интеллектуальных компьютерных систем;
\scnaddlevel{1}уточнить требования, представляемые к интеллектуальным компьютерным системам
(что такое интеллектуальная компьютерная система);
почему этого нет;
как эти требования удовлетворяют интеллектуальных компьютерных систем
;уточнить требования к технологии
;понять, уточнить, почему(что мешает)созданию технологии\scnaddlevel{1}
;сложность объекта;
отсутствие понимания того, что задача такой сложности требует создания принципиально нового творческого коллектива с принципиально новой организацией взаимодействия
\scnaddlevel{-1}
;как это сделать(принципы, лежащие в основе создания технологии интеллектуальных компьютерных систем)
\scnaddlevel{-2}
;Что такое ИИ(как наука)\\
\scniselement{научно-техническая дисциплина}
;Что такое интеллектуальная кибернетическая  система\\
\scnsubset{кибернетическая система}
;Что такое технология проектирования и реализации интеллектуальная кибернетическая  система
\scnaddlevel{1}
;проблемы создания технологии проектирования;
технология реализации от традиционных компьютеров к компьютерам, ориентированным на реализацию интеллектуальных кибернетических систем
\scnaddlevel{-1}
;Результат использования технологии проектирования и реализации
это не отдельные интеллектуальные компьютерные системы и Экосистема из интеллектуальных компьютерных систем и людей
\scnaddlevel{1}
;структура Экосистемы -- иерархическая система специализированных сообществ;
Чем нас не устраивают те, интеллектуальные компьютерные системы, которые мы разрабатываем сейчас;
Чем нас не устраивают современные технологии ИИ;
Какие интеллектуальные компьютерные системы нам нужны; 
Какими свойствами и способностями мы хотели бы их наделить\scnaddlevel{1}
;высокая степень обучаемости в разных направлениях\scnaddlevel{1};
расширение знаний без введения новых понятий;
введение новых понятий без расширения многообразия видов знаний;
расширение многообразия видов знаний;
расширение моделей решения задач(новый вид методов + их интерпретация)\scnaddlevel{-2}
;Какие технологии нам нужны;
Почему таких икс и технологий ещё нет; 
Что мешает?;
Что делать?;
Какие недостатки имеют современные интеллектуальные системы;
\scnaddlevel{1}недостаточно высокий уровень интеллектуальности;
нет эффективного взаимодействия(координации);
высокая степень обучаемости в разных направлениях;
\scnaddlevel{-1} 
Какие недостатки имеют современные технологии Искусственного интеллекта
;Какова трудоёмкость разработки выбранных
икс
;Какова трудоёмкость системной интеграции икс и их компонентов;
Обеспечивается ли совместимость компонентов
икс, разрабатываемых с помощью различных 
}
\scntext{аннотация}{Предлагаемое вашему вниманию рассмотрение методологических проблем современного состояния работ в области \textit{Искусственного интеллекта} состоит из следующих частей:
\scnaddlevel{1}
\begin{scnitemize}

\item Анализ актуальных проблем, препятствующих дальнейшему развитию  \textit{Искусственного интеллекта} как  \textit{научно-технической дисциплины,}:
\begin{scnitemizeii}
\item \scnaddlevel{1} Проблемы развития научных исследований в области \textit{Искусственного интеллекта} 
\item Проблемы разработки технологий проектирования и реализации \textit{интеллектуальных компьютерных систем};
\item Проблемы формирования рынка \textit{интеллектуальных компьютерных систем}; 
\item Образовательные проблемы в области \textit{Искусственного интеллекта};
\item Проблемы развития бизнеса в области \textit{Искусственного интеллекта}.
\end{scnitemizeii}
\item Анализ проблем автоматизации сложных видов деятельности:
\begin{scnitemizeii}
\item научно-исследовательской деятельности в рамках различных научных дисциплин;
\item создание \textit{технологий проектирования} и производства (реализации) сложных технических систем;
\item \textit{инженерной деятельности} по разработке сложных технических систем;
\item \textit{образовательной деятельности} по наукоёмким техническим специальностям
\end{scnitemizeii}
\item Формулировка принципов, лежащих в основе \textit{технологии OSTIS},предназначенных для решения указанных выше проблем;
\item Рассмотрение структуры \textit{Экосистемы OSTIS}, построенной по \textit{технологии OSTIS} и обеспечивающей комплексную автоматизацию всех видов человеческой деятельности}
\end{scnitemize}

\filemodefalse
\scnaddlevel{-2}
\scnrelfromset{используемые знаки общих понятий и иных сущностей}{деятельность\\
\scnaddlevel{1}
\scnidtf{область деятельности}
\scnsuperset{человеческая деятельность}
\scnaddlevel{-1}
;вид деятельности\\
\scnaddlevel{1}
\scnhaselement{проектирование}
\scnaddlevel{1}
\scnidtf{проектная деятельность}
\scnaddlevel{-1}
\scnhaselement{производство}
\scnaddlevel{1}
\scnidtf{производственная деятельность}
\scnaddlevel{-1}
\scnhaselement{наука}
\scnaddlevel{1}
\scnidtf{научная деятельность}
\scnaddlevel{-2}
;проект\\
\scnaddlevel{1}
\scnsuperset{открытый проект}
\scnaddlevel{-1}
;консорциум
;технология\\
\scnaddlevel{1}
\scnsuperset{информационная технология}
\scnaddlevel{1}
\scnsuperset{технология искусственного интеллекта}
\scnaddlevel{-2}
;кибернетическая система\\
\scnaddlevel{1}
\scnsuperset{интеллектуальная система}
\scnaddlevel{1}
\scnsuperset{интеллектуальная компьютерная система}
\scnaddlevel{1}
\scnidtf{искусственная интеллектуальная система}
\scnaddlevel{-3}
;конвергенция{\^}
\scnaddlevel{1}
\scnidtf{уровень конвергенции(?)}
\scnaddlevel{-1}
;интеграция*\\
\scnaddlevel{1}
\scnsuperset{интеграция кибернетических систем*}
\scnsuperset{эклектичная интеграция*}
\scnsuperset{глубокая интеграция}
\scnaddlevel{-1}
;интегрированная система\\
\scnaddlevel{1}
\scnsuperset{эклектичная система}
\scnsuperset{гибридная система}
\scnaddlevel{-1}
;экосистема интеллектуальных компьютерных систем
;рынок знаний\\
\scnaddlevel{1}
\scnidtf{рыночная организация порождения эволюции и применения знаний}
\scnaddlevel{-1}
;smart-общество\\
\scnaddlevel{1}
\scnidtf{общество,в основе которого лежит экосистема интеллектуальных компьютерных систем и рынок знаний}
}
 
\filemodefalse
\scnaddlevel{-1}
\scnrelfromset{ключевые знаки}
{Искусственный интеллект\\
\scnaddlevel{1}
\scniselement{научно-техническая дисциплина}
\scnaddlevel{1}
\scnsubset{научно-техническая деятельность} 
\scnaddlevel{-2}
;интеллектуальная система\\
\scnaddlevel{1}
\scnsuperset{интеллектуальная компьютерная система}
\scnaddlevel{-1}
;Общая теория интеллектуальных систем
;Базовая комплексная технология проектирования интеллектуальных компьютерных систем
;Технология производства спроектированных интеллектуальных компьютерных систем
;Специализированная инженерия в области  Искусственного интеллекта
 ;Образовательная деятельность в области Искусственного интеллекта
 ;Бизнес-деятельность в области Искусственного интеллекта
 ;Технология OSTIS 
 ;ostis-система
 ;смысловое преставление информаци
 ;агентно-ориентированная модель обработки информации в памяти стандартизация ostis-систем
\textit{SC-код}
;конвергенция знаний в памяти
;ostis-систем
;конвергенция моделей решения задач в  ostis-системе
;интеграция знаний в памяти  ostis-системы
;интеграция моделей решения задач в  ostis-системе
;ostis-сообщество
;ostis-технология
;\textit{Ядро Технологии OSTIS
\scnaddlevel{1}
\scnsuperset{ostis-технология проектирования}
\scnsuperset{ostis-технология производства}
\scnsuperset{технология эксплуатации ostis-систем}
;\scnsuperset{технология реинжиниринга ostis-систем}
\scnaddlevel{-1}
;Проект Программной реализации универсальной абстрактной sc-машины
;проект разработки Универсального sc-компьютера
; Специализированная инженерия, осуществляемая на основе технологии OSTIS;
  образовательная деятельность в области искусственного интеллекта осуществляемые на основе технологии OSTIS;
  Консорциум OSTIS;
  Экосистема OSTIS;
  человеческая деятельность;
  вид человеческой деятельности;
  автоматизация человеческой деятельности;
  качество человеческой деятельности;
  субъект Экосистемы OSTIS;
  Рынок знаний, реализованный в рамках Экосистемы OSTIS;
  smart-общество}
 
\newpage
\ActivateBG


\end{SCn}
