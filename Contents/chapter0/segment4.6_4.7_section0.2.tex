\bigskip
\scnfragmentcaption

\scnheader{Бизнес-деятельность в области Искусственного интеллекта}
\scntext{Предлагаемый подход}{Бизнес-деятельность в области Искусственного интеллекта, осуществляемая на основе \textit{Технологии OSTIS}}
    \scnaddlevel{1}
    \scntext{субъект}{ OSTIS-сообщество, Бизнес-деятельности в области Искусственного интеллекта, осуществляемой на основе \textit{Технологии OSTIS}}
        \scnaddlevel{1}
        \scnidtf{Глобальное (максимальное) OSTIS-сообщество, осуществляющее Бизнес-деятельность в области Искусственного интеллекта}
        \scnrelto{часть}{Экосистема OSTIS}
        \scniselement{ostis-сообщество}
        \scnexplanation{Речь идет об ostis-сообществе, которое включает в себя все компании и лаборатории, работающие в области Искусственного интеллекта и желающие на взаимовыгодных условиях сотрудничать в направлении совместного, перманентного и интенсивного развития стандартов, методов и средств комплексного проектирования и производства семантически совместимых и договороспособных интеллектуальных компьютерных систем, способных самостоятельно и целенаправленно взаимодействовать друг с другом. Кроме указанных компаний и лабораторий, в состав рассматриваемого ostis-сообщества входят:
            \begin{scnitemize}
            \item семейство корпоративных ostis-систем, которые “представляют интересы” указанных компаний и лабораторий в рамках рассматриваемого ostis-сообщества и которые обеспечивают автоматизацию “внутренней” деятельности (бизнес-процессов) этих компаний и лабораторий, включая делопроизводство, юридический мониторинг, бухгалтерскую деятельность, административно-хозяйственную деятельность, управление персоналом, управление выполняемыми проектами  и т.д.;
            \item Корпоративная ostis-система OSTIS-сообщества, являющегося субъектом Бизнес-деятельности в области Искусственного интеллекта. Через эту корпоративную ostis-систему осуществляется взаимодействие между членами рассматриваемого ostis-сообщества - между компаниями и лабораториями, работающими в области искусственного интеллекта.
            \end{scnitemize}}
        \scnaddlevel{-1}
    \scnaddlevel{-1}


\scnheader{Консорциум OSTIS}
\scniselement{ostis-сообщество}
\scntext{пояснение}{Весь комплекс деятельности в области \textit{Искусственного интеллекта} мы декомпозировали на шесть форм (частей).Для каждой из этих форм деятельности создается свое \textit{ostis-сообщество}, каждому из которых, в свою очередь, соответствует своя \textit{корпоративная ostis-система}. \textit{Консорциум OSTIS} объединяет все указанные \textit{ostis-сообщества}, включая в свой состав (в состав \textit{Консорциума OSTIS}) прежде всего все \textit{корпоративные ostis-системы} указанных \textit{ostis-сообществ}. Кроме того, для координации деятельности членов самого \textit{Консорциума OSTIS} создается \textit{Корпоративная ostis-система Консорциума OSTIS}. Напомним, что для каждого \textit{ostis-сообщества}, создается соответствующая ему \textit{корпоративная ostis-система}, являющаяся ключевым членом этого \textit{ostis-сообщества} и осуществляющая координацию всех остальных его членов.}

\scnheader{Консорциум OSTIS}
\scniselement{ostis-сообщество}
\scnrelfromlist{член ostis-сообщества}{
Корпоративная система Консорциума OSTIS\\
    \scnaddlevel{1}
    \scnrelto{корпоративная ostis-система}{Консорциум OSTIS}
        \scnaddlevel{1}
        \scnrelto{субъект}{Деятельность в области Искусственного интеллекта, осуществляемая на основе технологии OSTIS}
        \scnaddlevel{-1}
    \scnaddlevel{-1};
OSTIS-портал научных знаний в области Искусственного интеллекта\\
    \scnaddlevel{1}
    \scnrelto{корпоративная ostis-система}{OSTIS-сообщество научно-исследовательской деятельности в области искусственного интеллекта}
        \scnaddlevel{1}
        \scnrelto{субъект}{научно-исследовательская деятельность в области Искусственного интеллекта, осуществляемая на основе технологии OSTIS}
        \scnaddlevel{-1}
    \scnaddlevel{-1};
Метасистема IMS.ostis\\
    \scnaddlevel{1}
    \scnrelto{корпоративная ostis-система}{OSTIS-сообщество Проекта IMS.ostis}
        \scnaddlevel{1}
        \scnrelto{субъект}{Проект IMS.ostis}
        \scnaddlevel{-1}
    \scnaddlevel{-1};
Корпоративная система OSTIS-сообщества Проекта разработки универсального интерпретатора логико-семантических моделей ostis-систем\\
    \scnaddlevel{1}
    \scnrelto{корпоративная ostis-система}{Корпоративная ostis-система OSTIS-сообщество Проекта разработки универсального интерпретатора логико-семантических моделей ostis-систем}
        \scnaddlevel{1}
        \scnrelto{субъект}{Проект разработки универсального интерпретатора логико-семантических моделей ostis-систем}
        \scnaddlevel{-1}
    \scnaddlevel{-1};
Корпоративная система OSTIS-сообщества специализированной инженерии в области Искусственного интеллекта, осуществляемой на основе Технологии OSTIS\\
    \scnaddlevel{1}
    \scnrelto{корпоративная ostis-система}{OSTIS-сообщество Специализированной инженерии в области Искусственного интеллекта, осуществляемой на основе технологии OSTIS}
        \scnaddlevel{1}
        \scnrelto{субъект}{Специализированная инженерия в области Искусственного интеллекта, осуществляемая на основе технологии OSTIS}
        \scnaddlevel{-1}
    \scnaddlevel{-1};
Корпоративная система OSTIS-сообщества образовательной деятельности в области Искусственного интеллекта, осуществляемой на основе технологии OSTIS\\
    \scnaddlevel{1}
    \scnrelto{корпоративная ostis-система}{OSTIS-сообщество Образовательной деятельности в области Искусственного интеллекта, осуществляемой на основе технологии OSTIS}
        \scnaddlevel{1}
        \scnrelto{субъект}{Образовательная деятельность в области Искусственного интеллекта, осуществляемая на основе технологии OSTIS}
        \scnaddlevel{-1}
    \scnaddlevel{-1};
Корпоративная система OSTIS-сообщества Бизнес-деятельности в области Искусственного интеллекта, осуществляемой на основе Технологии OSTIS\\
    \scnaddlevel{1}
    \scnrelto{корпоративная ostis-система}{OSTIS-сообщество Бизнес-деятельности в области Искусственного интеллекта, осуществляемой на основе технологии OSTIS}
        \scnaddlevel{1}
        \scnrelto{субъект}{Бизнес-деятельность в области Искусственного интеллекта, осуществляемая на основе Технологии OSTIS}
        \scnaddlevel{-1}}

\scnheader{Консорциум OSTIS}
\scnnote{Конвергенция и интеграция различных форм и направлений деятельности в области \textit{Искусственного интеллекта} должна проходить через каждого персонального члена \textit{Консорциума OSTIS} - желательно, чтобы большинство из них были одновременно: 
\begin{scnitemize}
\item и участниками научно-исследовательской деятельности в области \textit{Искусственного интеллекта} (аспирантами докторами и т.д.);
\item и участниками совершенствования (развития) целостного комплекса методов и средств проектирования и реализации \textit{интеллектуальных компьютерных систем};
\item и разработчиками различных прикладных \textit{интеллектуальных компьютерных систем};
\item и преподавателями, участвующими в подготовке молодых специалистов в области \textit{Искусственного интеллекта}.
\end{scnitemize}}

\scnheader{Консорциум OSTIS}
\scnidtf{OSTIS-сообщество субъектов всех форм и направлений деятельности в области Искусственного интеллекта}
\scnrelto{часть}{Экосистема OSTIS}

\scnheader{Консорциум OSTIS}
\scnidtf{Научно-техническое и учебное объединение специалистов и организаций, работающих в области Искусственного интеллекта}

\scnheader{Консорциум OSTIS}
\scntext{перспективы}{Создание Консорциума OSTIS на основе широкого применения Технологии OSTIS может и должно осуществляться с поэтапным расширением состава участников и поэтапным повышением уровня автоматизации деятельности Консорциума OSTIS. Ключевыми направлениями деятельности Консорциума OSTIS являются:
\begin{scnitemize}
\item Существенное повышение темпов эволюции Ядра технологии OSTIS, темпов перехода на все более совершенные версии стандартов интеллектуальных компьютерных систем, проектных библиотек и средств автоматизации проектирования интеллектуальных компьютерных систем;
\item Разработка компьютеров нового поколения, ориентированных на интерпретацию логико-семантических моделей интеллектуальных компьютерных систем;
\item Разработка иерархического семейства семантически совместимых специализированных технологий проектирования различных классов интеллектуальных компьютерных систем;
\item Создание условий для развития технологий искусственного интеллекта в направлении унификации интеллектуальных компьютерных систем для обеспечения их конвергенции и семантической совместимости.
\end{scnitemize}}