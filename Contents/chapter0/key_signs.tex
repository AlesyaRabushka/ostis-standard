\newpage

\scnheader{Теория и Технология проектирования интеллектуальных компьютерных систем}
\scnrelfromvector{ключевые знаки}{Технология проектирования интеллектуальных компьютерных систем
;технология\\
	\scnaddlevel{1}
	\scnsuperset{открытая технология}
		\scnaddlevel{1}
		\scnidtf{технология, доступная не только для тех, кто желает ее использовать, но и для тех, кто желает участвовать в ее развитии -- для того, чтобы технология быстро развивалась, она должна иметь широкий круг своих пользователей и разработчиков}
		\scnaddlevel{-1}
	\scnsuperset{технология проектирования}
	\scnsuperset{технология производства}
	\scnsuperset{технология эксплуатации}
	\scnsuperset{онтологическая технология}
		\scnaddlevel{1}
		\scnidtf{технология выполнения соответствующего вида деятельности, в основе которой лежит иерархическая система формальных онтологий, обеспечивающая четкую стратификацию указанной деятельности и наследование свойств между различными уровнями детализации этой
деятельности}
		\scnsuperset{технология онтологического проектирования}
		\scnsuperset{технология онтологического производства}
		\scnsuperset{технология онтологической эксплуатации}
		\scnaddlevel{-1}
		\scnaddlevel{-1}
;интеллектуальная компьютерная система
			\scnaddlevel{1}
			\scnidtf{искусственная интеллектуальная система}
			\scnsubset{интеллектуальная система}
				\scnaddlevel{1}
				\scnidtf{интеллектуальная кибернетическая система}
				\scnsubset{кибернетическая система}
				\scnaddlevel{-1}
		\scnsuperset{гибридная интеллектуальная компьютерная система}
		\scnaddlevel{1}
		\scnidtf{интеллектуальная компьютерная система, в которой глубоко интегрированы различные виды знаний и различные модели решения задач}
		\scnaddlevel{-1}
	\scnaddlevel{-1}
;семантическая совместимость интеллектуальных компьютерных систем\scnsupergroupsign
	\scnaddlevel{1}
	\scnidtf{свойство, определяющее степень (уровень) семантической совместимости каждой пары интеллектуальных компьютерных систем\scnsupergroupsign}
	\scnaddlevel{-1}
;семейство семантически совместимых интеллектуальных компьютерных систем*
	\scnaddlevel{1}
	\scnidtf{семейство интеллектуальных компьютерных систем, все пары которых имеют одинаковый уровень семантической совместимости (взаимопонимания)}
	\scnaddlevel{-1}
;семантическая унификация интеллектуальных компьютерных систем 
	\scnaddlevel{1}
	\scnidtf{процесс обеспечения высокого уровня семантической совместимости интеллектуальных компьютерных систем в ходе их проектирования, эксплуатации и реинжиниринга}
	\scnaddlevel{-1}
;документация
	\scnaddlevel{1}
	\scnidtf{документация соответствующего объекта, представленная в виде раздела базы знаний некоторой интеллектуальной компьютерной системы}
	\scnaddlevel{-1}
;публикация документации
	\scnaddlevel{1}
	\scnidtf{внешнее представление документации, доступное широкому кругу читателей}
	\scnsubset{внешнее представление фрагмента базы знаний интеллектуальной компьютерной системы}
;публикация документации*
	\scnaddlevel{1}
	\scnidtf{быть публикацией заданной документации*}
	\scnidtfexp{Бинарное ориентированное \textit{отношение}, каждая \textit{пара} которого связывает знак некоторого \textit{раздела базы знаний} со знаком \textit{файла}, который является внешним представлением указанного раздела, а также является либо копией электронной публикации материалов этого раздела, либо оригинал-макетом бумажной публикации указанных материалов*}
	\scnaddlevel{-1}
;оглавление публикации документации\\
;оглавление публикации документации*
	\scnaddlevel{1}
	\scnidtf{быть оглавлением заданной публикации соответствующей документации*}
	\scnidtfexp{Бинарное ориентированное \textit{отношение}, каждая \textit{пара} которого связывает знак некоторого \textit{раздела базы знаний} либо знак \textit{файла}, содержащего некоторый \textit{документ}, с описанием иерархии \uline{всех} \textit{разделов}, входящих в состав указанного \textit{раздела базы знаний} либо указанного \textit{документа}*}
	\scnaddlevel{-1}
;база знаний интеллектуальной компьютерной системы
;раздел базы знаний интеллектуальной компьютерной системы
;конкатенация подразделов*\\
	\scnaddlevel{1}
	\scnexplanation{Бинарное ориентированное \textit{отношение}, каждая \textit{пара} которого связывает \textit{знак} некоторого \textit{раздела базы знаний} либо знак \textit{файла}, содержащего некоторый \textit{документ}, с упорядоченным множеством всех \uline{непосредственных} подразделов указанного \textit{раздела базы знаний} или указанного \textit{документа}*}
	\scnaddlevel{-1}
;Искусственный интеллект
	\scnaddlevel{1}	
	\scnidtf{Научно-техническая дисциплина, направленная на изучение интеллектуальных систем для построения искусственных интеллектуальных систем}
	\scniselement{научно-техническая дисциплина}
	\scnidtf{кибернетическая система, имеющая достаточно высокий уровень интеллекта}
	\scnsubset{кибернетическая система}
	\scnaddlevel{1}
	\scnidtf{система, в основе функционирования которой лежит обработка информации}
	\scnaddlevel{-1}
	\scnsuperset{интеллектуальная компьютерная система}
	\scnaddlevel{1}
	\scnidtf{искусственная интеллектуальная система}
	\scnidtf{интеллектуальная компьютерная система}
	\scnaddlevel{-1}
	\scnaddlevel{-1}	
;Внутренний язык интеллектуальных компьютерных систем
;информационная конструкция, принадлежащая Внутреннему языку интеллектуальных компьютерных систем
;знак, входящий в состав информационной конструкции, принадлежащей Внутреннему языку интеллектуальных компьютерных систем
;внешний идентификатор знака, входящего в текст Внутреннего языка интеллектуальной компьютерной системы
;внешний язык интеллектуальных компьютерных систем
	\scnaddlevel{1}
	\scnidtf{язык коммуникации интеллектуальных компьютерных систем с их пользователями и другими интеллектуальными компьютерными системами}
	\scnhaselement{Язык графического представления знаний интеллектуальных компьютерных систем}
	\scnhaselement{Язык линейного представления знаний интеллектуальных компьютерных систем}
	\scnhaselement{Язык структурированного представления знаний интеллектуальных компьютерных систем}
	\scnaddlevel{-1}
;знание интеллектуальной компьютерной системы\\
	\scnaddlevel{1}
	\scnsuperset{внутреннее представление знания интеллектуальной компьютерной системы}
	\scnsuperset{внешнее представление знания интеллектуальной компьютерной системы}
	\scnaddlevel{-1}
;предметная область
	\scnaddlevel{1}
	\scnidtf{предметная область, представленная в памяти интеллектуальной компьютерной системы}
	\scnidtf{sc-модель предметной области}
	\scnaddlevel{-1}
;онтология
	\scnaddlevel{1}
	\scnidtf{формальная онтология, представленная в памяти интеллектуальной компьютерной системы}
	\scnaddlevel{-1}
;предметная область и онтология
	\scnaddlevel{1}
	\scnidtf{объединение предметной области и онтологии}
	\scnaddlevel{-1}
;кибернетическая система\\
	\scnaddlevel{1}
	\scnsuperset{интеллектуальная система}
	\scnsuperset{компьютерная система}
		\scnaddlevel{1}
		\scnsuperset{интеллектуальная компьютерная система}
		\scnaddlevel{-1}
	\scnaddlevel{-1}
;решатель задач кибернетической системы
;интерфейс кибернетической системы
;база знаний
;традиционная компьютерная технология
	\scnaddlevel{1}
	\scnidtf{современная информационная технология}
	\scnaddlevel{-1}
;технология искусственного интеллекта
;логико-семантическая модель интеллектуальной компьютерной системы
	\scnaddlevel{1}
	\scnidtf{результат проектирования интеллектуальной компьютерной системы}
	\scnidtf{стартовое состояние базы знаний проектируемой интеллектуальной компьютерной системы}
	\scnaddlevel{-1}
;семантическая сеть
;семантический язык\\
	\scnaddlevel{1}
	\scnidtf{язык, построенный на основе семантических сетей}
	\scnaddlevel{-1}
;семантическая модель базы знаний
;семантическая модель решателя задач
;семантическая модель интерфейса компьютерной системы
;семантическая окрестность\\
	\scnaddlevel{1}
	\scnidtf{семантическая спецификация}
	\scnaddlevel{-1}
;логическая формула
;логическая онтология\\
	\scnaddlevel{1}
	\scnsubset{онтология}
	\scnaddlevel{-1}
;внешняя информационная конструкция интеллектуальной компьютерной системы
;файл интеллектуальной компьютерной системы\\
;свойство\\
	\scnaddlevel{1}
	\scnidtf{параметр}
	\scnidtf{множество множеств сущностей, имеющих некоторое общее свойство}
	\scnsubset{класс классов}
	\scnaddlevel{-1}
;величина\\
	\scnaddlevel{1}
	\scnidtf{значение свойства}
	\scnaddlevel{-1}
;шкала
;структура\\
	\scnaddlevel{1}
	\scnidtf{фрагмент базы знаний интеллектуальной компьютерной системы}
	\scnaddlevel{-1}
;теоретико-множественная онтология\\
	\scnaddlevel{1}
	\scnsubset{онтология}
	\scnaddlevel{-1}
;материальная сущность
;пространственная сущность
;темпоральная сущность\\
	\scnaddlevel{1}
	\scnidtf{временная сущность}
	\scnidtf{временно существующая сущность}
	\scnaddlevel{-1}
;темпоральная сущность базы знаний интеллектуальной компьютерной системы\\
	\scnaddlevel{1}
	\scnsubset{темпоральная сущность}
	\scnaddlevel{-1}
;действие
;задача
;план
;протокол
;метод\\
	\scnaddlevel{1}
	\scnidtf{метод решения задач заданного класса}
	\scnaddlevel{-1}
;задачная онтология\\
	\scnaddlevel{1}
	\scnsubset{онтология}
	\scnidtf{онтология методов решения задач в рамках заданной предметной области}
	\scnaddlevel{-1}
;внутренний агент интеллектуальной компьютерной системы
;Базовый язык программирования интеллектуальных компьютерных систем
;денотационная семантика языка программирования
;операционная семантика языка программирования
;информационно-поисковый агент интеллектуальной компьютерной системы
;логическое исчисление
;логический агент интеллектуальной компьютерной системы
;сообщение
;интерфейсное действие интеллектуальной компьютерной системы
;агент пользовательского интерфейса интеллектуальной компьютерной системы
;знаковая конструкция
;естественный язык
;методика разработки интеллектуальных компьютерных систем
;средства разработки интеллектуальных компьютерных систем
;базовый интерпретатор логико-семантических моделей интеллектуальных компьютерных систем
;семантический ассоциативный компьютер
;Библиотека многократно используемых компонентов интеллектуальных компьютерных систем
;встраиваемая интеллектуальная компьютерная система
;понимание информации
;противоречие в базе знаний интеллектуальной компьютерной системы
;информационная дыра в базе знаний интеллектуальной компьютерной системы\\
	\scnaddlevel{1}
	\scnidtf{неполнота в базе знаний интеллектуальной компьютерной системы}
	\scnaddlevel{-1}
;разработчик баз знаний интеллектуальных компьютерных систем
;Экосистема нтеллектуальных компьютерных систем
;интеллектуальный портал научно-технических знаний
;интеллектуальная справочная система
;интеллектуальная help-система
;интеллектуальная корпоративная система
;интеллектуальная система в сфере образования\\
	\scnaddlevel{1}
	\scnsuperset{интеллектуальная обучающая система}
	\scnaddlevel{-1}
;интеллектуальная система автоматизации проектирования
;интеллектуальная система управления проектированием
;интеллектуальная система управления производством
;Метасистема интеллектуальных компьютерных систем
}

\newpage