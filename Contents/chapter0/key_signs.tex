\scnrelfromvector{ключевые знаки}{Технология OSTIS
	\scnaddlevel{1}
	\scnidtf{Open Semantic Technology for Intelligent Systems}
	\scnidtf{Открытая технология онтологического проектирования, производства и эксплуатации семантически совместимых гибридных интеллектуальных компьютерных систем}
	\scnaddlevel{-1}
;технология\\
	\scnaddlevel{1}
	\scnsuperset{открытая технология}
		\scnaddlevel{1}
		\scnidtf{технология, доступная не только для тех, кто желает ее использовать, но и для тех, кто желает участвовать в ее развитии -- для того, чтобы технология быстро развивалась, она должна иметь широкий круг своих пользователей и разработчиков}
		\scnaddlevel{-1}
	\scnsuperset{технология проектирования}
	\scnsuperset{технология производства}
	\scnsuperset{технология эксплуатации}
	\scnsuperset{онтологическая технология}
		\scnaddlevel{1}
		\scnidtf{технология выполнения соответствующего вида деятельности, в основе которой лежит иерархическая система формальных онтологий, обеспечивающая четкую стратификацию указанной деятельности и наследование свойств между различными уровнями детализации этой
деятельности}
		\scnsuperset{технология онтологического проектирования}
		\scnsuperset{технология онтологического производства}
		\scnsuperset{технология онтологической эксплуатации}
		\scnaddlevel{-1}
		\scnaddlevel{-1}
;интеллектуальная компьютерная система
			\scnaddlevel{1}
			\scnidtf{искусственная интеллектуальная система}
			\scnsubset{интеллектуальная система}
				\scnaddlevel{1}
				\scnidtf{интеллектуальная кибернетическая система}
				\scnsubset{кибернетическая система}
				\scnaddlevel{-1}
		\scnsuperset{гибридная интеллектуальная компьютерная система}
		\scnaddlevel{1}
		\scnidtf{интеллектуальная компьютерная система, в которой глубоко интегрированы различные виды знаний и различные модели решения задач}
		\scnaddlevel{-1}
	\scnaddlevel{-1}
;семантическая совместимость интеллектуальных компьютерных систем\scnsupergroupsign
	\scnaddlevel{1}
	\scnidtf{свойство, определяющее степень (уровень) семантической совместимости каждой пары интеллектуальных компьютерных систем\scnsupergroupsign}
	\scnaddlevel{-1}
;семейство семантически совместимых интеллектуальных компьютерных систем*
	\scnaddlevel{1}
	\scnidtf{семейство интеллектуальных компьютерных
систем, все пары которых имеют одинаковый уровень семантической совместимости (взаимопонимания)}
	\scnaddlevel{-1}
;семантическая унификация интеллектуальных компьютерных систем 
	\scnaddlevel{1}
	\scnidtf{процесс обеспечения высокого уровня семантической совместимости \textit{интеллектуальных компьютерных систем} в ходе их проектирования, эксплуатации и реинжиниринга}
	\scnaddlevel{-1}
;ostis-система
	\scnaddlevel{1}
	\scnidtf{интеллектуальная компьютерная система, разработанная по \textit{Технологии OSTIS}}
	\scnsubset{интеллектуальная компьютерная система}
	\scnaddlevel{-1}
;ostis-документация
	\scnaddlevel{1}
	\scnidtf{документация соответствующего объекта, представленная в виде \textit{раздела базы знаний} некоторой \textit{ostis-системы}}
	\scnhaselement{Документация Технологии OSTIS}
	\scnaddlevel{-1}
;публикация ostis-документации
	\scnaddlevel{1}
	\scnidtf{внешнее представление \textit{ostis-документации}, доступное широкому кругу читателей}
	\scnsubset{внешнее представление фрагмента базы знаний ostis-системы}
	\scnhaselement{Публикация Документации Технологии OSTIS-2021}
		\scnaddlevel{1}
		\scnnote{Здесь имеется в виду Документации Технологии OSTIS версии 2021 года}
		\scnaddlevel{-1}
	\scnaddlevel{-1}
;публикация ostis-документации*
	\scnaddlevel{1}
	\scnidtf{быть публикацией заданной \textit{ostis-документации}*}
	\scnidtfexp{Бинарное ориентированное \textit{отношение}, каждая \textit{пара} которого связывает знак некоторого \textit{раздела базы знаний} со знаком \textit{файла}, который является внешним представлением указанного раздела, а также является либо копией электронной публикации материалов этого раздела, либо оригинал-макетом бумажной публикации указанных материалов*}
	\scnaddlevel{-1}
;оглавление публикации ostis-документации\\
	\scnaddlevel{1}
	\scnhaselement{Оглавление Публикации Документации Технологии OSTIS-2021}
	\scnaddlevel{-1}
;оглавление публикации ostis-документации*
	\scnaddlevel{1}
	\scnidtf{быть оглавлением заданной публикации соответствующей ostis-документации*}
	\scnidtfexp{Бинарное ориентированное \textit{отношение}, каждая \textit{пара} которого связывает знак некоторого \textit{раздела базы знаний} либо знак \textit{файла}, содержащего некоторый \textit{документ}, с описанием иерархии \uline{всех} \textit{разделов}, входящих в состав указанного \textit{раздела базы знаний} либо указанного \textit{документа}*}
	\scnaddlevel{-1}
;база знаний ostis-системы
;раздел базы знаний \textit{ostis-системы}
%% Текст в кружке, не знаю как выделить, пока поставил textit
;конкатенация подразделов*\\
	\scnaddlevel{1}
	\scnexplanation{Бинарное ориентированное \textit{отношение}, каждая \textit{пара} которого связывает \textit{знак} некоторого \textit{раздела базы знаний} либо знак \textit{файла}, содержащего некоторый \textit{документ}, с упорядоченным множеством всех \uline{непосредственных} подразделов указанного \textit{раздела базы знаний} или указанного \textit{документа}*}
	\scnaddlevel{-1}
;Искусственный интеллект\\
	\scnaddlevel{1}	
	\scnidtf{Научно-техническая дисциплина, направленная на изучение \textit{интеллектуальных систем} для построения \textit{искусственных интеллектуальных систем}}
	\scniselement{научно-техническая дисциплина}
	\scnaddlevel{-1}	
;интеллектуальная система\\
	\scnaddlevel{1}
	\scnidtf{кибернетическая система, имеющая достаточно высокий уровень интеллекта}
	\scnsubset{кибернетическая система}
	\scnaddlevel{1}
	\scnidtf{система, в основе функционирования которой лежит обработка \textit{информации}}
	\scnaddlevel{-1}
	\scnsuperset{интеллектуальная компьютерная система}
	\scnaddlevel{1}
	\scnidtf{искусственная \textit{интеллектуальная система}}
	\scnidtf{интеллектуальная \textit{компьютерная система}}
	\scnaddlevel{-1}	
	\scnaddlevel{-1}	
;SC-код
	\scnaddlevel{1}
	\scnidtf{Semantic Computer Code}
	\scnidtf{Внутренний язык \textit{ostis-систем}}
	\scnaddlevel{-1}
;sc-конструкция
	\scnaddlevel{1}
	\scnidtf{информационная конструкция, принадлежащая \textit{SC-коду}}
	\scnsuperset{sc-текст}
	\scnaddlevel{1}
	\scnidtf{синтаксически целостная sc-конструкция}
	\scnsuperset{sc-знание}
	\scnaddlevel{1}
	\scnidtf{семантически целостный sc-текст}
	\scnaddlevel{-1}
	\scnaddlevel{-1}
	\scnaddlevel{-1}
;sc-элемент
	\scnaddlevel{1}
	\scnidtf{\textit{знак}, входящий в в состав \textit{sc-конструкции}}
	\scnaddlevel{-1}
;sc-идентификатор
	\scnaddlevel{1}
	\scnidtf{внешний идентификатор \textit{sc-элемента}}
	\scnidtf{внешний идентификатор знака, входящего в текст Внутреннего языка \textit{ostis-системы}}	
	\scnaddlevel{-1}
;внешний язык ostis-систем
	\scnaddlevel{1}
	\scnidtf{язык коммуникации \textit{ostis-систем} с их пользователями и другими \textit{ostis-системами}}
	\scnhaselement{SCg-код}
		\scnaddlevel{1}
		\scnidtf{Semantic Computer Code graphical}
		\scnidtf{Язык графического представления знаний \textit{ostis-систем}}
		\scnaddlevel{-1}
	\scnhaselement{SCs-код}
		\scnaddlevel{1}
		\scnidtf{Semantic Computer Code string}
		\scnidtf{Язык линейного представления знаний \textit{ostis-систем}}
		\scnaddlevel{-1}
	\scnhaselement{SCn-код}
		\scnaddlevel{1}
		\scnidtf{Semantic Computer Code natural}
		\scnidtf{Язык структурированного представления знаний \textit{ostis-систем}}
		\scnaddlevel{-1}
	\scnaddlevel{-1}
;знание ostis-системы\\
	\scnaddlevel{1}
	\scnsuperset{внутреннее представление знания ostis-системы}
	\scnsuperset{внешнее представление знания ostis-системы}
	\scnaddlevel{-1}
;предметная область
	\scnaddlevel{1}
	\scnidtf{предметная область, представленная в памяти \textit{ostis-системы}}
	\scnidtf{\textit{sc-модель} предметной области}
	\scnaddlevel{-1}
;онтология
	\scnaddlevel{1}
	\scnidtf{формальная онтология, представленная в памяти \textit{ostis-системы}}
	\scnidtf{\textit{sc-модель} онтологии}
	\scnaddlevel{-1}
;предметная область и онтология
	\scnaddlevel{1}
	\scnidtf{объединение \textit{предметной области} и онтологии}
	\scnaddlevel{-1}
;кибернетическая система\\
	\scnaddlevel{1}
	\scnsuperset{интеллектуальная система}
	\scnsuperset{компьютерная система}
		\scnaddlevel{1}
		\scnsuperset{интеллектуальная компьютерная система}
		\scnaddlevel{-1}
	\scnaddlevel{-1}
;решатель задач кибернетической системы
;интерфейс кибернетической системы
;база знаний
;традиционная компьютерная технология
	\scnaddlevel{1}
	\scnidtf{современная информационная технология}
	\scnaddlevel{-1}
;технология искусственного интеллекта
;логико-семантическая модель ostis-системы
	\scnaddlevel{1}
	\scnidtf{формальная логико-семантическая модель \textit{ostis-системы}, представленная в \textit{SC-коде}}
	\scnidtf{sc-модель \textit{ostis-системы}}
	\scnidtf{результат проектирования \textit{ostis-системы}}
	\scnidtf{стартовое состояние \textit{базы знаний} проектируемой \textit{ostis-системы}}
	\scnaddlevel{-1}
;семантическая сеть
;семантический язык\\
	\scnaddlevel{1}
	\scnidtf{\textit{язык}, построенный на основе \textit{семантических сетей}}
	\scnaddlevel{-1}
;семантическая модель базы знаний\\
	\scnaddlevel{1}
	\scnidtf{\textit{база знаний}, представленная на \textit{семантическом языке}}
	\scnsuperset{sc-модель базы знаний}
	\scnaddlevel{-1}
;семантическая модель решателя задач\\
\scnaddlevel{1}
\scnsuperset{sc-модель решателя задач}
\scnaddlevel{-1}
;семантическая модель интерфейса кибернетической системы\\
\scnaddlevel{1}
\scnsuperset{sc-модель интерфейса кибернетической системы}
\scnaddlevel{-1}
;семантическая окрестность\\
	\scnaddlevel{1}
	\scnidtf{семантическая спецификация}
	\scnidtf{sc-окрестность}
	\scnaddlevel{-1}
;логическая формула\\
\scnaddlevel{1}
	\scnidtf{sc-модель логической формулы}
\scnaddlevel{-1}
;логическая онтология\\
	\scnaddlevel{1}
	\scnsubset{онтология}
	\scnaddlevel{-1}
;внешняя информационная конструкция ostis-системы\\
	\scnaddlevel{1}
	\scnidtf{информационная конструкция, записанная не в SC-коде}
	\scnaddlevel{-1}
;файл ostis-системы\\
;свойство\\
	\scnaddlevel{1}
	\scnidtf{параметр}
	\scnidtf{семейство множеств сущностей, эквивалентных в заданном семействе}
	\scnsubset{класс классов}
	\scnaddlevel{-1}
;величина\\
	\scnaddlevel{1}
	\scnidtf{значение свойства}
	\scnaddlevel{-1}
;шкала
;структура\\
	\scnaddlevel{1}
	\scnidtf{фрагмент базы знаний \textit{ostis-системы}}
	\scnaddlevel{-1}
;теоретико-множественная онтология\\
	\scnaddlevel{1}
	\scnsubset{онтология}
	\scnaddlevel{-1}
;материальная сущность
;пространственная сущность
;темпоральная сущность\\
	\scnaddlevel{1}
	\scnidtf{временная сущность}
	\scnidtf{временно существующая сущность}
	\scnaddlevel{-1}
;темпоральная сущность базы знаний ostis-системы\\
	\scnaddlevel{1}
	\scnsubset{темпоральная сущность}
	\scnaddlevel{-1}
;действие
;задача
;план
;протокол
;метод\\
	\scnaddlevel{1}
	\scnidtf{метод решения задач заданного класса}
	\scnaddlevel{-1}
;задачная онтология\\
	\scnaddlevel{1}
	\scnsubset{онтология}
	\scnidtf{онтология методов решения задач в рамках заданной предметной области}
	\scnaddlevel{-1}
;внутренний агент ostis-системы
;Язык SCP\\
	\scnaddlevel{1}
	\scnidtf{Базовый язык программирования ostis-систем}
	\scnidtf{Semantic Code Programming}
	\scnaddlevel{-1}
;денотационная семантика языка программирования
;операционная семантика языка программирования
;информационно-поисковый агент ostis-системы
;логическое исчисление
;логический агент ostis-системы
;сообщение
;интерфейсное действие ostis-системы
;агент пользовательского интерфейса ostis-системы
;знаковая конструкция
;естественный язык
;методика разработки ostis-систем
;средство разработки ostis-систем
;базовый интерпретатор логико-семантических моделей ostis-систем
;семантический ассоциативный компьютер
;Библиотека многократно используемых компонентов ostis-систем\\
	\scnaddlevel{1}
	\scnidtf{Библиотека \textit{технологии OSTIS}}
	\scnaddlevel{-1}
;встраиваемая ostis-система
;понимание информации
;противоречие в базе знаний ostis-системы
;информационная дыра в базе знаний ostis-системы\\
	\scnaddlevel{1}
	\scnidtf{неполнота в базе знаний \textit{ostis-системы}}
	\scnaddlevel{-1}
;разработчик баз знаний ostis-систем
;Экосистема OSTIS
;интеллектуальный портал научно-технических знаний
;интеллектуальная справочная система
;интеллектуальная help-система
;интеллектуальная корпоративная система
;интеллектуальная система в сфере образования\\
	\scnaddlevel{1}
	\scnsuperset{интеллектуальная обучающая система}
	\scnaddlevel{-1}
;интеллектуальная система автоматизации проектирования
;интеллектуальная система управления проектированием
;интеллектуальная система управления производством
;Метасистема IMS.ostis
;ostis-система\\
\scnaddlevel{1}
\scnidtf{интеллектуальная компьютерная система, построенная на Технологии OSTIS}
\scnaddlevel{-1}
;интеллектуальная компьютерная система\\
\scnaddlevel{1}
\scnidtf{компьютерная система, основанная на знаниях}
\scnaddlevel{-1}
;база знаний
\scnaddlevel{1}
\scnidtf{система знаний интеллектуальной компьютерной системы}
\scnaddlevel{-1}
}