\scnheader{Разработка технологии производства спроектированных интеллектуальных компьютерных систем}
\scntext{предлагаемый подход}{Проект разработки универсального интерпретатора логико-семантических моделей ostis-систем.}
\scnaddlevel{1}
    \scntext{продукт}{универсальный интерпретатор логика семантических моделей ostis-систем}
    \scnaddlevel{1}
    	\scnidtf{"пустая"{} ostis-система -- ostis-система, на базе которой можно построить любую 					ostis-систему, если логико-семантическую модель этой системы, загрузить в память 					указанной выше "пустой"{} ostis-системы.}
    \scnaddlevel{-1}
\scnaddlevel{-1}

\scnheader{Проект разработки универсального интерпретатора логико-семантических моделей ostis-систем.}
\scnidtf{Проект реализации универсальной абстрактной sc-машины}
\scntext{альтернативный подпроект}{Проект Программной реализации универсальной абстрактной sc-машины}
\scntext{продукт}{универсальный интерпретатор логико-семантических моделей ostis-систем}
\scnaddlevel{1}
\scnidtf{Базовый интерпретатор логико-семантических моделей ostis-систем}
\scnidtf{Интерпретатор Универсальной абстрактной sc-машины}
\scnaddlevel{-1}
\scntext{альтернативный подпроект}{Проект разработки Универсального sc-компьютера}
\scntext{применение}{Подчеркнём, что разные альтернативные варианты реализации универсального интерпретатора логико-семантических моделей ostis-систем (универсальной абстрактной sc-машины) никоим образом не влияет на процесс и результат проектирования ostis-систем, то есть на процесс и результат построения логико-семантических моделей разрабатываемых ostis-систем.
Другими словами, принципы представления и структуризации логико-семантических моделей ostis-систем и архитектура универсального интерпретатора этих моделей чётко стратифицированы и, следовательно, могут эволюционировать в достаточной степени независимо друг от друга. Тем не менее некоторая зависимость всё же есть -- согласованная трактовка понятия универсальной sc-машины и согласованная форма (язык) передача логико-семантической модели разрабатываемой ostis-системы из Метасистемы IMS.ostis в "пустую" ostis-систему}
}

\scnheader{Универсальная абстрактная sc-машина}
\scnidtf{Базовый язык программирования ostis-систем с его синтаксисом, денотационной семантикой и операционный семантикой.}
\scnidtf{Языки программирования SCP (Semantic Code Programming)}
\scntext{смотрите}{Раздел 2.1.3.2
}
\scnidtf{Абстрактная scp-машина}



\scnheader{Проект программной реализации универсальной абстрактной sc-машины.}
\scnidtf{Проект разработки программной реализации универсальной абстрактной sc-машины на современных компьютерах.}
\scntext{продукт}{
Программная реализация универсальной абстрактной sc-машины.

\scnaddlevel{1}
    \scntext{смотрите}{Раздел 2.3.1.1}
\scnaddlevel{-1}
}



\scnheader{Проект разработки универсального sc-компьютера.}
\scnidtf{Проект разработки аппаратной реализации универсальной абстрактной sc-машины в виде компьютера нового поколения, ориентированного на использование в интеллектуальных компьютерных системах( в нашем случае – в ostis-системах)}
\scntext{продукт}{
Универсальный sc-компьютер.
\scnaddlevel{1}
	\scnidtf{Универсальный компьютер универсальный ostis-компьютер.}
	\scnidtf{Семантический ассоциативный компьютеры для ostis-систем.}
	\scnidtf{Аппаратная реализация абстрактный sc-машины.}
    \scntext{смотрите}{Раздел 2.3.1.2}
    \scntext{примечание}{Тот факт, что универсальный sc-компьютер, разрабатывается под конкретную технологию проектирование интеллектуальных компьютерных систем (технология OSTIS), которая развивается накапливает опыт разработки и внедрение самых различных прикладных интеллектуальных систем независимо от наличия универсальных sc-компьютеров, имеет принципиальное значение. Опыт создания компьютеров, имеющих принципиально новую архитектуру, показывает, что разработка компьютеров нового поколения, без серьезной подготовки технологий их применения, без подготовки соответствующий инфраструктуры приводит к неэффективному использованию результатов разработки и к  быстрому их моральному старению.}
\scnaddlevel{-1}
}

\scntext{примечание}{
Разработка Универсального sc-компьютера является важнейшим следующим этапом развития технологии OSTIS, который обеспечит существенное повышение производительности (быстродействия) ostis-систем.;Развитие технологий искусственного интеллекта неизбежно приведёт к необходимости создания компьютеров, принципиально нового поколения, предназначенных для использования в интеллектуальных компьютерных системах. Поэтому изначально ориентация Технологии OSTIS на компьютеры нового поколения является принципиальной и весьма перспективной особенностью Технологии OSTIS, обеспечивающей её высокую конкурентоспособность.
}





\scnheader{Проект разработки универсального интерпретации логико-семантических моделей ostis-систем}
\scntext{примечание}{
При построении любого интерпретатора любой информационной машины (в нашем случае – абстрактной из sc-машины) должны быть чётко полно, а самое главное на формальном языке (в нашем случае –  sc-коде) описано следующее:

	\begin{scnitemize}    
 		\item синтаксис,денотационная семантика и операционная семантика интерпретируемой машины ( 		в нашем случае для абстрактной sc-машины это синтаксис и денотационная семантика sc-кода и 		языка SCP, а также операционной системой языка SCP). 
 
 		\item синтаксис,денотационная семантика и операционная семантика интерпретирующей 					информационный машины.
 
 		\item соотношение между указанными формальными моделями интерпретируемой информационной 			машины и интерпретирующей информационной машины, определяющее семантическую и операционную 		(функциональную) эквивалентность.
 	\end{scnitemize} 
		Подчеркнем, что без построения указанной строгой формальной модели соответствия 					(эквивалентности) интегрируемой и интерпретирующей информационной машины организовать 				качественную коллективную разработку интерпретаторов сложной информационной машины (				например, абстрактной sc-машины) невозможно, так как будет совершаться большое количество 			поздно обнаруживаемых ошибок.

}



\scnheader{Проект разработки универсального интерпретатора логико-семантических моделей ostis-систем}
\scntext{примечание}{
	Разрабатываемые сейчас варианты реализации \textit{универсального интерпретатора логико-семантических моделей ostis-систем }(программный и аппаратный вариант) являются в известной мере привычными объектами проектирования для современных технологий проектирования программных систем и технологий проектирования интегрированых микросхем и их комплексов.\\
	Тем не менее, повышение уровня сложности указанных обьектов проектирования и указанных характеристик проектирования требует существенного повышения уровня интеллекта у соответствующих систем автоматизации (поддержки) проектирования). \textit{Технология OSTIS} уже имеет достаточный опыт разработки.\\
	\textit{ostis-систем автоматизации проектирования} (достаточно указать метасистему), обеспечивающую автоматизацию проектирования ostis-систем).Таким образом для повышения качества разработки \textit{программной реализации универсальной абстрактной sc-машины} и разработки \textit{универсального sc-компьютера} целесообразно разработать, соответственно, ostis-систему поддержки проектирования сложных программных систем, а также о систему поддержки проектирования интегральных микросхем и их комплексов.Здесь речь может идти об интеллектуальных настройках над существующими средствами автоматизации проектирования и управления проектами.\\
	При проектировании \textit{программной реализации универсальной абстрактной sc-машины}, а также \textit{универсального sc-компьютера} такая интеллектуальная надстройка абсолютно необходима, поскольку при проектировании указанных объектов необходимо четко отслеживать соответствия между компонентами этих объектов и компонентами интерпретируемой ими \textit{универсальной абстрактной машины}. Актуальность указанной интеллектуальной надстройки обусловлена также тем, что \textit{универсальная абстрактная машина} может корректироваться. \\
	Следует отметить возможную связь между процессом проектирования \textit{программной реализация универсальной абстрактной sс-машины} и проектированием \textit{универсального sc-компьютера}. Дело в том, что \textit{программную реализацию универсальной абстрактной sc-смашины} можно и нужно рассматривать как программную модель не только интегрируемой \textit{универсальной абстрактной sc-машины}, но и проектируемого \textit{универсального sc-компьютера}. Таким образом, реализацию универсальной sc-машины можно развивать в 2 направлениях:
		\begin{scnitemize}    
 		\item в направлении повышения её производительности;
 		\item в направлении более детальной эмуляции универсального sc-компьютера на уровне взаимодействия всё более «мелких» компонентов этого компьютера.
 	\end{scnitemize} 
}


















