\bigskip
\scnfragmentcaption

\scnheader{следует отличать*}
\scnhaselementset{
конвергенция
\scnaddlevel{1}
\scnidtf{Процесс сближения структурных и/или функциональных характеристик нескольких (как минимум двух) заданных сущностей}
\scnidtf{Процесс конвергенции заданных сущностей в ходе их изменения, совершенствование, эволюции}
\scnsubset{процесс}
\scnaddlevel{-1};
конвергенция\scnsupergroupsign 
\scnaddlevel{1}
\scnidtf{Степень близости (сходство) заданных сущностей}
\scniselement{свойство}
\scnaddlevel{-1}
}
\scnheader{конвергенция} 
\scnnote{ 
\textit{Конвергенция} пар конкретных искусственных сущностей (например, технических систем) есть стремление их унификацию (в частности, к стандартизации), т.е. стремление к минимизации многообразия форм решения аналогичных практических задач -- стремление к тому, чтобы все, что можно сделать одинаково, сделалось одинаково, но без ущерба требуемого качества. Последнее очень важно, так как безграмотная стандартизация может привести к существенному торможению прогресса. Ограничение многообразия форм не должно приводить к ограничению содержания, возможностей. Образно говоря, "словам должно быть тесно, а мыслям -- свободно".}
\scnnote{Методологически конвергенция искусственно создаваемых сущностей (артефактов) сводится (1) к выявлению (обнаружению) принципиальных сходств между этими сущностями, которые часто весьма закамуфлированы и их трудно "увидеть", и (2) к реализации обнаруженных сходств одинаковым образом (в одинаковой форме, в одинаковом "синтаксисе"). Образно говоря, от "семантической"{} (смысловой) эквивалентности требуется перейти и к "синтаксической" эквивалентности. Кстати, в этом как раз и заключается суть (идея) смыслового представления информации (знаний), целью которого является создание такой языковой среды (\textit{смыслового пространства}), в рамках которого (1) семантически эквивалентные информационные конструкции полностью совпадали, а (2) конвергенция информационных конструкций сводилась бы к выявлению изоморфных фрагментов этих конструкций.}
\scnnote{Очень важно уточнить, формализовать понятие конвергенции (конвергенции знаний, методов, модели решения задач, конвергенции интеллектуальных компьютерных систем в целом)}
\scnsuperset{конвергенция информационных конструкций}
\scnaddlevel{1}
\scnidtf{конвергенция синтаксических и семантических свойств информационных конструкций }
\scnaddlevel{-1}
\scnsuperset{конвергенция языков}
\scnsuperset{конвергенция научных дисциплин}
\scnaddlevel{1}
\scnidtf{конвергенция различных научных дисциплин или различных направлений одной и той же и дисциплины}
\scnaddlevel{-1}
\scnsuperset{конвергенция баз знаний}
\scnsuperset{конвергенция моделей решения задач}
\scnsuperset{конвергенция гибридных решателей задач}
\scnsuperset{конвергенция кибернетических систем}
\scnsuperset{конвергенция интеллектуальных систем}
\scnaddlevel{1}
\scnsuperset{конвергенция интеллектуальных систем, направленная на обеспечение их \uline{семантической совместимости}}
\scnaddlevel{-1}

\scnheader{конвергенция результатов научно-технической деятельности}
\scnnote{Важным препятствием для конвергенции результатов научно-технической деятельности является сформировавшийся в науке и технике акцент на выявлении не сходств, а отличий. Чтобы убедиться в этом достаточно обратить внимание на то, что уровень научных результатов оценивается научной \uline{новизной}, которая может имитироваться новизной не по существу, а по форме представления (например, с помощью новых понятий или даже новых терминов). Результаты в технике, например, в патентах также оцениваются \uline{отличиями} от предшествующих технических решений. Но для конвергенции нужны другие акценты -- ни поиск отличий, а выявление неочевидных сходств и превращения их в очевидные сходства, представленные в одинаковой \uline{форме}.}

\scnheader{совместимость\scnsupergroupsign}
\scnidtf{совместимость заданных двух или более сущностей\scnsupergroupsign}
\scnidtf{простота интеграции заданной группы сущностей\scnsupergroupsign}
\scnidtf{интегрируемость\scnsupergroupsign}
\scnnote{Степень (уровень) совместимости заданных сущностей может рассматриваться как оценка результата их конвергенции. Чем качественнее (основательнее, глубже) проведена конвергенция заданных сущностей, тем выше уровень их совместимости и, собственно, тем легче их интегрировать.}

\scnsuperset{cовместимость информационных конструкций\scnsupergroupsign}
\scnaddlevel{1}
\scnsuperset{семантическая совместимость информационных конструкций\scnsupergroupsign}
\scnaddlevel{-1}
\scnsuperset{совместимость языков\scnsupergroupsign}
\scnaddlevel{1}
\scnsuperset{семантическая совместимость языков\scnsupergroupsign}
\scnaddlevel{-1}
\scnsuperset{семантическая совместимость научных дисциплин\scnsupergroupsign}
\scnsuperset{совместимость баз знаний\scnsupergroupsign}
\scnsuperset{совместимость моделей решения задач\scnsupergroupsign}
\scnsuperset{совместимость кибернетических систем\scnsupergroupsign}
\scnaddlevel{1}
\scnsuperset{семантическая совместимость кибернетических систем\scnsupergroupsign}
\scnaddlevel{-1}
\scnsuperset{семантическая совместимость\scnsupergroupsign}

\scnheader{интеграция*}
\scnidtf{объединение нескольких разных сущностей, в результате чего возникает некоторая объединённая целостная сущность*}
\scnsuperset{эклектичная интеграция*}
\scnaddlevel{1}
\scnidtf{Интеграция разнородных (гетерогенных) сущностей, которой не предшествует конвергенция (сближение) этих сущностей*}
\scnaddlevel{-1}
\scnsuperset{глубокая интеграция*}
\scnnote{Понятие \textit{интеграции*} и особенно понятие \textit{глубокой интеграции*} имеет тесную связь с понятием \textit{конвергенции\scnsupergroupsign}. Чем выше степень конвергенции (степень сближения) интегрируемых объектов, тем выше качество результата интеграции. Особенно, если речь идёт о глубокой интеграции.}

\scnheader{глубокая интеграция*}
\scnidtf{"бесшовная"{} интеграция*}
%TODO ссылка на Грибову
\scnidtf{интеграция однородных сущностей, предполагающая глубокую взаимную "диффузию"{} (сращивание) соединяемых сущностей, которая не обязательно должна осуществляться физически}
\scnnote{Примером виртуальной глубокой интеграции является формирование коллектива \uline{семантический совместимых} индивидуальный кибернетических систем}
\scnidtf{бесшовная интеграция*}
\scnidtf{гибридизация*}
\scnidtf{интеграция, результатом которой являются гибридные объекты*}
\scnidtf{интеграция, которой предшествует высокий уровень конвергенции интегрируемых объектов*}
\scnidtf{(конвергенция + интеграция)*}
\scnidtf{"бесшовная"{} интеграция}
\scnidtf{интеграция, в результате которой возникает гибридная система*}
\scnidtf{интеграция, которой предшествует конвергенция (в частности, унификация) интегрируемых систем, приведение этих систем к максимально похожему виду (общему знаменателю)*}
%TODO сложно при чтении воспринимать, конвергенция и приведение как-то сливаются, становится не совсем понятно, к чему относится приведение к конвергенции или к интеграции, может как-то более явно указать, что конвергенция это то приведение?
\scnidtf{интеграция с "диффузией"{} , взаимопроникновением на основе унификации того, что можно сделать одинаковым*}

\scnheader{интеграция*}
\scnsuperset{интеграция информационных конструкций}
\scnsuperset{интеграция языков}
\scnsuperset{интеграция научных дисциплин}
\scnsuperset{интеграция баз знаний}
\scnsuperset{интеграция моделей решения задач}
\scnsuperset{интеграции индивидуальных кибернетических систем}
\scnaddlevel{1}
\scnsuperset{слияние индивидуальных кибернетических систем}
\scnaddlevel{1}
\scnidtf{преобразование нескольких \uline{искусственных} индивидуальных кибернетических систем в интегрированную индивидуальную кибернетическую систему, которая способна решать все задачи, каждая из которых могла бы быть решена в рамках какой-либо из интегрируемых систем}
\scnaddlevel{-1}
\scnsuperset{формирование коллектива индивидуальных кибернетических систем}
\scnaddlevel{1}
\scnidtf{формирования многоагентной системы, состоящей из индивидуальных кибернетических систем}
\scnaddlevel{-1}
\scnnote{Эффективность интеграции индивидуальных кибернетических систем определяется тем, насколько объем задач, решаемых коллективом индивидуальных кибернетических систем, превысит объединение объёмов задач, решаемых членами коллектива в отдельности.}
\scnaddlevel{-1}

\bigskip
\scnendstruct \scninlinesourcecommentpar{Завершили Сегмент ``\textit{Текущее состояние и проблемы дальнейшего развития деятельности в области Искусственного интеллекта}''}