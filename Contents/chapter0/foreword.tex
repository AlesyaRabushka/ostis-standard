\scsection{Предисловие к Документации Технологии OSTIS}
\label{foreword}

\begin{SCn}

\scnsectionheader{\currentname}

\scnstartsubstruct

\bigskip
\scnfilelong{Данная \textit{публикация} посвящена описанию предлагаемой нами \textbf{\textit{Технологии OSTIS}} (Открытой Семантической Технологии Проектирования Интеллектуальных Компьютерных Систем -- Open Semantic Technology for Intelligent Systems). \textit{Интеллектуальные компьютерные системы}, разработанные на этой технологии, мы назвали \textbf{ostis-системами}. Особенностью \textit{публикации} является то, что она оформлена в виде \uline{внешнего представления} основной части \textit{базы знаний} специальной \textit{ostis-системы}, которая предназначена для комплексной поддержки проектирования \uline{семантически совместимых} \textit{ostis-систем}. Эту систему мы назвали \textbf{\textit{Метасистемой IMS.ostis}} (Intelligent MetaSystem for ostis-systems). Последовательность изложения материала во внешнем представлении \textit{базы знаний} не является единственно возможным маршрутом прочтения (просмотра) \textit{базы знаний}. Каждый читатель, войдя в \textbf{\textit{Метасистему IMS.ostis}}, может выбрать любой другой маршрут навигации по этой \textit{базе знаний}, задавая указанной метасистеме те \textit{вопросы}, которые в текущий момент его интересуют. Таким образом, читая предлагаемый вашему вниманию текст и одновременно работая с \textbf{\textit{Метасистемой IMS.ostis}}, можно значительно быстрее усвоить детали \textbf{\textit{Технологии OSTIS}} и значительно быстрее приступить к непосредственному использованию указанной технологии. Этому также способствует большое количество примеров семантических моделей различных фрагментов \textit{интеллектуальных компьютерных систем}. 

Основной вид \textit{разделов базы знаний \textbf{ostis-системы}} -- это формальное представление различных \textbf{\textit{предметных областей}} вместе с соответствующими им \textbf{\textit{онтологиями}}. При этом явно указываются связи между этими \textbf{\textit{предметными областями} и \textit{онтологиями}}. Таким образом, \textit{база знаний} \textbf{\textit{Метасистемы IMS.ostis}}, как и любых других \textit{интеллектуальных компьютерных систем}, построенных по \textbf{\textit{Технологии OSTIS}}, представляет собой иерархическую систему связанных между собой формальных моделей \textit{предметных областей} и соответствующих им \textit{онтологий}.

В основе \textit{Технологии OSTIS} лежит предлагаемая нами унификация \textit{интеллектуальных компьютерных систем}, основанная, в свою очередь, на \textit{смысловом представлении знаний} в \textit{памяти интеллектуальных компьютерных систем}. Таким образом, данную \textit{публикацию} можно рассматривать как проект \textit{стандарта} семантических моделей \textit{интеллектуальных компьютерных систем}. Последующие \textit{публикации}, посвящённые детальному описанию различных компонентов \textit{Технологии OSTIS}, будут также оформляться как внешнее представление соответствующих \textit{разделов базы знаний} \scnbigspace \textit{Метасистемы IMS.ostis} и будут отражать следующие этапы развития  \textit{Технологии OSTIS}, следующие версии этой технологии, и, соответственно, следующие версии \textit{Метасистемы IMS.ostis}. 

Надеемся на то, что состав авторов таких \textit{публикаций} будет расширяться при четкой спецификации вклада каждого из них. 

Все основные положения \textit{Технологии OSTIS} рассматривались и обсуждались на ежегодных \textit{конференциях OSTIS}, которые стали важным стимулирующим фактором становления и развития \textit{Технологии OSTIS}. Мы благодарим всех активных участников этих конференций.

Важной задачей данной  монографии   была выработка стилистики формализованного представления научно-технической информации, которая одновременно была бы понятна как человеку, так и интеллектуальной компьютерной системе. По сути это принципиально новый подход к оформлению научно-технических результатов, позволяющий:
\begin{scnitemize}
 \item существенно повысить уровень автоматизации анализа качества (корректности, целостности) научно-технической информации;
 \item интеллектуальным компьютерным системам непосредственно (без какой-либо дополнительной “ручной” доработки) использовать информацию (знания)содержащуюся в разработанных специалистами документах;
 \item существенно упростить согласование точек зрения различных специалистов, входящих в коллектив разработчиков той или  иной научно-технической документации.
\end{scnitemize}
Для достижения указанной цели нам было важно привлечь к обсуждению и анализу материала данной монографии как можно больше коллег, участвующих и участвовавших в развитии и применении Технологии OSTIS. При этом некоторых коллег мы включили в число соавторов соответствующих разделов монографии.
Основной целью написания данной монографии является создание технологических и организационных предпосылок к принципиально новому  подходу к организации \textit{научно-технической деятельности} в любой области и, в частности, в области создания и перманентного развития комплексной технологии проектирования и производства семантически совместимых интеллектуальных компьютерных систем (\textit{Технологии OSTIS}). Суть указанного подхода заключается  в глубокой конвергенции и интеграции результатов деятельности всех специалистов, участвующих в создании и развитии \textit{Технологии OSTIS}, путем организации коллективной разработки \textit{базы знаний}, являющейся формальным представлением полной \textit{Документации Технологии OSTIS}, отражающей текущее состояние этой технологии. Для обеспечения высоких темпов и высокого качества данной деятельности мы заинтересованы в расширении контингента участников этой деятельности.

На данном этапе к разработке и оформлению различных разделов Документации по текущей версии \textit{Технологии OSTIS} были привлечены студенты, магистранты, аспиранты и преподаватели кафедры интеллектуальных информационных технологий Белорусского государственного университета информатики и радиоэлектроники и кафедры интеллектуальных информационных технологий Брестского государственного технического университета, а также сотрудники ОАО «Савушкин продукт» и ООО «Интелиджент семантик системс». Так, например,
\begin{scnitemize}
\item соавторами Раздела \ref{sec:sd_neuronetworks} "\nameref{sec:sd_neuronetworks}"{} являются В.А. Головко, М.В. Ковалёв, А.А. Крощенко, Е.В. Михно;
\item соавторами Раздела \ref{sec:sd_ecosys_enterprise} "\nameref{sec:sd_ecosys_enterprise}"{} являются В.В. Таберко, Д.С. Иванюк, В.В. Касьяник;
\item соавторами Раздела \ref{sec:sd_interfaces} "\nameref{sec:sd_interfaces}"{} являются М.Е. Садовский, В.А. Захарьев, С.А. Никифоров, Р.А. Коршунов;
\end{scnitemize}
\scnauthorcomment{Добавить соавторов}
}

\scnheader{ostis-система}
\scnidtf{интеллектуальная компьютерная система, построенная на Технологии OSTIS}
\scnheader{интеллектуальная компьютерная система}
\scnidtf{компьютерная система, основанная на знаниях}
\scnheader{база знаний}
\scnidtf{система знаний интеллектуальной компьютерной системы}

\scnendstruct

\end{SCn}
