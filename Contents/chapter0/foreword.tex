
\scsection{Предисловие к Документации Технологии OSTIS}
\label{foreword}

\begin{SCn}

\scnsectionheader{\currentname}

\scnstartsubstruct

\bigskip
\scnfilelong{Данная публикация посвящена описанию предлагаемой нами Открытой Семантической Технологии Проектирования Интеллектуальных Компьютерных Систем (Технологии OSTIS -- Open Semantic Technology for Intelligent Systems). Интеллектуальные компьютерные системы, разработанные на этой технологии, мы назвали ostis-системами. Особенностью публикации является то, что она оформлена в виде \uline{исходного текста} основной части базы знаний специальной ostis-системы, которая предназначена для комплексной поддержки проектирования семантически совместимых ostis-систем. Эту систему мы назвали \textbf{\textit{Метасистемой IMS.ostis}} (Intelligent MetaSystem for ostis-systems). Последовательность изложения материала в исходном тексте базы знаний не является единственно возможным маршрутом прочтения (просмотра) базы знаний. Каждый читатель, войдя в \textbf{\textit{Метасистему IMS.ostis}}, может выбрать любой другой маршрут навигации по этой базе знаний, задавая указанной метасистеме те вопросы, которые в текущий момент его интересуют. Таким образом, читая данный исходный текст и одновременно работая с \textbf{\textit{Метасистемой IMS.ostis}}, можно значительно быстрее усвоить детали \textbf{\textit{Технологии OSTIS}} и значительно быстрее приступить к непосредственному использованию указанной технологии. Этому также способствует большое количество примеров семантических моделей различных фрагментов интеллектуальных компьютерных систем. 

Основной вид разделов базы знаний ostis-системы -- это формальное представление различных \textbf{\textit{предметных областей}} вместе с соответствующими им \textbf{\textit{онтологиями}}. При этом явно указываются связи между этими \textbf{\textit{предметными областями}} и \textbf{\textit{онтологиями}}. Таким образом, база знаний \textbf{\textit{Метасистемы IMS.ostis}}, как и любых других интеллектуальных компьютерных систем, построенных по \textbf{\textit{Технологии OSTIS}}, представляет собой иерархическую систему связанных между собой формальных моделей предметных областей и соответствующих им онтологий.

В основе Технологии OSTIS лежит предлагаемая нами унификация интеллектуальных компьютерных систем, основанная, в свою очередь, на смысловом представлении знаний в памяти интеллектуальных компьютерных систем. Таким образом, данную публикацию можно рассматривать как проект стандарта семантических моделей интеллектуальных компьютерных систем. Последующие публикации, посвящённые детальному описанию различных компонентов Технологии OSTIS, будут также оформляться как исходные тексты соответствующих разделов базы знаний Метасистемы IMS.ostis и будут отражать следующие этапы развития технологии OSTIS, следующие версии этой технологии, и, соответственно, следующие версии Метасистемы IMS.ostis. 

Надеемся на то, что состав авторов таких публикаций будет расширяться при четкой спецификации вклада каждого из них. 

Все основные положения Технологии OSTIS рассматривались и обсуждались на конференциях OSTIS, которые стали важным стимулирующим фактором становление и развития Технологии OSTIS. Мы благодарим всех активных участников ежегодных конференций OSTIS.
}

\scnendstruct

\end{SCn}
