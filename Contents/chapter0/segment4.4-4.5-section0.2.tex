\bigskip
\scnfragmentcaption

\scnheader{Специализированная инженерия в области Искусственного интеллекта}
\scnrelfrom{предлагаемый подход}{Специализированная инженерия, осуществляемая на основе Технологии OSTIS}
\scnaddlevel{1}
\scnrelfromset{декомпозиция}{Разработка ostis-систем автоматизации проектирования различных классов ostis-систем\\
    \scnaddlevel{1}
    \scnidtf{Разработка специализированных ostis-технологий}
    \scnrelfromlist{часть}{Разработка ostis-систем автоматизации проектирования ostis-систем автоматизации проектирования\\
        \scnaddlevel{1}
        \scnidtf{Разработка ostis-технологий проектирования}
        \scnaddlevel{-1}
    ;Разработка ostis-систем автоматизации проектирования ostis-систем автоматизации производства\\
        \scnaddlevel{1}
        \scnidtf{Разработка ostis-технологий управления производством}
        \scnaddlevel{-1}
    ;Разработка ostis-систем автоматизации проектирования ostis-систем управления транспортными системами\\
        \scnaddlevel{1}
        \scnidtf{Разработка ostis-технологий управления транспортными системами}
        \scnaddlevel{-1}
    ;Разработка ostis-систем автоматизации проектирования диагностических ostis-систем\\
        \scnaddlevel{1}
        \scnidtf{Разработка ostis-технологий диагностики (технической, медицинской)}
        \scnaddlevel{-1}
    ;Разработка ostis-систем автоматизации проектирования обучающих ostis-систем\\
        \scnaddlevel{1}
        \scnidtf{Разработка ostis-технологий обучения людей}
        \scnaddlevel{-1}
    ;Разработка ostis-систем автоматизации проектирования ostis-систем управления "умными"{} домами\\
        \scnaddlevel{1}
        \scnidtf{Разработка ostis-технологий управления "умными"{} домами}
        \scnaddlevel{-1}
    ;Разработка ostis-систем автоматизации проектирования ostis-систем управления "умными"{} больницами\\
        \scnaddlevel{1}
        \scnidtf{Разработка ostis-технологий управления "умными"{} больницами}
        \scnaddlevel{-1}
    ;Разработка ostis-систем автоматизации проектирования ostis-систем управления "умными"{} поликлиниками\\
        \scnaddlevel{1}
        \scnidtf{Разработка ostis-технологий управления "умными"{} поликлиниками}
        \scnaddlevel{-1}
    ;Разработка ostis-систем автоматизации проектирования ostis-систем управления "умными"{} городскими районами\\
        \scnaddlevel{1}
        \scnidtf{Разработка ostis-технологий управления "умными"{} городскими районами}
        \scnaddlevel{-1}
    ;Разработка ostis-систем автоматизации проектирования ostis-систем управления "умными"{} городами\\
        \scnaddlevel{1}
        \scnidtf{Разработка ostis-технологий управления "умными"{} городами}
        \scnaddlevel{-1}}
    \scnaddlevel{-1}
;Разработка (на основе соответствующих ostis-технологий проектирования) ostis-систем автоматизации проектирования различных классов объектов, не являющихся ostis-системами\\
    \scnaddlevel{1}
    \scnrelfromlist{часть}{Разработка семейства ostis-систем автоматизации проектирования различных видов интегральных микросхем;Разработка семейства ostis-систем автоматизации проектирования различных видов автомобилей;Разработка семейства ostis-систем автоматизации проектирования различных видов строительных объектов}
    \scnaddlevel{-1}
;Разработка ostis-систем автоматизации производства\\
    \scnaddlevel{1}
    \scnidtf{Разработка интеллектуальных систем управления производственными предприятиями}
    \scnaddlevel{-1}
;Разработка ostis-систем управления транспортными средствами;Разработка диагностических ostis-систем;Разработка обучающих ostis-систем;Разработка ostis-систем управления "умными"{} домами;Разработка ostis-систем управления "умными"{} больницами;Разработка ostis-систем управления "умными"{} поликлиниками;Разработка ostis-систем управления "умными"{} городскими районами;Разработка ostis-систем управления "умными"{} городами}
\scnaddlevel{-1}

\bigskip
\scnfragmentcaption

\scnheader{Образовательная деятельность в области Искусственного интеллекта}
\scnrelfrom{предлагаемый подход}{Образовательная деятельность в области Искусственного интеллекта, осуществляемая на основе Технологии OSTIS}
    \scnaddlevel{1}
    \scniselement{образовательная деятельность}
    \scniselement{человеческая деятельность, осуществляемая на основе Технологии OSTIS}
        \scnaddlevel{1}
        \scnidtf{человеческая деятельность, комплексная автоматизация которой осуществляется либо индивидуальной \textit{ostis-системой}, либо \textit{коллективом ostis-систем} (сетью ostis-систем)}
        \scnaddlevel{-1}
    \scnrelfrom{субъект}{OSTIS-сообщество Образовательной деятельности в области Искусственного интеллекта, осуществляемой на основе Технологии OSTIS}
		\scnaddlevel{1}    	
    	\scnidtf{глобальное (максимальное) OSTIS-сообщество, осуществляющее Образовательную деятельность в области Искусственного интеллекта и обеспечивающее активное и взаимовыгодное сотрудничество между всеми заинтересованными в этом субъектами и, в первую очередь, с соответствующими кафедрами различных вузов}
    	
    	\scnrelto{часть}{Экосистема OSTIS}
        	\scnaddlevel{1}
	        \scnidtf{глобальная сеть ostis-систем вместе с их пользователями}
    	    \scnidtf{глобальное ostis-сообщество}
	        \scnaddlevel{-1}
    	\scniselement{ostis-сообщество}
        	\scnaddlevel{1}
	        \scnidtf{локальная сеть ostis-систем вместе с их пользователями}
	        \scnaddlevel{-1}
	    \scnexplanation{Данное \textit{ostis-сообщество} включает в себя:
\begin{scnitemize}
    \item все кафедры, которые готовят молодых специалистов в области Искусственного интеллекта и которые могут входить в состав самых различных вузов;
    \item все те организации, которые разрабатывают или эксплуатируют интеллектуальные компьютерные системы и которые готовы сотрудничать с вузами для повышения квалификации поступающих к ним молодых специалистов в области Искусственного интеллекта;
    \item студентов, магистрантов и аспирантов, обучающихся в области Искусственного интеллекта в разных вузах;
    \item их преподавателей;
    \item семейство интеллектуальных обучающих ostis-систем по различным дисциплинам (направлениям) Искусственного интеллекта, которые семантически совместимы и тесно связаны с \textit{OSTIS-порталом научных знаний по Искусственному интеллекту} и с \textit{Метасистемой IMS.ostis};
    \item \textit{OSTIS-портал научных знаний по Искусственному интеллекту}, осуществляющий поддержку развития Общей теории интеллектуальных систем как естественного, так и искусственного происхождения;
    \item \textit{Метасистема IMS.ostis}, осуществляющая поддержку развития Общей теории интеллектуальных компьютерных систем (искусственных интеллектуальных систем) и поддержку развития Базовой универсальной комплексной технологии проектирования интеллектуальных компьютерных систем;
    \item семейство персональных ostis-ассистентов студентов, магистрантов и аспирантов, обучающихся в области Искусственного интеллекта;
    \item семейство персональных ostis-ассистентов преподавателей, осуществляющих подготовку молодых специалистов в области Искусственного интеллекта;
    \item семейство кафедральных корпоративных ostis-систем, осуществляющих управление учебным процессом на уровне кафедр, обеспечивающих подготовку молодых специалистов в области Искусственного интеллекта. В рамках таких корпоративных ostis-систем осуществляется:
    \begin{scnitemizeii}
        \item составление кафедрального расписания занятий на следующий семестр и его согласование с расписанием других кафедр этого же вуза;
        \item распределение учебной нагрузки на очередной семестр и учебный год;
        \item мониторинг проведения различного вида занятий (лекций, консультаций, семинаров, практических занятий, зачетов/экзаменов);
        \item мониторинг самостоятельной деятельности обучаемых (курсовых и дипломных проектов, рефератов, диссертаций, тестов и др.);
        \item фиксация текущего соответствия между учебными дисциплинами и разделами Общей теории интеллектуальных систем и Базовой универсальной комплексной технологии проектирования интеллектуальных компьютерных систем (речь идет не только о дисциплинах, непосредственно относящихся к Искусственному интеллекту, но и о различных общеобразовательных и смежных дисциплинах, таких, как теория познания, методология, иностранные языки, современные компьютерные системы и сети, компьютеры нового поколения, теория алгоритмов и программ, ориентированных на современные компьютеры, семантическая теория алгоритмов и программ, ориентированных на обработку баз знаний и др.). Принципиально важно сформировать у студентов, магистрантов и аспирантов целостную картину проблематики Искусственного интеллекта и место Искусственного интеллекта в общей Картине Мира. Барьеров между учебными дисциплинами быть не должно.
    \end{scnitemizeii}
    \item Корпоративная ostis-система OSTIS-сообщества, являющегося субъектом Образовательной деятельности в области Искусственного интеллекта. Через эту корпоративную ostis-систему осуществляется взаимодействие между всеми членами указанного ostis-сообщества и, прежде всего между кафедрами, осуществляющими подготовку молодых специалистов в области Искусственного интеллекта.
\end{scnitemize}}
        \scnaddlevel{-1}        
    \scnaddlevel{-1}

\scnrelfromvector{принципы, лежащие в основе}{\scnfileitem{Подготовка молодых специалистов в области Искусственного интеллекта должна осуществляться путем поэтапного и непосредственного их подключения к реальным коллективным проектам:
\begin{scnitemize}
    \item к развитию базы знаний по Общей теории интеллектуальных систем, хранимой в памяти соответствующего интеллектуального портала знаний{;}
    \item к развитию базы знаний по \textit{Общей теории интеллектуальных компьютерных систем}, хранимой в памяти соответствующего интеллектуального портала знаний (в памяти \textit{Метасистемы IMS.ostis}){;}
    \item к развитию базы знаний по Базовой комплексной технологии проектирования интеллектуальных компьютерных систем, хранимой в памяти интеллектуальной компьютерной системы автоматизации проектирования интеллектуальных компьютерных систем (в памяти Метасистемы IMS.ostis){;}
    \item к развитию различных методов и средств проектирования различных компонентов интеллектуальных компьютерных систем{;}
    \item к развитию различных специализированных технологий проектирования различных классов интеллектуальных компьютерных систем{;}
    \item к разработке различных прикладных интеллектуальных компьютерных систем на основе развиваемой Базовой (универсальной) комплексной технологии проектирования интеллектуальных компьютерных систем.
\end{scnitemize}}
;\scnfileitem{Каждый студент и магистрант в процессе обучения привлекается к нескольким разным формам деятельности в области Искусственного интеллекта и, в частности, обязательно и к разработке приложений, и к развитию технологий. Специалист, занимающийся автоматизацией какой-либо деятельности должен на себе прочувствовать проблемы и трудности этой автоматизируемой деятельности}
;\scnfileitem{Все студенты, магистранты и преподаватели должны активно участвовать в анализе эффективности своей образовательной деятельности и активно способствовать повышению эффективности и повышению уровня автоматизации этой деятельности с помощью развиваемой технологии проектирования и производства интеллектуальных компьютерных систем. Данный принцип можно условно назвать устранением синдрома "сапожника без сапог"{}.}
;\scnfileitem{Результаты самостоятельной работы студентов и магистрантов (лабораторных работ, практических занятий, рефератов, курсовых работ и проектов, дипломных работ и проектов, магистерских диссертаций) должны быть востребованы в тех проектах, к которым они подключены и должны быть доведены до уровня внедрения в эти проекты, т. е. должны быть по соответствующей процедуре согласованы и одобрены. При этом приветствуется и соответствующим образом поощряется любая такого рода инициатива студентов и магистрантов. Указанная востребованность (полезность) результатов самостоятельной работы студентов и магистрантов предполагает то, что отчеты по этим результатам оформляются в формализованном виде -- в виде исходных текстов соответствующих фрагментов баз знаний. При этом указанные результаты могут требовать как весьма высокой квалификации, так и не очень высокой (например, квалификации первокурсника). К таким несложным, но весьма полезным работам относятся:
\begin{scnitemize}
    \item введение в базы знаний полезных библиографических ссылок и цитат;
    \item сравнительный анализ различных положений, представленных в некоторой разрабатываемой базе знаний;
    \item различные пояснения, примечания и комментарии, вводимые в базу знаний;
    \item спецификация выявленных в разрабатываемой базе знаний ошибок,  противоречий, информационных дыр и информационного мусора;
    \item примеры, иллюстрирующие различные понятия;
    \item упражнения к различным разделам разрабатываемых баз знаний, которые особенно актуальны для интеллектуальных компьютерных систем, используемых в учебном процессе (это не только интеллектуальные обучающие системы).
\end{scnitemize}}
;\scnfileitem{Вклад каждого студента и магистранта в развитие всех проектов, в которых он принимает участие, фиксируется и при подведении итогов по каждому семестру соответствующим образом оценивается. Это своего рода предтеча будущего рынка знаний.}
;\scnfileitem{Учебным пособием по каждой учебной дисциплине должна быть база знаний или некоторый раздел базы знаний некоторой интеллектуальной компьютерной системы. Такой может быть либо интеллектуальная обучающая система, либо, например, Метасистема IMS.ostis. Условием максимально эффективного проведения лекционного занятия является предварительное прочтение студентами или магистрантами материала предстоящей лекции (соответствующего раздела базы знаний). Тогда на лекции можно акцентировать внимание не на изложение материала, опубликованного в виде базы знаний, а на обсуждение непонятных фрагментов этого материала, на обсуждение проблем, касающихся содержания (принципиальных положений) этого материала. Все это формирует культуру взаимопонимания и согласования различных точек зрения, а также способствует повышению качества базы знаний, представляющей материал соответствующей учебной дисциплины.}
;\scnfileitem{Важнейшей задачей подготовки молодых специалистов является формирование у них:
\begin{scnitemize}
	\item высокой математической культуры (культуры формализации);
	\item высокой системной культуры (понимания того, что количество далеко не всегда переходит в ожидаемое качество);
	\item высокого уровня технологической культуры, технологической дисциплины, четкого соблюдения текущих стандартов и способности участвовать в эволюции стандартов;
	\item способности работать в наукоемких проектах в составе творческих коллективов с децентрализованным управлением;
	\item способности к достижению семантической совместимости (взаимопонимания) со своими коллегами;
	\item договороспособности (способности к согласованию различных точек зрения).
\end{scnitemize}}
;\scnfileitem{Подготовку молодых специалистов в области Искусственного интеллекта можно осуществлять с ориентацией на следующие условно выделенные уровни их квалификации:
\begin{scnitemize}
	\item инженерия прикладных интеллектуальных компьютерных систем по заданной технологии{;}
	\item инженерия специализированных технологий проектирования различных классов прикладных интеллектуальных компьютерных систем (на основе базовой универсальной комплексной технологии проектирования интеллектуальных компьютерных систем){;}
	\item инженерия базовой универсальной комплексной технологии проектирования интеллектуальных компьютерных систем{;}
	\item инженерия программных и аппаратных средст, интерпретации логико-семантических моделей интеллектуальных компьютерных систем{;}
	\item инженерия комплексов интеллектуальных компьютерных систем{;}
	\item научно-исследовательская деятельность по развитию \textit{Общей формальной теории интеллектуальных компьютерных систем}.
\end{scnitemize}}}