\begin{SCn}
	\scnsegmentheader{Структура Деятельности в области Искусственного интеллекта}
	
	\scntext{аннотация}{Для того, чтобы рассмотреть проблемы дальнейшего развития \textit{деятельности} в области \textit{Искусственного интеллекта} как \textit{научно-технической дисциплины} и, в частности, проблемы комплексной автоматизации этой \textit{деятельности}, необходимо уточнить структуру указанной \textit{деятельности}.
	}
	
	\scnheader{Искусственный интеллект}
	\scniselement{область человеческой деятельности}
	\scniselement{научно-техническая дисциплина}
	\scnaddlevel{1}
	\scniselement{вид человеческой деятельности}
	\scnaddlevel{-1}
	\scnidtf{Область человеческой деятельности, основными целями которой являются:
		
		\begin{scnitemize}
			\item построение теории  интеллектуальных систем;
			\item создание  технологии разработки интеллектуальных компьютерных систем (искусственных интеллектуальных систем);
			\item переход на принципиально новый уровень комплексной автоматизации всех \textit{видов человеческой деятельности}, который основан на массовом применение \textit{интеллектуальных    компьютерных систем} и который предполагает:
			\begin{scnitemizeii}
				\item не только наличие \textit{интеллектуальных компьютерных систем}, способных понимать друг друга и согласовывать свою деятельность,
				\item но и построение \textit{общей теории  человеческой деятельности},  осуществляемый в условиях нового уровня её автоматизации, ( теории деятельности \textit{smart-общества}), которая должна быть "  понятна" \textit{используемым интеллектуальным компьютерным системам} и которая потребует существенного переосмысления современной организации \textit{человеческой деятельности}.
			\end{scnitemizeii}
		\end{scnitemize}
	}
	
	
	\scnheader{Искусственный интеллект}
	\scnidtf{Научно-техническая деятельность, направленная на построение теории интеллектуальных систем, а также на создание технологии проектирования и производства  искусственных интеллектуальных систем (\textit{интеллектуальных компьютерных систем})}
	\scnidtf{Научно-техническая деятельность в области \textit{Искусственного интеллекта}}
	\scnidtf{Деятельность в области \textit{Искусственного интеллекта}}
	
	\scnidtf{Научно-техническая деятельность, направленная на исследование феномена \textit{интеллекта}, а также на создание искусственных интеллектуальных систем( \textit{интеллектуальных компьютерных систем}) и включающая в себя соответствующую научно-исследовательскую деятельность, инженерно-технологическую, инженерно-прикладную, образовательную и организационную деятельность}
	
	\scnexplanation{Междисциплинарная (трансдисциплинарная) область \textit{научно-технической деятельности}, направленная на разработку и эксплуатацию \textit{интеллектуальных  компьютерных систем},  обеспечивающих автоматизацию различных сфер \textit{человеческой деятельности}.}
	
	\scnidtf{Научно-техническая дисциплина направленная на разработку теории индивидуальных \textit{интеллектуальных компьютерных систем} и \textit{интеллектуальных сообществ} (коллективов) таких систем, а также средств поддержки их проектирования и реализации)}
	
	\scnidtf{\textit{Научно-техническая  дисциплина}, являющаяся частью \textit{кибернетики} (теория кибернетических систем), объектом исследования которой являются  \textit{интеллектуальные компьютерные системы}  (искусственные \textit{интеллектуальные системы})и  целями которой являются
		(1) разработка \textit{теории интеллектуальных компьютерных систем}, 
		(2) разработка \textit{технологии(методов и средств)} \textit{проектирования и производства компьютерных систем}, а также,
		(3)разработка конкретных интеллектуальных компьютерных систем различного назначения.}
	
	
	\scnheader{Искусственный интеллект}
	\scnrelfrom{декомпозиция}{Декомпозиция Искусственного интеллекта по формам деятельности}
	\scnaddlevel{1}
	
	
	
	\scneqtoset{Научно-исследовательская деятельность в области Искусственного интеллекта\\
		\scnaddlevel{1}
		\scnrelfrom{продукт}{Общая теория интеллектуальных систем}
		\scnaddlevel{1}
		\scnidtf{Теория, уточняющая структуру и принципы функционирования \textit{интеллектуальных систем}, а также акцентирующая внимание на причинах (предпосылках) возникновение свойства интеллектуальности (феномены \textit{интеллекта})}
		\scnaddlevel{-1}
		\scniselement{коллективная научно-техническая деятельность}
		\scnaddlevel{1}
		\scniselement{вид человеческой деятельности}
		\scnidtf{научно-исследовательская дисциплина или направление} 
		\scnrelboth{следует отличать}{продукт научно-исследовательской деятельности}
		\scnaddlevel{-2};
		Разработка Базовой комплексной технологии проектирования интеллектуальных компьютерных систем\\
		\scnaddlevel{1}
		\scnrelfrom{продукт}{Базовая комплексная технология проектирования интеллектуальных компьютерных систем}
		\scnaddlevel{1}
		\scnnote{Комплексность данной \textit{технологии}  заключается в том, что она ориентирована
			(1) на проектирование \textit{интеллектуальных компьютерных систем} в целом, а не только отдельных их компонентов и
			(2) на создание объединённой \textit{технологии}, объединяющей самые различные технологические подходы на основе их  \textit{конвергенции} и глубокой  \textit{интеграции}}
		\scnaddlevel{-1}
		\scniselement{разработка технологии проектирования искусственных объектов заданного класса}
		\scnaddlevel{1}
		\scniselement{вид человеческой деятельности}
		\scnaddlevel{-2};
		Разработка Технологии производства спроектированных интеллектуальных компьютерных систем\\
		\scnaddlevel{1}
		\scnrelfrom{продукт}{Технология производства спроектированных интеллектуальных компьютерных систем}
		\scnaddlevel{1}
		\scnidtf{технология реализации (сборки и установки) спроектированных интеллектуальных компьютерных систем}
		\scnnote{Очевидно, что данная \textit{технология} должна быть самым тесным образом связана с \textit{Базовой комплексной технологией проектирования интеллектуальных компьютерных систем} (по крайней мере данная \textit{технология}  должна "знать", какой форме ей на "вход" передаётся результат проектирования). Поэтому имеет смысл говорить об объединённой технологии проектирования и производства \textit{интеллектуальных компьютерных систем}}
		\scnaddlevel{-1}
		\scniselement{производства спроектированных искусственных объектов заданного класса}
		\scnaddlevel{1}
		\scniselement{вид человеческой деятельности}
		\scnaddlevel{-2};
		Специализированная инженерия в области Искусственного интеллекта\\
		\scnaddlevel{1}
		\scnidtf{Множество Процессов разработки (проектирования и производства)  \textit{интеллектуальных компьютерных систем} различного назначения, кроме \textit{интеллектуальных компьютерных систем автоматизации проектирования} и автоматизации производства \textit{интеллектуальных компьютерных систем}}
		\scnrelfrom{продукт}{множество специализированных интеллектуальных компьютерных систем}
		\scnaddlevel{1}
		\scnrelfrom{основной sc-идентификатор*}{прикладная интеллектуальная компьютерная система}
		\scnaddlevel{-1}
		\scnsubset{проектирование и производство искусственного объекта заданного класса на основе заданной технологии}
		\scnaddlevel{1}
		\scniselement{вид человеческой деятельности}
		\scnaddlevel{-2};
		Образовательная деятельность в области Искусственного интеллекта\\
		\scnaddlevel{1}
		\scnidtf{Деятельность, направленная на подготовку молодых специалистов области \textit{Искусственного интеллекта} на перманентное повышение квалификации действующих специалистов в этой области}
		\scnnote{Сложность и высокая степень наукоемкости задач, больших своего решения на текущем этапе развития \textit{Искусственного интеллекта}, добавляют к специалистам, работающим в этой области высокие требования к уровню их:
			\begin{scnitemize}
				\item математической культуры (культуры формализации) ;
				\item системной культуры;
				\item технологической культуры;
				\item инженерная культура;
				\item умения работать коллективных наукоемких процесс проектах.
			\end{scnitemize}
		}
		\scnaddlevel{-1}
		\scnsubset{образовательная деятельность}
		\scnaddlevel{1}
		\scniselement{вид человеческой деятельности}
		\scnaddlevel{-1};
		Бизнес деятельность в области искусственного интеллекта\\
		\scnaddlevel{1}
		\scnexplanation{Речь идет о бизнес-деятельности в широком смысле как деятельности, направленный на создание  инфраструктурных условий для качественного выполнения всех \textit{видов деятельности} в области \textit{Искусственного интеллекта}, а именно:
			\begin{scnitemize}
				\item на разработку и реализацию грамотной научно-технической политики, связывающей как тактические, так и стратегические цели;
				\item на глубокую \textit{конвергенцию} всех форм и \textit{видов деятельности} в области \textit{Искусственного интеллекта};
				\item на организацию взаимовыгодного сотрудничество различных школ и коллективов, работающих в области \textit{Искусственного интеллекта};
				\item на финансовое обеспечение;
				\item на кадровое обеспечение;
				\item на материально-техническое обеспечение;
				\item на организацию проведения различных мероприятий (конференций, выставок, семинаров);
				\item на публикационную деятельность и защиту интеллектуальной собственности;
				\item на материально-техническое обеспечение;
			\end{scnitemize}
		}
		\scnsubset{бизнес-деятельность научно-технической области}
		\scnaddlevel{1}
		\scniselement{вид человеческой деятельности}
		\scnaddlevel{-2}
	}
	
	\scnheader{Разработка базовой комплексной технологии проектирования интеллектуальных компьютерных систем}
	\scnrelfromset{декомпозиция}{Разработка общей теории интеллектуальных компьютерных систем\\
		\scnaddlevel{1}  
		\scnrelfrom{продукт}{Общая теория интеллектуальных компьютерных систем}
		\scniselement{разработка теории искусственных объектов заданного класса}
		\scnaddlevel{1}
		\scniselement{вид человеческой деятельности}
		\scnaddlevel{-2}
		;Разработка общей теории интеллектуальных компьютерных систем\\
		\scnaddlevel{1}
		\scnrelfrom{продукт}{Теория проектирования интеллектуальных компьютерных систем}
		\scnaddlevel{1}
		\scnnote{В состав этой теории входят методы проектирования, библиотеки проектирования и спецификация используемых индустриальных средств.}
		\scnaddlevel{-1}
		\scnidtf{Разработка \textit{Теории проектной деятельности} по построению формальных моделей \textit{интеллектуальных компьютерных систем}}
		\scniselement{теории проектирования интеллектуальных объектов заданного класса}
		\scnaddlevel{1}
		\scniselement{вид человеческой деятельности}
		\scnaddlevel{-2};
		Разработка комплекса средств автоматизации проектирования интеллектуальных компьютерных систем\\
		\scnaddlevel{1}
		\scniselement{разработка комплекса средств автоматизации проектирования искусственных объектов заданного класса}
		\scnaddlevel{1}
		\scniselement{вид человеческой деятельности}
		\scnaddlevel{-2}
	}
	
	\scnheader{Разработка Технологии производства спроектированных интеллектуальных компьютерных систем}
	\scnrelfromset{декомпозиция}{Разработка Теории производства спроектированных интеллектуальных компьютерных систем\\
		\scnaddlevel{1}  
		\scnidtf{Разработка \textit{Теории производственной деятельности} по реализации (сборке и установке) \textit{интеллектуальных компьютерных систем}}
		\scnrelfrom{продукт}{Теория производства спроектированных интеллектуальных компьютерных систем}
		\scnaddlevel{1}
		\scnnote{В состав этой \textit{теории} входят \textit{методы производства} (сборки и установки) \textit{интеллектуальных компьютерных систем}, а также спецификация \textit{используемых инструментальных средств}.}
		\scnaddlevel{-2}
		;Разработка комплекса средств автоматизации производства интеллектуальных компьютерных систем\\
		\scnaddlevel{1}
		\scniselement{разработка комплекса средств автоматизации производства искусственных объектов заданного класса}
		\scnaddlevel{1}
		\scniselement{вид человеческой деятельности}
		\scnaddlevel{-2}
	}
	
	\scnheader{Специализированная инженерия в области Искусственного интеллекта}
	\scnidtf{проектирование и производство конкретной \textit{интеллектуальной компьютерной системы} по заданной технологии}  
	\scnrelfromset{обобщенная декомпозиция}{проектирование конкретные интеллектуальные компьютерные системы\\
		\scnaddlevel{1}
		\scnrelfromset{обобщенное разбиение}{проектирование интеллектуальной компьютерной системы автоматизации проектирования соответствующего класса интеллектуальных компьютерных систем; проектирование интеллектуальной компьютерной системы автоматизации проектирования соответствующего класса объектов, не являющихся интеллектуальными компьютерными системами; проектирование интеллектуальной компьютерной системы,не являющейся системой автоматизации проектирования
		}
		\scnaddlevel{-1}
		;производство конкретной спроектированной интеллектуальной компьютерной системы\\
	}
	
	\scnheader{Производство конкретной спроектированной интеллектуальной компьютерной системы}
	\scnnote{Производственная деятельность направленная на \textit{производство} (реализацию) спроектированной \textit{интеллектуальной компьютерной системы} значительно уступает по уровню сложности деятельности проектированию этой \textit{интеллектуальной компьютерной системы}, так как это производство сводится к сборке результата \textit{проектирования} (формальной логико-семантической модели разрабатываемой \textit{интеллектуальной компьютерной системы}) и загрузки этой модели в память компьютера или программного \textit{универсального интерпретатора логико-семантических моделей интеллектуальных компьютерных систем}, в качестве которого может быть использован
		(1) либо специально разработанный для этого \textit{компьютер}, ориентированные на обработку \textit{баз знаний} и интерпретацию различных \textit{интеллектуальных моделей решения задач},
		(2) либо программная эмуляция такого \textit{компьютера}, реализованная на современных  \textit{компьютерах} фон-неймановской архитектуры.
		
		Простота производства спроектированных систем характерна для производства не только интеллектуальных, но и любых других \textit{компьютерных систем}.
		
		Мы выделяем "производственный" этап реализации \textit{интеллектуальных компьютерных систем} для того, чтобы по аналогии рассматривать этап производства (массового, мелкосерийного, разового производства) спроектированных искусственных объектов любого другого вида(микросхем, автомобилей, зданий,  компьютеров).
		
		Очевидно, что массовое производство некоторых видов продукции может иметь весьма большой уровень сложности, но при этом суть \textit{производственной деятельности} как процесса перехода от проекта (спецификации) некоторого объекта к его реализации остаётся одной и той же независимо от уровня сложности  реализуемого объекта (производимой продукции).
	}	
	\scnaddlevel{-1}	
	\scnheader{следует отличать*}
	\scnhaselementset{специализированная интеллектуальная компьютерная система;интеллектуальная компьютерная система автоматизации проектирования интеллектуальных компьютерных систем;интеллектуальная компьютерная система автоматизации производства спроектированных интеллектуальных компьютерных систем} 
	\scnhaselementset{человеческая деятельность\\
		\scnaddlevel{1}
		\scnsuperset{научно-исследовательская деятельность}
		\scnsuperset{научно-техническая деятельность}
		\scnaddlevel{-1}
		;продукт человеческой деятельности\\
		\scnaddlevel{1}
		\scnsuperset{продукт научно-исследовательской деятельности}
		\scnsuperset{продукты научно-технической деятельности стакан}
		\scnaddlevel{-1}
	}	
	
	
	\scnheader{Искусственный интеллект}
	\scnrelfrom{декомпозиция}{Декомпозиция Искусственного интеллекта по направлениям}
	\scnaddlevel{1}
	
	\scneqtoset{Разработка теории представление знаний и технологии проектирования баз знаний актуальных компьютерных систем\\
		;Разработка теории решение задач и технологии проектирования решателей задач интеллектуальных компьютерных систем\\
		\scnaddlevel{1}
		\scnrelfromset{декомпозиция}{Разработка Теории решений интерфейсных задач и технологии проектирования соответствующих решателей\\
			\scnaddlevel{1}  
			\scnrelfrom{часть}{Разработка теории естественно языковых интерфейсов интеллектуальных компьютерных систем и технологии их проектирования}
			\scnaddlevel{-1}
			;Разработка теории решения информационных задач в базах знаний интеллектуальных компьютерных систем и технологии проектирования соответствующих решателей\\
			;Разработка теории решения поведенческих задач во внешней среде интеллектуальных компьютерных систем и технологии проектирования соответствующих решателей\\
		}
		\scnrelfromset{декомпозиция}{Разработка логических моделей решения задач и технологии проектирования соответствующих решателей;Разработка нейросетевых моделей решение задач и технологии проектирования соответствующих решателей} 
		\scnaddlevel{-1}
		;Разработка универсальных интерпретаторов базовых моделей обработки баз знаний интеллектуальных компьютерных систем\\
		;Разработка общей теории человеческой деятельности автоматизируемой с помощью комплекса взаимодействующих интеллектуальных компьютерных систем\\
	}
\end{SCn}