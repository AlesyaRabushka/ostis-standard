\scnheader{конвергенция в области Искусственного интеллекта}

\scnrelfrom{разбиение}{Направления конвергенции в области Искусственного интеллекта}
     \scnaddlevel{1}
\scnhaselement{конвергенция Искусственного интеллекта со смежными научными дисциплинами}
    \scnaddlevel{1}
	\scnrelfrom{примечание}{\scnstartsetlocal\\
		\scnheaderlocal{Искусственный интеллект}
	\scnrelbothlist{смежная дисциплина}{Логика;Психология человека;Зоопсихология;Нейропсихология;Этология;Кибернетика;Общая теория систем;Семиотика;Лингвистика}
		\scnendstruct
	}
	\scnaddlevel{-1}

\scnhaselement{конвергенция различных направлений Искусственного интеллекта}
\scnaddlevel{1}
	\scnidtf{Конвергенция различных направлений исследований в области Искусственного интеллекта, результатом которой должна быть формализованная практически ориентированная общая теория интеллектуальных систем и, в частности, интеллектуальных компьютерных систем}
	\scnnote{Разобщенность различных направлений исследований в области искусственного интеллекта является главным препятствием создания общей комплексной технологии проектирования интеллектуальных компьютерных систем}
	\scnidtf{Конвергенция между различными направлениями и продуктами научных исследований в области искусственного интеллекта, результатом (целевым продуктом) которой должна стать общая формальная теория интеллектуальных компьютерных систем}
\scnaddlevel{-1}
\scnhaselement{конвергенция различного вида знаний в памяти интеллектуальной компьютерной системы}
\scnaddlevel{1}
	\scnidtf{Конвергенция и интеграция внутреннего представления в памяти интеллектуальной компьютерной системы различного вида знаний}
\scnaddlevel{-1}
\scnhaselement{конвергенция различных моделей решения задач в памяти интеллектуальной компьютерной системы}
\scnaddlevel{1}
	\scnidtf{Конвергенция и интеграция различных моделей решения задач, которая включает логико-семантическую типологию задач и типологию моделей решения задач и требует уточнения семантики таких понятий как задача, класс задач, метод, класс методов, модель решения задач (иерархический метод интерпретации класса методов)}
\scnaddlevel{-1}
\scnhaselement{конвергенция интеллектуальных компьютерных систем}
\scnaddlevel{1}
	\scnidtf{Обеспечение семантической совместимости (взаимопонимания) интеллектуальных систем, согласование используемых онтологий}
	\scnidtf{Конвергенция между различными прикладными компьютерными системами, результатом (целевым продуктом) которой должна стать экосистема, состоящая из перманентно эволюционирующих, семантически совместимых и взаимодействующих интеллектуальных компьютерных систем, а также их пользователей}
	\scnexplanation{Конвергенция (семантическая совместимость) всех разрабатываемых интеллектуальных компьютерных систем (в том числе прикладных), преобразующая набор индивидуальных (самостоятельных) интеллектуальных компьютерных систем различного назначения в коллектив активно взаимодействущих интеллектуальных компьютерных систем для совместного (коллективного) решения сложных (комплексных) задач и для перманентной поддержки семантической совместимости в ходе индивидуальной эволюции каждой интеллектуальной компьютерной системы.}
\scnaddlevel{-1}
\scnhaselement{конвергенция средств автоматизации проектирования различного вида компонентов интеллектуальных компьютерных систем}
\scnaddlevel{1}
	\scnidtf{Конвергенция (семантическая совместимость) средств автоматизации проектирования различного вида компонентов интеллектуальных компьютерных систем, результатом которой должен быть общий комплекс средств автоматизации проектирования всех компонентов интеллектуальных компьютерных систем}
	\scnidtf{Конвергенция между инструментальными средствами, обеспечивающими автоматизацию проектирования различных компонентов или различных классов интеллектуальных компьютерных систем, результатом (целевым продуктом) которой должен стать единый комплекс методологических и инструментальных средств, ориентированный на поддержку комплексного проектирования любых интеллектуальных компьютерных систем}
\scnaddlevel{-1}
\scnhaselement{конвергенция логико-семантических моделей интеллектуальных компьютерных систем}
\scnaddlevel{1}
	\scnnote{\textit{логико-семантические модели интеллектуальных компьютерных систем} являются результатом ("сухим"{} остатком) \textit{проектирования} этих систем и представляют собой формальное представления исходного (начального) состояния \textit{баз знаний} разрабатываемых \textit{интеллектуальных компьютерных систем}}
\scnaddlevel{-1}
\scnhaselement{конвергенция средств интерпретации логико-семантических моделей разрабатываемых интеллектуальных компьютерных систем}
\scnaddlevel{1}
\scnexplanation{Конвергенция (совместимость) средств реализации (производства) интеллектуальных компьютерных систем на основе спроектированных формальных моделей создаваемых интеллектуальных компьютерных систем (средств интерпретации спроектированных моделей интеллектуальных компьютерных систем). Такая интерпретация может осуществляться либо программным путем на современных компьютерах, либо путем создания принципиально новых компьютеров, специально ориентированных на интерпретацию формальных моделей интеллектуальных компьютерных систем, помещаемых в память указанных компьютеров}
\scnaddlevel{-1}
\scnhaselement{конвергенция между информационно-программным и 	аппаратным обеспечением интеллектуальных компьютерных систем}
\scnaddlevel{1}
	\scnidtf{Конвергенция между Software и Hardware интеллектуальных компьютерных систем}
\scnaddlevel{-1}
\scnhaselement{Конвергенция различных форм деятельности в области Искусственного интеллекта}
\scnaddlevel{1}
\scnexplanation{Конвергенция между
	\begin{scnitemize}
    \item научными исследованиями по созданию общей теории интеллектуальных компьютерных систем;
    \item разработкой средств автоматизации проектирования интеллектуальных компьютерных систем;
    \item разработкой средств интерпретации спроектированных формальных моделей интеллектуальных компьютерных систем;
    \item разработкой прикладных интеллектуальных компьютерных систем различного назначения;
    \item подготовкой и перманентным повышением квалификации кадров, способных эффективно участвовать во всех перечисленных направлениях деятельности.
    \end{scnitemize}
Глубокая конвергенция между всеми этими формами деятельности возможна только тогда, когда \uline{каждый} участник создания комплексной технологии искусственного интеллекта является участником \uline{каждой} из перечисленных форм деятельности.    
}
\scnidtf{Конвергенция между (1) научно-исследовательской деятельностью в области искусственного интеллекта; (2) инженерно-технологической деятельностью, которая направлена на разработку комплексной технологии проектирования интеллектуальных компьютерных систем и которая имеет высокий уровень наукоемкости; (3) инженерно-прикладной деятельностью, которая направлена на разработку прикладных интеллектуальных систем и которая также имеет высокий уровень наукоемкости, обусловленной необходимостью качественной формализации соответствующих предметных областей и, в частности, методов решения задач в этих областях; (4) образованием (образовательной деятельностью) в области искусственного интеллекта, повышение эффективности которого настоятельно требует раннего и поэтапного вовлечения студентов в реальные, а не учебные проекты - сначала в инженерно-прикладные, потом в инженерно- исследовательские проекты; (5) деятельностью, направленной на создание инфраструктуры, обеспечивающей поддержку открытого массового активного международного сотрудничества по консолидации усилий, направленных на решение современных проблем в области искусственного интеллекта; (6) бизнесом в области искусственного интеллекта, который не просто должен обеспечить финансовую поддержку перечисленных видов деятельности, но и обеспечить грамотный баланс между ними, грамотное сочетания тактических и стратегических целей}

\scnheader{Искусственный интеллект}
\scnrelfromset{методологические проблемы текущего состояния}{
\scnfileitem{Далеко не всеми учеными, работающими в области искусственного интеллекта принимается прагматичность практической направленности этой науки}
;\scnfileitem{Не всеми принимается необходимость конвергенции различных направлений искусственного интеллекта и необходимость их интеграции в целях построения общей теории интеллектуальных систем}
;\scnfileitem{Нет движения к построению общей компьютерной технологии интеллектуальных компьютерных систем}
;\scnfileitem{Нет движения к построению экосистем интеллектуальных компьютерных систем}
;\scnfileitem{Не всеми принимается необходимость конвергенции различных форм деятельности в области искусcтвенного интеллекта}}
\scnnote{Современная трактовка целей и задач Искусственного интеллекта как научно-технической дисциплины требует переосмысления, так как, к сожалению, носит несогласованный, а часто и значительно более узкий характер, чем этого требует текущее положение}
