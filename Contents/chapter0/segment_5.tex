\scnsegmentheader{Понятие Экосистемы OSTIS}

\scnstartsubstruct

\scnidtf{Использование \textit{Технологии OSTIS} для повышения качества и, в частности, уровня автоматизации всех \textit{областей человеческой деятельности}}
\scnidtf{Понятие \textbf{\textit{Экосистемы OSTIS}} как формы реализации \textit{smart-общества}, представляющего собой сеть взаимодействующих людей, интеллектуальных компьютерных систем, "умных"{} домов, "умных"{} предприятий, "умных"{} больниц, "умных"{} учебных заведений, "умных"{} городов, "умных"{} транспортных систем и т.п.}

\scnrelfromset{рассматриваемые вопросы}{
\scnfileitem{Какова архитектура \textit{Экосистемы OSTIS}};
\scnfileitem{Какова архитектура \textit{ostis-сообщества}, входящего в состав \textit{Экосистемы OSTIS}};
\scnfileitem{Как взаимодействуют между собой различные \textit{ostis-сообщества} в рамках \textit{Экосистемы OSTIS}};
\scnfileitem{Как интегрируется \textit{деятельность} различных \textit{ostis-сообществ} и результаты этой \textit{деятельности}};
\scnfileitem{Какова типология \textit{ostis-сообществ} и по каким признакам классификации можно эту типологию проводить};
\scnfileitem{Можно ли опыт автоматизации деятельности в области Искусственного интеллекта с помощью \textit{Технологии OSTIS} расширить на все многообразие областей и видов человеческой деятельности};
\scnfileitem{Как выглядит систематизация областей и видов человеческой деятельности};
\scnfileitem{Как осуществляется конвергенция и интеграция различных областей и видов человеческой деятельности};
\scnfileitem{Как взаимодействуют ostis-системы, осуществляющие автоматизацию различных областей видов человеческой деятельности};
\scnfileitem{Как может выглядеть \uline{комплексная} автоматизация всех областей и видов \textit{человеческой деятельности} с помощью \textit{Технологии OSTIS}}
}
\scnrelfromvector{план изложения}{
\scnfileitem{Что такое Экосистема OSTIS};
\scnfileitem{Структура Экосистемы OSTIS};
\scnfileitem{Что такое ostis-система, являющаяся агентом Экосистемы OSTIS};
\scnfileitem{Что такое ostis-сообщетво, являющееся агентом Экосистемы OSTIS};
\scnfileitem{Что такое Проект создания Экосистемы OSTIS};
\scnfileitem{Цель создания и основные свойства Экосистемы OSTIS};
\scnfileitem{Как структурируется человеческая деятельность};
\scnfileitem{Как выглядит рынок знаний, реализуемый в рамках Экосистемы OSTIS};
\scnfileitem{Чем определяется качество человеческой деятельности};
\scnfileitem{Что такое эффективная автоматизация человеческой деятельности};
\scnfileitem{Почему повышение эффективности человеческой деятельности невозможно без интеллектуальных компьютерных систем};
\scnfileitem{Какие достоинства имеет Экосистема OSTIS}
}

\bigskip
\scnfragmentcaption

\scnheader{Экосистема OSTIS}
\scntext{вопрос}{Какова структура Экосистемы OSTIS}
\scnexplanation{Популяция
\begin{scnitemize}
\item семантически совместимых
\item эволюционируемых
\item активно взаимодействующих  
\item способных координировать(согласовывать) свою деятельность с другими субъектами
\end{scnitemize}
интеллектуальных компьютерных систем (\textit{ostis-систем}). При этом указанная популяция \textit{ostis-систем} поддерживает децентрализованное управление собственной деятельностью, а также деятельностью людей(пользователей \textit{ostis-систем}) и человеко-машинных сообществ (\textit{ostis-сообществ}), обеспечивая тем самым автоматизацию системной интеграции любых новых субъектов (\textit{ostis-систем}, людей, \textit{ostis-сообществ}) в состав \textit{Экосистемы OSTIS}.
}

\scnrelfromvector{принципы, лежащие в основе}
{
\scnfileitem{\textit{Экосистема OSTIS} представляет собой сеть \textit{ostis-сообществ}};
\scnfileitem{Каждому \textit{ostis-сообществу} взаимно однозначно соответствует \textit{корпоративная ostis-система} этого \textit{ostis-сообщества}, которая:
\begin{scnitemize}
\item обеспечивает координацию деятельности членов соответствующего \textit{ostis-сообщества};
\item является "представителем"{} этого \textit{ostis-сообщества} в других \textit{ostis-сообществах}, членом которых указанное \textit{ostis-сообщество} является.
\end{scnitemize}
};
\scnfileitem{Каждое \textit{ostis-сообщество} может входить в состав любого другого \textit{ostis-сообщества} по своей инициативе. Формально это означает, что \textit{корпоративная ostis-система} первого \textit{ostis-сообщества} является членом другого \textit{ostis-сообщества}.};
\scnfileitem{Каждому специалисту, входящему в состав Экосистемы OSTIS ставится во взаимнооднозначное соответствие его \textit{персональный ostis-ассистент}, который трактуется как \textit{корпоративная ostis-система} вырожденного \textit{ostis-сообщества}, состоящего из одного человека.}
}

\scnheader{следует отличать*}
\scnhaselementset{корпоративная ostis-система*
\scnaddlevel{1}
\scnidtf{корпоративная ostis-система данного ostis-сообщества*}
\scnaddlevel{-1};
корпоративная ostis-система
\scnaddlevel{1}\\
\scnrelto{второй домен}{корпоративная ostis-система*}
\scnaddlevel{-1};
член ostis-сообщества*;
персональный ostis-ассистент*
\scnaddlevel{1}
\scnidtf{персональный ostis-ассистент данного специалиста*}
\scnsubset{корпоративная ostis-система*}
\scnaddlevel{-1};
персональный ostis-ассистент
\scnaddlevel{1}\\
\scnrelto{второй домен}{персональный ostis-ассистент*}
\scnsubset{корпоративная ostis-система}
\scnaddlevel{-1}
}

\scnheader{есть сходства*}
\scnhaselementset{Экосистема OSTIS;
ostis-сообщество\\
\scnaddlevel{1}
\scnhaselement{Экосистема OSTIS}
\scnaddlevel{-1}
}
\scnaddlevel{1}
\scnexplanation{Экосистема OSTIS является максимальным ostis-сообществом, включающим в себя все существующее ostis-сообщества}
\scnaddlevel{-1}

\scnheader{Экосистема OSTIS}
\scnidtf{Максимальное \textit{иерархическое ostis-сообщество}, обеспечивающее комплексную автоматизацию \uline{всех} видов \textit{человеческой деятельности}}
\scnidtf{Максимальное ostis-сообщество такое ostis-сообщество, для которого не существует другого ostis-сообщества, содержащее указанное выше ostis-сообщество в качестве своего члена}
\scnidtf{Симбиоз людей и \textit{компьютерных систем}(точнее, \textit{ostis-систем}) являющийся вариантом реализации \textit{smart-общества}}
\scniselement{иерархическое ostis-сообщество}
\scnaddlevel{1}
\scnidtf{такое \textit{ostis-сообщество}, по крайней мере одним из членов которого является некоторое другое \textit{ostis-сообщество}}
\scnsubset{ostis-сообщество}
\scnaddlevel{1}
\scnrelboth{следует отличать}{коллектив ostis-систем}
\scnaddlevel{-1}
\scnaddlevel{-1}{
\scnrelto{основной продукт}{Технология OSTIS}
\scnrelto{вариант реализации}{smart-общество}
\scnheader{агент Экосистемы OSTIS}
\scnidtf{субъект Экосистемы OSTIS}
\scnidtf{субъект, входящий в состав Экосистемы OSTIS}
\scnsuperset{когнитивный агент Экосистемы OSTIS}
\scnsubdividing{индивидуальная ostis-система Экосистемы OSTIS
\scnaddlevel{1}
\scnidtf{индивидуальная ostis-система, входящая в состав Экосистемы OSTIS}
\scnaddlevel{-1};
ostis-сообщество Экосистемы OSTIS\\
\scnaddlevel{1}
\scnsubdividing{
простое ostis-сообщество Экосистемы OSTIS;
иерархическое ostis-сообщество Экосистемы OSTIS
}
\scnaddlevel{-1};
пользователь Экосистемы OSTIS
}

\scnrelfrom{правила поведения}{Правила поведения агентов Экосистемы OSTIS}
\scnaddlevel{1}
\scneqtoset{
\scnfileitem{Согласовывать денотационную семантику всех используемых знаков(в первую очередь \uline{понятий})};
\scnfileitem{Согласовывать терминологию, соответствующую введенным знакам устранять противоречия и информационные дыры};
\scnfileitem{Постоянно бороться с синонимией и омонимией как на уровне sc-элементов(внутренних знаков), так и на уровне соответствующих им терминов и прочих внешних идентификаторов(внешних обозначений)};
\scnfileitem{Каждый агент Экосистемы OSTIS по своей инициативе может стать членом любого ostis-сообщества Экосистемы OSTIS после соответствующей регистрации}}
\scnaddlevel{-1}

\scnheader{Правила поведения агентов Экосистемы OSTIS}
\scnnote{Существенно подчеркнуть, что все правила функционирования(поведения) в рамках агентов Экосистемы OSTIS должны соблюдать не только ostis-системы, являющиеся агентами(субъектами) этой Экосистемы, но и люди, которые являются её агентами. И здесь возникают очень важные проблемы, обусловленные человеческим фактором. Дело в том, что убедить человека соблюдать правила, пусть даже те которые направлены на максимальную его самореализацию и в совершенствовании которых он может реально участвовать, очень непросто, поскольку любые  подобные правила многими воспринимаются как ограничение их творческой свободы. Другими словами корректное поведение ostis-системы в роли агентов Экосистемы OSTIS значительно проще, чем корректное поведение людей в качестве таких агентов. Поведение пользователей (естественных агентов) Экосистемы OSTIS необходимо внимательно мониторить и контролировать, постоянно способствуя повышению уровня их квалификации как агентов Экосистемы OSTIS, а также повышению уровня их мотивации, целенаправленности, самореализации.}

\scnheader{следует отличать*}
\scnhaselementset{
агент Экосистемы OSTIS;
член ostis-сообщества*
}

\scnheader{Экосистема OSTIS}
\scntext{архитектура}{В Экосистеме OSTIS можно выделить следующие уровни иерархии:
\begin{scnitemize}
\item индивидуальные компьютерные системы (\textit{индивидуальные ostis-системы} и \textit{люди}, являющиеся конечными пользователями ostis-систем);
\item иерархическая система ostis-сообществ, членами каждого из которых могут быть \textit{индивидуальные ostis-системы}, люди, а также другие \textit{ostis-сообщества};
\item \textit{Максимальное ostis-сообщество} \scnbigspace \textit{Экосистемы OSTIS}, не являющееся членом никакого другого \textit{ostis-сообщества}, входящего в состав \textit{Экосистемы OSTIS}.
\end{scnitemize}
Подчеркнем, что качество \textit{Экосистемы OSTIS} во многом определяется эффективностью взаимодействия каждой \textit{ostis-системы} (в том числе и каждого \textit{ostis-сообщества}), а также каждого \textit{человека} со своей \textit{внешней средой*}, а также качеством(чистотой), самой \textit{внешней среды*}.
Но внешняя среда каждого \textit{субъекта} каждой ostis-системы и каждого человека, входящего в \textit{Экосистему OSTIS} -- это не только \textit{материальная внешняя среда*}, но и \textit{информационная внешняя среда*}, представляющая собой виртуальный распределенный информационный ресурс, являющийся интеграцией(объединением) информации, хранящейся в текущий момент в памяти всех других(остальных) \textit{субъектов}, входящих в \textit{Экосистему OSTIS}. Основной целью \textit{Экосистемы OSTIS} является повышение качества(в том числе чистоты) \textit{информационной внешней среды*} для \uline{всех} \textit{субъектов}, входящих в \textit{Экосистему OSTIS}. Фактически речь идет об \textbf{\textit{Информационной экологии человеческого общества}}.
}

\scnheader{Информационная экология человеческого общества}
\scnnote{Говоря об \textit{Информационной экологии человеческого общества} необходимо заметить следующее. Современные подходы к развитию взаимодействия с информационной средой человеческого общества можно разбить на два направления:
\begin{scnitemize} 
\item на разработку средств приспособления к недостаткам текущего состояния этой среды
\item на устранение этих недостатков путем наведения порядка в устной информационной среде и её систематизации.
\end{scnitemize}
Технология OSTIS и реализация Экосистемы OSTIS целенаправленно и в известной степени радикально ориентирована на второе направление, памятуя искусственный (рукотворный) характер происхождения этой информационной среды.}
\scnendstruct

