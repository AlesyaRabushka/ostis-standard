\scnsegmentheader{\currentname}

\scnstartsubstruct

\scnheader{ostis-сообщество}
\scnidtf{Человеко-машинный симбиоз, представляющий собой коллектив, состоящий из людей и ostis-систем и обеспечивающий высокий уровень автоматизации определённого (соответствующего) вида человеческой деятельности.}
\scnnote{В состав каждого ostis-сообщества входит корпоративная ostis-система, которая в рамках этого ostis-сообщества выполняет: 
\begin{scnitemize}
\item роль координатора деятельности членов данного ostis-сообщества;
\item роль памяти ostis-сообщества, т.е. хранителя общих (обобществляемых, общедоступных) знаний для всех членов данного ostis-сообщества, которое несет ответственность за совершенствование этих знаний, а также для всех членов всех тех ostis-сообществ, в состав которых данное ostis-сообщество входит (указанные субъекты являются пользователями рассматриваемых общих знаний). Таким образом, корпоративная ostis-система некоторого ostis-сообщества является "официальным"{} представителем этого ostis-сообщества во всех ostis-сообществах, в состав которых входит, и, следовательно, является координатором деятельности даного ostis-сообщества (как единого целого) в рамках всех ostis-сообществ, в состав которых оно входит;
\end{scnitemize}}

\scnheader{есть сходства*}
\scnhaselementset{ostis-сообщество; решатель задач ostis-системы}
	\scnaddlevel{1}
	\scnrelfrom{пояснение}{\scnstartsetlocal
		
		\scnheaderlocal{ostis-сообщество}
		\scnsuperset{многоагентная система,в которой управление агентами осуществляется через общую для них память}
		\scnsuperset{многоагентная система, с децентрализованным управлением агентами}
		\scnsuperset{многоагентная система, в которой областью деятельности её агентов является как внешняя среда, так и память этой системы}

		\bigskip
		\scnheaderlocal{решатель задач ostis-системы}
		\scnsuperset{многоагентная система,в которой управление агентами осуществляется через общую для них память}
		\scnsuperset{многоагентная система, с децентрализованным управлением агентами}
		\scnsuperset{многоагентная система, в которой областью деятельности её агентов является как внешняя среда, так и память этой системы}
		\scnsuperset{агентно-ориентированная модель обработки информации в памяти}
		
		\scnendstruct
	}

\scnheader{многоагентная система с децентрализованным управлением агентами}
\scnrelfromlist{пример}{оркестр, играющий без дирижера или даже без композитора\\
	  \scnaddlevel{1}
	  \scntext{необходимое требование}{каждый участник оркестра должен иметь квалификацию дирижера или композитора}
	  \scnaddlevel{-1};
комплексная строительная бригада, работающая без прораба\\
	  \scnaddlevel{1}
	  \scntext{необходимое требование}{каждый участник строительной бригады должен иметь квалификацию прораба}
	  \scnaddlevel{-1};
научно-исследовательская лаборатория, работающая без заведующего и научного руководителя\\
	  \scnaddlevel{1}
	  \scntext{необходимое требование}{каждый участник научно-исследовательской лаборатории должен иметь квалификацию заведующего или научного руководителя}
	  \scnaddlevel{-1};
кафедра, работающая без заведующего и ученого секретаря\\
	  \scnaddlevel{1}
	  \scntext{необходимое требование}{каждый участник кафедры должен иметь квалификацию заведующего и ученого секретаря}
	  \scnaddlevel{-1}
}

\scnheader{наукоемкий проект}
\scnidtf{непредсказуемый проект}
\scnrelfromset{специфика реализации}{
	\scnfileitem{цель};
	\scnfileitem{мотивация общего результата}; 
	\scnfileitem{квалификация участников на уровне дирижера-композитора (прораба-архитектора)}; \scnfileitem{коммуникабельность};
	\scnfileitem{моральные принципы -- не как я красиво играю, а какая красивая музыка, которую мы вместе делаем}
}
\scnnote{Новый подход к наукоемкому project-менеджменту предполагает:
\begin{scnitemize}
	 \item каждый "строитель"{} должен иметь квалификацию прораба (оркестр без дирижера, нот и указаний). 
	 \item распределение сфер ответственности, а не задач
\end{scnitemize} 
	 Таким образом, технологии разработки, эксплуатации и совершенствования (реинжиниринга) нового поколения должны быть устроена на основе децентрализованного, не целенаправленного управления.
}
\scntext{необходимые факторы}{
\begin{scnenumerate}
\item высокая квалификация и человеческие качества (моральные) участников;
\item согласованное формирование целей и подцелей;
\item персонификация вкладов (самоконтроль);
\item открытость.
\end{scnenumerate}}

\bigskip

\scnendstruct \scninlinesourcecommentpar{Завершили рассмотрение понятия ostis-сообщества}