\scnsegmentheader{Начало раздела "\currentname"}

\scnstartsubstruct

\scnheader{Методологические проблемы современного состояния работ в области Искусственного интеллекта}

\scnrelfromvector{конкатенация сегментов}
{Структура деятельности в области Искусственного интеллекта;}
\scnauthorcomment{дополнить список}

\scnrelfromset{рассматриваемые вопросы}{
\scnfileitem{Каковы основные стратегические цели (сверхзадачи) научно-технической деятельности в области \textit{Искусственного интеллекта}?};
\scnfileitem{Какие проблемы являются на сегодняшний день актуальными для дальнейшего развития различных направлений \textit{Искусственного интеллекта} и для развития \textit{Искусственного интеллекта} в целом как общей (объединённой) \textit{научно-технической дисциплины}, а также для развития различных форм деятельности в этой области (научно-исследовательской деятельности создания технологий разработки интеллектуальных компьютерных систем, образовательной деятельности, бизнеса)?};
\scnfileitem{Какие проблемы являются на сегодняшний день актуальными для развития других \textit{научно-технических дисциплин} и являются ли эти проблемы аналогичными тем, которые актуальны для развития \textit{Искусственного интеллекта}?};
\scnfileitem{Какие можно предложить подходы к решению указанных выше проблем и как для этого можно использовать создаваемый сейчас новый технологический уклад в области \textit{Искусственного интеллекта} (следующий уровень технологий искусственного интеллекта)?};
\scnfileitem{Как будет выглядеть на основе следующего уровня \textit{технологий Искусственного интеллекта} комплексная автоматизация вех \textit{видов человеческой деятельности}, а также взаимодействие различных \textit{видов человеческой деятельности}, т.е. как будет выглядеть архитектура \textit{smart-общества}?};
\scnfileitem{Устраивает ли нас уровень семантической совместимости взаимопонимания между современными виртуальными компьютерными системами и что необходимо сделать для повышения этого уровня?};
\scnfileitem{Устраивает ли нас уровень семантической совместимости взаимопонимания между современными интеллектуальными компьютерными системами их пользователями и что необходимо сделать для повышения этого уровня?}}
\scntext{аннотация}{Предлагаемое вашему вниманию рассмотрение методологических проблем современного состояния работ в области \textit{Искусственного интеллекта} состоит из следующих частей:
\begin{scnitemize}
\item Анализ актуальных проблем, препятствующих дальнейшему развитию  \textit{Искусственного интеллекта} как \textit{научно-технической дисциплины}:
\begin{scnitemizeii}
\item Проблемы развития научных исследований в области \textit{Искусственного интеллекта} 
\item Проблемы разработки технологий проектирования и реализации \textit{интеллектуальных компьютерных систем};
\item Проблемы формирования рынка \textit{интеллектуальных компьютерных систем}; 
\item Образовательные проблемы в области \textit{Искусственного интеллекта};
\item Проблемы развития бизнеса в области \textit{Искусственного интеллекта}.
\end{scnitemizeii}
\item Анализ проблем автоматизации сложных видов деятельности:
\begin{scnitemizeii}
\item научно-исследовательской деятельности в рамках различных научных дисциплин;
\item создание \textit{технологий проектирования} и производства (реализации) сложных технических систем;
\item \textit{инженерной деятельности} по разработке сложных технических систем;
\item \textit{образовательной деятельности} по наукоёмким техническим специальностям
\end{scnitemizeii}
\item Формулировка принципов, лежащих в основе \textit{Технологии OSTIS}, предназначенной для решения указанных выше проблем;
\item Рассмотрение структуры \textit{Экосистемы OSTIS}, построенной по \textit{Технологии OSTIS} и обеспечивающей комплексную автоматизацию всех видов человеческой деятельности
\end{scnitemize}}

\scnrelfromset{используемые знаки общих понятий и иных сущностей}{деятельность\\
\scnaddlevel{1}
\scnidtf{область деятельности}
\scnsuperset{человеческая деятельность}
\scnaddlevel{-1}
;вид деятельности\\
\scnaddlevel{1}
\scnhaselement{проектирование}
\scnaddlevel{1}
\scnidtf{проектная деятельность}
\scnaddlevel{-1}
\scnhaselement{производство}
\scnaddlevel{1}
\scnidtf{производственная деятельность}
\scnaddlevel{-1}
\scnhaselement{наука}
\scnaddlevel{1}
\scnidtf{научная деятельность}
\scnaddlevel{-2}
;проект\\
\scnaddlevel{1}
\scnsuperset{открытый проект}
\scnaddlevel{-1}
;консорциум
;технология\\
\scnaddlevel{1}
\scnsuperset{информационная технология}
\scnaddlevel{1}
\scnsuperset{технология искусственного интеллекта}
\scnaddlevel{-2}
;кибернетическая система\\
\scnaddlevel{1}
\scnsuperset{интеллектуальная система}
\scnaddlevel{1}
\scnsuperset{интеллектуальная компьютерная система}
\scnaddlevel{1}
\scnidtf{искусственная интеллектуальная система}
\scnaddlevel{-3}
;конвергенция\scnsupergroupsign
\scnaddlevel{1}
\scnidtf{уровень конвергенции (близости)}
\scnsuperset{конвергенция кибернетических систем\scnsupergroupsign}
\scnaddlevel{-1}
;интеграция*\\
\scnaddlevel{1}
\scnsuperset{интеграция кибернетических систем*}
\scnsuperset{эклектичная интеграция*}
\scnsuperset{глубокая интеграция*}
\scnaddlevel{-1}
;интегрированная система\\
\scnaddlevel{1}
\scnsuperset{эклектичная система}
\scnsuperset{гибридная система}
\scnaddlevel{-1}
;экосистема интеллектуальных компьютерных систем
;рынок знаний\\
\scnaddlevel{1}
\scnidtf{рыночная организация порождения эволюции и применения знаний}
\scnaddlevel{-1}
;smart-общество\\
\scnaddlevel{1}
\scnidtf{общество,в основе которого лежит экосистема интеллектуальных компьютерных систем и рынок знаний}
\scnaddlevel{-1}
}
 
\scnrelfromset{ключевые знаки}
{Искусственный интеллект\\
\scnaddlevel{1}
\scniselement{научно-техническая дисциплина}
\scnaddlevel{1}
\scnsubset{научно-техническая деятельность} 
\scnaddlevel{-2};
интеллектуальная система\\
\scnaddlevel{1}
\scnsuperset{интеллектуальная компьютерная система}
\scnaddlevel{-1};
Общая теория интеллектуальных систем;
Базовая комплексная технология проектирования интеллектуальных компьютерных систем;
Технология производства спроектированных интеллектуальных компьютерных систем;
Специализированная инженерия в области Искусственного интеллекта;
Образовательная деятельность в области Искусственного интеллекта;
Бизнес-деятельность в области Искусственного интеллекта\bigskip;
\scnkeyword{Технология OSTIS};
\scnkeyword{ostis-система};
смысловое преставление информации;
агентно-ориентированная модель обработки информации в памяти; стандартизация ostis-систем;
\scnkeyword{SC-код};
абстрактная sc-машина;
конвергенция знаний в памяти;
ostis-систем;
конвергенция моделей решения задач в  ostis-системе;
интеграция знаний в памяти  ostis-системы;
интеграция моделей решения задач в  ostis-системе;
ostis-сообщество;
ostis-технология\\
\scnaddlevel{1}
\scnsuperset{ostis-технология проектирования}
\scnsuperset{ostis-технология производства}
\scnsuperset{технология эксплуатации ostis-систем}
\scnsuperset{технология реинжиниринга ostis-систем}
\scnaddlevel{-1};
\scnkeyword{Ядро Технологии OSTIS}\bigskip;
OSTIS-портал научных знаний в области Искусственного интеллекта;
Проект IMS.ostis;
\scnkeyword{Метасистема IMS.ostis};
Проект Программной реализации универсальной абстрактной sc-машины;
Проект разработки Универсального sc-компьютера;
Специализированная инженерия, осуществляемая на основе Технологии OSTIS;
Образовательная деятельность в области Искусственного интеллекта, осуществляемая на основе технологии OSTIS;
\scnkeyword{Консорциум OSTIS}\bigskip;
\scnkeyword{Экосистема OSTIS};
человеческая деятельность;
вид человеческой деятельности;
автоматизация человеческой деятельности;
качество человеческой деятельности;
субъект Экосистемы OSTIS;
Рынок знаний, реализованный в рамках Экосистемы OSTIS;
smart-общество}