\bigskip
\scnfragmentcaption

\scnheader{Объединенная человеческая деятельность}
\scnidtf{максимальная область человеческой деятельности}
\scnidtf{вся человеческая деятельность}
\scnidtf{человеческая деятельность в целом}
\scnidtf{объединение всевозможных областей человеческой деятельности}
\scniselement{человеческая деятельность} 
	\scnaddlevel{1}
	\scnidtf{область человеческой деятельности}
	\scnidtf{система целенаправленных действий некоторого количества(возможно одного) людей над некоторыми объектами с помощью некоторых инструментов}
	\scnsubset{деятельность}
		\scnaddlevel{1}
		\scnidtf{трудно выполнимое сложное действие}
		\scnidtf{область деятельности}
		\scnidtf{система целенаправленных действий некоторых (возможно одного) субъектов над некоторыми объектами с помощью некоторых инструментов}
		\scnaddlevel{-1}
	\scnaddlevel{-1}
	
\scnheader{следует отличать*}
\scnhaselementset{Объединенная человеческая деятельность\\
	\scnaddlevel{1}
	\scnidtf{человеческая деятельность в целом}
	\scnidtf{максимальная область человеческой деятельности}
	\scnaddlevel{-1}
;область человеческой деятельности\\
	\scnaddlevel{1}
	\scnidtf{фрагмент (часть, раздел) человеческой деятельности}
	\scnidtf{человеческая деятельность}
	\scnidtf{деятельность, осуществляемая либо одним человеком (индивидуальная человеческая деятельность), либо коллективом людей}
	\scnsubset{деятельность}
		\scnaddlevel{1}
		\scnsubset{действие}
		\scnaddlevel{-1}
	\scnsuperset{индивидуальная человеческая деятельность}
	\scnsuperset{коллективная человеческая деятельность}
	\scnidtf{множество всевозможных областей человеческой деятельности}
	\scnaddlevel{-1}
;вид человеческой деятельности\\
	\scnaddlevel{1}
	\scnidtf{класс однотипных областей человеческой деятельности, которому можно поставить в соответствие некоторую технологию}
	\scnsubset{вид деятельности}
		\scnaddlevel{1}
		\scnsubset{класс действий}
		\scnaddlevel{-1}
	\scnaddlevel{-1}
}

\scnheader{следует отличать*}
\scnhaselementset{Объединенная человеческая деятельность\\
	\scnaddlevel{1}
	\scnidtf{максимальный процесс человеческой деятельности, включающий в себя деятельность всех людей и всех сообществ}
	\scnaddlevel{-1}
;человеческая деятельность\\
	\scnaddlevel{1}
	\scnidtf{множество всевозможных целостных, целенаправленных фрагментов \textit{Объединенной человеческой деятельности}}
	\scnnote{На данном множестве заданы такие отношения, как \textit{часть*}, \textit{декомпозиция*}. Т.е. конкретный экземпляр (элемент) данного множества может быть \textit{частью*} (входить в состав) другой конкретной человеческой деятельности. Более того, целесообразно рассматривать достаточно сложную иерархию процессов человеческой деятельности.}
	\scnidtf{конкретный процесс человеческой деятельности}
	\scnidtf{бизнес-процесс}
	\scnidtf{деятельность, основными субъектами которой являются люди и различные сообщества людей}
	\scnsubset{деятельность}
		\scnaddlevel{1}
		\scnsubset{действие}
		\scnaddlevel{-1}
	\scnnote{Если для автоматизации человеческой деятельности используются интеллектуальные компьютерные системы, то эти системы также становятся достаточно самостоятельными полноценными субъектами этой деятельности, мнение которых обязательно принимается во внимание, но при этом интеллектуальные компьютерные системы не становятся основными субъектами человеческой деятельности.}
	\scnhaselement{объединенная человеческая деятельность}
		\scnaddlevel{1}	
		\scnidtf{максимальная человеческая деятельность, для которой не существует никакой другой конкретной человеческой деятельности, частью* которой указанная Максимальная человеческая деятельность является.}
		\scnaddlevel{-1}
	\scnnote{каждая конкретная человеческая деятельность (каждый бизнес-процесс) может быть:
	\begin{scnitemize}
		\item либо полностью автоматизирована – от человека требуется только корректно сформулировать соответствующую команду (цель инициируемого действия);
		\item либо автоматизирована, но требующая от человека управления функционированием соответствующего одного инструментального средства;
		\item либо состоящая из фрагментов (подпроцессов, частных бизнес-процессов), некоторые из которых автоматизированы, а некоторые нет;
		\item либо полностью неавтоматизирована (т.е. выполняется "вручную"{})
	\end{scnitemize}}
	\scnnote{Когда речь идет о спецификации конкретной человеческой деятельности (конкретного бизнес-процесса), важно провести четкую грань между теми действиями, которые выполняются автоматически (в том числе интеллектуальными компьютерными системами), и действиям, которые выполняются людьми "вручную"{} - это как минимум действия по "формулировке"{} команд, которые адресуются соответствующим инструментальным средствам (язык и, соответственно, интерфейс формулировки таких команд для разных инструментальных средств может сильно отличаться).
Отсутствие унификации языка взаимодействия (интерфейсы) между людьми и различными инструментальными средствами (автомобилями, станками, холодильниками, газовыми плитами, микроволновками, компьютерными системами различного назначения) существенно снижает комплексную эффективность автоматизации человеческой деятельности, т.к. вынуждает людей тратить много времени на усвоение не сути (смысла) автоматизации, а формы (синтаксиса) своей деятельности по организации использования различных средств автоматизации.}
	\scnaddlevel{-1}
;вид человеческой деятельности\\
	\scnaddlevel{1}
	\scnidtf{класс (множество однотипных) процессов человеческой деятельности}
	\scnidtf{класс бизнес-процессов}
	\scnidtf{множество всевозможных классов бизнес-процессов}
	\scnsubset{вид деятельности}
	\scnnote{Каждый конкретный вид человеческой деятельности (т. е. каждый элемент множества "\textit{вид человеческой деятельности}"{}) является \textit{подмножеством*} множества "человеческая деятельность"{}.
Каждому виду человеческой деятельности соответствует своя \textit{технология человеческой деятельности}, т.е. свой набор \textit{методов} и \textit{средств}, обеспечивающих выполнение каждой конкретной деятельности, принадлежащей этому виду.}
	\scnaddlevel{-1}
;область человеческой деятельности\\
	\scnaddlevel{1}
	\scnsubset{человеческая деятельность}
	\scnidtf{достаточно крупный фрагмент человеческой деятельности}
	\scnidtf{раздел человеческой деятельности}
	\scnaddlevel{-1}
}

\scnheader{Экосистема OSTIS}
\scnnote{Содержательную типологию \textit{ostis-систем}, входящих в состав \textit{Экосистемы OSTIS} следует проводить на основе глубокого анализа содержательной структуры человеческой деятельности, требующей взаимодействия человека с другими людьми и даже с организациями. Очевидно, что эффективность такого взаимодействия во многом определяется качеством организации информационного взаимодействия, уровнем взаимопонимания, уровнем квалификации участников, оперативностью получения качественной консультативной помощи по любому (!) вопросу.}

\scnheader{вид человеческой деятельности, продуктом которой является информационная модель некоторого объекта или класса объектов}
\scnidtf{вид человеческой деятельности, направленной на построение описания (спецификации) некоторого объекта исследования или класса таких объектов}
\scnsubset{вид человеческой деятельности}
\scnhaselement{научно-исследовательская деятельность}
	\scnaddlevel{1}	
	\scnhaselement{Научно-исследовательская деятельность в области Искусственного интеллекта}
		\scnaddlevel{1}
		\scnidtf{разработка Общей теории интеллектуальных систем}
		\scnaddlevel{-1}
	\scnaddlevel{-1}
\scnhaselement{разработка теории искусственных объектов заданного класса}
	\scnaddlevel{1}
	\scnhaselement{Разработка Общей теории интеллектуальных компьютерных систем}
		\scnaddlevel{1}
		\scnidtf{разработка стандарта интеллектуальных компьютерных систем}
		\scnaddlevel{-1}
	\scnaddlevel{-1}
\scnhaselement{разработка теории проектирования искусственных объектов заданного класса}
	\scnaddlevel{1}
	\scnidtf{разработка системы проектных действий для искусственных объектов (артефактов) заданного класса}
	\scnhaselement{разработка теории проектирования интеллектуальных компьютерных систем}
		\scnaddlevel{1}
		\scnidtf{разработка стандарта организации коллективных проектных действий для проектирования интеллектуальных компьютерных систем}
		\scnaddlevel{-1}
	\scnaddlevel{-1}
\scnhaselement{разработка теории производства спроектированных искусственных объектов заданного класса}
	\scnaddlevel{1}
	\scnidtf{разработка стандарта системы производственных действий, методов и инструментов, обеспечивающих производство спроектированных артефактов заданного класса}
	\scnhaselement{Разработка Теории производства спроектированных интеллектуальных компьютерных систем}
	\scnaddlevel{-1}
\scnhaselement{проектирование искусственного объекта заданного класса}
	\scnaddlevel{1}
	\scnidtf{проектная деятельность, направленная на построение такой информационной модели (спецификации) искусственно создаваемого объекта (артефакта) заданного класса, которой достаточно для производства этого объекта}
	\scnsuperset{проектирование конкретной интеллектуальной компьютерной системы}
		\scnaddlevel{1}
		\scnidtf{процесс проектирования некоторой компьютерной системы по заданной технологии проектирования}
		\scnaddlevel{-1}
	\scnaddlevel{-1}
\scnexplanation{Данный вид человеческой деятельности характерен следующими особенностями:
\begin{scnitemize}
	\item очень часто продукт этой деятельности (создаваемая информационная конструкция) имеет высокую степень сложности и, следовательно, указанная деятельность не может быть индивидуальной, а несет коллективный характер;
	\item основными факторами качественного коллективного построения сложной информационной конструкции являются семантическая совместимость (взаимопонимание) авторов, а также согласованность их действий;
	\item важнейшим направлением автоматизации коллективной деятельности, объект и продукт которой представляет собой сложную информационную конструкцию, является автоматизация редактирования коллективно создаваемого информационного объекта, а также автоматизация обеспечения семантической совместимости и согласованности продуктов индивидуальной деятельности всех соавторов;
	\item указанную автоматизацию легко реализовать с помощью корпоративной интеллектуальной компьютерной системы, объединяющей всех соавторов создаваемого информационного объекта и снабженной мощными средствами поддержки коллективного проектирования различных разделов базы знаний этой системы. Примерами таких систем являются интеллектуальные порталы различного вида знаний.
\end{scnitemize}}
\scnheader{вид человеческой деятельности, продуктом которой является информационная модель некоторого объекта или класса объектов}
\scnexplanation{Здесь речь идет о коллективной человеческой деятельности, которая принципиально не может быть полностью автоматизирована (исследовательская, проектная), то основной проблемой ее автоматизации являются
\begin{scnitemize}
	\item недостаточный уровень семантической совместимости и взаимопонимания между людьми и отсутствие сознания серьезности этой проблемы;
	\item недостаточный уровень договоренности и отсутствия понимания серьезности этой проблемы;
	\item отсутствие четкой методики согласования точек зрения и отсутствие понимая серьезности этой проблемы.
\end{scnitemize}
Интеллектуальные компьютерные системы могут и должны создать корпоративную среду для решения этих проблем.
По сути это не что иное, как поддержка коллективного проектирования соответствующих разделов баз знаний интеллектуальной компьютерной системы, реализуемая на \uline{семантическом уровне}, когда интеллектуальная компьютерная система становится самостоятельным полноправным участником деятельности, в обязанности которого входит:
\begin{scnitemize}
	\item анализ семантической совместимости точек уровня различных участников, 
	\begin{scnitemizeii}
		\item выявление противоречий и альтернатив 
	\end{scnitemizeii}
	\item фиксация авторства
	\item отмена современной формы представления интеллектуального продукта (статьи, книги, документы)
\end{scnitemize}
Недостаточно высокий уровень семантической согласованности используемых понятий приводит к огромному количеству искусственно создаваемых противоречий.
При этом следует отличать семантические противоречия (например, синонимию вводимых знаков) и, соответственно, методику их устранения или разногласия по поводу системы вводимых понятий от терминологических разногласий, методика устранения которых может и должна быть максимально простой и лишенной эмоциональной окраски. Излишнее увлечение терминологическими спорами существенно тормозит творческий процесс, но и несерьезное отношение к постоянному совершенствованию и соблюдению \uline{правил} построения терминов также недопустимо.}