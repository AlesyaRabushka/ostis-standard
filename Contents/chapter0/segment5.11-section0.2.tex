\bigskip
\scnfragmentcaption

\scnheader{Экосистема OSTIS}
\scntext{вопрос}{Какие достоинства имеет Экосистема OSTIS}
\scntext{достоинство}{Важнейшей особенностью Экосистемы OSTIS является то, что входящие в нее \textit{ostis-системы} благодаря высокому уровню их интеллекта и, в частности, высокому уровню их социализации, становятся самостоятельными, активными и полноправными субъектами, участвующими в реализации самых различных видов человеческой деятельности, что существенно повышает уровень её автоматизации.}
\scnrelfromset{Что такое интеллектуальная система}{
\uline{система}(!)свойств
;В.К. Финн
;требования, предъявленные к интеллектуальным компьютерным системам}
\scnrelfromset{достоинства}{
\scnfileitem{семантическая совместимость
\begin{scnitemize}
\item интеллектуальных компьютерных систем между собой
\item интеллектуальных компьютерных систем с их пользователями и разработчиками
\item семантическая совместимость = взаимопонимание
\end{scnitemize}};
\scnfileitem{Перманентная поддержка семантической совместимости};
\scnfileitem{Способность координировать свои действия (договороспособность, координация(!)) при коллективном решении задач автоматизации \textbf{системной интеграции} интеллектуальных компьютерных систем, "ручная"{} реализация системной интеграции -- главный тормоз комплексной автоматизации. Многоагентная система из интеллектуальных компьютерных систем + \textbf{людей}.
\begin{scnitemize}
\item Каждая ostis-система является, кроме всего прочего, способной обучать(повышать квалификацию) своих пользователей т.е. повышать эффективность своей эксплуатации\\
ostis-система\\
\scnsubset{интеллектуальная обучающая система}
\end{scnitemize}};
\scnfileitem{Экосистема интеллектуальных компьютерных систем\\ 
\scneq{комплексная автоматизация человеческой деятельности}
Достоинства Экосистемы(преимущества) и перспективы создания и развития Технологии OSTIS
\begin{scnitemize}
\item Технология OSTIS как основа эволюции человеческого общества => переход к smart-обществу, к более интеллектуальному обществу
\item Экосистема OSTIS как продукт Технологии, т.е продукт технологии -- не отдельные интеллектуальные компьютерные системы, а целая Экосистема
\end{scnitemize}
это вариант smart-общества\\
smart-предприятие, smart-город\\
Требования к технологии разработки интеллектуальных компьютерных систем
\begin{scnitemize}
\item smart-сообщество разработчиков интеллектуальных компьютерных систем и разработчиков самой технологии
\item консорциум!!
\item стандарты интеллектуальных компьютерных систем
\item стандарты процесса разработки различных интеллектуальных компьютерных систем 
\item стандарты процесса совершенствования самой технологии
\item ориентация на новые компьютеры
\end{scnitemize}
Экосистема OSTIS -> цель\\
\scneq{smart-общество = общество 5.0 как интеграция всевозможных специализированных smart-сообществ(...)}};%?
\scnfileitem{конвергенция в области Искусственного интеллекта и не только(!!)(это необходимо для Экосистемы интеллектуальных компьютерны систем)};
\scnfileitem{глубокая интеграция(совместимость)};
\scnfileitem{новое поколение компьютеров};
\scnfileitem{Консорциум OSTIS по разработке глобального комплекса семантически совместимых технологий, обеспечивающих комплексную автоматизацию всевозможных видов человеческой деятельности(для Экосистемы интеллектуальных компьютерных систем).Достоинства эволюции интеллектуальных компьютерных систем автоматизации проектирования интеллектуальных компьютерных систем распространяется на все технические дисциплины и соответственно сообщества.};
\scnfileitem{система Проектов OSTIS, реализуемых консорциумом OSTIS -- как прообраз project-management нового типа, ориентированного на реализацию \uline{наукоемких} проектов с децентрализованным управлением и с перманентным коллективным уточнением и детализацией целей};
\scnfileitem{\textbf{Рынок знаний и его реализация}
\begin{scnitemize}
\item Смысловые представления знаний и глобальный характер минимизирует субъективизм, предвзятость, более эффективно защищает авторские права
\item Тормозит научно-техническое развитие(прогресс) становиться труднее
\item Выигрывает тот, кто действительно способствует прогрессу, а не тормозит его
\end{scnitemize}
Нет центральных и периферийных публикаций -- есть общая база знаний, в которой нет семантической эквивалентности(и, следовательно, нет плагиата). 
%До....?
Монография OSTIS \textbf{+ IMS.ostis} как новый уровень автоматизации создания и эволюции научно-технического к?
	от статей и монографий к семантически совместимым базам и порталам научно-технических знаний!!
	достоинства эволюции портала научных знаний по Искусственному интеллекту и соответственно сообщества ученых распространяется на все научные дисциплины
}
}

\bigskip
\scnendstruct \scninlinesourcecommentpar{Завершили Сегмент "\textit{Понятие Экосистемы OSTIS}"}