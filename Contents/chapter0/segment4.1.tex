\scnsegmentheader{Использование Технологии OSTIS для повышения качества человеческой деятельности в области Искусственного интеллекта}
\scnstartsubstruct

\scnrelfromlist{рассматриваемые вопросы}{
\scnfileitem{Могут ли \uline{практические} результаты работ в области Искусственного интеллекта, существенно повысить эффективность развития Искусственного интеллекта как научно-технической дисциплины, включая \uline{все формы} деятельности в области Искусственного интеллекта};
\scnfileitem{Какова перспектива использования Технологии OSTIS для автоматизации других областей и видов человеческой деятельности}}

\bigskip
\scnfragmentcaption

\scnheader{Научно-исследовательская деятельность в области Искусственного интеллекта}
\scnrelfrom{субъект деятельности}{OSTIS-сообщество научно-исследовательской деятельности в области Искусственного интеллекта}
	\scnaddlevel{1}
	\scnnote{Имеются ввиду специалисты разных стран и разных направлений Искусственного интеллекта}
	\scnaddlevel{-1}
\scnrelfromset{направления деятельности}{Конвергенция и интеграция различных направлений Искусственного интеллекта
;Конвергенция Искусственного интеллекта как отдельной научно-технической дисциплины с другими смежными научными дисциплинами\\
	\scnaddlevel{1}
	\scnnote{Конвергенция с математикой, кибернетикой, информатикой, общей теорией систем, психологией, семиотикой, лингвистикой, гносеологией, логикой, методологией и др.}
	\scnaddlevel{-1}
;Разработка Общей теории интеллектуальных систем\\
	\scnaddlevel{1}
	\scnnote{Речь идет как о естественных, так и об искусственных интеллектуальных системах.}
	\scnaddlevel{-1}}
\scnrelfrom{средство автоматизации}{OSTIS-портал научных знаний в области Искусственного интеллекта}
\scnrelfrom{технология}{OSTIS-технология организации коллективной научно-теоретической деятельности }
 	\scnaddlevel{1}
 	\scnrelto{частная технология}{Технология \textbf{\textit{реинжиниринга}} ostis-систем}
 		\scnaddlevel{1}
 		\scnidtf{Технология коллективной \textbf{\textit{реинженерии}} баз знаний ostis-систем, обеспечивающая конвергенцию, интеграцию и согласование различных точек зрения и реализуемая абсолютно одинаковыми ostis-системами, которые встраиваются (интегрируются) в состав каждой ostis-системы}
 		\scnrelfrom{реализация}{Встраиваемая ostis-система поддержки реижиниринга ostis-систем}
 		\scnaddlevel{-1}
 	\scnaddlevel{-1}
\scnrelfrom{продукт}{Общая формальная теория интеллектуальных систем}
	\scnaddlevel{1}
	\scnnote{В рамках \textit{Технологии OSTIS} \scnbigspace \textit{Общая формальная теория интеллектуальных систем} представляется в виде \textit{базы знаний} \scnbigspace \textit{OSTIS-портала научных знаний в области Искусственного интеллекта}.}
	\scnaddlevel{-1}
	
\scnheader{Общая формальная теория интеллектуальных систем}
\scnnote{Зачем нужна \textit{Общая теория интеллектуальных систем}?\\
Очевидно, что без этой теории невозможно построить набор методов и средств, обеспечивающий комплексную поддержку разработки \textit{интеллектуальных компьютерных систем} различного назначения и с различным набором навыков (способностей, возможностей), которыми могут обладать \textit{интеллектуальные компьютерные системы}, но необязательно каждая из них.
При этом важно не просто построить \textit{Общую теорию интеллектуальных систем} и довести ее до строгого формального уровня, но также довести представление такой формальной теории до уровня базы знаний соответствующего \textit{портала научных знаний}.}

\scnheader{OSTIS-технология организации коллективной научно-теоретической деятельности}
\scnexplanation{Подчеркнем, что \textit{Технология OSTIS} создает достаточно удобные (конструктивные) условия для решения таких проблем, как
	\begin{scnitemize}
	\item cогласование систем понятий разных научных дисциплин (в частности, разных направлений \textit{Искуственного интеллекта}) и, как следствие, возможность реализации
достаточно качественной семантической совместимости;
	\item конвергенция разных научных дисциплин, важным механизмом которой является увеличение числа общих понятий, используемых этими дисциплинами (в частности, этого можно добиться путем введения таких понятий, каждое из которых является обобщением, например, двух понятий, одно из которых относится к одной дисциплине, а другое - к другой
	\item интеграция научных дисциплин.
	\end{scnitemize}
Удобство (конструктивность, формализованность) решения указанных проблем обусловлено тем, что каждая научная дисциплина представляется постоянно развивающейся базой знаний, которая в \textit{Технологии OSTIS} представляется в виде специальной
семантической сети (в виде текста \textit{SC-кода}), которой соответствуют достаточно простые синтаксические и семантические правила.
Важной проблемой организации научно-теоретической деятельности
является реализация эффективной процедуры согласования различных точек зрения и обеспечения их конвергенции и глубокой (бесшовной) интеграции. В рамках коллективного развития базы знаний портала научных знаний можно обеспечить:
	\begin{scnitemize}
	\item существенное сокращение времени, затрачиваемого на согласование используемых понятий;
	\item существенное повышение эффективности рецензирования самых различных предложений;
	\item существенное сокращение времени, затрачиваемого на публикацию научных результатов, так как меняется форма публикаций-публикации. Эти результаты оформляются в "смысловом"{} виде как фрагменты соответствующей базы знаний, что предполагает отсутствие дублирования
научных текстов, т.е. отсутствие возможности представления одного и того же результата во многих формах в разных статьях и монографиях;
	\item автоматизацию анализа качества новых знаний, предлагаемых в состав совершенствуемой базы знаний;
	\item автоматизацию мониторинга общего качества всей базы знаний.
	\end{scnitemize}
}

\scnheader{OSTIS-портал научных знаний в области Искусственного интеллекта}
\scntext{в перспективе}{В перспективе каждый \textit{ostis-портал научных знаний} может преобразоваться
в сеть семантически совместимых \textit{ostis-порталов научных знаний}, соответствующих различным направлениям заданной \textit{научной дисциплиной} (например, различным направлениям \textit{Искусственного интеллекта})}

\scnheader{Коллектив специалистов в области Искусственного интеллекта}
\scntext{в перспективе}{OSTIS-сообщество субъектов научно-исследовательской деятельности в области Искусственного интеллекта}
	\scnaddlevel{1}
	\scntext{уточнение}{указанными субъектами являются объединенные в сеть специалисты в области Искусственного интеллекта, неформальные группы таких специалистов организации или подразделения организации, работающих в области Искусственного интеллекта и ostis-порталы научных знаний в
области Искусственного интеллекта}
	\scnrelto{часть}{Экосистема OSTIS}
	\scnaddlevel{-1}

\bigskip
\scnfragmentcaption

\scnheader{Разработка Базовой Комплексной технологии проектирования интеллектуальных компьютерных систем}
\scnrelfrom{субъект деятельности}{Коллектив разработчиков Базовой Комплексной технологии проектирования интеллектуальных компьютерных систем}
	\scnaddlevel{1}
	\scnnote{Речь идет об открытом проекте разработки указанной технологии и, соответственно, об открытом международном коллективе разработчиков, формируемом на добровольной основе}
	\scnrelfromset{направления деятельности}{Разработка Общей теории интеллектуальных компьютерных систем\\
	\scnaddlevel{1}
	\scnnote{Речь идет об \uline{искусственных} (компьютерных) интеллектуальных системах и о разработке \uline{стандарта} таких технологических систем.}
	\scnaddlevel{-1}
;Разработка Теории проектирования интеллектуальных компьютерных систем\\
	\scnaddlevel{1}
	\scnnote{Имеются в виду интеллектуальные компьютерные системы, соответствующие стандарту, разработанному в виде общей теории таких систем, имеется в виду рассмотрение самого процесса проектирования таких систем, т.е. рассмотрение методов их проектирования и проектных библиотек.}
	\scnaddlevel{-1}
;Разработка комплекса средств автоматизации проектирования
интеллектуальных компьютерных систем\\
	\scnaddlevel{1}
	\scnnote{Данные средства автоматизации проектирования (средства решения проектных задач) при их реализации с помощью \textit{Технологии OSTIS} входят в состав решателя задач \textit{Метасистемы IMS.оstis}.}
	\scnaddlevel{-1}
;Конвергенция и интеграция различного вида знаний, хранимых в памяти проектируемых интеллектуальных компьютерных систем
;Конвергенция и интеграция различных моделей решения задач,
используемых проектируемыми интеллектуальными компьютерными
системами}
	\scnaddlevel{-1}

\scnheader{Разработка Базовой Комплексной технологии проектирования интеллектуальных компьютерных систем}
\scnrelfrom{предлагаемых подход}{\textbf{Проект IMS.ostis}}
	\scnaddlevel{1}
	\scnidtf{Разработка Базовой Комплексной оstis-технологии проектирования оstis-систем}
	\scnidtf{Разработка Базовой Комплексной технологии проектирования оstis-систем с помощью специально предназначенной для этого \textit{оstis-системы}, которая
названа нами \textit{Метасистемой IMS.ostis}}
	\scnidtf{Проект разработки Метасистемы IMS.ostis}
	\scnrelfromset{принципы, лежащие в основе}{
\scnfileitem{Речь идет о проектировании не просто интеллектуальных компьютерных систем, а ostis-систем в виде которых можно построить любую интеллектуальную компьютерную систему. Соблюдение этого принципа является важнейшей
целью эволюции Технологии ОSTIS}
;\scnfileitem{Система автоматизации проектирования ostis-систем реализуется также в виде ostis-системы -- Метасистемы IMS.ostis}
;\scnfileitem{Эволюция технологии проектирования ostis-систем сводится к эволюции (реинженирингу) базы знаний Метасистемы IMS.ostis.}}
	\scnnote{Если речь вести о \textit{Технологии ОSTIS}, то следует говорить не только о самой данной технологии, но и о проекте, направленном на создание и
перманентное совершенствование этой технологии, так как важнейшей особенностью и достоинством \textit{Технологии ОSTIS} являются высокие темпы ее эволюции. Указанное достоинство обеспечивается прежде всего тем, что Технология ОSTIS реализуется в виде ostis-системы (\textit{Метасистемы IMS.ostis}).}
	\scnrelfrom{средство автоматизации}{Метасистема IMS.ostis}
		\scnaddlevel{1}
		\scnidtf{ОSTIS-система автоматизации комплексного проектирования ostis-систем}
		\scnnote{При \textit{Разработке Базовой Комплексной технологии проектирования интеллектуальных компьютерных систем} (точнее ostis-систем) средством автоматизации этой деятельности является не вся \textit{Метасистема IMS.ostis}, а только ее часть -- входящая в состав \textit{Метасистемы  IMS.ostis} типовая \textit{Встраиваемая ostis-система поддержки реижиниринга ostis-систем}, которая поддерживает деятельность разработчиков базы знаний Метасистемы IMS.оstis. Это обусловлено тем, что вся деятельность по \textit{Разработке Базовой Комплексной технологии проектирования интеллектуальных компьютерных систем} (ostis-систем) сводится к разработке (инженирингу) и обновлению (совершенствованию, реинжинирингу) \textit{Базы знаний Метасистемы IMS.ostis}).}
		\scnaddlevel{-1}
	\scnrelfrom{технология}{Технология реинжиниринга ostis-систем}
		\scnaddlevel{1}
		\scnrelfrom{реализация}{Встраиваемая ostis-система поддержки реинжиниринга ostis-систем}
		\scnaddlevel{-1}
	\scnrelfrom{продукт}{Комплексная ostis-технология проектирования ostis-систем}
		\scnaddlevel{1}
		\scnrelfrom{реализация}{Метасистема IMS.оstis}
		\scnaddlevel{-1}
	\scnaddlevel{-1}