\scnfragmentcaption

\scnheader{ostis-система, являющаяся агентной Экосистемой OSTIS}
\scntext{Примечание}{Вокруг каждой (!) \textit{ostis-системы} формируется коллектив её разработчиков, несущих ответственность (1) за её качественную эксплуатацию и (2) за её перманентное совершенствование в ходе эксплуатации. При этом интеллект \textit{ostis-системы} должен быть использован для максимально возможной автоматизации указанной деятельности разработчиков (для мониторинга состояния эксплуатируемой \textit{ostis-системы}, для своевременной реакции на подозрительные ситуации или события), а также для координации деятельности разработчиков. Таким образом, каждая \textit{ostis-система} (в том числе каждая персональная \textit{ostis-система}) для своих разработчиков фактически является корпоративной системой. Следовательно, для каждой \textit{ostis-системы} следует отличать (1) координируемый (управляемый, обслуживаемый) ею коллектив конечных пользователей, управляемых \textit{ostis-сообществ} и \textit{ostis-систем} (2) и координируемый ею коллектив разработчиков, в состав которого входит также и "породившая"{} эту \textit{ostis-систему} ostis-система автоматизации проектирования и реализации \textit{ostis-систем} соответствующего класса (эта система должна рассматриваться как полноценный активный член коллектива разработчиков, как \textit{ostis-система}, несущая полную ответственность за все те \textit{ostis-системы}, которые она "породила"). Таким образом, в рамках \textit{Экосистемы OSTIS} можно выделить (1) иерархическую сеть \textit{ostis-систем}, обеспечивающую автоматизированную комплексную реализацию человеческой деятельности с помощью указанных \textit{ostis-систем} и (2) сеть \textit{ostis-систем}, предназначенную для поддержки их эффективной эксплуатации и их совершенствования (постоянного обновления), включая постоянное совершенствование \textit{баз знаний ostis-систем}, осуществляемое не столько специалистами в области \textit{Искусственного интеллекта}, сколько специалистами в соответствующих предметных областях. Ярким примером ostis-систем, совершенствуемых не только специалистами в областях Искусственного интеллекта, являются порталы научно-технических знаний. Подчеркнём, что ответственность за качество сети \textit{ostis-систем}, предназначенной для разработчиков, постоянно совершенствующих эти \textit{ostis-системы} несёт \textit{Консорциум OSTIS.}}

\scnheader{персональный ostis-ассистент}
\scnidtf{ostis-система, являющаяся персональным ассистентом для соответсвующей персоны, входящей в состав Экосистемы OSTIS}
\scnidtf{"официальный"{} представитель соответствующей персоны во всех \textit{ostis-сообществах}, членом которых эта персона является}
\scnidtf{\textit{ostis-система}, являющаяся посредником соответствующей персоны в его взаимодействии с членами всех коллективом (\textit{ostis-сообществ}), в состав которых эта персона входит}

\scnnote{Каждой персоне, входящей в состав \textit{Экосистемы OSTIS} взаимно однозначно соответствует его личный (персональный) ассистент в виде персонального \textit{ostis-ассистента}. Таким образом, количество персональных \textit{ostis-ассистентов}, входящих в состав \textit{Экосистемы OSTIS}, совпадает с числом персон, входящих в состав \textit{Экосистемы OSTIS}.}

\scnnote{Коллектив, состоящий из персоны и соответствующего ей \textit{персонального ostis-ассистента}, фактически является \textit{минимальным ostis-сообществом}, которое также можно назвать терминальным \textit{ostis-сообществом}. Следовательно, персонального ostis-ассистента можно условно(!) считать \textit{корпоративной ostis-системой} минимального \textit{ostis-сообщества}, которая, как и любая другая корпоративная \textit{ostis-система} (1) организует "внутреннюю"{} деятельность соответствующего \textit{ostis-сообщества} и (2) осуществляет "внешнее"{} взаимодействие этого \textit{ostis-сообщества} (как единого целого) с другими агентами (субъектами) в рамках \textit{ostis-сообществ}, находящихся на следующем (более высоком) уровне иерархии.}

\scnnote{Строго говоря, все \textit{ostis-сообщества}, кроме \textit{минимальных ostis-сообществ}, являются не коллективами, состоящими из \textit{персон и ostis-систем}, а коллективами, состоящими только из \textit{ostis-систем}, поскольку формально в неминимальное \textit{ostis-сообщество} входят не персоны, а соответствующие им, организующие их деятельность \textit{персональные ostis-ассистенты}.}

\scnidtf{корпоративная система минимального ostis-сообщества}
\scnrelboth{следует отличать}{корпоративная ostis-система}
	\scnaddlevel{1}
	\scnidtf{корпоративная (центральная) ostis-система либо неминимального ostis-сообщества, либо коллектива ostis-систем}
	\scnaddlevel{-1}
	
\scnrelboth{следует отличать}{ostis-система массового обслуживания индивидуальных пользователей}
	\scnaddlevel{1}
	\scnidtf{ostis-система, не осуществляющая координацию деятельность своих конечных пользователей]}
	\scnidtf{ostis-сообщества массового обслуживания групп пользователей}

\scnidtf{ostis-сообщества, поддерживающая взаимодействия своих конечных пользователей, не только в рамках самостоятельных групп пользователей}

\scnheader{корпоративная ostis-система}
\scnidtf{ostis-система осуществляющая поддержку организации деятельности некоего количества, а также поддержка эволюции (совершенствования) этой деятельности управления проектами}

\scntext{принципы, лежащие в основе}{В основе взаимодействия и взаимосвязи корпоративных ostis-систем, входящих в состав Экосистемы OSTIS лежат следующие принципы:
\begin{scnitemize} 
	\item глубокая конвергенция различных научных дисциплин, формальных моделей различных видов деятельности и все, трансдисциплинарность, что может быть сделано одинаково и должно быть сделано одинаково.
	\item обобщение различных видов деятельности, построение частной иерархии различных моделей.
\end{scnitemize}
}

\scnheader{ostis-система, являющаяся агентом Экосистемы OSTIS}

\scnsuperset{персональный ostis-ассистент}
	\scnaddlevel{1}
	\scnrelboth{следует отличать}{персональный ostis-ассистент*}
		\scnaddlevel{1}
		\scnidtf{быть персональным ostis-ассистентом данной персоны*}
	\scnaddlevel{-1}
\scnaddlevel{-1}

\scnsuperset{корпоративная ostis-система}
	\scnaddlevel{1}
	\scnrelboth{следует отличать}{корпоративная ostis-система*}
		\scnaddlevel{1}
		\scnidtf{быть корпоративной ostis-системой данного ostis-сообщества*}
	\scnaddlevel{-1}
\scnaddlevel{-1}
\scnsuperset{ostis-портал научно-технических знаний}
	\scnaddlevel{1}
	\scnidtf{ostis-портал знаний по некоторой научно-технической дисциплине}}
\scnaddlevel{-1}
\scnsuperset{ostis-система автоматизации проектирования}
\scnsuperset{ostis-система автоматизации производства}
	\scnaddlevel{1}
	\scnidtf{ostis-система управления производством}
\scnaddlevel{-1}
\scnsuperset{ostis-система автоматизации образовательной деятельности}
	\scnaddlevel{1}
	\scnsuperset{обучающаяся ostis-система}
	\scnsuperset{корпоративная ostis-система виртуальной кафедры}
		\scnaddlevel{1}
		\scnidtf{корпоративная ostis-система, обеспечивающая интеграцию деятельности кафедр одинакового профиля и, возможно, различных вузов}
	\scnaddlevel{-1}
\scnaddlevel{-1}
\scnsuperset{ostis-система автоматизации бизнес-деятельности}
\scnsuperset{ostis-система автоматизации управления}
	\scnaddlevel{1}
	\scnsuperset{ostis-система управления проектами соответствующего вида}
	\scnsuperset{ostis-система сенсо-моторной координации при выполнении определённого вида  сложных действий во внешней среде}
		\scnaddlevel{1}
		\scnsuperset{ostis-система управления самостоятельным перемещением робота по пересеченной местности}
	\scnaddlevel{-1}
\scnaddlevel{-1}

\scnheader{обучающая ostis-система}
\scntext{Примечание}{Поскольку качество эксплуатации каждой ostis-системы зависит не только от неё, но и от квалификации пользователя (семантическая совместимость, знания о возможностях системы), каждая ostis-система должна быть способна обучать пользователя знаниям и навыкам эффективного её использования.}