\scnheader{Примеры синтаксической трансформации sc.s-предложений с использованием Расширенного алфавита sc.s-коннекторов}
\scnstartsubstruct

\bigskip
\scnfilelong{\textbf{\textit{si}}~$\Rightarrow$~\textit{включение*}:~\textbf{\textit{sj}}}
\scnrelfrom{синтаксическая трансформация}{
	\scnfilelong{\textbf{\textit{si}}~$\supseteq$~\textbf{\textit{sj}}}}
\scnaddlevel{1}
\scnrelboth{семантическая эквивалентность}{\scnfilescg{figures/intro/scs/sc.s-connectors/examples/scs_transf_inclusion_const.png}}
\scnaddlevel{-1}

\bigskip
\scnfilelong{\textbf{\textit{si}}~$\textunderscore\Rightarrow$~\textit{включение*}::~\textbf{\textit{sj}}}
\scnrelfrom{синтаксическая трансформация}{
	\scnfilelong{\textbf{\textit{si}}~$\textunderscore\supseteq$~\textbf{\textit{sj}}}}
\scnaddlevel{1}
\scnrelboth{семантическая эквивалентность}{\scnfilescg{figures/intro/scs/sc.s-connectors/examples/scs_transf_inclusion_var.png}}
\scnaddlevel{-1}

\bigskip
\scnfilelong{\textbf{\textit{si}}~$\textunderscore\textunderscore\Rightarrow$~\textit{включение*}:::~\textbf{\textit{sj}}}
\scnrelfrom{синтаксическая трансформация}{
	\scnfilelong{\textbf{\textit{si}}~$\textunderscore\textunderscore\supseteq$~\textbf{\textit{sj}}}}
\scnaddlevel{1}
\scnrelboth{семантическая эквивалентность}{\scnfilescg{figures/intro/scs/sc.s-connectors/examples/scs_transf_inclusion_meta.png}}
\scnaddlevel{-1}

\bigskip
\scnfilelong{\textbf{\textit{si}}~$\Rightarrow$~\textit{строгое включение*}:~\textbf{\textit{sj}}}
\scnrelfrom{синтаксическая трансформация}{
	\scnfilelong{\textbf{\textit{si}}~$\supset$~\textbf{\textit{sj}}}}
\scnaddlevel{1}
\scnrelboth{семантическая эквивалентность}{\scnfilescg{figures/intro/scs/sc.s-connectors/examples/scs_transf_strict_inclusion_const.png}}
\scnaddlevel{-1}

\bigskip
\scnfilelong{\textbf{\textit{si}}~$\textunderscore\Rightarrow$~\textit{строгое включение*}::~\textbf{\textit{sj}}}
\scnrelfrom{синтаксическая трансформация}{
	\scnfilelong{\textbf{\textit{si}}~$\textunderscore\supset$~\textbf{\textit{sj}}}}
\scnaddlevel{1}
\scnrelboth{семантическая эквивалентность}{\scnfilescg{figures/intro/scs/sc.s-connectors/examples/scs_transf_strict_inclusion_var.png}}
\scnaddlevel{-1}

\bigskip
\scnfilelong{\textbf{\textit{si}}~$\textunderscore\textunderscore\Rightarrow$~\textit{строгое включение*}:::~\textbf{\textit{sj}}}
\scnrelfrom{синтаксическая трансформация}{
	\scnfilelong{\textbf{\textit{si}}~$\textunderscore\textunderscore\supset$~\textbf{\textit{sj}}}}
\scnaddlevel{1}
\scnrelboth{семантическая эквивалентность}{\scnfilescg{figures/intro/scs/sc.s-connectors/examples/scs_transf_strict_inclusion_meta.png}}
\scnaddlevel{-1}

\bigskip
\scnfilelong{\textbf{\textit{si}}~$\Rightarrow$~\textit{порядок величин*}:~\textbf{\textit{sj}}}
\scnrelfrom{синтаксическая трансформация}{
	\scnfilelong{\textbf{\textit{si}}~$\geq$~\textbf{\textit{sj}}}}
\scnaddlevel{1}
\scnrelboth{семантическая эквивалентность}{\scnfilescg{figures/intro/scs/sc.s-connectors/examples/scs_transf_value_order_const.png}}
\scnaddlevel{-1}

\bigskip
\scnfilelong{\textbf{\textit{si}}~$\textunderscore\Rightarrow$~\textit{порядок величин*}::~\textbf{\textit{sj}}}
\scnrelfrom{синтаксическая трансформация}{
	\scnfilelong{\textbf{\textit{si}}~$\textunderscore\geq$~\textbf{\textit{sj}}}}
\scnaddlevel{1}
\scnrelboth{семантическая эквивалентность}{\scnfilescg{figures/intro/scs/sc.s-connectors/examples/scs_transf_value_order_var.png}}
\scnaddlevel{-1}

\bigskip
\scnfilelong{\textbf{\textit{si}}~$\textunderscore\textunderscore\Rightarrow$~\textit{порядок величин*}:::~\textbf{\textit{sj}}}
\scnrelfrom{синтаксическая трансформация}{
	\scnfilelong{\textbf{\textit{si}}~$\textunderscore\textunderscore\geq$~\textbf{\textit{sj}}}}
\scnaddlevel{1}
\scnrelboth{семантическая эквивалентность}{\scnfilescg{figures/intro/scs/sc.s-connectors/examples/scs_transf_value_order_meta.png}}
\scnaddlevel{-1}

\bigskip
\scnfilelong{\textbf{\textit{si}}~$\Rightarrow$~\textit{строгий порядок величин*}:~\textbf{\textit{sj}}}
\scnrelfrom{синтаксическая трансформация}{
	\scnfilelong{\textbf{\textit{si}}~$>$~\textbf{\textit{sj}}}}
\scnaddlevel{1}
\scnrelboth{семантическая эквивалентность}{\scnfilescg{figures/intro/scs/sc.s-connectors/examples/scs_transf_value_strict_order_const.png}}
\scnaddlevel{-1}

\bigskip
\scnfilelong{\textbf{\textit{si}}~$\textunderscore\Rightarrow$~\textit{строгий порядок величин*}::~\textbf{\textit{sj}}}
\scnrelfrom{синтаксическая трансформация}{
	\scnfilelong{\textbf{\textit{si}}~$\textunderscore>$~\textbf{\textit{sj}}}}
\scnaddlevel{1}
\scnrelboth{семантическая эквивалентность}{\scnfilescg{figures/intro/scs/sc.s-connectors/examples/scs_transf_value_strict_order_var.png}}
\scnaddlevel{-1}

\bigskip
\scnfilelong{\textbf{\textit{si}}~$\textunderscore\textunderscore\Rightarrow$~\textit{строгий порядок величин*}:::~\textbf{\textit{sj}}}
\scnrelfrom{синтаксическая трансформация}{
	\scnfilelong{\textbf{\textit{si}}~$\textunderscore\textunderscore>$~\textbf{\textit{sj}}}}
\scnaddlevel{1}
\scnrelboth{семантическая эквивалентность}{\scnfilescg{figures/intro/scs/sc.s-connectors/examples/scs_transf_value_strict_order_meta.png}}
\scnaddlevel{-1}

\bigskip
\scnfilelong{\textbf{\textit{si}}~$\Rightarrow$~\textit{внешний идентификатор*}:~\textbf{\textit{sj}}}
\scnrelfrom{синтаксическая трансформация}{
	\scnfilelong{\textbf{\textit{si}}~$:=$~\textbf{\textit{sj}}}}
\scnaddlevel{1}
\scnrelboth{семантическая эквивалентность}{\scnfilescg{figures/intro/scs/sc.s-connectors/examples/scs_transf_external_idtf_const.png}}
\scnaddlevel{-1}

\bigskip
\scnfilelong{\textbf{\textit{si}}~$\textunderscore\Rightarrow$~\textit{внешний идентификатор*}::~\textbf{\textit{sj}}}
\scnrelfrom{синтаксическая трансформация}{
	\scnfilelong{\textbf{\textit{si}}~$\textunderscore:=$~\textbf{\textit{sj}}}}
\scnaddlevel{1}
\scnrelboth{семантическая эквивалентность}{\scnfilescg{figures/intro/scs/sc.s-connectors/examples/scs_transf_external_idtf_var.png}}
\scnaddlevel{-1}

\bigskip
\scnfilelong{\textbf{\textit{si}}~$\textunderscore\textunderscore\Rightarrow$~\textit{внешний идентификатор*}:::~\textbf{\textit{sj}}}
\scnrelfrom{синтаксическая трансформация}{
	\scnfilelong{\textbf{\textit{si}}~$\textunderscore\textunderscore:=$~\textbf{\textit{sj}}}}
\scnaddlevel{1}
\scnrelboth{семантическая эквивалентность}{\scnfilescg{figures/intro/scs/sc.s-connectors/examples/scs_transf_external_idtf_meta.png}}
\scnaddlevel{-1}

\bigskip
\scnfilelong{\textbf{\textit{si}}~$\Leftrightarrow$~\textit{синонимия*}:~\textbf{\textit{sj}}}
\scnrelfrom{синтаксическая трансформация}{
	\scnfilelong{\textbf{\textit{si}}~$=$~\textbf{\textit{sj}}}}
\scnaddlevel{1}
\scnrelboth{семантическая эквивалентность}{\scnfilescg{figures/intro/scs/sc.s-connectors/examples/scs_transf_synonymy_const.png}}
\scnaddlevel{-1}


\bigskip
\scnfilelong{\textbf{\textit{si}}~$\Rightarrow$~\textit{погружение*}:~\textbf{\textit{sj}}}
\scnrelfrom{синтаксическая трансформация}{
	\scnfilelong{\textbf{\textit{si}}~$\supset=$~\textbf{\textit{sj}}}}
\scnaddlevel{1}
\scnrelboth{семантическая эквивалентность}{\scnfilescg{figures/intro/scs/sc.s-connectors/examples/scs_transf_insertion_const.png}}
\scnaddlevel{-1}

\scnendstruct\\

\scnnote{Аналогичным образом может быть описана трансформация предложений, содержащих любые классы sc.s-коннекторов, за исключением тех классов sc.s-коннекторов, которые соответствуют классам sc-коннекторов, входящим в Ядро SC-кода.}

\scnnote{В общем случае \textit{sc-элементы}, инцидентные \textit{sc-коннекторам}, классы которых описаны в данном примере, могут быть как \textit{sc-константами}, так и \textit{sc-переменными} (в том числе \textit{sc-метапеременными}). При этом как \textit{переменному sc-коннектору} может соответствовать \textit{константный sc-узел}, так и \textit{константному sc-коннектору} может соответствовать \textit{переменный sc-узел} (например, если возникает необходимость переменному sc-узлу приписать \textit{внешний идентификатор*}). Последняя ситуация встречается не очень часто и возникает в случае, когда область определения соответствующего \textit{отношения} имеет непустое пересечение с классом \textit{sc-переменных}.}