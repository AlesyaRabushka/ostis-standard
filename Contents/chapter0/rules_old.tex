\scsection{Общие принципы оформления внутреннего и внешнего представления информационных конструкций в ostis-системах}
\label{intro_rules}

\begin{SCn}

\scnsectionheader{\currentname}

\scnstartsubstruct

\scnreltovector{конкатенация сегментов}{Первый сегмент Раздела \dq{}\currentname\dq{};Форма внутреннего представления знаний ostis-систем;Структуризация баз знаний ostis-систем;Формальная спецификация знаний ostis-системы;Представление предметных областей и онтологий в базах знаний ostis-систем;
Принципы структуризации и оформления внешнего представления знаний ostis-систем;Последний сегмент раздела \dq{}\currentname\dq{}}
\scntext{аннотация}{***}

\scnsegmentheader{Формы внутреннего представления знаний ostis систем}

\scnstartsubstruct

\scnrelto{сегмент раздела базы знаний}{\currentname}
\scnrelfrom{следующий сегмент базы знаний}{Структуризация баз знаний ostis-систем}

\scnheader{знание ostis-системы}
\scnidtf{семантически целостный фрагмент \textit{базы знаний} \scnbigspace \textit{ostis-системы}}
\scnidtf{\textit{sc-текст}, имеющий однозначную семантическую интерпретацию в рамках \textit{SC-пространства}}
\scnidtf{семантически целостная \textit{информационная конструкция}, представленная в \textit{SC-коде} и хранимой в \textit{памяти ostis-системы} в составе ее \textit{базы знаний}}
\scnidtf{\textit{знание}, хранимое в \textit{памяти ostis-системы}}
\scnidtf{\textit{знание}, хранимое в \textit{sc-памяти}}
\scnidtf{\textit{знание}, входящее в состав \textit{базы знаний ostis-системы}}
\scnidtf{внутреннее представление \textit{знания ostis-системы}}
\scnnote{Не каждый sc-текст и не каждый файл ostis-системы является знанием}
\scnsubdividing{sc-знание ostis-системы\\
\scnaddlevel{1}
    \scnsubdividing{sc-знание, представленное текстом Ядра SC-кода;sc-знание, представленное sc-текстом, не принадлежащим Ядру SC-кода\\
    \scnaddlevel{1}
        \scnidtf{\textit{знание ostis-системы}, в состав которого входят как \textit{sc-элементы} с содержимым так и \textit{sc-элементы} без содержимого}
    \scnaddlevel{-1}}
\scnaddlevel{-1}
;файл-знание ostis-системы\\
\scnaddlevel{1}
    \scnidtf{\textit{знание}, представленное \textit{файлом ostis-системы}}
\scnaddlevel{-1}}

\scnsuperset{sc-спецификация}
\scnaddlevel{1}
    \scnidtf{описание (спецификация) заданной \textit{сущности}, представленное в виде \textit{sc-текста}}
    \scnidtf{семантически целостная окрестность заданного \textit{sc-элемента} в рамках \textit{SC-пространства}}
    \scnsubset{sc-окрестность}
    \scnaddlevel{1}
	    \scnidtf{окрестность заданного \textit{sc-элемента} в \textit{SC-пространстве}}
		\scnidtf{семантическая окрестность заданного \textit{sc-элемента}}
	\scnaddlevel{-1}
\scnaddlevel{-1}
\scnsuperset{раздел базы знаний}
\scnaddlevel{1}
\scnidtf{\textit{раздел}\scnbigspace \textit{базы знаний} \scnbigspace \textit{ostis-системы}}
\scnaddlevel{-1}
\scnsuperset{сегмент базы знаний}
\scnaddlevel{1}
\scnidtf{структурно выделяемый фрагмент \textit{атомарного раздела базы знаний}, а также либо начала, либо завершения \textit{неатомарного раздела} \scnbigspace \textit{базы знаний} \scnbigspace \textit{ostis-системы}}
\scnaddlevel{-1}

\scnsuperset{предметная область}
\scnaddlevel{1}
\scnidtf{sc-модель предметной области}
\scnaddlevel{-1}
\scnsuperset{онтология}
\scnaddlevel{1}
\scnidtf{sc-модель онтологии}
\scnsuperset{логическая онтология}
    \scnaddlevel{1}
    \scnidtf{sc-модель формальной теории (не обязательно классической)}
    \scnaddlevel{-1}
\scnaddlevel{-1}
\scnsuperset{логическое высказывание}
\scnaddlevel{1}
\scnidtf{sc-представление логического высказывания}
\scnaddlevel{-1}

\bigskip
\scnstartset
\scnheader{знание ostis-системы}
\scnsubset{sc-текст}
\scnaddlevel{1}
   \scnsubset{sc-структура}
\scnaddlevel{-1}
\scnsubset{sc-структура}
\scnendstruct


\scnnote{Не каждая \textit{sc-структура} и не каждый \textit{sc-текст} является \textit{знанием}. В отличие от \textit{sc-текста} каждая \textit{sc-структура} является \uline{синтаксически} связным множеством \textit{sc-элементов} (связной графовой структурой, состоящей из \textit{sc-элементов}). В отличие от \textit{sc-текстов} и \textit{sc-структур} в знаниях важна \uline{семантическая} целостность (полнота) \textit{информационных конструкций}. Так, например, \textit{логическая формула} со свободными переменными не является \textit{знанием}. Но полный текст \textit{высказывания}, включающий в себя все компоненты всех логических связок (вплоть до \textit{атомарных логических формул}) \textit{знанием} является. Второй пример: \textit{sc-текст}, в состав которого входит \textit{sc-коннектор} или \textit{sc-связка}, но не входят все компоненты этого \textit{sc-коннектора} или \textit{sc-связки},\scnbigspace \textit{знанием} не является.}

\scnheader{знание ostis-системы}
\scnnote{семантическая целостность \textit{знания ostis-системы} означает, во-первых, то, что такое \textit{знание} представляет собой \textit{информационную конструкцию}, являющуюся высказыванием, то есть информационную конструкцию, имеющую истинностное значение, которое может быть подтверждено или опровергнуто, например, экспертом (рецензентом) \textit{базы знаний ostis-системы}. Во-вторых, \textit{знание ostis-системы} должно содержать достаточно полную и \uline{однозначную} спецификацию по возможности всех входящих в него неидентифицированных (неименованных) \textit{sc-элементов}. Это необходимо для того, чтобы внешнее представление знания ostis-системы можно было по возможности однозначно погрузить ("вставить") в \textit{базу знаний} \scnbigspace \textit{ostis-системы}.}

\scnheader{файл ostis-системы}
\scnidtf{файл, хранимый в памяти ostis-системы}
\scnsubdividing{файл ostis-системы, не являющийся знанием;файл-знание ostis-системы\\
\scnaddlevel{1}
	\scnidtf{знание, представленное \textit{файлом ostis-системы}}
	\scnidtf{\textit{файл ostis-системы}, являющийся знанием}
	\scnnote{Далеко не каждый \textit{файл ostis-системы} является знанием}
	\scnnote{Для знания, представленного \textit{файлом ostis-системы}, в \textit{базе знаний} этой \textit{ostis-системы} может присутствовать \textit{семантически эквивалентное*} этому файлу знание, представленное \textit{текстом Ядра SC-кода}. Это является вариантом описания семантической интерпретации некоторых фрагментов \textit{базы знаний}. Кроме того, \textit{знание}, представленное \textit{файлом ostis-системы}, может быть предварительным этапом формализации этого \textit{знания}, предполагающим последующую трансляцию этого знания на язык \textit{Ядра SC-кода}. Использование таких файлов является важнейшим механизмом коллективной разработки \textit{баз знаний} \scnbigspace \textit{ostis-систем}. Следует также заметить, что некоторые \textit{знания}, представленные \textit{файлами ostis-систем}, не требует трансляции на язык \textit{Ядра SC-кода}, а носят вспомогательный характер, позволяющий разработчикам и конечным пользователям \textit{ostis-систем} ускорить процесс установления семантического контакта (взаимопонимания) с \textit{ostis-системами}.}
\scnaddlevel{-1}}

\scnheader{файл}
\scnsubdividing{строковый файл\\
\scnaddlevel{1}
    \scnidtf{файл, представляющий собой строку символов}
    \scnsuperset{sc.s-файл}
    \scnsuperset{ея-файл}
\scnaddlevel{-1}
;матричный файл\\
\scnaddlevel{1}
	\scnidtf{файл, представляющий собой двухмерную матрицу символов}
	\scnsuperset{sc.n-файл}
\scnaddlevel{-1}
;статическое изображение в 2D\\
\scnaddlevel{1}
	\scnsuperset{sc.g-файл в 2D}
\scnaddlevel{-1}
;статическое изображение в 3D\\
\scnaddlevel{1}
	\scnsuperset{sc.g-файл в 3D}
\scnaddlevel{-1}
;динамическое изображение в 2D\\
\scnaddlevel{1}
	\scnsuperset{динамический sc.g-файл в 2D}
\scnaddlevel{-1}
;динамическое изображение в 3D\\
\scnaddlevel{1}
	\scnsuperset{динамический sc.g-файл в 3D}
\scnaddlevel{-1}
}

\scnheader{файл}
\scnidtf{"электронная"\ форма представления информационной конструкции}
\scnsubdividing{текстовый файл;графический файл;файл-изображение;аудио-файл;видео-файл}
\scnsuperset{файл ostis-системы}
\scnaddlevel{1}
\scnsubdividing{файл-экземпляр;файл-образец
\scnaddlevel{1}
    \scnidtf{файл ostis-системы, обозначающий класс файлов-экземпляров, синтаксически эквивалентных заданному образцу}
\scnaddlevel{-1}}
	\scnsuperset{sc.g-файл ostis-системы}
	\scnsuperset{sc.s-файл ostis-системы}
	\scnsuperset{sc.n-файл ostis-системы}
	\scnsuperset{ея-файл ostis-системы}
\scnaddlevel{-1}

\scnheader{файл ostis-системы}
\scnidtf{файл, хранимый в памяти ostis-системы и который не обязательно в текущий момент должен быть сформированным (построенным)}
\scnsubdividing{файл-экземпляр ostis-системы;файл-образец ostis-системы\\
\scnaddlevel{1}
\scnidtf{класс синтаксически эквивалентных файлов-экземпляров ostis-системы}
\scnidtf{множество всевозможных копий заданного файла}
\scnaddlevel{-1}}

\scnheader{следует отличать*}
\scnhaselementset{ея-файл ostis.системы;ея-текст}

\scnheader{ея-файл ostis-системы}
\scnidtf{естественно-языковой файл ostis-системы}
\scnsubset{текстовый файл ostis-системы}
\scnidtf{файл ostis-системы, содержимым которого является текст одного из естественных языков (Русского, Английского, Немецкого, Французского и т.д.)}

\scnheader{ея-файл ostis-систем}
\scnrelfrom{разделители}{\scnset{...}}
\scnrelfrom{ограничители}{\scnset{...}}
\filemodetrue
\scnrelfromlist{оформление}{При оформлении текстов в естественно-языковых файлах (ея-файлах) ostis-систем используются обычные разделители (точки в аббревиатурах и между предложениями\char59 круглые скобки, запятые, пробелы\char59 символ, используемый в ея-файлах ostis-систем как разделитель $\square$), а также целый ряд ограничителей, позволяющих выделять некоторые фрагменты ея-текстов:
\begin{scnitemize}
\item подчеркивание выделяет логически важные фрагменты в предложениях\char59
\item цитатные кавычки являются ограничителем кратких цитат\char59
\item угловые двойные кавычки ограничивают длинные цитаты\char59
\item прямые кавычки ограничивают иносказательные термины, метафоры\char59
\item ограничители "косая черта со звездочкой"\ ограничивают комментарии, которые непосредственно в хранимым ея-файл не входят, но могут входить в исходные и отображаемые ея-тексты\char59
\item жирным курсивом увеличенного размера и с увеличенным расстоянием между символами выделяются заголовки разделов базы знаний\char59
\item жирным курсивом стандартного размера с увеличенным расстоянием между символами выделяются заголовки сегментов атомарных разделов, а также начал и завершений неатомарных разделов\char59
\item жирным курсивом стандартного размера выделяются идентификаторы элементов базы знаний, являющиеся ключевыми для заданного контекста\char59
\item жирным курсивом стандартного размера выделяются также условные внешние идентификаторы (обозначения) условно вводимых sc-элементов, например, условные обозначения sc-элементов произвольного структурного типа ($\bm{e_i}$, $\bm{e_j}$, ...), условные обозначения sc-узлов ($\bm{v_i}$, $\bm{v_j}$, ...), условные обозначения sc-коннекторов ($\bm{c_i}$, $\bm{c_j}$, ...), sc-дуг, sc-связок, не являющихся sc-коннекторами, sc-структур и так далее\char59
\item нежирным курсивом стандартного размера выделяются идентификаторы элементов базы знаний, не являющиеся ключевыми для данного ея-текста.
\end{scnitemize}
Если при этом в ея-тексте идентификаторы элементов базы знаний (sc-идентификаторы, чаще всего -- простые sc-идентификаторы), выделенные курсивом одинаковой жирности, следуют друг за другом, то в каждом из этих идентификаторов пробелы заменяются на знаки подчеркивания, а пробелы между разными идентификаторами сохраняются, и для наглядности число таких пробелов может быть увеличено. Из этого следует, что и в любом идентификаторе элемента базы знаний, который выделен курсивом, пробелы должны быть заменены на знаки подчеркивания. Иначе пробелы будут считаться разделителями разных идентификаторов.;
Текст ея-файла ostis-системы может иметь \uline{любые} вставки не являющиеся естественно-языковыми текстами, в том числе, и фрагменты, являющиеся формальными внешними текстами представления знаний для ostis-систем (текстами SCg-кода, SCs-кода, SCn-кода). При этом указанные формальные фрагменты (вставки) могут быть как транслируемыми на внутренний язык ostis-системы (SC-код) и погружаемыми в состав ее базы знаний (т.е. фактически являться sc-текстами), так и нетранслируемыми формальными фрагментами, которые входят в состав базы знаний ostis-системы в виде содержимого соответствующих файлов. Все указанные выше "вставки" в ея-файл ostis-системы оформляются как ссылки на соответствующие sc-тексты или файлы ostis-системы. Каждая такая ссылка представляет собой sc.s-идентификатор соответствующего sc-текста или файла ostis-системы и выделяется в ея-файле жирным курсивом стандартного размера со стандартным расстоянием между символами.
Таким образом, если в естественно-языковой файл ostis-системы, необходимо "вставить" информационную конструкцию иного рода (sc.g-текст, рисунок, таблицу, изображение), то (1) указанная конструкция оформляется как отдельный файл (2) которому приписывается имя (название), построенное по установленным правилам, и (3) на который в указанном естественно-языковом файле делается ссылка.
В естественно-языковых файлах ostis-систем можно делать ссылки не только на другие файлы ostis-системы, но и на \uline{именуемые} (!) фрагменты базы знаний, которые во внешнем представлении базы знаний оформляются в виде именуемых (идентифицируемых) sc.n-контуров.
Файлы и sc-тексты, на которые делаются ссылки из ея-файла, во внешнем представлении (при визуализации) базы знаний размещаются после указанного ея-файла в порядке первого их упоминания в этом ея-файле, если, конечно, на эти файлы или sc-тексты не было ссылок из ранее представленных ея-файлов.;
В ея-текстах все основные идентификаторы описываемых в базе знаний сущностей должны быть выделены жирным или нежирным курсивом.;
В ея-текстах используются только основные идентификаторы (термины). Используемые синонимы явно указываются как неосновные идентификаторы.;
Если в ея-тексте необходимо выделить (жирным или нежирным курсивом внешний идентификатор sc-элемента, состоящий из нескольких слов, то пробелы в таком словосочетании заменяются знаками подчеркивания.;
Основная часть (содержимого текста) ея-файла ostis-системы оформляется стандартным печатным шрифтом.
}
\filemodefalse

\scnsourcecomment{Завершили перечень правил оформления содержимого ея-файлов ostis-систем}

\scnheader{ея-файл ostis-системы}
\scnnote{Выделенные курсивом в ея-файле ostis-системы sc.s-идентификаторы sc-элементов могут являться \uline{ар\-гу\-мен\-та\-ми} различного вида \uline{запросов} к базе знаний ostis-системы и, в первую очередь запросов типа "Что это такое", предполагающих выделение из \uline{текущего} состояния базы знаний ostis-системы семантической окрестности (спецификации) указываемого sc-элемента, содержащей основную информацию о сущности, обозначаемой этим sc-элементом.}

\scnendstruct

\scnsegmentheader{Структуризация баз знаний ostis-систем}
\scnstartsubstruct

\scnrelto{сегмент раздела базы знаний}{\currentname}
\scnrelfrom{следующий сегмент базы знаний}{Формальная спецификация знаний ostis-систем}
\scnidtf{Структурная типология знаний ostis-систем}
\scntext{введение}{База знаний ostis-системы имеет достаточно развитую иерархическую структуру. База знаний делится на разделы. Разделы бывают атомарными и неатомарными.
Неатомарный раздел состоит из подразделов, а также включает в себя свое начало и завершение.
Атомарные разделы не имеют подразделов, но могут декомпозироваться на сегменты. Аналогичным образом начало и завершение неатомарного раздела также может декомпозироваться на сегменты. Разделы базы знаний ostis-системы могут иметь самое различное назначение. Так, например, база знаний Метасистемы IMS.ostis включает в себя:
\begin{scnitemize}
    \item раздел, содержащий текущее состояние постоянно пополняемой и совершенствуемой \textit{Документации Технологии OSTIS};
    \item раздел, посвященный описанию конечных пользователей и разработчиков Метасистемы IMS.ostis;
    \item раздел, посвященный описанию истории эксплуатации Метасистемы IMS.ostis;
    \item раздел, посвященный описанию истории эволюции Метасистемы IMS.ostis (в т.ч. истории эволюции ее базы знаний);
    \item раздел, посвященный описанию интеллектуальных компьютерных систем, разработанных (порожденных) с помощью Метасистемы IMS.ostis.
\end{scnitemize}}

\scnheader{структурно выделяемое sc-знание ostis-системы}
\scnsubset{sc-знание ostis-системы}
\scnidtf{фрагмент базы знаний ostis-системы, являющийся структурным компонентом базы знаний}
\scnsubdividing{раздел базы знаний\\
    \scnaddlevel{1}
    \scnsubdividing{атомарный раздел базы знаний\\
        \scnaddlevel{1}
        \scnsubdividing{структурированный атомарный раздел базы знаний;неструктурированный атомарный раздел базы знаний}
        \scnaddlevel{-1}
    ;неатомарный раздел базы знаний}
    \scnsuperset{первый раздел базы знаний}
    \scnsuperset{последний раздел базы знаний \scnsourcecomment{первый или последний в рамках подраздела}}
    \scnaddlevel{-1}
;начало неатомарного раздела базы знаний\\
    \scnaddlevel{1}
    \scnsubdividing{структурированное начало неатомарного раздела базы знаний;неструктурированное начало неатомарного раздела базы знаний}
    \scnaddlevel{-1}
;завершение неатомарного раздела базы знаний\\
    \scnaddlevel{1}
    \scnsubdividing{структурированное завершение неатомарного раздела базы знаний;неструктурированное завершение неатомарного раздела базы знаний}
    \scnaddlevel{-1}
;сегмент базы знаний\\
    \scnaddlevel{1}
    \scnsuperset{первый сегмент базы знаний}
    \scnsuperset{последний сегмент базы знаний \scnsourcecomment{первый или последний в рамках фрагмента базы знаний, в состав которого сегмент входит}}
    \scnaddlevel{-1}
;sc-знание ostis-системы нижнего структурного уровня} \scnaddlevel{1}
\scneq{Структурная классификация знаний ostis-системы}
\scnaddlevel{-1}
\scnsourcecomment{Завершили классификацию структурно выделяемых фрагментов баз знаний.}

\scnheader{структурно выделяемое sc-знание ostis-системы}
\scnsubset{sc-структура}
\scnnote{Каждое структурно выделяемое sc-знание ostis-системы -- это всегда sc-структура, которая может декомпозироваться на более "мелкие" структурно выделяемые sc-знания ostis-системы (в случае, если эти sc-знания являются структурированными), а могут и не декомпозироваться (в случае, если эти sc-знания являются неструктурированными).}

\scnheader{структурированное структурно выделяемое sc-знание ostis-системы}
\scnidtf{структурно выделяемый фрагмент базы знаний, в состав которого входят другие структурно выделяемые фрагменты (подразделы или сегменты)}
\scnsubdividing{структурированный атомарный раздел базы знаний;неатомарный раздел базы знаний \scnsourcecomment{он всегда структурирован, т.к. состоиз из разделов более низкого уровня (подразделов)};структурированное начало неатомарного раздела;структурированное завершение неатомарного раздела}

\scnheader{неструктурированное структурно выделяемое sc-знание ostis-системы}
\scnidtf{структурно выделяемый фрагмент базы знаний, не содержащий ни разделов, ни сегментов}
\scnsubdividing{неструктурированный атомарный раздел базы знаний;неструктурированное начало неатомарного раздела базы знаний;неструктурированное завершение неатомарного раздела базы знаний;сегмент базы знаний \scnsourcecomment{он всегда неструктурирован}}
\scnnote{Простейшей формой сегмента базы знаний, неструктурированного атомарного раздела базы знаний, неструктурированного начала или завершения неатомарного раздела базы знаний является просто последовательность файлов ostis-системы. Некоторые из этих файлов могут быть идентифицированными (именованными), если на них ссылаются другие файлы или sc.n-предложения, а некоторые из них могут быть связаны с другими файлами различными отношениями (в частности, один файл может быть пояснением другого). Кроме того, некоторые из этих файлов могут быть формально специфицированы (например, указаны соответствующие им ключевые sc-элементы).\\
В самом простом случае неструктурированное структурно выделяемое знание ostis-системы может быть sc-структурой, состоящей из \uline{одного} (!) sc-узла, обозначающего файл ostis-системы (чаще всего, ея-файл ostis-системы). Т.е. сам файл ostis-системы может быть знанием ostis-системы (файл ostis-системы $\supset$ знание ostis-системы), но не может быть структурно выделяемым знанием ostis-системы. При этом sc-узел, обозначающий файл ostis-системы, являющийся знанием, может быть единственным элементом простейшего вида структурно выделяемого знания ostis-системы.}

\scnheader{раздел базы знаний}
\scnexplanation{Для каждого неатомарного раздела базы знаний существует одно и только одно начало этого раздела, а также одно и только одно его завершение. В общем случае неатомарный раздел базы знаний может иметь неограниченное число подразделов. Каждый атомарный раздел базы знаний, а также каждое начало и завершение неатомарного раздела базы знаний не может иметь подразделов, но может иметь неограниченное число сегментов базы знаний. В этом случае они называются структурированными структурно выделяемыми знаниями ostis-системы. Сегменты базы знаний не могут состоять из других сегментов (подсегментов). В этом смысле сегменты базы знаний имеют атомарный характер.}
\scnidtf{раздел базы знаний ostis-системы}
\scnidtf{модуль (блок) базы знаний}
\scnsubdividing{неатомарный раздел базы знаний\\
    \scnaddlevel{1}
    \scnidtf{раздел, декомпозируемый на подразделы}
    \scnaddlevel{-1}
;атомарный раздел базы знаний}
\scnexplanation{В общем случае многим разделам ставятся в соответствие такие тексты, как \uline{введение}, \uline{заключение}, \uline{аннотация}, \uline{оглавление}, упражнения.\\
Некоторые из этих текстов могут иметь статус разделов (как, например, в Документации Технологии OSTIS).}
\scnsubdividing{начало неатомарного раздела базы знаний\\
    \scnaddlevel{1}
    \scnsubset{введение в неатомарный раздел базы знаний.}
    \scnaddlevel{-1}
;завершение неатомарного раздела базы знаний\\
    \scnaddlevel{1}
    \scnsubset{заключение неатомарного раздела базы знаний.}
    \scnaddlevel{-1}
;первый сегмент базы знаний\\
    \scnaddlevel{1}
    \scnidtf{первый сегмент атомарного раздела базы знаний, либо начала неатомарного раздела базы знаний, либо завершения неатомарного раздела базы знаний}
    \scnaddhind{-1}
    \scnsubset{введение в атомарный раздел базы знаний, либо спецификация неатомарного раздела базы знаний и его начала, либо спецификация завершения неатомарного раздела базы знаний.}
    \scnaddlevel{-1}
;последний сегмент базы знаний\\
    \scnaddlevel{1}
    \scnsubset{заключение атомарного раздела базы знаний, либо заключение введения (начала) неатомарного раздела базы знаний, либо заключение заключения (завершения) неатомарного раздела базы знаний}
    \scnidtf{последний сегмент атомарного раздела базы знаний, либо начала неатомарного раздела базы знаний, либо завершения неатомарного раздела базы знаний}
    \scnaddlevel{-1}}
\scnnote{Некоторые разделы базы знаний могут несколько введений и несколько заключений. При этом начало неатомарного раздела базы знаний трактуется как одно из введений в этот раздел, а первый сегмент атомарного раздела базы знаний трактуется как одно из введений в этот атомарный раздел. Завершение неатомарного раздела базы знаний трактуется как одно из заключений к этому неатомарному разделу, а последний сегмент атомарного раздела базы знаний трактуется как одно из заключений в этот атомарный раздел.}

\scnheader{сегмент базы знаний}
\scnidtf{сегмент базы знаний ostis-системы}
\scnidtf{структурно выделяемое sc-знание ostis-системы, структурный уровень которого ниже уровня разделов базы знаний}
\scnnote{В отличие от разделов базы знаний сегменты базы знаний не могут иметь иерархической структуры, т.е. не могут состоять из сегментов более низкого структурного уровня.}
\scnnote{Сегменты базы знаний входят в состав (1) структурированных атомарных разделов базы знаний, а также (2) структурированных начал и завершений неатомарных разделов базы знаний.}
\scnsubset{неструктурированное структурно выделяемое sc-знание ostis-системы} 

\scnheader{sc-знание ostis-системы нижнего структурного уровня}
\scnidtf{sc-знание, входящее в состав либо неструктурированного атомарного раздела базы знаний, либо неструктурированного начала неатомарного раздела базы знаний, либо неструктурированного завершения неатомарного раздела базы знаний, либо сегмента базы знаний.}
\scnnote{Для наглядного отображения (визуализации) неструктурированного структурно выделяемого sc-знания ostis-системы целесообразно представить указанное sc-знание в виде конкатенации (последовательности) таких sc-знаний, которые, во-первых, были бы достаточно крупными и логико-семантически значимыми для соответствующего неструктурированного структурно выделяемого sc-знания ostis-системы и, во-вторых, для которых существовал бы алгоритм \uline{однозначного} (!) размещения (на экране) внешнего представления этих sc-знаний (в SCg-коде или в SCn-коде).\\
Однозначность здесь означает наличие легко усваиваемого пользователями стандартного \uline{стиля визуализации} sc-знаний и заключается в том, что многократная визуализация одного и того же sc-знания с помощью указанного алгоритма должна приводить к синтаксически эквивалентным, а в случае SCg-кода и к геометрически конгруэнтным текстам. Очевидно, что для произвольных sc-знаний большого объёма такого алгоритма не существует, но для sc-знаний, содержащих описание собственной структуры и семантической типологии собственных фрагментов, разработка такого алгоритма вполне реальна при наличии достаточных указанных метазнаний о структуре отображаемых (визуализируемых) sc-знаний.}

\scnendstruct

\scnsegmentheader{Формальная спецификация знаний ostis-систем}

\scnstartsubstruct

\scnrelto{сегмент раздела базы знаний}{\currentname}
\scnrelto{следующий сегмент базы знаний}{Представление предметных областей и онтологий в базах знаний ostis-систем}
\scntext{введение}{Формальная спецификация знания ostis-системы представляет собой sc-структуру, описывающую свойства специфицируемого знания и включающую в себя: 
\begin{scnitemize}
\item связи принадлежности специфицируемого знания соответствующим классам знаний ostis-систем;
\item связи, указывающие логически предшествующее и логически следующее знание ostis-системы;
\item связь, описывающую декомпозицию специфицируемого знания на последовательность знаний более низкого структурного уровня;
\item различного вида связи с другими знаниями ostis-систем, которые сами "целиком"\ входят в состав спецификации специфицируемого знания (такими знаниями могут быть аннотации, предисловия, введения, оглавления, заключения);
\item различного вида связи с другими знаниями ostis-систем, которые сами не входят в состав спецификации специфицируемого знания (такого рода связями могут быть связи семантической близости специфицируемого знания с другими знаниями, связи семантической эквивалентности, связи семантического включения, связи противоречивости знаний);
\item связи, указывающие различного вида ключевые sc-элементы, соответствующие специфицируемому знанию;
\item связи специфицируемого знания с авторским коллективом, коллективом рецензентов, с датой его последнего обновления;
\item для каждого нового целостного фрагмента, вводимого в состав базы знаний, в истории эволюции этой базы знаний указываются:
\begin{scnitemizeii}
\item \textit{автор*} или \textit{авторы*} первой версии этого фрагмента;
\item отметка времени появления (дата-час-минута) всех версий этого фрагмента (в том числе и окончательно утверждённой, согласованной версии, которая, собственно, и становится фрагментом, включенным в согласованную часть базы знаний);
\item \textit{рецензии*} (замечания к доработке) всех предварительных версий разрабатываемого фрагмента базы знаний;
\item \textit{авторы*} всех указанных рецензий;
\item отметка времени появления всех указанных рецензий;
\item события по одобрению, утверждению различных предварительных версий разрабатываемого фрагмента базы знаний различными рецензентами и экспертами с указанием отметки времени появления этих событий;
\item темпоральная последовательность предварительных версий.
\end{scnitemizeii}
\end{scnitemize}
}

\scnheader{класс знаний ostis-системы}
\scnidtf{вид (тип) знаний ostis-системы, указание принадлежности которому входит в спецификацию знания ostis-системы}
\scnhaselement{файл-знаний ostis-системы}
\scnhaselement{sc-знание ostis-системы}
\scnaddlevel{1}
\scnsuperset{структурно выделяемое sc-знание ostis-системы}
\scnaddlevel{-1}

\scnheader{класс структурно выделяемых знаний ostis-системы}
\scnidtf{класс \uline{синтаксически} выделяемых знаний ostis-системы}
\scnhaselement{раздел базы знаний}
\scnhaselement{атомарный раздел базы знаний}
\scnhaselement{неатомарный раздел базы знаний}
\scnhaselement{завершение неатомарного раздела базы знаний}
\scnhaselement{сегмент базы знаний}
\scnhaselement{sc-знание нижнего структурного уровня}

\bigskip

\scnstartset
\scnheader{структурно выделяемое знание ostis-системы}
\scnsubset{рефлексивная sc-структура}
\scnidtf{структура, спецификация (описание свойств) которой входит в её состав, то есть в состав самой этой структуры}
\scnendstruct\\
\scntext{следствие}{спецификация каждого структурно выделяемого sc-знания ostis-системы входит в состав этого sc-знания}

\scnheader{класс семантически выделяемых знаний ostis-системы}
\scnidtf{класс знаний ostis-систем, выделяемых в базах знаний по их семантическим свойствам}
\scnhaselement{спецификация знания ostis-системы}
\scnaddlevel{1}
    \scnsuperset{структурная спецификация знания ostis-системы}
    \scnsuperset{оглавление}
    \scnsuperset{предисловие}
    \scnaddlevel{1}
        \scnidtf{исторический аспект эволюции специфицируемого знания}
    \scnaddlevel{-1}
    \scnsuperset{аннотация}
\scnaddlevel{-1}

\scnhaselement{введение}
\scnaddlevel{1}
    \scnidtf{Описание \uline{актуальности} соответствующего знания, рассматриваемой в этом знании \uline{проблемы} (постановки задачи), \uline{предыстории} ее рассмотрения}
\scnaddlevel{-1}

\scnhaselement{спецификация знания ostis-системы и введение в него}
\scnaddlevel{1}
    \scnidtf{начальная информация о знании ostis-системы}
\scnaddlevel{-1}

\scnhaselement{заключение}
\scnaddlevel{1}
    \scnidtf{резюме, выводы, следствия, перспективы применения, сравнительный анализ, библиография соответствующего знания ostis-системы}
\scnaddlevel{-1}

\scnheader{бинарное ориентированное отношение, каждая пара которого связывает знание (или множество знаний) ostis-системы с sc-элементом, обозначающим сущность, описываемую этим знанием (или знаниями)}
\scnnote{Вторыми компонентами ориентированных пар отношений, входящих в данный класс бинарных отношений, могут быть знаки как \textit{файлов-знаний ostis-систем}, так и \textit{sc-знаний ostis-систем}}
\scnhaselement{определение*}
\scnaddlevel{1}
    \scnidtf{быть определением заданного понятия*}
\scnaddlevel{-1}
\scnhaselement{однозначная спецификация*}
\scnaddlevel{1}
    \scnidtf{быть однозначной спецификацией заданной сущности*}
    \scnsuperset{определение}
    \scnaddlevel{1}
        \scnidtf{быть однозначной спецификацией заданного понятия*}
    \scnaddlevel{-1}
\scnaddlevel{-1}
\scnhaselement{пояснение*}
\scnhaselement{примечание*}
\scnhaselement{утверждение*}
\scnhaselement{обоснование целесообразности создания*}
\scnhaselement{принципы, лежащие в основе*}
\scnhaselement{требования*}
\scnhaselement{правила построения*}
\scnhaselement{достоинства*}
\scnhaselement{недостатки*}
\scnhaselement{сравнительный анализ*}
\scnhaselement{ключевой знак*}
\scnaddlevel{1}
    \scnidtf{быть ключевым sc-элементом, который семантически входит в состав заданного знания ostis-систем*}
    \scnaddlevel{1}
        \scnnote{для sc-знания ostis-системы ключевой sc-элемент непосредственно принадлежит этому sc-знанию, которое формально трактуется как множество всех sc-элементов, входящих в состав этого sc-знания}
    \scnaddlevel{-1}
    \scnnote{данное отношение в основном используется для спецификации разделов базы знаний и сегментов базы знаний}
\scnaddlevel{-1}
\scnhaselement{описание заданной сущности*}

\scnheader{пояснение*}
\scnidtf{быть пояснением смысла сущности, обозначаемой заданным (указываемым) sc-элементом*}
\scnnote{поясняемой сущностью может быть любая сущность, в том числе, и знание ostis-системы}
\scnidtf{достаточно полное описание смысла данной сущности*}
\scnidtf{толкование данной сущности*}
\scnidtf{что это такое*}
\scnidtf{знание ostis-системы, являющееся ответом на вопрос "что это такое?"*}

\scnheader{примечание*}
\scnidtf{быть уточняющим (дополнительным) пояснением смысла сущности, обозначаемой заданным sc-элементом*}
\scnnote{дополнительно уточняемой сущностью может быть сущность любого вида, в том числе, и знание ostis-системы}

\scnheader{утверждение*}
\scnidtf{логическое высказывание (закономерность, аксиома, теорема), описывающее свойство заданной сущности*}

\scnheader{принципы, лежащие в основе*}
\scnidtf{принципы, лежащие в основе сущностей заданного класса или некоторой конкретной заданной сущности*}

\scnheader{требования*}
\scnidtf{требования, предъявляемые к искусственным объектам заданного класса или к некоторому конкретному заданному искусственному объекту*}

\scnheader{правила построения*}
\scnidtf{правила построения (проектирования, синтеза, разработки, реализации) искусственных объектов заданного класса либо некоторого конкретного заданного искусственного объекта*}

\scnheader{достоинства*}
\scnidtf{описание достоинств (преимуществ, положительных свойств) сущностей заданного класса либо некоторой конкретной заданной сущности*}

\scnheader{недостатки*}
\scnidtf{описание недостатков сущностей заданного класса либо некоторой конкретной заданной сущности*}

\scnheader{сравнительный анализ*}
\scnidtf{сравнительный анализ описываемой сущности с различными аналогичными сущностями*}
\scnnote{в таком сравнительном анализе перечисляются сущности, аналогичные описываемой, и указываются их сходства и отличия по отношению к описываемой сущности. Для формальной детализации сравнительного анализа вводятся отношения \textit{сравнение*}, \textit{отличия*}, \textit{сходства*}}

\scnheader{описание данной сущности*}
\scnidtf{семантическая окрестность данной сущности*}
\scnidtf{спецификация данной сущности*}
\scnidtf{бинарное ориентированное отношение, каждая пара которого связывает (1) некоторый sc-элемент, обозначающий описываемую сущность с (2) соответствующим знанием ostis-системы, описывающим указанную сущность}
\scnrelfrom{первый домен}{sc-элемент}
\scnrelfrom{второй домен}{знание ostis-системы}
\scnrelboth{обратное отношение}{ключевой знак*}
\scnsuperset{однозначная спецификация*}
\scnsuperset{пояснение*}
\scnsuperset{примечание*}
\scnsuperset{обоснование целесообразности создания*}
\scnsuperset{принципы, лежащие в основе*}
\scnsuperset{требования*}
\scnsuperset{правила построения*}
\scnsuperset{достоинства*}
\scnsuperset{недостатки*}

\scnheader{бинарное ориентированное отношение, каждая пара которого связывает знание (или множество знаний) ostis-системы с \uline{парой} (!) sc-элементов, описываемых этим знанием (или знаниями)}
\scnhaselement{сравнение*}
	\scnaddlevel{1}
	\scnidtf{сравнение двух заданных сущностей*}
	\scnaddlevel{-1}
\scnhaselement{сравнения*}
	\scnaddlevel{1}
	\scnidtf{множество фактов, описывающих различные сходства и различия двух сравниваемых сущностей*}
	\scnaddlevel{-1}
\scnhaselement{отличие*}
	\scnaddlevel{1}
	\scnidtf{отличие двух сравниваемых сущностей*}
	\scnaddlevel{-1}
\scnhaselement{отличия*}
	\scnaddlevel{1}
	\scnidtf{множество отличий двух сравниваемых сущностей*}
	\scnaddlevel{-1}
\scnhaselement{сходство*}
	\scnaddlevel{1}
	\scnidtf{сходство двух сравниваемых сущностей*}
	\scnaddlevel{-1}
\scnhaselement{сходства*}
	\scnaddlevel{1}
	\scnidtf{множество сходств двух сравниваемых сущностей*}
	\scnaddlevel{-1}

\scnheader{отношение, связывающее знания ostis-системы между собой}
\scnhaselement{синтаксическая эквивалентность*}
\scnhaselement{семантическая эквивалентность*}
\scnhaselement{семантическое включение*}
\scnhaselement{логическое следствие*}
\scnhaselement{текст доказательства*}
\scnhaselement{конкатенация сегментов базы знаний*}
\scnhaselement{следующий сегмент базы знаний*}
\scnhaselement{конкатенация разделов базы знаний*}
\scnhaselement{следующий раздел базы знаний*}
\scnhaselement{подраздел базы знаний*}
\scnhaselement{сегмент базы знаний*}
	\scnaddlevel{1}
	\scnsuperset{первый сегмент базы знаний*}
	\scnsuperset{последний сегмент базы знаний*}
	\scnaddlevel{-1}	
\scnhaselement{начало неатомарного раздела базы знаний*}
\scnhaselement{завершение неатомарного раздела базы знаний*}
\scnhaselement{введение*}
\scnhaselement{аннотация*}
\scnhaselement{предисловие*}
\scnhaselement{оглавление*}
\scnhaselement{упражнения*}
\scnhaselement{спецификация знания ostis-системы и введение в него*\\
	\scnaddlevel{1}
	\scnnote{Первый сегмент атомарного раздела базы знаний всегда является спецификацией этого раздела и введением в него.\\
	Первый сегмент начала неатомарного раздела всегда является спецификацией этого неатомарного раздела}
	\scnaddlevel{-1}}
\scnhaselement{заключение*\\
	\scnaddlevel{1}
	\scnnote{Последний сегмент атомарного раздела базы знаний семантически трактуется как заключение этого атомарного раздела. Завершение неатомарного раздела базы знаний трактуется как заключение этого неатомарного раздела}
	\scnaddlevel{-1}}

\scnheader{конкатенация*}
\scnidtf{соединение указываемых сущностей в заданной последовательности}
\scnnote{соединяемыми сущностями (объектами) могут быть строки символов, разделы или сегменты базы знаний ostis-системы, и т.д.}
\scnnote{последовательность (порядок) соединяемых сущностей при выполнении операции конкатенации* задается \uline{либо кортежем} знаков соединяемых сущностей, \uline{либо множеством} таких \uline{ориентированных пар} знаков соединяемых сущностей, которые принадлежат отношению порядка, заданному на множестве соединяемых сущностей}
	\scnaddlevel{1}
	\scntext{следствие}{Следовательно, с формальной точки зрения, отношение \textit{конкатенации*} представляет собой \uline{объединение} двух квазибинарных функциональных отношений, множество аргументов которых задается по-разному -- либо кортежем произвольного числа компонентов, либо множеством ориентированных пар (бинарных кортежей). Такое объединение возможно, поскольку во внутреннем представлении отношения конкатенации* (в SC-коде) легко отличить кортеж от множества бинарных кортежей (sc-дуг), поскольку явно вводится понятие кортежа и понятие бинарного ориентированного отношения (не обязательно бесконечного).}
	\scntext{целесообразность}{Целесообразность такого неоднозначного представления связок отношения конкатенации* обусловлена тем, что во внешнем представлении связок этого отношения иногда удобно использовать первый вариант представления этих связок, а иногда -- второй.}
	\scnaddlevel{-1}
\scnsuperset{конкатенация строк*}
\scnsuperset{конкатенация разделов базы знаний*}
\scnsuperset{конкатенация сегментов базы знаний*}
\scnsuperset{конкатенация подразделов*}

\scnheader{конкатенация разделов базы знаний*}
\scnidtf{декомпозиция неатомарного раздела, представленная в виде последовательности всех ближайших (непосредственных) его подразделов, а также начала и завершения этого раздела}

\scnheader{конкатенация сегментов базы знаний*}
\scnidtf{декомпозиция атомарного раздела или начала неатомарного раздела или завершения неатомарного раздела, представленная в виде последовательности сегментов, входящих в его состав}

\scnheader{конкатенация подразделов*}
\scnexplanation{Бинарное ориентированное \textit{отношение}, каждая \textit{пара} которого связывает \textit{знак} некоторого \textit{раздела базы знаний} либо знак \textit{файла}, содержащего некоторый \textit{документ}, с упорядоченным множеством всех \uline{непосредственных} подразделов указанного \textit{раздела базы знаний} или указанного \textit{документа}*}
	\scnaddlevel{1}
	\scnnote{Подчеркнем, что в указанное \textit{упорядоченное множество} подразделов заданного \textit{раздела базы знаний} или заданного \textit{документа} подразделы подразделов этого \textit{раздела базы знаний} (или этого \textit{документа}) \uline{не входят}.}
	\scnaddlevel{-1}


\scnheader{порядок*}
\scnidtf{Объединение всевозможных отношений порядка*}
\scnidtf{последовательность*}
\scnidtf{быть следующим*}
\scnidtf{следующий*}
\scnsuperset{следующий раздел базы знаний*\\
	\scnaddlevel{1}
	\scnidtf{порядок разделов базы знаний*}
	\scnaddlevel{-1}}
\scnsuperset{следующий сегмент базы знаний*\\
	\scnaddlevel{1}
	\scnidtf{порядок сегментов базы знаний*}
	\scnaddlevel{-1}}

\scnheader{следует отличать*}
\scnhaselementset{порядок*;отношение порядка\\
	\scnaddlevel{1}
	\scnidtf{Семейство всевозможных отношений порядка}
	\scnaddlevel{-1}}

\scnheader{подраздел базы знаний*}
\scnidtf{быть непосредственным подразделом \uline{данного} раздела базы знаний*}
\scnnote{Отношение "быть подразделом базы знаний" не является транзитивным, т.е. подраздел подраздела заданного раздела базы знаний не является подразделом этого (заданного) раздела -- "вассал моего вассала -- не мой вассал".}

\scnheader{сегмент базы знаний*}
\scnidtf{быть сегментом \uline{данного} атомарного раздела базы знаний или начала неатомарного раздела базы знаний или завершения неатомарного раздела базы знаний*}

\scnheader{первый сегмент базы знаний*}
\scnidtf{быть первым сегментом \uline{данного} атомарного раздела базы знаний или начала неатомарного раздела базы знаний или завершения неатомарного раздела базы знаний*}
\scnsubset{сегмент базы знаний*}

\scnheader{последний сегмент базы знаний*}
\scnidtf{быть последним сегментом \uline{данного} атомарного раздела базы знаний или начала неатомарного раздела базы знаний или завершения неатомарного раздела базы знаний*}
\scnsubset{сегмент базы знаний*}

\scnheader{начало неатомарного раздела базы знаний*}
\scnidtf{быть началом \uline{данного} неатомарного раздела базы знаний*}

\scnheader{завершение неатомарного раздела базы знаний*}
\scnidtf{быть завершением \uline{данного} неатомарного раздела базы знаний*}

\scnheader{введение*}
\scnidtf{быть введением в \uline{данное} sc-знание ostis-системы*}
\scnnote{введение может иметь sc-знание \uline{любого} вида}

\scnheader{аннотация*}
\scnidtf{быть аннотацией (рефератом) \uline{данного} раздела базы знаний*}

\scnheader{предисловие*}
\scnidtf{быть предисловием к \uline{данному} разделу базы знаний*}

\scnheader{оглавление*}
\scnidtf{быть полным оглавлением \uline{данного} раздела базы знаний, описывающим полную иерархию всех его подразделов до уровня атомарных разделов базы знаний*}

\scnheader{упражнения*}
\scnidtf{быть \uline{перечнем} упражнений для \uline{данного} раздела базы знаний*}
\scnidtf{Бинарное ориентированное отношение, каждая пара которого связывает раздел базы знаний с \uline{перечнем} упражнений (вопросов и задач), самостоятельное выполнение которых существенно повышает уровень усвоения содержания этого раздела*}
\scnnote{Упражнения к неатомарному разделу базы знаний входят в состав завершения этого раздела. Упражнения к структурированному атомарному разделу базы знаний входят в состав последнего сегмента этого раздела.}

\scnheader{логическое следствие*}
\scnidtf{быть ориентированной парой двух знаний ostis-системы, второе из которых логически следует из первого*}
\scnidtf{что из этого логически следует*}

\scnheader{последствие*}
\scnidtf{быть ориентированной парой двух знаний ostis-системы, первое из которых является описанием ситуации, являющейся причиной (предпосылкой, достаточным условием) возникновения ситуации, которая описывается вторым из указанных знаний ostis-системы*}
\scnidtf{причинно-следственная связь}
\scnrelboth{обратное отношение}{причина*}
	\scnaddlevel{1}
	\scnidtf{предпосылка*}
	\scnidtf{ситуация, являющаяся достаточным условием*}
	\scnaddlevel{-1}

\scnheader{отношение, связывающее знания ostis-системы с сущностями, которые этими знаниями не описываются и сами не являются знаниями}
\scnhaselement{авторы*}
\scnhaselement{рецензенты*}

\scnheader{следует отличать*}
\scnhaselementset{начало неатомарного раздела базы знаний\\
	\scnaddlevel{1}
	\scniselement{абсолютное понятие}
	\scnaddlevel{-1}
;начало неатомарного раздела базы знаний*\\
	\scnaddlevel{1}
	\scniselement{отношение}
	\scnaddlevel{-1}}

\scnresetlevel
\scnsegmentheader{Представление предметных областей и онтологий в базах знаний ostis-систем}
\scnstartsubstruct

\scnrelto{введение}{(Предметная область и онтология предметных областей $\cup$ Предметная область и онтология онтологий)\\
\scnaddlevel{-1}
\scnidtf{Введение в Предметную область и онтологию предметных областей и онтологий}}
\scnaddlevel{-1}

\scnheader{предметная область}
\scnexplanation{Для эффективной коллективной разработки и эксплуатации базы знаний ostis-системы важна не просто её структуризация, а такая структуризация, которая носит максимально \uline{объективный} характер, имеющий четкую \uline{семантическую интерпретацию} (!) и позволяющий на основе \uline{семантических} (!) связей между структурно выделяемыми фрагментами базы знаний легко определять (локализовывать) ``местоположение'' либо искомых знаний, либо новых знаний, вводимых в состав базы знаний. Такая семантическая структуризация базы знаний, формирование системы семантически связанных между собой ``семантических полочек'', на которых размещаются конкретные знания, удовлетворяющие четко заданным требованиям, существенно упрощает навигацию по базе знаний и четко \uline{локализует} эволюцию знаний, находящихся на каждой ``семантической полочке''. В основе указанной семантической структуризации базы знаний лежит иерархическая система предметных областей. С содержательной точки зрения предметная область представляет собой совокупность фактографических высказываний, описывающих \uline{все} (!) элементы \uline{заданного} множества объектов исследования' с помощью \uline{заданного} набора отношений и параметров (характеристик). Максимальный класс указанных объектов исследования, некоторые специально выделяемые подклассы этого класса, а также указанные отношения и параметры задают систему понятий, лежащих в основе предметной области, которую будем называть схемой предметной области. Подчеркнем то, что схема предметной области, определяющая семантическую ``систему координат'' в рамках соответствующей предметной области, в известной степени имеет \uline{субъективный} характер и является предметом \uline{соглашения} (!) (консенсуса) между специалистами в этой области. Подчеркнем также, что схема предметной области может эволюционировать (может меняться набор понятий, может уточняться их денотационная семантика).\\
Понятие предметной области можно считать обобщением понятия алгебраической системы, ориентированной на решение проблемы \uline{семантической} структуризации базы знаний. В памяти ostis-системы предметная область представляется обычно бесконечной sc-структурой, которая задается (1) неким множеством исследуемых сущностей, (2) семейством отношений, алгебраических операций и параметров (свойств), заданных на множестве исследуемых сущностей, каждое из которых либо рассматривается в рамках данной предметной области, либо ``наследуется'' из другой предметной области более высокого уровня. Каждая предметная область в базе знаний может быть представлена своим разделом. При этом на множестве таких разделов могут быть заданы не только рассмотренные выше ``синтаксические'' отношения, но и целый ряд семантически интерпретируемых отношений, например, отношение ``быть частной предметной областью*''. Примером связки этого отношения является связка между Предметной областью геометрических фигур в Евклидовом пространстве и Предметной областью планарных фигур Евклидова пространства. Таких отношений, заданных на множестве предметных областей и уточняющих характер соотношения между множествами исследуемых сущностей, а также рассматриваемыми или ``наследуемыми'' отношениями, алгебраическими операциями и параметрами, существует достаточно много. Но, если к этому добавить анализ соотношения не только между самими предметными областями, но и соотношения между представляющими их sc-структурами в рамках базы знаний конкретной ostis-системы, а также в рамках глобального смыслового пространства (SC-пространства), то число семантически интерпретируемых отношений, заданных на множестве предметных областей существенно расширится. К числу важных отношений, заданных на множестве предметных областей относятся также отношения, связывающие предметные области с различными структурно выделяемыми фрагментами базы знаний, которые предметными областями не являются. Ключевым отношением такого вида является \textit{Отношение, связывающее предметные области с соответствующими им онтологиями}.} 

\scnheader{предметная область}
\filemodetrue
\scnrelfromset{правила построения}{Понятия и соответствующие им термины в базе знаний ostis-системы группируются по четко выделенным предметным областям. При этом связи между предметными областями \uline{явно} (!) указываются.;
Изложение материала в базе знаний ostis-системы построено по принципу ``сверху вниз'', переходя от общих предметных областей и онтологий к частным для явного указания направления \uline{наследования свойств}.
;Во всех представляемых и описываемых предметных областях необходимо явно указывать \uline{все} (!) понятия, используемые в предметных областях с указанием каждой роли этих понятий в рамках предметной области -- исследуемых понятий' в рамках описываемой предметной области (максимальный класс объектов исследования'\char59 немаксимальный класс объектов исследования'\char59 исследуемое отношение, заданное на объектах исследования, вспомогательных понятий', используемых в описываемой предметной области, но исследуемых в других предметных областях).}
\filemodefalse

\scnheader{онтология}
\scnexplanation{Все рассматриваемые в базе знаний ostis-системы предметные области должны быть формально специфицированы, т.е. для каждой из них должна быть построена и указана соответствующая ей онтология и, в частности, должны быть указаны все известные связи с другими предметными областями и с иными фрагментами базы знаний ostis-системы. Можно выделить целый ряд частных онтологий, описывающих свойства соответствующих предметных областей с разных ``ракурсов''.}
\scnsuperset{схема предметной области}
	\scnaddlevel{1}
    \scnidtf{специфицикация внутренней структуры предметной области}
	\scnidtf{перечень всех понятий и некоторых ключевых объектов исследования, лежащих в основе специфицируемой предметной области, с указанием их роли в рамках этой предметной области}
	\scnaddlevel{-1}
\scnsuperset{специфицикация внешних связей предметной области}
	\scnaddlevel{1}
	\scnidtf{описание связей предметной области с онтологиями и с другими предметными областями}
	\scnaddlevel{-1}
\scnsuperset{онтология определений исследуемых понятий}
\scnsuperset{логическая онтология}
	\scnaddlevel{1}
	\scnidtf{онтология аксиом, теорем и текстов доказательств теорем}
	\scnaddlevel{-1}
\scnsuperset{онтология задач}
	\scnaddlevel{1}
	\scnidtf{онтология задач, решаемых в рамках предметной области, объединенной с соответствующей онтологией}
	\scnaddlevel{-1}
\scnsuperset{онтология классов задач и методов их решения}
\scnsuperset{терминологическая онтология}
\scnsuperset{онтология эволюции предметной области и соответствующей ей онтологии}
\scnsuperset{онтология авторства}

\scnheader{семантический тип разделов баз знаний}
\scnexplanation{Различные предметные области, различные онтологии, а также предметные области, объединенные с соответствующими им онтологиями, являются важнейшими видами sc-знаний ostis-систем, обеспечивающими логически стройную систематизацию знаний ostis-систем и, соответственно, семантическую структуризацию баз знаний. При этом указанные типы знаний обычно представляются в виде разделов базы знаний, иерархия которых соответствует иерархии предметных областей. Но, кроме разделов базы знаний, являющихся предметными областями, онтологиями, предметными областями, объединенными с соответствующими им онтологиями, вводится и целый ряд других семантических типов разделов базы знаний, определяемых характером соотношения разделов базы знаний с используемыми (рассматриваемыми) предметными областями и онтологиями.}
\scnsubset{класс семантически выделяемых знаний ostis-системы}
\scnidtf{класс разделов базы знаний, задаваемый характером их соотношения с рассматриваемыми предметными областями и онтологиями}
\scnhaselement{предметная область}
\scnhaselement{фрагмент предметной области}
\scnhaselement{интегрированная онтология}
\scnhaselement{частная онтология}
\scnhaselement{предметная область \& онтология}
    \scnaddlevel{1}
    \scnidtf{предметная область, объединенная (интегрированная) с соответствующей ей онтологией}
    \scnidtf{предметная область вместе с онтологией, которая её специфицирует}
    \scnaddlevel{-1}
    
\scnheader{семантический тип разделов баз знаний}
\scnnote{Если каждому разделу базы знаний ostis-системы будет четко соответствовать его семантический тип, то к "синтаксическим"{}
 связям между разделами базы знаний (конкатенация разделов базы знаний*, следующий раздел базы знаний* и др.) добавится большое количество "осмысленных"{} (семантически интерпретируемых) связей, определяющих "семантическое местоположение"{} ("семантические координаты"{}) каждого раздела базы знаний во множестве всех разделов, входящих в состав базы знаний ostis-системы.}

\scnheader{Представление предметных областей и онтологий в базах знаний ostis-систем}
\scniselement{сегмент базы знаний}
\scntext{заключение}{В основе представления баз знаний ostis-систем лежат развитые средства семантической структуризация баз знаний и семантической систематизации знаний ostis-систем. Можно выделить следующие уровни систематизации элементов и фрагментов смыслового пространства, построенного на основе SC-кода:
\begin{scnitemize}
	\item уровень знаков всевозможных сущностей (уровень sc-элементов);
	\item уровень вводимых понятий;
	\item уровень высказываний;
	\item уровень предметных областей, онтологий и разделов, семантический тип которых известен.
\end{scnitemize}} 

\scnendstruct

\scnsegmentheader{Принципы структуризации и оформления внешнего представления знаний ostis-систем}

\scnstartsubstruct

\scnheader{внешнее представление знаний ostis-системы}
\scnexplanation{внешнее представление некоторого фрагмента базы знаний ostis-системы, используемое для ввода новой информации в состав базы знаний ostis-системы или для вывода (отображения) запрашиваемого фрагмента базы знаний}
\scnnote{Способ представления исходных текстов баз знаний ostis-систем должен быть максимально возможным образом использован и для вывода (отображения) запрашиваемых пользователем фрагментов баз знаний, особенно, если запрашиваются достаточно большие фрагменты баз знаний, которые необходимо не только представлять, но и структурировать унифицированным образом. Очевидно, что для пользователей желательно, чтобы и для ввода информации в ostis-систему, и для ее вывода использовались одни и те же языковые средства и правила оформления}
\scnnote{Требования, предъявляемые к оформлению внешних текстов знаний ostis-систем (sc-знаний) носят достаточно противоречивый характер -- с одной стороны, речь идет о формальных текстах, легко воспринимаемых (понимаемых, транслируемых) ostis-системами, а, с другой стороны, желательно, чтобы эти же формальные тексты легко воспринимались (понимались) широким кругом людей и не требовали для этого от них длительной подготовки. Отметим при этом, что работа с формальными текстами требует от человека достаточно высокой культуры \uline{точного} мышления (математической культуры).

Отметим также, что использование формальных языков является важнейшим и необходимым этапом эволюции человеческой деятельности в любой области (в математике, в физике, в технике).

Тем не менее, проблема создания универсального языка представления исходных текстов различного вида знаний, который был бы достаточно удобен как для интеллектуальных компьютерных систем, так и для \uline{широкого} круга разработчиков баз знаний и экспертов, требует конкретного решения.

\scnauthorcomment{за счет чего, каким путем. Наш подход к решению это проблемы в чем заключается.}
}
\filemodetrue
\scnreltovector{требования}{Стиль и характер оформления внешнего представления sc-знаний должен обеспечить возможность интуитивного понимания смысла текста при отсутствии понимания различного рода синтаксических деталей. Для этого:
\begin{scnitemize}
\item формальный текст должен максимально возможным образом использовать привычную для широкого круга специалистов терминологию\char59
\item структуризация, форматирование формальных текстов также должны опираться на сформировавшиеся традиции\char59
\item внешнее представление (внешний текст) sc-знания должен включать в себя такое количество отображаемых ея-файлов, прочтение которых было бы достаточно для понимания смысла представляемого sc-знания, а также для понимания формальных средств его представления
\end{scnitemize};
Формальный текст (как внутреннего, так и внешнего представления sc-знаний) должен включать в себя средства для уточнения смысла используемых знаков и соответствующих им терминов, а также смысла некоторых фрагментов формального текста. Для этого в формальный язык вводятся естественно-языковые файлы, отображаемые в исходных текстах и поясняющие используемые термины, а также комментирующие или даже полностью переводящие на естественный язык различные фрагменты формального представления базы знаний.;
Все используемые в базе знаний ostis-системы (в том числе, и в её ея-файлах) внешние идентификаторы sc-элементов (термины, имена, условные обозначения) должны быть формально специфицированы средствами SC-кода. Подчеркнем, что здесь речь идет о спецификации не самих sc-элементов, а их внешних идентификаторов (в первую очередь, простых sc.s-идентификаторов) -- их происхождение, использование, авторство и т.д.;
Все отношения, параметры и другие понятия, используемые в формальных текстах должны быть пояснены в соответствующих формальных онтологиях. Первое упоминание во внешнем тексте каждого такого понятия должно быть кратко пояснено с помощью поясняющего ея-файла, а также сделана ссылка на раздел базы знаний, в которых приведена подробная и формальная спецификация указанного понятия с дополнительным указанием номера этого раздела с помощью нетранслируемого комментария.;
Аналогичным образом в отображаемом внешнем представлении sc-знания поясняются и комментируются все \uline{первые} (в рамках этого внешнего представления) использования средств формального представления знаний со ссылками на разделы базы знаний, где указанные языковые средства подробно описываются. С самого начала внешнего представления большого структурированного раздела базы знаний (каковым, в частности, является Документация Технологии OSTIS) с помощью нетранслируемых комментариев, не входящих в состав базы знаний, либо с помощью ея-файлов ostis-систем необходимо пояснять все нюансы формализации со ссылкой на ближайший раздел и сегмент, где это будет подробнее рассмотрено.;
При описании формальных средств приводить (формальным образом) конкретные \uline{примеры} со ссылкой на раздел или сегмент, где этот пример будет рассмотрен подробнее (например, на соответствующую предметную область и онтологию);
Все комментарии и примечания, которые можно представить средствами SCn-кода или SCg-кода, нужно оформлять именно так. Нетранслируемыми комментариями  \uline{не стоит увлекаться}.;
В формальных текстах и в естественно-языковых файлах, входящих в состав базы знаний ostis-системы, для идентификации (именования) sc-элементов должны использоваться только те термины, которые являются \uline{основными}(!) внешними идентификаторами соответствующих sc-элементов, выделяемыми жирным и нежирным курсивом. При этом, если идентификаторы (названия, имена) разделов, сегментов базы знаний, начал неатомарных разделов и завершений неатомарных разделов находятся в позиции \uline{заголовков} указанных фрагментов базы знаний, то они оформляются жирным курсивом с увеличенным расстоянием между символами, а заголовки разделов дополнительно выделяются увеличенным размером символов.;
В согласованной (общепризнанной) части базы знаний противоречия трактуются как выявленные ошибки в базе знаний, подлежащие устранению. Но в истории эволюции
базы знаний противоречия могут присутствовать как противоречия разных точек зрения разных авторов. Заметим при этом, что разные точки зрения далеко не всегда являются противоречивыми (взаимоисключающими). Они могут просто дополнять друг друга, описывать исследуемые сущности с разных "ракурсов". Умение видеть противоречия только там, где, они действительно есть, и умение локализовать эти противоречия (выделить их суть) -- это необходимые навыки для разработки \uline{практически полезных} баз знаний.}
\filemodefalse

\scnheader{внешнее sc.n-представление знаний ostis-системы}
\filemodetrue
\scnrelfromvector{правила оформления}{Основным языком внешнего представления баз знаний ostis-систем является SCn-код, рассмотренный в разделе \scnsourcecomment{0.3.4} \textit{Введение в язык структурированного представления баз знаний ostis-систем} и в разделе \scnsourcecomment{2.1.1.4} \textit{Предметная область и онтология SCn-кода} (Semantic Code natural). Но в состав текста SCn-кода могут входить тексты и других языков (тексты SCg-кода, тексты SCs-кода, тексты естественных языков, тексты различных искусственнных языков), а также различного рода нетекстовые информационные конструкции (рисунки, таблицы, чертежи, графики, фотографии). Указанные "инородные"{} для SCn-кода информационные конструкции, а также описываемые тексты самого SCn-кода оформляются во внешнем представлении базы знаний либо как нетранслируемые, но специфицируемые файлы, либо как транслируемые инородные для SCn-кода информационные конструкции, ограниченные, соответственно, либо sc.n-рамками (квадратными скобками), либо sc.n-контурами (фигурными скобками).
;Структуризация внешнего представления баз знаний ostis-систем является полным ограничением структуризации внутреннего представления баз знаний ostis-систем.
;В случае, если осуществляется внешнее представление \uline{полного} текста указываемого сложноструктурированного раздела базы знаний, как, например, внешнее представление раздела под названием \textit{``Документация Технологии OSTIS''} вместе со всеми его подразделами, подразделами подразделов и т.д., то последовательность и иерархическая структура отображения подразделов и сегментов указанного сложноструктурированного фрагмента базы знаний в точности соответствует отношению порядка этих подразделов и сегментов в рамках внутреннего представления базы знаний, а также в точности соответствует иерархической структуре представляемого (отображаемого, визуализируемого) раздела базы знаний, который в рамках внешнего текста, представляющего рассматриваемый сложноструктурированный раздел будем называть максимальным разделом* отображаемого фрагмента базы знаний.
;База знаний каждой ostis-системы представляет собой иерархическую систему разделов, к которым должны "привязываться"{} исходные тексты каждой новой информации, вводимой в базу знаний, и, прежде всего, исходные тексты достаточно крупных фрагментов баз знаний. \scnsourcecomment{К таким исходным текстам, в частности, относится и данная монография.} Очевидно при этом, что нумерация разделов баз знаний не может быть стабильна. Кроме того, при оформлении исходного текста крупного фрагмента базы знаний, обладающего достаточной целостностью и по научно-технической значимости достигшего уровня монографии или диссертации, желательно иметь собственную (свою, локальную) нумерацию разделов при сохранении их иерархической структуры. Это означает, что имеет смысл использовать только в рамках каждого номера разделов исходного вводимого текста и не должны использоваться в самой базе знаний. Таким образом, ссылаться на разделы базы следует по \uline{названию} разделов. При этом дополнительное указание номеров этих разделов, используемых в рамках заданного исходного текста возможно, но должно быть заключено в специальные ``скобки'', ограничивающие часть исходного текста, которая не учитывается при загрузке исходного текста в состав базы знаний (при трансляции во внутреннее представление базы знаний). Такого рода нетранслируемые комментарии к исходным текстам баз знаний, игнорируемые при их загрузке, могут потребоваться не только для указания номеров разделов. Все такие комментарии ограничиваются слева наклонной чертой и "звездочкой"{} ``/*'', и справа --  "звездочкой"{} и наклонной чертой ``*/''.
;Внешнее представление каждого структурно выделяемого фрагмента базы знаний ostis-системы за исключением sc-знаний нижнего уровня начинается с \uline{заголовка} (имени, названия) этого фрагмента. Указанный заголовок есть не что иное, как простой sc.s-идентификатор sc-узла, обозначающего представляемый фрагмент базы знаний (представляемое sc-знание). Рассматриваемый заголовок оформляется жирным курсивом с \uline{увеличенным расстоянием между символами}. При этом заголовки \uline{внешнего представления} разделов базы знаний, а также начал и завершений неатомарных разделов баз знаний имеют дополнительно \uline{увеличенный размер шрифта}. Заголовок внешнего представления sc-знания размещается с первого символа строчки, которой \uline{предшествует} строчка \uline{нетранслируемого комментария}, который 
\begin{scnitemize}
    \item для разделов базы знаний состоит из (1) слова ``Раздел'', (2) номера представляемого раздела базы знаний в рамках максимального представляемого фрагмента базы знаний (максимального раздела базы знаний) и далее (3) из ``звездочек'' \uline{до конца} строчки\char59
    \item для начал неатомарных разделов базы знаний состоит из (1) словосочетания ``Начало раздела'', (2) номера начинаемого неатомарного раздела базы знаний и (3) строки ``звездочек'' \uline{до середины} строчки\char59
    \item для завершений неатомарных разделов базы знаний состоит из (1) словосочетания ``Завершение раздела'', (2) номера завершаемого неатомарного раздела базы знаний и (3) строки, ``звездочек'' \uline{до конца} строчки\char59
    \item для сегментов атомарных разделов базы знаний, а также начал и завершений неатомарных разделов состоит из (1) словосочетания ``Первый сегмент'', либо ``Второй сегмент'', либо ``Третий сегмент'' и т.д. до ``Последний сегмент'', и (3) строки ``звездочек'' \uline{до середины} строчки.
\end{scnitemize}
После заголовка представляемого фрагмента базы знаний \uline{с новой строчки} размещается (1) sc.s-коннектор вида ``$\supset$='' (2) следом за ним на той же строчке левая фигурная скобка (открывающая фигурная скобка).\\
После этого для неструктурированных (т.е. не разбиваемых на сегменты) атомарных разделов, для неструктурированных начал и завершений неатомарных разделов, а также для сегментов приводится последовательность sc.n-предложений, представляющая указанный неструктурированный фрагмент базы знаний.\\
Для структуктурированного атомарного раздела, а также для структуктурированного начала или завершения неатомарного раздела после указанной левой фигурной скобки приводится заголовок первого сегмента с соответствующим нетранслируемым комментарием.\\
Нетрудно заметить, что внешнее представление каждого структурно выделяемого фрагмента базы знаний, не являющегося sc-знанием нижнего уровня иерархии, представляет собой \uline{одно} (!) sc.n-предложение, состоящее из (1) \uline{заголовка} представляемого фрагмента базы знаний, (2) sc.s-коннектора вида ``$\supset$='', (3) sc.s-выражения, ограничиваемого фигурными скобками и включающего в себя sc.n-текст, семантически эквивалентный sc-тексту представляемого фрагмента базы знаний. Если представляемый фрагмент базы знаний является неатомарным разделом базы знаний, \uline{то} в состав указанного sc.n-текста, ограниченного фигурными скобками, входят: (1) sc.n-предложение, представляющее \uline{начало} указанного неатомарного раздела базы знаний, (2) последовательность sc.n-предложений, представляющих \uline{все подразделы} указанного неатомарного раздела базы знаний (среди этих подразделов могут встречаться и неатомарные разделы базы знаний), (3) sc.n-предложение, представляющее \uline{завершение} указанного неатомарного раздела базы знаний. \uline{Если} представляемый фрагмент базы знаний является либо атомарным, но структурируемым разделом базы знаний, либо структурируемым началом или завершением неатомарного раздела, \uline{то} в состав указанного sc.n-текста, ограниченного фигурными скобками, входит последовательность sc.n-предложений, представляющих \uline{все сегменты} представляемого фрагмента базы знаний. Таким образом, внешнее представление сложноструктурированного фрагмента базы знаний представляет собой иерархическую (``матрешечную'') систему внешних представлений \uline{всех} (!) структурно выделяемых фрагментов базы знаний, входящих в состав представляемого сложноструктурированного фрагмента базы знаний.\\
Внешнее представление каждого раздела базы знаний, а также завершения каждого неатомарного раздела базы знаний начинается \uline{с новой страницы}. Соответственно, начало неатомарного раздела базы знаний и сегмент базы знаний могут начинаться \uline{не} с новой страницы.\\
Начало неатомарного раздела базы знаний должно включать в себя формальную спецификацию этого раздела базы знаний. В случае, \uline{если} указанное начало является структурируемым (т.е. разбивается на сегменты), \uline{то} спецификации указанного неатомарного раздела базы знаний должен быть полностью посвящен первый сегмент начала этого неатомарного раздела.\\
Атомарный раздел базы знаний должен содержать спецификацию самого себя. В случае, \uline{если} этот атомарный раздел структурируем (разбивается на сегменты), \uline{то} его спецификации должен быть полностью посвящен его первый сегмент.\\
Таким образом, формальная спецификация разделов базы знаний является важной информацией, входящей в состав самих этих разделов. В этом смысле разделы базы знаний ostis-систем имеют рефлексивный характер, как, впрочем, и любые другие формы sc-знаний.
;Названия (идентификаторы) начал и завершений неатомарных разделов строятся как sc-выражения, в которых указывается имя функции и sc-идентификатор одного из её аргументов.\\
Примеры:
\begin{scnitemize}
    \item начало раздела*(Вводный раздел Документации OSTIS)
    \item завершение раздела*(Вводный раздел Документации OSTIS)
    \item начало раздела*(Введение в описание внутреннего языка ostis-систем и близких к ему внешних языков, используемых для представления баз знаний)
\end{scnitemize}
Аналогичным образом строятся названия первых и последних сегментов атомарных разделов, а также начал и завершений неатомарных разделов.\\
Примеры:
\begin{scnitemize}
    \item первый сегмент* (Общие принципы оформления внутреннего и внешнего представления знаний ostis-систем)
    \item первый сегмент (начало раздела* (...))
    \item последний сегмент (начало раздела* (...))
\end{scnitemize}
;Внешнее представление каждого неструктурируемого структурно выделяемого фрагмента базы знаний (неструктурированного атомарного раздела базы знаний, сегмента базы знаний, неструктурированного начала неатомарного раздела базы знаний, неструктурированного завершения неатомарного раздела базы знаний) оформляется в виде sc.n-предложения, связывающего sc.s-идентификатор (имя, название) представляемого фрагмента базы знаний, оформленный в виде \uline{заголовка} внешнего текста этого фрагмента (признаком чего является жирный курсив с увеличенным расстоянием между символами), с sc.n-контуром, который фигурными скобками ограничивает sc.n-изображение (sc.n-визуализацию) представляемого фрагмента базы знаний.\\
В самом простом случае представляемым неструктурируемым структурно выделяемым фрагментом базы знаний является \uline{один} ея-файл ostis-системы.
;Внешний текст базы знаний может иметь самый разный объем и может касаться только \uline{одного раздела} или сегмента, а может включать в себя материалы \uline{нескольких разделов} или сегментов. Если представляемый (отображаемый) фрагмент базы знаний является sc-знанием нижнего уровня иерархии, т.е. частью \uline{неструктурированного} фрагмента базы знаний (например, частью сегмента базы знаний), то при его представлении необходимо указать, частью какого неструктурированного фрагмента базы знаний представляемый фрагмент является.\\
Для исходного текста sc-знания нижнего уровня эта информация необходима для того, чтобы знать, в какой фрагмент базы знаний требуется включить, ``погрузить'' данное вводимое sc-знание.
;Логическая последовательность sc-знаний в рамках sc-знания, в состав которого они \uline{непосредственно} входят, задается отношением ``порядок sc-знаний*''.\\
При этом при отображении структурированных sc-знаний порядок sc-знаний нижнего структурного уровня может \uline{явно} не отображаться.
;Виды выделений во внешнем представлении знаний ostis-системы:
\begin{scnitemize}
    \item с помощью шрифта
    \begin{scnitemizeii}
        \item вид шрифта (печатный, курсив)
        \item размер шрифта (стандартный, увеличенный)
        \item расстояние между символами (стандартное, увеличенное)
    \end{scnitemizeii}
    \item подчеркиванием
    \item с помощью символьных ограничителей (скобок различного вида)
\end{scnitemize}
Выделяемые объекты:
\begin{scnitemize}
    \item основные термины (имена, идентификаторы)
    \item специфицируемые файлы
    \begin{scnitemizeii}
        \item нетранслируемые в базу знаний файлы
        \item транслируемые в базу знаний файлы
    \end{scnitemizeii}
    \item нетранслируемые комментарии к внешнему тексту
    \item цитаты (и короткие, и длинные)
    \item ключевые фрагменты ея-текста
    \item метафорические термины
\end{scnitemize}
;На любой странице внешнего текста при распечатке делается ``разметка'' тонкими вертикальными линиями до середины строчек для четкой визуализации длины отступа от левого края страницы. Особенно это важно при переходе на новую страницу.} \scnsourcecomment{Завершили перечень правил оформления внешнего представления знаний ostis-системы} 
\filemodefalse

\scnheader{заголовок внешнего представления структурно выделяемого фрагмента базы знаний ostis-системы}
\scnsubset{sc.s-идентификатор структурно выделяемого фрагмента базы знаний ostis-системы}
\scnnote{Подчеркнем, что не каждое вхождение во внешний текст sc.s-идентификатора структурно выделяемого фрагмента базы знаний ostis-системы является \uline{заголовком} этого фрагмента в рамках его внешнего представления. Именно поэтому такие заголовки внешних текстов структурно выделяемых фрагментов базы знаний целесообразно ``синтаксически'' выделить (шрифтом и соответствующими правилами построения и размещения).}

\scnsuperset{заголовок внешнего представления раздела базы знаний ostis-системы}
\scnaddlevel{1}
    \scnsubset{жирный курсив увеличенного размера с увеличенным расстоянием между символами}
    \scnaddlevel{1}
        \scniselement{форма представления*}
    \scnaddlevel{-1}
    \scntext{размещение}{в начале листа с первого символа второй строчки сразу после строчки нетранслируемого комментария, указывающего номер раздела}
\scnaddlevel{-1}
\scnsuperset{заголовок внешнего представления начала неатомарного раздела базы знаний ostis-системы}
\scnsuperset{заголовок внешнего представления завершения неатомарного раздела базы знаний ostis-системы}
\scnsuperset{заголовок внешнего представления сегмента базы знаний ostis-системы}

\scnheader{максимальный раздел*}
\scnidtf{рассматриваемый (например, визуализируемый) фрагмент базы знаний ostis-системы, который имеет статус раздела базы знаний и в состав которого входит \uline{полный} (!) текст этого раздела, т.е. входят все подразделы указанного фрагмента базы знаний, все подразделы указанных разделов и т.д.}
\scnnote{Подчеркнем, что понятие максимального раздела* является понятием относительным. О максимальном разделе базы знаний мы имеем возможность говорить только тогда, когда точно определен визуализируемый или проектируемый фрагмент базы знаний и точно определено все множество разделов, входящих в состав указанного фрагмента базы знаний.}
\scnidtf{максимальный раздел \uline{для данного} (представляемого, отображаемого) фрагмента базы знаний ostis-системы*}
\scntext{определение}{С формальной точки зрения отношение ``быть максимальным разделом*'' является бинарным ориентированным отношением, каждая пара которого связывает некоторый файл ostis-системы, представляющий собой \uline{полный} \uline{внешний} (!) текст некоторого раздела базы знаний, с sc-узлом, обозначающим представляемый (отображаемый) раздел.}
\scnaddlevel{1}
    \scnnote{Следует четко отличать внешний текст (внешнее представление) раздела базы знаний и сам этот представляемый раздел, т.е его \uline{внутреннее} представление в памяти ostis-системы.}
\scnaddlevel{-1}
\scnsourcecommentpar{По отношению к тексту данной монографии максимальным разделом является раздел ``Документация Технологии OSTIS'', являющийся ключевым разделом базы знаний Метасистемы IMS.ostis}

\scnheader{аналоги*}
\scnhaselementset{максимальный раздел*;раздел-часть*;раздел-глава*;раздел-параграф*;раздел-пункт*;раздел-подпункт*;раздел-подподпункт*}
\scnaddlevel{1}
    \scnexplanation{Поскольку внешний текст представляемого сложноструктурированного раздела базы знаний может иметь большое число уровней иерархии для разделов, входящих в состав представляемого (максимального) раздела базы знаний, для структуризации \uline{внешнего} текста удобно разбить указанные разделы по уровням иерархии по отношению к максимальному представляемому разделу}
\scnaddlevel{-1}

\scnheader{раздел-часть*}
\scnidtf{раздел базы знаний, являющийся по отношению к заданному файлу внешнего текста непосредственным подразделом максимального (для указанного внешнего текста) раздела}

\scnheader{раздел-глава*}
\scnidtf{раздел базы знаний, являющийся по отношению к заданному файлу внешнего текста подразделом подраздела максимального (для указанного внешнего текста) раздела}
\scnaddlevel{1}
    \scnnote{Напомним, что отношение ``быть подразделом'' не является транзитивным.}
\scnaddlevel{-1}
\scnsourcecomment{Аналогичным образом определяются понятия раздела-параграфа*, раздела-пункта* и т.д.}

\scnheader{нетранслируемый комментарий к внешнему тексту}
\scnidtf{нетранслируемый комментарий к внешнему тексту отображаемого фрагмента базы знаний ostis-системы}
\scnexplanation{естественно-языковой текст, который ограничен слева наклонной чертой и звездочкой ``/*'' и справа -- звездочкой и наклонной чертой ``*/'' и который может находиться в \uline{любом} месте внешнего текста}
\scnnote{При загрузке и трансляции \uline{исходного} текста базы знаний ostis-системы все входящие в него нетранслируемые комментарии игнорируются -- и в том случае, если эти комментарии входят в состав изображения какого-либо файла ostis-системы (в частности, естественно-языкового файла), и в том случае, если эти комментарии входят в состав формального текста, транслируемого в SC-код.\\
При трансляции внутреннего текста (sc-текста) базы знаний ostis-системы во внешнее представление (например, в sc.n-текст) некоторые нетранслируемые комментарии могут автоматически генерироваться.\\
Например, нетранслируемые комментарии, предшествующие заголовкам внешнего представления структурно выделяемых фрагментов баз знаний ostis-систем (разделов, сегментов, начал неатомарных разделов, завершений неатомарных фрагментов), нетранслируемые комментарии к правым (закрывающим) фигурным и квадратным скобкам, если ограничиваемые этими скобками выражения размещаются на нескольких страницах, нетранслируемые комментарии к sc.s-идентификаторам, являющиеся ссылками на номера разделов и сегментов, где подробно специфицируется сущность, именуемая комментируемым sc.s-идентификатором.\\
Приведем несколько конкретных примеров:\\
\scnsourcecomment{представление данного файла будет продолжено на следующей странице}\\
\scnsourcecomment{продолжение представления файла}\\
\scnsourcecomment{представление данного sc-текста будет продолжено на следующей странице}\\
\scnsourcecomment{продолжение представления sc-текста}\\
\scnsourcecomment{sc-текст (файл), обозначаемый данным sc-узлом, представлен на следующей странице (на следующих страницах)}\\
Примерами нетранслируемых комментариев к закрывающим скобкам:
\scnsourcecomment{Завершили раздел i.j.k}\\
\scnsourcecomment{Завершили начальный сегмент раздела i.j.k}\\
\scnsourcecomment{Завершили примечание к i.j.k}}

\scnheader{нетранслируемый комментарий к внешнему тексту базы знаний}
\scnnote{Нетранслируемые комментарии к внешнему тексту базы знаний непосредственно в состав базы не входят. Указанными комментариями \uline{не следует злоупотреблять}, они должны касаться оформления внешнего текста, но не смысла представляемой базы знаний.\\
Содержательные комментарии к базе знаний должны оформляться в виде файлов, входящих в состав базы знаний.}

\scnsegmentheader{Итоговый сегмент раздела \dq{}\currentname\dq{}}
\scnstartsubstruct

\scnheader{следует отличать*}
\scnhaselementset{sc-структура;sc-текст;sc-знание}

\scnheader{следует отличать*}
\scnhaselementset{sc.s-идентификатор структурно выделяемого sc-знания ostis-системы;заголовок внешнего представления структурно выделяемого sc-знания ostis-системы}
\scnhaselementset{заголовок внешнего представления раздела базы знаний\\
    \scnaddlevel{1}
    \scnaddhind{-1}
    \scnsubset{жирный курсив увеличенного размера с увеличенным расстоянием между символами с начала страницы и с начала строчки}
    \scnaddlevel{-1}
    \bigskip
;заголовок внешнего представления сегмента базы знаний\\
    \scnaddlevel{1}
    \scnsubset{жирный курсив стандартного размера с увеличенным расстоянием между символами с начала строчки}
    \scnaddlevel{-1}
;заголовок внешнего представления начала неатомарного раздела базы знаний\\
    \scnaddlevel{1}
    \scnsubset{жирный курсив стандартного размера с увеличенным расстоянием между символами с начала строчки}
    \scnaddlevel{-1}
;заголовок внешнего представления завершения неатомарного радела базы знаний\\
    \scnaddlevel{1}
    \scnsubset{жирный курсив увеличенного размера с увеличенным расстоянием между строчками с начала страницы и с начала строчки}
    \scnaddlevel{-1}}
\bigskip
\scnresetlevel
\scnhaselementset{раздел базы знаний ostis-системы;внешнее представление раздела базы знаний ostis-системы\\
    \scnaddlevel{1}
    \scnsubset{формальный текст}
    \scnaddlevel{-1}
;ея-текст раздела базы знаний ostis-системы\\
    \scnaddlevel{1}
    \scnidtf{ея-текст, семантически эквивалентный разделу базы знаний ostis-системы}
    \scnaddlevel{-1}}
\bigskip
\scnhaselementset{ограничитель \uline{транслируемого} формального фрагмента \uline{ея-текста};ограничитель в рамках sc.n-текста транслируемого sc.g-текста, не являющегося внутренней частью sc.g-контура}
\bigskip
\scnhaselementset{sc-текст\\
    \scnaddlevel{1}
    \scnidtf{связное множество sc-элементов, удовлетворяющее синтаксическим правилам SC-кода}
    \scnidtf{правильно построенная конструкция SC-кода}
    \scnidtf{фрагмент базы знаний ostis-системы}
    \scnidtf{фрагмент внутреннего представления базы знаний ostis-системы}
    \scnidtf{sc-текст}
    \scnsuperset{раздел базы знаний}
    \scnsuperset{сегмент базы знаний}
    \scnsuperset{высказывание базы знаний}
    \scnaddlevel{-1}
;внешнее представление sc-структуры\\
    \scnaddlevel{1}
    \scnsuperset{sc.g-текст}
    \scnsuperset{sc.s-текст}
    \scnsuperset{sc.n-текст}
        \scnaddlevel{1}
            \scnsuperset{sc.n-текст раздела базы знаний}
            \scnsuperset{sc.n-текст сегмента базы знаний}
            \scnsuperset{sc.n-текст высказывания базы знаний}
        \scnaddlevel{-1}
    \scnaddlevel{-1}}

\scnendstruct \scninlinesourcecommentpar{Завершили последний сегмент Раздела}

\scnendstruct \scnendcurrentsectioncomment

\end{SCn}
