\scnsegmentheader{Понятие технологии OSTIS}

\scnstartsubstruct

\scnheader{Технология OSTIS}
\scnidtf{Комплекс (семейство) технологий, обеспечивающих проектирование, производство, эксплуатацию и реинжиниринг интеллектуальных \textit{компьютерных систем (ostis-систем)}, предназначенных для автоматизации самых различных видов человеческой деятельности и в основе которых лежит смысловое представление и онтологическая систематизация знаний, а также агентно-ориентированная обработка знаний}
\scnidtf{Open Semantic Technology for Intelligent Systems}
\scnaddlevel{1}
\scntext{сокращение}{OSTIS}
\scnaddlevel{-1}
\scnidtf{Семейство (комплекс) \textit{ostis-технологий}}
\scnidtf{Комплексная открытая семантическая технология проектирования, производства, эксплуатации и реинжиниринга гибридных, семантически совместимых, активных и договороспособных \textit{интеллектуальных компьютерных систем}}
\scnheader{Технология OSTIS}
\scnrelfromset{принципы, лежащие в основе}{
\scnfileitem{Ориентация на разработку \textit{интеллектуальных компьютерных систем}, имеющих высокий уровень \textit{интеллекта} и, в частности, высокий уровень \textit{социализации}. Указанные системы, разработанные по \textit{Технологии OSTIS}, будем называть \textbf{ostis-системами}}
;\scnfileitem{Ориентация на \uline{комплексную} автоматизацию всех видов и областей \textit{человеческой деятельности} путем создания сети взаимодействующих и координирующих свою деятельность \textit{ostis-систем}. Указанную сеть \textit{ostis-систем} вместе с их пользователями будем называть \textbf{\textit{Экосистемой OSTIS}}}
;\scnfileitem{Поддержка перманентной эволюции \textit{ostis-систем} в ходе их эксплуатации.}
;\scnfileitem{\textit{Технология OSTIS} реализуется в виде сети \textit{ostis-систем}, которая является частью \textit{Экосистемы OSTIS}.
Ключевой \textit{ostis-системой} указанной сети является \textbf{Метасистема IMS.ostis} (Intelligent MetaSystem), реализующая \textbf{Ядро Технологии OSTIS}, которое включает в себя базовые (предметно независимые) методы и средства проектирования и производства \textit{ostis-систем} с интеграцией в их состав типовых встроенных подсистем поддержки эксплуатации и реинжиниринга \textit{ostis-систем}. Остальные \textit{ostis-системы}, входящие в состав рассматриваемой сети, реализуют различные специализированные \textit{ostis-технологии} проектирования различных классов \textit{ostis-систем}, обеспечивающих автоматизацию любых областей и \textit{видов человеческой деятельности}, кроме \textit{проектирования ostis-систем}.}
;\scnfileitem{Конвергенция и интеграция на основе смыслового представления знаний всевозможных научных направлений \textit{Искусственного интеллекта} (в частности, всевозможных базовых знаний и навыков решения интеллектуальных задач) в рамках \textit{Общей формальной семантической теории ostis-систем}.}
;\scnfileitem{Ориентация на разработку компьютеров нового поколения, обеспечивающих эффективную (в том числе, производительную) интерпретацию логико-семантических моделей \textit{ostis-систем}, представленных базами знаний этих систем и имеющих смысловое представление.}}

\scnsegmentheader{Понятие ostis-системы}

\scnstartsubstruct

\scnheader{ostis-система}
\scnidtf{\textit{интеллектуальная компьютерная система}, спроектированная и реализованная по требованиям и стандартам \textit{Технологии OSTIS}, которые задокументированы в \textit{Общей теории ostis-систем}}

\scnheader{ostis-система}
\scnidtf{интеллектуальная компьютерная система, построенная в соответствии с принципами и требованиями Технологии OSTIS}
\scnidtf{Множество ostis-систем различного назначения}
\scnaddlevel{1}
\scniselement{имя собственное}
\scnaddlevel{-1}
\scnidtf{Множество всевозможных интеллектуальных компьютерных систем, построенных по Технологии OSTIS}

\scnheader{ostis-система}
\scnsubset{интеллектуальная компьютерная система}
\scnidtf{\textit{интеллектуальная компьютерная система}, которая построена в соответствии с требованиями и стандартами \textit{Технологии OSTIS}, что обеспечивает существенное развитие целого ряда \textit{свойств} (способностей) этой \textit{компьютерной системы}, позволяющих значительно повысить \textit{уровень интеллекта} этой системы (и, прежде всего, ее \textit{уровень обучаемости} и \textit{уровень социализации})} 

\scnauthorcomment{Добавить классификацию из пояснения}

\scnsubdividing{индивидуальная ostis-система;коллективная ostis-система\\
\scnaddlevel{1}
    \scnsubdividing{простой коллектив ostis-систем;иерархический коллектив ostis-систем}   
\scnaddlevel{-1}
}

\scnheader{ostis-система}
\scnexplanation{интеллектуальная компьютерная система, разработанная, разрабатываемая или совершенствуемая по технологии OSTIS}
\scnnote{Когда речь идет о таком компоненте технологии OSTIS, как модели ostis-систем, фактически имеется в виду теория ostis-систем, включающая в себя строгое формальное уточнение того, как устроена ostis-система, какова ее архитектура, принципы организации памяти, принципы организации представления информации, принципы организации интерфейса с внешней средой (в том числе, с пользователями)}

\scnheader{ostis-система}
\scnrelfromset{принципы, лежащие в основе}{
\scnfileitem{Информация, хранимая в памяти \textit{ostis-системы}, имеет смысловое представление.}
;\scnfileitem{В основе организации решения задач в памяти \textit{ostis-системы} лежит \textit{агентно-ориентированная модель обработки информации}, управляемая ситуациями и событиями, возникающими в обрабатываемой информации (точнее, в обрабатываемой базе знаний).}
;\scnfileitem{Унификация базового набора (базовой системы) используемых понятий, что является основой обеспечения \textit{семантической совместимости} всех \textit{ostis-систем}.}
;\scnfileitem{В основе структуризации информации (базы знаний), хранимой в памяти \textit{ostis-системы}, лежит иерархическая система \textit{предметных областей} и соответствующих им \textit{формальных онтологий}.}
;\scnfileitem{Способность к пониманию (к семантическому погружению, к семантической интеграции) новых приобретаемых знаний (и, в том числе, новых навыков) в состав текущего состояния \textit{базы знаний}.}
;\scnfileitem{Способность к \textit{семантической конвергенции} (к обнаружению сходств) новых приобретаемых знаний (и, в частности, навыков) со знаниями, входящими в состав текущего состояния базы знаний \textit{ostis-системы}.}
;\scnfileitem{Способность \textit{ostis-системы} поддерживать высокий уровень своей \textit{семантической совместимости} (высокий уровень взаимопонимания) с другими \textit{ostis-системами}.}
;\scnfileitem{Способность ostis-системы согласовывать, координировать свою деятельность с другими \textit{ostis-системами}.}
;\scnfileitem{\scnauthorcomment{статья на OSTIS-2020}}
;\scnfileitem{\scnauthorcomment{статья на OSTIS-2020}}
;\scnfileitem{\scnauthorcomment{статья на OSTIS-2020}}
;\scnfileitem{\scnauthorcomment{статья на OSTIS-2020}}}
\scntext{следовательно}{Перечисленные свойства \textit{ostis-систем} свидетельствуют о том, что они имеют существенно более высокий \textit{уровень интеллекта} и, в частности, более высокий \textit{уровень социализации} по сравнению с современными \textit{интеллектуальными компьютерными системами}. \scnauthorcomment{См. начало Раздела 1.1}}

\scnheader{ostis-система}
\scnrelfromset{принципы, лежащие в основе}{
\scnfileitem{смысловое представление информации в памяти компьютерных систем, направленное на устранение недостатков современных компьютерных систем и технологий путем повышения уровня интеллектуальности компьютерных систем}
;\scnfileitem{децентрализация управления решателем задач
\begin{itemize}
	\item внутренняя МАС
	\item внешняя МАС
\end{itemize}}
;\scnfileitem{интеграция различных видов знаний}
;\scnfileitem{интеграция различных моделей решателей задач}
;\scnfileitem{ориентация на компьютеры нового поколения}
;\scnfileitem{обеспечение семантической совместимости компьютерных систем}
;\scnfileitem{обеспечение поддержания семантической совместимости компьютерных систем в ходе эволюции}
;\scnfileitem{способность к координации деятельности}}

\scnheader{ostis-система}
\scnrelfromset{принципы, лежащие в основе}{
\scnfileitem{Память ostis-системы является графодинамической (т.е. нелинейной (графовой) и структурно перестраиваемой). Переработка информации в памяти ostis-системы сводится не столько к изменению состояния элементов памяти (это происходит только при изменении синтаксического типа элементов и при изменении содержимого тех элементов, которые обозначают файлы), сколько к изменению \uline{конфигурации связей} между ними.}
;\scnfileitem{Хранение информации в памяти ostis-системы ориентируется на \uline{смысловое} представление информации – без синонимов, омонимов знаков и без семантической эквивалентности информационных конструкций.}
;\scnfileitem{С точки зрения архитектуры ostis-система представляет собой \uline{иерархическую} многоагентную систему с общедоступной памятью (т.е. с памятью, общедоступной \uline{всем} агентам ostis-системы). 
Заметим при этом, что общая память большинства исследуемых в настоящее время многоагентных систем является не общедоступной, а распределенной, т.е. представляет собой абстрактное (виртуальное) объединение, в состав которого входит память каждого агента многоагентной системы. Координация деятельности агентов ostis-системы при выполнении сложных \textit{действий в памяти} ostis-системы реализуется также через \textit{память ostis-системы} с помощью хранимых в памяти \textit{методов} решения различных классов задач, а также с помощью хранимых в памяти \textit{планов} решения конкретных задач.
На основании этого можно строить неограниченную иерархическую систему агентов ostis-системы – от элементарных агентов, обеспечивающих выполнение базовых действий в памяти ostis-системы, до неэлементарных агентов, представляющих собой коллективы (группы) элементарных и/или неэлементарных агентов, обеспечивающих решение различных типовых задач с помощью соответствующих методов и планов.}
;\scnfileitem{Организация выполнения \textit{ostis-системой действий во внешней среде} осуществляется следующим образом:
\begin{scnitemize}
	\item Выделяются классы \textit{элементарных действий во внешней среде}, для реализации каждого из которых вводятся \textit{эффекторные агенты} ostis-системы.
	\item Координация деятельности \textit{эффекторных агентов} ostis-системы при выполнении \textit{сложных действий во внешней среде} осуществляется через \textit{память ostis-системы} с помощью хранимых в памяти \textit{методов} и \textit{планов} решения различных задач во \textit{внешней среде}, а также с помощью \textit{рецепторных агентов} ostis-системы, обеспечивающих обратную связь и, соответственно, сенсомоторную координацию.
\end{scnitemize}}}

\scnheader{ostis-система}
\scnrelfromset{принципы, лежащие в основе}{
\scnfileitem{Способность понимать друг друга, а также любого своего пользователя
\scnaddlevel{-2}
\scnidtf {Совместимость используемых понятий (по терминам и по денотационной семантике)}
\scnidtf {Семантическая совместимость}
\scnaddlevel{2}}
;\scnfileitem{Способность поддерживать взаимопонимание в процессе индивидуальной эволюции, приводящей к расширению и/или корректировке системы используемых понятий}
;\scnfileitem{Способность координировать свою деятельность с другими системами при решении задач, которые усилиями одной (индивидуальной) интеллектуальной компьютерной системы не могут быть решены либо принципиально, либо за разумное время}}

\scnheader{ostis-система}
\scnrelfromset{принципы, лежащие в основе}{
\scnfileitem{Высокая степень индивидуальной обучаемости интеллектуальных компьютерных систем
\begin{itemize}
	\item гибкости
	\item стратифицированности
	\item рефлексивности
	\item универсальность средств представления и образования знаний
\end{itemize}}
;\scnfileitem{Высокая степень семантической совместимости и, как следствие, коллективной обучаемости интеллектуальных компьютерных систем
\begin{itemize}
	\item семантической совместимости
\end{itemize}}
;\scnfileitem{Основа для автоматизации рынка знаний}}

\scnmakeset{память*;ostis-система}
\scnrelfrom{сужение второго домена заданного отношения для заданного первого домена}{память ostis-системы}
\scnaddlevel{1}
\scnsubset{смысловая память}
\scnaddlevel{-1}


\scnmakeset{информация, хранимая в памяти кибернетической системы*;  ostis-система}
\scnrelfrom{сужение второго домена заданного отношения для заданного первого домена}{база знаний ostis-системы}
\scnaddlevel{1}
\scnsubset{смысловое представление информации}
\scnaddlevel{-1}


\scnheader{память ostis-системы}
\scnsubset{смысловая память}

\scnheader{информация, хранимая в памяти ostis-системы}
\scnsubset{смысловое представление информации}

\scnheader{решатель задач ostis-системы }
\scnsubset{агентно-ориентированная модель обработки информации в памяти}