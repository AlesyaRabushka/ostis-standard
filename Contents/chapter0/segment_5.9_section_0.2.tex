\bigskip
\scnfragmentcaption

\scnheader{автоматизация человеческой деятельности}
\scnrelfromlist{вопрос}{
    \scnfileitem{В чем заключаются проблемы комплексной автоматизации человеческой деятельности}
    ;\scnfileitem{Как автоматизировать участие человеческая одновременно в нескольких разных действиях (разных областях деятельности), принадлежащих в общем случае разным видам деятельности}}

\scnheader{Экосистема OSTIS}
\scnnote{Во многом разработка принципов организации взаимодействия интеллектуальных компьютерных систем (ostis-систем) и людей, входящих в состав \textit{Экосистемы OSTIS} должна опираться на анализ того, как взаимодействие осуществляется между людьми, когда основные проблемы возникают из-за:
\begin{scnitemize}
    \item отсутствия взаимопонимания (семантической совместимости),
    \item противоречий между целями различных субъектов,
    \item имитации целенаправленных действий,
    \item нарушений каких-либо соглашений, договоренностей и даже законов (правил поведения и обязанностей).
\end{scnitemize}
Общая (общедоступная) \textit{база знаний} всей \textit{Экосистемы OSTIS}, а также корпоративная \textit{база знаний} каждого \textit{ostis-сообщества}, входящего в состав \textit{Экосистемы OSTIS}, является распределенной, но при этом обязательно целостной. Она поддерживается группой специальных \textit{ostis-систем}, являющихся \textit{порталами знаний} по самым различным областям. Для \textit{Технологии OSTIS} роль такого \textit{портала знаний} выполняет \textit{Метасистема IMS.ostis}. \textit{Экосистема OSTIS} представляет собой многоагентную социотехническую систему, в которой каждая \textit{индивидуальная ostis-система}, входящая в состав \textit{Экосистемы OSTIS}, каждый пользователь указанных \textit{ostis-систем}, а также каждое \textit{ostis-сообщество}, входящее в Экосистему, является её самостоятельным \textit{субъектом*} (когнитивным агентом). При этом каждый субъект \textit{Экосистемы OSTIS} должен соблюдать определенные правила, обеспечивающие качественную (эффективную) эксплуатацию и эволюцию \textit{Экосистемы OSTIS}.}
\scntext{резюме}{Сама идея комплексной автоматизации всех видов человеческой деятельности предполагает необходимость:
\begin{scnitemize}
    \item разработки достаточно детальных формальных теорий всех видов человеческой деятельности, причем, теорий, доведенных до уровня разделов баз знаний соответствующих корпоративных компьютерных систем -- это, фактически, строгое описание стандартов различных видов человеческой деятельности, доведенное до такого уровня, чтобы соответствующая корпоративная система "\underline{понимала}"{} , в какой деятельности она участвует, и могла быть активным и полноценным субъектом (участником) этой деятельности;
    \item серьезного отношения и научного подхода к формализации различных видов человеческой деятельности, к разработке самых различных стандартов;
    \item глубокой конвергенции различных областей (разделов) человеческой деятельности и, соответственно, различных видов человеческой деятельности, осуществляемой в условиях достигнутого уровня автоматизации этой деятельности. Это предполагает необходимость рассмотрения каждого вида человеческой деятельности в контексте \textbf{\textit{Общей теории человеческой деятельности}} в условиях \underline{текущего} состояния уровня автоматизации этой деятельности;
    \item обеспечения высоких темпов эволюции и, следовательно, высокого уровня \underline{гибкости} \textit{Общей теории человеческой деятельности} и теорий (стандартов) каждого \textit{вида деятельности} в силу их большой зависимости от текущего уровня автоматизации;
    \item автоматизации взаимодействия субъектов не только внутри каждой области (раздела) человеческой деятельности, но и между этими областями (разделами), что предполагает автоматизацию \underline{представительства} каждой области (раздела) человеческой деятельности во множестве всех таких областей;
    \item понимания того, что эффективность человеческой деятельности во многом определяется скоординированностью, адекватностью, грамотностью поведения каждого субъекта. Поэтому автоматизация человеческой деятельности должна быть направлена на более глубокую координацию этой деятельности на основе учета смысла и целей этой деятельности. А это "превращает"{} средства автоматизации в полноценных субъектов коллективной деятельности
\end{scnitemize}}

\scnheader{автоматизация человеческой деятельности}
\scnnote{Рассмотрение комплексной автоматизации человеческой \textit{деятельности в области Искусственного интеллекта} естественным образом можно расширить (обобщить) до рассмотрения комплексной автоматизации человеческой деятельности в целом.}
\scntext{проблемы текущего состояния}{Для того, чтобы обеспечить качественную автоматизацию любой \textit{области человеческой деятельности} с помощью \textit{интеллектуальных компьютерных систем}, необходимо построить \textit{формальную модель} этой области деятельности и довести эту модель до такого уровня формализации, чтобы она могла стать частью \textit{базы знаний интеллектуальной компьютерной системы}, используемой для автоматизации указанной \textit{области человеческой деятельности}. Очевидно, что, чем субъекты, участвующие в какой-либо коллективной деятельности, (люди и интеллектуальные компьютерные системы) лучше понимают суть, цели, критерии качества указанной коллективной деятельности, тем выше качество выполнения этой деятельности. В состав формальной модели автоматизируемой области человеческой деятельности входит спецификация:
\begin{scnitemize}
    \item объектов деятельности;
    \item среды деятельности;
    \item инструментов (инструментальных средств) деятельности;
    \item субъектов деятельности;
    \item текущего состояния деятельности (как процесса) -- в том числе, спецификация действий (целей, задач), выполняемых в текущий момент;
    \item формулировка различного рода учитываемых закономерностей -- в том числе, правил поведения субъектов деятельности;
    \item спецификация всех используемых субъектами деятельности методов выполнения сложных действий (решения задач).
\end{scnitemize}
Примерами автоматизируемых областей человеческой деятельности являются:
\begin{scnitemize}
    \item процесс взаимодействия "умного"{} дома с его жильцами и посетителями;
    \item процесс взаимодействия "умного"{} предприятия, выпускающего определенного вида продукцию, с его сотрудниками;
    \item процесс взаимодействия студентов и преподавателей в рамках "умной"{} кафедры, осуществляющей подготовку молодых специалистов по какой-либо инженерной специальности;
    \item процесс взаимодействия постояльцев, посетителей и сотрудников "умного"{} отеля;
    \item процесс взаимодействия посетителей и сотрудников "умного"{} музея;
    \item процесс взаимодействия пациентов и медицинского персонала "умной поликлиники"{}, "умной"{} больницы;
    \item процесс взаимодействия граждан и чиновников в рамках "умной"{} администрации некоторого региона;
    \item процесс взаимодействия жителей и гостей в рамках "умного"{} города;
    \item и т. д.
\end{scnitemize}
Но для комплексной автоматизации человеческой деятельности в целом (Объединенной человеческой деятельности) автоматизации отдельных областей человеческой деятельности явно не достаточно, поскольку тесные связи между различными областями человеческой деятельности требуют автоматизации не только деятельности внутри каждой из этих областей, но и внешней деятельности, обусловленной необходимостью взаимодействия между различными областями деятельности, например, в рамках более крупных областей деятельности. Так, например, каждое предприятие взаимодействует со своими поставщиками и потребителями. Очевидно, что автоматизация такой внешней деятельности и, тем более, автоматизация с использованием интеллектуальных компьютерных систем существенно упрощается, если будут совпадать (будут унифицированы) принципы, лежащие в основе автоматизации каждой области деятельности, а также принципы автоматизации крупных областей деятельности, в состав которых входит некоторое количество более "мелких"{} областей человеческой деятельности.

Таким образом, для комплексной автоматизации человеческой деятельности в целом с применением интеллектуальных компьютерных систем и для обеспечения эффективной интеграции различных областей человеческой деятельности необходима разработка \textbf{\textit{Общей формальной теории человеческой деятельности}}, которая объединила бы формальные модели всевозможных областей человеческой деятельности, а также включила бы:
\begin{scnitemize}
    \item формальные теории различных \textit{видов человеческой деятельности}. Поскольку каждый \textbf{\textit{вид человеческой деятельности}} -- это класс однотипных \textit{областей человеческой деятельности}, формальная теория каждого вида человеческой деятельности -- это формальное представление \underline{стандарта} соответствующего класса областей человеческой деятельности. Так, например, можно говорить о формальной модели конкретного предприятия рецептурного производства (например, предприятие "Савушкин продукт"{} , выпускающего молочную продукцию), но можно говорить и о формальной теории всего класса предприятий рецептурного производства -- о формальном представлении стандарта ISA-88. Формальная теория каждого вида человеческой деятельности включает в себя формальное описание технологии, обеспечивающей осуществление каждой области (фрагмента) человеческой деятельности, принадлежащей указанному виду деятельности. В описание технологии входит описание используемых методов, средств и основных объектов и субъектов деятельности;
    \item четкую иерархическую декомпозицию \textit{Объединенной человеческой деятельности} по нескольким признакам. Основными признаками такой декомпозиции являются региональный признак и целевая направленность деятельности. По региональному признаку на высшем уровне иерархии выделяются такие области человеческой деятельности, как Деятельность Франции, Деятельность Германии и далее деятельность всех стран. По признаку целевой направленности на высшем уровне иерархии выделяются: Научно-исследовательская деятельность человечества, Проектная деятельность человечества, Производственная деятельность человечества, Образовательная деятельность человечества, Здравоохранительная деятельность человечества, Природоохранная деятельность человечества, Административная деятельность человечества и др. Дальнейшая декомпозиция областей человеческой деятельности по признаку целевой направленности выделяет такие области деятельности, как:
    \begin{scnitemizeii}
    	\item \textit{Научно-исследовательская деятельность человечества в области Математики}
    	\item \textit{Научно-исследовательская деятельность человечества в области Лингвистики}
    	\item и др.
    \end{scnitemizeii}    
    Заметим при этом, что, в отличие от "чисто"{} научных дисциплин, дисциплины научно-технического типа (например, дисциплина \textit{Искусственный Интеллект}) представляют собой симбиоз фрагментов (областей) деятельности, принадлежащих разным видам деятельности:
    \begin{scnitemizeii}
        \item научно-исследовательской деятельности;
        \item деятельности по разработке технологии проектирования (CAD);
        \item деятельности по разработке технологии производства (CAM);
        \item проектная деятельность;
        \item производство спроектированного объекта;
        \item образовательной деятельности;
        \item бизнес-деятельности.
    \end{scnitemizeii}
    Кроме указанных областей человеческой деятельности выделяются области, соответствующие различным сочетаниям значений указанных признаков декомпозиции областей человеческой деятельности. Примерами таких областей являются: Научно-исследовательская деятельность Франции, Образовательная деятельность Германии. Подчеркнем то, что количество областей человеческой деятельности, выделенных в результате указанной иерархической декомпозиции \textit{Объединенной человеческой деятельности}, является, хоть и не очень большим, но конечным в каждый момент времени.
\end{scnitemize}\bigskip

При этом Общая формальная теория человеческой деятельности должна быть ориентирована:
\begin{scnitemize}
    \item на унификацию формального описания самых различных технологий для самых различных областей человеческой деятельности;
    \item на унификацию формального описания всевозможных видов человеческой деятельности;
    \item на унификацию формального описания связей между различными областями и видами человеческой деятельности, различными субъектами деятельности, объектами, средствами (инструментами);
    \item на глубокую конвергенцию всех видов человеческой деятельности, областей человеческой деятельности, используемых методов.
\end{scnitemize}}
\scnrelfromvector{что делать}{\scnfileitem{Необходим переход от локальной автоматизации различных областей и видов человеческой деятельности путем независимой друг от друга разработки систем автоматизации бизнес-процессов даже близких по виду деятельности предприятий к \underline{комплексной автоматизации человеческой деятельности} в целом прежде всего для обеспечения совместимости различных областей деятельности и исключения ужасающего и никому не нужного дублирования (многообразия форм) автоматизации аналогичных бизнес-процессов}
;\scnfileitem{Все многообразие человеческой деятельности необходимо четко стратифицировать, доведя эту стратификацию до строгого формального представления}
;\scnfileitem{Необходимо
\begin{scnitemize}
    \item четко выделить все виды человеческой деятельности, соответствующие текущему уровню развития человеческого общества;
    \item построить четкую иерархию этих видов на основании отношения, связывающего виды человеческой деятельности с их подвидами;
    \item унифицировать человеческую деятельность в рамках каждого выделенного вида, разработав соответствующие стандарты, для каждого из которых построить четкую систему используемых понятий;
    \item довести указанные стандарты до такого уровня формализации, чтобы они стали частью базы знаний интеллектуальной системы автоматизации соответствующего вида человеческой деятельности.
\end{scnitemize}}
;\scnfileitem{Необходимо обеспечить конвергенцию, семантическую совместимость и глубокую интеграцию различных видов и областей человеческой деятельности путем:
\begin{scnitemize}
    \item согласования систем понятий, соответствующих стандартам разных видов человеческой деятельности, и особенно согласования систем понятий между стандартами видов и подвидов человеческой деятельности;
    \item представления стандарта каждого вида человеческой деятельности в виде формальной онтологии;
    \item построения такой иерархической системы формальных онтологий, соответствующих всевозможным видам человеческой деятельности, в которой обеспечивалась бы конвергенция и \underline{семантическая совместимость} онтологий, входящих в эту систему, а также \underline{наследование свойств} от онтологии каждого вида человеческой деятельности к онтологии каждого подвида этого вида человеческой деятельности.
\end{scnitemize}}}
\scnaddlevel{1}
\scntext{следовательно}{Таким образом, в целях повышения эффективности автоматизации человеческой деятельности и, в первую очередь, в целях существенного снижения трудозатрат на такую автоматизацию необходимо с точки зрения общей теории систем фундаментально переосмыслить современную организацию человеческой деятельности, поскольку автоматизация беспорядка приводит к ещё большему беспорядку. На этом пути имеется только одно препятствие -- противодействие лени с высоким уровнем эгоизма, которым современный беспорядок организации человеческой деятельности выгоден.}
\scnaddlevel{-1}

%\scnheader{Математика}
%\scniselement{научная дисциплина}
%\scnidtf{Научно-исследовательская деятельность человечества в области математики}
%\scnheader{Лингвистика}
%\scniselement{научная дисциплина}
%\scnidtf{Научно-исследовательская деятельность человечества в области лингвистики}

\scnheader{следует отличать}
\scnhaselementset{вид человеческой деятельности\\
\scnaddlevel{1}
\scnhaselementlist{пример}{научно-исследовательская деятельность;проектирование\\
    \scnaddlevel{1}
    \scnidtf{проектная деятельность}
    \scnsuperset{проектирование интеллектуальной компьютерной системы\\
        \scnaddlevel{1}
        \scnsuperset{проектирование ostis-системы}
        \scnaddlevel{-1}
    \scnaddlevel{-1}}}
\scnaddlevel{-1}
;область человеческой деятельности\\
\scnaddlevel{1}
\scnhaselementlist{пример}{Научно-исследовательская деятельность в области Искусственного интеллекта\\
    \scnaddlevel{1}
    \scniselement{научно-исследовательская деятельность}
    \scnrelfrom{часть}{Научно-исследовательская деятельность РАИИ}
    \scnaddlevel{-1}
    ;Проектирование Метасистемы IMS.ostis\\
        \scnaddlevel{1}
        \scniselement{проектирование ostis-системы}
        \scnaddlevel{-1}}
\scnaddlevel{-1}
;подвид человеческой деятельности*\\
    \scnaddlevel{1}
    \scnsubset{включение*}
        \scnaddlevel{1}
        \scnidtf{подмножество*}
        \scnaddlevel{-1}
    \scnaddlevel{-1}
;подобласть человеческой деятельности*\\
    \scnaddlevel{1}
    \scnsubset{часть*}
    \scnaddlevel{-1}}

    
    
\scnheader{информационная технология}
\scnexplanation{Множество технологий, связанных с проектированием и производством компьютерных систем и их компонентов, с эксплуатацией компьютерных систем, а также с их использованием в качестве инструмента в составе самых различных технологий. В рамках различных информационных технологий компьютерные системы рассматриваются как инструментальные средства, как вспомогательные субъекты, обеспечивающие автоматизацию соответствующих видов деятельности. Но в некоторых информационных технологиях компьютерные системы являются также и \underline{объектами} автоматизируемых видов деятельности. Примерами таких технологий являются:
\begin{scnitemize}
    \item технология проектирования компьютерных систем;
    \item технология реализации (сборки) компьютерных систем;
    \item технология обновления компьютерных систем.
\end{scnitemize}}
\scnhaselement{Комплекс современных информационных технологий}
\scnhaselement{Комплекс современных технологий искусственного интеллекта}
\scnhaselement{Технология OSTIS}

\scnheader{автоматизация человеческой деятельности}
\scnidtf{человеческая деятельность, направленная на повышения уровня автоматизации человеческой деятельности, а также на повышение качества (в том числе, уровня интеллекта) человеческого общества как многоагентной кибернетической системы}
\scnsubset{вид человеческой деятельности}
\scnnote{Важнейшим этапом автоматизации человеческой деятельности в перспективе должен стать переход к существенно более высокому уровню \textit{интеллекта человеческого общества} как целостной кибернетической системы путем преобразования современного человеческого общества в сообщество взаимодействующих между собой людей и интеллектуальных компьютерных систем. Такое сообщество иногда называют smart-обществом, обществом 5.0.
	
Особо подчеркнем то, что переход к такому интеллектуальному обществу требует существенного переосмысления современной организации различных видов человеческой деятельности. Прежде всего, следует подчеркнуть, что эффективность (коэффициент полезного действия) современной организации человеческой деятельности в целом ужасающе низка, а, как известно, автоматизация беспорядка (даже с помощью интеллектуальных компьютерных систем) приводит к ещё большему беспорядку.}

\scnheader{вид человеческой деятельности}
\scnidtf{Множество всевозможных видов человеческой деятельности}
\scnrelfrom{разбиение}{Разбиение Множества видов человеческой деятельности по степени их автоматизируемости}
\scnaddlevel{1}
\scneqtoset{вид человеческой деятельности, который принципиально может быть автоматизирован полностью\\
    \scnaddlevel{1}
    \scnidtf{Множество полностью автоматизируемых видов человеческой деятельности}
    \scnaddlevel{-1}
;вид человеческой деятельности, который может быть автоматизирован только частично\\
    \scnaddlevel{1}
    \scnidtf{Множество частично автоматизируемых видов человеческой деятельности}
    \scnaddlevel{-1}
;вид человеческой деятельности, который принципиально никак не может быть автоматизирован\\
    \scnaddlevel{1}
    \scnidtf{Множество неавтоматизируемых видов человеческой деятельности, которые могут быть выполнены только "вручную"\ (точнее самими людьми с возможным использованием каких-либо "пассивных"\ инструментов -- топора, лопаты и т. п.)}
    \scnaddlevel{-1}}
\scnaddlevel{-1}

\scnheader{вид человеческой деятельности, который принципиально может быть автоматизирован полностью}
\scnnote{Есть виды человеческой деятельности, которые принципиально могут быть автоматизированы \underline{полностью}, но в текущий момент эта автоматизация не полна. Это ,например, частично автоматизированная деятельность по производству спроектированных искусственных объектов. Здесь важна \underline{четкость} распределения "обязанностей"{} между различными средствами автоматизации и поэтапное исключение неавтоматизированных действий, "вручную"{} выполняемых людьми (например, сотрудниками производственных предприятий), т. е. поэтапная автоматизация этих действий.}