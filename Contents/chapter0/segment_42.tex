\scnheader{Проект IMS.ostis}
\scnidtf{Человеко-машинная деятельность, осуществляемая в рамках \textit{Экосистемы OSTIS} и напрвленная на разработку и перманентное совершенствование \textit{Метасистемы 
IMS.ostis}, которая является формой представления (отображения) (1) текущего состояния \textit{Технологии OSTIS}, как комплекса методов и средств автоматизации (поддержки) разработки\textit{ostis-систем} и (2) текущего состояния самого \textit{Проекта IMS.ostis}.}
\scntext{примечание}{Принципы (правила) организации деятельности в рамках \textit{Проекта IMS.ostis} полностью совпадают с принципами (правилами) организации деятельности в рамках любого другого проекта, направленного на разработку и совершенствование любой другой ostis-системы.}
\scnrelto{ключевой подпроект}{Проект Экосистемы OSTIS}
\scnaddlevel{1}
\scnidtf{Совместная деятельность ученых, инженеров и ostis-систем, входящих в \textit{Экосистему OSTIS}, направленная на перманентное совершенствование \textit{Экосистемы OSTIS} -- на совершенствование (реинжиниринг) входящих в неё  \textit{ostis-систем} и на создание новых ostis-систем и их включение в состав \textit{Экосистемы OSTIS.}}
\scnaddlevel{-1}
\scntext{пояснение}{
ostis-система, являющаяся:
\begin{scnitemize}
	\item ostis-порталом научно-технических знаний по Технологии OSTIS, база знаний которого включает в себя:
	\begin{scnitemize}
		\item формальную теорию ostis-систем
		\item формальную теорию (методику) проектирования 󠇦 ostis-систем
		\item формальную спецификацию средств автоматизации проектирования ostis-систем
		\item библиотеку проектирования ostis-систем
		\item формальную спецификацию средств производства спроектированных ostis-систем
	\end{scnitemize}
	\item ostis-системой автоматизации (поддержки) проектирования ostis-систем
	\item ostis-системой поддержки производства (сборки, синтеза, генерации) спроектированных ostis-систем
	\item ostis-системой поддержки реинжиниринга ostis-систем в ходе их эксплуатации
\end{scnitemize}
}

\scnheader{Метасистема IMS.ostis}
\scnidtf{Универсальная базовая (предметно-независимая) ostis-система автоматизации проектирования ostis-систем (любых ostis-систем)}
\scnrelboth{следует отличать}{специализированная ostis-система автоматизации проектирования ostis-систем}
\scniselement{ostis-система}
\scnrelto{корпоративная ostis-система}
{Консорциум OSTIS}
\scnidtf{IMS.ostis}
\scnidtf{Интеллектуальная метасистема, построенная по стандартам \textit{технологии OSTIS} и предназначенная (1) для инженеров \textit{ostis-систем} -- для поддержки проектирования. Реализации и обновления (реинжиниринга) \textit{ostis-систем} и для разработчиков \textit{Технологии OSTIS} -- для поддержки коллективной деятельности по развитию стандартов и библиотек \textit{Технологии OSTIS.}}
\scnrelto{форма реализации}{Технология OSTIS}
\scnrelto{продукт}{Проект IMS.ostis}
\scnidtf{Интеллектуальная Метасистема, являющаяся формой (вариантом) реализации (представления, оформления) \textit{Технологии OSTIS} в виде \textit{ostis-системы}}
\scntext{примечание}{Тот факт, что Технология OSTIS реализуется в виде ostis-системы, является весьма важным для эволюции Технологии OSTIS, поскольку методы и средства эволюции (перманентного совершенствования) Технологии OSTIS становятся фактически совпадающими с методами и средствами разработки любой (!) ostis-системы на всех этапах их жизненного цикла.\\
Другими словами, эволюция Технологии OSTIS осуществляется методами и средствами самой этой технологии.}
\scnidtf{Система комплексной автоматизации (информационной и инструментальной поддержки) проектирования и реализации ostis-систем, которая сама реализована также в виде ostis-системы.}
\scnidtf{Портал знаний по Технологии OSTIS, интегрированный с САПРом ostis-систем и реализованный в виде ostis-системы.}
\scniselement{портал научно-технических знаний}

\bigskip
\scnstartset
\scnheader{Метасистема IMS.ostis}
\scniselement{система автоматизации проектирования}
\scnaddlevel{1}
\scnidtf{CAD-система}
\scnaddlevel{1}
\scnrelto{аббревиатура}{\scnfilelong{Computer Aided Design system}}
\scnaddlevel{-2}
\scniselement{интеллектуальная обучающая система}
\scnendstruct

\scnrelboth{семантическая эквивалентность}{\scnfilelong{Метасистема IMS.ostis является одновременно и системой автоматизации проектирования (ostis-систем) интеллектуальной системой обучающей методам  и средствам проектирования ostis-систем.}}
\scnaddlevel{1}
\scntext{следовательно}{этот факт существенно повышает качество проектирования прикладных ostis-систем, расширяет контингент разработчиков ostis-систем и интегрирует проектную (инженерную) деятельность в области искусственного интеллекта с образовательной деятельностью в этой области.}
\scnaddlevel{-1}



