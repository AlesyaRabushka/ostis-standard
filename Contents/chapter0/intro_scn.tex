\scsection{Описание языка структурированного представления знаний ostis-систем}
\label{intro_scn}

\begin{SCn}

\bigskip

\scnsectionheader{\currentname}

\scnstartsubstruct

\scnheader{SCn-код}
\scnidtf{Язык структурированного представления знаний \textit{ostis-систем}}
\scnexplanation{\textit{SCn-код} является языком структурированного внешнего представления текстов \textit{SC-кода} и представляет собой синтаксическое расширение \textit{SCs-кода}, направленное на повышение наглядности и компактности текстов \textit{SCs-кода}. 
	
SCn-код позволяет перейти от линейных текстов \uline{SCs-кода} к форматированным и фактически двухмерным текстам, в которых появляется декомпозиция исходного линейного текста \uline{SCs-кода} на \uline{строчки}, размещенные "по вертикали". При этом начало всех \uline{строчек} текста фиксировано и определяется известным и ограниченным набором правил, что дает возможность использовать это при форматировании \uline{sc.n-текста} (текста, принадлежащего SCn-коду.}
\scniselement{язык двухмерных текстов}
\scnaddlevel{1}
\scnidtf{язык, каждый \textit{текст} которого задается (1) множеством входящих в него \textit{символов}, (2) отношением порядка (последовательности) \textit{символов} по "горизонтали"{}, (3) отношением порядка(последовательности) \textit{символов} по "вертикали".}
\scnaddlevel{1}
\scnexplanation{Символ, входящий в состав \textit{двухмерного текста}, в общем случае может иметь четыре "соседних"{}\textit{символа}: (1) \textit{символ}, находящийся от него \uline{слева} в рамках той же \textit{строчки}, (2) \textit{символ}, находящийся от него \uline{справа} в рамках этой же \textit{строчки}, (3) \textit{символ}, находящийся строго \uline{над} ним в предыдущей \textit{строчке} и (4) \textit{символ}, находящийся строго \uline{под ним} в следующей \textit{строчке} текста.}
\scnaddlevel{-1}
\scnaddlevel{-1}
\scntext{сравнительный анализ}{Благодаря тому, что в состав sc.n-текстов могут входить и sc.s-тексты, и sc.g-тексты (ограниченные sc.n-контуром), SCn-код можно считать интегратором различных внешних языков представления знаний.  Это дает возможность при визуализации и разработке базы знаний ostis-системы недостатки одного из предлагаемых вариантов внешнего представления sc-текстов (SCg-кода, SCs-кода, SCn-кода) компенсировать достоинствами других вариантов.}

\bigskip

\scnmakeset{SCn-код;SCs-код}
\scnrelfrom{описание связи}{\scnstartsetlocal\\
\scnheaderlocal{SCn-код}
\scnrelfrom{синтаксическое расширение;синтаксическое ядро языка}{SCs-код}
\scnendstruct
}
\scntext{отличие}{Переход от линейности sc.s-текстов к двухмерности sc.n-текстов.}
\scnaddlevel{1}
    \scntext{уточнение}{Важной особенностью SCn-кода является "двухмерный"{} характер его текстов. Это проявляется в том, что для каждого фрагмента текста SCn-кода важное значение имеет величина отступа от левого края \textit{строчки}.
    
    В тексте \textit{SCn-кода} в отличие от текста \textit{SCs-кода} для каждого фрагмента текста важное значение имеет не только то, как этот фрагмент связан с другими фрагментами "по горизонтали"{}(какой фрагмент находится \uline{левее} и какой \uline{правее} на одной и той же \textit{строчке}), но и то, как он связан с другими фрагментами "по вертикали"{}(какой фрагмент находится \uline{выше} на предыдущей \textit{строчке} и какой находится \uline{ниже} на следующей \textit{строчке}). Так, например, если в тексте \textit{SCn-кода} некоторый \textit{sc-идентификатор}(внешний идентификатор \textit{sc-элемента}) размещен сразу после вертикальной табуляционной линии и точно \uline{под ним} размещен некоторый \textit{sc.s-коннектор}, то это означает, что указанный \textit{sc-элемент} инцидентен \textit{sc-коннектору}, изображенному указанным \textit{sc.s-коннектором}.
    
    Для того, чтобы обеспечить точное задание(формулировку) правил двухмерной инцидентности элементов(элементарных фрагментов) sc.n-текстов, вводится понятие \textit{\textbf{страницы} sc.n-текста}, понятие \textit{\textbf{строчки} sc.n-текста}, а также используется специальная \uline{разметка}, представляющая собой вертикальные табуляционные линии, расстояние между которыми примерно равняется максимальной длине sc.s-коннектора (обычно это расстояние равно ширине 4-5 символов).}
\scnaddlevel{-1}

\scnheader{sc.n-текст}
\scnidtf{текст SCn-кода}
\scnidtf{последовательность предложений SCn-кода}
\scnidtf{последовательность предложений SCn-кода, каждое из которых не является частью какого-либо другого предложения из \uline{этой} последовательности}

\scnheader{страница sc.n-текста}
\scnidtf{страница, на которой размещается sc.n-текст}
\scnnote{если sc.n-текст является частью какого-либо другого файла, разделяемого на страницы, например, публикации какой-либо части базы знаний, то sc.n-страницей считается только часть страницы, на которой изображен sc.n-текст, в то время как страница указанного файла может быть больше за счет, например, белых полей по краям страницы, необходимых для последующей распечатки.}

\scnheader{строчка sc.n-текста}
\scnnote{Максимальное количество символов в строчках sc.n-текста для каждого sc.n-текста фиксировано и определяется конкретным вариантом размещения sc.n-текста. При этом, в зависимости от отступов в рамках конкретного sc.n-предложения, строчка sc.n-текста может начинаться не с левого края sc.n-текста (но всегда с какой-то из вертикальных линий разметки) и иметь произвольную длину, ограничиваемую правой границей sc.n-страницы.}

\scnheader{линия разметки sc.n-текста}
\scnidtf{табуляционная линия sc.n-текста}
\scnidtf{вертикальная линия разметки sc.n-текста}
\scnidtf{вертикальная табуляционная линия}
\scnidtf{вертикальная линия, используемая для упрощения восприятия sc.n-текстов и показывающая уровень отступа для компонентов sc.n-предложений}
\scnexplanation{1-я линия разметки ограничивает левый край sc.n-страницы, 2-я линия разметки располагается примерно между 5 и 6 символами строчки и т.д. Расстояние между линиями разметки может меняться в зависимости от размера шрифта, однако в рамках одного sc.n-текста всегда остается одинаковым. Общее количество линий разметки ограничивается максимально возможной шириной sc.n-страницы в конкретном файле ostis-системы, содержащем данный sc.n-текст.}

\scnheader{следует отличать*}
\scnhaselementset{страница sc.n-текста;строчка sc.n-текста;строка}

\bigskip

\scnmakeset{SCn-код;SCs-код}
\scnrelfrom{сходство}{\scnstartsetlocal\\
\scnheaderlocal{Алфавит SCs-кода}
\scnreltolist{алфавит}{SCs-код;SCn-код}
\scnendstruct
}
\scnaddlevel{1}
	\scnrelboth{семантически эквивалентная информационная конструкция}			{\scnfilelong{Алфавит символов \textit{SCs-кода} является также алфавитом символов и \textit{SCn-кода}, т.е. \textit{алфавиты}* этих языков совпадают.}}
\scnaddlevel{-1}
\scntext{сходство}{Все компоненты sc.s-текстов используются также и в sc.n-текстах:
\begin{scnitemize}
\item sc-идентификаторы
\item sc.s-коннекторы
\item модификаторы sc.s-коннекторов с соответствующими разделителями (двоеточиями)
\item разделители, используемые в sc-выражениях, обозначающих sc-множества, заданные перечислением элементов с соответствующими разделителями (\textit{точкой с запятой} или \textit{круглым маркером})
\item \textit{круглые маркеры} в перечислениях идентификаторов sc-элементов, связанных однотипными sc-коннекторами с однотипными модификаторами с заданным sc-элементом
\item разделители предложений (двойные точки с запятой) (опускаются при преобразовании sc.s-предложений в sc.n-предложения)
\item ограничители присоединенных sc.s-предложений (опускаются при преобразовании sc.s-предложений в sc.n-предложения)
\end{scnitemize}
}
\scntext{отличие}{В отличие от sc.s-текстов в sc.n-текстах:
\begin{scnitemize}
\item добавляются новые виды sc-выражений (а именно -- sc-выражений, имеющих двухмерный характер);
\item добавляется новый вид разделителей предложений -- пустая строчка;
\item меняется размещение предложений с учетом двухмерного характера такого размещения.
\end{scnitemize}
}
\scntext{отличие}{В \textit{SCn-коде} по сравнению с \textit{SCs-кодом} добавляются новые виды \textit{sc-выражений}:
\begin{scnitemize}
\item \textit{sc-выражение}, представляющее собой двухмерный \textit{sc.n-текст}, ограниченный \textit{sc.n-контуром} или \textit{sc.n-рамкой}. Каждый \textit{sc.n-контур} изображается условно в виде \textit{открывающей фигурной скобки} и расположенной строго \uline{под} ней через несколько строчек \textit{закрывающей фигурной скобки}. Внутри указанных скобок (начиная от линии вертикальной разметки, на которой расположены сами скобки, и до правого края \textit{страницы}) размещается sc.n-текст. Полученный sc.n-контур является изображением sc-структуры, являющейся результатом трансляции указанного sc.n-текста в SC-код. Каждая \textit{sc.n-рамка} изображается аналогичным образом, только вместо \textit{фигурных скобок} в ней используются \textit{квадратные скобки}, либо \textit{квадратные скобки} с \textit{восклицательным знаком} (в случае файла-образца);
\item \textit{sc-выражение}, представляющее собой двухмерный \textit{sc.g-текст}, ограниченный \textit{sc.n-контуром} или \textit{sc.n-рамкой}.
\item \textit{sc-выражение}, представляющее собой ограниченное \textit{sc.n-рамкой} двухмерное графическое изображение \textit{информационной конструкции}, закодированной в некотором \textit{файле ostis-системы}. Такой \textit{информационной конструкцией} может быть таблица, рисунок, фотография, диаграмма, график и многое другое.
\end{scnitemize}
}
\scnaddlevel{1}
	\scnnote{Нетрудно заметить, что \textit{sc.n-контур} является, по сути, двухмерным эквивалентом \textit{sc-выражения sc-структуры}, а \textit{sc.n-рамка} -- двухмерным эквивалентом \textit{sc-выражения внутреннего файла ostis-системы} или \textit{sc-выражения, обозначающего файл-образец ostis-системы}.}
\scnaddlevel{-1}

\scnheader{sc.n-рамка}
\scnidtf{ограничитель изображения файла \uline{ostis-системы}, используемый в \uline{sc.n-предложениях}}

\scnhaselementrole{пример}{\scnsourcecomment{начало sc.n-рамки}\vspace{\baselineskip}\scnfileimage{\bigskip}
	\scnsourcecommentpar{конец sc.n-рамки}}
\scnnote{С формальной точки зрения \textit{sc.n-рамка} всегда представляет собой \uline{одну} \textit{строчку sc.n-текста}. Это означает, что \textit{sc.n-рамка} не может быть синтаксически разделена на части в рамках того \textit{sc.n-текста}, в котором она используется, и внутрь нее не могут вставляться, например, \textit{присоединенные sc.n-предложения} или какой-либо другой текст (за исключением случаев, когда \textit{sc.n-рамка} содержит \textit{sc.n-текст}, но в этом случае указанный \textit{sc.n-текст} все равно будет рассматриваться как целостный внешний файл, а не как фрагмент окружающего его \textit{sc.n-текста}).}

\scnheader{sc.n-контур}
\scnidtf{используемый в \uline{sc.n-предложениях} ограничитель, являющийся изображением sc-структуры}
\scnhaselementrole{пример}{\scnsourcecomment{начало sc.n-контура}\\\settab\scnstartsetlocal \bigskip\\ \scnendstruct \scnsourcecommentpar{конец sc.n-контура}}

\bigskip

\scnmakeset{sc.s-предложение;sc.n-предложение}
\scntext{сходство}{Понятие \textit{sc.n-предложения} является естественным обобщением понятия \textit{sc.s-предложения}. Более того, \uline{аналогичным} для \textit{sc.s-предложений} образом вводятся понятия:
\begin{scnitemize}
%\item \textit{тривиального sc.n-предложения}
\item \textit{простого sc.n-предложения}
\item \textit{сложного sc.n-предложения}
\item \textit{sc.n-предложения, содержащего присоединенные sc.n-предложения}
\item \textit{sc.n-предложения, не содержащего присоединенные sc.n-предложения}
\item \textit{присоединенного sc.n-предложения}
\item \textit{неприсоединенного sc.n-предложения}
\end{scnitemize}}
\filemodetrue
\scnrelfromlist{отличие}{\uline{Если} каждое \textit{неприсоединенное sc.s-предложение} \uline{либо} являетcя первым предложением \textit{sc.s-текста}, \uline{либо} начинается после \textit{разделителя sc.s-предложений} (\textit{двойной точки с запятой}), \uline{то} каждое \textit{неприсоединенное sc.n-предложение} начинается с начала новой строчки;
\uline{Если} каждое \textit{присоединенное sc.s-предложение} начинается либо после открывающего ограничителя присоединенных sc.s-предложений (открывающей круглой скобки со звездочкой), \uline{либо} после \textit{разделителя sc.s-предложений}, \uline{то} каждое \textit{присоединенное sc.n-предложение} начинается с новой строчки под sc-идентификатором, которым завершается то sc.n-предложение (и соответственно, sc.s-предложение), в которое встраивается данное \textit{присоединенное sc.n-предложение}; 
Первый \textit{sc-идентификатор}, входящий в состав \textit{sc.n-предложения} до \textit{sc.s-коннектора} выделяется \uline{жирным} курсивом;
В \textit{sc.n-предложениях двойная точка с запятой} не используется в качестве признака завершения этих предложений и, соответственно, не используется в качестве разделителя \textit{sc.n-предложений}. Таким разделителем является \textit{пустая строчка}.}

\scntext{отличие}{Благодаря двухмерности SCn-кода появляются более широкие возможности (степени свободы) для наглядного и компактного размещения sc.n-предложений.}
\scnaddlevel{1}
\scnrelfromlist{уточнение}{При оформлении sc.n-предложения осуществляется четкая \uline{табуляция} всех присоединенных к нему sc.n-предложений, присоединяемых к исходному "по вертикали". Вертикальная линия табуляции задает левую границу исходного (максимального) sc.n-предложения или левую границу присоединенного sc.n-предложения, присоединяемого "по вертикали". Левая граница sc.n-предложения задает начало первого sc-идентификатора, входящего в состав этого sc.n-предложения, а также начало sc.s-коннектора, инцидентного указанному sc-идентификатору и размещаемого \uline{строго под} этим sc-идентификатором. Расстояние между вертикальными табуляционными линиями фиксировано и примерно равно максимальной длине sc.s-коннектора;
В отличие от sc.s-текстов: в sc.n-текстах sc.s-коннектор может быть инцидентен предшествующему sc-идентификатору (как простому, так и sc-выражению) не только "по горизонтали"{}, но и "по вертикали". Для этого sc.s-коннектор размещается строго \uline{под} предшествующим ему sc-идентификатором;
Кроме того "по вертикали"\ sc-идентификатор может быть инцидентен не одному, а \uline{нескольким} sc.s-коннекторам, которые последовательно "по вертикали"\ размещаются \uline{под} указанным sc-идентификатором. Это позволяет в рамках одного sc.n-предложения представлять произвольное число "ответвлений"\ от каждого sc-идентификатора, т.е. произвольное число sc.s-коннекторов, инцидентных этому sc-идентификатору;
Каждый sc-идентификатор, включая sc-выражение, ограничиваемого фигурными или квадратными скобками, должен размещаться сразу правее вертикальной разметочной линии, если \uline{под ним} размещается sc.s-коннектор;
Каждый sc.s-коннектор выделяется жирным некурсивным шрифтом и, если он находится \uline{под} инцидентным ему sc-идентификатором, размещается строго между двумя вертикальными разметочными линиями, прижимаясь при этом к левой из этих двух разметочных линий.}
\scnaddlevel{-1}
\filemodefalse

\scnheader{SCn-код}
\scntext{правило синтаксической трансформации}{Поскольку по отношению к SCn-коду SCs-код является \textit{синтаксическим ядром языка*}, SCn-код можно рассматривать как результат интеграции нескольких направлений расширения SCs-кода, в основе которых лежат правила синтаксической трансформации sc.s-текстов и sc.n-текстов, ориентированные на повышение эффективности использования тех возможностей обеспечения наглядности и компактности sc.n-текстов, которые открываются при переходе от линейности sc.s-текстов к двухмерности sc.g-текстов}

\scnheader{sc.n-предложение}
\scnrelfromlist{заданная операция}{Операция преобразования sc.s-предложения в sc.n-предложение*\\
	\scnaddlevel{1}
	\scnsubset{синтаксическая трансформация*}
	\scnexplanation{Каждое \textit{sc.s-предложение}, записываемое линейно ("горизонтально") может быть преобразовано в соответствующее двухмерное \textit{sc.n-предложение}. 
	Перечислим основные правила трансформации sc.s-предложений в sc.n-предложения:
	\begin{scnitemize}
		\item sc.s-коннектор может размещаться на следующей строчке под предшествующим sc-идентификатором, начиная с того же символа следующей строчки, что и указанный sc-идентификатор\char59
		\item если sc-идентификатор переносится на следующую строчку, то его продолжение на следующей строчке осуществляется с таким же отступом от начала строчки, с каким указанный sc-идентификатор начинается\char59
		\item перечисление sc-идентификаторов, разделенных точкой с запятой, может осуществляться не "в строчку"\ , а "в столбик"\ при размещении каждого следующего sc-идентификатора строго под предшествующим. При этом, разделительная точка с запятой может быть заменена \textit{круглым маркером}, который помещается \uline{перед} каждым перечисляемым sc-идентификатором\char59
		\item закрывающая фигурная или квадратная скобка может быть размещена строго \uline{под} соответствующей открывающей скобкой\char59
		\item sc-идентификатор в sc.n-предложении может быть связан с другими sc-идентификаторами через несколько разных sc.s-коннекторов. При этом, каждый из этих sc.s-коннекторов размещается строго под предшествующим, но только после того, когда будет завершена запись всей, в общем случае разветвленной, цепочки sc.s-коннекторов и sc-идентификаторов, которая начинается с предшествующего sc.s-коннектора. В SCs-коде аналога таким предложениям с неограниченной возможностью описания “разветвленных” связей sc-идентификаторов нет. Следовательно, если в sc.s-тексте sc-идентификатор может быть инцидентен не более, чем двум sc.s-коннекторам (слева и справа от него), то в sc.n-тексте sc-идентификатор может быть дополнительно инцидентен неограниченному числу (причем, не обязательно одинаковых) sc.s-коннекторов, которые размещаются “по вертикали” строго под ним.
	\end{scnitemize}}
	\scnaddlevel{-1}
	;Операция присоединения sc.n-предложения*\\
		\scnaddlevel{1}
	\scnsubset{синтаксическая трансформация*}
	\scnexplanation{Некоторое sc.n-предложение может быть присоединено к другому sc.n-предложению, если в этом другом sc.n-предложении есть sc-идентификатор (но не sc.s-модификатор), с которого начинается первое (присоединяемое) sc.n-предложение.
	Присоединение в происходит следующим образом:
	\begin{scnitemize}
		\item начальный sc-идентификатор присоединяемого предложения опускается\char59
		\item оставшаяся часть sc.n-предложения, начиная от sc.s-коннектора, записывается под таким же sc-идентификатором, но входящим в состав того sc.n-предложения, к которому присоединяется данное sc.n-предложение. С учетом этого смещаются все отступы в присоединяемом sc.n-предложении.
	\end{scnitemize}
	Таким образом может формироваться произвольное число любых разветвлений.}
	\scnaddlevel{-1}}
\scnnote{По сути, семантика sc.n-предложения -- множество маршрутов в sc-тексте, возможно пересекающихся и исходящих из заданного sc-элемента}

\scnheader{Описание примеров выполнения операций, заданных на множестве sc.n-предложений}
\scnstartsubstruct

\bigskip

\scnfilelong{\textit{Треугк(ТчкВ;ТчкС;ТчкD)} $\Rightarrow$ \textit{сторона*}: \textit{Отр(ТчкВ;ТчкС)} (* $\in$ \textit{отрезок};; *);;}\\
\scnrelfrom{Операция преобразования sc.s-предложения в sc.n-предложен ие}{\scnfilescn{
		\scnheader{Треугк(ТчкВ;ТчкС;ТчкD)}
		\scnrelfrom{сторона}{Отр(ТчкВ;ТчкС)}
		\scnaddlevel{1}
			\scniselement{отрезок}
		\scnaddlevel{-1}
		}}
\scnresetlevel

\bigskip
\bigskip

\scnfilescn{
	\scnheader{Треугк(ТчкВ;ТчкС;ТчкD)}
	\scnrelfrom{сторона}{Отр(ТчкВ;ТчкС)}

	\scnheader{Отр(ТчкВ;ТчкС)}
	\scniselement{отрезок}
}
\scnrelfrom{Операция присоединения sc.n-предложения}{\scnfilescn{
		\scnheader{Треугк(ТчкВ;ТчкС;ТчкD)}
		\scnrelfrom{сторона}{Отр(ТчкВ;ТчкС)}
		\scnaddlevel{1}
		\scniselement{отрезок}
		\scnaddlevel{-1}
}}
\scnresetlevel

\bigskip
\scnendstruct               
\bigskip

\scnendstruct \scninlinesourcecommentpar{Закрывающий ограничитель раздела  \textit{\nameref{intro_scn}}}

\end{SCn}