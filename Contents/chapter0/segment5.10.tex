\bigskip
\scnfragmentcaption

\scnheader{автоматизация человеческой деятельности}
\scntext{вопрос}{Почему для комплексной автоматизации человеческой деятельности целесообразно использовать семантически совместимые, договороспособные, самостоятельные интеллектуальные компьютерные системы, которым можно делегировать права на принятие некоторых решений.}

\scntext{вопрос}{Почему для повышения уровня комплексной \textit{автоматизации человеческой деятельности} необходим переход от современных (традиционных) \textit{компьютерных систем} и соответствующих им информационных технологий, а также от \textit{современных интеллектуальных компьютерных систем} и соответствующих им современных технологий искусственного интеллекта к \textit{интеллектуальным компьютерным системам} \uline{нового поколения} и к соответствующей им Комплексной технологии проектирования таких систем.}
	\scnaddlevel{1}
	\scntext{ответ}{В силу отсутствия унификации представления обрабатываемой информации в традиционных компьютерных системах и, как следствие, отсутствия совместимости этих систем как на синтаксическом уровне, так и на семантическом уровне, принциально не существует универсального метода системной интеграции традиционных компьютерных систем и, следовательно, невозможна полная автоматизация решения этой задачи. Системная интеграция традиционных компьютерных систем практически всегда осуществляется "вручную"{} с учетом индивидуальной специфики каждой интегрируемой системы и, следовательно, является весьма трудоемкой и требующей высокой квалификации разработчиков.\\
	Но на данном этапе эволюции компьютерных систем крайне актуальной является полная автоматизация их интеграции без какого бы то ни было участия разработчиков и тем более конечных пользователей. Если компьютерные системы не приобретут способность \uline{самостоятельно} взаимодействовать между собой в целях решения сложных комплексных задач, то эффективность использования человечеством интенсивно расширяемого многообразия весьма полезных и качественно реализованных информационных ресурсов и сервисов будет весьма низкой. Традиционно компьютерные технологии позволяют реализовать \uline{любую} модель обработки информации (в том числе и \uline{любую} интеллектуальную модель -- нейросетевую, логическую и т.д.). Однако актуальным является не реализация самих этих моделей, а их интеграция, что требует обоспечения синтаксически и семантически совместимых компьютерных систем и полной автоматизации их системной интеграции. Следовательно необходим переход на принципиально новое поколение компьютерных технологий и, в частности, на принципиально новое поколение самих компьютеров, ориентированных на решение проблем совместимости компьютерных систем и полной автоматизации их системной интеграции.\\
	Таким образом, дальнейшее повышение уровня автоматизации различных видов человеческой деятельности потребует перехода на принципиально новый уровень информационных технологий -- от современных (традиционных) \textit{компьютерных систем} к компьютерным системам, имеющим существенно более \textit{высокий уровень интеллекта} и способным не только индивидуально решать достаточно сложные (в том числе, интеллектуальные) задачи, но и эфффективно \uline{самостоятельно} взаимодействовать между собой, координируя свою деятельность при решении задач, принадлежащих априори неизвестным (заранее не предусмотренным) классам задач и требующих коллективного (корпоративного) решения.\\
	Основные проблемы автоматизации \textit{человеческой деятельности} в настоящее время лежат не в области разработки средств автоматизации решения различных конкретных классов задач (в том числе и весьма сложных, интеллектуальных, труднорешаемых задач), а в области системной интеграции этих средств в комплексы, компоненты которых способны самостоятельно кооперироваться для совместного (коллективного) решения сложных задач. Но для этого указанные компоненты должны уметь согласовывать, координировать свои действия, должны понимать друг друга, должны быть семантически совместимы.}
	\scnaddlevel{-1}
	
\scnheader{интеллектуальная компьютерная система}
\scnnote{Различные интеллектуальные компьютерные системы могут быть эффективно использованы в качестве средств автоматизации самых различных видов человеческой деятельности. Но, поскольку все виды человеческой деятельности взаимосвязаны (как минимум потому, что каждый человек может одновременно участвовать сразу в нескольких видах деятельности, причем в разные моменты времени этот набор видов деятельности для каждого человека может быть различным), интеллектуальная компьютерная система автоматизации каждого вида человеческой деятельности должна эффективно взаимодействовать с другими интеллектуальными компьютерными системами, осуществляющими автоматизацию других видов человеческой деятельности. Другими словами, необходимо переходить от автоматизации отдельных видов человеческой деятельности к автоматизации комплекса всех видов человеческой деятельности. Для этого необходимо:
	\begin{scnitemize}
	\item не просто достаточно детально разработать \uline{теорию каждого вида деятельности}, выделив (1) все классы автоматизируемых действий, (2) все классы неавтоматизируемых действий, (3) соответствующие этим классам методы выполнения действий (в частности, это могут быть обобщенные бизнес-процессы), которые по сравнению с используемыми в настоящий момент могут потребовать существенного реинжиниринга бизнес-процессов \scncite{Popov};
	\item но и представить эти теории в формализованном и унифицированном виде в качестве фрагментов баз знаний соответствующих интеллектуальных компьютерных систем, обеспечив при этом высокую степень конвергенции этих теорий.
	\end{scnitemize}}
	
\scnheader{интеллектуальная компьютерная система}
\scnrelfromlist{возможное амплуа}{средство автоматизации проектирования\\
	\scnaddlevel{1}
	\scnsuperset{средство автоматизации проектирования интеллектуальных компьютерных систем}
		\scnaddlevel{1}
		\scnnote{В силу большой сложности процесса проектирования интеллектуальных компьютерных систем для автоматизации этого процесса необходимо использовать именно интеллектуальные компьютерные системы.}
		\scnaddlevel{-1}
	\scnaddlevel{-1}
;средство автоматизации производства
;средство повышения качества эксплуатации сложного объекта\\
	\scnaddlevel{1}
	\scnsuperset{средство help-поддержки конечных пользователей}
	\scnsuperset{средство управления процессом повышения качества деятельности конечных пользователей}
	\scnsuperset{средство поддержки оптимальных эксплуатационных свойств эксплуатируемого объекта}
		\scnaddlevel{1}
		\scnexplanation{Здесь имеется в виду мониторинг состояния эксплуатируемого объекта, контроль условий эксплуатации, своевременная профилактика и ремонт.}
		\scnaddlevel{-1}
	\scnsuperset{средство поддержки совершенствования эксплуатируемого объекта в ходе его эксплуатации}
	\scnnote{Для интеллектуальных компьютерных систем все средства повышения качества их эксплуатации целесообразно встраивать в эти системы. Имеется в виду слияние нескольких интеллектуальных компьютерных систем в одну интегрированную.}
	\scnaddlevel{-1}
;средство автоматизации научно-исследовательской деятельности в рамках заданной научной дисциплины
;средство автоматизации образовательной деятельности\\
	\scnaddlevel{1}
	\scnnote{Автоматизация образовательной деятельности может осуществляться в рамках:
		\begin{scnitemize}
		\item заданной учебной дисциплины;
		\item заданной учебной специальности;
		\item заданного учебного заведения;
		\item заданного государства.
		\end{scnitemize}}
	\scnaddlevel{-1}
;средство автоматизации бизнес-деятельности в заданной научно-технической области
	\scnaddlevel{1}
	\scnnote{Здесь важна автоматизация контроля за реализацией всех направлений организационной деятельности с учетом разработанных и постоянно уточняемых и корректируемых планов, а также с учетом согласованных приоритетов.}
	\scnaddlevel{-1}
;средство автоматизации деятельности в области здравоохранения
;средство автоматизации административной деятельности
;средство автоматизации деятельности жилищно-коммунального хозяйства
;юриспруденция
;правоохранительная деятельность
;транспорт}

\scnheader{Экосистема OSTIS}
\scnnote{\textit{Экосистема OSTIS} является основой для перевода уровня информатизации различных областей человеческой деятельности на принципиально новый уровень, а также для интеграции соответствующих проектов -- "Общество 5.0"{}, "Industry 4.0"{}, "University 3.0"{}, "Умный дом"{}, "Умный город"{} и других (без интеллектуальных компьютерных систем все эти проекты невозможны).\\
Все эти проекты должны быть приведены в единую стройную иерархическую систему взаимосвязанных проектов, охватывающих весь объем и многообразие человеческой деятельности.}