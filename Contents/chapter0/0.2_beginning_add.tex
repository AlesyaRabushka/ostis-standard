\\
\scnaddlevel{1}
\scnrelfromset{подвопрос}{
	\scnfileitem{Недостатки современных интеллектуальных компьютерных систем};
	\scnfileitem{Недостатки современной технологии Искусственного интеллекта};
	\scnfileitem{Каким требованиям должна удовлетворять качественная технология разработки интеллектуальных компьютерных систем}\\
	\scnaddlevel{1}
		\scnrelfromset{подвопрос}{
			\scnfileitem{уточнить требования, представляемые к интеллектуальным компьютерным системам (что такое интеллектуальная компьютерная система)};
			\scnfileitem{уточнить, почему этого нет};
			\scnfileitem{как эти требования удовлетворить в рамках интеллектуальных компьютерных систем (принципы)};
			\scnfileitem{уточнить требования к технологии};
			\scnfileitem{понять, уточнить, почему, что мешает созданию технологии}
			\scnaddlevel{1}
				\scnrelfromset{причина}{
						\scnfileitem{сложность объекта};
						\scnfileitem{отсутствие понимания того, что задача такой сложности требует создания принципиально нового творческого коллектива с принципиально новой организацией взаимодействия}}
			\scnaddlevel{-1};
	\scnfileitem{как это сделать (принципы, лежащие в основе создания технологии интеллектуальных компьютерных систем)}
	\scnaddlevel{-1}};
	\scnfileitem{Что такое ИИ (как наука)};
%	\scniselement{научно-техническая дисциплина}
	;Что такое интеллектуальная кибернетическая  система\\
%	\scnsubset{кибернетическая система}
	;Что такое технология проектирования и реализации интеллектуальная кибернетическая система
	\scnaddlevel{1}
	;проблемы создания технологии проектирования;
	технология реализации от традиционных компьютеров к компьютерам, ориентированным на реализацию интеллектуальных кибернетических систем
	\scnaddlevel{-1}
	;Результат использования технологии проектирования и реализации
	это не отдельные интеллектуальные компьютерные системы и Экосистема из интеллектуальных компьютерных систем и людей
	\scnaddlevel{1}
	;структура Экосистемы -- иерархическая система специализированных сообществ;
	Чем нас не устраивают те, интеллектуальные компьютерные системы, которые мы разрабатываем сейчас;
	Чем нас не устраивают современные технологии ИИ;
	Какие интеллектуальные компьютерные системы нам нужны;
	Какими свойствами и способностями мы хотели бы их наделить\scnaddlevel{1}
	;высокая степень обучаемости в разных направлениях\scnaddlevel{1};
	расширение знаний без введения новых понятий;
	введение новых понятий без расширения многообразия видов знаний;
	расширение многообразия видов знаний;
	расширение моделей решения задач(новый вид методов + их интерпретация)\scnaddlevel{-2}
	;Какие технологии нам нужны;
	Почему таких икс и технологий ещё нет;
	Что мешает?;
	Что делать?;
	Какие недостатки имеют современные интеллектуальные системы;
	\scnaddlevel{1}недостаточно высокий уровень интеллектуальности;
	нет эффективного взаимодействия(координации);
	высокая степень обучаемости в разных направлениях;
	\scnaddlevel{-1}
	Какие недостатки имеют современные технологии Искусственного интеллекта
	;Какова трудоёмкость разработки выбранных
	икс
	;Какова трудоёмкость системной интеграции икс и их компонентов;
	Обеспечивается ли совместимость компонентов
	икс, разрабатываемых с помощью различных
}
\scnaddlevel{-1}
}