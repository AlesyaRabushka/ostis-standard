\bigskip
\scnfragmentcaption

\scnheader{Понятие ostis-технологии}

\scnstartsubstruct

\scnheader{ostis-технология}
\scnreltoset{объединение}{
ostis-технология проектирования\\
\scnaddlevel{1}
	\scnsubdividing{
		ostis-технология проектирования ostis-систем соответствующего класса\\
		\scnaddlevel{1}
			\scnhaselement{Базовая ostis-технология проектирования ostis-систем}
		\scnaddlevel{-1}
		;ostis-технология проектирования соответствующего класса компонентов ostis-систем\\
		\scnaddlevel{1}
			\scnhaselement{Базовая ostis-технология проектирования баз знаний ostis-систем}
			\scnhaselement{Базовая ostis-технология проектирования решателей задач ostis-систем}
			\scnhaselement{Базовая ostis-технология проектирования интерфейсов ostis-систем}
		\scnaddlevel{-1}
		;ostis-технология проектирования объектов заданного класса, не являющихся ostis-системами\\
	}
\scnaddlevel{-1}
;ostis-технология производства\\
\scnaddlevel{1}
	\scnsuperset{технология производства спроектированных ostis-систем}
	\scnsuperset{ostis-технология управления производством спроектированных продуктов заданного класса, не являющихся ostis-системами}
\scnaddlevel{-1}
;технология эксплуатации ostis-систем\\
\scnaddlevel{1}
	\scnhaselement{Базовая технология эксплуатации ostis-систем}
	\scnsuperset{технология эксплуатации ostis-систем соответствующего класса}
	\scnaddlevel{1}
		\scnsuperset{ostis-технология управления производством спроектированных продуктов заданного класса, не являющихся ostis-системами}
		\scnaddlevel{1}
			\scnidtf{технология эксплуатации ostis-систем управления производством спроектированных продуктов заданного класса, не являющихся ostis-системами}
			\scnaddlevel{-1}
	\scnaddlevel{-1}
\scnaddlevel{-1}
;технология реинжиниринга ostis-систем\\
\scnaddlevel{1}
	\scnhaselement{Базовая технология реинжиниринга ostis-систем}
	\scnsuperset{технология реинжиниринга ostis-систем соответствующего класса}
\scnaddlevel{-1}
}

\scnheader{ostis-технология}
\scnidtf{компонент Технологии OSTIS}
\scnhaselement{Ядро Технологии OSTIS}
\scnaddlevel{1}
	\scnidtf{Базовая ostis-технология}
\scnaddlevel{-1}
\scnsuperset{частная ostis-технология}
\scnaddlevel{1}
	\scnsuperset{ostis-технология проектирования соответствующего класса компонентов ostis-систем}
	\scnaddlevel{1}
		\scnhaselement{Технология проектирования баз знаний ostis-систем}
		\scnhaselement{Технология проектирования решателей задач ostis-систем}
		\scnhaselement{Технология проектирования невербальных интерфейсов ostis-систем с внешней средой}
		\scnhaselement{Технология проектирования интерфейсов ostis-систем с другими техническими системами}
		\scnhaselement{Технология проектирования пользовательских интерфейсов ostis-систем}
	\scnaddlevel{-1}
\scnaddlevel{-1}
\scnsuperset{специализированная ostis-технология проектирования ostis-систем соответствующего класса}
\scnaddlevel{1}
	\scnhaselement{Технология проектирования ostis-систем управления предприятиями рецептурного производства}
	\scnhaselement{Технология проектирования ostis-систем управления предприятиями производства молочной продукции}
	\scnhaselement{Технология проектирования интеллектуальных обучающих ostis-систем}
	\scnhaselement{Технология проектирования интеллектуальных обучающих ostis-систем для школьников}
	\scnhaselement{Технология проектирования интеллектуальных обучающих ostis-систем для подготовки специалистов в области Математики}
	\scnhaselement{Технология проектирования интеллектуальных обучающих ostis-систем для подготовки специалистов в области Искуственного интеллекта}
\scnaddlevel{-1}

\scnheader{ostis-технология проектирования}
\scnnote{Каждой ostis-технологии проектирования соответсвует своя ostis-система автоматизации проектирования соответствующего класса объектов}
\scnrelfrom{соответствующее семейство средств автоматизации}{ostis-система автоматизации проектирования}
\scnrelfrom{соответствующее семейство классов проектируемых объектов}{{\normalfont(}ostis-система автоматизации проектирования ostis-систем $\cup$ ostis-система автоматизации проектирования объектов, не являющихся ostis-системами{\normalfont)}}
\scnsuperset{ostis-технология проектирования ostis-систем соответствующего класса}

\scnheader{ostis-технология проектирования ostis-систем соответствующего класса}
\scnidtf{технология проектирования \textit{ostis-систем} соответствующего (заданного) класса, который, в свою очередь, соответствует определенному \textit{виду человеческой деятельности}, подвиды которого автоматизируются с помощью указанных выше проектируемых \textit{ostis-систем}}

\scnheader{ostis-технология}
\scnrelfromlist{отношение, заданное на данном множестве}{частная технология*; специализированная технология*; комплекс специализированных технологий*}
\scnexplanation{Базовая частная или специализированная технология, входящая в состав комплексной \textit{Технологии OSTIS}, которая:
\begin{scnitemize}
	\item направлена на автоматизацию конкретного вида человеческой деятельности;
	\item ориентирована на использование ostis-систем (как индивидуальных, так и коллективных) в качестве самостоятельных субъектов или активных интеллектуальных инструментов, либо на использование человеко-машинных ostis-сообществ при решении:
	\begin{scnitemizeii}
		\item как задач, выполняемых в памяти ostis-систем (в т.ч. в памяти коллективов ostis-систем);
		\item так и задач, выполняемых во внешней среде ostis-систем, в процессе решения которых субъектами соответствующих действий либо ostis-системы (индивидуальные или коллективные), либо конкретные персоны, либо ostis-сообщества.
	\end{scnitemizeii}
\end{scnitemize}
}
\scnidtf{Множество всевозможных технологий, соответствующих стандартам технологии OSTIS и направленных на автоматизацию различных конкретных видов человеческой деятельности}
\scnrelboth{следует отличать}{Технология OSTIS}
\scnaddlevel{1}
 	\scnnote{\textit{Технология OSTIS} в отличие от понятия \textit{ostis-технологии} представляет собой не множество технологий, а комплекс взаимосвязанных между собой самых различных технологий, превращающий указанное множество технологий в единую объединенную технологию, в сумму взаимосвязанных глубоко интегрированных технологий. В этом смысле Технология OSTIS является максимальной ostis-технологией, в состав которой входят все ostis-технологии.}
\scnaddlevel{-1}
\scnrelfromlist{включение;~пример}{
ostis-технология проектирования и перепроектирования;
ostis-технология производства;
ostis-технология образования}
\scnhaselementlist{пример}{
Технология OSTIS;
OSTIS-технология публикации и согласования результатов научно-технической деятельности (в широком смысле);
OSTIS-технология проектирования, реализации и реинжиниринга ostis-систем;
OSTIS-технология разработки стандартов Технологии OSTIS}

\scnheader{ostis-технология коллективной разработки информационных ресурсов}
\scnsuperset{ostis-технология коллективного проектирования}
\scnsuperset{ostis-технология коллективной разработки планов}
\scnsuperset{ostis-технология публикации и согласования результатов научно-технической деятельности}
\scnsubset{ostis-технология}

\scnheader{ostis-технология эксплуатации ostis-систем}
\scnidtf{Общие методы и средства (языковые и интерфейсные) организации взаимодействия ostis-систем со своими конечными пользователями}
\scnsubset{ostis-технология}
\scnnote{Поскольку в рамках Экосистемы OSTIS каждому человеку придется взаимодействовать с больщим числом ostis-систем разного назначения, принципы организации взаимодействия всех ostis-систем со своими пользователями должны быть абсолютно одинаковыми. Удобство (usability) пользовательских интерфейсов должно быть направлено не только на синтаксическую красоту, но и на простую семантическую интерпретацию (понятность).}

\scnheader{ostis-технология проектирования ostis-систем}
\scnidtf{Технология построения (разработки) логико-семантических моделей (sc-моделей) ostis-систем}
\scniselement{ostis-технология}
\scnnote{Продуктом каждого завершенного (целостного) коллективного проекта, реализованного в рамках этой технологии, является полная \textit{логико-семантическая модель ostis-системы}.}
\scnrelfrom{класс продуктов}{логико-семантическая модель ostis-системы}
\scnrelfrom{средство}{Метасистема IMS OSTIS}
\scnrelfrom{класс субъектов}{коллектив разработчиков ostis-системы}
\scnrelfrom{класс исходных данных}{исходная спецификация ostis-системы}

\scnheader{ostis-технология производства ostis-систем}
\scnidtf{Технология сборки и установки ostis-систем}
\scniselement{ostis-технология}
\scnrelfrom{исходная информация}{логико-семантическая модель ostis-системы}
\scnrelfrom{комплектация}{универсальный интерпретатор логико-семантических моделей ostis-систем}
\scnaddlevel{1}
	\scnnote{Это, своего рода, "мотор"{}, "движок"{} ostis-систем}
\scnaddlevel{-1}
\scnrelfrom{методы}{Методика производства ostis-систем}
\scnrelfrom{активный инструмент}{Метасистема IMS OSTIS}
\scnrelfrom{продукты}{ostis-система}

\scnheader{ostis-технология реинжиниринга ostis-систем}
\scnidtf{Технология обновления (перепроектирования) ostis-систем в ходе их эксплуатации}
\scniselement{ostis-технология}
\scnheader{следует отличать*}
\scnhaselementset{Технология реинжиниринга ostis-систем; Технология проектирования ostis-систем}
\scnaddlevel{1}
	\scnnote{Эти технологии сходны. Их методы и средства совпадают. Не совпадают только исходные данные и результаты, которыми в \textit{Технологии обновления ostis-систем} являются предшествующие и последующие состояния ostis-систем. В \textit{Технологии проектирования ostis-систем} исходными данными являются исходные спецификации (замыслы) проектируемых ostis-систем, и результатами -- полные логико-семантические модели этих систем}
\scnaddlevel{-1}

\scnheader{Технология OSTIS}
\scnidtf{Совокупность (интеграция, объединение) всех \textit{ostis-технологий}}
\scnrelto{интеграция}{ostis-технология}
\scnidtf{Комплекс (множество) семантически совместимых \textit{технологий}, в состав которого входит \textit{Ядро Технологии OSTIS} и иерархическая система \textit{ostis-технологий}, каждая из которых ориентирована на \textit{проектирование}, \textit{производство}, \textit{эксплуатацию} или \textit{реинжиниринг} соответствующего \textit{класса ostis-систем}, обеспечивающих автоматизацию соответствующего \textit{вида человеческой деятельности}. При этом каждая такая проектируемая \textit{ostis-система} автоматизирует либо область, либо \textit{вид человеческой деятельности}, которая (который) является соответственно либо экземпляром (элементом), либо подвидом (подклассом) указанного выше \textit{вида человеческой деятельности}, соответствующего используемой \textit{специализированной \textit{ostis-технологии}}.}

\scnheader{Ядро Технологии OSTIS}
\scnidtf{Универсальная базовая \textit{ostis-технология}}
\scnidtf{Универсальный компонент Технологии OSTIS}

\scniselementrole{ключевой элемент}{ostis-технология}
\scnrelto{ядро}{Технология OSTIS}
\scnhaselement{технология}
\scnrelfrom{вид деятельности, выполняемой с помощью технологии}{проектирование, производство, эксплуатация и реинжиниринг ostis-системы}
\scnaddlevel{1}
	\scnreltoset{объединение}{
		проектирование ostis-системы\\
		\scnaddlevel{1}
			\scnidtf{построение логико-семантической модели \textit{ostis-системы}}
		\scnaddlevel{-1}
		;производство ostis-системы\\
		\scnaddlevel{1}
			\scnidtf{сборка логико-семантической модели ostis-системы и загрузка этой модели в память универсального интерпретатора таких моделей}
		\scnaddlevel{-1}
		;эксплуатация ostis-системы\\
		\scnaddlevel{1}
			\scnidtf{базовый (предметно-независимый) уровень организации деятельности конечного пользователя ostis-системы с помощью соответствующих методов	и средств}
		\scnaddlevel{-1}
		;реинжиниринг ostis-системы\\
		\scnaddlevel{1}
			\scnidtf{совершенствование \textit{ostis-системы} в процессе её эксплуатации}
		\scnaddlevel{-1}
	}
	\scnrelfrom{создаваемые продукты}{ostis-система\\
		\scnidtf{\textit{интеллектуальная компьютерная система}, построенная в соответствии со стандартом \textit{Технологии OSTIS}, предъявляемым к продуктам, создаваемым с помощью этой технологии}
		\scnaddlevel{1}
			\scnnote{Указанный стандарт продуктов, создаваемых с помощью технологии OSTIS есть не что иное, как \textit{общая формальная семантическая теория интеллектуальных компьютерных систем}}
		\scnaddlevel{-1}
	}
\scnaddlevel{-1}
\scnrelfromlist{частная технология}{
	Базовая Технология Проектирования ostis-систем\\
	\scnaddlevel{1}
		\scnrelfromlist{частная технология}{
			Технология проектирования баз знаний ostis-систем\\
			;Технология проектирования решателей задач ostis-систем\\
			;Технология проектирования интерфейсов ostis-систем\\
			\scnrelfromlist{частная технология}{
				Технология проектирования невербальных интерфейсов ostis-систем с внешней средой\\
				;Технология проектирования интерфейсов ostis-систем с другими техническими системами\\
				;Технология проектирования пользовательских интерфейсов ostis-систем
			}
		}
		\scnrelfrom{реализация}{Метасистема IMS.ostis}
		\scnaddlevel{1}
			\scnidtf{Intelligent MetaSystem for ostis-systems design}
			\scnidtf{OSTIS-система автоматизации проектирования ostis-систем}
		\scnaddlevel{-1}
	\scnaddlevel{-1}
	;Технология производства ostis-систем\\
	\scnaddlevel{1}
		\scnexplanation{Основным компонентом, точнее, инструментальным средством \textit{технологии производства ostis-систем} является \textit{универсальный интерпретатор логико-семантических моделей ostis-систем}. Указанные \textit{логико-семантические модели ostis-систем} являются результатом \textit{проектирования ostis-систем} и представляют собой начальные (исходные) состояния \textit{баз знаний} разрабатываемых \textit{ostis-систем}. В отличие от \textit{инструмента производства ostis-систем}, методика их производства весьма проста и сводится к сборке разработанных логико-семантических моделей (начального состояния \textit{баз знаний}) разрабатываемых \textit{ostis-систем} и загрузке этих моделей в память \textit{универсального интерпретатора логико-семантических моделей ostis-систем}.}
		\scnrelfrom{реализация}{универсальный интерпретатор логико-семантических моделей ostis-систем}
		\scnaddlevel{1}
			\scnexplanation{Такой интерпретатор логико-семантических моделей ostis-систем может быть реализован либо программно на \textit{современных компьютерах}, либо аппаратно в виде компьютеров нового поколения, ориентированных на реализацию интеллектуальных компьютерных систем.}
			\scnexplanation{С формальной точки зрения универсальный интерпретатор логико-семантических моделей ostis-систем является "пустой"{} ostis-системой, которая способна приобретать и записывать формализованную информацию в свою память.}
		\scnaddlevel{-1}
	\scnaddlevel{-1}
	;Базовая технология эксплуатации ostis-систем\\
	\scnaddlevel{1}
		\scnidtf{Общая технология эксплуатации ostis-систем, включающая в себя общие методы и средства, используемые в процессе эксплуатации любых ostis-систем}
		\scnrelfrom{реализация}{встраиваемая ostis-система поддержки эксплуатации ostis-систем}
		\scnaddlevel{1}
			\scnexplanation{Данная ostis-система входит (интегрирована) в состав каждой ostis-системы.}
		\scnaddlevel{-1}
	\scnaddlevel{-1}
	;Базовая технология реинжиниринга ostis-систем\\
	\scnaddlevel{1}
		\scnrelfrom{реализация}{встраиваемая ostis-система поддержки реинжиниринга ostis-систем}
		\scnaddlevel{1}
			\scnexplanation{Данная ostis-система входит (интегрирована) в состав каждой ostis-системы и обеспечивает внесение изменений "руками"{} инженеров, сопровождающих эксплуатацию ostis-системы, или авторов базы знаний этой ostis-системы в текущее состояние базы знаний ostis-системы в ходе её экспуатации}
		\scnaddlevel{-1}
	\scnaddlevel{-1}
}
\scnrelfromlist{специализированная технология}{
	Общая технология проектирования ostis-систем автоматизации проектирования\\
	\scnaddlevel{1}
		\scnrelfromlist{специализированная технология}{
		Технология проектирования ostis-систем автоматизации проектирования строительных объектов\\
		;Технология проектирования ostis-систем автоматизации проектирования автомобилей\\
		;Технология проектирования ostis-систем автоматизации проектирования интегральных микросхем
		}
	\scnaddlevel{-1}
	;Технология проектирования ostis-систем управления производством\\
	\scnaddlevel{1}
		\scnrelfromlist{специализированная технология}{
		Технология проектирования ostis-систем управления строительством различных объектов\\
		;Технология проектирования ostis-систем управления производством автомобилей\\
		;Технология проектирования ostis-систем управления производством микросхем\\
		;Технология проектирования ostis-систем управления предприятиями рецептурного производства\\
		\scnaddlevel{1}
			\scnrelfrom{специализированная технология}{				Технология проектирования ostis-систем управления предприятиями производства молочной продукции}
		\scnaddlevel{-1}
		}
	\scnaddlevel{-1}
	;Технология проектирования интеллектуальных обучающих ostis-систем\\
	\scnaddlevel{1}
		\scnrelfromset{комплекс специализированных технологий}{
		Технология проектирования интеллектуальных обучающих ostis-систем для школьников\\
		;Технология проектирования интеллектуальных обучающих ostis-систем для студентов по общеобразовательным дисциплинам\\
		;Технология проектирования интеллектуальных обучающих ostis-систем для студентов по профильным дисциплинам\\
		;Технология проектирования интеллектуальных обучающих ostis-систем для магистрантов
		}
		\scnrelfromset{комплекс специализированных технологий}{
		Технология проектирования интеллектуальных обучающих ostis-систем по Математике\\
		;Технология проектирования интеллектуальных обучающих ostis-систем по Искусственному интеллекту
		}
	\scnaddlevel{-1}
}

\scnheader{специализированная ostis-технология}
\scnnote{Приведённый нами перечень \textit{специализированных ostis-технологий} охватывает только некоторые области (фрагменты) \textit{человеческой деятельности}, подлежащие автоматизации с помощью \textit{ostis-технологий} в рамках \textit{Экосистемы OSTIS}.}

\scnheader{Ядро Технологии OSTIS}
\scnnote{Форма реализации \textit{Ядра Технологии OSTIS} (в виде ostis-системы \textit{IMS.ostis}) позволяет:
\begin{scnitemize}
	\item использовать достоинства \textit{Технологии OSTIS} для повышения уровня автоматизации развития самой \textit{Технологии OSTIS} и для существенного повышения темпов такого развития;
	\item приобрести очень важный опыт применения \textit{Технологии OSTIS};
	\item создать центрально ядро \textit{Экосистемы OSTIS}, обеспечивающее поддержку семантической совместимости всех \textit{ostis-систем} и \textit{ostis-сообществ}, входящих в состав \textit{Экосистемы OSTIS}.
\end{scnitemize}
}

\scnendstruct \scninlinesourcecommentpar{Завершили рассмотрение \textit{понятия ostis-технологии}}