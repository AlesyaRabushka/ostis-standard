\scnsegmentheader{Уточнение Понятия Экосистемы OSTIS}

\scnstartsubstruct

\scnidtf{Использование \textit{Технологии OSTIS} для повышения качества и, в частности, уровня автоматизации всех \textit{областей человеческой деятельности}}
\scnidtf{Понятие \textit{Экосистемы OSTIS} как формы реализации \textit{smart-общества}, представляющего собой сеть взаимодействующих людей, интеллектуальных компьютерных систем, "умных"{} домов, "умных"{} предприятий, "умных"{} больниц, "умных"{} учебных заведений, "умных"{} городов, "умных"{} транспортных систем и т.п.}

\scnrelfromset{рассматриваемые вопросы}{
\scnfileitem{Какова архитектура \textit{Экосистемы OSTIS}};
\scnfileitem{Какова архитектура \textit{ostis-сообщества}, входящего в состав \textit{Экосистемы OSTIS}};
\scnfileitem{Как взаимодействуют между собой различные \textit{ostis-сообщества} в рамках \textit{Экосистемы OSTIS}};
\scnfileitem{Как интегрируется \textit{деятельность} различных \textit{ostis-сообществ} и результаты этой \textit{деятельности}};
\scnfileitem{Какова типология \textit{ostis-сообществ} и по каким признакам классификации можно эту типологию проводить};
\scnfileitem{Можно ли опыт автоматизации деятельности в области \textit{Искусственного интеллекта} с помощью \textit{Технологии OSTIS} расширить на все многообразие областей и видов человеческой деятельности};
\scnfileitem{Как выглядит систематизация областей и видов человеческой деятельности};
\scnfileitem{Как осуществляется конвергенция и интеграция различных областей и видов человеческой деятельности};
\scnfileitem{Как взаимодействуют ostis-системы, осуществляющие автоматизацию различных областей видов человеческой деятельности};
\scnfileitem{Как может выглядеть \uline{комплексная} автоматизация всех областей и видов \textit{человеческой деятельности} с помощью \textit{Технологии OSTIS}}
}
\scnrelfromvector{план изложения}{
\scnfileitem{Что такое \textit{Экосистема OSTIS}};
\scnfileitem{Структура \textit{Экосистемы OSTIS}};
\scnfileitem{Что такое \textit{ostis-система}, являющаяся агентом \textit{Экосистемы OSTIS}};
\scnfileitem{Что такое \textit{ostis-сообщетво}, являющееся агентом \textit{Экосистемы OSTIS}};
\scnfileitem{Что такое Проект создания \textit{Экосистемы OSTIS}};
\scnfileitem{Цель создания и основные свойства \textit{Экосистемы OSTIS}};
\scnfileitem{Как структурируется \textit{человеческая деятельность}};
\scnfileitem{Как выглядит \textit{рынок знаний}, реализуемый в рамках \textit{Экосистемы OSTIS}};
\scnfileitem{Чем определяется качество \textit{человеческой деятельности}};
\scnfileitem{Что такое эффективная автоматизация \textit{человеческой деятельности}};
\scnfileitem{Почему повышение эффективности \textit{человеческой деятельности} невозможно без \textit{интеллектуальных компьютерных систем}};
\scnfileitem{Какие достоинства имеет \textit{Экосистема OSTIS}}
}

\bigskip
\scnfragmentcaption

\scnheader{Экосистема OSTIS}
\scntext{вопрос}{Какова структура Экосистемы OSTIS}
\scnexplanation{Популяция
\begin{scnitemize}
\item семантически совместимых
\item эволюционируемых
\item активно взаимодействующих  
\item способных координировать(согласовывать) свою деятельность с другими субъектами
\end{scnitemize}
интеллектуальных компьютерных систем (\textit{ostis-систем}). При этом указанная популяция \textit{ostis-систем} поддерживает децентрализованное управление собственной деятельностью, а также деятельностью людей(пользователей \textit{ostis-систем}) и человеко-машинных сообществ (\textit{ostis-сообществ}), обеспечивая тем самым автоматизацию системной интеграции любых новых субъектов (\textit{ostis-систем}, людей, \textit{ostis-сообществ}) в состав \textit{Экосистемы OSTIS}.
}

\scnrelfromvector{принципы, лежащие в основе}
{
\scnfileitem{\textit{Экосистема OSTIS} представляет собой сеть \textit{ostis-сообществ}};
\scnfileitem{Каждому \textit{ostis-сообществу} взаимно однозначно соответствует \textit{корпоративная ostis-система} этого \textit{ostis-сообщества}, которая:
\begin{scnitemize}
\item обеспечивает координацию деятельности членов соответствующего \textit{ostis-сообщества};
\item является "представителем"{} этого \textit{ostis-сообщества} в других \textit{ostis-сообществах}, членом которых указанное \textit{ostis-сообщество} является.
\end{scnitemize}
};
\scnfileitem{Каждое \textit{ostis-сообщество} может входить в состав любого другого \textit{ostis-сообщества} по своей инициативе. Формально это означает, что \textit{корпоративная ostis-система} первого \textit{ostis-сообщества} является членом другого \textit{ostis-сообщества}.};
\scnfileitem{Каждому специалисту, входящему в состав \textit{Экосистемы OSTIS} ставится во взаимнооднозначное соответствие его \textit{персональный ostis-ассистент}, который трактуется как \textit{корпоративная \mbox{ostis-система}} вырожденного \textit{ostis-сообщества}, состоящего из одного человека.}
}

\scnheader{следует отличать*}
\scnhaselementset{корпоративная ostis-система*
\scnaddlevel{1}
\scnidtf{корпоративная ostis-система данного ostis-сообщества*}
\scnaddlevel{-1};
корпоративная ostis-система
\scnaddlevel{1}\\
\scnrelto{второй домен}{корпоративная ostis-система*}
\scnaddlevel{-1};
член ostis-сообщества*;
персональный ostis-ассистент*
\scnaddlevel{1}
\scnidtf{персональный ostis-ассистент данного специалиста*}
\scnsubset{корпоративная ostis-система*}
\scnaddlevel{-1};
персональный ostis-ассистент
\scnaddlevel{1}\\
\scnrelto{второй домен}{персональный ostis-ассистент*}
\scnsubset{корпоративная ostis-система}
\scnaddlevel{-1}
}

\scnheader{есть сходства*}
\scnhaselementset{Экосистема OSTIS;
ostis-сообщество\\
\scnaddlevel{1}
\scnhaselement{Экосистема OSTIS}
\scnaddlevel{-1}
}
\scnaddlevel{1}
\scnexplanation{\textit{Экосистема OSTIS} является максимальным \textit{ostis-сообществом}, включающим в себя все существующее \textit{ostis-сообщества}}
\scnaddlevel{-1}

\scnheader{Экосистема OSTIS}
\scnidtf{Максимальное \textit{иерархическое ostis-сообщество}, обеспечивающее комплексную автоматизацию \uline{всех} видов \textit{человеческой деятельности}}
\scnidtf{Максимальное ostis-сообщество такое ostis-сообщество, для которого не существует другого \mbox{ostis-сообщества}, содержащее указанное выше ostis-сообщество в качестве своего члена}
\scnidtf{Симбиоз людей и \textit{компьютерных систем} (точнее, \textit{ostis-систем}) являющийся вариантом реализации \textit{smart-общества}}
\scniselement{иерархическое ostis-сообщество}
\scnaddlevel{1}
\scnidtf{такое \textit{ostis-сообщество}, по крайней мере одним из членов которого является некоторое другое \textit{ostis-сообщество}}
\scnsubset{ostis-сообщество}
\scnaddlevel{1}
\scnrelboth{следует отличать}{коллектив ostis-систем}
\scnaddlevel{-1}
\scnaddlevel{-1}{
\scnrelto{основной продукт}{Технология OSTIS}
\scnrelto{вариант реализации}{smart-общество}
\scnheader{агент Экосистемы OSTIS}
\scnidtf{субъект Экосистемы OSTIS}
\scnidtf{субъект, входящий в состав Экосистемы OSTIS}
\scnsuperset{когнитивный агент Экосистемы OSTIS}
\scnsubdividing{индивидуальная ostis-система Экосистемы OSTIS
\scnaddlevel{1}
\scnidtf{индивидуальная ostis-система, входящая в состав Экосистемы OSTIS}
\scnaddlevel{-1};
ostis-сообщество Экосистемы OSTIS\\
\scnaddlevel{1}
\scnsubdividing{
простое ostis-сообщество Экосистемы OSTIS;
иерархическое ostis-сообщество Экосистемы OSTIS
}
\scnaddlevel{-1};
пользователь Экосистемы OSTIS
}

\scnrelfrom{правила поведения}{Правила поведения агентов Экосистемы OSTIS}
\scnaddlevel{1}
\scneqtoset{
\scnfileitem{Согласовывать денотационную семантику всех используемых знаков(в первую очередь \uline{понятий})};
\scnfileitem{Согласовывать терминологию, соответствующую введенным знакам устранять противоречия и информационные дыры};
\scnfileitem{Постоянно бороться с синонимией и омонимией как на уровне sc-элементов (внутренних знаков), так и на уровне соответствующих им терминов и прочих внешних идентификаторов(внешних обозначений)};
\scnfileitem{Каждый агент \textit{Экосистемы OSTIS} по своей инициативе может стать членом любого \textit{\mbox{ostis-сообщества}} \scnbigskip \textit{Экосистемы OSTIS} после соответствующей регистрации}}
\scnaddlevel{-1}

\scnheader{Правила поведения агентов Экосистемы OSTIS}
\scnnote{Существенно подчеркнуть, что все правила функционирования(поведения) агентов в рамках \textit{Экосистемы OSTIS} должны соблюдать не только \textit{ostis-системы}, являющиеся агентами(субъектами) этой \textit{Экосистемы}, но и люди, которые являются её агентами. И здесь возникают очень важные проблемы, обусловленные человеческим фактором. Дело в том, что убедить человека соблюдать правила, пусть даже те которые направлены на максимальную его самореализацию и в совершенствовании которых он может реально участвовать, очень непросто, поскольку любые  подобные правила многими воспринимаются как ограничение их творческой свободы. Другими словами, корректное поведение \textit{ostis-систем} в роли агентов \textit{Экосистемы OSTIS} значительно проще обеспечить, чем корректное поведение людей в качестве таких агентов. Поведение пользователей (естественных агентов) \textit{Экосистемы OSTIS} необходимо внимательно мониторить и контролировать, постоянно способствуя повышению уровня их квалификации как агентов \textit{Экосистемы OSTIS}, а также повышению уровня их мотивации, целенаправленности, самореализации.}

\scnheader{следует отличать*}
\scnhaselementset{
агент Экосистемы OSTIS;
член ostis-сообщества*
}

\scnheader{Экосистема OSTIS}
\scntext{архитектура}{В \textit{Экосистеме OSTIS} можно выделить следующие уровни иерархии:
\begin{scnitemize}
\item индивидуальные компьютерные системы (\textit{индивидуальные ostis-системы} и \textit{люди}, являющиеся конечными пользователями ostis-систем);
\item иерархическая система ostis-сообществ, членами каждого из которых могут быть \textit{индивидуальные ostis-системы}, люди, а также другие \textit{ostis-сообщества};
\item \textit{Максимальное ostis-сообщество} \scnbigspace \textit{Экосистемы OSTIS}, не являющееся членом никакого другого \textit{\mbox{ostis-сообщества}}, входящего в состав \textit{Экосистемы OSTIS}.
\end{scnitemize}
Подчеркнем, что качество \textit{Экосистемы OSTIS} во многом определяется эффективностью взаимодействия каждой \textit{ostis-системы} (в том числе и каждого \textit{ostis-сообщества}), а также каждого \textit{человека} со своей \textit{внешней средой*}, а также качеством(чистотой), самой \textit{внешней среды*}.
Но внешняя среда каждого \textit{субъекта} каждой ostis-системы и каждого человека, входящего в \textit{Экосистему OSTIS} -- это не только \textit{материальная внешняя среда*}, но и \textit{информационная внешняя среда*}, представляющая собой виртуальный распределенный информационный ресурс, являющийся интеграцией(объединением) информации, хранящейся в текущий момент в памяти всех других(остальных) \textit{субъектов}, входящих в \textit{Экосистему OSTIS}. Основной целью \textit{Экосистемы OSTIS} является повышение качества(в том числе чистоты) \textit{информационной внешней среды*} для \uline{всех} \textit{субъектов}, входящих в \textit{Экосистему OSTIS}. Фактически речь идет об \textbf{\textit{Информационной экологии человеческого общества}}.
}

\scnheader{Информационная экология человеческого общества}
\scnnote{Говоря об \textit{Информационной экологии человеческого общества} необходимо заметить следующее. Современные подходы к развитию взаимодействия с информационной средой человеческого общества можно разбить на два направления:
\begin{scnitemize} 
\item на разработку средств приспособления к недостаткам текущего состояния этой среды
\item на устранение этих недостатков путем наведения порядка в устной информационной среде и её систематизации.
\end{scnitemize}
Технология OSTIS и реализация Экосистемы OSTIS целенаправленно и в известной степени радикально ориентирована на второе направление, памятуя искусственный (рукотворный) характер происхождения этой информационной среды.}
\scnendstruct

\scnfragmentcaption

\scnheader{ostis-система, являющаяся агентом Экосистемы OSTIS}
\scntext{Примечание}{Вокруг каждой (!) \textit{ostis-системы} формируется коллектив её разработчиков, несущих ответственность (1) за её качественную эксплуатацию и (2) за её перманентное совершенствование в ходе эксплуатации. При этом интеллект \textit{ostis-системы} должен быть использован для максимально возможной автоматизации указанной деятельности разработчиков (для мониторинга состояния эксплуатируемой \textit{ostis-системы}, для своевременной реакции на подозрительные ситуации или события), а также для координации деятельности разработчиков. Таким образом, каждая \textit{ostis-система} (в том числе каждая персональная \textit{ostis-система}) для своих разработчиков фактически является корпоративной системой. Следовательно, для каждой \textit{ostis-системы} следует отличать (1) координируемый (управляемый, обслуживаемый) ею коллектив конечных пользователей, управляемых \textit{ostis-сообществ} и \textit{ostis-систем} (2) и координируемый ею коллектив разработчиков, в состав которого входит также и "породившая"{} эту \textit{ostis-систему} ostis-система автоматизации проектирования и реализации \textit{ostis-систем} соответствующего класса (эта система должна рассматриваться как полноценный активный член коллектива разработчиков, как \textit{ostis-система}, несущая полную ответственность за все те \textit{ostis-системы}, которые она "породила"). Таким образом, в рамках \textit{Экосистемы OSTIS} можно выделить (1) иерархическую сеть \textit{ostis-систем}, обеспечивающую автоматизированную комплексную реализацию человеческой деятельности с помощью указанных \textit{ostis-систем} и (2) сеть \textit{ostis-систем}, предназначенную для поддержки их эффективной эксплуатации и их совершенствования (постоянного обновления), включая постоянное совершенствование \textit{баз знаний ostis-систем}, осуществляемое не столько специалистами в области \textit{Искусственного интеллекта}, сколько специалистами в соответствующих предметных областях. Ярким примером ostis-систем, совершенствуемых не только специалистами в областях Искусственного интеллекта, являются порталы научно-технических знаний. Подчеркнём, что ответственность за качество сети \textit{ostis-систем}, предназначенной для разработчиков, постоянно совершенствующих эти \textit{ostis-системы} несёт \textit{Консорциум OSTIS.}}

\scnheader{персональный ostis-ассистент}
\scnidtf{ostis-система, являющаяся персональным ассистентом для соответсвующей персоны, входящей в состав Экосистемы OSTIS}
\scnidtf{"официальный"{} представитель соответствующей персоны во всех \textit{ostis-сообществах}, членом которых эта персона является}
\scnidtf{\textit{ostis-система}, являющаяся посредником соответствующей персоны в его взаимодействии с членами всех коллективом (\textit{ostis-сообществ}), в состав которых эта персона входит}

\scnnote{Каждой персоне, входящей в состав \textit{Экосистемы OSTIS} взаимно однозначно соответствует его личный (персональный) ассистент в виде персонального \textit{ostis-ассистента}. Таким образом, количество персональных \textit{ostis-ассистентов}, входящих в состав \textit{Экосистемы OSTIS}, совпадает с числом персон, входящих в состав \textit{Экосистемы OSTIS}.}

\scnnote{Коллектив, состоящий из персоны и соответствующего ей \textit{персонального ostis-ассистента}, фактически является \textit{минимальным ostis-сообществом}, которое также можно назвать терминальным \textit{ostis-сообществом}. Следовательно, персонального ostis-ассистента можно условно(!) считать \textit{корпоративной ostis-системой} минимального \textit{ostis-сообщества}, которая, как и любая другая корпоративная \textit{ostis-система} (1) организует "внутреннюю"{} деятельность соответствующего \textit{ostis-сообщества} и (2) осуществляет "внешнее"{} взаимодействие этого \textit{ostis-сообщества} (как единого целого) с другими агентами (субъектами) в рамках \textit{ostis-сообществ}, находящихся на следующем (более высоком) уровне иерархии.}

\scnnote{Строго говоря, все \textit{ostis-сообщества}, кроме \textit{минимальных ostis-сообществ}, являются не коллективами, состоящими из \textit{персон и ostis-систем}, а коллективами, состоящими только из \textit{ostis-систем}, поскольку формально в неминимальное \textit{ostis-сообщество} входят не персоны, а соответствующие им, организующие их деятельность \textit{персональные ostis-ассистенты}.}

\scnidtf{корпоративная система минимального ostis-сообщества}
\scnrelboth{следует отличать}{корпоративная ostis-система}
	\scnaddlevel{1}
	\scnidtf{корпоративная (центральная) ostis-система либо неминимального ostis-сообщества, либо коллектива ostis-систем}
	\scnaddlevel{-1}
	
\scnrelboth{следует отличать}{ostis-система массового обслуживания индивидуальных пользователей}
	\scnaddlevel{1}
	\scnidtf{ostis-система, не осуществляющая координацию деятельность своих конечных пользователей]}
	\scnidtf{ostis-сообщества массового обслуживания групп пользователей}

\scnidtf{ostis-сообщества, поддерживающая взаимодействия своих конечных пользователей, не только в рамках самостоятельных групп пользователей}

\scnheader{корпоративная ostis-система}
\scnidtf{ostis-система осуществляющая поддержку организации деятельности некоего количества, а также поддержка эволюции (совершенствования) этой деятельности управления проектами}

\scntext{принципы, лежащие в основе}{В основе взаимодействия и взаимосвязи корпоративных ostis-систем, входящих в состав Экосистемы OSTIS лежат следующие принципы:
\begin{scnitemize} 
	\item глубокая конвергенция различных научных дисциплин, формальных моделей различных видов деятельности и все, трансдисциплинарность, что может быть сделано одинаково и должно быть сделано одинаково.
	\item обобщение различных видов деятельности, построение частной иерархии различных моделей.
\end{scnitemize}
}

\scnheader{ostis-система, являющаяся агентом Экосистемы OSTIS}

\scnsuperset{персональный ostis-ассистент}
	\scnaddlevel{1}
	\scnrelboth{следует отличать}{персональный ostis-ассистент*}
		\scnaddlevel{1}
		\scnidtf{быть персональным ostis-ассистентом данной персоны*}
	\scnaddlevel{-1}
\scnaddlevel{-1}

\scnsuperset{корпоративная ostis-система}
	\scnaddlevel{1}
	\scnrelboth{следует отличать}{корпоративная ostis-система*}
		\scnaddlevel{1}
		\scnidtf{быть корпоративной ostis-системой данного ostis-сообщества*}
	\scnaddlevel{-1}
\scnaddlevel{-1}
\scnsuperset{ostis-портал научно-технических знаний}
	\scnaddlevel{1}
	\scnidtf{ostis-портал знаний по некоторой научно-технической дисциплине}}
\scnaddlevel{-1}
\scnsuperset{ostis-система автоматизации проектирования}
\scnsuperset{ostis-система автоматизации производства}
	\scnaddlevel{1}
	\scnidtf{ostis-система управления производством}
\scnaddlevel{-1}
\scnsuperset{ostis-система автоматизации образовательной деятельности}
	\scnaddlevel{1}
	\scnsuperset{обучающаяся ostis-система}
	\scnsuperset{корпоративная ostis-система виртуальной кафедры}
		\scnaddlevel{1}
		\scnidtf{корпоративная ostis-система, обеспечивающая интеграцию деятельности кафедр одинакового профиля и, возможно, различных вузов}
	\scnaddlevel{-1}
\scnaddlevel{-1}
\scnsuperset{ostis-система автоматизации бизнес-деятельности}
\scnsuperset{ostis-система автоматизации управления}
	\scnaddlevel{1}
	\scnsuperset{ostis-система управления проектами соответствующего вида}
	\scnsuperset{ostis-система сенсо-моторной координации при выполнении определённого вида  сложных действий во внешней среде}
		\scnaddlevel{1}
		\scnsuperset{ostis-система управления самостоятельным перемещением робота по пересеченной местности}
	\scnaddlevel{-1}
\scnaddlevel{-1}

\scnheader{обучающая ostis-система}
\scntext{Примечание}{Поскольку качество эксплуатации каждой ostis-системы зависит не только от неё, но и от квалификации пользователя (семантическая совместимость, знания о возможностях системы), каждая ostis-система должна быть способна обучать пользователя знаниям и навыкам эффективного её использования.}

\bigskip
\scnfragmentcaption

\scnheader{ostis-сообщество, являющееся агентом Экосистемы OSTIS}
\scnidtf{это естественный этап перехода (эволюции) творчески ориентированных коллективов людей в принципиально новое существенно более интеллектуальное и, соответственно, более позитивное качество}
\scnidtf{устойчивый фрагмент Экосистемы OSTIS, обеспечивающий (1) комплексную автоматизированную определенной части коллективной человеческой деятельности (2) перманентное повышение эффективности (в т.ч. уровня автоматизации) указанной части человеческой деятельности}
\scnnote{Каждому ostis-сообществу, являющемуся агентом, Экосистемы OSTIS соответствует своя \textit{корпоративная ostis-система}, обеспечивающая интеграцию соответствующих корпоративных знаний и поддержку взаимодействий всех ostis-систем и пользователей, входящих в это ostis-сообщество}

\scnheader{ostis-сообщество}
\scnreltolist{перманентно-решаемая задача}{
	перманентная поддержка семантической совместимости членов ostis-сообщества;
	перманентная поддержка высокого качества базы знаний, доступной всем  членам ostis-сообщества\\
	\scnaddlevel{1}
	\scntext{пояснение}{Имеется в виду поддержка непротиворечивости (корректности отсутствия синонимов, омонимов и противоречий), полноты (отсутствия информационных дыр) и чистоты (отсутствия информационного мусора)}
	\scnaddlevel{-1};
	перманентная поддержка мониторинга эффективности распределения работ между членами ostis-сообщества и контроля исполнительской дисциплины;
	перманентная поддержка мониторинга динамики роста квалификации каждого члена ostis-сообщества
}

\scnheader{коллектив людей}
\scnidtf{человеческое общество}
\scntext{примечание}{Низкий уровень интеллекта современных коллективов людей определяется (1) низким уровнем качества организации общей памяти каждого такого коллектива (общей памяти всех его членов) 
(2) низкий уровень качества знаний, хранимых в этой памяти (как минимум это наличие большого количества синонимов, омонимов и противоречий). Поэтому переход от современных коллективов людей к соответствующим ostis-сообществам существенно повышает уровень интеллекта этих коллективов}

\bigskip
\scnfragmentcaption

\scnheader{Экосистема OSTIS}
\scniselement{ostis-сообщество}
\scnrelto{субъект}{Объединенная человеческая деятельность, осуществляемая на основе Технологии OSTIS}
\scnrelfrom{корпоративная ostis-система}{Корпоративная система Экосистемы OSTIS}
\scnaddlevel{1}
\scnexplanation{Основным назначением \textit{Корпоративной системы Экосистемы OSTIS} является организация общего взаимодействия при выполнении самых различных видов и \textit{областей человеческой деятельности}, которые могут быть либо полностью автоматизированными, либо частично автоматизированными, либо вообще неавтоматизированными. Из этого следует, что база знаний \textit{Корпоративной системы Экосистемы OSTIS} должна содержать \textit{Общую формальную теорию человеческой деятельности}, включающей в себя типологию видов и областей \textit{человеческой деятельности}, а также общую \textit{методологию} этой \textit{деятельности}.}
\scnaddlevel{-1}

\scnheader{Деятельность в области Искусственного интеллекта, осуществляемая на основе Технологии OSTIS}
\scnrelfrom{субъект}{Консорциум OSTIS}
\scnaddlevel{1}
\scnrelfrom{корпоративная ostis-система}{Корпоративная ostis-система Консорциума OSTIS}
\scnaddlevel{-1}
\scnrelfrom{основной продукт}{Экосистема OSTIS}
\scnaddlevel{1}
    \scnrelfrom{частное ostis-сообщество}{Консорциум OSTIS}
    \scnrelfrom{член ostis-сообщества}{Консорциум OSTIS}
    \scnaddlevel{1}
    \scnrelfrom{примечание}{\scnstartsetlocal\\
    \scnheaderlocal{член ostis-сообщества*}
    \scnsubset{частное ostis-сообщество*}
    \scnaddlevel{1}
        \scnidtf{ostis-сообщество, входящее в состав заданного либо непосредственно (в качестве члена), либо в качестве члена члена заданного ostis-сообщества и т. д.}
    \scnaddlevel{-1}
    \scnendstruct}

\scnaddlevel{-2}
\scnidtf{Проект, основной целью (продуктом) которого является создание \textit{Экосистемы OSTIS}}
\scnrelfrom{часто используемый sc-идентификатор}{Проект Экосистемы OSTIS}
\scnidtf{Деятельность, направленная на создание и перманентное развитие \textit{Экосистемы OSTIS}}
\scnidtf{Проект, направленный на проектирование, производство и реинжиниринг \textit{ostis-систем}, входящих в сеть \textit{Экосистемы OSTIS}, а также на проектирование и реинжиниринг Экосистемы OSTIS в целом (как сети \textit{ostis-систем} и их пользователей)}
\scnrelfromlist{подпроект}{Проект IMS.ostis;Проект программной реализации абстрактной sc-машины;Проект разработки универсального sc-компьютера}
\scnnote{В состав \textit{Проекта Экосистемы OSTIS} входит большое количество \textit{проектов} (\textit{подпроектов}*), направленных на \textit{проектирование} и \textit{производство ostis-систем} самого различного назначения}
\scnexplanation{Распространим предлагаемый нами подход к повышению эффективной и человеческой \textit{Деятельности в области Искусственного интеллекта} на всю \textit{Объединенную человеческую деятельность} в целом, т.е. рассмотрим структуру Глобального \textit{ostis-сообщества} (\textit{Экосистемы OSTIS})}
\scntext{эпиграф}{От \textit{Консорциума OSTIS} к \textit{Экосистеме OSTIS}}

\scnheader{Экосистема OSTIS}
\scnnote{Подчеркнем, что \textit{Экосистема OSTIS} является:
    \begin{scnitemize}
        \item с одной стороны, \textit{основным продуктом*} человеческой \textit{Деятельности в области Искусственного интеллекта, осуществляемой на основе Технологии OSTIS} (эту Деятельность мы также будем называть Проектом Экосистемы OSTIS),
        \item а, с другой стороны, \textit{субъектом* Объединенной человеческой деятельности, осуществляемой на основе Технологии OSTIS}.
    \end{scnitemize}

    Особо подчеркнем то, что продуктом человеческой \textit{Деятельности в области Искусственного интеллекта, осуществляемой на основе Технологии OSTIS}, является не просто множество \textit{ostis-систем} различного назначения, а Экосистема, состоящая из \underline{взаимодействующих} \textit{ostis-систем} и их пользователей}
\scnexplanation{Принципиальным является то, что продуктом (результатом применения) \textit{Технологии OSTIS} является не просто множество \textit{ostis-систем}, а целая система, состоящая из \textit{ostis-систем} и их пользователей, взаимодействующих между собой и осуществляющих комплексную автоматизацию всех \textit{видов человеческой деятельности}, а также комплексное повышение уровня эффективности организации человеческой деятельности (и, в частности, повышение уровня автоматизации этой деятельности).}

\bigskip
\scnfragmentcaption

\scnheader{Экосистема OSTIS}
\scnrelfromlist{\scnkeyword{вопрос}}{Каковы основные свойства Экосистемы OSTIS;Какова основная цель создания Экосистемы OSTIS}
\scnidtf{Экосистема ostis-систем и их пользователей}
\scnrelto{общий создаваемый продукт}{Технология OSTIS}
\scnidtf{Расширяемый коллектив эволюционируемых, семантически совместимых и взаимодействующих ostis-систем и их пользователей}
\scniselement{многоагентная система}
\scnidtf{Многоагентная система, агентами которой являются ostis-системы, а также их конечные пользователи и разработчики}

\scnheader{Экосистема интеллектуальных компьютерных систем}
\scnidtf{Smart-сообщество}
\scnidtf{Smart-сообщество интеллектуальных компьютерных систем и людей}
\scnidtf{Интеллектуальная многоагентная система, состоящая из интеллектуальных компьтерных систем и людей}
\scnnote{Многоагентная система может состоять из кибернетических систем, не являющихся интеллектуальными.}
\scnnote{Многоагентная система может состоять из интеллектуальных систем, но сама не быть интеллектуальной. Количество далеко не всегда переходит в нужное качество.}
\scnidtf{Экосистема ostis-систем, а также их разработчиков и пользователей}
\scnidtf{Эволюционирумая сеть ostis-систем, обеспечивающая конвергенцию и интеграцию всех видов человеческой деятельности}

\scnheader{Экосистема OSTIS}
\scnidtf{Глобальная компьютерная сеть ostis-систем, обеспечивающая комплексную автоматизацию всевозможных видов и областей человеческой деятельности и отражающая иерархию уровней этой деятельности}
\scnidtf{Глобальная \textit{многоагентная система}, состоящая из людей и семантически совместимых \textit{интеллектуальных компьютерных систем}, построенных по \textit{Технологии OSTIS}, которые, взаимодействуя между собой и с людьми, обеспечивают существенное повышение уровня автоматизации всех видов \textit{человеческой деятельности} и существенное повышение эффективности человеческого взаимодействия}
\scnidtf{Предлагаемых нами подход к реализации smart-общества}
\scnidtf{Smart-общество, построенное на основе Технологии OSTIS}
\scnidtf{следующий этап развития человеческого общества, обеспечивающий существенное повышение уровня общественного (коллективного) интеллекта путем преобразования человеческого общества в экосистему, состоящую из людей и семантически совместимых интеллектуальных систем}
\scntext{обоснование}{Необходимо обеспечить не только повышение уровня автоматизации человеческой деятельности (как информационной (умственной), так и "физической"{}), но и существенное повышение уровня интеллекта человеческого общества как социальной  кибернетической системы путем создания многоагентной кибернетической системы, сосотоящей из \textit{интеллектуальных компьютерных систем} и людей и имеющей \textit{высокий уровень интеллекта}. Человечества пока не умеет создавать инетеллектуальные сообщества (коллективы) людей и, тем более, интеллектуальные общества людей и интеллектуальных компьютерных системы.
Уровень интеллекта каждого такого сообщества обычно определяется уровнем интеллекта его руководителя (лица, принимающего решение).А надо, чтобы уровень интеллекта сообщества был результатом интеграции интеллектуального потенциала всех его членов. При этом следует помнить, что интеллект определяется не только и не столько множеством решаемых задач, а \uline{скоростью} расширения этого множества.}

\scnnote{Предметом инженерной деятельности в области \textit{искусственного интеллекта} следует считать не множество \textit{интеллектуальных компьютерных систем} (например, \textit{ostis-систем}), а весь комплекс взаимодействующих между собой \textit{интеллектуальных компьютерных систем}. Назовём такой комплекс \textit{Экосистемой интеллектуальных компьютерных систем} (в нашем случае – это Экосистема OSTIS – Экосистема взаимодействующих \textit{ostis-систем}) здесь важно построить архитектуру таковой экосистемы, в основе которой должна лежать комплексная формальная модель всевозможных видов человеческой деятельности, автоматизируемых с помощью интеллектуальных компьютерных систем (ostis-систем). Указання комплексная модель человеческой деятельности является необходимой основой создания smart-общества (общества 5.0)}

\scnrelto{общий создаваемый продукт}{Технология OSTIS}
\scnidtf{Общий (объединенный, интегрированный) продукт использования \textit{Технологии OSTIS}, представляющий собой глобальную сеть \textit{ostis-систем}, обеспечивающий комплексную автоматизацию и интеграцию всевозможных \textit{видов человеческой деятельности} и, в частности, включающий в себя (в виде соответствующего \mbox{\textit{ostis-сообщества}}) \textit{консорциум OSTIS}, т.е. инфраструктуру, направленную на перманентное развитие \textit{Технологии OSTIS} (как Ядра Технологии OSTIS, так и иерархического семейства \textit{специализированных ostis-технологий})}
\scntext{следовательно}{\textit{Экосистема OSTIS} представляет собой саморазвивающуюся сеть ostis-систем}

\scnexplanation{Сверхзадачей \textit{Экосистемы OSTIS} является не просто комплексная автоматизация всех \textit{видов человеческой деятельности} (разумеется, только тех видов деятельности, автоматизация которых целесообразна), но и существенное повышение уровня интеллекта различных человеческих (точнее человеко-машинных) сообществ и всего человеческого общества в целом. Это потребует соблюдения ряда требований, предъявляемых не только к \textit{интеллектуальным компьютерным системам}, но и к людям, входящим в состав \textit{Экосистемы OSTIS}}

\scnexplanation{\textit{Экосистема OSTIS} представляет собой открытый коллектив взаимодействующих интеллектуальных систем, состав которого входят \textit{ostis-системы} и их пользователи (конечные пользователи и разработчики,участвующие в совершенствование этих \textit{ostis-систем}). Особое место среди \textit{ostis-систем}, входящих в состав \textit{экосистемы OSTIS}, занимают \textit{корпоративные ostis-системы}, через которое осуществляется координация и эволюция деятельности некоторых групп \textit{ostis-систем} и их пользователей. Основная цель корпоративных \textit{ostis-систем} -- локализовать базы знаний указанных групп ostis-систем, перевести их из статуса виртуальных в статус реальных и автоматизировать их эволюцию.}

\scnidtf{Сообщество ostis-систем или людей, обеспечивающее принципиально новый уровень автоматизации человеческой деятельности и принципиально \uline{новый уровень интеллекта человеческого общества}}

\scnheader{Экосистема OSTIS}
\scnnote{Очень важно проектировать не только саму \textit{Экосистему OSTIS}, ну и процесс \uline{поэтапного перехода} от современной глобальной сети \textit{компьютерных систем} к глобальной сети \textit{ostis-систем} (т.е. к \textit{Экосистеме OSTIS}). В рамках такого переходного периода \textit{ostis-системы} могут выполнять роль системных интеграторов различных ресурсов и сервисов, реализованных современными \textit{компьютерными системами}, поскольку уровень интеллекта \textit{ostis-систем} позволяет им с любой степенью детализации специфицировать интегрируемые \textit{компьютерные системы} и, следовательно, достаточно адекватно "понимать"{}, что знает и/или умеет каждая из них, а также достаточно качественно координировать их деятельность и обеспечивать "релевантный"{} поиск нужного ресурса и сервиса. Кроме того системы могут выполнять роль интеллектуальных help-систем -- помощников и консультантов по вопросам эффективной эксплуатации различных \textit{компьютерных систем} со сложными функциональными возможностями, имеющими пользовательский интерфейс с нетривиальной семантикой и использующимися в сложных предметных областях. Такие интеллектуальные help-системы можно сделать интеллектуальными посредниками между соответствующими компьютерными системами их пользователями. При этом пользователь может работать одновременно и с help-системой и с соответствующей эксплуатируемой компьютерной системой, консультируясь с help-системой в затруднительных для него ситуациях. Основными недостатками такого варианта является то,что: (1) пользователь должен использовать два разных интерфейса и (2) help-система не может мониторить деятельность пользователя и, следовательно, пользователь сам должен сообщать системе о своих затруднительных ситуациях. Указанные недостатки можно устранить, если компьютерную систему, которая построена по современным технологиям и эксплуатация которой нуждается в качественной консультационной (help-овой) поддержке, интегрировать с соответствующей ей help-системой, построенной по стандартам технологии OSTIS, так, чтобы пользовательским интерфейсом такой интегрированной системы стал пользовательский интерфейс, соответствующий стандартам технологии OSTIS и важнейшим достоинством которого является чёткая формализация семантики всех элементов управления пользовательским интерфейсом. Благодаря этому взаимодействие пользователя с пользовательским интерфейсом ostis-систем становится, во-первых, осмысленным и, во-вторых, позволяющим легко переносить опыт интерфейсного взаимодействия с одной ostis-системы на другую ostis-систему.
}

\bigskip
\scnfragmentcaption

\scnheader{Объединенная человеческая деятельность}
\scnidtf{максимальная область человеческой деятельности}
\scnidtf{вся человеческая деятельность}
\scnidtf{человеческая деятельность в целом}
\scnidtf{объединение всевозможных областей человеческой деятельности}
\scniselement{человеческая деятельность} 
	\scnaddlevel{1}
	\scnidtf{область человеческой деятельности}
	\scnidtf{система целенаправленных действий некоторого количества(возможно одного) людей над некоторыми объектами с помощью некоторых инструментов}
	\scnsubset{деятельность}
		\scnaddlevel{1}
		\scnidtf{трудно выполнимое сложное действие}
		\scnidtf{область деятельности}
		\scnidtf{система целенаправленных действий некоторых (возможно одного) субъектов над некоторыми объектами с помощью некоторых инструментов}
		\scnaddlevel{-1}
	\scnaddlevel{-1}
	
\scnheader{следует отличать*}
\scnhaselementset{Объединенная человеческая деятельность\\
	\scnaddlevel{1}
	\scnidtf{человеческая деятельность в целом}
	\scnidtf{максимальная область человеческой деятельности}
	\scnaddlevel{-1}
;область человеческой деятельности\\
	\scnaddlevel{1}
	\scnidtf{фрагмент (часть, раздел) человеческой деятельности}
	\scnidtf{человеческая деятельность}
	\scnidtf{деятельность, осуществляемая либо одним человеком (индивидуальная человеческая деятельность), либо коллективом людей}
	\scnsubset{деятельность}
		\scnaddlevel{1}
		\scnsubset{действие}
		\scnaddlevel{-1}
	\scnsuperset{индивидуальная человеческая деятельность}
	\scnsuperset{коллективная человеческая деятельность}
	\scnidtf{множество всевозможных областей человеческой деятельности}
	\scnaddlevel{-1}
;вид человеческой деятельности\\
	\scnaddlevel{1}
	\scnidtf{класс однотипных областей человеческой деятельности, которому можно поставить в соответствие некоторую технологию}
	\scnsubset{вид деятельности}
		\scnaddlevel{1}
		\scnsubset{класс действий}
		\scnaddlevel{-1}
	\scnaddlevel{-1}
}

\scnheader{следует отличать*}
\scnhaselementset{Объединенная человеческая деятельность\\
	\scnaddlevel{1}
	\scnidtf{максимальный процесс человеческой деятельности, включающий в себя деятельность всех людей и всех сообществ}
	\scnaddlevel{-1}
;человеческая деятельность\\
	\scnaddlevel{1}
	\scnidtf{множество всевозможных целостных, целенаправленных фрагментов \textit{Объединенной человеческой деятельности}}
	\scnnote{На данном множестве заданы такие отношения, как \textit{часть*}, \textit{декомпозиция*}. Т.е. конкретный экземпляр (элемент) данного множества может быть \textit{частью*} (входить в состав) другой конкретной человеческой деятельности. Более того, целесообразно рассматривать достаточно сложную иерархию процессов человеческой деятельности.}
	\scnidtf{конкретный процесс человеческой деятельности}
	\scnidtf{бизнес-процесс}
	\scnidtf{деятельность, основными субъектами которой являются люди и различные сообщества людей}
	\scnsubset{деятельность}
		\scnaddlevel{1}
		\scnsubset{действие}
		\scnaddlevel{-1}
	\scnnote{Если для автоматизации человеческой деятельности используются интеллектуальные компьютерные системы, то эти системы также становятся достаточно самостоятельными полноценными субъектами этой деятельности, мнение которых обязательно принимается во внимание, но при этом интеллектуальные компьютерные системы не становятся основными субъектами человеческой деятельности.}
	\scnhaselement{объединенная человеческая деятельность}
		\scnaddlevel{1}	
		\scnidtf{максимальная человеческая деятельность, для которой не существует никакой другой конкретной человеческой деятельности, частью* которой указанная Максимальная человеческая деятельность является.}
		\scnaddlevel{-1}
	\scnnote{каждая конкретная человеческая деятельность (каждый бизнес-процесс) может быть:
	\begin{scnitemize}
		\item либо полностью автоматизирована – от человека требуется только корректно сформулировать соответствующую команду (цель инициируемого действия);
		\item либо автоматизирована, но требующая от человека управления функционированием соответствующего одного инструментального средства;
		\item либо состоящая из фрагментов (подпроцессов, частных бизнес-процессов), некоторые из которых автоматизированы, а некоторые нет;
		\item либо полностью неавтоматизирована (т.е. выполняется "вручную"{})
	\end{scnitemize}}
	\scnnote{Когда речь идет о спецификации конкретной человеческой деятельности (конкретного бизнес-процесса), важно провести четкую грань между теми действиями, которые выполняются автоматически (в том числе интеллектуальными компьютерными системами), и действиям, которые выполняются людьми "вручную"{} - это как минимум действия по "формулировке"{} команд, которые адресуются соответствующим инструментальным средствам (язык и, соответственно, интерфейс формулировки таких команд для разных инструментальных средств может сильно отличаться).
Отсутствие унификации языка взаимодействия (интерфейсы) между людьми и различными инструментальными средствами (автомобилями, станками, холодильниками, газовыми плитами, микроволновками, компьютерными системами различного назначения) существенно снижает комплексную эффективность автоматизации человеческой деятельности, т.к. вынуждает людей тратить много времени на усвоение не сути (смысла) автоматизации, а формы (синтаксиса) своей деятельности по организации использования различных средств автоматизации.}
	\scnaddlevel{-1}
;вид человеческой деятельности\\
	\scnaddlevel{1}
	\scnidtf{класс (множество однотипных) процессов человеческой деятельности}
	\scnidtf{класс бизнес-процессов}
	\scnidtf{множество всевозможных классов бизнес-процессов}
	\scnsubset{вид деятельности}
	\scnnote{Каждый конкретный вид человеческой деятельности (т. е. каждый элемент множества "\textit{вид человеческой деятельности}"{}) является \textit{подмножеством*} множества "человеческая деятельность"{}.
Каждому виду человеческой деятельности соответствует своя \textit{технология человеческой деятельности}, т.е. свой набор \textit{методов} и \textit{средств}, обеспечивающих выполнение каждой конкретной деятельности, принадлежащей этому виду.}
	\scnaddlevel{-1}
;область человеческой деятельности\\
	\scnaddlevel{1}
	\scnsubset{человеческая деятельность}
	\scnidtf{достаточно крупный фрагмент человеческой деятельности}
	\scnidtf{раздел человеческой деятельности}
	\scnaddlevel{-1}
}

\scnheader{Экосистема OSTIS}
\scnnote{Содержательную типологию \textit{ostis-систем}, входящих в состав \textit{Экосистемы OSTIS} следует проводить на основе глубокого анализа содержательной структуры человеческой деятельности, требующей взаимодействия человека с другими людьми и даже с организациями. Очевидно, что эффективность такого взаимодействия во многом определяется качеством организации информационного взаимодействия, уровнем взаимопонимания, уровнем квалификации участников, оперативностью получения качественной консультативной помощи по любому (!) вопросу.}

\scnheader{вид человеческой деятельности, продуктом которой является информационная модель некоторого объекта или класса объектов}
\scnidtf{вид человеческой деятельности, направленной на построение описания (спецификации) некоторого объекта исследования или класса таких объектов}
\scnsubset{вид человеческой деятельности}
\scnhaselement{научно-исследовательская деятельность}
	\scnaddlevel{1}	
	\scnhaselement{Научно-исследовательская деятельность в области Искусственного интеллекта}
		\scnaddlevel{1}
		\scnidtf{разработка Общей теории интеллектуальных систем}
		\scnaddlevel{-1}
	\scnaddlevel{-1}
\scnhaselement{разработка теории искусственных объектов заданного класса}
	\scnaddlevel{1}
	\scnhaselement{Разработка Общей теории интеллектуальных компьютерных систем}
		\scnaddlevel{1}
		\scnidtf{разработка стандарта интеллектуальных компьютерных систем}
		\scnaddlevel{-1}
	\scnaddlevel{-1}
\scnhaselement{разработка теории проектирования искусственных объектов заданного класса}
	\scnaddlevel{1}
	\scnidtf{разработка системы проектных действий для искусственных объектов (артефактов) заданного класса}
	\scnhaselement{разработка теории проектирования интеллектуальных компьютерных систем}
		\scnaddlevel{1}
		\scnidtf{разработка стандарта организации коллективных проектных действий для проектирования интеллектуальных компьютерных систем}
		\scnaddlevel{-1}
	\scnaddlevel{-1}
\scnhaselement{разработка теории производства спроектированных искусственных объектов заданного класса}
	\scnaddlevel{1}
	\scnidtf{разработка стандарта системы производственных действий, методов и инструментов, обеспечивающих производство спроектированных артефактов заданного класса}
	\scnhaselement{Разработка Теории производства спроектированных интеллектуальных компьютерных систем}
	\scnaddlevel{-1}
\scnhaselement{проектирование искусственного объекта заданного класса}
	\scnaddlevel{1}
	\scnidtf{проектная деятельность, направленная на построение такой информационной модели (спецификации) искусственно создаваемого объекта (артефакта) заданного класса, которой достаточно для производства этого объекта}
	\scnsuperset{проектирование конкретной интеллектуальной компьютерной системы}
		\scnaddlevel{1}
		\scnidtf{процесс проектирования некоторой компьютерной системы по заданной технологии проектирования}
		\scnaddlevel{-1}
	\scnaddlevel{-1}
\scnexplanation{Данный вид человеческой деятельности характерен следующими особенностями:
\begin{scnitemize}
	\item очень часто продукт этой деятельности (создаваемая информационная конструкция) имеет высокую степень сложности и, следовательно, указанная деятельность не может быть индивидуальной, а несет коллективный характер;
	\item основными факторами качественного коллективного построения сложной информационной конструкции являются семантическая совместимость (взаимопонимание) авторов, а также согласованность их действий;
	\item важнейшим направлением автоматизации коллективной деятельности, объект и продукт которой представляет собой сложную информационную конструкцию, является автоматизация редактирования коллективно создаваемого информационного объекта, а также автоматизация обеспечения семантической совместимости и согласованности продуктов индивидуальной деятельности всех соавторов;
	\item указанную автоматизацию легко реализовать с помощью корпоративной интеллектуальной компьютерной системы, объединяющей всех соавторов создаваемого информационного объекта и снабженной мощными средствами поддержки коллективного проектирования различных разделов базы знаний этой системы. Примерами таких систем являются интеллектуальные порталы различного вида знаний.
\end{scnitemize}}
\scnheader{вид человеческой деятельности, продуктом которой является информационная модель некоторого объекта или класса объектов}
\scnexplanation{Здесь речь идет о коллективной человеческой деятельности, которая принципиально не может быть полностью автоматизирована (исследовательская, проектная), то основной проблемой ее автоматизации являются
\begin{scnitemize}
	\item недостаточный уровень семантической совместимости и взаимопонимания между людьми и отсутствие сознания серьезности этой проблемы;
	\item недостаточный уровень договоренности и отсутствия понимания серьезности этой проблемы;
	\item отсутствие четкой методики согласования точек зрения и отсутствие понимая серьезности этой проблемы.
\end{scnitemize}
Интеллектуальные компьютерные системы могут и должны создать корпоративную среду для решения этих проблем.
По сути это не что иное, как поддержка коллективного проектирования соответствующих разделов баз знаний интеллектуальной компьютерной системы, реализуемая на \uline{семантическом уровне}, когда интеллектуальная компьютерная система становится самостоятельным полноправным участником деятельности, в обязанности которого входит:
\begin{scnitemize}
	\item анализ семантической совместимости точек уровня различных участников, 
	\begin{scnitemizeii}
		\item выявление противоречий и альтернатив 
	\end{scnitemizeii}
	\item фиксация авторства
	\item отмена современной формы представления интеллектуального продукта (статьи, книги, документы)
\end{scnitemize}
Недостаточно высокий уровень семантической согласованности используемых понятий приводит к огромному количеству искусственно создаваемых противоречий.
При этом следует отличать семантические противоречия (например, синонимию вводимых знаков) и, соответственно, методику их устранения или разногласия по поводу системы вводимых понятий от терминологических разногласий, методика устранения которых может и должна быть максимально простой и лишенной эмоциональной окраски. Излишнее увлечение терминологическими спорами существенно тормозит творческий процесс, но и несерьезное отношение к постоянному совершенствованию и соблюдению \uline{правил} построения терминов также недопустимо.}

\bigskip
\scnfragmentcaption

\scnheader{Рынок знаний, реализуемый в рамках Экосистемы OSTIS}
\scnexplanation{Важнейшим видом предметно-независимой человеческой деятельности, осуществляемой в рамках \textit{Экосистемы OSTIS} является перманентный реинжиниринг всех \textit{ostis-систем}, входящих в \textit{Экосистему OSTIS}. Указанная деятельность должна быть направлена на перманентную и быструю эволюцию всех ostis-систем и, самое важное, на эволюцию \textit{Экосистемы OSTIS} в целом. Особо следует подчеркнуть, что эволюция \textit{ostis-систем} и \textit{Экосистемы OSTIS} в целом представляет собой весьма сложный творческий, коллективный процесс, который принципиально может быть автоматизирован \uline{только частично}. При этом от людей, участвующих в этом процессе требуется высокая квалификация, высочайшая системная культура на уровне глубокого знания общей теории систем, высокая математическая культура -- культура формализации, высокая культура конвергенции (обнаружения сходств, доведение их до формальных аналогий), высокая культура глубокой интеграции, высокий уровень договороспособности.

Кроме указанных требований необходим высочайший уровень мотивации к тому, чтобы эволюция отдельных компонентов \textit{Экосистемы OSTIS} (в частности, отдельных \textit{ostis-систем}) не осуществлялась в ущерб эволюции \textit{Экосистемы OSTIS} в целом, например, путём привнесения эклектичности, многообразия форм решения похожих проблем, путем ослабления фундаментального требования \uline{максимально возможной простоты} и логичности принципов, лежащих в основе Экосистемы OSTIS.

Существенно подчеркнуть, что эволюция \textit{ostis-систем} и \textit{Экосистем OSTIS} в целом сводится к коллективному реинжинирингу \textit{баз знаний ostis-систем}, что, в свою очередь сводится к:
	\begin{scnitemize}
	\item "ручной"{} генерации предлагаемых дополнительных (новых) знаний в базу знаний указываемой \mbox{ostis-системы};
	\item "ручной"{} генерации предлагаемых изменений текущего состояния
базы знаний указываемой ostis-системы;
	\item автоматическому назначению компетентных и заинтересованных
рецензентов;
	\item "ручному"{} рецензированию каждого поступившего предложения,
результатом чего является:
		\begin{scnitemizeii}
		\item либо полное одобрение;
		\item либо полное неодобрения с предлагаемой аргументацией;		
		\item либо детальная рекомендация доработки, предположения;		
		\end{scnitemizeii}
	\item автоматическому назначению достаточно широкого круга компетентных и заинтересованных специалистов для утверждения
поступившего предложения (после получения одобрения от всех назначенных экспертов);
	\item автоматическому принятию решения по одобрению поступившего
предложения на основании мнения всех привлечённых экспертов и
специалистов.
	\end{scnitemize}

Таким образом в \textit{базе знаний} каждой \textit{ostis-системы} можно (и нужно!)
фиксировать весь процесс обсуждения каждого поступившего предложения с
указанием (1) моментов времени всех привлечённых событий; (2) участников каждого события (авторов предложений, авторов рецензий участников голосования).

Кроме того, каждая \textit{ostis-система}, анализируя процесс использования
хранимых ею знаний в процессе эксплуатации, может оценивать частоту
непосредственного и опосредованного использования этих знаний, т.е. может оценить степень востребованности этих знаний.	

Следовательно, в перспективе \textit{Экосистема OSTIS} может с достаточно
высокой степенью \uline{объективности} может оценивать объем и значимость вклада каждого специалиста в развитие распределенной базы знаний \textit{Экосистемы OSTIS}. Это является фундаментальной основой для
формирования достаточно объективного (честного) \textit{рынка знаний}.}

\scnheader{Рынок знаний, реализуемый в рамках Экосистемы OSTIS}
\scntext{правило для авторов в рамках Экосистемы OSTIS}{знания, \uline{предлагаемые} для рецензирования, согласования,
утверждения и публикации в базе знаний соответствующей ostis-системы должны быть специфицированы (указана ostis-
система, атомарный раздел базы знаний, дата и время, автор,новый вид публикации, рынок знаний,
защита авторского права не на уровне документов, а на уровне смысла.}
\scntext{коллективное совершенствование базы знаний}{Абсолютно идеальных решений (в том числе проектных) не бывает. Поэтому (1) не надо бояться ошибок и (2) надо минимизировать степень ошибочности за счёт (2.1) \uline{оперативности} исправления ошибок и (2.2) повышения качества (уровня) анализа при принятии решения путем (2.2.1) \uline{коллективного} характера экспертизы,
(2.2.2) достаточного количества привлекаемых
экспертов и (2.2.3) учёта уровня осведомленности
(квалифицированности и  погруженности в
соответствующую предметную область и
онтологию). Для каждого эксперта, привлекаемого к принятию решения нужен постоянно уточняемый, по объективным критериям коэффициент осведомленности-авторитетности каждого эксперта к каждой конкретной предметной области.}
\scntext{правила редактирования Общей базы знаний коллектива интеллектуальной системы}{
	\begin{scnitemize}
	\item Если Вы в рамках базы знаний разрабатываемой Вами ostis-системы хотите ввести знак новой ранее не описываемой сущности, то Вы должны проверить, что эта сущность
действительно не описывалась в рамках виртуальной базы знаний всей Экосистемы OSTIS
		\begin{scnitemizeii}
		\item Если в результате такой проверки выяснилось, что указанная сущность уже рассматривалась, то Вы должны использовать
введенный ранее основной внешний идентификатор этой сущности (Если он Вам не нравится, можете предложить, но пока не использовать, свой)
		\item Если сущность не рассматриваласть, нужно специфицировать, связать с семантически
близким (особенно для понятий)
		\end{scnitemizeii}
	\end{scnitemize}

От толковых словарей и энциклопедий -- к стройной
семантической сети таких спецификации \uline{всех} описываемых сущностей, которые позволяют установить (желательно автоматически) наличие или отсутствие в рамках технического состояния базы
знаний синонимичного знака для любого нового знака, вводимого в базу знаний.}
\scntext{cтруктура качественной спецификации}{
Нужно стремиться:
	\begin{scnitemize}
	 	\item к однозначности такой спецификации;
		\item "координаты"{} в пространстве декомпозиций
		\item к семантической близости;
		\item сходства, отличия;
	\end{scnitemize}}
	

\scnheader{качество человеческой деятельности}
\scnidtf{качество деятельности человеческого общества}
\scnexplanation{Поскольку человеческого общество в целом является кибернетической системой, (которая принадлежит классу иерархических многоагентных систем, качество деятельности человеческого общества можно оценивать по критериям качества кибернетических систем.\\
На основании этих критериев можно оценивать:
	\begin{scnitemize}
	\item качество информационной среды, формируемой человеческим обществом, т.е. качество накапливаемой и общедоступной информации;
	\item качество текущего состояния общечеловеческих знаний;
	\item качество методов и технологий, используемых для решения задач как в рамках накопленных человечеством знаний, так и в
рамках внешней среды человеческого общества;
	\item качество организаций человеческой деятельности в целом;
	\item обучаемость (темпы эволюции) человеческого общества в целом.
	\end{scnitemize}
	
Современный этап развития науки и техники характерен тем, что при оценке качества научно-технических результатов акцентируется внимание на новизне результатов, на их \uline{отличиях} от текущего положения дел. Это создает почву и для
имитации этой новизны и для увеличения барьеров между различными дисциплинами, что существенно препятствует конвергенции и интеграции различных дисциплин. Указанная конвергенция и инеграция, в частности, необходима для \uline{комплексной} автоматизации \uline{всех} видов человеческой
деятельности в рамках smart-общества. Очевидно, что основной такой комплексной автоматизации должна быть \textbf{\textit{Общая формальная теория человеческой деятельности}}.}

\scnheader{уровень конвергенции и интерации человеческой деятельности и её результатов}
\scnrelto{свойство-предпосылка}{Качество человеческой деятельности}
\scnexplanation{Повышение уровня конвергенции и интеграции различных видов человеческой деятельности и, соответственно, результатов
этой деятельности является важнейшим фактором (важнейшим направлением) повышения качества
(эффективности) человеческой деятельности, а, следовательно, и качества самого человеческого общества как сложной распределенной социотехнической кибернетической системы.}
\scntext{вопрос}{Что является главным препятствием существенному повышению уровня конвергенции и интеграции человеческой деятельности.}
	\scnaddlevel{1}
	\scntext{ответ}{Главным препятствием повышению уровня конвергенции и интеграции человеческой деятельности является то, что на текущем этапе эволюции человеческого общества основным механизмом эволюции является конкуренция. Конкуренция предполагает противопоставление результатов своей деятельности результатам конкурентов. т.е. акцентирует внимание на отличиях, новизне, преимуществах своих результатов по отношению к результатам своих конкурентов. При этом мысль о целесообразности объединения усилий со своими конкурентами чаще обусловлена стремлением повысить уровень конкурентоспособности и прибыли по отношению к другим более сильным конкурентам и значительно реже обусловлена искренним стремлением получить более качественный результат.\\
	Таким образом, повышение уровня конвергенции и интеграции всех видов человеческой деятельности требует весьма сложного перехода от использования механизма конкуренции в её современном виде к созданию мощной технологической основы, обеспечивающей широкое взаимовыгодное сотрудничество и гарантированные возможности самореализации каждого человека и каждого коллектива. Фундаментом указанной технологической основы может и должен стать общечеловеческий рынок знаний, который построен на базе сети интеллектуальных компьютерных систем и в рамках которого фиксируется и объективно оценивается значимость вклада каждого человека и каждого коллектива.}
	\scnaddlevel{-1}

\bigskip
\scnfragmentcaption

\scnheader{автоматизация человеческой деятельности}
\scnrelfromlist{вопрос}{
    \scnfileitem{В чем заключаются проблемы комплексной автоматизации человеческой деятельности}
    ;\scnfileitem{Как автоматизировать участие человеческая одновременно в нескольких разных действиях (разных областях деятельности), принадлежащих в общем случае разным видам деятельности}}

\scnheader{Экосистема OSTIS}
\scnnote{Во многом разработка принципов организации взаимодействия интеллектуальных компьютерных систем (ostis-систем) и людей, входящих в состав \textit{Экосистемы OSTIS} должна опираться на анализ того, как взаимодействие осуществляется между людьми, когда основные проблемы возникают из-за:
\begin{scnitemize}
    \item отсутствия взаимопонимания (семантической совместимости),
    \item противоречий между целями различных субъектов,
    \item имитации целенаправленных действий,
    \item нарушений каких-либо соглашений, договоренностей и даже законов (правил поведения и обязанностей).
\end{scnitemize}
Общая (общедоступная) \textit{база знаний} всей \textit{Экосистемы OSTIS}, а также корпоративная \textit{база знаний} каждого \textit{ostis-сообщества}, входящего в состав \textit{Экосистемы OSTIS}, является распределенной, но при этом обязательно целостной. Она поддерживается группой специальных \textit{ostis-систем}, являющихся \textit{порталами знаний} по самым различным областям. Для \textit{Технологии OSTIS} роль такого \textit{портала знаний} выполняет \textit{Метасистема IMS.ostis}. \textit{Экосистема OSTIS} представляет собой многоагентную социотехническую систему, в которой каждая \textit{индивидуальная ostis-система}, входящая в состав \textit{Экосистемы OSTIS}, каждый пользователь указанных \textit{ostis-систем}, а также каждое \textit{ostis-сообщество}, входящее в Экосистему, является её самостоятельным \textit{субъектом*} (когнитивным агентом). При этом каждый субъект \textit{Экосистемы OSTIS} должен соблюдать определенные правила, обеспечивающие качественную (эффективную) эксплуатацию и эволюцию \textit{Экосистемы OSTIS}.}
\scntext{резюме}{Сама идея комплексной автоматизации всех видов человеческой деятельности предполагает необходимость:
\begin{scnitemize}
    \item разработки достаточно детальных формальных теорий всех видов человеческой деятельности, причем, теорий, доведенных до уровня разделов баз знаний соответствующих корпоративных компьютерных систем -- это, фактически, строгое описание стандартов различных видов человеческой деятельности, доведенное до такого уровня, чтобы соответствующая корпоративная система "\underline{понимала}"{} , в какой деятельности она участвует, и могла быть активным и полноценным субъектом (участником) этой деятельности;
    \item серьезного отношения и научного подхода к формализации различных видов человеческой деятельности, к разработке самых различных стандартов;
    \item глубокой конвергенции различных областей (разделов) человеческой деятельности и, соответственно, различных видов человеческой деятельности, осуществляемой в условиях достигнутого уровня автоматизации этой деятельности. Это предполагает необходимость рассмотрения каждого вида человеческой деятельности в контексте \textbf{\textit{Общей теории человеческой деятельности}} в условиях \underline{текущего} состояния уровня автоматизации этой деятельности;
    \item обеспечения высоких темпов эволюции и, следовательно, высокого уровня \underline{гибкости} \textit{Общей теории человеческой деятельности} и теорий (стандартов) каждого \textit{вида деятельности} в силу их большой зависимости от текущего уровня автоматизации;
    \item автоматизации взаимодействия субъектов не только внутри каждой области (раздела) человеческой деятельности, но и между этими областями (разделами), что предполагает автоматизацию \underline{представительства} каждой области (раздела) человеческой деятельности во множестве всех таких областей;
    \item понимания того, что эффективность человеческой деятельности во многом определяется скоординированностью, адекватностью, грамотностью поведения каждого субъекта. Поэтому автоматизация человеческой деятельности должна быть направлена на более глубокую координацию этой деятельности на основе учета смысла и целей этой деятельности. А это "превращает"{} средства автоматизации в полноценных субъектов коллективной деятельности
\end{scnitemize}}
\bigskip
\bigskip
\scnheader{автоматизация человеческой деятельности}
\scnnote{Рассмотрение комплексной автоматизации человеческой \textit{деятельности в области Искусственного интеллекта} естественным образом можно расширить (обобщить) до рассмотрения комплексной автоматизации человеческой деятельности в целом.}
\scntext{проблемы текущего состояния}{Для того, чтобы обеспечить качественную автоматизацию любой \textit{области человеческой деятельности} с помощью \textit{интеллектуальных компьютерных систем}, необходимо построить \textit{формальную модель} этой области деятельности и довести эту модель до такого уровня формализации, чтобы она могла стать частью \textit{базы знаний интеллектуальной компьютерной системы}, используемой для автоматизации указанной \textit{области человеческой деятельности}. Очевидно, что, чем субъекты, участвующие в какой-либо коллективной деятельности, (люди и интеллектуальные компьютерные системы) лучше понимают суть, цели, критерии качества указанной коллективной деятельности, тем выше качество выполнения этой деятельности. В состав формальной модели автоматизируемой области человеческой деятельности входит спецификация:
\begin{scnitemize}
    \item объектов деятельности;
    \item среды деятельности;
    \item инструментов (инструментальных средств) деятельности;
    \item субъектов деятельности;
    \item текущего состояния деятельности (как процесса) -- в том числе, спецификация действий (целей, задач), выполняемых в текущий момент;
    \item формулировка различного рода учитываемых закономерностей -- в том числе, правил поведения субъектов деятельности;
    \item спецификация всех используемых субъектами деятельности методов выполнения сложных действий (решения задач).
\end{scnitemize}
Примерами автоматизируемых областей человеческой деятельности являются:
\begin{scnitemize}
    \item процесс взаимодействия "умного"{} дома с его жильцами и посетителями;
    \item процесс взаимодействия "умного"{} предприятия, выпускающего определенного вида продукцию, с его сотрудниками;
    \item процесс взаимодействия студентов и преподавателей в рамках "умной"{} кафедры, осуществляющей подготовку молодых специалистов по какой-либо инженерной специальности;
    \item процесс взаимодействия постояльцев, посетителей и сотрудников "умного"{} отеля;
    \item процесс взаимодействия посетителей и сотрудников "умного"{} музея;
    \item процесс взаимодействия пациентов и медицинского персонала "умной поликлиники"{}, "умной"{} больницы;
    \item процесс взаимодействия граждан и чиновников в рамках "умной"{} администрации некоторого региона;
    \item процесс взаимодействия жителей и гостей в рамках "умного"{} города;
    \item и т. д.
\end{scnitemize}
Но для комплексной автоматизации человеческой деятельности в целом (Объединенной человеческой деятельности) автоматизации отдельных областей человеческой деятельности явно не достаточно, поскольку тесные связи между различными областями человеческой деятельности требуют автоматизации не только деятельности внутри каждой из этих областей, но и внешней деятельности, обусловленной необходимостью взаимодействия между различными областями деятельности, например, в рамках более крупных областей деятельности. Так, например, каждое предприятие взаимодействует со своими поставщиками и потребителями. Очевидно, что автоматизация такой внешней деятельности и, тем более, автоматизация с использованием интеллектуальных компьютерных систем существенно упрощается, если будут совпадать (будут унифицированы) принципы, лежащие в основе автоматизации каждой области деятельности, а также принципы автоматизации крупных областей деятельности, в состав которых входит некоторое количество более "мелких"{} областей человеческой деятельности.

Таким образом, для комплексной автоматизации человеческой деятельности в целом с применением интеллектуальных компьютерных систем и для обеспечения эффективной интеграции различных областей человеческой деятельности необходима разработка \textbf{\textit{Общей формальной теории человеческой деятельности}}, которая объединила бы формальные модели всевозможных областей человеческой деятельности, а также включила бы:
\begin{scnitemize}
    \item формальные теории различных \textit{видов человеческой деятельности}. Поскольку каждый \textbf{\textit{вид человеческой деятельности}} -- это класс однотипных \textit{областей человеческой деятельности}, формальная теория каждого вида человеческой деятельности -- это формальное представление \underline{стандарта} соответствующего класса областей человеческой деятельности. Так, например, можно говорить о формальной модели конкретного предприятия рецептурного производства (например, предприятие "Савушкин продукт"{} , выпускающего молочную продукцию), но можно говорить и о формальной теории всего класса предприятий рецептурного производства -- о формальном представлении стандарта ISA-88. Формальная теория каждого вида человеческой деятельности включает в себя формальное описание технологии, обеспечивающей осуществление каждой области (фрагмента) человеческой деятельности, принадлежащей указанному виду деятельности. В описание технологии входит описание используемых методов, средств и основных объектов и субъектов деятельности;
    \item четкую иерархическую декомпозицию \textit{Объединенной человеческой деятельности} по нескольким признакам. Основными признаками такой декомпозиции являются региональный признак и целевая направленность деятельности. По региональному признаку на высшем уровне иерархии выделяются такие области человеческой деятельности, как Деятельность Франции, Деятельность Германии и далее деятельность всех стран. По признаку целевой направленности на высшем уровне иерархии выделяются: Научно-исследовательская деятельность человечества, Проектная деятельность человечества, Производственная деятельность человечества, Образовательная деятельность человечества, Здравоохранительная деятельность человечества, Природоохранная деятельность человечества, Административная деятельность человечества и др. Дальнейшая декомпозиция областей человеческой деятельности по признаку целевой направленности выделяет такие области деятельности, как:
    \begin{scnitemizeii}
    	\item \textit{Научно-исследовательская деятельность человечества в области Математики}
    	\item \textit{Научно-исследовательская деятельность человечества в области Лингвистики}
    	\item и др.
    \end{scnitemizeii}    
    Заметим при этом, что, в отличие от "чисто"{} научных дисциплин, дисциплины научно-технического типа (например, дисциплина \textit{Искусственный Интеллект}) представляют собой симбиоз фрагментов (областей) деятельности, принадлежащих разным видам деятельности:
    \begin{scnitemizeii}
        \item научно-исследовательской деятельности;
        \item деятельности по разработке технологии проектирования (CAD);
        \item деятельности по разработке технологии производства (CAM);
        \item проектная деятельность;
        \item производство спроектированного объекта;
        \item образовательной деятельности;
        \item бизнес-деятельности.
    \end{scnitemizeii}
    Кроме указанных областей человеческой деятельности выделяются области, соответствующие различным сочетаниям значений указанных признаков декомпозиции областей человеческой деятельности. Примерами таких областей являются: Научно-исследовательская деятельность Франции, Образовательная деятельность Германии. Подчеркнем то, что количество областей человеческой деятельности, выделенных в результате указанной иерархической декомпозиции \textit{Объединенной человеческой деятельности}, является, хоть и не очень большим, но конечным в каждый момент времени.
\end{scnitemize}\bigskip

При этом Общая формальная теория человеческой деятельности должна быть ориентирована:
\begin{scnitemize}
    \item на унификацию формального описания самых различных технологий для самых различных областей человеческой деятельности;
    \item на унификацию формального описания всевозможных видов человеческой деятельности;
    \item на унификацию формального описания связей между различными областями и видами человеческой деятельности, различными субъектами деятельности, объектами, средствами (инструментами);
    \item на глубокую конвергенцию всех видов человеческой деятельности, областей человеческой деятельности, используемых методов.
\end{scnitemize}}
\scnrelfromvector{что делать}{\scnfileitem{Необходим переход от локальной автоматизации различных областей и видов человеческой деятельности путем независимой друг от друга разработки систем автоматизации бизнес-процессов даже близких по виду деятельности предприятий к \underline{комплексной автоматизации человеческой деятельности} в целом прежде всего для обеспечения совместимости различных областей деятельности и исключения ужасающего и никому не нужного дублирования (многообразия форм) автоматизации аналогичных бизнес-процессов}
;\scnfileitem{Все многообразие человеческой деятельности необходимо четко стратифицировать, доведя эту стратификацию до строгого формального представления}
;\scnfileitem{Необходимо
\begin{scnitemize}
    \item четко выделить все виды человеческой деятельности, соответствующие текущему уровню развития человеческого общества;
    \item построить четкую иерархию этих видов на основании отношения, связывающего виды человеческой деятельности с их подвидами;
    \item унифицировать человеческую деятельность в рамках каждого выделенного вида, разработав соответствующие стандарты, для каждого из которых построить четкую систему используемых понятий;
    \item довести указанные стандарты до такого уровня формализации, чтобы они стали частью базы знаний интеллектуальной системы автоматизации соответствующего вида человеческой деятельности.
\end{scnitemize}}
;\scnfileitem{Необходимо обеспечить конвергенцию, семантическую совместимость и глубокую интеграцию различных видов и областей человеческой деятельности путем:
\begin{scnitemize}
    \item согласования систем понятий, соответствующих стандартам разных видов человеческой деятельности, и особенно согласования систем понятий между стандартами видов и подвидов человеческой деятельности;
    \item представления стандарта каждого вида человеческой деятельности в виде формальной онтологии;
    \item построения такой иерархической системы формальных онтологий, соответствующих всевозможным видам человеческой деятельности, в которой обеспечивалась бы конвергенция и \underline{семантическая совместимость} онтологий, входящих в эту систему, а также \underline{наследование свойств} от онтологии каждого вида человеческой деятельности к онтологии каждого подвида этого вида человеческой деятельности.
\end{scnitemize}}}
\scnaddlevel{1}
\scntext{следовательно}{Таким образом, в целях повышения эффективности автоматизации человеческой деятельности и, в первую очередь, в целях существенного снижения трудозатрат на такую автоматизацию необходимо с точки зрения общей теории систем фундаментально переосмыслить современную организацию человеческой деятельности, поскольку автоматизация беспорядка приводит к ещё большему беспорядку. На этом пути имеется только одно препятствие -- противодействие лени с высоким уровнем эгоизма, которым современный беспорядок организации человеческой деятельности выгоден.}
\scnaddlevel{-1}

%\scnheader{Математика}
%\scniselement{научная дисциплина}
%\scnidtf{Научно-исследовательская деятельность человечества в области математики}
%\scnheader{Лингвистика}
%\scniselement{научная дисциплина}
%\scnidtf{Научно-исследовательская деятельность человечества в области лингвистики}

\scnheader{следует отличать}
\scnhaselementset{вид человеческой деятельности\\
\scnaddlevel{1}
\scnhaselementlist{пример}{научно-исследовательская деятельность;проектирование\\
    \scnaddlevel{1}
    \scnidtf{проектная деятельность}
    \scnsuperset{проектирование интеллектуальной компьютерной системы\\
        \scnaddlevel{1}
        \scnsuperset{проектирование ostis-системы}
        \scnaddlevel{-1}
    \scnaddlevel{-1}}}
\scnaddlevel{-1}
;область человеческой деятельности\\
\scnaddlevel{1}
\scnhaselementlist{пример}{Научно-исследовательская деятельность в области Искусственного интеллекта\\
    \scnaddlevel{1}
    \scniselement{научно-исследовательская деятельность}
    \scnrelfrom{часть}{Научно-исследовательская деятельность РАИИ}
    \scnaddlevel{-1}
    ;Проектирование Метасистемы IMS.ostis\\
        \scnaddlevel{1}
        \scniselement{проектирование ostis-системы}
        \scnaddlevel{-1}}
\scnaddlevel{-1}
;подвид человеческой деятельности*\\
    \scnaddlevel{1}
    \scnsubset{включение*}
        \scnaddlevel{1}
        \scnidtf{подмножество*}
        \scnaddlevel{-1}
    \scnaddlevel{-1}
;подобласть человеческой деятельности*\\
    \scnaddlevel{1}
    \scnsubset{часть*}
    \scnaddlevel{-1}}

    
    
\scnheader{информационная технология}
\scnexplanation{Множество технологий, связанных с проектированием и производством компьютерных систем и их компонентов, с эксплуатацией компьютерных систем, а также с их использованием в качестве инструмента в составе самых различных технологий. В рамках различных информационных технологий компьютерные системы рассматриваются как инструментальные средства, как вспомогательные субъекты, обеспечивающие автоматизацию соответствующих видов деятельности. Но в некоторых информационных технологиях компьютерные системы являются также и \underline{объектами} автоматизируемых видов деятельности. Примерами таких технологий являются:
\begin{scnitemize}
    \item технология проектирования компьютерных систем;
    \item технология реализации (сборки) компьютерных систем;
    \item технология обновления компьютерных систем.
\end{scnitemize}}
\scnhaselement{Комплекс современных информационных технологий}
\scnhaselement{Комплекс современных технологий искусственного интеллекта}
\scnhaselement{Технология OSTIS}

\scnheader{автоматизация человеческой деятельности}
\scnidtf{человеческая деятельность, направленная на повышения уровня автоматизации человеческой деятельности, а также на повышение качества (в том числе, уровня интеллекта) человеческого общества как многоагентной кибернетической системы}
\scnsubset{вид человеческой деятельности}
\scnnote{Важнейшим этапом автоматизации человеческой деятельности в перспективе должен стать переход к существенно более высокому уровню \textit{интеллекта человеческого общества} как целостной кибернетической системы путем преобразования современного человеческого общества в сообщество взаимодействующих между собой людей и интеллектуальных компьютерных систем. Такое сообщество иногда называют smart-обществом, обществом 5.0.
	
Особо подчеркнем то, что переход к такому интеллектуальному обществу требует существенного переосмысления современной организации различных видов человеческой деятельности. Прежде всего, следует подчеркнуть, что эффективность (коэффициент полезного действия) современной организации человеческой деятельности в целом ужасающе низка, а, как известно, автоматизация беспорядка (даже с помощью интеллектуальных компьютерных систем) приводит к ещё большему беспорядку.}

\scnheader{вид человеческой деятельности}
\scnidtf{Множество всевозможных видов человеческой деятельности}
\scnrelfrom{разбиение}{Разбиение Множества видов человеческой деятельности по степени их автоматизируемости}
\scnaddlevel{1}
\scneqtoset{вид человеческой деятельности, который принципиально может быть автоматизирован полностью\\
    \scnaddlevel{1}
    \scnidtf{Множество полностью автоматизируемых видов человеческой деятельности}
    \scnaddlevel{-1}
;вид человеческой деятельности, который может быть автоматизирован только частично\\
    \scnaddlevel{1}
    \scnidtf{Множество частично автоматизируемых видов человеческой деятельности}
    \scnaddlevel{-1}
;вид человеческой деятельности, который принципиально никак не может быть автоматизирован\\
    \scnaddlevel{1}
    \scnidtf{Множество неавтоматизируемых видов человеческой деятельности, которые могут быть выполнены только "вручную"\ (точнее самими людьми с возможным использованием каких-либо "пассивных"\ инструментов -- топора, лопаты и т. п.)}
    \scnaddlevel{-1}}
\scnaddlevel{-1}

\scnheader{вид человеческой деятельности, который принципиально может быть автоматизирован полностью}
\scnnote{Есть виды человеческой деятельности, которые принципиально могут быть автоматизированы \underline{полностью}, но в текущий момент эта автоматизация не полна. Это ,например, частично автоматизированная деятельность по производству спроектированных искусственных объектов. Здесь важна \underline{четкость} распределения "обязанностей"{} между различными средствами автоматизации и поэтапное исключение неавтоматизированных действий, "вручную"{} выполняемых людьми (например, сотрудниками производственных предприятий), т. е. поэтапная автоматизация этих действий.}

\bigskip
\scnfragmentcaption

\scnheader{автоматизация человеческой деятельности}
\scntext{вопрос}{Почему для комплексной автоматизации человеческой деятельности целесообразно использовать семантически совместимые, договороспособные, самостоятельные интеллектуальные компьютерные системы, которым можно делегировать права на принятие некоторых решений.}

\scntext{вопрос}{Почему для повышения уровня комплексной \textit{автоматизации человеческой деятельности} необходим переход от современных (традиционных) \textit{компьютерных систем} и соответствующих им информационных технологий, а также от \textit{современных интеллектуальных компьютерных систем} и соответствующих им современных технологий искусственного интеллекта к \textit{интеллектуальным компьютерным системам} \uline{нового поколения} и к соответствующей им Комплексной технологии проектирования таких систем.}
	\scnaddlevel{1}
	\scntext{ответ}{В силу отсутствия унификации представления обрабатываемой информации в традиционных компьютерных системах и, как следствие, отсутствия совместимости этих систем как на синтаксическом уровне, так и на семантическом уровне, принциально не существует универсального метода системной интеграции традиционных компьютерных систем и, следовательно, невозможна полная автоматизация решения этой задачи. Системная интеграция традиционных компьютерных систем практически всегда осуществляется "вручную"{} с учетом индивидуальной специфики каждой интегрируемой системы и, следовательно, является весьма трудоемкой и требующей высокой квалификации разработчиков.\\
	Но на данном этапе эволюции компьютерных систем крайне актуальной является полная автоматизация их интеграции без какого бы то ни было участия разработчиков и тем более конечных пользователей. Если компьютерные системы не приобретут способность \uline{самостоятельно} взаимодействовать между собой в целях решения сложных комплексных задач, то эффективность использования человечеством интенсивно расширяемого многообразия весьма полезных и качественно реализованных информационных ресурсов и сервисов будет весьма низкой. Традиционно компьютерные технологии позволяют реализовать \uline{любую} модель обработки информации (в том числе и \uline{любую} интеллектуальную модель -- нейросетевую, логическую и т.д.). Однако актуальным является не реализация самих этих моделей, а их интеграция, что требует обоспечения синтаксически и семантически совместимых компьютерных систем и полной автоматизации их системной интеграции. Следовательно необходим переход на принципиально новое поколение компьютерных технологий и, в частности, на принципиально новое поколение самих компьютеров, ориентированных на решение проблем совместимости компьютерных систем и полной автоматизации их системной интеграции.\\
	Таким образом, дальнейшее повышение уровня автоматизации различных видов человеческой деятельности потребует перехода на принципиально новый уровень информационных технологий -- от современных (традиционных) \textit{компьютерных систем} к компьютерным системам, имеющим существенно более \textit{высокий уровень интеллекта} и способным не только индивидуально решать достаточно сложные (в том числе, интеллектуальные) задачи, но и эфффективно \uline{самостоятельно} взаимодействовать между собой, координируя свою деятельность при решении задач, принадлежащих априори неизвестным (заранее не предусмотренным) классам задач и требующих коллективного (корпоративного) решения.\\
	Основные проблемы автоматизации \textit{человеческой деятельности} в настоящее время лежат не в области разработки средств автоматизации решения различных конкретных классов задач (в том числе и весьма сложных, интеллектуальных, труднорешаемых задач), а в области системной интеграции этих средств в комплексы, компоненты которых способны самостоятельно кооперироваться для совместного (коллективного) решения сложных задач. Но для этого указанные компоненты должны уметь согласовывать, координировать свои действия, должны понимать друг друга, должны быть семантически совместимы.}
	\scnaddlevel{-1}
	
\scnheader{интеллектуальная компьютерная система}
\scnnote{Различные интеллектуальные компьютерные системы могут быть эффективно использованы в качестве средств автоматизации самых различных видов человеческой деятельности. Но, поскольку все виды человеческой деятельности взаимосвязаны (как минимум потому, что каждый человек может одновременно участвовать сразу в нескольких видах деятельности, причем в разные моменты времени этот набор видов деятельности для каждого человека может быть различным), интеллектуальная компьютерная система автоматизации каждого вида человеческой деятельности должна эффективно взаимодействовать с другими интеллектуальными компьютерными системами, осуществляющими автоматизацию других видов человеческой деятельности. Другими словами, необходимо переходить от автоматизации отдельных видов человеческой деятельности к автоматизации комплекса всех видов человеческой деятельности. Для этого необходимо:
	\begin{scnitemize}
	\item не просто достаточно детально разработать \uline{теорию каждого вида деятельности}, выделив (1) все классы автоматизируемых действий, (2) все классы неавтоматизируемых действий, (3) соответствующие этим классам методы выполнения действий (в частности, это могут быть обобщенные бизнес-процессы), которые по сравнению с используемыми в настоящий момент могут потребовать существенного реинжиниринга бизнес-процессов;
	\item но и представить эти теории в формализованном и унифицированном виде в качестве фрагментов баз знаний соответствующих интеллектуальных компьютерных систем, обеспечив при этом высокую степень конвергенции этих теорий.
	\end{scnitemize}}
	
\scnrelfromlist{возможное амплуа}{средство автоматизации проектирования\\
	\scnaddlevel{1}
	\scnsuperset{средство автоматизации проектирования интеллектуальных компьютерных систем}
		\scnaddlevel{1}
		\scnnote{В силу большой сложности процесса проектирования интеллектуальных компьютерных систем для автоматизации этого процесса необходимо использовать именно интеллектуальные компьютерные системы.}
		\scnaddlevel{-1}
	\scnaddlevel{-1}
;средство автоматизации производства
;средство повышения качества эксплуатации сложного объекта\\
	\scnaddlevel{1}
	\scnsuperset{средство help-поддержки конечных пользователей}
	\scnsuperset{средство управления процессом повышения качества деятельности конечных пользователей}
	\scnsuperset{средство поддержки оптимальных эксплуатационных свойств эксплуатируемого объекта}
		\scnaddlevel{1}
		\scnexplanation{Здесь имеется в виду мониторинг состояния эксплуатируемого объекта, контроль условий эксплуатации, своевременная профилактика и ремонт.}
		\scnaddlevel{-1}
	\scnsuperset{средство поддержки совершенствования эксплуатируемого объекта в ходе его эксплуатации}
	\scnnote{Для интеллектуальных компьютерных систем все средства повышения качества их эксплуатации целесообразно встраивать в эти системы. Имеется в виду слияние нескольких интеллектуальных компьютерных систем в одну интегрированную.}
	\scnaddlevel{-1}
;средство автоматизации научно-исследовательской деятельности в рамках заданной научной дисциплины
;средство автоматизации образовательной деятельности\\
	\scnaddlevel{1}
	\scnnote{Автоматизация образовательной деятельности может осуществляться в рамках:
		\begin{scnitemize}
		\item заданной учебной дисциплины;
		\item заданной учебной специальности;
		\item заданного учебного заведения;
		\item заданного государства.
		\end{scnitemize}}
	\scnaddlevel{-1}
;средство автоматизации бизнес-деятельности в заданной научно-технической области
	\scnaddlevel{1}
	\scnnote{Здесь важна автоматизация контроля за реализацией всех направлений организационной деятельности с учетом разработанных и постоянно уточняемых и корректируемых планов, а также с учетом согласованных приоритетов.}
	\scnaddlevel{-1}
;средство автоматизации деятельности в области здравоохранения
;средство автоматизации административной деятельности
;средство автоматизации деятельности жилищно-коммунального хозяйства
;юриспруденция
;правоохранительная деятельность
;транспорт}

\scnheader{Экосистема OSTIS}
\scnnote{\textit{Экосистема OSTIS} является основой для перевода уровня информатизации различных областей человеческой деятельности на принципиально новый уровень, а также для интеграции соответствующих проектов -- "Общество 5.0"{}, "Industry 4.0"{}, "University 3.0"{}, "Умный дом"{}, "Умный город"{} и других (без интеллектуальных компьютерных систем все эти проекты невозможны).\\
Все эти проекты должны быть приведены в единую стройную иерархическую систему взаимосвязанных проектов, охватывающих весь объем и многообразие человеческой деятельности.}

\iffalse %TODO Update fragment
\bigskip
\scnfragmentcaption

\scnheader{Экосистема OSTIS}
\scntext{вопрос}{Какие достоинства имеет Экосистема OSTIS}
\scntext{достоинство}{Важнейшей особенностью Экосистемы OSTIS является то, что входящие в нее \textit{ostis-системы} благодаря высокому уровню их интеллекта и, в частности, высокому уровню их социализации, становятся самостоятельными, активными и полноправными субъектами, участвующими в реализации самых различных видов человеческой деятельности, что существенно повышает уровень её автоматизации.}
\scnrelfromset{Что такое интеллектуальная система}{
\uline{система}(!)свойств
;В.К. Финн
;требования, предъявленные к интеллектуальным компьютерным системам}
\scnrelfromset{достоинства}{
\scnfileitem{семантическая совместимость
\begin{scnitemize}
\item интеллектуальных компьютерных систем между собой
\item интеллектуальных компьютерных систем с их пользователями и разработчиками
\item семантическая совместимость = взаимопонимание
\end{scnitemize}};
\scnfileitem{Перманентная поддержка семантической совместимости};
\scnfileitem{Способность координировать свои действия (договороспособность, координация(!)) при коллективном решении задач автоматизации \textbf{системной интеграции} интеллектуальных компьютерных систем, "ручная"{} реализация системной интеграции -- главный тормоз комплексной автоматизации. Многоагентная система из интеллектуальных компьютерных систем + \textbf{людей}.
\begin{scnitemize}
\item Каждая ostis-система является, кроме всего прочего, способной обучать(повышать квалификацию) своих пользователей т.е. повышать эффективность своей эксплуатации\\
ostis-система\\
\scnsubset{интеллектуальная обучающая система}
\end{scnitemize}};
\scnfileitem{Экосистема интеллектуальных компьютерных систем\\ 
\scneq{комплексная автоматизация человеческой деятельности}
Достоинства Экосистемы(преимущества) и перспективы создания и развития Технологии OSTIS
\begin{scnitemize}
\item Технология OSTIS как основа эволюции человеческого общества => переход к smart-обществу, к более интеллектуальному обществу
\item Экосистема OSTIS как продукт Технологии, т.е продукт технологии -- не отдельные интеллектуальные компьютерные системы, а целая Экосистема
\end{scnitemize}
это вариант smart-общества\\
smart-предприятие, smart-город\\
Требования к технологии разработки интеллектуальных компьютерных систем
\begin{scnitemize}
\item smart-сообщество разработчиков интеллектуальных компьютерных систем и разработчиков самой технологии
\item консорциум!!
\item стандарты интеллектуальных компьютерных систем
\item стандарты процесса разработки различных интеллектуальных компьютерных систем 
\item стандарты процесса совершенствования самой технологии
\item ориентация на новые компьютеры
\end{scnitemize}
Экосистема OSTIS -> цель\\
\scneq{smart-общество = общество 5.0 как интеграция всевозможных специализированных smart-сообществ(...)}};%?
\scnfileitem{конвергенция в области Искусственного интеллекта и не только(!!)(это необходимо для Экосистемы интеллектуальных компьютерны систем)};
\scnfileitem{глубокая интеграция(совместимость)};
\scnfileitem{новое поколение компьютеров};
\scnfileitem{Консорциум OSTIS по разработке глобального комплекса семантически совместимых технологий, обеспечивающих комплексную автоматизацию всевозможных видов человеческой деятельности(для Экосистемы интеллектуальных компьютерных систем).Достоинства эволюции интеллектуальных компьютерных систем автоматизации проектирования интеллектуальных компьютерных систем распространяется на все технические дисциплины и соответственно сообщества.};
\scnfileitem{система Проектов OSTIS, реализуемых консорциумом OSTIS -- как прообраз project-management нового типа, ориентированного на реализацию \uline{наукоемких} проектов с децентрализованным управлением и с перманентным коллективным уточнением и детализацией целей};
\scnfileitem{\textbf{Рынок знаний и его реализация}
\begin{scnitemize}
\item Смысловые представления знаний и глобальный характер минимизирует субъективизм, предвзятость, более эффективно защищает авторские права
\item Тормозит научно-техническое развитие(прогресс) становиться труднее
\item Выигрывает тот, кто действительно способствует прогрессу, а не тормозит его
\end{scnitemize}
Нет центральных и периферийных публикаций -- есть общая база знаний, в которой нет семантической эквивалентности(и, следовательно, нет плагиата). 
%До....?
Монография OSTIS \textbf{+ IMS.ostis} как новый уровень автоматизации создания и эволюции научно-технического к?
	от статей и монографий к семантически совместимым базам и порталам научно-технических знаний!!
	достоинства эволюции портала научных знаний по Искусственному интеллекту и соответственно сообщества ученых распространяется на все научные дисциплины
}
}

\fi

\bigskip
\scnendstruct \scnendsegmentcomment{Уточнение Понятия Экосистемы OSTIS}