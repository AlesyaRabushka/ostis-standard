\scnsegmentheader{Текущее состояниеи и проблемы дальнейшего развития деятельности в области Искусственного интеллекта}
\scnstartsubstruct

\scntext{аннотация}{Рассмотрим в каких направлениях должна происходить эволюция повышенного качества деятельности в области \textit{Искусственного интеллекта}, а также эволюция продуктов этой деятельности}

\scnheader{Научно-исследовательская деятельность в области Искуссвенного интеллекта}
\scntext{текущее состояние}{}
\scnheader{Научно-исследовательская деятельность в области Искуссвенного интеллекта}
\scnrelfromset{проблемы текущего состояния}{
\scnfileitem{Отсутствует согласованность систем \textit{понятий} в разных направлениях \textit{Искусственного интеллекта} и, как следствие, отсутствует \textit{семантическая совместимость} и \textit{конвергенция} этих направлений, в результате чего ни о каком движении в напралении построения \textit{общей теории интеллектуальных систем} с высоким уровнем формализации и речи быть не может. Существование и продолжающееся увеличение "высоты барьеров" между различными направлениями исследований в области \textit{Искусственного интеллекта} проявляется в том, что специалист, работающий в рамках какого-либо направления \textit{Искусственного интеллекта}, посещая заседания "не своей" секции на конференции по \textit{Искусственному интеллекту}, мало что там может понять и, соответсвенно, извлечь полезного для себя.};
\scnfileitem{Отсутствует мотивация и осознание острой необходимости в указанной \textit{конвергенции} между разлиными направлениями \textit{Искусственного интеллекта}.};
\scnfileitem{Отсутствует реальное движение в направлении построения \textit{Общей теории интеллектуальных систем}, поскольку отсутствует соответствующая мотивация и осознание острой практической необходимости в этом.}
}

\scnheader{Разработка базовой компклексной технологии проектирования интеллектуальных компьютерных систем}
\scntext{текущее состояние}{Современная технология \textit{Искусственного интеллекта} представляет собой целое семейство всевозможных частных технологий, ориентированных на разработку и сопровождение различного вида компонентов \textit{интеллектуальных компьютерных систем}, реализующих самые различные модели представления и обработки информации, различные модели решения задач, ориентированных на разработку различных классов \textit{интеллектуальных компьютерных систем}.}
\scnrelfromset{проблемы текущего состояния}{
\scnfileitem{высокая трудоемкость разработки};
\scnfileitem{необходимая высокая квалификация разработчиков};
\scnfileitem{современные технологии \textit{Искусственного интеллекта} принципиально не обеспечивают разработки таких \textit{интеллектуальных компьютерных систем}, в которых устраняются недостатки современных \textit{интеллектуальных компьтерных систем}};
\scnfileitem{совместимость частных технолгий \textit{Искусственного интеллекта} практически отсутствует и, как следствие, отсутствует \textit{семантическая совместимость} разрабатываемых \textit{интеллектуальных компьютерных систем}, поэтому их системная интеграция осуществляется \uline{вручную}.};
\scnfileitem{Разрабатываемые \textit{интеллектуальные компьютерные системы} не способны \uline{самостоятельно} координировать свою деятельность друг с другом следовательно
\begin{scnitemize}
\item{нет общей комплекснойтехнологии проектирования интеллектуальных компьютерных систем};
\item{не обеспечивается совместимость и взаимодействие разрабатываемых систем (синтаксическая и семантическая совместимость)};
\item{нет совместимости между существующими частными технологиями проектирования различных компонентов интеллектуальных компьютерных систем (б.д.,нейроны,интерфейсы и т.д.)};
\item{Есть инструментальные средства по компоновке, но "склеивать" (соединять, интегрировать) это надо вручную};
\item{Нет системы и инструментальных средств}
\end{scnitemize}
}
}

\scnheader{Разработка технологии призводства спроектированных интеллектальных компьютерных систем}
\scntext{текущее состояние}{Было сделано целый ряд попыток разработки \textit{компьютеров} нового поколения, ориентированных на использование в \textit{интеллектуальных компьютерных системах}. Но все они оказались неудачными, так как не были ориентированы на всё многообразие моделей решения задач в \textit{интеллектуальных компьютерных системах}. В этом смысле они не были \textit{\uline{универсальными} компьютерами} для \textit{интеллектуальных компьютерных систем}.}
\scnrelfromset{проблемы текущего состояния}{
\scnfileitem{Разрабатываемые \textit{интеллектальные компьтерные системы} могут использовать самые различные комбинации \textit{моделей решения интеллектуальных задач} (логических моделей, соответствующих различного вида логикам, нейросетевых моделей различного вида, моделей целеполагания, синтеза планов, моделей управления сложными объектами, моделей понимания и синтеза текстов естественного языка и т.д.). Современные (традиционные, фон-неймановские) \textit{компьютеры} не в состоянии достаточно производительно интерпритировать всё многообразие указанных моделей решения задач. При этом разработка специализированных \textit{компьютеров}, ориентированных на интерпритацию какой-либо одной модели решения задач (нейросетевой модели или какой-либо логической модели) проблему не решает, так как в \textit{интеллектуальной компьютерной системе} необходимо использовать сразу несколько разных моделей решения задач, причём в различных сочетаниях.}
}

\scnheader{Специализированная инженерия в области Искусственного интеллекта}
\scnidtf{Деятельность, направленная на разработку \textit{интеллектуалных компьютерных систем} различного назначения с использованием имеющихся для этого моделей, методов и средств}
\scnidtf{Деятельность по проектированию и производству \textit{интеллектуальных компьютерных систем}}
\scnidtf{Деятельность, направленная на формирование рынка \textit{интеллектуальных компьютерных систем}}
\scnrelfrom{в перспективе}{Специализированная инженерия в области \textit{Искусственного интеллекта}, осушествляемая специальной частью экосистемы OSTIS\\
\scnrelfrom{продукт}{экосистема OSTIS}
\scnrelfrom{субъект действия}{часть экосистемы OSTIS, осуществляющая специализированную инженерию в области \textit{Искусственного интеллекта}}
}
\scntext{текущее состояние}{}
\scnrelfromset{проблемы текущего состояния}{
\scnfileitem{Отсутствует четкая систематизация многообразия \textit{интеллектуальных компьютерных систем}, соответствующая систематизации автоматизируемых \textit{видов человеческой деятельности}.};
\scnfileitem{Отсутствует \textit{конвергенция} \scnbigspace \textit{интеллектуальных компьютерных систем}, обеспечивающих автоматизацию \textit{областей человеческой деятельности}, принадлежащих однму и тому же \textit{виду человеческой деятельности}.};
\scnfileitem{Отсутствует \textit{семантическая совместимость}(семантическая унификация, взаимнопонимание) между \textit{интеллектуальными компьютерными системами}, основной причиной чего является отсутствие согласованной системы общих используемых \textit{понятий}.};
\scnfileitem{семантическая недружественность \textit{пользовательского иннтерфейса} и отсутствие встроенной справочной системы, позволяющией запрашивать информацию об элементах интерфейса и возможностях системы, приводят к низкой эффективности эксплуатации всех возможностей \textit{интеллектуальной компьютерной системы}.};
\scnfileitem{Анализ проблем автоматизации всех \textit{видов человеческой деятельности} убеждает в том, дальнейшая автоматизация \textit{человеческой деятельности} требует не только повышения уровня \textit{интеллекта} соответствующих \textit{интеллектуальных компьютерных систем}, но и реализации их способности
\begin{scnitemize}
\item устанавливать свою \textit{семантическую совместимость} (взаимопонимание) как с другими \textit{компьютерными системами}, так и со своими пользователями
\item поддерживать эту \textit{семантическую совместимость} в процессе собственной эволюции, а также эволюции пользователей и других \textit{компьютерных систем}.
\item координировать свою деятельность с пользователями и другими \textit{компьютерными системами} при коллективно решении различных задач
\item участвовать в распределении работ (подзадач) при коллективном решении различных задач
\end{scnitemize}
Важно подчеркнуть то, что реализация вышеперечисленных способностей создаст возможность для существенной и даже полной автоматизации \textit{системной интеграции} \scnbigspace \textit{компьютерных систем} в комплексы взаимодействующих систем и автоматизации реинжиниринга таких комплексов. Такая автоматизация системной интеграции и её реинжиниринга:
\begin{scnitemize}
\item во-первых, даст возможность комплексам кибернетических систем \uline{самостоятельно} адаптироваться к решению новых задач
\item во-вторых, существенно повыситэффективность эксплуатации таких комплексов компьютерных систем, так как реинжиниринг системной интеграции компьютерных систем в такой комплекс, часто востребован (например, при реконструкции предприятия)
\item в-третьих, существенно сокращает число ошибок по сравнению с "ручной" (неавтоматизированной) выполнением \textit{системной интеграции} и её \textit{реинжиниринга} которая, к тому же, требует высокой квалификации.
\end{scnitemize}
Таким образом следующий этап повышения уровня автоматизации \textit{человеческой деятельности} настоятельно требует создания таких \textit{интеллектуальных компьютерных систем}, которые могли бы легко сами (без системного интегратора) объединяться для совместного решения сложных задач. 
}
}

\scnheader{Образовательная деятельность в области искусственного интеллекта}
\scntext{текущее состояние}{Целеноправленная подготовк специалистов в области искусственного интеллект имеет богатую историю и осуществляется во многих вудущих университетах (Стенфорд,MTI,МГУ,МЭИ,РГГУ,Питер(Гаврилово),Владивосток,Новосибирск, Киев,КПИ,БГУИР,БГУ,БрГТУ)}
\scnrelfromset{проблемы текущего состояния}{
\scnfileitem{Поскольку деятельнсоть в области \textit{Искусственного интеллекта} сочетает в себе и высокую степень наукоемкости и высокую степен сложности инженерных работ, подготовка специалистов в этой области требует одновременного формирования у них как научно-исследовательских навыков, культуры и стиля мышления, так и инженерно-практических навыков, культуры и стиля мышления. С точки зрения методики и психологии обучения сочетание фундаментальной научной и инженерно-практической подготовки специалистов является весьма сложный образовательной педагогической задачей.};
\scnfileitem{Отстутствует \textit{семантическая совместимость} между различными учебными дисциплинами это приводит к "мозаичности" восприятия информации};
\scnfileitem{отсутствует системный подход к подготовке молодых специалистов в области \textit{Искусственного интеллекта}};
\scnfileitem{Нет персонификации обучения};
\scnfileitem{Нет установки на выявление, раскрытие и развитие таланта проект творческого раскрытия};
\scnfileitem{Отсутствует целенаправленное формирование мотивации к творчеству};
\scnfileitem{Нет формирования навыков работы в реальных коллективах разработчиков};
\scnfileitem{Отсутствует адаптация к реальной практической деятельности};
\scnfileitem{Любая современная технология (в том числе и технология OSTIS) должна иметь высокие темпы своего развития, поскольку без этого невозможно поддерживать высокий уровень её конкурентоспособности. Но для быстро развиваемой технологии требуются:
\begin{scnitemize}
\item не просто высокая квалификация кадров, использующих и развивающих технологию
\item но и высокие темпы повышения уровня этой квалификации, так как без этого невозможно эффективно использовать и развивать быстро меняющуюся технологию.
\end{scnitemize}
Из этого следует, что образовательная деятельность в области \textit{Искусственного интеллекта} и соответствующая ей технология должна быть не просто важной частью деятельности в области искусственного интеллекта, а частью, глубоко интегрированной во все остальные виды деятельности в области \textit{Искусственного интеллекта}. Так, например, каждая \textit{интеллектуальная компьютерная система} должная быть ориетированна не только на обслуживание своих конечных ползователей, не только на организацию целенаправленного взаимодействия со своими разработчиками, которые постоянно совершенствуют эту систему, и не только на обеспечение минимального "порога вхождения" для новых конечных пользователей и разработчиков, но и на организацию постоянного и персонифицированного повышения квалификации каждого своего конечного пользователя и разработчика в условиях постоянных изменений, вносимых в указанную \textit{интеллектуальную компьютерную систему}. Для этого эксплуатируемая \textit{интеллектуальная компьютерная система} должна "знать", что в ней изменилось, на что она способна и как эти способности инициировать (содержание и форма, соответствующих пользовательских команд)
}
}

\scnendstruct

