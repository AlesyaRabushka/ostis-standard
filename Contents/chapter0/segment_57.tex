\bigskip
\scnfragmentcaption

\scnheader{Рынок знаний, реализуемый в рамках Экосистемы OSTIS}
\scnexplanation{Важнейшим видом предметно-независимой человеческой деятельности, осуществляемой в рамках Экосистемы OSTIS является перманентный реинжиниринг всех ostis-систем, входящих в Экосистему OSTIS. Указанная деятельность должна быть направлена на перманентную и быструю эволюцию всех ostis-систем и, самое важное, на эволюцию Экосистемы OSTIS в целом. Особо следует подчеркнуть, что эволюция ostis-систем и Экосистемы OSTIS в целом представляет собой весьма сложный творческий, коллективный процесс, который принципиально может быть автоматизирован \uline{только частично}. При этом от людей, участвующих в этом процессе требуется высокая квалификация, высочайшая системная культура на уровне глубокого знания общей теории систем, высокая математическая культура -- культура формализации, высокая культура конвергенции (обнаружения сходств, доведение их до формальных аналогий), высокая культура глубокой интеграции, высокий уровень договороспособности.

Кроме указанных требований необходим высочайший уровень мотивации к тому, чтобы эволюция отдельных компонентов Экосистемы OSTIS (в частности, отдельных ostis-систем) не осуществлялась в ущерб эволюции Экосистемы OSTIS в целом, например, путём привнесения эклектичности, многообразия форм решения похожих проблем, путем ослабления фундаментального требования \uline{максимально возможной простоты} и логичности принципов, лежащих в основе Экосистемы OSTIS.

Существенно подчеркнуть, что эволюция ostis-систем и Экосистем OSTIS в целом сводится к коллективному реинжинирингу баз знаний ostis-систем, что, в свою очередь сводится к:
	\begin{scnitemize}
	\item "ручной"{} генерации предлагаемых дополнительных (новых) знаний в базу знаний указываемой ostis-системы;
	\item "ручной"{} генерации предлагаемых изменений текущего состояния
базы знаний указываемой ostis-системы;
	\item автоматическому назначению компетентных и заинтересованных
рецензентов;
	\item "ручному"{} рецензированию каждого поступившего предложения,
результатом чего является:
		\begin{scnitemize}
		\item либо полное одобрение;
		\item либо полное неодобрения с предлагаемой аргументацией;		
		\item либо детальная рекомендация доработки, предположения;		
		\end{scnitemize}
	\item автоматическому назначению достаточно широкого круга компетентных и заинтересованных специалистов для утверждения
поступившего предложения (после получения одобрения от всех назначенных экспертов);
	\item автоматическому принятию решения по одобрению поступившего
предложения на основании мнения всех привлечённых экспертов и
специалистов.
	\end{scnitemize}

Таким образом в базе знаний каждой OSTIS-системы можно (и нужно!)
фиксировать весь процесс обсуждения каждого поступившего предложения с
указанием (1) моментов времени всех привлечённых событий; (2) участников каждого события (авторов предложений, авторов рецензий участников голосования).

Кроме того, каждая ostis-система, анализируя процесс использования
хранимых ею знаний в процессе эксплуатации, может оценивать частоту
непосредственного и опосредованного использования этих знаний, т.е. может оценить степень востребованности этих знаний.	

Следовательно, в перспективе Экосистема OSTIS может с достаточно
высокой степенью \uline{объективности} может оценивать объем и значимость вклада каждого специалиста в развитие распределенной базы знаний Экосистемы OSTIS. Это является фундаментальной основой для
формирования достаточно объективного (честного) рынка знаний.}

\scnauthorcomment{Фрагмент 407-412, вычитать}

\scnheader{Рынок знаний, реализуемый в рамках Экосистемы OSTIS}
\scnrelfromset{правила для авторов знаний, публикуемых в рамках баз знаний различных ostis-систем, входящих в состав экосистемы OSTIS}{
\scnfileitem{знания, \uline{предлагаемые} для рецензирования, согласования,
утверждения и публикации в базе знаний соответствующей ostis-системы должны быть специализированы (указана ostis-
система, атомарный раздел базы знаний, дата и время, автор,новый вид публикации, рынок знаний,
защита авторского права не на уровне документов, а на уровне смысла.}
;\scnfileitem{}}
\scntext{коллективное совершенствование базы знаний}{Абсолютно идеальных решений (в том числе проектных) не бывает. Поэтому (1) не надо бояться ошибок и (2) надо минимизировать степень ошибочности за счёт (2.1) \uline{оперативности} исправления ошибок и (2.2) повышения качества (уровня) анализа при принятии решения путем (2.2.1) \uline{коллективного} характера экспертизы,
(2.2.2) достаточного количества привлекаемых
экспертов и (2.2.3) учёта уровня осведомленности
(квалифицированности и  погруженности в
соответствующую предметную область и
онтологию). Для каждого эксперта, привлекаемого к принятию решения нужен постоянно уточняемый, по объективным критериям коэффициент осведомленности-авторитетности каждого эксперта к каждой конкретной предметной области.}
\scntext{правила редактирования Общей базы знаний коллектива интеллектуальной системы}{
	\begin{scnitemize}
	\item Если Вы в рамках базы знаний разрабатываемой Вами ostis-системы хотите ввести знак новой ранее не описываемой сущности, то Вы должны проверить, что эта сущность
действительно не описывалась в рамках виртуальной базы знаний всей Экосистемы OSTIS
		\begin{scnitemize}
		\item Если в результате такой проверки выяснилось, что указанная сущность уже рассматривалась, то Вы должны использовать
введенный ранее основной внешний идентификатор этой сущности (Если он Вам не нравится, можете предложить, но пока не использовать, свой)
		\item Если сущность не рассматриваласть, нужно специфицировать, связать с семантически
близким (особенно для понятий)
		\end{scnitemize}
	\end{scnitemize}

От толковых словарей и энциклопедий – к стройной
семантической сети таких спецификации \uline{всех} описываемых сущностей, которые позволяют установить (желательно автоматически) наличие или отсутствие в рамках технического состояния базы
знаний синонимичного знака для любого нового знака, вводимого в базу знаний.}
\scntext{cтруктура качественной спецификации}{
Нужно стремиться:
	\begin{scnitemize}
	 	\item к однозначности такой спецификации;
		\item "координаты"{} в пространстве декомпозиций
		\item к семантической близости;
		\item сходства, отличия;
	\end{scnitemize}}
	

\scnheader{качество человеческой деятельности}
\scnidtf{качество деятельности человеческого общества}
\scnexplanation{Поскольку человеческого общество в целом является кибернетической системой, (которая принадлежит классу иерархических многоагентных систем, качество деятельности человеческого общества можно оценивать по критериям качества кибернетических систем.\\
На основании этих критериев можно оценивать:
	\begin{scnitemize}
	\item качество информационной среды, формируемой человеческим обществом, т.е. качество накапливаемой и общедоступной информации;
	\item качество текущего состояния общечеловеческих знаний;
	\item качество методов и технологий, используемых для решения задач как в рамках накопленных человечеством знаний, так и в
рамках внешней среды человеческого общества;
	\item качество организаций человеческой деятельности в целом;
	\item обучаемость (темпы эволюции) человеческого общества в целом.
	\end{scnitemize}
	
Современный этап развития науки и техники характерен тем, что при оценке качества научно-технических результатов акцентируется внимание на новизне результатов, на их \uline{отличиях} от текущего положения дел. Это создает почву и для
имитации этой новизны и для увеличения барьеров между различными дисциплинами, что существенно препятствует конвергенции и интеграции различных дисциплин. Указанная конвергенция и инеграция, в частности, необходима для \uline{комплексной} автоматизации \uline{всех} видов человеческой
деятельности в рамках smart-общества. Очевидно, что основной такой комплексной автоматизации должна быть \textbf{Общая формальная теория человеческой деятельности}.}

\scnheader{уровень конвергенции и интерации человеческой деятельности и её результатов}
\scnrelto{свойство-предпосылка}{Качество человеческой деятельности}
\scnexplanation{Повышение уровня конвергенции и интеграции различных видов человеческой деятельности и, соответственно, результатов
этой деятельности является важнейшим фактором (важнейшим направлением) повышения качества
(эффективности) человеческой деятельности, а, следовательно, и качества самого человеческого общества как сложной распределенной социотехнической кибернетической системы.}
\scntext{вопрос}{Что является главным препятствием существенному повышению уровня конвергенции и интеграции человеческой деятельности.}
	\scnaddlevel{1}
	\scntext{ответ}{Главным препятствием повышению уровня конвергенции и интеграции человеческой деятельности является то, что на текущем этапе эволюции человеческого общества основным механизмом эволюции является конкуренция. Конкуренция предполагает противопоставление результатов своей деятельности результатам конкурентов. т.е. акцентирует внимание на отличиях, новизне, преимуществах своих результатов по отношению к результатам своих конкурентов. При этом мысль о целесообразности объединения усилий со своими конкурентами чаще обусловлена стремлением повысить уровень конкурентоспособности и прибыли по отношению к другим более сильным конкурентам и значительно реже обусловлена искренним стремлением получить более качественный результат.\\
	Таким образом, повышение уровня конвергенции и интеграции всех видов человеческой деятельности требует весьма сложного перехода от использования механизма конкуренции в её современном виде к созданию мощной технологической основы, обеспечивающей широкое взаимовыгодное сотрудничество и гарантированные возможности самореализации каждого человека и каждого коллектива. Фундаментом указанной технологической основы может и должен стать общечеловеческий рынок знаний, который построен на базе сети интеллектуальных компьютерных систем и в рамках которого фиксируется и объективно оценивается значимость вклада каждого человека и каждого коллектива.}
	\scnaddlevel{-1}
	