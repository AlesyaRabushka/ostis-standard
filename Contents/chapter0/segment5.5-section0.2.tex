\bigskip
\scnfragmentcaption

\scnheader{Деятельность в области Искусственного интеллекта, осуществляемая на основе Технологии OSTIS}
\scnidtf{Проект Экосистемы OSTIS}

\scnheader{Экосистема OSTIS}
\scniselement{ostis-сообщество}
\scnrelto{субъект}{Объединенная человеческая деятельность, осуществляемая на основе Технологии OSTIS}
\scnrelfrom{корпоративная ostis-система}{Корпоративная система Экосистемы OSTIS}
\scnaddlevel{1}
\scnexplanation{Основным назначением \textit{Корпоративной системы Экосистемы OSTIS} является организация общего взаимодействия при выполнении самых различных видов и \textit{областей человеческой деятельности}, которые могут быть либо полностью автоматизированными, либо частично автоматизированными, либо вообще неавтоматизированными. Из этого следует, что база знаний \textit{Корпоративной системы Экосистемы OSTIS} должна содержать \textit{Общую формальную теорию человеческой деятельности}, включающей в себя типологию видов и областей \textit{человеческой деятельности}, а также общую \textit{методологию} этой \textit{деятельности}.}
\scnaddlevel{-1}

\scnheader{Деятельность в области Искусственного интеллекта, осуществляемая на основе Технологии OSTIS}
\scnrelfrom{субъект}{Консорциум OSTIS}
\scnaddlevel{1}
\scnrelfrom{корпоративная ostis-система}{Корпоративная ostis-система Консорциума OSTIS}
\scnaddlevel{-1}
\scnrelfrom{основной продукт}{Экосистема OSTIS}
\scnaddlevel{1}
    \scnrelfrom{частное ostis-сообщество}{Консорциум OSTIS}
    \scnrelfrom{член ostis-сообщества}{Консорциум OSTIS}
    \scnaddlevel{1}
    \scnrelfrom{примечание}{\scnstartsetlocal\\
    \scnheaderlocal{член ostis-сообщества*}
    \scnsubset{частное ostis-сообщество*}
    \scnaddlevel{1}
        \scnidtf{ostis-сообщество, входящее в состав заданного либо непосредственно (в качестве члена), либо в качестве члена члена заданного ostis-сообщества и т. д.}
    \scnaddlevel{-1}
    \scnendstruct}

\scnaddlevel{-2}
\scnidtf{Проект, основной целью (продуктом) которого является создание Экосистемы OSTIS}
\scnrelfrom{часто используемый sc-идентификатор}{Проект Экосистемы OSTIS}
\scnidtf{Деятельность, направленная на создание и перманентное развитие \textit{Экосистемы OSTIS}}
\scnidtf{Проект, направленный на проектирование, производство и реинжиниринг \textit{ostis-систем}, входящих в сеть \textit{Экосистемы OSTIS}, а также на проектирование и реинжиниринг Экосистемы OSTIS в целом (как сети \textit{ostis-систем} и их пользователей)}
\scnrelfromlist{подпроект}{Проект IMS.ostis;Проект программной реализации абстрактной sc-машины;Проект разработки универсального sc-компьютера}
\scnnote{В состав \textit{Проекта Экосистемы OSTIS} входит большое количество \textit{проектов} (\textit{подпроектов}*), направленных на \textit{проектирование} и \textit{производство ostis-систем} самого различного назначения}
\scnexplanation{Распространим предлагаемый нами подход к повышению эффективной и человеческой \textit{Деятельности в области Искусственного интеллекта} на всю \textit{Объединенную человеческую деятельность} в целом, т.е. рассмотрим структуру Глобального \textit{ostis-сообщества} (\textit{Экосистемы OSTIS})}
\scntext{эпиграф}{От \textit{Консорциума OSTIS} к \textit{Экосистеме OSTIS}}

\scnheader{Экосистема OSTIS}
\scnnote{Подчеркнем, что \textit{Экосистема OSTIS} является:
    \begin{scnitemize}
        \item с одной стороны, \textit{основным продуктом*} человеческой \textit{Деятельности в области Искусственного интеллекта, осуществляемой на основе Технологии OSTIS} (эту Деятельность мы также будем называть Проектом Экосистемы OSTIS),
        \item а, с другой стороны, \textit{субъектом* Объединенной человеческой деятельности, осуществляемой на основе Технологии OSTIS}.
    \end{scnitemize}

    Особо подчеркнем то, что продуктом человеческой \textit{Деятельности в области Искусственного интеллекта, осуществляемой на основе Технологии OSTIS}, является не просто множество \textit{ostis-систем} различного назначения, а Экосистема, состоящая из \underline{взаимодействующих} \textit{ostis-систем} и их пользователей}
\scnexplanation{Принципиальным является то, что продуктом (результатом применения) \textit{Технологии OSTIS} является не просто множество \textit{ostis-систем}, а целая система, состоящая из \textit{ostis-систем} и их пользователей, взаимодействующих между собой и осуществляющих комплексную автоматизацию всех \textit{видов человеческой деятельности}, а также комплексное повышение уровня эффективности организации человеческой деятельности (и, в частности, повышение уровня автоматизации этой деятельности).}

\bigskip
\scnfragmentcaption

\scnheader{Экосистема OSTIS}
\scnrelfromlist{вопрос}{Каковы основные свойства Экосистемы OSTIS;Какова основная цель создания Экосистемы OSTIS}
\scnidtf{Экосистема ostis-систем и их пользователей}
\scnrelto{общий создаваемый продукт}{Технология OSTIS}
\scnidtf{Расширяемый коллектив эволюционируемых, семантически совместимых и взаимодействующих ostis-систем и их пользователей}
\scniselement{многоагентная система}
\scnidtf{Многоагентная система, агентами которой являются ostis-системы, а также их конечные пользователи и разработчики}

\scnheader{Экосистема интеллектуальных компьютерных систем}
\scnidtf{Smart-сообщество}
\scnidtf{Smart-сообщество интеллектуальных компьютерных систем и людей}
\scnidtf{Интеллектуальная многоагентная система, состоящая из интеллектуальных компьтерных систем и людей}
\scnnote{Многоагентная система может состоять из кибернетических систем, не являющихся интеллектуальными.}
\scnnote{Многоагентная система может состоять из интеллектуальных систем, но сама не быть интеллектуальной. Количество далеко не всегда переходит в нужное качество.}
\scnidtf{Экосистема ostis-систем, а также их разработчиков и пользователей}
\scnidtf{Эволюционирумая сеть ostis-систем, обеспечивающая конвергенцию и интеграцию всех видов человеческой деятельности}

\scnheader{Экосистема OSTIS}
\scnidtf{Глобальная компьютерная сеть ostis-систем, обеспечивающая комплексную автоматизацию всевозможных видов и областей человеческой деятельности и отражающая иерархию уровней этой деятельности}
\scnidtf{Глобальная \textit{многоагентная система}, состоящая из людей и семантически совместимых \textit{интеллектуальных компьютерных систем}, построенных по \textit{Технологии OSTIS}, которые, взаимодействуя между собой и с людьми, обеспечивают существенное повышение уровня автоматизации всех видов \textit{человеческой деятельности} и существенное повышение эффективности человеческого взаимодействия}
\scnidtf{Предлагаемых нами подход к реализации smart-общества}
\scnidtf{Smart-общество, построенное на основе Технологии OSTIS}
\scnidtf{следующий этап развития человеческого общества, обеспечивающий существенное повышение уровня общественного (коллективного) интеллекта путем преобразования человеческого общества в экосистему, состоящую из людей и семантически совместимых интеллектуальных систем}
\scntext{обоснование}{Необходимо обеспечить не только повышение уровня автоматизации человеческой деятельности (как информационной (умственной), так и "физической"{}), но и существенное повышение уровня интеллекта человеческого общества как социальной  кибернетической системы путем создания многоагентной кибернетической системы, сосотоящей из \textit{интеллектуальных компьютерных систем} и людей и имеющей \textit{высокий уровень интеллекта}. Человечества пока не умеет создавать инетеллектуальные сообщества (коллективы) людей и, тем более, интеллектуальные общества людей и интеллектуальных компьютерных системы.
Уровень интеллекта каждого такого сообщества обычно определяется уровнем интеллекта его руководителя (лица, принимающего решение).А надо, чтобы уровень интеллекта сообщества был результатом интеграции интеллектуального потенциала всех его членов. При этом следует помнить, что интеллект определяется не только и не столько множеством решаемых задач, а \uline{скоростью} расширения этого множества.}

\scnnote{Предметом инженерной деятельности в области \textit{искусственного интеллекта} следует считать не множество \textit{интеллектуальных компьютерных систем} (например, \textit{ostis-систем}), а весь комплекс взаимодействующих между собой \textit{интеллектуальных компьютерных систем}. Назовём такой комплекс \textit{Экосистемой интеллектуальных компьютерных систем} (в нашем случае – это Экосистема OSTIS – Экосистема взаимодействующих \textit{ostis-систем}) здесь важно построить архитектуру таковой экосистемы, в основе которой должна лежать комплексная формальная модель всевозможных видов человеческой деятельности, автоматизируемых с помощью интеллектуальных компьютерных систем (ostis-систем). Указання комплексная модель человеческой деятельности является необходимой основой создания smart-общества (общества 5.0)}

\scnrelto{общий создаваемый продукт}{Технология OSTIS}
\scnidtf{Общий (объединенный, интегрированный) продукт использования \textit{Технологии OSTIS}, представляющий собой глобальную сеть \textit{ostis-систем}, обеспечивающий комплексную автоматизацию и интеграцию всевозможных \textit{видов человеческой деятельности} и, в частности, включающий в себя (в виде соответствующего \textit{ostis-сообщества}) \textit{консорциум OSTIS}, т.е. инфраструктуру, направленную на перманентное развитие \textit{Технологии OSTIS} (как Ядра Технологии OSTIS, так и иерархического семейства \textit{специализированных ostis-технологий})}
\scntext{следовательно}{\textit{Экосистема OSTIS} представляет собой саморазвивающуюся сеть ostis-систем}

\scnexplanation{Сверхзадачей \textit{Экосистемы OSTIS} является не просто комплексная автоматизация всех \textit{видов человеческой деятельности} (разумеется, только тех видов деятельности, автоматизация которых целесообразна), но и существенное повышение уровня интеллекта различных человеческих (точнее человеко-машинных) сообществ и всего человеческого общества в целом. Это потребует соблюдения ряда требований, предъявляемых не только к \textit{интеллектуальным компьютерным системам}, но и к людям, входящим в состав \textit{Экосистемы OSTIS}}

\scnexplanation{\textit{Экосистема OSTIS} представляет собой открытый коллектив взаимодействующих интеллектуальных систем, состав которого входят \textit{ostis-системы} и их пользователи (конечные пользователи и разработчики,участвующие в совершенствование этих \textit{ostis-систем}). Особое место среди \textit{ostis-систем}, входящих в состав \textit{экосистемы OSTIS}, занимают \textit{корпоративные ostis-системы}, через которое осуществляется координация и эволюция деятельности некоторых групп \textit{ostis-систем} и их пользователей. Основная цель корпоративных \textit{ostis-систем} -- локализовать базы знаний указанных групп ostis-систем, перевести их из статуса виртуальных в статус реальных и автоматизировать их эволюцию.}
\scnheader{Экосистема OSTIS}
\scnidtf{Сообщество ostis-систем или людей, обеспечивающее принципиально новый уровень автоматизации человеческой деятельности и принципиально \uline{новый уровень интеллекта человеческого общества}}

\scnheader{Экосистема OSTIS}
\scnnote{Очень важно проектировать не только саму \textit{Экосистему OSTIS}, ну и процесс \uline{поэтапного перехода} от современной глобальной сети \textit{компьютерных систем} к глобальной сети \textit{ostis-систем} (т.е. к \textit{Экосистеме OSTIS}). В рамках такого переходного периода \textit{ostis-системы} могут выполнять роль системных интеграторов различных ресурсов и сервисов, реализованных современными \textit{компьютерными системами}, поскольку уровень интеллекта \textit{ostis-систем} позволяет им с любой степенью детализации специфицировать интегрируемые \textit{компьютерные системы} и, следовательно, достаточно адекватно "понимать"{}, что знает и/или умеет каждая из них, а также достаточно качественно координировать их деятельность и обеспечивать "релевантный"{} поиск нужного ресурса и сервиса. Кроме того системы могут выполнять роль интеллектуальных help-систем -- помощников и консультантов по вопросам эффективной эксплуатации различных \textit{компьютерных систем} со сложными функциональными возможностями, имеющими пользовательский интерфейс с нетривиальной семантикой и использующимися в сложных предметных областях. Такие интеллектуальные help-системы можно сделать интеллектуальными посредниками между соответствующими компьютерными системами их пользователями. При этом пользователь может работать одновременно и с help-системой и с соответствующей эксплуатируемой компьютерной системой, консультируясь с help-системой в затруднительных для него ситуациях. Основными недостатками такого варианта является то,что: (1) пользователь должен использовать два разных интерфейса и (2) help-система не может мониторить деятельность пользователя и, следовательно, пользователь сам должен сообщать системе о своих затруднительных ситуациях. Указанные недостатки можно устранить, если компьютерную систему, которая построена по современным технологиям и эксплуатация которой нуждается в качественной консультационной (help-овой) поддержке, интегрировать с соответствующей ей help-системой, построенной по стандартам технологии OSTIS, так, чтобы пользовательским интерфейсом такой интегрированной системы стал пользовательский интерфейс, соответствующий стандартам технологии OSTIS и важнейшим достоинством которого является чёткая формализация семантики всех элементов управления пользовательским интерфейсом. Благодаря этому взаимодействие пользователя с пользовательским интерфейсом ostis-систем становится, во-первых, осмысленным и, во-вторых, позволяющим легко переносить опыт интерфейсного взаимодействия с одной ostis-системы на другую ostis-систему.
}
