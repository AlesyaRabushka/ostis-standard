\begin{SCn}

\scnsectionheader{\currentname}

\scnstartsubstruct

\scnheader{Встраиваемая ostis-система комплексной поддержки проектирования баз знаний ostis-систем}
\scnidtf{Встраиваемая типовая интеллектуальная система комплексной автоматизации проектирования, а также управления процессом коллективного проектирования и совершенствования баз знаний интеллектуальных систем на всех этапах их жизненного цикла}
\scnidtf{Интеллектуальная система автоматизированного проектирования баз знаний}
\scnidtf{Встраиваемая интеллектуальная система, поддержки проектирования и совершенствования баз знаний интеллектуальных систем на всех этапах их жизненного цикла}
\scnidtf{Интеллектуальный компьютерный фреймворк баз знаний интеллектуальных систем, разрабатываемых по Технологии OSTIS}
\scnexplanation{Известно, что разработка базы знаний интеллектуальной системы является весьма трудоемким процессом, во много определяющим качество интеллектуальной системы. Очевидно также, что сокращение сроков разработки базы знаний возможно путем организации коллективной разработки, но при условии решения ряда задач, например:

\begin{scnitemize}
    \item Как в рамках коллектива разработчиков одной и той же базы знаний предотвратить синдром "лебедя, рака и щуки"\,, или синдром "семи нянек"\ и как снизить накладные расходы на согласование их деятельности по созданию качественной базы знаний.
    \item Как обеспечить возможность включения любых уже формализованных знаний в базу знаний любой интеллектуальной системы (если они там необходимы) без какой-либо "ручной"\ корректировки этих знаний и тем самым полностью исключить повторную разработку и адаптацию этих знаний.
\end{scnitemize}

Качество базы знаний определяется следующими ее характеристиками:
\begin{scnitemize}
    \item полнота = целостность = отсутствие информационных дыр
    \item непротиворечивость = корректность = отсутствие ошибок
\end{scnitemize}    
\begin{scnitemize}

\item актуальность = соответствие текущему состоянию внешней среды и текущему состоянию общечеловеческих знаний о внешней среде

\item структуризация.
\end{scnitemize}

Проектирование интеллектуальных систем заключается в построении семантической модели этой интеллектуальной системы, включающей в себя и модель обрабатываемых знаний, и различные модели решения различных классов задач, и различные модели взаимодействия интеллектуальных систем с ее внешней средой. При этом обрабатываемыми знаниями могут быть и модели решения задач в базе знаний, и модели решения интерфейсных задач, которые, соответственно, также должны входить в состав базы знаний интеллектуальных систем.

Комплекс средств проектирования интеллектуальных систем можно разделить на

\begin{scnitemize}
    \item средства проектирования баз знаний;
    \item средства проектирования решателей интеллектуальных систем;
    \item средства проектирования интерфейсов интеллектуальных систем.
\end{scnitemize}

При этом существенно подчеркнуть, что проектирование решателя задач интеллектуальной системы заключается в проектировании знаний специального вида — навыков и спецификаций агентов, осуществляющих интерпретацию этих навыков при решении конкретных задач. Проектирование интерфейсов интеллектуальных систем сводится к проектированию знаний, представляющих собой семантическую модель встроенной интеллектуальной системы, ориентированной на решение интерфейсных задач.

Данная встраиваемая интеллектуальная система осуществляет:
\begin{scnitemize}
    \item мониторинг деятельности каждого участника процесса проектирования баз знаний, что необходимо для защиты его авторских прав, для оценки объема и значимости его вклада в проектную деятельность, для оценки его профессиональной квалификации, для качественного распределения новых проектных работ с учетом его текущей квалификации и планируемого направления повышения этой квалификации, для реализации откатов, то есть отмены ошибочных решений, принятых администраторами или менеджерами проектируемой базы знаний;
    \item контроль версий проектируемой базы знаний, реализацию необходимых возвратов к предшествующим версиям;
    \item контроль исполнительской дисциплины;
    \item анализ текущего состояния и динамики процесса проектирования, выявление критических ситуаций;
    \item семантический анализ корректности результатов проектных работ всех участников;
    \item оценку объема и значимости деятельности каждого участника проекта;
    \item оценку текущего состояния и динамики развития квалификационного портрета каждого участника проекта;
    \item формирование рекомендаций по повышению квалификации каждого участника проекта;
    \item контроль качества (непротиворечивости, целостности, полноты, чистоты) текущего состояния проектируемой и совершенствуемой базы знаний.
\end{scnitemize}

Каждый участник процесса проектирования базы знаний может выполнять различные виды проектных работ:
\begin{scnitemize}
    \item предложить новый фрагмент в согласованную часть базы знаний или некоторую корректировку (удаление, изменение) в этой части базы знаний;
    \item высказать согласие или несогласие с предложенной кем-то корректировкой или добавлением в согласованную часть базы знаний;
    \item провести верификацию, тестирование, рецензирование предложенной кем-то корректировки или добавления в согласованную часть базы знаний и написать замечания к доработке этого предложения;
    \item предложить формулировку нового проектного задания, например, на устранение указываемого противоречия (ошибки), на заполнение указываемой информационной дыры;
    \item высказать конструктивные критические замечания к формулировке нового проектного задания;
    \item  предложить исполнителя или группу исполнителей для выполнения пока не исполняемого проектного задания;
    \item высказать конструктивные критические замечания к предложенным исполнителям некоторого свободного проектного задания.
\end{scnitemize}

}

\scnendstruct

\end{SCn}