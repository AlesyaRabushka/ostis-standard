\begin{SCn}

\scnsectionheader{\currentname}
\scnrelfromlist{подраздел}{Предметная область и онтология многократно используемых компонентов баз знаний ostis-систем;Предметная область и онтология многократно используемых внутренних агентов ostis-систем;Предметная область и онтология многократно используемых интерпретируемых ostis-системами методов;Предметная область и онтология многократно используемых компонентов интерфейсов ostis-систем;Библиотека многократно используемых встраиваемых ostis-систем}

\scnstartsubstruct

\scnheader{Предметная область многократно используемых компонентов ostis-систем}
\scnsdmainclasssingle{...}
\scnsdclass{}
\scnsdrelation{}

\scnheader{Библиотека многократно используемых компонентов OSTIS}
\scnidtf{Библиотека OSTIS}
\scnidtf{многократно используемый компонент OSTIS}
\scnidtf{многократно используемый компонент интеллектуальных систем, построенных по Технологии OSTIS}
\scnexplanation{Под \textbf{\textit{многократно используемым компонентом OSTIS}} понимается компонент некоторой ostis-системы, который может быть использован в рамках другой ostis-системы. Для этого необходимо выполнение как минимум двух условий:
\begin{scnitemize}
    \item есть техническая возможность встроить компонент в другую ostis-систему, либо путем физического копирования, переноса и встраивания его в проектируемую систему, либо использования компонента, размещенного в исходной системе наподобие сервиса, то есть без явного копирования и переноса компонента. Трудоемкость встраивания зависит, в том числе, от реализации компонента;
    \item использование компонента в каких-либо ostis-системах, кроме материнской, является целесообразным, то есть компонентом не может быть частное решение, ориентированное на узкий круг задач. Стоит, однако, отметить, что в общем случае практически каждое решение может быть использовано в каких-либо других системах, круг которых определяется степенью общности и предметной зависимостью такого решения.
\end{scnitemize}
С формальной точки зрения каждый \textbf{\textit{многократно используемый компонент OSTIS}} представляет собой \textit{структуру}, которая содержит все те (и только те) \textit{sc-элементы}, которые необходимы для функционирования компонента в \textit{дочерней ostis-системе} и, соответственно, должны быть в нее скопированы при включении компонента в одну из таких систем. Конкретный состав данной \textit{структуры} зависит от типа компонента и уточняется для каждого типа отдельно. По сути, данная \textit{структура} представляет собой эталон или образец, который копируется при включении соответствующего компонента в дочернюю систему.

Каждый \textbf{\textit{многократно используемый компонент OSTIS}} может быть атомарным, либо неатомарным, то есть состоять из более простых самодостаточных компонентов.

В зависимости от типа компонента в его составе, т.е. в составе соответствующей \textit{структуры}, могут дополнительно вводиться роли некоторых \textit{sc-элементов}, если это необходимо. Например, в случае \textit{многократно используемого sc-агента}, сам \textit{sc-узел}, обозначающий \textit{sc-агент}, будет являться \textit{ключевым sc-элементом'} в рамках компонента.

В каждый момент времени в текущем состоянии \textit{sc-памяти} каждый многократно используемый компонент может представлен полностью, т.е. в памяти явно присутствуют все \textit{sc-дуги принадлежности}, соединяющие соответствующую компоненту \textit{структуру} и все ее элементы, или представлен неявно, например, при помощи указания \textit{ключевых sc-элементов'} данного компонента или путем задания декомпозиции данного компонента на более частные.

Каждый \textbf{\textit{многократно используемый компонент OSTIS}} имеет формальную спецификацию, то есть некоторую \textit{семантическую окрестность}, характеризующую данный компонент. На основе формальной спецификации осуществляется поиск подходящего компонента в библиотеке, сравнение его с другими компонентами и т.д.

Данная спецификация включает, как минимум:
\begin{scnitemize}
    \item Информацию об авторстве компонента, то есть связь компонента со знаком автора (физического лица, коллектива и т.д.) при помощи отношения \textit{автор*};
    \item Информацию о типе компонента, посредством указания принадлежности компонента какому-либо классу многократно используемых компонентов;
    \item Описание назначения компонента, его особенностей;
    \item Историю изменений компонента по версиям;
    \item При необходимости сведения об открытости компонента и возможностях его использования в различных системах с точки зрения проприетарности;
    \item И др.
\end{scnitemize}
Не следует путать понятия \textbf{версии компонента} и \textbf{модификации компонента}. Версии отражают историю изменений компонента (как правило, какие-либо улучшения или устранения ошибок). Модификации представляют собой функционально эквивалентные, но разные варианты реализации одного и того же компонента, которые могут быть синтаксически эквивалентны (то есть быть реализованными при помощи одних и тех же языковых средств). В качестве примера синтаксически не эквивалентной модификации можно привести реализацию одного и того же \textit{sc-агента} на одном и том же языке но с отличиями в алгоритме, в качестве синтаксически эквивалентной модификации – платформенно-зависимую и платформенно-независимую реализацию одного и того же \textit{sc-агента}.

В общем случае система \textit{IMS} как материнская система взаимодействует со всеми своими \textit{дочерними ostis-системами} (с системами, построенными по \textit{Технологии OSTIS}), обеспечивая в дочерних системах автоматическое обновление версий \textit{многократно используемых компонентов OSTIS}. Любая дочерняя система, построенная по \textit{Технологии OSTIS}, в том числе, выполняет роль посредника между разработчиком такой системы и системой \textit{IMS}. Разработчик имеет возможность выбрать интересующий его компонент или набор компонентов в одной из библиотек, и включить их в разрабатываемую дочернюю систему. Таким образом, можно говорить о том, что \textbf{разработчик} систем, построенных по \textit{Технологии OSTIS}, является \textbf{конечным пользователем} системы \textit{IMS}. При обеспечении такого механизма взаимодействия между системами, построенными на основе \textit{Технологии OSTIS}, указанные системы формируют Глобальную базу знаний, в пределах которой различные системы могут координироваться и решать более глобальные задачи, нежели это может делать одна отдельно взятая система. В случае намеренной изоляции какой-либо из систем из такого коллектива, в частности, потери связи с системой \textit{IMS}, она теряет возможность получать своевременные обновления используемых компонентов, а также использовать знания, накопленные в других системах для решения стоящих перед ней задач. Речь в данном случае идет не только о физической изоляции рассматриваемой системы, которая может быть легко устранена, а о рассогласовании знаний данной системы и других систем, что не позволит безболезненно интегрировать компоненты из библиотек в такую систему. Таким образом, разработчик каждой системы обязан следить за тем, чтобы его система находилась в постоянном согласовании с глобальным смысловым пространством, что позволит пользователям в полной мере использовать все возможности коллектива систем.

В некоторых случаях может оказаться, что для использования одного \textbf{\textit{многократно используемого компонента OSTIS}} целесообразно или даже необходимо дополнительно использовать несколько других \textbf{\textit{многократно используемых компонентов OSTIS}}. Например, может оказаться целесообразным вместе с каким либо \textit{sc-агентом информационного поиска} использовать соответствующую команду интерфейса, которая представлена отдельным компонентом и позволит пользователю задавать вопрос для указанного \textit{sc-агента} через интерфейс системы. В таких случаях для связи компонентов используется отношение \textit{сопутствующий компонент*}. Наличие таких связей позволяет устранить возможные проблемы неполноты знаний и навыков в дочерней системе, из-за которых какие-либо из компонентов могут не выполнять свои функции. Связки отношения \textit{сопутствующий компонент*} связывают \textbf{\textit{многократно используемые компоненты OSTIS}}, которые целесообразно или необходимо использовать в дочерней системе вместе. При этом каждая связка направляется от зависящего компонента к зависимому. Каждая такая связка может дополнительно быть снабжена \textit{комментарием} или \textit{пояснением}, отражающим суть указываемой зависимости.

Включение компонента в \textit{дочернюю ostis-систему} в самом общем случае состоит из следующих этапов:
\begin{scnitemize}
    \item поиск подходящего компонента (или набора компонентов) во множестве библиотек, входящих в состав \textit{IMS}. Для облегчения задачи поиска могут быть реализованы специализированные поисковые агенты. В любом случае, поиск и выделение компонента будет осуществляться на основе спецификации компонента. Данный этап с точки зрения пользователя не зависит от типа компонента и особенностей его реализации. Конкретные действия на следующих  этапах сильно зависят от реализации и типа компонента и будут более детально описаны при рассмотрении подклассов \textbf{\textit{многократно используемых компонентов OSTIS}};
    \item выделение компонента (или набора компонентов) в рамках \textit{IMS} в виде, удобном для транспортировки в \textit{дочернюю ostis-систему} (при необходимости – создание физической копии компонента);
    \item транспортировка выделенного компонента в \textit{дочернюю sc-систему};
    \item интеграция компонента в \textit{дочернюю ostis-систему}. Если в системе уже использовалась более старая версия компонента, то необходимо произвести либо локальное обновление, либо полную замену устаревшей версии компонента. Дальнейший процесс интеграции зависит от типа компонента, например, в случае добавления нового \textit{sc-агента} он должен быть помечен как \textit{активный sc-агент} и т.п.
\end{scnitemize}

Для обеспечения возможности встраивания \textbf{\textit{многократно используемых компонентов OSTIS}} в дочернюю систему, каждая такая система обязана иметь в своем составе средства, обеспечивающие интеграцию новых компонентов в систему и, при необходимости, удаление устаревших версий этих компонентов (или автоматического локального обновления компонентов до более новой версии).}

\scnauthorcomment{разбиение согласовать со структурой разделов или убрать вообще}

\scnsubdividing{Семейство платформ интерпретации sc-моделей компьютерных систем
;Библиотека многократно используемых компонентов sc-моделей баз знаний;Библиотека шаблонов типовых компонентов sc-моделей компьютерных систем;Библиотека многократно используемых компонентов абстрактных sc-машин;Библиотека многократно используемых компонентов sc-моделей интерфейсов компьютерных систем;Библиотека типовых подсистем компьютерных систем, разрабатываемых по Технологии OSTIS}
\scnsubdividing{атомарный многократно используемый компонент OSTIS;неатомарный многократно используемый компонент OSTIS}
\scnsubdividing{платформенно-зависимый многократно используемый компонент OSTIS;платформенно-независимый многократно используемый компонент OSTIS}

\scnheader{шаблон типового компонента OSTIS}

\scnheader{атомарный многократно используемый компонент OSTIS}
\scnexplanation{Под \textbf{\textit{атомарным многократно используемым компонентом OSTIS}} подразумевается компонент, который в текущем состоянии библиотеки компонентов рассматривается как неделимый, то есть не содержит в своем составе других компонентов, представленных в какой-либо из библиотек компонентов в рамках \textit{IMS}. В общем случае атомарный компонент может перейти в разряд неатомарных в случае, если будет принято решение выделить какую-то из его частей в качестве отдельного компонента. Все вышесказанное, однако, не означает, что даже в случае его платформенной независимости, атомарный компонент всегда хранится в sc-памяти как сформированная sc-структура. Например, \textit{платформенно-независимая реализация sc-агента} всегда будет представлена набором \textit{scp-программ}, но будет \textbf{\textit{атомарным многократно используемым компонентом OSTIS}} в случае, если ни одна из этих программ не будет представлять интереса как самостоятельный компонент.}

\scnheader{неатомарный многократно используемый компонент OSTIS}
\scnexplanation{Под \textbf{\textit{неатомарным многократно используемым компонентом OSTIS}} подразумевается компонент, который в текущем состоянии библиотеки компонентов содержит в своем составе более простые компоненты, представленные в каких-либо библиотеках компонентов в рамках \textit{IMS}. В общем случае неатомарный компонент может перейти в разряд атомарных в случае, если будет принято решение по каким-либо причинам исключить все его части из рассмотрения в качестве отдельных компонентов.

Следует отметить, что неатомарный компонент необязательно должен складываться полностью из других компонентов, в его состав могут также входить и части, не являющиеся самостоятельными компонентами. Например, в состав реализованного на \textit{языке SCP sc-агента}, являющего \textbf{\textit{неатомарным многократно используемым компонентом}} могут входить как \textit{scp-программы}, которые могут являться \textit{многократно используемыми компонентами} (а могут и не являться), а также \textit{агентная scp-программа}, которая не имеет смысла как многократно используемый компонент.}

\scnheader{платформенно-зависимый многократно используемый компонент OSTIS}
\scnexplanation{Под \textbf{\textit{платформенно-зависимым многократно используемым компонентом OSTIS}} понимается компонент, частично или полностью реализованный при помощи каких-либо сторонних с точки зрения \textit{Технологии OSTIS} средств. Основной недостаток платформенно-зависимых компонентов состоит в том, что их интеграция в интеллектуальные системы может сопровождаться дополнительными трудностями, зависящими от конкретных средств реализации компонента. В качестве возможного преимущества \textbf{\textit{платформенно-зависимых многократно используемых компонентов OSTIS}} можно выделить их, как правило, более высокую производительность за счет реализации их на более приближенном к платформе уровне.

В общем случае \textbf{\textit{платформенно-зависимый многократно используемый компонент OSTIS}}  может поставляться как в виде набора исходных кодов, так и бинарном виде, например в виде скомпилированной библиотеки.

Процесс интеграции \textbf{\textit{платформенно-зависимого многократно используемого компонента OSTIS}} в дочернюю систему, разработанную по \textit{Технологии OSTIS}, сильно зависит от технологий реализации данного компонента и в каждом конкретном случае может состоять из различных этапов.

Для того чтобы \textbf{\textit{платформенно-зависимый многократно используемый компонент OSTIS}} мог быть успешно встроен в дочернюю систему, необходимо выполнение следующих условий:
\begin{scnitemize}
    \item в состав \textit{структуры}, соответствующей компоненту, должны входить знаки \textit{файлов}, обозначающих исходные тексты компонента или уже собранной его версии, то есть ссылки на внешние ресурсы или явно включенные в систему файлы компонента в виде указанных \textit{файлов};
    \item каждый \textbf{\textit{платформенно-зависимый многократно используемый компонент OSTIS}} должен иметь соответствующую подробную, корректную и понятную инструкцию по его установке и внедрению в дочернюю систему;
\end{scnitemize}
}

\scnheader{платформенно-независимый многократно используемый компонент OSTIS}
\scnexplanation{Под \textbf{\textit{платформенно-независимым многократно используемым компонентом OSTIS}} понимается компонент, который целиком и полностью представлен в \textit{SC-коде}. В случае \textit{неатомарного многократно используемого компонента} это означает, что все более простые компоненты, входящие в его состав также обязаны быть \textbf{\textit{платформенно-независимыми многократно используемыми компонентами OSTIS}}.

Процесс интеграции \textbf{\textit{платформенно-зависимого многократно используемого компонента OSTIS}} в дочернюю систему, разработанную по \textit{Технологии OSTIS}, существенно упрощается за счет использования общей унифицированной формальной основы представления и обработки знаний.

В случае \textbf{\textit{платформенно-независимого многократно используемого компонента OSTIS }} процесс интеграции конкретизируется до следующих этапов:
\begin{scnitemize}
    \item формирование \textit{структуры}, явно содержащей все \textit{sc-элементы}, входящие в состав компонента, а также все \textit{sc-элементы}, входящие в спецификацию компонента, необходимую для его функционирования в дочерней системе. В случае \textbf{\textit{неатомарного многократно используемого компонента OSTIS}} в указанную \textit{структуру} должны быть полностью включены и все более частные компоненты;
    \item транспортировка компонента в дочернюю систему. В худшем случае может быть осуществлена путем выгрузки всей \textit{структуры} компонента в какой-либо формат, например \textit{SCs-код}, с последующим переносом файла в дочернюю систему разработчиком вручную. В общем случае, дочерняя система по команде разработчика должна самостоятельно обращаться к родительской и осуществлять загрузку необходимых компонентов;
    \item интеграция нового компонента в \textit{дочернюю ostis-систему} либо обновление компонента с более старой версии. Если функция обновления не поддерживается, то интеграция может проводиться в два этапа – удаление старой версии компонента и добавление в систему более новой версии. В зависимости от типа компонента (\textit{многократно используемый компонент баз знаний} или \textit{многократно используемый компонент машин обработки знаний}) интеграция осуществляется по-разному.
\end{scnitemize}
}

\scnheader{Библиотека многократно используемых компонентов абстрактных sc-машин}
\scnidtf{многократно используемый компонент абстрактных sc-машин обработки знаний}
\scntext{примечание}{Если \textbf{\textit{многократно используемый компонент абстрактных sc-машин обработки знаний}} является \textit{платформенно-зависимым многократно используемым компонентом OSTIS}, то его интеграция производится в соответствии с инструкцией, как и для любого компонента такого рода. В противном случае, процесс интеграции можно конкретизировать в зависимости от подклассов данного типа компонентов.}
\scnsubdividing{Библиотека многократно используемых абстрактных sc-агентов;Библиотека многократно используемых программ обработки sc-текстов}

\scnheader{шаблон типового компонента sc-моделей компьютерных систем}
\scnidtf{образец типового компонента OSTIS}
\scnidtf{атомарная логическая формула, описывающая структуру аналогичных (чаще всего изоморфных) компонентов баз знаний ostis-систем}
\scnidtf{Библиотека шаблонов типовых компонентов sc-моделей компьютерных систем}
\scnexplanation{В процессе использования \textbf{\textit{шаблона типового компонента sc-моделей компьютерных систем}} при формировании баз знаний проектируемых \textit{ostis-систем} вместо \textit{sc-переменных}, входящих в состав компонента, подставляются их значения.}

\scnheader{сопутствующий компонент*}

\scnheader{декомпозиция компонента библиотеки*}
\scniselement{квазибинарное отношение}
\scniselement{отношение декомпозиции}

\scnendstruct

\end{SCn}