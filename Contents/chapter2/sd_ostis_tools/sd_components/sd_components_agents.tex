\begin{SCn}

\scnsectionheader{\currentname}
\scnsuperset{Предметная область многократно используемых  внутренних агентов ostis-систем}
\scnaddlevel{1}
\scnsdmainclasssingle{...}
\scnsdclass{}
\scnsdrelation{}
\scnaddlevel{-1}

\scnstartsubstruct

\scnheader{Библиотека многократно используемых абстрактных sc-агентов}
\scnidtf{многократно используемый абстрактный sc-агент}
\scnsubdividing{Библиотека sc-агентов информационного поиска;Библиотека sc-агентов погружения интегрируемого знания в базу знаний;Библиотека sc-агентов выравнивания онтологии интегрируемого знания с основной онтологией текущего состояния базы знаний;Библиотека sc-агентов планирования решения явно сформулированных задач;Библиотека sc-агентов логического вывода;Библиотека sc-агентов обнаружения и удаления информационного мусора;Библиотека координирующих sc-агентов;Библиотека sc-моделей языков программирования высокого уровня и соответствующих им интерпретаторов}
\scnexplanation{Под \textbf{\textit{многократно используемым абстрактным sc-агентом}} подразумевается компонент, соответствующий некоторому \textit{абстрактному sc-агенту}, который может быть использован в других системах, возможно, в составе более сложных \textit{неатомарных абстрактных sc-агентов}. Указанный абстрактный sc-агент входит в соответствующую компоненту \textit{структуру} под атрибутом \textit{ключевой sc-элемент'}. Каждый \textbf{\textit{многократно используемый абстрактный sc-агент}} должен содержать всю информацию, необходимую для функционирования соответствующего \textit{sc-агента} в дочерней системе.

Таким образом, соответствующая \textbf{\textit{многократно используемому абстрактному sc-агенту}} \textit{структура} формируется следующим образом:
\begin{scnenumerate}
    \item в нее включается \textit{sc-узел}, обозначающий соответствующий \textit{абстрактный sc-агент}, и вся его спецификация, то есть, как минимум, указание \textit{ключевых sc-элементов sc-агента*}, \textit{условия инициирования и результат*}, \textit{первичного условия инициирования*}, \textit{sc-описание поведения sc-агента} и класса решаемых им задач;
    \item в случае, если входящий в \textbf{\textit{многократно используемый sc-агент}} \textit{абстрактный sc-агент} рассматривается как \textit{неатомарный абстрактный sc-агент}, то \textbf{\textit{многократно используемый sc-агент}} будет содержать \textit{sc-узлы}, обозначающие все более частные \textit{абстрактные sc-агенты}, а также все их спецификации согласно п.1. Для каждого включенного в \textbf{\textit{многократно используемый sc-агент}} \textit{абстрактного sc-агента} необходимо выполнить п.2 и п.3;
    \item для каждого \textit{атомарного абстрактного sc-агента}, знак которого вошел в \textbf{\textit{многократно используемый абстрактный sc-агент}} необходимовыбрать вариант его реализации (то есть элемент класса \textit{платформенно-независимый абстрактный sc-агент} или \textit{платформенно-зависимый абстрактный sc-агент}, связанный с исходным \textit{атомарным абстрактным sc-агентом} связкой отношения \textit{включение*}) и включить в \textbf{\textit{многократно используемый абстрактный sc-агент}} sc-узел, обозначающий указанную реализацию, а также знаки всех программ, входящие во множество, связанное с указанной реализацией отношением \textit{программа sc-агента*}. Выбранная реализация включается в \textbf{\textit{многократно используемый абстрактный sc-агент}} под атрибутом \textit{ключевой sc-элемент'}.
    \item в \textbf{\textit{многократно используемый абстрактный sc-агент}} включаются также все связки отношений, указанных в п.1-3, связывающие уже включенные в его состав sc-элементы, а также сами знаки этих отношений (например, \textit{включение*}, \textit{программа sc-агента*} и т.д.);
\end{scnenumerate}
После того, как \textbf{\textit{многократно используемый абстрактный sc-агент}} был скопирован в дочернюю систему, необходимо сгенерировать \textit{sc-узел}, обозначающий конкретный \textit{sc-агент}, работающий в данной системе, принадлежащий выбранной реализации \textit{абстрактного sc-агента} и добавить его во множество \textit{активных sc-агентов} при необходимости.

Также каждую \textit{scp-программу}, попавшую в \textit{дочернюю ostis-систему} при копировании \textbf{\textit{многократно используемого абстрактного sc-агента}}, необходимо добавить ее во множество \textit{корректных scp-программ} (корректность верифицируется при попадании в библиотеку компонентов в рамках IMS).}

\scnheader{Библиотека многократно используемых программ обработки sc-текстов}
\scnidtf{многократно используемая программа обработки sc-текстов}
\scnrelfrom{включение}{Библиотека многократно используемых scp-программ}
\scnexplanation{Под \textbf{\textit{многократно используемой программой обработки sc-текстов}} подразумевается компонент, соответствующий программе, записанной на произвольном языке программирования, которая ориентирована именно на обработку знаний, то есть с точки зрения \textit{Технологии OSTIS}, обработку \textit{структур}, хранящихся в памяти \textit{ostis-системы}. Приоритетным в данном случае является использование \textit{scp-программ} по причине их платформенной независимости, за исключением случаев проектирования некоторых компонентов интерфейса, когда полная платформенная независимость невозможна (например, при проектировании \textit{эффекторных sc-агентов} и \textit{рецепторных sc-агентов}).}

\scnheader{Библиотека многократно используемых scp-программ}
\scnidtf{многократно используемая scp-программа}
\scnexplanation{Под \textbf{\textit{многократно используемой scp-программой}} понимается компонент, соответствующий некоторой достаточно универсальной \textit{scp-программе}, которая может быть использована в составе сразу нескольких \textit{sc-агентов}. 

В \textbf{\textit{многократно используемую scp-программу}} включается полный текст \textit{scp-программы}, то есть все \textit{sc-элементы}, принадлежащие \textit{структуре}, являющейся \textit{scp-программой}, а так же все пары принадлежности между этой \textit{структурой} и ее элементами и знак самой этой \textit{структуры}. При этом сам \textit{sc-узел}, обозначающий \textit{scp-программу}, входит в соответствующий компонент под атрибутом \textit{ключевой sc-элемент'}.

После того, как \textbf{\textit{многократно используемая scp-программа}} была скопирована в дочернюю систему, необходимо добавить ее во множество \textit{корректных scp-программ} (корректность верифицируется при попадании в библиотеку компонентов в рамках \textit{IMS}).}

\scnendstruct

\end{SCn}