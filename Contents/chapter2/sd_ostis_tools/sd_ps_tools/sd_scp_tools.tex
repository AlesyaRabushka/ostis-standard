\begin{SCn}

\scnsectionheader{Семантическая модель средств поддержки проектирования программ Базового языка программирования ostis-систем}

\scnstartsubstruct

\scnheader{Средства поддержки проектирования программ Базового языка программирования ostis-систем}
\scnreltoset{базовая декомпозиция}{База знаний средств поддержки проектирования программ Базового языка программирования ostis-систем;Решатель задач средств поддержки проектирования программ Базового языка программирования ostis-систем;Пользовательский интерфейс средств поддержки проектирования программ Базового языка программирования ostis-систем}

\scnheader{точка останова*}
\scniselement{квазибинарное отношение}
\scnexplanation{Связки отношения \textit{точки останова*} связывают \textit{scp-программу} с некоторым множеством sc-переменных, соответствующих \textit{scp-операторам} в рамках этой программы. При генерации каждого \textit{scp-процесса}, соответствующего этой \textit{scp-программе}, все \textit{scp-операторы}, соответствующие таким переменным, будут добавлены во множество \textit{точка останова}, т. е. выполнение данного scp-процесса будет прерываться при достижении каждого из этих \textit{scp-операторов}.
Использование данного отношения приводит к указанию точек останова для всех \textit{scp-процессов}, формируемых на основе заданной \textit{\mbox{scp-программы}}. Для указания точки останова в рамках отдельно взятого \textit{scp-процесса} нужный scp-оператор явно включается во множество \textit{точка останова}.}

\scnheader{точка останова}
\scnrelto{включение}{scp-оператор}
\scnexplanation{Во множество \textit{точка останова} входят все \textit{scp-операторы}, являющиеся точками останова в рамках какого-либо \textit{scp-процесса}. Это означает, что в момент, когда в соответствии с переходами между \textit{scp-операторами} по связкам отношения \textit{последовательность действий*} указанный \textit{scp-оператор} должен стать \textit{активным действием}, он становится \textit{отложенным действием}, и, соответственно, выполнение всего \textit{scp-процесса} по данной ветке приостанавливается. Чтобы продолжить выполнение, необходимо удалить указанный \textit{\mbox{scp-оператор}} из множества \textit{отложенных действий} и добавить его во множество \textit{активных действий}.}

\scnheader{некорректность в scp-программе}
\scnrelto{включение}{некорректная структура}
\scnrelfrom{включение}{ошибка в scp-программе}
\scnrelfromlist{включение}{недостижимый scp-оператор;потенциально бесконечный цикл}
\scnexplanation{Под \textit{некорректностью в scp-программе} понимается \textit{некорректная структура}, описывающая некорректность (не обязательно делающую невозможным выполнение соответствующих данной \textit{scp-программе scp-процессов}), выявленную в рамках какой-либо конкретной \textit{scp-программы}.}

\scnheader{ошибка в scp-программе}
\scnsubdividing{синтаксическая ошибка в scp-программе;семантическая ошибка в scp-программе}
\scnsubdividing{ошибка в scp-программе на уровне программы;ошибка в scp-программе на уровне множества параметров;ошибка в scp-программе на уровне множества операторов;ошибка в scp-программе на уровне оператора;ошибка в scp-программе на уровне операнда}
\scnexplanation{Под \textit{ошибкой в scp-программе} понимается такая \textit{некорректность в \mbox{scp-программе}}, которая делает невозможным успешное выполнение любого \textit{scp-процесса}, соответствующего данной \textit{scp-программе}, или даже создание такого \textit{scp-процесса}.}

\scnheader{синтаксическая ошибка в scp-программе}
\scnexplanation{Под \textit{синтаксической ошибкой в \mbox{scp-программе}} понимается \textit{ошибка в \mbox{scp-программе}}, в состав которой входит некоторая конструкция, не соответствующая синтаксису \textit{scp-программы} или какой-либо ее части, например, конкретного \textit{scp-оператора}.}

\scnheader{семантическая ошибка в scp-программе}
\scnexplanation{Под \textit{семантической ошибкой в \mbox{scp-программе}} понимается \textit{ошибка в \mbox{scp-программе}}, в состав которой входит некоторая конструкция, корректная с точки зрения синтаксиса, но некорректная с семантической точки зрения, например, нарушающая логическую целостность \textit{scp-программы}.}

\scnheader{ошибка в scp-программе на уровне программы}
\scnexplanation{Каждая \textit{ошибка в scp-программе на уровне программы} описывает некорректный фрагмент, выявление которого требует анализа всей \textit{scp-программы} как единого целого, и не может быть выполнено путем анализа ее отдельных частей, например, конкретных \textit{scp-операторов}.}
\scnrelfromlist{включение}{отсутствует scp-процесс, соответствующий данной scp-программе\\
\scnaddlevel{1}
\scniselement{синтаксическая ошибка в scp-программе}
\scnaddlevel{-1}
;не указана декомпозиция scp-процесса, соответствующего данной scp-программе\\
\scnaddlevel{1}
\scniselement{синтаксическая ошибка в scp-программе}
\scnaddlevel{-1}
}

\scnheader{ошибка в scp-программе на уровне множества параметров}
\scnexplanation{Каждая \textit{ошибка в scp-программе на уровне множества параметров} описывает некорректный фрагмент, для выявления которого достаточно анализа параметров некоторой \textit{scp-программы}, т. е. явным образом выделенных аргументов \textit{действия (scp-процессе)}, соответствующего данной scp-программе. К такого рода ошибкам относятся, например, неверное указание ролей этих аргументов в рамках данного действия.}
\scnrelfromlist{включение}{не указан тип параметра scp-программы\\
\scnaddlevel{1}
\scniselement{синтаксическая ошибка в scp-программе}
\scnaddlevel{-1}
;не указан порядковый номер параметра scp-программы\\
\scnaddlevel{1}
\scniselement{синтаксическая ошибка в scp-программе}
\scnaddlevel{-1}
}

\scnheader{ошибка в scp-программе на уровне множества операторов}
\scnexplanation{Каждая \textit{ошибка в scp-программе на уровне множества операторов} описывает некорректный фрагмент, для выявления которого достаточно анализа множества операторов некоторой \textit{scp-программы}, т. е. элементов декомпозиции \textit{действия (scp-процесса)}, соответствующего данной \textit{scp-программе}. К таким ошибкам относится, например, факт отсутствия \textit{начального оператора' scp-программы} или факт отсутствия в программе \textit{scp-оператора завершения выполнения программы}.}
\scnrelfromlist{включение}{декомпозиция scp-процесса не содержит ни одного элемента\\
\scnaddlevel{1}
\scniselement{синтаксическая ошибка в scp-программе}
\scnaddlevel{-1}
;отсутствует scp-оператор завершения выполнения программы\\
\scnaddlevel{1}
\scniselement{синтаксическая ошибка в scp-программе}
\scnaddlevel{-1}
;scp-оператор, к которому осуществляется переход, не является частью текущего scp-процесса\\
\scnaddlevel{1}
\scniselement{синтаксическая ошибка в scp-программе}
\scnaddlevel{-1}
;не указана последовательность действий после выполнения текущего scp-оператора\\
\scnaddlevel{1}
\scniselement{синтаксическая ошибка в scp-программе}
\scnaddlevel{-1}
;отсутствует начальный оператор scp-программы\\
\scnaddlevel{1}
\scniselement{синтаксическая ошибка в scp-программе}
\scnaddlevel{-1}
}

\scnheader{ошибка в scp-программе на уровне оператора}
\scnexplanation{Каждая \textit{ошибка в scp-программе на уровне оператора} описывает некорректный фрагмент, для выявления которого достаточно анализа одного конкретного \textit{scp-оператора}, при этом не важно, в состав какой \textit{scp-программы} он входит. К такого рода ошибкам относится, например, факт указания количества операндов \textit{scp-оператора}, не соответствующего спецификации соответствующего класса \textit{scp-операторов}.}
\scnrelfromlist{включение}{scp-оператор не принадлежит ни одному из атомарных классов scp-операторов\\
\scnaddlevel{1}
\scniselement{синтаксическая ошибка в scp-программе}
\scnaddlevel{-1}
;ни один операнд scp-оператора удаления не помечен как удаляемый sc-элемент\\
\scnaddlevel{1}
\scniselement{синтаксическая ошибка в scp-программе}
\scnaddlevel{-1}
;в scp-операторе поиска пятиэлементной конструкции совпадает второй и четвертый операнд\\
\scnaddlevel{1}
\scniselement{синтаксическая ошибка в scp-программе}
\scnaddlevel{-1}
;scp-оператор поиска не содержит ни одного операнда с заданным значением\\
\scnaddlevel{1}
\scniselement{синтаксическая ошибка в scp-программе}
\scnaddlevel{-1}
;scp-оператор поиска с формированием множеств не содержит ни одного операнда с атрибутом формируемое множество\\
\scnaddlevel{1}
\scniselement{синтаксическая ошибка в scp-программе}
\scnaddlevel{-1}
;атрибутом формируемое множество отмечен операнд, которому соответствует операнд с заданным значением\\
\scnaddlevel{1}
\scniselement{синтаксическая ошибка в scp-программе}
\scnaddlevel{-1}
;количество операндов scp-оператора не совпадает со спецификацией\\
\scnaddlevel{1}
\scniselement{синтаксическая ошибка в scp-программе}
\scnaddlevel{-1}
}

\scnheader{ошибка в scp-программе на уровне оператора}
\scnexplanation{Каждая \textit{ошибка в scp-программе на уровне операнда} описывает некорректный фрагмент, для выявления которого достаточно анализа одного конкретного операнда в рамках scp-программы, точнее sc-дуги принадлежности, связывающей указанный операнд и соответствующий \textit{scp-оператор}, при этом не важно, какой именно \textit{scp-оператор}. К такого рода ошибкам относится, например, факт отсутствия ролевого отношения, указывающего на номер операнда в рамках \textit{scp-оператора}.}
\scnrelfrom{включение}{не указан номер операнда в рамках scp-оператора\\
\scniselement{синтаксическая ошибка в scp-программе}
}

\scnheader{некорректность в scp-программе*}
\scniselement{бинарное отношение}
\scnrelfrom{первый домен}{некорректность в scp-программе}
\scnrelfrom{второй домен}{scp-программа}

\scnheader{scp-программа поиска некорректности в scp-программе*}
\scnrelfrom{первый домен}{некорректность в scp-программе}
\scnrelfrom{второй домен}{scp-программа}
\scnexplanation{Отношение \textit{scp-программа поиска некорректности в scp-программе*} связывает класс \textit{некорректностей в scp-программе} и \textit{scp-программу}, которая может использоваться для выявления соответствующей некорректности в какой-либо другой \textit{scp-программе}.

Указанная \textit{scp-программа} должна иметь единственный параметр, который является \textit{in-параметром’} и, в зависимости от соответствующего класса некорректностей в \textit{scp-программе}, обозначает:
\begin{scnitemize}
\item саму \textit{scp-программу} в случае выявления \textit{некорректности в \mbox{scp-программе}} вообще или \textit{ошибки в scp-программе на уровне программы};

\item \textit{scp-процесс}, являющийся \textit{ключевым sc-элементом} данной \textit{\mbox{scp-программы}} в случае выявления ошибки в \textit{scp-программе на уровне множества параметров};

\item \textit{множество операторов} данной \textit{\mbox{scp-программы}} в случае выявления \textit{ошибки в scp-программе на уровне множества операторов};

\item \textit{знак конкретного scp-оператора} в случае выявления ошибки в \textit{\mbox{scp-программе} на уровне оператора};

\item \textit{sc-дугу принадлежности} в случае выявления \textit{ошибки в scp-программе на уровне операнда}.

\end{scnitemize}

Если в результате верификации \textit{scp-программы} выявлена некорректность, то формируется соответствующая \textit{структура} и генерируется связка отношения \textit{некорректность в scp-программе*}.}


\scnheader{Решатель задач средств поддержки проектирования программ Базового языка программирования ostis-систем}
\scnreltoset{декомпозиция sc-агента}{Абстрактный sc-агент верификации scp-программ;Абстрактный sc-агент отладки scp-программ}

\scnheader{Абстрактный sc-агент верификации scp-программ}
\scnexplanation{Алгоритм работы \textit{Абстрактного sc-агента верификации scp-программ} сводится к поиску некорректностей в рамках \textit{scp-программы} на основе определений, соответствующих различным классам таких некорректностей, а также посредством запуска соответствующих данным классам некорректностей \textit{scp-программ поиска некорректности в scp-программе*}.

Результатом работы \textit{Абстрактного sc-агента верификации scp-программ} является формирование в \textit{sc-памяти структур}, описывающих некорректности в исследуемой \textit{scp-программе}, если таковые имеются.

Единственным аргументом класса действий, соответствующего \textit{Абстрактному sc-агенту верификации scp-программ}, является знак верифицируемой \textit{scp-программы}.}

\scnheader{Абстрактный sc-агент отладки scp-программ}
\scnreltoset{декомпозиция sc-агента}{Абстрактный sc-агент запуска заданной scp-программы для заданного множества входных данных;Абстрактный sc-агент запуска заданной scp-программы для заданного множества входных данных в режиме пошагового выполнения;Абстрактный sc-агент поиска всех scp-операторов в рамках scp-программы;Абстрактный sc-агент поиска всех точек останова в рамках scp-процесса;Абстрактный sc-агент добавления точки останова в scp-программу;Абстрактный sc-агент удаления точки останова из scp-программы;Абстрактный sc-агент добавления точки останова в scp-процесс;Абстрактный sc-агент удаления точки останова из scp-процесса;Абстрактный sc-агент продолжения выполнения scp-процесса на один шаг;Абстрактный sc-агент продолжения выполнения scp-процесса до точки останова или завершения;Абстрактный sc-агент просмотра информации об scp-процессе;Абстрактный sc-агент просмотра информации об scp-операторе}

\scnheader{~}
\scnfilelong{Классы действий, соответствующие \textit{Абстрактному sc-агенту запуска заданной scp-программы для заданного множества входных данных} и \textit{Абстрактному sc-агенту запуска заданной scp-программы для заданного множества входных данных в режиме пошагового выполнения}, имеют два аргумента. Первый аргумент является знаком запускаемой scp-программы, второй -- знаком связки, в которую под соответствующими атрибутами входят sc-элементы, которые станут аргументами соответствующего scp-процесса.

В режиме пошагового выполнения предполагается, что на каждом шаге инициируется выполнение всех scp-операторов в рамках заданного \mbox{scp-процесса}, для которых предыдущий scp-оператор стал прошлой сущностью (выполнился). В свою очередь, шаг заканчивается, когда все инициированные таким образом операторы закончат выполнение. Таким образом, в случае, если в рамках scp-программы есть параллельные ветви, то на одном шаге могут одновременно инициироваться два и более scp-оператора.}
\scnreltolist{пояснение}{Абстрактный sc-агент запуска заданной scp-программы для заданного множества входных данных;Абстрактный sc-агент запуска заданной scp-программы для заданного множества входных данных в режиме пошагового выполнения}

\scnheader{~}
\scnfilelong{Классы действий, соответствующие \textit{Абстрактному sc-агенту добавления точки останова в scp-программу, Абстрактному sc-агенту удаления точки останова из scp-программы, Абстрактному sc-агенту добавления точки останова в scp-процесс} и \textit{Абстрактному sc-агенту удаления точки останова из scp-процесса}, имеют два аргумента. Первый аргумент является знаком \mbox{scp-программы} или scp-процесса соответственно, второй -- знаком \mbox{scp-оператора}, входящего в состав этой scp-программы или scp-процесса.}
\scnreltolist{пояснение}{Абстрактный sc-агент добавления точки останова в scp-программу;Абстрактный sc-агент удаления точки останова из scp-программы;Абстрактный sc-агент добавления точки останова в scp-процесс}

\scnheader{~}
\scnfilelong{Единственным аргументом классов действий, соответствующих \textit{Абстрактному sc-агенту поиска всех точек останова в рамках scp-процесса, Абстрактному sc-агенту продолжения выполнения scp-процесса на один шаг, Абстрактному sc-агенту продолжения выполнения scp-процесса до точки останова или завершения} и \textit{Абстрактному sc-агенту просмотра информации об scp-процессе}, является знак scp-процесса, с которым будет выполнено соответствующее действие.}
\scnreltolist{пояснение}{Абстрактный sc-агент поиска всех точек останова в рамках scp-процесса;Абстрактный sc-агент продолжения выполнения scp-процесса на один шаг; Абстрактный sc-агент продолжения выполнения scp-процесса до точки останова или завершения}

\scnheader{Абстрактный sc-агент поиска всех scp-операторов в рамках scp-программы}
\scnexplanation{Единственным аргументом класса действий, соответствующего \textit{Абстрактному sc-агенту поиска всех scp-операторов в рамках scp-программы}, является знак этой scp-программы.}

\scnheader{Абстрактный sc-агент просмотра информации об scp-операторе}
\scnexplanation{Единственным аргументом класса действий, соответствующего \textit{Абстрактному sc-агенту просмотра информации об scp-операторе}, является знак scp-оператора, входящего в состав некоторого scp-процесса. Результатом работы данного агента является структура, описывающая значения операндов данного scp-оператора, его атомарный тип и другую служебную информацию, полезную для разработчика.}

\scnendstruct

\end{SCn}