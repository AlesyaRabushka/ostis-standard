\begin{SCn}

\scnsectionheader{Предметная область и онтология семантических ассоциативных компьютеров для ostis-систем}

\scnstartsubstruct

\scnheader{Предметная область и онтология семантических ассоциативных компьютеров для ostis-систем}
\scnsdmainclasssingle{***}
\scnsdclass{***}
\scnsdrelation{***}

\scnheader{семантический ассоциативный компьютер}
\scnidtf{аппаратно реализованный интерпретатор семантических моделей (sc-моделей) компьютерных систем}
\scnidtf{семантический ассоциативный компьютер, управляемый знаниями}
\scnidtf{компьютер с нелинейной структурно перестраиваемой (графодинамической) ассоциативной памятью, переработка информации в которой сводится не к изменению состояния элементов памяти, а к изменению конфигурации связей между ними}
\scnidtf{sc-компьютер}
\scnidtf{scp-компьютер}
\scnidtf{компьютер, управляемый знаниями, представленными в SC-коде}
\scnidtf{компьютер, ориентированный на обработку текстов SC-кода}

\filemodetrue
\scnrelfromlist{принцип}{
нелинейная память — каждый элементарный фрагмент хранимого в памяти текста может быть инцидентен неограниченному числу других элементарных фрагментов этого текста;
структурно перестраиваемая (реконфигурируемая) память — процесс отработки хранимой в памяти информации сводится не только к изменению состояния элементов, но и к реконфигурации связей между ними;
в качестве внутреннего способа кодирования знаний, хранимых в памяти семантического ассоциативного компьютера, используется универсальный (!) способ нелинейного (графоподобного) смыслового представления знаний, названный нами SC-кодом (семантическим, смысловым компьютерным кодом);
обработка информации осуществляется коллективом агентов, работающих над общей памятью. Каждый из них реагирует на соответствующую ему ситуацию или событие в памяти (компьютер, управляемый хранимыми знаниями);
есть программно реализуемые агенты, поведение которых описывается хранимыми в памяти агентно-ориентированными программами, которые интерпретируются соответствующими коллективами агентов;
есть базовые агенты, которые не могут быть реализованы программно (в частности, это агенты интерпретации агентных программ, базовые рецепторные агенты-датчики, базовые эффекторные агенты);
все агенты работают над общей памятью одновременно. Более того, если для какого-либо агента в некоторый момент времени в различных частях памяти возникает сразу несколько условий его применения, разные акты указанного агента в разных частях памяти могут выполняться одновременно (акт агента — это неделимый, целостный процесс деятельности агента);
для того, чтобы акты агентов, параллельно выполняемые в общей памяти не "мешали"\ друг другу, для каждого акта в памяти фиксируется и постоянно актуализируется его текущее состояние. То есть каждый акт сообщает всем остальным о своих намерениях и пожеланиях, которым остальные агенты не должны препятствовать (например, это различного рода блокировки используемых элементов семантической памяти);
кроме того, агенты (точнее, выполняемые ими акты) должны соблюдать "этику"\,, стараясь не в ущерб себе создавать максимально благоприятные условия для других агентов (актов), например, не жадничать, быстрее возвращать, не захватывать (не блокировать) лишние элементы памяти, как можно скорее освобождать (деблокировать) заблокированные элементы памяти;
процессор и память семантического ассоциативного компьютера глубоко интегрированы и составляют единую процессоро-память. Процессор семантического ассоциативного компьютера равномерно "распределен"\ по его памяти так, что процессорные элементы одновременно являются и элементами памяти компьютера. Обработка информации в семантическом ассоциативном компьютере сводится к реконфигурации каналов связи между процессорными элементами,  следовательно память такого компьютера есть не что иное, как \uline{коммутатор} (!) указанных каналов связи. Таким образом, текущее состояние конфигурации этих каналов связи и есть текущее состояние обрабатываемой информации}
\filemodefalse

\scnendstruct

\end{SCn}