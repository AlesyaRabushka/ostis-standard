\begin{SCn}
	
\scnsectionheader{\currentname}
\scnstartsubstruct

\scnheader{программный вариант реализации платформы интерпретации sc-моделей компьютерных систем}
\scnidtf{программный вариант реализации базового интерпретатора логико-семантических моделей компьютерных систем}
\scnidtf{вариант реализации базового интерпретатора логико-семантических моделей компьютерных систем на традиционных компьютерах с архитектурой фон Неймана}
\scnsuperset{web-ориентированный вариант реализации платформы интерпретации sc-моделей компьютерных систем}
\scnaddlevel{1}
	\scnidtf{вариант реализации платформы интерпретации sc-моделей компьютерных систем предполагающий взаимодействие пользователей с системой посредством сети Интернет}
	\scnsubset{многопользовательский вариант реализации платформы интерпретации sc-моделей компьютерных систем}
	\scnhaselement{}
\scnaddlevel{-1}

\scnheader{Программный вариант реализации платформы интерпретации sc-моделей компьютерных систем}
\scnrelfromset{декомпозиция программной системы}{Реализация sc-памяти;Реализация интерпретатора sc-моделей пользовательских интерфейсов;Реализация scp-интерпретатора}

\scnheader{Реализация sc-памяти}
\scnidtf{sc-machine}
\scntext{основной репозиторий исходных текстов}{https://github.com/ostis-dev/sc-machine.git}
\scnrelfromlist{компонент программной системы}{Реализация sc-хранилища и механизма доступа к нему;Реализация базового набора платформенно-зависимых sc-агентов;Реализация подсистемы взаимодействия с внешней средой с использованием протокола SCTP;Реализация подсистемы взаимодействия с внешней средой с использованием протоколов на основе формата JSON;Реализация вспомогательных инструментальных средств в рамках реализации sc-памяти}

\scnheader{Реализация вспомогательных инструментальных средств в рамках реализации sc-памяти}
\scnrelfrom{компонент программной системы}{Реализация сборщика базы знаний из исходных текстов, записанных в SCs-коде}

\scnheader{Реализация sc-хранилища и механизма доступа к нему}
\scnrelfrom{компонент программной системы}{Реализация файловой памяти ostis-системы}
\scntext{принципы реализации}{В рамках данной программной реализации \textit{sc-хранилища} \scnbigspace \textit{sc-память} моделируется в виде набора \textit{сегментов} фиксированного размера ($2^{16}-1=65535$ \textit{sc-элементов}). Максимально возможный набор сегментов ограничивается настройками программной реализации sc-хранилища (в настоящее время по умолчанию установлено количество $2^{16}-1=65535$ сегментов). Таким образом, технически максимальное количество sc-элементов в текущей реализации составляет около $4.3 \times 10^{9}$ sc-элементов.

По умолчанию все сегменты физически располагаются в оперативной памяти, если объема памяти не хватает, то предусмотрен механизм выгрузки части сегментов на жесткий диск.

Каждый сегмент состоит из набора структур данных, описывающих \textit{sc-элементы} (далее будем говорить, что сегмент состоит из sc-элементов). Независимо от типа sc-элемент имеет фиксированный размер (в текущий момент -- 56 байт), что обеспечивает удобство их хранения. Таким образом, максимальный размер базы знаний в текущий момент в физическом выражении может достигнуть 223 Гб (без учета содержимого \textit{внутренних файлов ostis-системы}).

Каждый sc-элемент в текущей реализации может быть однозначно задан его адресом (sc-адресом), состоящим из номера сегмента и номера sc-элемента в рамках сегмента.

Каждый sc-элемент описывается его синтаксическим типом (меткой), а также независимо от типа указывается sc-адрес первой входящей sc-дуги и первой выходящей sc-дуги (могут быть пустыми, если таких sc-дуг нет). Далее в зависимости от типа указывается либо содержимое (для внутреннего файла, может быть пустым, если sc-узел не является знаком файла), либо спецификация sc-дуги.

В текущей реализации набор синтаксических типов sc-элементов  включает в себя следующие признаки:
\begin{scnitemize}
	\item принадлежность одному из базовых классов sc-элементов, т.е. классов уровня Алфавита SC-кода (sc-узел, sc-дуга общего вида, sc-дуга принадлежности, sc-ребро, sc-ссылка (внутренний файл ostis-системы));
	\item признак константности/переменности (для всех sc-элементов);
	\item признак позитивности/негативности/нечеткости (для sc-дуг принадлежности);
	\item признак стационарности/нестационарности (для sc-дуг принадлежности);	
	\item принадлежность одному из следующих классов сущностей -- \textit{sc-структура}, \textit{sc-связка}, \textit{ролевое отношение}, \textit{неролевое отношение}, \textit{sc-класс}, \textit{абстрактная терминальная сущность}, \textit{материальная сущность}.
\end{scnitemize}

Спецификация sc-дуги включает указание sc-адресов начального и конечного sc-элементов, инцидентных данной sc-дуге, а также sc-адреса следующей и предыдущей sc-дуг, входящих в конечный для данной sc-дуги sc-элемент, и sc-адреса следующей и предыдущей sc-дуг, выходящих из начального для данной sc-дуги sc-элемента. 

Таким образом каждый sc-элемент (в том числе, sc-дуга) не хранит список связанных с ним sc-элементов, а хранит адреса одной выходящей и одной входящей дуги, каждая из которых в свою очередь хранит адреса следующей и предыдущей дуг в списке исходящих и входящих для соответствующих элементов. Это позволяет сделать размер всех sc-элементов фиксированным.
	
Каждый sc-узел в текущей реализации может иметь содержимое (может стать \textit{внутренним файлом ostis-системы}). Под содержимое отводится 48 байт (объем, эквивалентный спецификации sc-дуги). Если объем содержимого не превышает это значение, то содержимое хранится прямо в sc-элементе в виде набора байт. В противном случае оно помещается в специальным образом организованную файловую память (за ее организацию отвечает отдельный модуль платформы, который в общем случае может быть устроен по-разному), а в содержимое sc-узла записывается уникальный идентификатор, позволяющий быстро найти нужное содержимое в файловой памяти.
}
\scnaddlevel{1}
	\scnnote{sc-ребра в текущий момент хранятся так же, как sc-дуги, то есть имеют начальный и конечный sc-элементы, отличие только в типе (sc-edge вместо sc-arc). Это приводит к ряду неудобств при обработке, но sc-ребра используются пока достаточно редко.}
\scnaddlevel{-1}

\scnheader{Реализация сборщика базы знаний из исходных текстов, записанных в SCs-коде}
\scnidtf{sc-builder}

\scnendstruct

\end{SCn}