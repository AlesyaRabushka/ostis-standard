\begin{SCn}

\scnsectionheader{Семантическая модель Метасистемы IMS.ostis}

\scnstartsubstruct

\scnheader{Метасистема IMS.ostis}
\scntext{назначение}{Эффективность любой технологии, в том числе и \textit{\textbf{Технологии OSTIS}}~\cite{IMS} определяется не только сроками создания искусственных систем соответствующего класса, но и темпами совершенствования самой технологии (темпами совершенствования средств автоматизации и темпами совершенствования системы стандартов, лежащих в основе технологии).

Для фиксации текущего состояния \textit{Технологии OSTIS}, а также для организации ее эффективного использования и ее перманентного совершенствования с участием ученых, работающих в области искусственного интеллекта, и инженеров, разрабатывающих семантические компьютерные системы различного назначения, в состав \textit{Экосистемы OSTIS} вводится \textit{Метасистема IMS.ostis}~\cite{IMS}, назначение которой делает ее \underline{ключевой} \textit{ostis-системой} в рамках \textit{Экосистемы OSTIS}.}

\scnheader{Метасистема IMS.ostis}
\scnidtf{Интеллектуальная метасистема комплексной информационной и инструментальной поддержки проектирования совместимых семантических компьютерных систем, которая является формой реализации общей теории и технологии проектирования семантических компьютерных систем и которая поддерживает высокий темп эволюции указанной теории и технологии}
\scnidtf{Intelligent MetaSystem for intelligent systems design}
\scnidtf{IMS.ostis}
\scnidtf{Фреймворк интеллектуальных систем}
\scnidtf{Интеллектуальная метасистема комплексной поддержки проектирования совместимых семантических компьютерных систем по Технологии OSTIS}
\scnidtf{Фреймворк ostis-систем}
\scnidtf{Фреймворк IMS.ostis}

\scnheader{Метасистема IMS.ostis}
\scntext{назначение}{\textit{Метасистема IMS.ostis} является в \textit{Экосистеме OSTIS} ключевой интеллектуальной системой, которая поддерживает не только проектирование новых интеллектуальных систем и не только замену устаревших компонентов в интеллектуальных системах, входящих в состав \textit{Экосистемы OSTIS}, но и включение (интеграция) в состав \textit{Экосистемы OSTIS} новых создаваемых интеллектуальных систем.

\textit{Метасистема IMS.ostis} ориентирована на разработку и практическое внедрение методов и средств \textbf{компонентного проектирования} семантически совместимых интеллектуальных систем, которая предоставляет возможность быстрого создания интеллектуальных приложений различного назначения. Подчеркнем при этом, что сферы практического применения методики компонентного проектирования семантически совместимых интеллектуальных систем ничем не ограничены.}

\scnheader{Метасистема IMS.ostis}
\scnidtf{реализация технологии проектирования семантически совместимых компьютерных систем в виде метасистемы, построенной по той же технологии и обеспечивающей комплексную информационную и инструментальную поддержку проектирования семантически совместимых компьютерных систем}
\scntext{декомпозиция}{
\begin{scnitemize}
    \item полное описание самой Технологии OSTIS;
    \item история эволюции Технологии OSTIS;
    \item описание правил использования Технологии OSTIS;
    \item описание организационной инфраструктуры, направленной на развитие Технологии OSTIS;
    \item библиотека многократно используемых и семантически совместимых компонентов ostis-систем;
    \item методы и инструментальные средства проектирования различного вида компонентов ostis-систем;
    \item технические средства координации деятельности участников проекта, направленные на постоянное совершенствование Технологии OSTIS.
\end{scnitemize}
}

\scnheader{Проект IMS.ostis}
\scntext{подзадачи}{
\begin{scnitemize}
    \item Разработать \textit{Метасистему IMS.ostis}, обеспечивающую быстрое компонентное проектирование семантически совместимых компьютерных систем различного назначения.
    \item Разработать методы и средства, обеспечивающие интенсивное развитие рынка семантически совместимых прикладных интеллектуальных систем, созданных на основе \textit{Метасистемы IMS.ostis}.
    \item Разработать методы и средства, обеспечивающие стимулирование интенсивного развития самой \textit{Метасистемы IMS.ostis}.
\end{scnitemize}
}

\scnheader{Метасистема IMS.ostis}
\scntext{новизна}{Новизна \textit{Метасистемы IMS.ostis} заключается в унификации представления различного вида информации в памяти компьютерных систем на основе смыслового (семантического) представления этой информации, что обеспечивает:
\begin{scnitemize}
    \itemустранение дублирования одной и той же информации в разных интеллектуальных системах и в разных компонентах одной и той же системы;
    \item семантическую совместимость различных компонентов интеллектуальных систем и различных интеллектуальных систем в целом;
    \item существенное расширение библиотек совместимых многократно используемых компонентов компьютерных систем за счет "крупных"\ компонентов и, в частности, типовых подсистем.
\end{scnitemize}
}

\scnheader{Метасистема IMS.ostis}
\scntext{принципы реализации}{Принципы технической реализации \textit{Метасистемы IMS.ostis} полностью совпадают с принципами технической реализации прикладных интеллектуальных систем, разрабатываемых с помощью этой метасистемы.}
\scnidtf{интеллектуальная система, предназначенная для комплексной информационной и инструментальной поддержки проектирования семантически совместимых компьютерных систем, на назначение которых не накладывается никаких ограничений}

\scnheader{База знаний Метасистемы IMS.ostis}
\scntext{декомпозиция}{
\begin{scnitemize}
    \item текущее состояние моделей и методов, используемых при разработке интеллектуальных систем с помощью \textit{Метасистемы IMS.ostis};
    \item систематизированную библиотеку многократно используемых и совместимых компонентов интеллектуальных систем;
    \item описание инструментальных средств проектирования различного вида компонентов интеллектуальных систем (фрагментов баз знаний, решателей задач, пользовательских интерфейсов);
    \item описание средств координации коллективной деятельности, направленной на постоянное развитие \textit{Метасистемы IMS.ostis};
    \item описание истории эволюции \textit{Метасистемы IMS.ostis};
    \item описание средств проектирования различных классов интеллектуальных систем.
\end{scnitemize}
}

\scnheader{Проект IMS.ostis}
\scntext{принципы организации}{Организация \textit{Проекта IMS.ostis} реализуется в форме взаимодействия \textit{Метасистемы IMS.ostis} с его пользователями и основана на следующих принципах:
\begin{scnitemize}
    \item Решатель задач и пользовательский интерфейс \textit{Метасистемы IMS.ostis} обеспечивают поддержку всего комплекса проектных задач, решаемых разработчиками прикладных интеллектуальных систем, а также разработчиками самой \textit{Метасистемы IMS.ostis}.
    \item Для стимулирования развития рынка совместимых прикладных интеллектуальных систем, разработанных с помощью \textit{Метасистемы IMS.ostis} и развития самой этой метасистемы используются технические средства анализа и оценки объекта и значимости персонального вклада каждого разработчика в специальных условных единицах.
    \item Для стимулирования развития рынка совместимых прикладных интеллектуальных систем, разработанных с помощью \textit{Метасистемы IMS.ostis}, за каждую такую интеллектуальную систему, зарегистрированную и специфицированную в рамках \textit{Метасистемы IMS.ostis}, разработчикам выделяется вознаграждение в используемых условных единицах после того, как эта прикладная система будет протестирована на предмет семантической совместимости с другими системами, разработанными с помощью \textit{Метасистемы IMS.ostis}. При этом \textit{Метасистемы IMS.ostis} становится площадкой для рекламы и распространения интеллектуальных систем, разработанных с его помощью.
    \item Стимулирование развития самой \textit{Метасистемы IMS.ostis} осуществляется следующим образом. Участие в развитии \textit{Метасистемы IMS.ostis} носит открытый характер, для чего достаточно соответствующим образом зарегистрироваться. Авторские права каждого разработчика \textit{Метасистемы IMS.ostis} защищаются и каждый его вклад в зависимости от его ценности автоматически измеряется и фиксируется в используемых условных единицах.
    \item Участие в развитии \textit{Метасистемы IMS.ostis} может иметь самые различные формы (в простейшем случае, это может быть указание на конкретные ошибки, на конкретные трудности, с которыми пользователь столкнулся, формулировка конкретных пожеланий; более сложным вкладом является добавление в базу знаний метасистемы новых знаний, новых компонентов в библиотеку многократно используемых компонентов). При этом автор нового многократно используемого компонента, включенного в библиотеку \textit{Метасистемы IMS.ostis}, может выбрать любую лицензию для его распространения и, в том числе, назначить ему любую цену.
    \item Использование \textit{Метасистемы IMS.ostis} зарегистрированными пользователями  для ознакомления с ним носит бесплатный открытый характер. При коммерческой разработке прикладных интеллектуальных систем стоимость каждого обращения к библиотекам \textit{Метасистемы IMS.ostis} вполне доступна, но существенно снижается в зависимости от степени активности пользователя в развитии \textit{Метасистемы IMS.ostis}. Это еще один механизм стимулирования участия в развитии \textit{Метасистемы IMS.ostis}.
\end{scnitemize}

Таким образом, указанные принципы организации \textit{Метасистемы IMS.ostis} обеспечивают на постоянной основе привлечение к разработке \textit{Метасистемы IMS.ostis} и к формированию рынка семантически совместимых прикладных интеллектуальных систем неограниченные научные, технические и финансовые ресурсы и, в частности, привлечение любых специалистов, желающих участвовать в этом открытом проекте.
}

\scnendstruct

\end{SCn}