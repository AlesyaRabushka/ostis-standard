\begin{SCn}
	\scnsectionheader{\currentname}
	
	\scnstartsubstruct
	
	\scnheader{Предметная область и онтология семантически совместимых интеллектуальный корпоративных ostis-систем различного назначения}
	\scniselement{предметная область}
	\scnsdmainclasssingle{интеллектуальная комната данных}
	
	
	\scnheader{интеллектуальная комната данных}
	\scnidtf{Intelligent Data Room}
	\scnidtf{IDR}
	\scnidtf{система, которая позволяет отслеживать, анализировать и постепенно автоматизировать все процессы обработки данных в компании}
	
	\scnrelfromset{как это работает}
	{\scnfileitem{Интеллектуальные подсистемы (агенты) упорядочивают структуру ваших данных таким образом, что актуальная информация всегда доступна, а устаревшая информация автоматически архивируется или удаляется в соответствии с законами о хранении и защите данных;};
	\scnfileitem{Запросы к системе выполняются в виде простых инструкций, система помогает менеджерам вводить необходимую информацию, осуществляет частичную или полную автоматизацию обновления информации из баз данных, доступных через Интернет;};
	\scnfileitem{Искусственный интеллект (ИИ) выполняет структуризацию и классификацию документов и информации для принятия быстрых и правильных решений, автоматически обрабатывает документы и доступные базы данных для отбора ключевой информации, необходимой в данный момент и в будущем;};
	\scnfileitem{Существующее системное окружение на предприятии может быть легко подключено к ИИ через открытые интерфейсы, вся информация остается доступной. Все ключевые системы данных синхронизируются с ИИ, данные постоянно сравниваются друг с другом, чтобы избежать потерь.};
	\scnfileitem{Вся информация доступна в базе знаний, которая является источником данных для рабочих процессов, отчетности и комплексных проверок.};
	\scnfileitem{Таким образом, предлагаемая платформа на основе ИИ позволяет представить всю компанию единым целостным образом.}
	}

	\scnrelfromset{достоинства внедрения IDR}
	{\scnfileitem{IDR помогает собирать и оценивать информацию без преднамеренных искажений или ошибок, связанных с человеческим фактором;};
	\scnfileitem{Компания с IDR полностью контролирует свои данные;};
	\scnfileitem{Система предоставляет только высококачественные, достоверные и актуальные данные.};
	\scnfileitem{Цифровое представление всех процессов компании обеспечивает интегрированную обработку информации внутри компании, что дает полную прозрачность управления, облегчает доступ ко всей информации и ее анализ.};
	\scnfileitem{Благодаря поддержке подсистем ИИ все необходимые данные из документов, процессов и внешних источников могут быть извлечены, структурированы и грамотно оценены.};
	\scnfileitem{ИИ предоставляет инструмент для интеллектуальной оцифровки и интеграции знаний вашей компании и взаимодействия между всеми заинтересованными сторонами в рамках вашей компании, как следствие - обеспечивает автоматическую поддержку соответствующих бизнес-процессов и устраняет локальные изолированные решения внутри компании, превращая ее в единую согласованную систему.}
}

\scnaddhind{1}
\scnrelfrom{примечание}{
	Идея \textit{интеллектуальной комнаты данных} в общем случае может реализовываться двумя путями. Любой из перечисленных вариантов может реализовываться постепенно, с подключением к IDR все новых и новых информационных ресурсов.}
\scnaddhind{-1}


\scnsuperset{Вариант Цифровой сотрудник}
\scnaddlevel{1}
	\scnexplanation
	{Надстройка над уже существующими информационными ресурсами предприятия (различные базы данных, облачные и физические хранилища документов и т.д.). Для реализации этого варианта необходимо в базе знаний IDR в виде семейства онтологий описать метаинформацию об имеющихся информационных ресурсах (схемы баз данных, структуру и расположение документов и т.д.), а также механизмы доступа к этим ресурсам (например, их физическое расположение, языки запросов). На основе такого описания IDR сможет автоматически построить необходимый набор запросов к нужным информационным ресурсам, интегрировать полученные ответы и выдать ответ пользователю в удобной ему форме. При этом пользователю системы не нужно знать, где именно и в какой форме хранится нужная ему информация, запрос к системе делается на языке, близком к естественному.}
	
	\scnrelfromset{достоинства варианта}
	{\scnfileitem{Нет необходимости в рамках базы знаний IDR дублировать информацию, которая уже содержится в использовавшихся ранее информационных ресурсах, она может и дальше храниться в тех же местах и в той же форме. При этом данные могут быстро меняться, это никак не повлияет на работу системы};
	\scnfileitem{Нет необходимости вносить изменения в уже налаженные процессы и используемое на предприятии ПО, резко переобучать людей и менять отлаженные схемы работы};
	\scnfileitem{Такой вариант в общем случае проще и дешевле в реализации и позволяет экспериментальным путем выявить наиболее “больные” места предприятия, где автоматизация информационных процессов и обеспечение их прозрачности позволит сэкономить наибольшее количество средств и минимизировать число ошибок}
	}

	\scnrelfromset{недостатки варианта}
	{\scnfileitem{Такой вариант не решает проблемы, связанные с дублированием информации, представленной в разной форме в разных местах, и только частично решает проблемы, связанные с ошибками при внесении новой или редактированием имеющейся информации в разнородные информационные ресурсы};
	 \scnfileitem{Из-за отсутствия унификации представления информации система IDR ограничена в своих возможностях, в частности при верификации информации. Проверить корректность, непротиворечивость, полноту информации намного проще, если вся информация представлена в унифицированном виде (как по форме, так и по смыслу)}
	}
\scnaddlevel{-1}



\scnsuperset{Вариант полноценной комнаты данных}
\scnaddlevel{1}
\scnexplanation{
Данный вариант предполагает полный перенос в базу знаний IDR части или всей информации, хранящейся в электронном виде в информационных ресурсах предприятия. Для этого необходимо описывать в базе знаний как онтологии (системы понятий) предметной области предприятия, так и конкретные экземпляры и связи между ними. При этом очевидно, что если онтологии изменяются относительно редко и могут обновляться вручную, то конкретные экземпляры и их описание должно формироваться автоматически.}

	\scnrelfromset{достоинства варианта}
		{\scnfileitem{Полностью исключается необходимость дублирования информации в разных местах и в разных формах, таким образом снижается объем хранимой информации и минимизируется количество ошибок};
		\scnfileitem{В онтологиях описывается только смысл информации, нет необходимости описывать существующую структуру информационных ресурсов предприятия и ориентироваться на нее};
		\scnfileitem{Полностью задействуются возможности IDR по верификации информации предприятия на предмет корректности, непротиворечивости и полноты};
		\scnfileitem{Наращивание функциональных возможностей такой реализации значительно упрощается за счет гибкости подходов, лежащих в основе IDR, в частности, подхода к разработке и структуризации базы знаний и многоагентного подхода к обработке информации}
	}
	
	\scnrelfromset{недостатки варианта}
	{	\scnfileitem{Требуются изменения в уже налаженных процессах на предприятии, отказ от части используемого в настоящий момент ПО, переобучение сотрудников, обслуживающих и развивающих информационные системы предприятия};
		\scnfileitem{В общем случае такой вариант может оказаться более трудоемким при первоначальной реализации, поскольку предполагает больший объем работ по формализации информации, обеспечению надежности ее хранения и доступа к ней, эффективности ее редактирования и дополнения}
	}
\scnaddlevel{-1}

\scnaddhind{1}
\scnrelfrom{примечание}{
Оба представленных варианта дают возможность описать в базе знаний IDR не только информацию, которая уже активно используется на предприятии, но и дополнительные знания, которые могут оказаться полезными, например:}
\scnaddlevel{1}
	\scnaddhind{-1}
	\scneqtoset
	{	Различные стандарты и документы, регламентирующие работу предприятия;
		Описание структуры предприятия, правил работы, основных контактов;
		Учебные материалы и руководства для новых сотрудников;
		и многое другое
	}
\scnaddlevel{-1}

\end{SCn}