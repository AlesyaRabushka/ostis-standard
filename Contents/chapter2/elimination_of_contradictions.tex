\begin{SCn}

\bigskip
\scnheader{Семантическая модель средств обнаружения и анализа противоречий в базах знаний ostis-систем}

\scnrelfromset{рассматриваемые вопросы}{
	\scnfileitem{исследуемые объекты и отношения};
	\scnfileitem{какие существуют противоречия};
	\scnfileitem{Механизм устранения противоречий в базе знаний};
	\scnfileitem{Спецификации агентов поиска и устранения противоречий};
	\scnfileitem{Спецификация агентов для устранения конкретных противоречий}
}

\scnsdclass{противоречие;действие поиска противоречия;агент поиска противоречий;действие устранения противоречия;агент устранения противоречий;дублирование системных идентификаторов;действие устранение дублирования системного идентификатора;синонимия;омонимия}
\scnsdrelation{требующее внимания разработчика*}

\bigskip
\scnfragmentcaption

\scnheader{Противоречие в базе знаний}
\scnexplanation{ситуация, когда в базе знаний существуют семантически несовместимые конструкции !!(уточнить определение у ВВ)}
\scnnote{Как при добавлении новых фрагментов базы знаний, так и в процессе работы системы, в ней могут появляться противоречия}
\scnrelfrom{виды противоречий}{
\scnfilescg{figures/sd_agents/struct_contradictions.png}}}

\scnheader{Модель средств обнаружения и анализа противоречий}
\scnrelfromlist{включение}{поиск противоречий; поиск решения противоречий;применение решения;}

\bigskip
\scnfragmentcaption

\scnheader{Агент поиска противоречий}
\scntext{ключевые sc-элементы}{}
\scntext{условие инициирования}{Появление в sc-памяти инициированного действия, принадлежащего классу \textit{действие поиска противоречий}.}
\scnaddlevel{1}
\scnnote{Аргументами могут быть несколько структур.}
\scnaddlevel{1}

\scntext{входные параметры}{фрагмент базы знаний}
\scnaddlevel{1}
\scnrelfrom{пример}{
		\scgfileitem{figures/sd_agents/fragment_examle.png}
}
\scnaddlevel{-1}

\scntext{результат}{Множество пар для исходного фрагмента БЗ, где первый элемент это структура устраненного противоречия, а второй это множество элементов, которые должны быть удалены при выполнении устранения противоречия.}
\scnaddlevel{1}
\scnrelfrom{пример}{
	\scgfileitem{figures/sd_agents/find_contradiction_result.png}
}
\scnaddlevel{-1}

\scntext{описание поведения}{
	\begin{scnenumerate}
		\item \scgfileitem{figures/sd_agents/find_contradiction_step1.png}
			\scnexplanation{Генерация экземпляра каждого действия, принадлежащего \textit{классу действий поиска противоречий в базе знаний}}
		\item \scgfileitem{figures/sd_agents/find_contradiction_step2.png}
		    \scnexplanation{Добавления экземпляра действия в множество \textit{выполненных действий}} 
		\item \scgfileitem{figures/sd_agents/find_contradiction_step3.png}
		\scnexplanation{Для каждого найденного противоречия генерация экземпляра \textit{действия устранения противоречий}}
		\item \scgfileitem{figures/sd_agents/find_contradiction_step4.png}
		    \scnexplanation{Добавления экземпляра действия в множество \textit{выполненных действий}}
		\item Если действие принадлежит множеству \textit{безуспешно выполненное действие}, то добавить противоречие в множество \textit{требующее внимания разработчика}; Перейти к пункту 7
		\item Если действие принадлежит множеству \textit{успешно выполненное действие}, то добавить \textit{результат} действия в множество, являющееся \textit{результатом} для экземпляра \textit{действие поиска противоречий}
		\item удалить выполненные экземпляры \textit{действия поиска противоречий} и \textit{действия устранения противоречий}
		\item Если не осталось незавершенных действий, завершить работу. Иначе перейти к пункту 3.
	\end{scnenumerate}
}


\bigskip
\scnfragmentcaption

\scnheader{Агент поиска дублирования системного идентификатора}
\scntext{условие инициирования}{Появление в sc-памяти инициированного действия, принадлежащего классу \textit{действие поиска дублирования системного идентификатора}}
\scntext{входные параметры}{фрагменты базы знаний}
\scnaddlevel{1}
\scnrelfrom{пример}{
	\scgfileitem{figures/sd_agents/find_sys_id_duplication_fragment.png}
}
\scnaddlevel{-1}

\scntext{результат}{множество структур с дублированием системного идентификатора}
\scnaddlevel{1}
\scnrelfrom{пример}{
	\scgfileitem{figures/sd_agents/find_sys_id_duplication_result.png}
}
\scnaddlevel{-1}

\scntext{описание поведения}{
	\begin{scnenumerate}
		\item Поиск различных элементов с одинаковым системным идентификатором по всей базе знаний
		\item Проверка принадлежат ли найденные проблемные элементы входным фрагментам
		\scnaddlevel{1}
		\scnrelfrom{шаблон проверки}{
			\scgfileitem{figures/sd_agents/template_search.jpg}
		}
		\scnaddlevel{-1}
		\item Если да, то генерация структуры, принадлежащей классу \textit{ структура с дублированием системного идентификатора}.
	\end{scnenumerate}
}

\bigskip
\scnfragmentcaption

\scnheader{Агент устранения дублирования системного идентификатора}
\scntext{условие инициирования}{Появление в sc-памяти инициированного действия, принадлежащего классу \textit{действие устранения противоречия}}
\scntext{входные параметры}{структура с противоречием}
\scnaddlevel{1}
\scnrelfrom{пример}{
	\scgfileitem{figures/sd_agents/sys_id_elimination_fragment.png}
}
\scnaddlevel{-1}

\scntext{результат}{Пара, которая состоит из структуры решения противоречия и множества элементов, подлежащих удалению при применении решении противоречия.}
\scnaddlevel{1}
\scnrelfrom{пример}{
	\scgfileitem{figures/sd_agents/sys_id_elimination_result_pos.png}
	\scnexplanation{результат при успешном выполнении действия}
	
	\scgfileitem{figures/sd_agents/sys_id_elimination_result_neg.png}
	\scnexplanation{результат при неуспешном выполнении действия}
}
\scnaddlevel{-1}

\scntext{описание поведения}{
	\begin{scnenumerate}
		\item проверить что структура с противоречием принадлежит классу \textit{структура с дублированием системного идентификатора} Если нет, то перейти к пункту 7.
 		\item проверить что элементы с дублированием системного идентификатора совместимы, если нет, перейти к пункту 7.
		\item создание нового элемента, пересоздание всех связок старых элементов на новый, и их зависимостей.
		\item добавление всех зависимых элементов, подлежащие удалению в множество.
		\item оформить \textit{результат} выполнения агента. Результатом будет являться пара - структура, с новым элементом из пункта 3 и множество элементов, сформированное в пункте 4.
		\item добавить действие в множество \textit{успешно выполненное действие} и завершить работу.
		\item добавить действие в множество \textit{неуспешно выполненное действие} и завершить работу.
		
	\end{scnenumerate}
}

\bigskip
\scnfragmentcaption

\scnheader{Агент слияния структур}
\scntext{условие инициирования}{Появление в sc-памяти инициированного действия, принадлежащего классу \textit{действие слияния структур}}
\scntext{входные параметры}{множество структур для слияния. Наличие отношения \textit{rrel\_1} обязательно}
\scnaddlevel{1}
\scnrelfrom{пример}{
	\scgfileitem{figures/sd_agents/contour_merge_init.png}
}
\scnaddlevel{-1}

\scntext{результат}{единая структура, содержащая все элементы}
\scnaddlevel{1}
\scnrelfrom{пример}{
	\scgfileitem{figures/sd_agents/contour_merge_result.png}
}
\scnaddlevel{-1}

\scntext{описание поведения}{
	\begin{scnenumerate}
		\item перемещение всех элементов в единую структуру.
	\end{scnenumerate}
}

\end{SCn}