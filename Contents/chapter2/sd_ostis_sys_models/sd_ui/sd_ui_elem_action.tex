\begin{SCn}

\scnsectionheader{\currentname}

\scnstartsubstruct

\scnheader{Предметная область элементарных пользовательских действий}
\scniselement{предметная область}
\scnsdmainclasssingle{***}
\scnsdclass{***}

\scnheader{элементарное пользовательское действие}
\scnexplanation{Интерфейсное действие пользователя ostis-системы, для которого не существует других входящих в его состав интерфейсных действий этого же пользователя.}
\scnrelfromset{включение}{элементарное пользовательское действие над сущностью\\
	\scnrelfromset{включение}{действие отмены последнего элементарного пользовательского действия; 	действие вывода полной семантической окрестности указанной сущности\\
		\scnrelfromset{включение}{действие вывода минимальной семантической окрестности указанной 			сущности;действие вывода частной семантической окрестности заданной сущности;действие 				вывода базовой декомпозиции указанной сущности};
	элементарное пользовательское действие над окном ostis-системы\\
		\scnrelfromset{включение}{действие манипулирования рабочей областью окна ostis-системы\\
			\scnrelfromset{включение}{действие сужения рабочей области окна;действие расширения 				рабочей области окна};
		действие изменения конфигурации окна ostis-системы\\
			\scnrelfromset{включение}{действие перемещения окна ostis-системы;действие изменения 				размеров окна ostis-системы}
		}
	};
элементарное пользовательское действие над множеством сущностей\\
	\scnrelfromset{включение}{элементарное пользовательское действие уточнения спецификации 			команды\\
		\scnrelfromset{включение}{действие указания аргумента инициируемого действия;действие 				указания типа инициируемого действия;действие указания факта завершения перечисления 				аргументов;действие отмены указанных аргументов инициируемого действия};
	элементарное пользовательское действие над sc-текстом\\
		\scnrelfromset{включение}{действие копирования sc-текста;действие выделения sc-						текста;действие удаления sc-текста;действие вставки sc-текста}
	}
}

\scnheader{элементарное пользовательское действие над сущностью}
\scnexplanation{Элементарное пользовательское действие, аргументом которого является знак сущности.}

\scnheader{элементарное пользовательское действие над множеством сущностей}
\scnexplanation{Элементарное пользовательское действие, аргументом которого является знак множества сущностей.}

\scnheader{действие отмены последнего элементарного пользовательского действия}
\scnexplanation{Элементарное пользовательское действие над сущностью, которое удаляет из sc-памяти спецификацию последнего инициированного элементарного пользовательского действия и результат его выполнения.}

\scnheader{действие вывода полной семантической окрестности указанной сущности}
\scnexplanation{Элементарное пользовательское действие над сущностью, в результате которого выводится полная семантическая окрестность некоторой сущности в рамках выбранной структуры. Данное действие может быть реализовано нажатием левой кнопки мыши по сущности.}

\scnheader{элементарное пользовательское действие над окном ostis-системы}
\scnexplanation{Элементарное пользовательское действие над сущностью, аргументом которого является окно ostis-системы.}

\scnheader{действие вывода минимальной семантической окрестности указанной сущности}
\scnexplanation{Элементарное пользовательское действие над сущностью, в результате которого выводится минимальная семантическая окрестность некоторой сущности в рамках выбранной структуры. Данное действие может быть реализовано наведением курсора мыши на некоторую сущность.}

\scnheader{действие вывода частной семантической окрестности указанной сущности}
\scnexplanation{Элементарное пользовательское действие над сущностью, в результате которого выводится часть семантической окрестности некоторой сущности в рамках выбранной структуры. Данное действие может быть реализовано прокруткой рабочей области окна.}

\scnheader{действие вывода базовой декомпозиции указанной сущности}
\scnexplanation{Элементарное пользовательское действие над сущностью, в результате которого выводится базовая декомпозиция некоторой сущности в рамках выбранной структуры. Данное действие может быть реализовано нажатием левой кнопки мыши по сущности, которая имеет базовую декомпозицию.}

\scnheader{действие манипулирования рабочей областью окна ostis-системы}
\scnexplanation{Элементарное пользовательское действие над окном ostis-системы, направленное на изменение области просмотра sc-текста, отображённого в некотором окне ostis-системы.}

\scnheader{действие изменения конфигурации окна ostis-системы}
\scnexplanation{Элементарное пользовательское действие над окном ostis-системы, связанное с изменением формы и расположения окна ostis-системы.}

\scnheader{действие сужения рабочей области окна}
\scnexplanation{Действие манипулирования рабочей областью окна ostis-системы, направленное на уменьшение области просмотра информации, отображённой в некотором окне ostis-системы.
Данное действие может быть реализовано прокруткой колеса мыши вперёд или нажатием комбинации клавиш Ctrl и “+”.}

\scnheader{действие расширения рабочей области окна}
\scnexplanation{Действие манипулирования рабочей областью окна ostis-системы, направленное на увеличение области просмотра информации, отображённой в некотором окне ostis-системы.
Данное действие может быть реализовано прокруткой колеса мыши назад или нажатием комбинации клавиш Ctrl и “-”.}

\scnheader{действие перемещения окна ostis-системы}
\scnexplanation{Действие изменения конфигурации окна ostis-системы, связанное с изменением расположения окна ostis-системы относительно некоторой системы координат и относительно других окон ostis-системы. Данное действие может быть реализовано одновременным нажатием левой кнопки мыши по выбранному окну ostis-системы и передвижением курсора мыши в точку экрана, куда нужно переместить это окно.}

\scnheader{действие изменения размеров окна ostis-системы}
\scnexplanation{Действие изменения конфигурации окна ostis-системы, связанное с изменением формы окна ostis-системы. Данное действие может быть реализовано одновременным нажатием левой кнопки мыши у края окна ostis-системы и передвижением курсора мыши в сторону растяжения/сжатия этого окна.}

\scnheader{элементарное пользовательское действие уточнения спецификации команды}
\scnexplanation{Элементарное пользовательское действие над множеством сущностей, аргументом которого является спецификация действия, соответствующая некоторой команде.}

\scnheader{элементарное пользовательское действие над sc-текстом}
\scnexplanation{Элементарное пользовательское действие над множеством сущностей, аргументом которого является знак sc-текста.}

\scnheader{действие указания аргумента инициируемого действия}
\scnexplanation{Элементарное пользовательское действие уточнения спецификации команды, связанное с внесением выбранной сущности в список аргументов некоторого действия. Данное действие может быть реализовано нажатием правой кнопки мыши по sc-элементу.}

\scnheader{действие указания типа инициируемого действия}
\scnexplanation{Элементарное пользовательское действие уточнения спецификации команды, связанное с соотнесением знака некоторого действия со знаком класса действия.}

\scnheader{действие указания факта завершения перечисления аргументов}
\scnexplanation{Элементарное пользовательское действие уточнения спецификации команды, связанное с инициированием команды, аргументами которой становятся все сущности, соответствующие списку аргументов действия, входящего в состав спецификации. Данное действие может быть реализовано двойным нажатием левой кнопки мыши.}

\scnheader{действие отмены указанных аргументов инициируемого действия}
\scnexplanation{Элементарное пользовательское действие уточнения спецификации команды, связанное с удалением аргументов из списка аргументов некоторого действия. Данное действие может быть реализовано нажатием кнопки «Backspace» на клавиатуре.}

\scnheader{действие копирования sc-текста}
\scnexplanation{Элементарное пользовательское действие над sc-текстом, связанное с сохранением некоторого предварительно выделенного фрагмента sc-текста для вставки в другой sc-текст. Данное действие может быть реализовано комбинацией клавиш «Сtrl» + «С» на клавиатуре.}

\scnheader{действие выделения sc-текста}
\scnexplanation{Элементарное пользовательское действие над sc-текстом, связанное с выделением некоторого фрагмента sc-текста произвольного размера и содержания. Данное действие может быть реализовано либо нажатием левой кнопки мыши и передвижением курсора мыши в сторону растяжения/сжатия области выделения, либо нажатием комбинации клавиш «Сtrl» + «->» или «Сtrl» + «<-».}

\scnheader{действие удаления sc-текста}
\scnexplanation{Элементарное пользовательское действие над sc-текстом, связанное с удалением некоторого предварительно выделенного фрагмента sc-текста. Данное действие может быть реализовано нажатием клавиши «Delete» на клавиатуре.}

\scnheader{действие вставки sc-текста}
\scnexplanation{Элементарное пользовательское действие над sc-текстом, связанное со вставкой фрагмента sc-текста в указанное место другого sc-текста. Данное действие может быть реализовано комбинацией клавиш «Сtrl» + «V» на клавиатуре.}

\scnendstruct

\end{SCn}
