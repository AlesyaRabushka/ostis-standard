\begin{SCn}

\scnsectionheader{\currentname}

\scnstartsubstruct

\scnheader{Предметная область входящих в ostis-систему и выходящих из ostis-системы сообщений}
\scnsdmainclass{сообщение}
\scnsdclass{***}
\scnsdrelation{отправитель*; получатель*}

\scnheader{сообщение}
\scnidtf{дискретная информационная конструкция, используемая в процессе передачи от \textbf{отправителя*} к \textbf{получателю*}}
\scnrelto{включение}{дискретная информационная конструкция}
\scnreltoset{разбиение}{сообщение пользователя ostis-системы; сообщение системы}
%\scnreltoset{разбиение}{графическое сообщение; аудио-сообщение; текстовое сообщение}
\scnreltoset{разбиение}{атомарное сообщение; неатомарное сообщение}
\scnreltoset{разбиение}{сообщение на естественном языке; сообщение на формальном языке}
\scnreltoset{разбиение}{сообщение в прошедшем времени; сообщение в настоящем времени; сообщение в будущем времени}
\scnrelfromlist{включение}{графическое сообщение; аудио-сообщение; обонятельное сообщение; текстовое сообщение}

%Разбиение по признаку автора сообщения
\scnheader{сообщение пользователя ostis-системы}
\scnidtf{сообщение, \textbf{отправителем*} которого является пользователь ostis-системы}
\scnreltoset{разбиение}{сообщение пользователя на внешнем языке; сообщение пользователя на внутреннем языке}

\scnheader{сообщение пользователя на внешнем языке}
\scnidtf{сообщение пользователя ostis-системы, сформированное на языке интерфейсных действий (интерфейсных команд), представляющее собой последовательность действий с указанием объектов, на которых эти действия заданы, и типов действий}

\scnheader{сообщение пользователя на внутреннем языке}
\scneq{сообщение пользователя на SC-коде}
\scnidtf{сообщение пользователя ostis-системы, представляющее собой некоторый sc-текст, предназначенный для использования ostis-системой}

\scnheader{сообщение системы}
\scnidtf{сообщение, \textbf{отправителем*} которого является некоторая система}
\scnrelfrom{включение}{сообщение ostis-системы}

\scnheader{сообщение ostis-системы}
\scnidtf{сообщение, \textbf{отправителем*} которого является ostis-система}
\scnreltoset{разбиение}{эффекторное сообщение ostis-системы; рецепторное сообщение ostis-системы}

%Разбиение по признаку адресата сообщения
\scnheader{эффекторное сообщение ostis-системы}
\scnidtf{сообщение ostis-системы, инициируемое самой ostis-системой при возникновении некоторых ситуаций}
\scnnote{К ситуациям, инициирующим возникновение эффекторных сообщений, можно отнести:
\begin{scnitemize}
    \item {ситуации, возникающие при анализе деятельности самого пользователя. Например, задание аргументов, не соответствующих типу инициируемого действия или появление подсказок при использовании компонентов пользовательского интерфейса;}
    \item {ситуации, возникающие при анализе синтаксиса текстов внешних языков. Например, неполнота сформированного предложения на внешнем языке или использование конструкций, нехарактерных или некорректно использованных в контексте отдельно взятого внешнего языка.}
\end{scnitemize}}

\scnheader{рецепторное сообщение ostis-системы}
\scnidtf{сообщение ostis-системы, являющееся реакцией на императивное сообщение.}
\scnnote{Возможными реакциями ostis-системы на императивное сообщение пользователя являются:
\begin{scnitemize}
    \item {указание факта завершения выполнения некоторой задачи, что, например, характерно для поведенческих действий;}
    \item {получение ответа* на поставленную задачу, формируемого либо в результате анализа базы знаний пользовательского интерфейса, либо в результате анализа предметной части базы знаний самой ostis-системы.}
\end{scnitemize}}

%разбиение по атомарности
\scnheader{атомарное сообщение}
\scnidtf{сообщение, в состав которого не входят другие сообщения}
\scnreltoset{разбиение}{вопросительное сообщение; побудительное сообщение; повествовательное сообщение}

\scnheader{вопросительное сообщение}
\scnidtf{атомарное сообщение, выражающее вопрос}
\scnrelfrom{включение}{сообщение запроса информации}

\scnheader{сообщение запроса информации}
\scnidtf{атомарное сообщение, являющееся запросом какой-либо новой информации}

\scnheader{побудительное сообщение}
\scnidtf{атомарное сообщение, побуждающее к какому-либо действию}
\scnreltoset{разбиение}{императивное сообщение; сообщение пожелания}

\scnheader{императивное сообщение}
\scnidtf{побудительное сообщение, содержащее в себе побуждение к действию}

\scnheader{сообщение пожелания}
\scnidtf{побудительное сообщение, содержащее какое-либо пожелание}

\scnheader{повествовательное сообщение}
\scnidtf{атомарное сообщение, сообщающее о каких-либо новых фактах}
\scnreltoset{разбиение}{информационное сообщение; нейтральное сообщение}

\scnheader{информационное сообщение}
\scnidtf{повествовательное сообщение, сообщающее какую-либо новую информацию}
\scnreltoset{разбиение}{сообщение информирования; сообщение отрицания информации; сообщение подтверждения информации}

\scnheader{сообщение информирования}
\scnidtf{информационное сообщение, предназначенное для оповещения о некоторых событиях, произошедших с \textbf{объектом\scnrolesign} сообщения}
\scnrelfromlist{включение}{сообщение информирования о достижениях; сообщение информирования о местоположении; сообщение информирования об объекте; сообщение информирования об отношении; сообщение информирования о проблемах; сообщение информирования о состоянии здоровья}

%================================
\scnheader{сообщение информирования о достижениях}
\scnidtf{сообщение информирования, предназначенное для оповещения о достижениях \textbf{объекта\scnrolesign} сообщения}

\scnheader{сообщение информирования о местоположении}
\scnidtf{сообщение информирования, предназначенное для оповещения о местоположении \textbf{объекта\scnrolesign} сообщения}

%\scnheader{сообщение информирования об объекте}
%\scnidtf{сообщение информирования, предназначенное для оповещения об \textbf{объекте\scnrolesign} сообщения}

\scnheader{сообщение информирования об отношении}
\scnidtf{сообщение информирования, предназначенное для оповещения о связи \textbf{объекта\scnrolesign} сообщения с иной сущностью}

\scnheader{сообщение информирования о проблемах}
\scnidtf{сообщение информирования, предназначенное для оповещения о проблемах \textbf{объекта\scnrolesign} сообщения}

\scnheader{сообщение информирования о состоянии здоровья}
\scnidtf{сообщение информирования, предназначенное для оповещения об состоянии здоровья \textbf{объекта\scnrolesign} сообщения}
%================================

\scnheader{сообщение отрицания информации}
\scnidtf{информационное сообщение, подтверждающее какую-либо упомянутую ранее в диалоге информацию}

\scnheader{сообщение подтверждения информации}
\scnidtf{информационное сообщение, отрицающее какую-либо упомянутую ранее в диалоге информацию}

\scnheader{нейтральное сообщение}
\scnidtf{повествовательное сообщение, не несущее какой-либо новой информации, а также не подверждающее и не опровергающее какую-либо озвученную ранее информацию}

\scnheader{неатомарное сообщение}
\scnidtf{сообщение, в состав которого входят атомарные сообщения}

%разбиение по типу информации
\scnheader{графическое сообщение}
\scnidtf{сообщение, содержащее графическую информацию}
\scnrelfrom{включение}{видео-сообщение}

\scnheader{видео-сообщение}
\scnidtf{сообщение, содержащее видео-информацию}

\scnheader{аудио-сообщение}
\scneq{речевое сообщение}
\scnidtf{сообщение, представленное в звуковой форме}

\scnheader{обонятельное сообщение}
\scnidtf{сообщение, содержащее информацию о запахах}

\scnheader{текстовое сообщение}
\scnidtf{сообщение, содержащее текстовую информацию}

\scnheader{сообщение на естественном языке}
\scnidtf{сообщение, написанное с использованием естественного языка}

\scnheader{сообщение на формальном языке}
\scneq{сообщение на искусственном языке}
\scnidtf{сообщение, написанное с использованием формального языка}

%
\scnheader{отправитель*}
\scneq{автор сообщения*}
\scnidtf{бинарное ориентированное отношение, связывающее сообщение с его отправителем}
\scnrelfrom{первый домен}{сообщение}
\scnrelfrom{второй домен}{человек $\cup$ компьютерная система}
\scniselement{бинарное отношение}
\scniselement{ориентированное отношение}
\scniselement{антирефлексивное отношение}
\scniselement{антитранзитивное отношение}
\scniselement{асимметричное отношение}

\scnheader{получатель*}
\scnidtf{бинарное ориентированное отношение, связывающее сообщение с его получателем}
\scnrelfrom{первый домен}{сообщение}
\scnrelfrom{второй домен}{человек $\cup$ компьютерная система}
\scniselement{бинарное отношение}
\scniselement{ориентированное отношение}
\scniselement{антирефлексивное отношение}
\scniselement{антитранзитивное отношение}
\scniselement{асимметричное отношение}


\end{SCn}