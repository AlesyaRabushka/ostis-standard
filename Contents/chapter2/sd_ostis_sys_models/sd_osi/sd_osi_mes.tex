\begin{SCn}

\scnsectionheader{\currentname}

\scnstartsubstruct

\scnheader{Предметная область входящих в ostis-систему и выходящих из ostis-системы сообщений}
\scnsdmainclass{сообщение}
\scnsdclass{***}
\scnsdrelation{авторы*(?); отправитель*; получатель*}

\scnheader{сообщение}
\scnidtf{дискретная информационная конструкция, используемая в процессе передачи от \textbf{отправителя*} к \textbf{получателю*}}
\scnrelto{включение}{дискретная информационная конструкция}
\scnreltoset{разбиение}{сообщение пользователя ostis-системы; сообщение системы}
\scnreltoset{разбиение}{атомарное сообщение; 
неатомарное сообщение\\
\scnaddlevel{1}
\scnidtf{сообщение, в состав которого входят атомарные сообщения}
\scnaddlevel{-1}}
\scnreltoset{разбиение}{сообщение на естественном языке
\scnaddlevel{1}
\scnidtf{сообщение, написанное с использованием естественного языка}
\scnaddlevel{-1}; 
сообщение на формальном языке\\
\scnaddlevel{1}
\scneq{сообщение на искусственном языке}
\scnidtf{сообщение, написанное с использованием формального языка}
\scnaddlevel{-1}}
\scnrelfromlist{включение}{графическое сообщение\\
\scnaddlevel{1}
\scnidtf{сообщение, содержащее графическую информацию}
\scnrelfrom{включение}{видео-сообщение}
\scnaddlevel{1}
\scnidtf{сообщение, содержащее видео-информацию}
\scnaddlevel{-1}
\scnaddlevel{-1}; 
аудио-сообщение\\
\scnaddlevel{1}
\scneq{речевое сообщение}
\scnidtf{сообщение, представленное в звуковом формате}
\scnaddlevel{-1};
обонятельное сообщение\\
\scnaddlevel{1}
\scnidtf{сообщение, содержащее информацию о запахах}
\scnaddlevel{-1}; 
текстовое сообщение\\
\scnaddlevel{1}
\scnidtf{сообщение, содержащее текстовую информацию}
\scnaddlevel{-1}
}
\scnrelfromlist{включение}{
сообщение, требующее приведения используемых в тексте слов к словарной форме
\scnaddlevel{1}
\scnidtf{сообщение, которое необходимо передать для обработки агенту нормализации текста}
\scnaddlevel{-1};
сообщение, требующее лексического разбора
\scnaddlevel{1}
\scnidtf{сообщение, которое необходимо передать для обработки агенту построения последовательности лексем в сообщении}
\scnaddlevel{-1};
сообщение, требующее синтаксического разбора
\scnaddlevel{1}
\scnidtf{сообщение, которое необходимо передать для обработки агенту синтаксического разбора}
\scnaddlevel{-1};
сообщение, имеющее неоднозначность
\scnaddlevel{1}
\scnidtf{сообщение, которое необходимо передать для обработки агенту устранения неоднозначностей}
\scnaddlevel{-1};
сообщение, требующее классификации
\scnaddlevel{1}
\scnidtf{сообщение, которое необходимо передать для обработки агенту классификации сообщений}
\scnaddlevel{-1};
сообщение, требующее выделения аргументов
\scnaddlevel{1}
\scnidtf{сообщение, которое необходимо передать для обработки агенту выделения аргументов сообщения пользователя}
\scnaddlevel{-1};
полностью сформированное сообщение пользователя
\scnaddlevel{1}
\scnidtf{сообщение, прошедшее обработку агентами}
\scnaddlevel{-1};
сообщение, требующее композиции текста
\scnaddlevel{1}
\scnidtf{сообщение, которое необходимо передать для обработки агенту}
\scnaddlevel{-1};
}
\scnrelfromlist{включение}{
сообщение, требующее трансляции
\scnaddlevel{1}
\scnidtf{сообщение, которое необходимо сформировать системой для дальнейшей передачи его пользователю}
\scnaddlevel{-1};
протранслированное сообщение
\scnaddlevel{1}
\scnidtf{сообщение, которое было сформировано системой для дальнейшей передачи его пользователю}
\scnaddlevel{-1};
}

%Разбиение по признаку автора сообщения
\scnheader{сообщение пользователя ostis-системы}
\scnidtf{сообщение, \textbf{отправителем*} которого является пользователь ostis-системы}
\scnreltoset{разбиение}{сообщение пользователя на внешнем языке
\scnaddlevel{1}
\scnidtf{сообщение пользователя ostis-системы, сформированное на языке интерфейсных действий (интерфейсных команд), представляющее собой последовательность действий с указанием объектов, на которых эти действия заданы, и типов действий}
\scnaddlevel{-1}; 
сообщение пользователя на внутреннем языке\\
\scnaddlevel{1}
\scneq{сообщение пользователя на SC-коде}
\scnidtf{сообщение пользователя ostis-системы, представляющее собой некоторый sc-текст, предназначенный для использования ostis-системой}
\scnaddlevel{-1}}

%\scnheader{сообщение пользователя на внешнем языке}
%\scnidtf{сообщение пользователя ostis-системы, сформированное на языке интерфейсных действий (интерфейсных команд), представляющее собой последовательность действий с указанием объектов, на которых эти действия заданы, и типов действий}

%\scnheader{сообщение пользователя на внутреннем языке}
%\scneq{сообщение пользователя на SC-коде}
%\scnidtf{сообщение пользователя ostis-системы, представляющее собой некоторый sc-текст, предназначенный для использования ostis-системой}

\scnheader{сообщение системы}
\scnidtf{сообщение, \textbf{отправителем*} которого является некоторая система}
\scnrelfrom{включение}{сообщение ostis-системы}

\scnheader{сообщение ostis-системы}
\scnidtf{сообщение, \textbf{отправителем*} которого является ostis-система}
\scnreltoset{разбиение}{эффекторное сообщение ostis-системы; рецепторное сообщение ostis-системы}

%Разбиение по признаку адресата сообщения
\scnheader{эффекторное сообщение ostis-системы}
\scnidtf{сообщение ostis-системы, инициируемое самой ostis-системой при возникновении некоторых ситуаций}
\scnnote{К ситуациям, инициирующим возникновение эффекторных сообщений, можно отнести:
\begin{scnitemize}
    \item {ситуации, возникающие при анализе деятельности самого пользователя. Например, задание аргументов, не соответствующих типу инициируемого действия или появление подсказок при использовании компонентов пользовательского интерфейса;}
    \item {ситуации, возникающие при анализе синтаксиса текстов внешних языков. Например, неполнота сформированного предложения на внешнем языке или использование конструкций, нехарактерных или некорректно использованных в контексте отдельно взятого внешнего языка.}
\end{scnitemize}}

\scnheader{рецепторное сообщение ostis-системы}
\scnidtf{сообщение ostis-системы, являющееся реакцией на императивное сообщение.}
\scnnote{Возможными реакциями ostis-системы на императивное сообщение пользователя являются:
\begin{scnitemize}
    \item {указание факта завершения выполнения некоторой задачи, что, например, характерно для поведенческих действий;}
    \item {получение ответа* на поставленную задачу, формируемого либо в результате анализа базы знаний пользовательского интерфейса, либо в результате анализа предметной части базы знаний самой ostis-системы.}
\end{scnitemize}}

%разбиение по атомарности
\scnheader{атомарное сообщение}
\scnidtf{сообщение, в состав которого не входят другие сообщения}
\scnreltoset{разбиение}{вопросительное сообщение; императивное сообщение; повествовательное сообщение}
\scnreltoset{разбиение}{сообщение без эмоциональной окраски\\
\scnaddlevel{1}
\scnidtf{атомарное сообщение, не несущее эмоциональную окраску в какую-либо из эмоций}
\scnaddlevel{-1}; 
сообщение, имеющее эмоциональную окраску}
\scnreltoset{разбиение}{сообщение о прошлом
\scnaddlevel{1}
\scnidtf{атомарное сообщение, содержащее информацию о прошлом}
\scnaddlevel{-1}; 
сообщение о настоящем
\scnaddlevel{1}
\scnidtf{атомарное сообщение, содержащее информацию о настоящем}
\scnaddlevel{-1}; 
сообщение о будушем
\scnaddlevel{1}
\scnidtf{атомарное сообщение, содержащее информацию о будущем}
\scnaddlevel{-1}}

\scnheader{вопросительное сообщение}
\scnidtf{атомарное сообщение, выражающее вопрос}
\scnrelfrom{включение}{сообщение запроса информации\\
\scnaddlevel{1}
\scnidtf{атомарное сообщение, являющееся запросом какой-либо новой информации}
\scnaddlevel{-1}}

%\scnheader{сообщение запроса информации}
%\scnidtf{атомарное сообщение, являющееся запросом какой-либо новой информации}

\scnheader{императивное сообщение}
\scnidtf{атомарное сообщение, побуждающее к какому-либо действию}
\scnrelfrom{включение}{сообщение пожелания\\
\scnaddlevel{1}
\scnidtf{побудительное сообщение, содержащее какое-либо пожелание}
\scnaddlevel{-1}}

%\scnheader{сообщение пожелания}
%\scnidtf{побудительное сообщение, содержащее какое-либо пожелание}

\scnheader{повествовательное сообщение}
\scnidtf{атомарное сообщение, предназначенное для передачи информации об \textbf{объекте\scnrolesign} сообщения}
\scnreltoset{разбиение}{информационное сообщение; 
нейтральное сообщение
\scnaddlevel{1}
\scnidtf{повествовательное сообщение, не несущее новой информации и не подверждающее/отрицающее ранее известную информацию}
\scnaddlevel{-1}}

\scnheader{информационное сообщение}
\scnidtf{повествовательное сообщение, предназначенное для передачи новой информации об \textbf{объекте\scnrolesign} сообщения}
\scnreltoset{разбиение}{сообщение информирования; 
сообщение отрицания информации
\scnaddlevel{1}
\scnidtf{информационное сообщение, отрицающее какую-либо ранее известную информацию}
\scnaddlevel{-1};
сообщение подтверждения информации
\scnaddlevel{1}
\scnidtf{информационное сообщение, подтверждающее какую-либо ранее известную информацию}
\scnaddlevel{-1}}

\scnheader{сообщение информирования}
\scnidtf{информационное сообщение, предназначенное для передачи ранее неизвестной информации об \textbf{объекте\scnrolesign} сообщения}
\scnrelfromlist{включение}{сообщение информирования о достижениях\\
\scnaddlevel{1}
\scnidtf{сообщение информирования, предназначенное для оповещения о достижениях \textbf{объекта\scnrolesign} сообщения}
\scnaddlevel{-1}; 
сообщение информирования о местоположении\\
\scnaddlevel{1}
\scnidtf{сообщение информирования, предназначенное для оповещения о местоположении \textbf{объекта\scnrolesign} сообщения}
\scnaddlevel{-1};
сообщение информирования об отношении\\
\scnaddlevel{1}
\scnidtf{сообщение информирования, предназначенное для оповещения о связи \textbf{объекта\scnrolesign} сообщения с иной сущностью}
\scnaddlevel{-1};
сообщение информирования о проблемах\\
\scnaddlevel{1}
\scnidtf{сообщение информирования, предназначенное для оповещения о проблемах \textbf{объекта\scnrolesign} сообщения}
\scnaddlevel{-1};
сообщение информирования о состоянии здоровья
\scnaddlevel{1}
\scnidtf{сообщение информирования, предназначенное для оповещения об состоянии здоровья \textbf{объекта\scnrolesign} сообщения}
\scnaddlevel{-1}}

%================================
%\scnheader{сообщение информирования о достижениях}
%\scnidtf{сообщение информирования, предназначенное для оповещения о достижениях \textbf{объекта\scnrolesign} сообщения}

%\scnheader{сообщение информирования о местоположении}
%\scnidtf{сообщение информирования, предназначенное для оповещения о местоположении \textbf{объекта\scnrolesign} сообщения}

%\scnheader{сообщение информирования об объекте}
%\scnidtf{сообщение информирования, предназначенное для оповещения об \textbf{объекте\scnrolesign} сообщения}

%\scnheader{сообщение информирования об отношении}
%\scnidtf{сообщение информирования, предназначенное для оповещения о связи \textbf{объекта\scnrolesign} сообщения с иной сущностью}

%\scnheader{сообщение информирования о проблемах}
%\scnidtf{сообщение информирования, предназначенное для оповещения о проблемах \textbf{объекта\scnrolesign} сообщения}

%\scnheader{сообщение информирования о состоянии здоровья}
%\scnidtf{сообщение информирования, предназначенное для оповещения об состоянии здоровья \textbf{объекта\scnrolesign} сообщения}
%================================

%\scnheader{сообщение отрицания информации}
%\scnidtf{информационное сообщение, отрицающее какую-либо ранее известную информацию}

%\scnheader{сообщение подтверждения информации}
%\scnidtf{информационное сообщение, подтверждающее какую-либо ранее известную информацию}

%\scnheader{нейтральное сообщение}
%\scnidtf{повествовательное сообщение, не несущее новой информации и не подверждающее/отрицающее ранее известную информацию}

%\scnheader{сообщение без эмоциональной окраски}
%\scnidtf{атомарное сообщение, не несущее эмоциональную окраску в какую-либо из эмоций}

\scnheader{сообщение, имеющее эмоциональную окраску}
\scnidtf{атомарное сообщение, несущее эмоциональную окраску в какую-либо из эмоций}
\scnreltoset{разбиение}{сообщение, имеющее негативную эмоциональную окраску\\
\scnaddlevel{1}
\scnidtf{атомарное сообщение, имеющее эмоциональную окраску в одну из эмоций, являющихся негативным эмоциональным состоянием}
\scnaddlevel{-1};
сообщение, имеющее нейтральную эмоциональную окраску\\
\scnaddlevel{1}
\scnidtf{атомарное сообщение, имеющее эмоциональную окраску в одну из эмоций, являющихся нейтральным эмоциональным состоянием}
\scnaddlevel{-1};
сообщение, имеющее положительную эмоциональную окраску\\
\scnaddlevel{1}
\scnidtf{атомарное сообщение, имеющее эмоциональную окраску в одну из эмоций, являющихся положительным эмоциональным состоянием}
\scnaddlevel{-1};
сообщение с неопределенной эмоциональной окраской
\scnaddlevel{1}
\scnidtf{атомарное сообщение, имеющее неопределенную эмоциональную окраску, на основе которой трудно определить эмоцию}
\scntext{Примечание}{Данный тип сообщений может быть вызван в случае, когда человек испытывает сразу несколько, как правило, противоречащих друг другу эмоций.}
\scnaddlevel{-1}}

%\scnheader{сообщение, имеющее негативную эмоциональную окраску}
%\scnidtf{атомарное сообщение, имеющее эмоциональную окраску в одну из эмоций, являющихся негативным эмоциональным состоянием}

%\scnheader{сообщение, имеющее нейтральную эмоциональную окраску}
%\scnidtf{атомарное сообщение, имеющее эмоциональную окраску в одну из эмоций, являющихся нейтральным эмоциональным состоянием}

%\scnheader{сообщение, имеющее положительную эмоциональную окраску}
%\scnidtf{атомарное сообщение, имеющее эмоциональную окраску в одну из эмоций, являющихся положительным эмоциональным состоянием}

%\scnheader{сообщение с неопределенной эмоциональной окраской}
%\scnidtf{атомарное сообщение, имеющее неопределенную эмоциональную окраску, на основе которой трудно определить эмоцию}
%\scntext{Примечание}{Данный тип сообщений может быть вызван в случае, когда человек испытывает сразу несколько, как правило, противоречащих друг другу эмоций.}

%\scnheader{сообщение о прошлом}
%\scnidtf{атомарное сообщение, содержащее информацию о прошлом}

%\scnheader{сообщение о настоящем}
%\scnidtf{атомарное сообщение, содержащее информацию о настоящем}

%\scnheader{сообщение о будушем}
%\scnidtf{атомарное сообщение, содержащее информацию о будущем}

%\scnheader{неатомарное сообщение}
%\scnidtf{сообщение, в состав которого входят атомарные сообщения}

%разбиение по типу информации
%\scnheader{графическое сообщение}
%\scnidtf{сообщение, содержащее графическую информацию}
%\scnrelfrom{включение}{видео-сообщение}

%\scnheader{видео-сообщение}
%\scnidtf{сообщение, содержащее видео-информацию}

%\scnheader{аудио-сообщение}
%\scneq{речевое сообщение}
%\scnidtf{сообщение, представленное в звуковой форме}

%\scnheader{обонятельное сообщение}
%\scnidtf{сообщение, содержащее информацию о запахах}

%\scnheader{текстовое сообщение}
%\scnidtf{сообщение, содержащее текстовую информацию}

%\scnheader{сообщение на естественном языке}
%\scnidtf{сообщение, написанное с использованием естественного языка}

%\scnheader{сообщение на формальном языке}
%\scneq{сообщение на искусственном языке}
%\scnidtf{сообщение, написанное с использованием формального языка}

\scnheader{временная точка отправления сообщения}
\scnrelto{включение}{квазиматериальная сущность}
\scnidtf{временная точка, в которой сообщение было отправлено}

\scnheader{временная точка получения сообщения}
\scnrelto{включение}{квазиматериальная сущность}
\scnidtf{временная точка, в которой сообщение было получено}

%отношения
\scnheader{отправитель*}
\scneq{отправитель сообщения*}
\scnidtf{бинарное ориентированное отношение, связывающее сообщение с его отправителем}
\scnrelfrom{первый домен}{сообщение}
\scnrelfrom{второй домен}{человек $\cup$ компьютерная система}
\scniselement{бинарное отношение}
\scniselement{ориентированное отношение}
\scniselement{антирефлексивное отношение}
\scniselement{антитранзитивное отношение}
\scniselement{асимметричное отношение}

\scnheader{получатель*}
\scneq{получатель сообщения*}
\scnidtf{бинарное ориентированное отношение, связывающее сообщение с его получателем}
\scnrelfrom{первый домен}{сообщение}
\scnrelfrom{второй домен}{человек $\cup$ компьютерная система}
\scniselement{бинарное отношение}
\scniselement{ориентированное отношение}
\scniselement{антирефлексивное отношение}
\scniselement{антитранзитивное отношение}
\scniselement{асимметричное отношение}


\end{SCn}