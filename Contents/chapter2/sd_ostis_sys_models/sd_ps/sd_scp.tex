\begin{SCn}

\scnsectionheader{\currentname}

\scnstartsubstruct


\scnheader{Предметная область Базового языка программирования ostis-систем (языка SCP -- Semantic Code Programming)}
\scnidtf{Предметная область Базового языка программирования ostis-систем}
\scnidtf{Предметная область Языка SCP}
\scnnote{В данную предметную область включаются все тексты программ Языка SCP. В ней исследуется типология операторов этих программ и заданные на них отношения.}
\scniselement{предметная область}
\scnsdmainclasssingle{scp-программа}
\scnsdclass{агентная scp-программа;scp-процесс;scp-оператор;атомарный тип scp-оператора}
\scnsdrelation{начальный оператор';параметр scp-программы';in-параметр’;out-параметр’;scp-операнд’}


\scnheader{Язык SCP}
\scnidtftext{часто используемый sc-идентификатор}{scp-программа}
\scnexplanation{В качестве базового языка для описания программ обработки текстов
	\textit{SC-кода} предлагается \textit{Язык SCP}.
	
	\textit{Язык SCP} -- это графовый язык процедурного программирования,
	предназначенный для эффективной обработки \textit{sc-текстов}. \textit{Язык SCP} является языком параллельного асинхронного программирования.
	
	Языком представления данных для текстов \textit{Языка SCP}
	(\textit{scp-программ}) является \textit{SC-код} и, соответственно, любые
	варианты его внешнего представления. \textit{Язык SCP} сам построен на
	основе \textit{SC-кода}, вследствие чего \textit{scp-программы} сами по себе
	могут входить в состав обрабатываемых данных для \textit{scp-программ}, в т.ч. по отношению к самим себе. Таким образом, \textit{язык SCP} предоставляет возможность
	построения реконфигурируемых программ. Однако для обеспечения
	возможности реконфигурирования программы непосредственно в процессе ее
	интерпретации необходимо на уровне интерпретатора \textit{Языка SCP
		(Aбстрактной scp-машины)} обеспечить уникальность каждой исполняемой
	копии исходной программы. Такую исполняемую копию, сгенерированную на
	основе \textit{scp-программы}, будем называть \textit{scp-процессом}.
	Включение знака некоторого \textit{действия в sc-памяти} во множество
	\textit{scp-процессов} гарантирует тот факт, что в декомпозиции данного
	действия будут присутствовать только знаки элементарных действий
	(\textit{scp-операторов}), которые может интерпретировать реализация
	\textit{Aбстрактной scp-машины} (интерпретатора scp-программ).
	
	\textit{Язык SCP} рассматривается как ассемблер для семантического компьютера.}


\scnheader{Базовая модель обработки sc-текстов}
\scnreltoset{объединение}{Предметная область Базового языка программирования ostis-систем;Модель Абстрактной scp-машины}
\scnrelfromset{особенности}{\scnfileitem{Тексты программ \textit{Языка SCP} записываются при помощи тех же
	унифицированных семантических сетей, что и обрабатываемая информация, таким образом, можно сказать, что \textit{Синтаксис языка SCP} на базовом уровне совпадает с \textit{Синтаксисом SC-кода}.};\scnfileitem{Подход к интерпретации \textit{scp-программ} предполагает создание при
	каждом вызове \textit{scp-программы} уникального \textit{scp-процесса}.}}
\scnrelfromset{достоинства}{\scnfileitem{Одновременно в общей памяти могут выполняться несколько независимых
	\textit{sc-агентов}, при этом разные копии \textit{sc-агентов} могут выполняться на разных серверах, за счет распределенной реализации интерпретатора sc-моделей (\textit{платформы реализации sc-моделей компьютерных систем}). Более
	того, \textit{Язык SCP} позволяет осуществлять параллельные асинхронные
	вызовы подпрограмм с последующей синхронизацией, и даже параллельно
	выполнять операторы в рамках одной \textit{scp-программы}.};
\scnfileitem{Перенос \textit{sc-агента} из одной системы в другую заключается в простом переносе фрагмента базы знаний, без каких-либо дополнительных операций, зависящих от платформы интерпретации.};
\scnfileitem{Тот факт, что спецификации \textit{sc-агентов} и их программы могут быть записаны
	на том же языке, что и обрабатываемые знания, существенно сокращает
	перечень специализированных средств, предназначенных для
	проектирования машин обработки знаний, и упрощает их разработку за
	счет использования более универсальных компонентов.};
\scnfileitem{Тот факт, что для интерпретации \textit{scp-программы} создается
	соответствующий ей уникальный \textit{scp-процесс}, позволяет по
	возможности оптимизировать план выполнения перед его реализацией и
	даже непосредственно в процессе выполнения без потенциальной опасности
	испортить общий универсальный алгоритм всей программы. Более того,
	такой подход к проектированию и интерпретации программ позволяет
	говорить о возможности создания самореконфигурируемых программ.}}


\scnheader{Абстрактная scp-машина}
\scnrelfrom{модель}{Модель Абстрактной scp-машины}
\scnnote{\textit{Абстрактная scp-машина} представляет собой интерпретатор \textit{scp-программ}, который должен являться частью \textit{платформы интерпретации sc-моделей компьютерных систем} (хотя в общем случае могут существовать варианты платформы, не содержащие такого интерпретатора, что, однако, не позволит использовать достоинства предлагаемой базовой модели}


\scnheader{scp-программа}
\scnsubset{программа в sc-памяти}
\scnexplanation{Каждая \textbf{\textit{scp-программа}} представляет собой \textit{обобщенную структуру}, описывающую один из вариантов декомпозиции действий некоторого класса, выполняемых в sc-памяти. Знак \textit{sc-переменной}, соответствующей конкретному декомпозируемому действию является в рамках \textbf{\textit{scp-программы}} \textit{ключевым sc-элементом'}. Также явно указывается принадлежность данного знака множеству \textit{scp-процессов}.
	
Принадлежность некоторого действия множеству \textit{scp-процессов} гарантирует тот факт, что в декомпозиции данного действия будут присутствовать только знаки элементарных действий (\textit{scp-операторов}), которые может интерпретировать реализация абстрактной scp-машины.

Таким образом, каждая \textbf{\textit{scp-программа}} описывает в обобщенном виде декомпозицию некоторого \textit{scp-процесса} на взаимосвязанные \textit{scp-операторы}, с указанием, при их наличии, аргументов для данного \textit{scp-процесса}.

По сути каждая \textbf{\textit{scp-программа}} представляет собой описание последовательности элементарных операций, которые необходимо выполнить над семантической сетью, чтобы выполнить более сложное действие некоторого класса.}
\scnrelfrom{типичная семантическая окрестность}{\scnfilescg{figures/sd_scp/program_example.png}}\\
\scnaddlevel{1}
	\scnexplanation{В приведенном примере показана \textit{scp-программа}, состоящая из трех \textit{scp-операторов}. Данная программа проверяет, содержится ли в заданном множестве (первый параметр) заданный элемент (второй параметр), и, если нет, то добавляет его в это множество.}	
\scnaddlevel{-1}
\scnhaselementrole{пример}{\scnfilescn{
		\scnheader{\_scp\_process}
		\scnisvarelement{scp-процесс}
		\scnhasvarelementrole{1;in-параметр}{\textunderscore set1}
		\scnhasvarelementrole{2;in-параметр}{\textunderscore element1}
		\scnvarrelto{декомпозиция действия}{\textunderscore ...}
		\scnaddlevel{1}
		\scnhasvarelementrole{1}{\textunderscore operator1}
		\scnaddlevel{1}
		\scnisvarelement{searchElStr3}
		\scnhasvarelementrole{1; scp-операнд с заданным значением; scp-константа}{\textunderscore set1}
		\scnhasvarelementrole{2; scp-операнд со свободным значением; scp-переменная; sc-дуга основного вида}{\textunderscore arc1}
		\scnhasvarelementrole{3; scp-операнд с заданным значением; scp-константа}{\textunderscore element1}
		\scnvarrelfrom{последовательность действий при отрицательном результате}{\textunderscore operator2}
		\scnvarrelfrom{последовательность действий при положительном результате}{\textunderscore operator3}
		\scnaddlevel{-1}
		\scnhasvarelement{\textunderscore operator2}
		\scnaddlevel{1}
		\scnisvarelement{genElStr3}
		\scnhasvarelementrole{1; scp-операнд с заданным значением; scp-константа}{\textunderscore set1}
		\scnhasvarelementrole{2; scp-операнд со свободным значением; scp-переменная; sc-дуга основного вида}{\textunderscore arc1}
		\scnhasvarelementrole{3; scp-операнд с заданным значением; scp-константа}{\textunderscore element1}
		\scnvarrelfrom{следующий оператор}{\textunderscore operator3}
		\scnaddlevel{-1}
		\scnhasvarelement{\textunderscore operator3}
		\scnaddlevel{1}
		\scnisvarelement{return}
		\scnaddlevel{-1}
		\scnaddlevel{-1}
}}


\scnheader{агентная scp-программа}
\scnrelto{включение}{scp-программа}
\scnexplanation{\textbf{\textit{агентные scp-программы}} представляют собой частный случай \textit{scp-программ} вообще, однако заслуживают отдельного рассмотрения, поскольку используются наиболее часто. \textit{Scp-программы} данного класса представляют собой реализации программ агентов обработки знаний, и имеют жестко фиксированный набор параметров. Каждая такая программа имеет ровно два \textit{in-параметра’}. Значение первого параметра является знаком бинарной ориентированной пары, являющейся вторым компонентом связки отношения \textit{первичное условие инициирования*} для абстрактного \textit{sc-агента}, в множество \textit{программ sc-агента*} которого входит рассматриваемая \textbf{\textit{агентная scp-программа}}, и, по сути, описывает класс событий, на которые реагирует указанный sc-агент.

Значением второго параметра является \textit{sc-элемент}, с которым непосредственно связано событие, в результате возникновения которого был инициирован соответствующий \textit{sc-агент}, т.е., например, сгенерированная либо удаляемая \textit{sc-дуга} или \textit{sc-ребро}.}


\scnheader{абстрактный sc-агент, реализуемый на Языке SCP}
\scnrelfromset{принципы реализации}{
	\scnfileitem{общие принципы организации взаимодействия
		\textit{sc-агентов} и пользователей \textit{ostis-системы} через общую
		\textit{sc-память}};
	\scnfileitem{в результате появления в sc-памяти некоторой конструкции,
		удовлетворяющей условию инициирования какого-либо \textit{абстрактного
			sc-агента}, реализованного при помощи \textit{Языка SCP}, в
		\textit{sc-памяти} генерируется и инициируется \textit{scp-процесс}. В
		качестве шаблона для генерации используется \textit{агентная
			scp-программа}, указанная во множестве программ соответствующего
		\textit{абстрактного sc-агента}};
	\scnfileitem{каждый такой \textit{scp-процесс}, соответствующий некоторой
		\textit{агентной scp-программе}, может быть связан с набором структур,
		описывающих блокировки различных типов. Таким образом, синхронизация
		взаимодействия параллельно выполняемых \textit{scp-процесcов}
		осуществляется так же, как и в случае любых других \textit{действий в
			sc-памяти}};
	\scnfileitem{В рамках \textit{scp-процесса} могут создаваться дочерние
		\textit{scp-процессы}, однако синхронизация между ними при необходимости
		осуществляется посредством введения дополнительных внутренних
		блокировок. Таким образом, каждый \textit{scp-процесс} с точки зрения
		\textit{процессов в sc-памяти} является атомарным и законченным актом
		деятельности некоторого \textit{sc-агента}};
	\scnfileitem{во избежание нежелательных изменений в самом теле \textit{scp-процесса},
		вся конструкция, сгенерированная на основе некоторой
		\textit{scp-программы} (весь текст \textit{scp-процесса}), должна быть
		добавлена в \textit{полную блокировку}, соответствующую данному \textit{scp-процессу}};
	\scnfileitem{все конструкции, сгенерированные в процессе выполнения
		\textit{scp-процесса}, автоматически попадают в \textit{полную
			блокировку}, соответствующую данному \textit{scp-процессу}.
		Дополнительно следует отметить, что знак самой этой структуры и вся метаинформация о ней также включаются в эту структуру};
	\scnfileitem{при необходимости можно вручную разблокировать или заблокировать   некоторую конструкцию каким-либо типом блокировки, используя   соответствующие \textit{scp-операторы} класса \textit{scp-оператор управления
			блокировками}};
	\scnfileitem{после завершения выполнения некоторого \textit{scp-процесса} его текст как
		правило, удаляется из \textit{sc-памяти}, а все заблокированные
		конструкции освобождаются (разрушаются знаки структур, обозначавших
		блокировки)};
	\scnfileitem{несмотря на то, что каждый \textit{scp-оператор} представляет собой атомарное
		\textit{действие в sc-памяти}, дополнительные блокировки, соответствующие
		одному оператору не вводятся, чтобы избежать громоздкости и избытка
		дополнительных системных конструкций, создаваемых при выполнении
		некоторого \textit{scp-процесса}. Вместо этого используются блокировки, общие
		для всего \textit{scp-процесса}. Таким образом, агенты \textit{Абстрактной scp-машины}
		при интерпретации \textit{scp-операторов} работают только с учетом блокировок,
		общих для всего интерпретируемого \textit{scp-процесса}};
	\scnfileitem{как правило, частный \textit{класс действий}, соответствующий конкретной
		\textit{scp-программе} явно не вводится, а используется более общий
		класс \textit{scp-процесс}, за исключением тех случаев, когда введение
		специального \textit{класса действий} необходимо по каким-либо другим
		соображениям}}


\scnheader{scp-процесс}
\scnexplanation{Под \textbf{\textit{scp-процессом}} понимается некоторое \textit{действие в sc-памяти}, однозначно описывающее конкретный акт выполнения некоторой \textit{scp-программы} для заданных исходных данных. Если \textit{scp-программа} описывает алгоритм решения какой-либо задачи в общем виде, то \textit{scp-процесс} обозначает конкретное действие, реализующее данный алгоритм для заданных входных параметров.

По сути, \textbf{\textit{scp-процесс}} представляет собой уникальную копию, созданную на основе \textit{scp-программы}, в которой каждой \textit{sc-переменной}, за исключением \textit{scp-переменных'}, соответствует сгенерированная \textit{sc-константа}.

Принадлежность некоторого действия множеству \textit{scp-процессов} гарантирует тот факт, что в декомпозиции данного действия будут присутствовать только знаки элементарных действий (\textit{scp-операторов}), которые может интерпретировать реализация \textit{Абстрактной scp-машины}.}
\scnrelfromvector{пример выполнения}{\scgfileitem{figures/sd_scp/process_example.png}\\
	\scnaddlevel{1}
\scnexplanation{Осуществляется вызов \textit{scp-программы}. Генерируется соответствующий \textit{scp-процесс}. Происходит инициирование начального оператора scp-процесса \textit{Operator1}}
	\scnaddlevel{-1}
;\scgfileitem{figures/sd_scp/process_example2.png}\\
\scnaddlevel{1}
\scnexplanation{Оператор \textit{Operator1} оказался безуспешно выполненным. Производится инициирование \textit{scp-оператора генерации трёхэлементной конструкции} \scnbigspace \textit{Operator2}}
\scnaddlevel{-1}
;\scgfileitem{figures/sd_scp/process_example3.png}
\scnaddlevel{1}
\scnexplanation{Оператор \textit{Operator2} выполнился. Производится инициирование \textit{scp-оператора завершения выполнения программы} \scnbigspace \textit{Operator3}}
\scnaddlevel{-1}
;\scgfileitem{figures/sd_scp/process_example4.png}\\
\scnaddlevel{1}
\scnexplanation{Оператор \textit{Operator3} выполнился. Выполнение \textit{scp-процесса} завершается.}
\scnaddlevel{-1}
}}
\scnaddlevel{1}
\scnexplanation{В приведенном примере последовательно показаны состояния \textit{scp-процесса}, соответствующего \textit{scp-программе}, добавляющей заданный элемент в заданное множество, если он там ранее не содержался. В примере предполагается, что рассматриваемый элемент (\textit{Element1}) изначально не содержится во множестве (\textit{Set1}).}
\scnaddlevel{-1}


\scnheader{scp-оператор}
\scnrelto{включение}{действие в sc-памяти}
\scnrelto{семейство подмножеств}{атомарный тип scp-оператора}
\scnexplanation{Каждый \textbf{\textit{scp-оператор}} представляет собой некоторое элементарное \textit{действие в sc-памяти}. Аргументы \textit{scp-оператора} будем называть операндами. Порядок операндов указывается при помощи соответствующих ролевых отношений (\textit{1’}, \textit{2’}, \textit{3’} и так далее). Операнд, помеченный ролевым отношением \textit{1’}, будем называть первым операндом, помеченный ролевым отношением \textit{2’} – вторым операндом, и т.д. Тип и смысл каждого операнда также уточняется при помощи различных подклассов отношения \textit{scp-операнд'}. В общем случае операндом может быть любой \textit{sc-элемент}, в том числе, знак какой-либо \textit{scp-программы}, в том числе самой программы, содержащей данный оператор.

Каждый \textbf{\textit{scp-оператор}} должен иметь один и более операнд, а также указание того \textbf{\textit{scp-оператора}} (или нескольких), который должен быть выполнен следующим. Исключение их данного правила составляет \textit{scp-оператор завершения выполнения программы}, который не содержит ни одного операнда и после выполнения которого никакие \textit{scp-операторы} в рамках данной программы выполняться не могут.}


\scnheader{атомарный тип scp-оператора}
\scnexplanation{Каждый \textbf{\textit{атомарный тип scp-оператора}} представляет собой класс \textit{scp-операторов}, который не разбивается на более частные, и, соответственно, интерпретируется реализацией \textit{Aбстрактной scp-машины}.}


\scnheader{начальный оператор'}
\scnrelto{включение}{1'}
\scnexplanation{Ролевое отношение \textbf{\textit{начальный оператор'}} указывает в рамках декомпозиции соответствующего \textit{scp-программе} \textit{scp-процесса} те \textit{scp-операторы}, которые должны быть выполнены в первую очередь, т.е. те, с которых собственно начинается выполнение \textit{scp-процесса}.}


\scnheader{параметр scp-программы'}
\scnrelto{включение}{аргумент действия'}
\scnsubdividing{in-параметр’;out-параметр’}
\scnexplanation{Ролевое отношение \textbf{\textit{параметр scp-программы'}} связывает знак соответствующего \textit{scp-программе} \textit{scp-процесса} с его аргументами.}


\scnheader{in-параметр’}
\scnexplanation{Параметры типа \textbf{\textit{in-параметр'}} хоть и соответствуют \textit{переменным scp-программы’}, не могут менять значение в процессе ее интерпретации. Фиксированное значение переменной устанавливается при создании уникальной копии \textit{scp-программы} (\textit{scp-процесса}) для ее интерпретации, и, таким образом, соответствующая \textit{scp-переменная'} на момент начала ее интерпретации становится \textit{scp-константой'} в рамках каждого \textit{scp-оператора}, в котором встречалась данная \textit{scp-переменная'}. Использование \textit{in-параметров} можно рассматривать по аналогии с использованием варианта механизма передачи по значению в традиционных языках программирования, с тем условием, что значение локальной переменной в рамках дочерней программы не может быть изменено.}


\scnheader{out-параметр’}
\scnexplanation{Параметры типа \textbf{\textit{out-параметр'}} соответствуют \textit{переменным scp-программы'} и обладают всеми теми же соответствующими свойствами. Чаще всего предполагается, что значение данного параметра необходимо родительской \textit{scp-программе}, содержащей оператор вызова текущей \textit{scp-программы}. При этом на момент начала интерпретации в качестве параметра дочернему процессу передается непосредственно узел, обозначающий переменную (а точнее, ее уникальную копию в рамках процесса) родительского процесса. Указанная переменная может при необходимости иметь значение, либо не иметь. После завершения и во время интерпретации дочернего процесса родительский процесс по-прежнему может работать с переменной, переданной в качестве \textit{out-параметра'}, при необходимости просматривая или изменяя ее значение. Использование out-параметра можно рассматривать по аналогии с использованием механизма передачи по ссылке в традиционных языках программирования.}


\scnendstruct

\end{SCn}
