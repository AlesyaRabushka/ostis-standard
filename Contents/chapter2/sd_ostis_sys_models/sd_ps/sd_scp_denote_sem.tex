\begin{SCn}

\scnsectionheader{\currentname}

\scnstartsubstruct

\scnheader{Предметная область денотационной семантики языка SCP}
\scniselement{предметная область}
\scnsdmainclasssingle{***}
\scnsdclass{***}

\scnauthorcomment{По scp-операторам есть подробное описание буквально по каждому классу с примерами (некоторые устарели, но можно подправить), там методичка около 100 страниц, нужно это сюда и в какой степени}

\scnheader{scp-оператор}
\scnrelto{включение}{действие в sc-памяти}
\scnrelto{семейство подмножеств}{атомарный тип scp-оператора}
\scnsubdividing{scp-оператор генерации конструкций\\
    \scnaddlevel{1}
    \scnsubdividing{scp-оператор генерации конструкции по произвольному образцу;scp-оператор генерации пятиэлементной конструкции;scp-оператор генерации трехэлементной конструкции;scp-оператор генерации одноэлементной конструкции}
    \scnaddlevel{-1}
;scp-оператор ассоциативного поиска конструкций\\
    \scnaddlevel{1}
    \scnsubdividing{scp-оператор поиска конструкции по произвольному образцу;scp-оператор поиска пятиэлементной конструкции с формированием множеств;scp-оператор поиска трехэлементной конструкции с формированием множеств;scp-оператор поиска пятиэлементной конструкции;scp-оператор поиска трехэлементной конструкции}
    \scnaddlevel{-1}
;scp-оператор удаления конструкций\\
    \scnaddlevel{1}
    \scnsubdividing{scp-оператор удаления множества элементов трехэлементной конструкции;scp-оператор удаления одноэлементной конструкции;scp-оператор удаления пятиэлементной конструкции;scp-оператор удаления трехэлементной конструкции}
    \scnaddlevel{-1}
;scp-оператор проверки условий\\
    \scnaddlevel{1}
    \scnsubdividing{scp-оператор сравнения числовых содержимых файлов;scp-оператор проверки равенства числовых содержимых файлов;scp-оператор проверки совпадения значений операндов;scp-оператор проверки наличия содержимого у файла;scp-оператор проверки наличия значения у переменной;scp-оператор проверки типа sc-элемента}
    \scnaddlevel{-1}
;scp-оператор управления значениями операндов\\
    \scnaddlevel{1}
    \scnsubdividing{scp-оператор удаления значения переменной;scp-оператор присваивания значения переменной}
    \scnaddlevel{-1}
;scp-оператор управления scp-процессами\\
    \scnaddlevel{1}
    \scnsubdividing{scp-оператор удаления значения переменной;scp-оператор завершения выполнения программы;конъюнкция предшествующих scp-операторов;scp-оператор ожидания завершения выполнения множества scp-программ;scp-оператор ожидания завершения выполнения scp-программы;scp-оператор асинхронного вызова подпрограммы}
    \scnaddlevel{-1}
;scp-оператор управления событиями\\
    \scnaddlevel{1}
    \scnreltoset{разбиение}{scp-оператор ожидания события}
    \scnaddlevel{-1}
;scp-оператор обработки содержимых файлов\\
    \scnaddlevel{1}
    \scnsubdividing{scp-оператор вычисления арксинуса числового содержимого файла;scp-оператор вычисления арккосинуса числового содержимого файла;scp-оператор деления числовых содержимых файлов;scp-оператор умножения числовых содержимых файлов;scp-оператор вычитания числовых содержимых файлов;scp-оператор сложения числовых содержимых файлов;scp-оператор вычисления тангенса числового содержимого файла;scp-оператор вычисления косинуса числового содержимого файла;scp-оператор вычисления синуса числового содержимого файла;scp-оператор вычисления логарифма числового содержимого файла;scp-оператор возведения числового содержимого файла в степень;scp-оператор удаления содержимого файла;scp-оператор копирования содержимого файла;scp-оператор нахождения остатка от деления числовых содержимых файлов;scp-оператор нахождения целой части от деления числовых содержимых файлов;scp-оператор вычисления арктангенса числового содержимого файла;scp-оператор перевода в верхний регистр строкового содержимого файла;scp-оператор перевода в верхний регистр строкового содержимого файла;scp-оператор замены определенной части строкового содержимого файла на содержимое указанной файла;scp-оператор проверки совпадения конца строкового содержимого файла со строковом содержимым другого файла;scp-оператор проверки совпадения начальной части строкового содержимого файла со строковом содержимым другого файла;scp-оператор получения части строкового содержимого файла по индексам;scp-оператор поиска строкового содержимого файла в строковом содержимом другого файла;scp-оператор вычисления длины строкового содержимого файла;scp-оператор разбиения строки на подстроки;scp-оператор лексикографического сравнения строковых содержимых файлов;scp-оператор проверки равенства строковых содержимых файлов}
    \scnaddlevel{-1}
;scp-оператор управления блокировками\\
    \scnaddlevel{1}
    \scnsubdividing{scp-оператор снятия всех блокировок данного scp-процесса;scp-оператор снятия блокировки с sc-элемента;scp-оператор установки полной блокировки на sc-элемент;scp-оператор установки блокировки на изменение sc-элемента;scp-оператор установки блокировки на удаление sc-элемента;scp-оператор снятия блокировки со структуры;scp-оператор установки полной блокировки на структуру;scp-оператор установки блокировки на изменение структуры;scp-оператор установки блокировки на удаление структуры}}

\scnresetlevel

\scnheader{scp-операнд’}
\scnrelto{включение}{аргумент действия'}
\scniselement{неосновное понятие}
\scniselement{ролевое отношение}
\scnsubdividing{scp-константа';scp-переменная'}
\scnsubdividing{scp-операнд с заданным значением';scp-операнд со свободным значением'}
\scnsubdividing{константный sc-элемент';переменный sc-элемент'}
\scnrelfromlist{включение}{формируемое множество'\\
    \scnaddlevel{1}
    \scnsubdividing{формируемое множество 1';формируемое множество 2';формируемое множество 3';формируемое множество 4';формируемое множество 5'}
    \scnaddlevel{-1}
;удаляемый sc-элемент';тип sc-элемента'\\
    \scnaddlevel{1}
    \scnsubdividing{sc-узел'\\
        \scnaddlevel{1}
        \scnsubdividing{структура';отношение'\\
            \scnaddlevel{1}
            \scnrelfrom{включение}{ролевое отношение'}
            \scnaddlevel{-1}
        ;класс'}
        \scnaddlevel{-1}
    ;sc-дуга'\\
        \scnaddlevel{1}
        \scnsubdividing{sc-дуга общего вида';sc-дуга принадлежности'\\
            \scnaddlevel{1}
            \scnrelfrom{включение}{sc-дуга основного вида'}
            \scnsubdividing{позитивная sc-дуга принадлежности';негативная sc-дуга принадлежности';нечеткая sc-дуга принадлежности'}
            \scnsubdividing{временная sc-дуга принадлежности';постоянная sc-дуга принадлежности'}
            \scnaddlevel{-1}}
        \scnaddlevel{-1}
    ;sc-ребро';файл'}
    \scnaddlevel{-1}}
\scnexplanation{Ролевое отношение \textit{scp-операнд’} является неосновным понятием и указывает на принадлежность аргументов \textit{scp-оператору}. Помимо указания какого-либо класса \textit{scp-операндов’} порядок аргументов \textit{scp-оператора} дополнительно уточняется \textit{ролевыми отношениями 1'}, \textit{2'} и т. д.}

\scnheader{scp-константа'}
\scnexplanation{В рамках \textit{scp-программы} \textit{scp-константы'} явно участвуют в \textit{\mbox{scp-операторах}} в качестве элементов (в теоретико-множественном смысле) и напрямую обрабатываются при интерпретации \textit{scp-программы}. Константами в рамках \textit{scp-программы} могут быть \textit{sc-элементы} любого типа, как \textit{\mbox{sc-константы}}, так и \textit{sc-переменные}. Константа в рамках \textit{scp-программы} остается неизменной в течение всего срока интерпретации. Константа \textit{\mbox{scp-программы}} может быть рассмотрена как переменная, значение которой совпадает с самой переменной в каждый момент времени, и изменено быть не может. Таким образом, далее будем считать, что \textit{scp-константа'} и ее значение это одно и то же. Каждый \textit{in-параметр'} при интерпретации каждой конкретной копии \textit{scp-программы} становится \textit{scp-константой'} в рамках всех ее операторов, хотя в исходном теле данной программы в каждом из этих операторов он является \textit{scp-переменной’}.}

\scnheader{scp-переменная'}
\scnexplanation{В рамках \textit{scp-программы} textit{scp-переменные'} не обрабатываются явно при интерпретации, обрабатываются значения переменных. Каждая переменная \textit{scp-программы} может иметь одно значение в каждый момент времени, т. е. представляет собой ситуативный \textit{синглетон}, элементом которого является текущее значение \textit{scp-переменной'}. Значение каждой \textit{scp-переменной’} может меняться в ходе интерпретации \textit{scp-программы}. При этом интерпретатор при обработке \textit{scp-оператора} работает непосредственно со значениями \textit{\mbox{scp-переменных’}}, а не самими \textit{scp-переменными’} (которые также являются узлами той же семантической сети).}

\scnheader{scp-операнд с заданным значением'}
\scnexplanation{Значение операндов, помеченных ролевым отношением \textit{scp-операнд с заданным значением'}, считается заданным в рамках текущего \textit{scp-оператора}. Данное значение учитывается при выполнении \textit{scp-оператора} и остается неизменным после окончания выполнения \textit{scp-оператора}. Каждая \textit{scp-константа'} по умолчанию рассматривается как \textit{scp-операнд с заданным значением'}, в связи с чем явное использование данного ролевого отношения в таком случае является избыточным. В таком случае в качестве значения рассматривается непосредственно сам операнд. В случае если отношением \textit{\mbox{scp-операнд} с заданным значением'} помечена \textit{scp-переменная'}, то осуществляется попытка поиска значения для данной \textit{scp-переменной'} (ее элемента). Если попытка оказалась безуспешной, то возникает ошибка времени выполнения, которая должна быть обработана соответствующим образом.

Любой \textit{scp-операнд с заданным значением'} независимо от конкретного типа \textit{scp-оператора} может быть \textit{scp-переменной'}.}

\scnheader{scp-операнд со свободным значением'}
\scnexplanation{Значение операндов, помеченных ролевым отношением \textit{scp-операнд со свободным значением'}, считается свободным (не заданным заранее) в рамках текущего \textit{scp-оператора}. В начале выполнения \textit{scp-оператора} связь между \textit{scp-переменной'}, помеченной данным ролевым отношением, и ее элементом (значением) всегда удаляется. В результате выполнения данного оператора может быть либо сгенерировано новое значение \textit{scp-переменной'}, либо не сгенерировано, тогда \textit{scp-переменная'} будет считаться не имеющей значения. Ни одна \textit{scp-константа'} не может быть помечена как \textit{scp-операнд со свободным значением'}, поскольку константа не может изменять свое значение в ходе интерпретации \textit{scp-программы}.

Таблица \ref{table_operands_roles} показывает возможные сочетания различных ролевых отношений, указывающих роль операнда в рамках scp-оператора:

%\begin{table}[H]
%  \caption{Роли операндов в рамках scp-оператора}\label{table_operands_roles}
%\begin{tabularx}{\hsize}{| p{43mm} | X | X |}
%  \hline
%  \textbf{Тип значения}
%  & \multirow{2}{*}{\textbf{\shortstack[l]{scp-операнд с\\ заданным значением'}}} & \multirow{2}{*}{\textbf{\shortstack[l]{scp-операнд со\\ свободным значением'}}} \\
%  \cline{0-0}
%  \textbf{Константность} & & \\
%\hline
%\textbf{scp-константа'} & Разрешено, может быть опущено & Запрещено \\
%\hline
%\textbf{scp-переменная'} & Разрешено, значение останется неизменным & Разрешено, значение переменной будет изменено либо потеряно\\
%\hline
%\end{tabularx}
%\end{table}
}

\scnheader{тип \mbox{sc-элемента'}}
\scnexplanation{Ролевое отношение \textit{тип \mbox{sc-элемента'}} используется для уточнения типа \textit{sc-элемента}, выступающего в роли значения некоторого операнда. \textit{тип \mbox{sc-элемента'}} имеет смысл указывать только для операндов, помеченных как \textit{scp-операнд со свободным значением'}, тогда данное уточнение типа \textit{\mbox{sc-элемента}} будет использовано для сужения области поиска либо уточнения параметров генерации каких-либо конструкций. Значением \textit{scp-операндов с заданным значением'} является конкретный, известный на момент начала выполнения \textit{scp-оператора sc-элемент} с конкретным типом, не зависящим от указания \textit{типа sc-элемента'}, в связи с чем использование ролевого отношения \textit{тип sc-элемента'} в данном случае является некорректным.

Допускается использование комбинаций семантически непротиворечащих друг другу подмножеств указанного отношения. Например, допускается комбинация \textit{константный sc-элемент'} и \textit{sc-дуга общего вида'}, но не допускается комбинация \textit{sc-узел'} и \textit{sc-дуга'}.}

\scnheader{sc-дуга основного вида'}
\scneq{(константный sc-элемент' $\cap$ позитивная sc-дуга принадлежности' $\cap$ постоянная sc-дуга принадлежности')}

\scnheader{формируемое множество'}
\scnexplanation{Ролевое отношение \textit{формируемое множество'} используется для указания того факта, что в результате выполнения \textit{scp-оператора} должно быть сформировано либо дополнено некоторое множество \textit{sc-элементов}, являющееся значением одного из операндов данного \textit{scp-оператора}. При этом если данный операнд помечен как \textit{scp-операнд со свободным значением'}, то множество будет сформировано с нуля (сгенерирован новый \textit{sc-элемент}, обозначающий данное множество), в противном случае уже существующее множество может быть дополнено. Использование данного ролевого отношения предполагает, что при его отсутствии множество бы не формировалось, а значением указанного операнда стал бы произвольный \textit{sc-элемент} из данного множества. 

Ролевое отношение \textit{формируемое множество'} без уточнения порядкового номера используется только в \textit{scp-операторах обработки произвольных конструкций}. Для явного указания номера операнда, которому соответствует \textit{формируемое множество'}, используются подмножества данного ролевого отношения, аналогичные ролевым отношениям, задающим порядок элементов в ориентированном множестве (\textit{1', 2', 3'} и т. д.), например \textit{формируемое множество 1'}, \textit{формируемое множество 2'} и т. д. Указанные ролевые отношения используются только в \textit{scp-операторах поиска конструкций с формированием множеств}.}

\scnheader{удаляемый sc-элемент'}
\scnexplanation{Ролевое отношение \textit{удаляемый sc-элемент'} используется для указания тех операндов, значение которых должно быть удалено в процессе выполнения \textit{scp-операторов удаления}. Данным ролевым отношением может быть помечен как \textit{scp-операнд с заданным значением'}, так и \textit{scp-операнд со свободным значением'}. При этом удаляемым \textit{sc-элементом} может быть как \textit{scp-константа'}, так и \textit{scp-переменная'} (в случае \textit{scp-переменной'} удаляется не только связка принадлежности между этой \textit{scp-переменной'} и ее значением, но и непосредственно сам \textit{sc-элемент}, являющийся значением).}

\scnendstruct

\end{SCn}
