\begin{SCn}
	
\scnsectionheader{\currentname}
	
\scnstartsubstruct
	
\scnheader{Предметная область искусственных нейронных сетей}
\scniselement{предметная область}
\scnsdmainclasssingle{искусственная нейронная сеть}

\scnrelfromset{частная предметная область}{
Предметная область ИНС с заданным направлением связей\\
    \scnaddlevel{1}
    \scnrelfromset{частная предметная область}{
    Предметная область ИНС с прямым связями\\
        \scnaddlevel{1}
        \scnrelfromset{частная предметная область}{
        Предметная область персептронов\\
            \scnaddlevel{1}
            \scnrelfromset{частная предметная область}{
            Предметная область персептронов Розенблатта
            ;Предметная область персептронов Румельхарта
            ;Предметная область автоэнкодерных ИНС
            }
            \scnaddlevel{-1}
        ;Предметная область ИНС радиально-базисных функций
        ;Предметная область машин опорных векторов
        }
        \scnaddlevel{-1}
    ;Предметная область ИНС с обратными связями\\
        \scnaddlevel{1}
        \scnidtf{Предметная область рекуррентных ИНС}
        \scnrelfromset{частная предметная область}{
        Предметная область ИНС Джордана
        ;Предметная область ИНС Элмана
        ;Предметная область LSTM-элементов
        ;Предметная область GRU-элементов
        }
        \scnaddlevel{-1}
    }
    \scnaddlevel{-1}
;Предметная область обучения ИНС\\
    \scnaddlevel{1}
    \scnrelfromset{частная предметная область}{
    Предметная область ИНС, обучающихся с учителем
    ;Предметная область ИНС, обучающихся без учителя\\
        \scnaddlevel{1}
        \scnrelfromset{частная предметная область}{
        Предметная область обучающихся автоэнкодерных ИНС
        ;Предметная область ИНС глубокого доверия
        ;Предметная область генеративно-состязательных ИНС
        ;Предметная область самоорганизующихся карт Кохонена
        ;Предметная область ИНС Хопфилда
        ;Предметная область подкрепляющего обучения ИНС
        }
        \scnaddlevel{-1}
    }
    \scnaddlevel{-1}
;Предметная область топологий ИНC\\
    \scnaddlevel{1}
    \scnrelfromset{частная предметная область}{
    Предметная область полносвязных ИНC
    ;Предметная область многослойных ИНC
    ;Предметная область слабосвязных ИНC
    }
    \scnaddlevel{-1}
;Предметная область задач, решаемых с помощью ИНС\\
    \scnaddlevel{1}
    \scnrelfromset{частная предметная область}{
    Предметная область ИНС, решающих задачу классификации
    ;Предметная область ИНС, решающих задачу аппроксимации
    ;Предметная область ИНС, решающих задачу управления
    ;Предметная область ИНС, решающих задачу фильтрации
    ;Предметная область ИНС, решающих задачу детекции
    ;Предметная область ИНС, решающих задачу с ассоциативной памятью
    }
    \scnaddlevel{-1}
;Предметная область интеграции ИНС с базой знаний
}
\scnheader{Искусственная нейронная сеть}
\scnaddlevel{1}
\scnidtf{ИНС}
\scnidtf{нейронная сеть}
\scnidtf{биологически инспирированная модель естественных нейронных цепей, обладающая способностью к решению определенной задачи после выполнения процедуры обучения}
\scnaddlevel{-1}

\scnheader{Обучение}
\scnaddlevel{1}
\scnidtf{процесс итеративного изменения параметров искусственной нейронной сети, минимизирующий некоторую функцию ошибки}
\scnaddlevel{-1}

\scnheader{Функция ошибки}

\scnheader{Параметры нейронной сети}
\scnendstruct \scnendcurrentsectioncomment

\end{SCn}