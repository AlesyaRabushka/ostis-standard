\begin{SCn}

\scnsectionheader{Предметная область и онтология SCs-кода}

\scnstartsubstruct

\scnheader{Предметная область SCs-кода}
\scnsdmainclasssingle{SCs-код}
\scnsdclass{***}
\scnsdrelation{***}

\scnheader{SCs-код}
\scnidtf{Semantic Code string}
\scnexplanation{SCs-код - строковый (линейный) вариант представления SC-кода. Предназначен для представления sc-графов (текстов SC-кода) в виде последовательностей символов, которые могут быть отредактированы как при помощи стандартных текстовых редакторов, так и при помощи специализированного sc.s-редактора. Следовательно, одним из требований, предъявляемых к SCs-коду, помимо полноты и непротиворечивости, является возможность набора исходных текстов без помощи специализированного редактора, используя только символы стандартной клавиатуры. В отличие от SCn-кода, также являющегося текстовым вариантом отображения sc-графов, в SCs-коде форматирование не имеет значения, все предложения могут быть записаны в одну строку.

В отличие от SCg-кода, в SCs-коде нет специальных обозначений для указания структурных типов sc-узлов, однако есть изображения sc-коннекторов, соответствующие ядру и расширению SCg-кода, а также изображения sc-коннекторов специального типа, для которых нет соответствия в SCg-коде. С формальной точки зрения SCs-код - множество sc.s-текстов. Каждый sc.s-текст представляет собой последовательность sc.s-предложений, каждое из которых оканчивается разделителем ;; (двойная точка с запятой). Множество sc.s-предложений разбивается на множество простых и множество сложных sc.s-предложений. Каждое сложное sc.s-предложение содержит в своем составе встроенные предложения, ограниченные ограничителем (*...*).

В рамках sc.s-текста любого уровня допустимо использование комментариев следующего вида:

// однострочный комментарий

/* многострочный комментарий */ 

В начале файла, содержащего sc.s-текст настоятельно рекомендуется указывать уровень и версию используемого SCs-кода. 
Для этого используются комментарий специального вида:
/* SCs\_code\textless" версия "\textgreater.Level\textless" номер уровня "\textgreater*/

Например:

/* SCs\_code0.1.0.Level 6 */

Следует отметить, что комментарий вида /*…*/, использованный внутри sc.s-рамки (см. соотв. раздел) не опускается при разборе sc.s-текста, и становится частью содержимого соответствующей sc-ссылки. Например при разборе предложения

X => r1: [TEXT1/*COMMENT*/TEXT2];;

Будет создана sc-ссылка с содержимым TEXT1/*COMMENT*/TEXT2, а не TEXT1TEXT2.

В SCs-коде условно выделяются 7 расширений. Все расширения равнозначны по возможностям представления знаний, однако расширения более высоких уровней описывают sc-графы более лаконично и удобно. Введение расширений призвано облегчить работу разработчиков баз знаний при наборе sc.s-текстов.
}

\scnheader{sc.s-предложение}
\scnsubdividing{простое sc.s-предложение\\
\scnaddlevel{1}
\scnidtf{sc.s-предложение, не содержащее встроенных sc.s-предложений}
\scnaddlevel{-1}
;сложное sc.s-предложение\\
\scnaddlevel{1}
\scnidtf{sc.s-предложение, содержащее встроенные sc.s-предложения}
\scnaddlevel{-1}}
\scnsubdividing{sc.s-предложение уровня 1;sc.s-предложение уровня 2;sc.s-предложение уровня 3;sc.s-предложение уровня 4;sc.s-предложение уровня 5;sc.s-предложение уровня 6;sc.s-предложение уровня 7}

\scnheader{sc.s-предложение уровня 1}
\scnexplanation{sc.s-предложение уровня 1 - sc.s-предложение, содержащее только sc.s-разделители инцидентности и имена sc-элементов.}
\scntext{пример}{включение* | sc-arc-main\#arc2 | sc\_arc\_common\#pair1;;}
\scntext{пример}{Четырехугк(ТчкА;ТчкВ;ТчкС;ТчкD) | sc\_arc\_common\#pair1 | Отр(ТчкВ;ТчкС);;}

\scnheader{sc.s-предложение уровня 2}
\scntext{пример}{сторона* $\ni$ (Четырехугк(ТчкА;ТчкВ;ТчкС;ТчкD) =\textgreater~Отр(ТчкВ;ТчкС) );;}
\scntext{пример}{включение* $\ni$ ( Треугк(ТчкВ;ТчкС;ТчкD) =\textgreater~Отр(ТчкВ;ТчкС) );;}

\scnheader{sc.s-предложение уровня 3}
\scntext{пример}{Четырехугк(ТчкА;ТчкВ;ТчкС;ТчкD) =\textgreater~ сторона* : Отр(ТчкВ;ТчкС);;}
\scntext{пример}{треугольник =\textgreater идентификатор* : основной англоязычный идентификатор*: [triangle];;}

\scnheader{sc.s-предложение уровня 4}
\scncomment{sc.s-предложение уровня 4 можно считать описанием семантической окрестности радиуса 1 для заданного sc-элемента, т.е описанием известных связей заданного sc-элемента с другими смежными ему sc-элементами.}
\scntext{пример}{Четырехугк(ТчкА;ТчкВ;ТчкС;ТчкD) =\textgreater~ сторона*: включение*: Отр(ТчкВ;ТчкС);\\ Отр(ТчкС;ТчкD);;}


\scnendstruct

\end{SCn}