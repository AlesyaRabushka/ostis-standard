\begin{SCn}

\scnsectionheader{\currentname}

\scnstartsubstruct

\scnheader{Предметная область базовой денотационной семантики SC-кода}
\scnidtf{Предметная область сущностей}
\scnsdmainclasssingle{сущность}
\scnsdclass{...}

\scnheader{SC-код}
\scnexplanation{Денотационная семантика любой знаковой конструкции (в том числе и \textit{sc-текста}) с формальной точки зрения -- это соответствие (морфизм) между множеством всех знаков, входящих в знаковую конструкцию, и множеством денотатов этих знаков (т. е. сущностей, обозначаемых этими знаками), а также между множеством всех семантически значимых (семантически интерпретируемых) связей, связывающих знаки, и множеством соответствующих им связей, связывающих либо денотаты всех указанных знаков, либо денотаты некоторых из указанных знаков непосредственно с самими остальными знаками из числа указанных знаков.

Рассмотрим денотационную семантику \textit{sc-элементов}, принадлежащих разным \textit{синтаксически выделяемым классам sc-элементов}, т.е. имеющих разные синтаксические метки. 

Если \textit{sc-элемент}, имеет метку \textit{\textbf{sc-элемента}}, то он может обозначать \uline{любую} описываемую \textbf{сущность}. 

Если \textit{sc-элемент} имеет метку \textit{\textbf{sc-коннектора}}, который инцидентен \textit{sc-элементу} \textit{\textbf{ei}} и \textit{sc-элементу} \textit{\textbf{ej}}, то он, с одной стороны, является знаком \textit{\textbf{пары}} \{{\textit{\textbf{ei}}, \textit{\textbf{ej}}}\}, а, с другой стороны, является моделью (отражением, описанием) связи либо между денотатом \textit{sc-элемента} \textit{\textbf{ei}} и денотатом \textit{sc-элемента} \textit{\textbf{ej}}, либо между денотатом \textit{sc-элемента} \textit{\textbf{ei}} и самим \textit{sc-элементом} \textit{\textbf{ej}}, либо между денотатом \textit{sc-элемента} \textit{\textbf{ej}} и самим \textit{sc-элементом} \textbf{\textit{ei}}.

Если \textit{sc-элемент} имеет метку \textit{\textbf{sc-узла}}, то он обозначает \textbf{сущность, не являющуюся парой}.

Если \textit{sc-элемент} имеет метку \textit{\textbf{sc-ребра}}, которое инцидентно sc-элементу \textit{\textbf{ei}} и \textit{sc-элементу} \textit{\textbf{ej}}, то он, с одной стороны, является знаком \textit{\textbf{неориентированной пары}} \mbox{\{{\textit{\textbf{ei}}, \textit{\textbf{ei}}}\}}, а с другой стороны, является моделью (отражением, описанием) связи либо между денотатом \textit{sc-элемента} \textit{\textbf{ei}} и денотатом \textit{sc-элемента} \textit{\textbf{ej}}, либо между денотатом \textit{sc-элемента} \textit{\textbf{ei}} и самим \textit{sc-элементом} \textit{\textbf{ej}}, либо между денотатом \textit{sc-элемента} \textit{\textbf{ej}} и самим \textit{sc-элементом} \textit{\textbf{ei}}, либо непосредственно между самими \textit{sc-элементами} \textit{\textbf{ei}} и \textit{\textbf{ej}}.

Если \textit{sc-элемент} имеет метку \textbf{\textbf{sc-дуги}}, которая выходит из \textit{sc-элемента} \textit{\textbf{ei}} и входит в \textit{sc-элемент} \textit{\textbf{ej}}, то он, с одной стороны, является знаком \textit{\textbf{ориентированной пары}} \mbox{<\textit{\textbf{ei}}, \textit{\textbf{ej}}>}, а с другой стороны, является моделью (отражением, описанием) связи либо между денотатом \textit{sc-элемента} \textit{\textbf{ei}} и денотатом \textit{sc-элемента} \textit{\textbf{ej}}, либо между денотатом \textit{sc-элемента} \textit{\textbf{ei}} и самим \textit{sc-элементом} \textit{\textbf{ej}}, либо между денотатом \textit{sc-элемента} \textit{\textbf{ej}} и самим \textit{sc-элементом} \textit{\textbf{ei}}, либо непосредственно между самими sc-элементами \textit{\textbf{ei}} и \textit{\textbf{ej}}.

Если \textit{sc-элемент} имеет метку \textit{\textbf{базовой sc-дуги}}, которая выходит из \textit{sc-элемента} \textit{\textbf{ei}} и входит в \textit{sc-элемент} \textit{\textbf{ej}}, то он, с одной стороны, является знаком ориентированной \textit{\textbf{пары константной позитивной постоянной принадлежности}} <\textit{\textbf{ei}}, \textit{\textbf{ej}}>, а, с другой стороны, является моделью (отражением, описанием) связи между множеством, которое обозначается \textit{sc-элементом} \textit{\textbf{ei}}, и \textit{sc-элементом} \textit{\textbf{ej}}, который является одним из элементов указанного множества.

Теперь перейдем к рассмотрению денотационной семантики пар \textit{\textbf{инцидентности}} \textit{\textbf{sc-коннек-торов}}. Напомним, что каждый \textit{sc-коннектор} семантически трактуется как знак \textit{\textbf{пары}} \textit{sc-элементов}, инцидентных этому \textit{sc-коннектору}. Соответственно этому каждая \textit{пара инцидентности sc-коннектора}, \uline{не являясь sc-элементом}, семантически трактуется как модель (отражение, описание) связи между \textit{парой} sc-элементов, обозначаемой этим \textit{sc-коннектором}, и одним из двух элементов этой \textit{пары}. При этом принадлежность указанного \textit{sc-элемента} указанной \textit{паре} может носить:

\begin{scnitemize}
    \item \textit{\textbf{константный}} либо \textit{\textbf{переменный}} характер в зависимости от константности или переменности указанного \textit{sc-коннектора};
    \item \textit{\textbf{стационарный}} (постоянный) либо \textit{\textbf{нестационарный}} (ситуативный) характер в зависимости от стационарности или нестационарности указанного \textit{sc-коннектора}.
\end{scnitemize}

Аналогичным образом задается денотационная семантика пар \textit{\textbf{инцидентности входящих sc-дуг}}. Каждая такая пара инцидентности трактуется как модель связи между \textbf{\textit{ориентированной парой}}, обозначаемой \textit{sc-дугой} и вторым компонентом этой пары (т.е. \textit{sc-элементом}, в которой \textit{sc-дуга} входит). И аналогично парам \textit{инцидентности sc-коннекторов} пары \textit{инцидентности входящих sc-дуг} могут иметь \textit{константный} и \textit{переменный} характер, а также \textit{стационарный} и \textit{нестационарный} в зависимости от характера соответствующей \textit{sc-дуги}.

Формальное описание денотационной семантики \textit{SC-кода} средствами самого \textit{SC-кода} осуществляется в виде иерархической системы \textbf{\textit{формальных онтологий}} верхнего уровня, представленных в виде текстов \textit{SC-кода}. В базе знаний \textit{\textbf{Метасистемы IMS.ostis}} все эти онтологии представлены~\cite{IMS}.
}

\scnheader{сущность}
\scnidtf{sc-элемент}
\scnsubdividing{sc-константа;sc-переменная\\
\scnaddlevel{1} 
\scnidtf{знак произвольной сущности из некоторого множества возможных значений}
\scnaddlevel{-1}}

\scnsubdividing{стационарная сущность\\
\scnaddlevel{1} 
\scnidtf{постоянная сущность}
\scnaddlevel{-1} 
;sc-переменная\\
\scnaddlevel{1} 
\scnidtf{нестационарная сущность}
\scnidtf{сущность, изменяющаяся во времени}
\scnsuperset{временная сущность}
\scnaddlevel{1} 
\scnidtf{временно существующая сущность}
\scnaddlevel{-2} }

\scnsubdividing{материальная сущность;терминальная абстрактная сущность;файл\\
\scnaddlevel{1}
\scnaddhind{-1}
\scnidtf{первичный (при восприятии) или конечный (при отображении) электронный образ внешней информационной конструкции}
\scnaddlevel{-1} 
;множество\\
\scnaddlevel{1}
\scnidtf{множество sc-элементов}
\scnsubdividing{связь;структура;класс\\
\scnaddlevel{1}
\scnsubdividing{класс терминальных сущностей;отношение\\
\scnaddlevel{1}
\scnidtf{класс связей}
\scnaddlevel{-1}
;класс классов\\
\scnaddlevel{1}
\scnsuperset{параметр}
\scnaddlevel{-1}
;класс структур}
\scnaddlevel{-1}
}
\scnaddlevel{-1}
}

\scnheader{связь}
\scnidtf{связка}
\scnsubdividing{пара\\
\scnaddlevel{1}
\scnidtf{бинарная связь}
\scnsuperset{sc-коннектор}
\scnaddlevel{1}
\scnnote{Некоторые пары sc-элементов в некоторые периоды времени могут быть не оформлены синтаксически как коннекторы, но такое преобразование обязательно происходит.}
\scnaddlevel{-2}
;небинарная связь}

\scnsubdividing{неориентированная связь\\
\scnaddlevel{1}
\scnsuperset{неориентированная пара}
\scnaddlevel{-1}
;ориентированная связь\\
\scnaddlevel{1}
\scnsuperset{ориентированная пара}
\scnaddlevel{-1}
}

\scnsubdividing{константная связь\\
\scnaddlevel{1}
\scneq{(связь $\cap$ sc-константа)}
\scnaddlevel{-1}
;переменная связь\\
\scnaddlevel{1}
\scneq{(связь $\cap$ sc-переменная)}
\scnsuperset{sc-переменная, значениями которой являются константные связи}
\scnsuperset{sc-переменная, значениями которой являются переменные связи}
\scnaddlevel{-1}
}

\scnheader{пара}
\scnidtf{обозначение двухмощного множества sc-элементов}
\scnsubdividing{неориентированная пара\\
\scnaddlevel{1}
\scnsuperset{sc-ребро}
\scnaddlevel{-1}
;ориентированная пара\\
\scnaddlevel{1}
\scnsuperset{sc-дуга}
\scnsuperset{пара принадлежности}
\scnaddlevel{-1}
}
\scnsubdividing{пара-петля\\
\scnaddlevel{1}
\scnidtf{пара, являющаяся петлей}
\scnidtf{пара, у которой инцидентные ей sc-элементы совпадают}
\scnidtf{пара, являющаяся мультимножеством}
\scnaddlevel{-1}
;пара, не являющаяся петлей}

\scnheader{пара принадлежности}
\addtocounter{hind}{1}
\scnidtf{связь, описывающая характер принадлежности некоторого sc-элемента некоторому множеству}
\addtocounter{hind}{-1}
\scnsubdividing{пара константной принадлежности\\
\scnaddlevel{1}
\scneq{(пара принадлежности $\cap$ sc-константа)}
\scnaddlevel{-1}
;пара переменной принадлежности\\
\scnaddlevel{1}
\scneq{(пара принадлежности $\cap$ sc-переменная)}
\scnaddlevel{-1}
}
\scnsubdividing{пара постоянной принадлежности\\
\scnaddlevel{1}
\scneq{(пара принадлежности $\cap$ стационарная сущность)} 
\scnidtf{пара стационарной принадлежности} 
\scnaddlevel{-1}
;пара временной принадлежности\\
\scnaddlevel{1}
\scneq{(пара принадлежности $\cap$ временная сущность)} 
\scnidtf{пара ситуативной принадлежности}
\scnaddlevel{-1}
}
\scnsubdividing{пара позитивной принадлежности\\
\scnaddlevel{1}
\scnidtf{пара действительной принадлежности} 
\scnaddlevel{-1}
;пара нечеткой принадлежности;
пара негативной принадлежности\\
\scnaddlevel{1}
\scnidtf{пара несуществующей принадлежности} 
\scnaddlevel{-1}
}

\scnsuperset{пара константной позитивной постоянной принадлежности}
\scnaddlevel{1}
\scneq{(пара позитивной принадлежности $\cap$ sc-константа $\cap$ стационарная сущность)}
\scnsuperset{базовая sc-дуга}
\scnaddlevel{-1}

\scnendstruct

\scnsourcecomment{Завершили представление фрагмента Предметной области базовой денотационной семантики SC-кода}

\end{SCn}

\begin{SCn}
\scnsectionheader{\currentname}
\scnsuperset{Онтология сущностей}
\scnaddlevel{1}
\scntext{детализация}{Нижеперечисленные онтологии уточняют (детализируют) понятия, введенные в \textit{\textbf{Онтологии сущностей}}.

В \textit{\textbf{Онтологии множеств}} (см. Раздел \textit{\nameref{sec:sd_sets}}) уточняется понятие \textit{множества} sc-элементов, рассматриваются различные классы множеств (конечные, бесконечные, счетные, континуальные, мультимножества, множества без кратных элементов), различные свойства (характеристики) и отношения, заданные на множествах (мощность множеств, включение, объединение, разбиение, пересечение и т.д.).

В \textit{\textbf{Онтологии отношений}} (см. Раздел \textit{\nameref{sec:sd_rels}}) рассматриваются такие понятия, как \textit{бинарное отношение, унарное отношение, тернарное отношение, класс связок одинаковой мощности, класс связок разной мощности, арность отношения, ориентированное, неориентированное отношение, ролевое отношение, атрибут отношения*, область определения отношения*, домен отношения по заданному атрибуту*, функция}, и т.д.

Для \textit{Онтологии отношений} вводится онтология более низкого уровня -- \textit{\textbf{Онтология бинарных отношений и соответствий}}, которая наследует все свойства отношений, описанных в \textit{Онтологии отношений}, уточняет понятие \textit{бинарного отношения} И рассматривает такие понятия, как \textit{транзитивное отношение, симметричное отношение, рефлексивное отношение, отношение эквивалентности, изоморфизм, гомоморфизм} и т.д.

Далее вводятся 
\begin{scnitemize}
    \item \textbf{\textit{Онтология параметров, величин и измерений}} (см. Раздел \textit{\nameref{sec:sd_params}}) 
    \item \textbf{\textit{Онтология структур}} (см. Раздел \textit{\nameref{sec:sd_structures}})
    \begin{scnitemizeii}
    \item \textit{\textbf{Онтология предметных областей}} (см. Раздел \textit{\nameref{sec:sd_sd}})
    \item \textit{\textbf{Онтология семантических окрестностей}} (см. Раздел \textit{\nameref{sec:sd_sem_neigh}})
    \item \textit{\textbf{Онтология баз знаний}} (см. Раздел \textit{\nameref{sec:sd_kb}})
    \end{scnitemizeii} 
    \item \textit{\textbf{Онтология логических формул}} (см. Раздел \textit{\nameref{sec:sd_logics}})
    \item \textit{\textbf{Онтология темпоральных сущностей}}, в которой рассматриваются такие понятия, как \textit{нестационарный параметр} (состояние), \textit{процесс, действие, ситуация, последовательность во времени*, темпоральная декомпозиция*} и др. (см. Раздел \textit{\nameref{sec:sd_temp_entities}})
    \begin{scnitemizeii}
    \item \textbf{\textit{Онтология действий, задач, планов, протоколов и методов}} (см. Раздел \textit{\nameref{sec:sd_actions}})
    \end{scnitemizeii}
    \item \textit{\textbf{Онтология внешних информационных конструкций}} (см. Раздел \textit{\nameref{sec:sd_files}})
\end{scnitemize}


Некоторые из \textit{онтологий}, представленных в \textit{SC-коде}, носят "общеобразовательный"\ характер. Это означает, что для качественного взаимопонимания между любыми субъектами (как пользователями, так и компьютерными системами), т.е. для качественной их семантической совместимости все эти "общеобразовательные"\ онтологии, причем в согласованном, унифицированном виде, должны знать \uline{все}(!). Иначе никакого взаимопонимания не будет.

Список \textit{онтологий} можно продолжать. Все \textit{онтологии} постоянно меняются (уточняются, совершенствуются). Важнейшим критерием качества иерархической системы \textit{онтологий} является стратифицированность способов решения задач, соответствующих различным \textit{онтологиям} -- для каждой решаемой задачи желательно априори знать, в рамках какой \textit{онтологии} она может быть решена.

Очевидно, что, кроме "общеобразовательных"\ \textit{онтологий} существует большое число профессиональных, специализированных \textit{онтологий}, согласованное представление и знание которых необходимо для взаимопонимания (совместимости) всех тех, кто работает в соответствующей профессиональной области. 

Таким образом, денотационная семантика \textit{SC-кода}, как и любого другого языка, претендующего на универсальность, отражает текущее состояние наших знаний и, следовательно, может изменяться. Очевидно, что наиболее интенсивно эти изменения происходят в специализированных и новых областях знаний.
}
\scnaddlevel{-1}

\end{SCn}