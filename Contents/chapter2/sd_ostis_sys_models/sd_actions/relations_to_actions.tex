\begin{SCn}

\bigskip
\scnfragmentcaptiontext{Отношения, заданные на множестве действий}

\scnheader{отношение, заданное на множестве*(действие)}
\scnidtf{отношение, заданное на множестве действий*}
\scnhaselement{\scnkeyword{поддействие*}}
	\scnaddlevel{1}
	\scnrelboth{обратное отношение}{наддействие*}
	\scnidtf{быть действием, являющимся частью заданного действия более высокого уровня иерархии*}
	\scnidtf{быть действием, направленным на решение задачи, которая является подзадачей по отношению к задаче, решение которой осуществляется заданным действием*}
	\scnsuperset{\scnkeyword{непосредственное поддействие}*}
		\scnaddlevel{1}
		\scnidtf{быть таким поддействием заданного действия, для которого не существует наддействия, которое было бы также поддействием заданного действия*}
		\scnaddlevel{-1}
	\scnaddlevel{-1}
\scnhaselement{\scnkeyword{последовательность действий}*}
\scnaddlevel{1}
\scnidtf{порядок выполнения (инициирования) действий*}
\scnidtf{передача управления от действия к действию*}
\scnidtf{goto*}
\scnidtf{Бинарное ориентированное отношение, каждая связка которого связывает два действия, первое из которых является действием, событие завершения которого является необходимым (но не обязательно достаточным) условием инициирования (начала выполнения) второго действия}
\scnnote{Связки Отношения \textit{последовательность действий*} могут иметь совпадающие первые компоненты. Это означает, что завершение действия может быть условием инициирования (передачи управления) сразу нескольким действиям, т.е. означает распараллеливание выполняемого сложного действия, состоящего из нескольких поддействий. Связки указанного Отношения могут иметь также совпадающие вторые компоненты. Это означает, что \uline{достаточным} условием инициирования действия, являющегося вторым компонентом указанных связок, является событие завершения всех непосредственно предшествующих действий.}
\scnnote{Безуспешно выполненное действие считается невыполненным и, следовательно, не может передать управление последующим действиям. Примерами таких действий являются действия, проверки наличия в текущий момент тех или иных ситуаций (условий). Если указанные ситуации (условия) альтернативны, то речь идет об условной передаче управления.}
\scnnote{Возможность безуспешного выполнения некоторых действий можно и нужно предусматривать при построении \textit{планов} выполнения сложных действий. Но при представлении \textit{протоколов}, описывающих то, как эти действия были выполнены на самом деле, все можно упростить -- можно удалить альтернативные "ветки"{} (цепочки) действий, которые следуют после безуспешно выполненных действий.}
\scnnote{Своего рода "штатным"{} вариантом синхронизации параллельно (одновременно) выполняемых или альтернативных "веток"{} (цепочек) с помощью Отношения \textit{последовательность действий*} является то, что инициирование действия осуществляется тогда и только тогда, когда завершается выполнение \uline{всех} непосредственно предшествующих ему действий. Для обеспечения более сложных вариантов синхронизации действий необходимо ввести два класса специальных \textit{действий}:
\begin{scnitemize}
\item \textit{и-синхронизация действий}
\item \textit{или-синхронизация действий}
\end{scnitemize}	
\scnaddlevel{-1}}

\scnheader{и-синхронизация действий}
\scnidtf{действие, которое успешно выполняется сразу после завершения выполнения всех непосредственно предшествующих действий и при этом ничего другого при выполнении этого действия не происходит}

\scnheader{или-синхронизация действий}
\scnidtf{действие, которое успешно выполняется сразу после завершения выполнения хотя бы одного непосредственно предшествующего действия}
	
\scnheader{субъект действия'}
\scnidtf{сущность, воздействующая на некоторую другую сущность в процессе некоторого действия'}
\scnidtf{сущность, создающая \uline{причину} изменений другой сущности (объекта действия)'}
\scnidtf{быть субъектом данного действия'}
\scnsuperset{субъект неосознанного воздействия'}
\scnsuperset{субъект осознанного воздействия'}
\scnaddlevel{1}
\scnidtf{субъект целенаправленного, активного воздействия'}
\scnaddlevel{-1}


\scnheader{объект действия'}
\scnidtf{аргумент действия'}
\scnidtf{сущность, на которую осуществляется воздействие в рамках заданного действия'}
\scnidtf{сущность, являющаяся в рамках заданного действия исходным условием (аргументов), необходимым для выполнения этого действия'}
\scnnote{Для разных действий количество объектов действий может быть различным.}

\scnheader{инструмент воздействия'}
\scnidtf{то, с помощью чего субъект осуществляет воздействие*}


\scnheader{продукт'}
\scnidtf{быть продуктом заданного действия*}
\scnidtf{результат*}
\scnidtf{"сухой"{} остаток*}
\scnidtf{то, ради чего может быть выполнено, выполняется или будет выполняться заданное действие*}
\scnnote{Продуктом действия может быть некоторая материальная сущность, некоторое множество (тираж) одинаковых материальных сущностей, некоторая информационная конструкция}
\scnrelboth{следует отличать}{цель*}
	\scnaddlevel{1}
	\scnidtf{спецификация продукта*}
	\scnnote{Следует также отмечать то, что является непосредственно результатом (продуктом) некоторого действия и то, что является предварительной (исходной, стартовой) спецификацией этого продукта.}
	\scnaddlevel{-1}

\bigskip
\scnfragmentcaptiontext{Отношения, заданные на множестве действий и связывающие действия с различного вида их спецификациями}
\bigskip

\scnheader{отношение, заданное на множестве*(действие)}
\scnhaselement{\scnkeyword{задача}*}
	\scnaddlevel{1}
	\scnidtf{формулировка задачи*}
	\scnidtf{спецификация действия, уточняющая то, \uline{что} должно быть сделано*}
\scnsubdividing{декларативная формулировка задачи*;процедурная формулировка задачи*}
\scnrelfrom{второй домен}{\scnkeyword{задача}}
	\scnaddlevel{1}
	\scnsuperset{задача обработки базы знаний}
	\scnsuperset{задача обработки файлов}
	\scnsuperset{задача, решаемая кибернетической системой во внешней среде}
	\scnsuperset{задача, решаемая кибернетической системой в собственной физической оболочке}
	\scnaddlevel{-1}
		\scnaddlevel{-1}
	
\scnhaselement{\scnkeyword{декларативная формулировка задачи}*}
\scnaddlevel{1}
\scnidtf{описание исходной ситуации и целевой ситуации специфицируемого действия*}
\scnexplanation{декларативная формулировка задачи включает в себя:
	\begin{scnitemize}
	\item связку отношения \textit{цель}*, связывающую специфицируемое действие с описанием целевой ситуации;
	\item само описание целевой ситуации;
	\item связку отношения \textit{исходная ситуация*}, связывающую специфицируемое действие с описанием исходной ситуации;
	\item непосредственно описание исходной ситуации;
	\item указание контекста (области решения) задачи.
	\end{scnitemize}
При этом указание и описание исходной ситуации может отсутствовать.}
\scnaddlevel{-1}

\scnhaselement{\scnkeyword{процедурная формулировка задачи}*}
\scnidtfexp{указание
	\begin{scnitemize}
	\item \textit{класса действий}, которому принадлежит специфицируемое \textit{действие}, а также 
	\item \textit{субъекта} или субъектов, выполняющих это действие (с дополнительным указанием роли каждого участвующего субъекта);
	\item \textit{объекта} или объектов, над которыми осуществляется действие (с указанием "роли"{} каждого такого объекта);
	\item используемых материалов;
	\item используемых инструментов (инструментальных средств);
	\item дополнительных темпоральных характеристик специфицируемого действия (сроки, длительность);
	\item приоритета (важности) специфицируемого действия.
	\end{scnitemize}}

\scnhaselement{\scnkeyword{исходная ситуация}*}
	\scnaddlevel{1}
\scnidtf{исходная ситуация, соответствующая заданному действию*}
\scnsubset{спецификация*} 
\scnrelfrom{второй домен}{ситуация}
\scnidtf{начальная ситуация*}
\scnidtf{описание того, что дано (что имеется) перед началом выполнения заданного (специфицируемого) действия*}
	\scnaddlevel{-1}

\scnhaselement{\scnkeyword{цель}*}
	\scnaddlevel{1}
	\scnidtf{целевая ситуация*}
	\scnsubset{спецификация*}
	\scnrelfrom{второй домен}{ситуация}
	\scnidtf{описание того, что требуется получить (какая ситуация должна быть достигнута) в результате выполнения заданного (специфицируемого) действия*}
	\scnidtf{цель выполнения действия*}
	\scnidtf{интенция, стремление, намерение, замысел, желание, устремление, направленность действия*}
	\scnaddlevel{-1}

\scnhaselement{\scnkeyword{план}*}
	\scnaddlevel{1}
	\scnidtf{план выполнения действия*}
	\scnidtf{процедурная спецификация выполнения действия*}
	\scnidtf{декомпозиция выполняемого действия на систему последовательно/параллельно выполняемых \uline{под}действий*}
	\scnidtf{описание того, как может быть выполнено соответствующее сложное действие*}
	\scnidtf{спецификация соответствующего действия, уточняющая то, \uline{как} предполагается выполнять это действие*}
	\scnidtf{план решения задачи (выполнения сложного действия) путем описания последовательности выполнения поддействий с описанием того, как передается управление от одних поддействий к другим, как осуществляется распараллеливание, как организуется выполнение циклов*}
	\scnaddlevel{-1}

\scnhaselement{\scnkeyword{декларативная спецификация выполнения действия*}}
	\scnaddlevel{1}
	\scnsubset{спецификация*}
	\scnrelfrom{второй домен}{декларативная спецификация выполнения действия}
	\scnaddlevel{1}
	\scnexplanation{В состав такой спецификации действия входят:
		\begin{scnitemize}
		\item \scnkeyword{контекст* \textit{действия}}, содержащий информацию, достаточную для его выполнения;
		\item \scnkeyword{множество используемых методов*} и инструментов, достаточных для выполнения действия.
		\end{scnitemize}}
	\scnidtf{непроцедурное описание выполнения сложного (неэлементарного) действия}
	\scnsuperset{функциональная спецификация выполнения действия}
	\scnsuperset{логическая спецификация выполнения действия}
	\scnnote{\textit{декларативная спецификация выполнения действия} -- это такой выделенный фрагмент \textit{базы знаний} (такой \textit{контекст*} выполнения соответствующего конкретного \textit{действия}), которого \uline{достаточно} для выполнения этого \textit{действия} с помощью заданного множества \textit{методов}, используемых в рамках указанного \textit{контекста*}. При этом важна \uline{минимизация} и самого \textit{контекста*} и \textit{множества используемых методов*}.}
	\scnaddlevel{-1}
	\scnaddlevel{-1}
\scnhaselement{\scnkeyword{контекст*}}
\scnhaselement{\scnkeyword{множество используемых методов*}}

\scnhaselement{\scnkeyword{протокол}*}
\scnaddlevel{1}
\scnidtf{декомпозиция выполненного действия на систему последовательно-параллельно выполненных его \uline{под}действий*}
\scnidtf{описание того, как действительно было выполнено соответствующее действие и, в частности, описание последовательности соответствующих ситуаций и событий*}
\scnidtf{протокол выполнения сложного действия, включающий в себя протоколы выполнения всех поддействий этого действия*}
\scnidtf{протокол решения задачи*}
\scnidtf{история решения выполненной задачи*}
\scnaddlevel{-1}

\scnhaselement{\scnkeyword{результативная часть протокола}*}
\scnaddlevel{1}
\scnidtf{часть протокола соответствующего выполненного действия, которая включает в себя только те его поддействия, которые действительно внесли вклад в построение результата ("сухого остатка"{}) этого выполненного действия*}
\scnnote{протокол выполненного действия и результативная часть этого протокола могут сильно отличаться. Примером тому является, например, соотношение между протоколом доказательства некоторой конкретной теоремы и результативной частью этого протокола, которая является подтверждением корректности проведенного доказательства и, соответственно, обоснованием истинности доказанной теоремы.}
\scnaddlevel{-1}
\end{SCn}