\begin{SCn}

\bigskip
\scnfragmentcaptiontext{Понятие класса действий и метода}

\scnheader{класс действий}
\scnrelto{семейство подклассов}{действие}
\scnidtfexp{\uline{максимальное} множество аналогичных (похожих в определенном смысле) действий, для которого существует (но не обязательно известный в текущий момент) по крайней мере один \textbf{метод} (или средство), обеспечивающий выполнение \uline{любого} действия из указанного множества действий} 
\scnidtf{множество однотипных действий}
\scnsuperset{класс элементарных действий}
\scnsuperset{класс легко выполнимых сложных действий}
\scnnote{Тот факт, что каждому выделяемому \textit{классу действий} соответствует по крайней мере один общий для них \textit{метод} выполнения этих \textit{действий}, означает то, что речь идет о \uline{семантической} "кластеризации"{} множества \textit{действий}, т.е. о выделении \textit{классов действий} по признаку \uline{семантической близости} (сходства) \textit{действий}, входящих в состав выделяемого \textit{класса действий}. При этом прежде всего учитывается аналогичность (сходство) \textit{исходных ситуаций} и \textit{целевых ситуаций} рассматриваемых \textit{действий}, т.е. аналогичность \textit{задач}, решаемых в результате выполнения соответствующих \textit{действий}. Поскольку одна и та же \textit{задача} может быть решена в результате выполнения нескольких \uline{разных} \textit{действий}, принадлежащих \uline{разным} \textit{классам действий}, следует говорить не только о \textit{классах действий} (множествах аналогичных действий), но и о \textbf{\textit{классах задач}} (о множествах аналогичных задач), решаемых этими \textit{действиями}. Так, например, на множестве \textit{классов действий} заданы следующие \textit{отношения}:
	\begin{scnitemize}
	\item \textit{отношение}, каждая связка которого связывает два разных (непересекающихся) \textit{класса действий}, осуществляющих решение одного и того же \textit{класса задач};
	\item \textit{отношение}, каждая связка которого связывает два разных \textit{класса действий}, осуществляющих решение разных \textit{классов задач}, один из которых является \textit{надмножеством} другого.
	\end{scnitemize}}

\scnheader{класс элементарных действий}
\scnidtf{множество элементарных действий, указание принадлежности которому является \uline{необходимым} и достаточным условием для выполнения этого действия}
\scnnote{Множество всевозможных элементарных действий, выполняемых каждым субъектом, должно быть \uline{разбито} на классы элементарных действий.}

\scnheader{класс легковыполнимых сложных действий}
\scnidtf{множество сложных действий, для которого известен и доступен по крайней мере один \textbf{\textit{метод}}, интерпретация которого позволяет осуществить полную (окончательную, завершающуюся элементарными действиями) декомпозицию на поддействия \uline{каждого} сложного действия из указанного выше множества}
\scnidtf{множество всех сложных действий, выполнимых с помощью известного \textit{метода}, соответствующего этому множеству}

\scnheader{спецификация*} 
\scnsuperset{сужение отношения по первому домену*(спецификация*; класс действий)*}
	\scnaddlevel{1}
  	\scnidtftext{часто используемый sc-идентификатор}{спецификация класса действий*}
  	\scnsubdividing{\textbf{обобщенная формулировка задач соответствующего класса*}\\
  	\scnaddlevel{1}
  	\scnsubdividing{\textbf{обобщенная декларативная формулировка задач соответствующего класса*}
  	;\textbf{обобщенная процедурная формулировка задач соответствующего класса*}}
  	\scnaddlevel{-1}
  	;\textbf{метод*}\\
  	\scnaddlevel{1}
  	\scnidtf{метод решения задач заданного класса*}
  	\scnidtf{метод выполнения действий соответствующего (заданного) класса*}
  	\scnsubdividing{\textbf{процедурный метод выполнения действий соответствующего класса*}\\
  	\scnaddlevel{1}
  	\scnidtf{обобщенный план выполнения действий заданного класса*}
  	\scnaddlevel{-1}
  	;\textbf{декларативный метод выполнения действий соответствующего класса*}\\
  	\scnaddlevel{1}
  	\scnidtf{обобщенная декларативная спецификация выполнения действий заданного класса*}
  	\scnaddlevel{-1}}
  	\scnaddlevel{-1}}
  	\scnaddlevel{-1}

\scnheader{класс задач}
\scnidtf{множество аналогичных действий}
\scnidtf{множество задач, для которого можно построить обобщенную формулировку задач, соответствующую всему этому множеству задач}
\scnnote{Каждая \textit{обобщенная формулировка задач соответствующего класса} по сути есть не что иное, как строгое логическое определение указанного класса задач.}

\scnheader{класс действий}
\scnsubdividing{\textbf{класс действий, однозначно задаваемый решаемым классом задач}\\
	\scnaddlevel{1}
	\scnidtf{\textit{класс действий}, обеспечивающих решение соответствующего \textit{класса задач} и использующих при этом любые, самые разные \textit{методы} решения задач этого класса}
	\scnaddlevel{-1}
	;\textbf{класс действий, однозначно задаваемый используемым методом решения задач}}

\scnheader{метод}
\scnrelto{второй домен}{метод*}
\scnidtf{описание того, \uline{как} может быть выполнено любое или почти любое действие, принадлежащее соответствующему классу действий}
\scnidtf{метод решения соответствующего класса задач, обеспечивающий решение любой или большинства задач указанного класса}
\scnidtf{обобщенная спецификация выполнения действий соответствующего класса}
\scnidtf{обобщенная спецификация решения задач соответствующего класса}
\scnidtf{программа решения задач соответствующего класса, которая может быть как процедурной, так и декларативной (непроцедурной)}
\scnidtf{знание о том, как можно решать задачи соответствующего класса}
\scnsubset{знание}
\scniselement{вид знаний}

\scnidtf{способ}
\scnidtf{знание о том, как надо решать задачи соответствующего класса задач (множества эквивалентных (однотипных, похожих) задач)}
\scnidtf{метод (способ) решения некоторого (соответствующего) класса задач}
\scnsubset{процедурная программа}
	\scnaddlevel{1}
	\scnsubset{алгоритм}
	\scnaddlevel{-1}
\scnidtf{информация (знание), достаточная для того, чтобы решить любую \textit{задачу}, принадлежащую соответствующему \textit{классу задач} с помощью соответствующей \textit{модели решения задач}}
\scnnote{В состав спецификации каждого \textit{класса задач} входит описание способа "привязки"{} \textit{метода} к исходным данным конкретной \textit{задачи}, решаемой с помощью этого \textit{метода}. Описание такого способа "привязки"{} включает в себя:
	\begin{scnitemize}
	\item набор переменных, которые входят как в состав \textit{метода}, так и в состав \textit{обобщенной формулировки задач соответствующего класса} и значениями которых являются соответствующие элементы исходных данных каждой конкретной решаемой задачи;
	\item часть \textit{обобщенной формулировки задач} того класса, которому соответствует рассматриваемый \textit{метод}, являющихся описанием \uline{условия применения} этого \textit{метода}.
	\end{scnitemize}
\bigskip
Сама рассматриваемая "привязка"{} \textit{метода} к конкретной \textit{задаче}, решаемой с помощью этого \textit{метода} осуществляется путем \uline{поиска} в \textit{базе знаний} такого фрагмента, который удовлетворяет условиям применения указанного \textit{метода}. Одним из результатов такого поиска и является установление соответствия между указанными выше переменными используемого \textit{метода} и значениями этих переменных в рамках конкретной решаемой \textit{задачи}. 

Другим вариантом установления рассматриваемого соответствия является явное обращение (вызов, call) соответствующего \textit{метода} (программы) с явной передачей соответствующих параметров. Но такое не всегда возможно, т.к. при выполнении процесса решения конкретной \textit{задачи} на основе декларативной спецификации выполнения этого действия нет возможности установить:
	\begin{scnitemize}
	\item когда необходимо инициировать вызов (использование) требуемого \textit{метода};
	\item какой конкретно \textit{метод} необходимо использовать;
	\item какие параметры, соответствующие конкретной инициируемой \textit{задачи}, необходимо передать для "привязки"{} используемого \textit{метода} к этой \textit{задаче}.
	\end{scnitemize}
	

Процесс "привязки"{} \textit{метода} решения \textit{задач} к конкретной \textit{задаче}, решаемой с помощью этого \textit{метода}, можно также представить как процесс, состоящий из следующих этапов:
	\begin{scnitemize}
	\item построение копии используемого \textit{метода};
	\item склеивание основных (ключевых) переменных используемого \textit{метода} с основными параметрами конкретной решаемой \textit{задачи}.
	\end{scnitemize}

В результате этого на основе рассматриваемого \textit{метода} используемого в качестве образца (шаблона) строится спецификация процесса решения конкретной задачи -- процедурная спецификация (\textit{план}) или декларативная.}
\scnnote{Заметим, что \textit{методы} могут использоваться даже при построении \textit{планов} решения конкретных \textit{задач}, в случае, когда возникает необходимость многократного повторения неких цепочек \textit{действий} при априори неизвестном количестве таких повторений. Речь идет о различного вида \textbf{циклах}, которые являются простейшим видом процедурных \textit{методов} решения задач, многократно используемых (повторяемых) при реализации \textit{планов} решения некоторых \textit{задач}.}

\scnheader{эквивалентность задач*}
\scnidtf{быть эквивалентной задачей*}
\scniselement{отношение}
\scntext{определение}{Задачи являются эквивалентными в том и только в том случае, если они могут быть решены путем интерпретации одного и того же \textit{метода} (способа), хранимого в памяти кибернетической системы.}
\scnnote{Некоторые \textit{задачи} могут быть решены разными \textit{методами}, один из которых, например, является обобщением другого.}

\scnheader{отношение, заданное на множестве методов}
\scnhaselement{подметод*}
	\scnaddlevel{1}
	\scnidtf{подпрограмма*}
	\scnidtf{быть методом, использование которого (обращение к которому) предполагается при реализации заданного метода*}
	\scnrelboth{следует отличать}{частный метод*}
	\scnaddlevel{1}
	\scnidtf{быть методом, обеспечивающим решение класса задач, который является подклассом задач, решаемых с помощью заданного метода*}
	\scnaddlevel{-1}

\scnheader{стратегия решения задач}
\scnsubset{метод}
\scnidtf{метаметод решения задач, обеспечивающий либо поиск одного релевантного известного метода, либо синтез целенаправленной последовательности аций применения в общем случае различных известных методов}
\scnnote{Можно говорить об универсальном метаметоде (универсальной стратегии) решения задач, объясняющем всевозможные частные стратегии.}
\scnexplanation{Можно говорить о нескольких глобальных \textit{стратегиях решения информационных задач} в базах знаний. Пусть в базе знаний появился знак инициированного действия с формулировкой соответствующей информационной цели, т.е. цели, направленной только на изменение состояния базы знаний. И пусть текущее состояние базы знаний не содержит контекста (исходных данных), достаточного для достижения указанной выше цели, т.е такого контекста, для которого в доступном пакете (наборе) методов (программ) имеется метод (программа), использование которого позволяет достигнуть указанную выше цель. Для достижения такой цели, контекст( исходные данные) которой недостаточен, существует три подхода (три стратегии): 
	\begin{scnitemize}
	\item декомпозиция (сведение изначальной цели к иерархической системе и/или подцелей (и/или подзадач) на основе анализа текущего состояния базы знаний и анализа того, чего в базе знаний не хватает для использования того или иного метода.) 
	
	При этом наибольшее внимание уделяется методам, для создания условий использования которых требуется меньше усилий. В конечном счете мы должны дойти (на самом нижнем уровне иерархии) до подцелей, контекст которых достаточен для применения одного из имеющихся методов (программ) решения задач;
	\item генерация новых знаний в семантической окрестности формулировки изначальной цели с помощью \uline{любых} доступных методов в надежде получить такое состояние базы знаний, которое будет содержать нужный контекст (достаточные исходные данные) для достижения изначальной цели с помощью какого-либо имеющегося метода решения задач;
	\item комбинация первого и второго подхода.
	\end{scnitemize}
Аналогичные стратегии существуют и для поиска пути решения задач, решаемых во внешней среде.}
\end{SCn}