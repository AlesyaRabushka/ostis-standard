\begin{SCn}

\scnsectionheader{\currentname}
\scnstartsubstruct

\scnheader{Предметная область синтаксиса SC-кода}
\scnsdmainclass{sc-элемент}
\scnsdclass{SC-код? \textbf{= sc-текст?};sc-узел;sc-коннектор;sc-дуга;sc-ребро;базовая sc-дуга;Алфавит SC-кода?}
\scnsdrelation{инцидентность sc-коннектора*;инцидентность входящей sc-дуги*}

\scnheader{SC-код}
\scnnote{Универсальность SC-кода позволяет с его помощью описывать любые объекты. Таким объектом может быть любой язык коммуникации с пользователями (в том числе и естественный язык), а также сам SC-код. Синтаксис и семантика SC-кода описываются в виде соответствующей формальных онтологий.}

\scnheader{sc-элемент}
\scnidtf{атомарный фрагмент хранимой в памяти знаковой конструкции, принадлежащей SC-коду}

\scnheader{Алфавит SC-кода}
\scneqtoset{sc-элемент;sc-узел;sc-коннектор;sc-дуга;sc-ребро;базовая sc-дуга}

\scnheader{sc-ребро}
\scnidtf{неориентированный sc-коннектор}
\scnheader{sc-дуга}
\scnidtf{ориентированный sc-коннектор}

\scnheader{Алфавит SC-кода}
\scnnote{В отличие от других языков, классы синтаксически выделяемых элементарных фрагментов текстов SC-кода могут пересекаться. Так, например, sc-элемент может одновременно принадлежать и классу \textit{sc-элементов} и классу \textit{sc-узлов}, а также может одновременно принадлежать и классу \textit{sc-элементов}, и классу \textit{sc-коннекторов}, и классу \textit{sc-дуг}, и классу \textit{базовых sc-дуг}.

Такая особенность \textit{Алфавита SC-кода} дает возможность строить синтаксически корректные \textit{\textbf{sc-тексты}} (тексты SC-кода) в условиях неполноты наших исходных знаний о некоторых \textit{sc-элементах}.}

\scnheader{sc-элемент}
\scnsubdividing{sc-узел;sc-коннектор}
\scnheader{sc-коннектор}
\scnsubdividing{sc-ребро;sc-дуга}
\scnheader{sc-дуга}
\scnsuperset{базовая sc-дуга}
\scnheader{инцидентность sc-коннектора*}
\scnrelfrom{первый домен}{sc-коннектор}
\scnrelfrom{второй домен}{sc-элемент}
\scnsuperset{инцидентность входящей sc-дуги*}
\scniselement{бинарное отношение}
\scniselement{ориентированное отношение}
\scniselement{отношение, среди элементов которого нет мультимножеств}
\scnaddlevel{1}
\scnnote{Для бинарных отношений принадлежность данному классу означает отсутствие петель}
\scnaddlevel{-1}
\scnheader{инцидентность входящей sc-дуги*}
\scnrelfrom{первый домен}{sc-дуга}
\scnrelfrom{второй домен}{sc-элемент}

\scnheader{sc-текст}
\scnnote{В процессе обработки \textbf{\textit{sc-текстов}} выполняются следующие правила уточнения их синтаксической разметки:
\begin{scnitemize}
    \item если стало известно, что \textit{sc-элемент}, имеющий метку \textit{sc-элемента}, является \textit{sc-узлом} или \textit{sc-коннектором}, то ему приписывается метка \textit{sc-узла} или  \textit{sc-коннектора}, а метка \textit{sc-элемента} удаляется;
    \item если стало известно, что \textit{sc-элемент}, имеющий метку \textit{sc-коннектора}, является \textit{sc-ребром} или \textit{sc-дугой}, то ему приписывается метка \textit{sc-ребра} или \textit{sc-дуги}, а метка \textit{sc-коннектора} удаляется;
    \item если стало известно, что \textit{sc-элемент}, имеющий метку \textit{sc-дуги}, является \textit{базовой sc-дугой}, то ему приписывается метка \textit{базовой sc-дуги}, а метка \textit{sc-дуги} удаляется.
\end{scnitemize}
}

\scnheader{SC-код}
\scntext{особенности}{Отметим некоторые синтаксические особенности SC-кода. 
\begin{scnitemize}
    \item Тексты SC-кода являются \textbf{абстрактными} в том смысле, что они абстрагируются от конкретного варианта их кодирования в памяти компьютерной системы. Кодирование текстов, в частности, зависит от варианта технической реализации памяти компьютерной системы. Так, например, актуальной является аппаратная реализация ассоциативной нелинейной памяти, в которой реализуется структурная реконфигурация хранимой информации, в которой обработка информации сводится не к изменению состояния элементов памяти, а к изменению конфигурации связей между ними.
    \item Тексты SC-кода являются структурами \textbf{графоподобного вида}. Все исследованные до настоящего времени графовые структуры легко представимы в SC-коде (неориентированные и ориентированные графы, мультиграфы, псевдографы, гиперграфы, сети и др.). Но, кроме этого, в SC-коде представимы и связи между связями, связи между целыми структурами и многое другое. SC-код фактически является \textbf{графовым языком}, текстами которого являются графоподобные структуры. Таким образом, теория графов при соответствующем ее расширении может стать основой описания синтаксиса SC-кода.
\end{scnitemize}

Простота синтаксиса \textit{\textbf{SC-кода}} обусловлена следующими \textbf{семантическими} свойствами \textit{sc-текстов} (знаковых конструкций, принадлежащих SC-коду).

\begin{scnitemize}
    \item \textbf{Все} (!) \textit{sc-элементы}, то есть элементарные (атомарные) фрагменты \textit{sc-текстов}, являются знаками (обозначениями) различных описываемых сущностей. При этом, каждая сущность, описываемая в тексте \textit{SC-кода}, должна быть представлена своим знаком;
    \item Никаких других знаков, кроме \textit{sc-элементов}, \textit{sc-тексты} не содержат (т. е. нет знаков, в состав которых входят другие знаки); 
    \item Любая сущность может быть описана \textit{sc-текстом} и, соответственно, представлена в этом \textit{sc-тексте} своим знаком;
    \item Все \textit{синтаксически выделяемые классы sc-элементов} (т.е. все элементы \textit{Алфавита SC-кода}) имеют четкую семантическую интерпретацию -- являются классами \textit{sc-элементов}, каждый из которых обозначает сущность, имеющую общие одинаковые свойства со всеми другими сущностями, обозначаемыми другими \textit{sc-элементами} этого же класса.
\end{scnitemize}}

\scnendstruct

\end{SCn}