\begin{SCn}
	
\scnsegmentheader{Уточнение понятий план сложного действия, классы действий, класса задач, метода}

\scnstartsubstruct
	
\scniselement{сегмент базы знаний}
	
\scnheader{план сложного действия}
\scnidtf{план}
\scnidtf{план выполнения сложного действия}
\scnidtf{план решения \textit{сложной задачи}}
\scnidtf{план выполнения действия}
\scnidtf{спецификация выполнения действия}
\scnidtf{декомпозиция выполняемого действия на систему последовательно/параллельно выполняемых поддействий*}
\scnidtf{описание того, как может быть выполнено соответствующее сложное действие}
\scnidtf{спецификация соответствующего действия, уточняющая то, \uline{как} предполагается выполнять это действие}
\scnidtf{план решения задачи (выполнения сложного действия) путем описания последовательности выполнения поддействий с описанием того, как передается управление от одних поддействий другим, как осуществляется распараллеливание, как организуется выполнение циклов}
	
\scndefinition{вид спецификации \textit{сложного действия}, представляющий собой систему \textit{задач}, \textit{интерпретация} которой (предполагающая решение указанных \textit{задач} в определенной последовательности) обеспечивает выполнение специфицируемого \textit{сложного действия}} 
\scnsubset{знание}
\scnexplanation{Каждый \textit{план} представляет собой \textit{семантическую окрестность, ключевым sc-элементом\scnrolesign} является \textit{действие}, для которого дополнительно детализируется предполагаемый процесс его выполнения. Основная задача такой детализации -- локализация области базы знаний, в которой предполагается работать, а также набора агентов, необходимого для выполнения  описываемого действия. При этом детализация не обязательно должна быть доведена до уровня элементарных действий, цель составления плана -- уточнение подхода к решению той или иной задачи, не всегда предполагающее составления подробного пошагового решения.
		
При описании \textit{плана} может быть использован как процедурный, так и декларативный подход. В случае процедурного подхода для соответствующего \textit{действия} указывает его декомпозиция на более частные поддействия, а также необходимая спецификация этих поддействий. В случае декларативного подхода указывается набор подцелей (например, при помощи логических утверждений), достижение которых необходимо для выполнения рассматриваемого \textit{действия}. На практике оба рассмотренных подхода можно комбинировать.
		
В общем случае \textit{план} может содержать и переменные, например в случае, когда часть плана задается в виде цикла (многократного повторения некоторого набора действий). Также план может содержать константы, значение которых в настоящий момент не установлено и станет известно, например, только после выполнения предшествующих ему \textit{действий}.
		
Каждый \textit{план} может быть задан заранее как часть спецификации \textit{действия}, т.е. \textit{задачи}, а может формироваться \textit{субъектов} уже собственно в процессе выполнения \textit{действия}, например, в случае использования стратегии разбиения задачи над подзадачи. В первом случае \textit{план} \textit{включается*} в \textit{задачу}, соответствующую тому же действию.}
	
\scnsubdividing{процедурный план сложного действия\\
	\scnaddlevel{1}
		\scnidtf{декомпозиция \textit{сложного действия} на множество последовательно и/или параллельно выполняемых \textit{поддействий}}
	\scnaddlevel{-1}
		;непроцедурный план сложного действия\\
		\scnaddlevel{1}
		\scnidtf{декомпозиция исходной \textit{задачи}, соответствующей заданному \textit{сложному действию}, на иерархическую систему и/или подзадач}
	\scnaddlevel{-1}
}
	
\scnheader{процедурный план сложного действия}
\scnnote{В \textit{процедурном плане выполнения сложного действия} соответствующие \textit{поддействия*} декомпозируемого \textit{сложного действия} представляются специфицирующими их \textit{задачами}. Но, кроме такого рода \textit{задач}, в \textit{процедурный план выполнения сложного действия} входят также \textit{задачи}, которые специфицируют \textit{действия}, обеспечивающие:
	\begin{scnitemize}
			\item синхронизацию выполнения \textit{поддействий*} заданного \textit{сложного действия};
			\item передачу управления указанным \textit{поддействиям*} (а точнее, соответствующим им \textit{задачам}), т.е. инициирование указанных \textit{поддействий*} (и соответствующих им \textit{задач}).
	\end{scnitemize}
}
	
\scnheader{действие управления интерпретацией процедурного плана сложного действия}
	
\scnrelboth{семантически близкий знак}{задача управления интерпретацией процедурного плана сложного действия}
\scnsuperset{безусловная подзадача управления от одного поддействия к другому}
\scnsuperset{инициирование  заданного поддействия при возникновении в базе знаний ситуации или события, заданного вида}
\scnsuperset{инициирование заданного множества поддействий при успешном завершении выполнения \uline{всех} поддействий другого заданного множества}
\scnsuperset{инициирование заданного множества поддействий при успешном завершении выполнения \uline{по крайней мере одного} поддействия другого заданного множества}
	
\scnheader{класс действий}
\scnrelto{семейство подклассов}{действие}
\scnidtfexp{\uline{максимальное} множество аналогичных (похожих в определенном смысле) действий, для которого существует (но не обязательно известных в текущий момент) по крайней мере один \textit{метод} (или средство), обеспечивающий выполнение \uline{любого} действия из указанного множества действий}
\scnidtf{множество однотипных действий}
\scnsuperset{класс элементарных действий}
\scnsuperset{класс легковыполнимых сложных действий}
\scnnote{Тот факт, что каждому выделяемому \textit{классу действий} соответствует по крайней мере один общий для них \textit{метод} выполнения этих \textit{действий}, означает то, что речь идет о \uline{семантической} "кластеризации"{} множества \textit{действий}, т.е. выделении \textit{классов действий} по признаку \uline{семантической близости} (сходства) \textit{действий}, входящих в состав выделяемого \textit{класса действий}. При этом прежде всего учитывается аналогичность (сходство) \textit{исходных ситуаций и целевых ситуаций} рассматриваемых \textit{действий}, т.е. аналогичность \textit{задач}, решаемых в результате выполнения соответствующих \textit{действий}. Поскольку одна и та же \textit{задача} может быть решена в результате выполнения нескольких \uline{разных} \textit{действий}, принадлежащих \uline{разным} \textit{классам действий}, следует говорить не только о \textit{классах действий} (множествах аналогичных действий), но и о \textit{классах задач} (о множествах аналогичных задач), решаемых этими \textit{действиями}. Так, например, на множестве \textit{классом действий} заданы следующие \textit{отношения}:
		\begin{scnitemize}
			\item \textit{отношение}, каждая связка которого связывает два разных (непересекающихся) \textit{класса действий}, осуществляющих решение одного и того же \textit{класса задач};
			\item \textit{отношение}, каждая связка которого связывает два разных \textit{класса действий}, осуществляющих решение разных \textit{классов задач}, один из которых является \textit{надмножеством} другого.
		\end{scnitemize}
}
\scntext{правило идентификации экземпляров}{Конкретные \textit{классы действий} в рамках \textit{Русского языка} именуются по следующим правилам:
	\begin{scnitemize}
		\item в начале идентификатора пишется слово ``\textit{действие}'' и ставится точка;
		\item далее со строчной буквы идет либо содержащее глагол совершенного вида в инфинитиве описание сути того, что требуется получить в результате выполнения действий данного класса, либо вопросительное предложение, являющееся спецификацией запрашиваемой (ответной) информации.
	\end{scnitemize}	
Например:
\newline
\textit{действие, сформировать полную семантическую окрестность указываемой сущности
\newline
действие, верифицировать заданную структуру}
			
Допускается использовать менее строгие идентификаторы, которые, однако, обязаны оперировать словом ``\textit{действие}'' и достаточно четко специфицировать суть действий описываемого класса.
			
Например:
\newline
\textit{действие редактирования базы знаний}\newline
\textit{действие, направленное на установление темпоральных характеристик указываемой сущности} }
	
\scnheader{класс элементарных действий}
\scnidtf{множество элементарных действий, указание принадлежности которому является \uline{необходимым} и достаточным условием для выполнения этого действия}
\scnnote{Множество всевозможных элементарных действий, выполняемых каждым субъектом, должно быть \uline{разбито} на классы элементарных действий.}
\scnexplanation{Принадлежность некоторого \textit{класса действий} множеству \textit{классу элементарных действий}, фиксирует факт того, что при указании всех необходимых аргументов принадлежности \textit{действия} данному классу достаточно для того, чтобы некоторый субъект мог приступить к выполнению этого действия.
		
При этом, даже если \textit{класс действий} принадлежит множеству \textit{класс элементарных действий}, не запрещается вводить более частные \textit{классы действий}, для которых, например, заранее фиксируется один из аргументов.
		
Если конкретный \textit{класс элементарных действий} является более частным по отношению к \textit{действиям в sc-памяти}, то это говорит о наличии в текущей версии системы как минимум одного \textit{sc-агента}, ориентированного на выполнение действий данного класса.}
	
\scnheader{класс легковыполнимых сложных действий}
\scnidtf{множество сложных действий, для которого известен и доступен по крайней мере один \textit{метод}, интерпретация которого позволяет осуществить полную (окончательную, завершающуюся элементарными действиями) декомпозицию на поддействия \uline{каждого} сложного действия из указанного выше множества}
\scnidtf{множество всех сложных действий, выполнимых с помощью известного \textit{метода}, соответствующего этому множеству}
\scnexplanation{Принадлежность некоторого \textit{класса действий} множеству \textit{класс легковыполнимых сложных действий} фиксирует факт того, что даже при указании всех необходимых аргументов принадлежности \textit{действия} данному классу недостаточно для того, чтобы некоторый \textit{субъект} приступил к выполнению этого действия, и требуются дополнительные уточнения.}
	
\scnheader{сужение отношения по первому домену*(спецификация*;класс действий*)}
\scnidtftext{часто используемый идентификатор}{спецификация класса действий*}
\scnsubdividing{обобщенная формулировка задач соответствующего класса*\\
	\scnaddlevel{1}
		\scnsubdividing{обобщенная декларативная формулировка задач соответствующего класса*\\
			;обобщенная процедурная формулировка задач соответствующего класса*\\
		\scnaddlevel{-1}}
		;метод*\\
		\scnaddlevel{1}
		\scnidtf{метод решения задач заданного класса*}
		\scnidtf{метод выполнения действий соответствующего (заданного) класса*}
		\scnaddlevel{1}
		\scnsubdividing{процедурный метод выполнения действий соответствующего класса*\\
			\scnaddlevel{1}
			\scnidtf{обобщенный план выполнения действий заданного класса*}
			\scnaddlevel{-1}
			;декларативный метод выполнения действий соответствующего класса*\\
			\scnaddlevel{1}
			\scnidtf{обобщенная декларативная спецификация выполнения действий заданного класса*}
			\scnaddlevel{-1}
			\scnaddlevel{-1}
			\scnaddlevel{-1}
		}
	}

\scnheader{класс задач}
\scnidtf{множество аналогичных задач}
\scnidtf{множество задач, для которого можно построить обобщенную формулировку задач, соответствующую всему этому множеству задач}
\scnnote{Каждая \textit{обобщенная формулировка задач соответствующего класса} по сути есть не что иное, как строгое логическое определение указанного класса задач.}
\scnrelto{семейство подмножеств}{задача}
\scntext{правило идентификации экземпляров}{Конкретные \textit{классы задач} в рамках \textit{Русского языка} именуются по следующим правилам:
	\begin{scnitemize}
			\item в начале идентификатора пишется слово ``\textit{задача}'' и ставится точка;
			\item далее с прописной буквы идет либо содержащее глагол совершенного вида в инфинитиве описание сути того, что требуется получить в результате решения данного \textit{класса задач}, либо вопросительное предложение, являющееся спецификацией запрашиваемой (ответной) информации.
	\end{scnitemize}
Например:\\
\textit{задача. сформировать полную семантическую окрестность указываемой сущности}\\
\textit{задача. верифицировать заданную структуру}
		
Допускается использовать менее строгие идентификаторы, которые, однако, обязаны оперировать словом ``\textit{задача}'' и достаточно четко специфицировать суть задач описываемого класса. 
		
Например:\\
\textit{задача на установление значения величины}\\
\textit{задача на доказательство}
	}
	
\scnheader{класс команд}
\scnrelto{семейство подмножеств}{задача}
\scnsuperset{класс интерфейсных пользовательских команд}
\scnaddlevel{1}
\scnsuperset{класс интерфейсных команд пользователя ostis-системы}
\scnaddlevel{-1}
\scnsuperset{класс команд без аргументов}
\scnsuperset{класс команд с одним аргументом}
\scnsuperset{класс команд с двумя аргументами}
\scnsuperset{класс команд с произвольным числом аргументов}
\scnexplanation{Идентификатор конкретного класса \textit{класса команд} в рамках \textit{Русского языка} пишется со строчной буквы и представляет собой либо содержащее глагол совершенного вида в инфинитиве описание сути того, что требуется получить в результате выполнения действий, соответствующих данному \textit{классу команд}, либо вопросительное предложение, являющееся спецификацией запрашиваемой (ответной) информации. 
		
		Например:\\
		\textit{сформировать полную семантическую окрестность указываемой сущности}\\
		\textit{верифицировать заданную структуру}
		
		Допускается использовать менее строгие идентификаторы, которые, однако, обязаны оперировать словом ``\textit{команда}'' и достаточно четко специфицировать суть задач описываемого класса. 
		
		Например:\\
		\textit{команда редактирования базы знаний}\\
		\textit{команда установления темпоральных характеристик указываемой сущности}}
\scnsubdividing{атомарный класс команд;неатомарный класс команд}
	
\scnheader{атомарный класс команд}
\scnexplanation{Принадлежность некоторого \textit{класса команд} множеству \textit{атомарных классов команд} фиксирует факт того, что данная спецификация является достаточной для того, чтобы некоторый субъект приступил к выполнению соответствующего действия.
		
При этом, даже если \textit{класса команд} принадлежит множеству \textit{атомарных классов команд} не запрещается вводить более частные \textit{классы команд}, в состав которых входит информация, дополнительно специфицирующая соответствующее \textit{действие}.
		
Если соответствующий данному \textit{классу команд класс действий} является более частным по отношению к \textit{действиям в sc-памяти}, то попадание данного класса команд во множество \textit{атомарных классов команд} говорит о наличии в текущей версии системы как минимум одного \textit{sc-агента}, условие инициирования которого соответствует формулировке команд данного класса.}
	
\scnheader{неатомарный класс команд}
\scnexplanation{Принадлежность некоторого \textit{класса команд} множеству \textit{неатомарных классов команд} фиксирует факт того, что данная спецификация не является достаточной для того, чтобы некоторый субъект приступил к выполнению соответствующего действия, и требует дополнительных уточнений.}
	
\scnheader{класс действий}
\scnsubdividing{\textit{класс действий, однозначно задаваемый решаемым классом задач}\\
		\scnaddlevel{1}
		\scnidtf{\textit{класс действий}, обеспечивающих решение соответствующего \textit{класса задач} и использующих при этом любые, самые разные \textit{методы} решения задач этого класса}
		\scnaddlevel{-1}
		;\textit{класс действий, однозначно задаваемый используемым методом решения задач}}
	
\scnheader{метод}
\scnrelto{второй домен}{метод*}
\scnidtf{описание того, \uline{как} может быть выполнено любое или почти любое действие, принадлежащее соответствующему классу действий}
\scnidtf{метод решения соответствующего класса задач, обеспечивающий решение любой или большинства задач указанного класса}
\scnidtf{обобщенная спецификация выполнения действий соответствующего класса}
\scnidtf{обобщенная спецификация решения задач соответствующего класса}
\scnidtf{программа решения задач соответствующего класса, которая может быть как процедурной, так и декларативной (непроцедурной)}
\scnidtf{знание о том, как можно решать задачи соответствующего класса}
\scnsubset{знание}
\scniselement{вид знаний}
\scnidtf{способ}
\scnidtf{знание о том, как надо решать задачи соответствующего класса задач (множества эквивалентных (однотипных, похожих) задач)}
\scnidtf{метод (способ) решения некоторого (соответствующего) класса задач}
\scnidtf{информация (знание), достаточная для того, чтобы решить любую \textit{задачу}, принадлежащую соответствующему \textit{классу задач} с помощью соответствующей \textit{модели решения задач}}
\scnidtf{обобщенный план выполнения некоторого класса сложных действий (или обобщенный план решения соответствующего класса сложных задач), "привязка"{} которого к конкретной задаче указанного класса и последующая интерпретация обеспечивает решение любой или почти любой задачи этого класса}
\scnnote{Очевидно, что трудоемкость разработки метода определяется не столько мощностью класса задач, решаемых с помощью разрабатываемого метода, сколько их семантической близостью, аналогичностью.}
\scnidtf{метод решения соответствующего класса задач}
\scnidtf{метод выполнения соответствующего класса действий}
\scnidtf{"пассивный"{} метод, хранимый в базе знаний и используемый соответствующими коллективами агентов при соответствующем их инициировании}
\scnidtf{метод выполнения действий некоторого класса или метод решения задач некоторого класса или метод, который может быть использован для выполнения некоторого конкретного действия или для решения некоторой конкретной задачи}
\scnidtf{обобщенное описание того, как можно выполнить действия из соответствующего класса действий или как можно решить задачу из соответствующего класса задач}
	
\scnsuperset{метод сведения задач к подзадачам}
\scnaddlevel{1}
\scnidtf{класс логически эквивалентных методов, обеспечивающих решение задач путем сведения этих задач к подзадачам и отличающихся только деталями формализации}
\scnaddlevel{-1}
\scnnote{В состав спецификации каждого \textit{класса задач} входит описание способа "привязки"{} \textit{метода} к исходным данным конкретной \textit{задачи}, решаемой с помощью этого \textit{метода}. Описание такого способа "привязки"{} включает в себя:
	\begin{scnitemize}
			\item набор переменных, которые входят как в состав \textit{метода}, так и в состав \textit{обобщенной формулировки задач соответствующего класса} и значениями которых являются соответствующие элементы исходных данных каждой конкретной решаемой задачи;
			\item часть \textit{обобщенной формулировки задач} того класса, которому соответствует рассматриваемый \textit{метод}, являющихся описанием \uline{условия применения} этого \textit{метода}.
		\end{scnitemize}
		\bigskip
Сама рассматриваемая "привязка"{} \textit{метода} к конкретной \textit{задаче}, решаемой с помощью этого \textit{метода} осуществляется путем \uline{поиска} в \textit{базе знаний} такого фрагмента, который удовлетворяет условиям применения указанного \textit{метода}. Одним из результатов такого поиска и является установление соответствия между указанными выше переменными используемого \textit{метода} и значениями этих переменных в рамках конкретной решаемой \textit{задачи}. 
		
Другим вариантом установления рассматриваемого соответствия является явное обращение (вызов, call) соответствующего \textit{метода} (программы) с явной передачей соответствующих параметров. Но такое не всегда возможно, т.к. при выполнении процесса решения конкретной \textit{задачи} на основе декларативной спецификации выполнения этого действия нет возможности установить:
		\begin{scnitemize}
			\item когда необходимо инициировать вызов (использование) требуемого \textit{метода};
			\item какой конкретно \textit{метод} необходимо использовать;
			\item какие параметры, соответствующие конкретной инициируемой \textit{задаче}, необходимо передать для "привязки"{} используемого \textit{метода} к этой \textit{задаче}.
		\end{scnitemize}
		
Процесс "привязки"{} \textit{метода} решения \textit{задач} к конкретной \textit{задаче}, решаемой с помощью этого \textit{метода}, можно также представить как процесс, состоящий из следующих этапов:
		\begin{scnitemize}
			\item построение копии используемого \textit{метода};
			\item склеивание основных (ключевых) переменных используемого \textit{метода} с основными параметрами конкретной решаемой \textit{задачи}.
		\end{scnitemize}
		
В результате этого на основе рассматриваемого \textit{метода} используемого в качестве образца (шаблона) строится спецификация процесса решения конкретной задачи -- процедурная спецификация (\textit{план}) или декларативная.}
\scnnote{Заметим, что \textit{методы} могут использоваться даже при построении \textit{планов} решения конкретных \textit{задач}, в случае, когда возникает необходимость многократного повторения неких цепочек \textit{действий} при априори неизвестном количестве таких повторений. Речь идет о различного вида \textit{циклах}, которые являются простейшим видом процедурных \textit{методов} решения задач, многократно используемых (повторяемых) при реализации \textit{планов} решения некоторых \textit{задач}.}
	
\scnidtf{программа}
\scnidtf{программа выполнения действий некоторого класса}
\scnnote{Одному \textit{классу действий} может соответствовать несколько \textit{методов} (программ).}
\scnsuperset{программа в sc-памяти}
\scnsuperset{процедурная программа}
\scnaddlevel{1}
\scnidtf{обобщенный процедурный план}
\scnidtf{обобщенный процедурный план выполнения некоторого класса действий}
\scnidtf{обобщенный процедурный план решения некоторого класса задач}
\scnidtf{обобщенная спецификация декомпозиции любого действия, принадлежащего заданному классу действий}
\scnidtf{знание о некотором классе действий (и соответствующем классе задач), позволяющее для каждого из указанных действий достаточно легко построить процедурный план его выполнения}
\scnsubset{алгоритм}
	
\scnexplanation{Каждая \textit{процедурная программа} представляет собой обобщенный процедурный план выполнения \textit{действий}, принадлежащих некоторому классу, то есть \textit{семантическую окрестность, ключевым sc-элементом\scnrolesign} является \textit{класс действий}, для элементов которого дополнительно детализируется процесс их выполнения.
		
В остальном описание \textit{процедурной программы} аналогично описанию \textit{плана} выполнения конкретного \textit{действия} из рассматриваемого \textit{класса действий}.
		
Входным параметрам \textit{процедурной программы} в традиционном понимании соответствуют аргументы, соответствующие каждому \textit{действию} из \textit{класса действий}, описываемого \textit{процедурной программой}. При генерации на основе \textit{процедурной программы} \textit{плана} выполнения конкретного \textit{действия} из данного класса эти аргументы принимают конкретные значения.
		
Каждая \textit{процедурная программа} представляет собой систему описанных действий с дополнительным указанием для действия:
		\begin{scnitemize}
			\item либо \textit{последовательности выполнения действий*} (передачи инициирования), когда условием выполнения (инициирования) действий является завершение выполнения одного из указанных или всех указанных действий;
			\item либо события в базе знаний или внешней среде, являющегося условием его инициирования;
			\item либо ситуации в базе знаний или внешней среде, являющейся условием его инициирования;
		\end{scnitemize}
	}
	\scnaddlevel{-1}
\scnnote{Отметим, что понятие \textit{метода} фактически позволяет локализовать область решения задач соответствующего класса, то есть ограничить множество знаний, которых достаточно для решения задач данного класса определенным способом. Это, в свою очередь, позволяет повысить эффективность работы системы в целом, исключая число лишних действий.}
	
\scnnote{Каждый конкретный метод рассматривается нами как важный вид спецификации соответствующего класса задач, но также и как \textit{объект}, который и сам нуждается в спецификации, обеспечивающей непосредственное применение этого метода. Другими словами, метод является не только спецификацией (спецификацией соответствующего класса задач), но и \uline{объектом} спецификации.}
	
\scnheader{эквивалентность задач*}
\scnidtf{быть эквивалентной задачей*}
\scniselement{отношение}
\scntext{определение}{Задачи являются эквивалентными в том и только в том случае, если они могут быть решены путем интерпретации одного и того же \textit{метода} (способа), хранимого в памяти кибернетической системы.}
\scnnote{Некоторые \textit{задачи} могут быть решены разными \textit{методами}, один из которых, например, является обобщением другого.}
	
\scnheader{отношение, заданное на множестве*(метод)}
\scnhaselement{подметод*}
\scnaddlevel{1}
\scnidtf{подпрограмма*}
\scnidtf{быть методом, использование которого (обращение к которому) предполагается при реализации заданного метода*}
\scnrelboth{следует отличать}{частный метод*}
\scnaddlevel{1}
\scnidtf{быть методом, обеспечивающим решение класса задач, который является подклассом задач, решаемых с помощью заданного метода*}
\scnaddlevel{-1}
	
\scnheader{стратегия решения задач}
\scnsubset{метод}
\scnidtf{метаметод решения задач, обеспечивающий либо поиск одного релевантного известного метода, либо синтез целенаправленной последовательности акций применения в общем случае различных известных методов}
\scnnote{Можно говорить об универсальном метаметоде (универсальной стратегии) решения задач, объясняющем всевозможные частные стратегии.}
\scnexplanation{Можно говорить о нескольких глобальных \textit{стратегиях решения информационных задач} в базах знаний. Пусть в базе знаний появился знак инициированного действия с формулировкой соответствующей информационной цели, т.е. цели, направленной только на изменение состояния базы знаний. И пусть текущее состояние базы знаний не содержит контекста (исходных данных), достаточного для достижения указанной выше цели, т.е такого контекста, для которого в доступном пакете (наборе) методов (программ) имеется метод (программа), использование которого позволяет достигнуть указанную выше цель. Для достижения такой цели, контекст (исходные данные) которой недостаточен, существует три подхода (три стратегии): 
		\begin{scnitemize}
			\item декомпозиция (сведение изначальной цели к иерархической системе и/или подцелей (и/или подзадач) на основе анализа текущего состояния базы знаний и анализа того, чего в базе знаний не хватает для использования того или иного метода.) 
			
			При этом наибольшее внимание уделяется методам, для создания условий использования которых требуется меньше усилий. В конечном счете мы должны дойти (на самом нижнем уровне иерархии) до подцелей, контекст которых достаточен для применения одного из имеющихся методов (программ) решения задач;
			\item генерация новых знаний в семантической окрестности формулировки изначальной цели с помощью \uline{любых} доступных методов в надежде получить такое состояние базы знаний, которое будет содержать нужный контекст (достаточные исходные данные) для достижения изначальной цели с помощью какого-либо имеющегося метода решения задач;
			\item комбинация первого и второго подхода.
		\end{scnitemize}
Аналогичные стратегии существуют и для поиска пути решения задач, решаемых во внешней среде.}
	
\scnheader{сужение отношения по первому домену(спецификация*, метод)}
\scnidtf{спецификация метода*}
\scnsubdividing{денотационная семантика метода*\\
	\scnaddlevel{1}
	\scnidtf{обобщенная формулировка класса задач, решаемых с помощью данного метода*}
	\scnrelboth{семантически близкий знак*}{обобщенная формулировка задач соответствующего класса*}
	\scnaddlevel{1}
	\scnnote{Данное отношение связывает обобщенную формулировку задач не с методом, а с классом задач}
	\scnaddlevel{-1}
	\scnaddlevel{-1}
	;операционная семантика метода*\\
	\scnaddlevel{1}
	\scnidtf{перечень обобщенных агентов, обеспечивающих интерпретацию метода*}
	\scnidtf{семейство методов интерпретации данного метода*}
	\scnidtf{формальное описание интерпретатора заданного метода*}
	\scnaddlevel{-1}
}
	
\bigskip
\scnendstruct \scnendsegmentcomment{Уточнение понятий план сложного действия, классы действий, класса задач, метода}
	
\end{SCn}