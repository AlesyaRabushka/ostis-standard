\begin{SCn}

\scnsectionheader{\currentname}
\scntext{введение}{Обработка информации в \textit{sc-памяти} (т.е. динамика базы знаний, хранимой в \textit{sc-памяти}) в конечном счете сводится:
\begin{scnitemize}
    \item к появлению в \textit{sc-памяти} новых актуальных \textit{sc-узлов} и \textit{sc-коннекторов};
    \item к логическому удалению актуальных \textit{sc-элементов}, т.е. к переводу их в неактуальное состояние (это необходимо для хранения протокола изменения состояния базы знаний, в рамках которого могут описываться действия по удалению \textit{sc-элементов});
    \item к возврату логически удаленных \textit{sс-элементов} в статус актуальных (необходимость в этом может возникнуть при откате базы знаний к какой-нибудь ее прошлой версии);
    \item к физическому удалению \textit{sc-элементов};
    \item к изменению состояния актуальных (логически не удаленных \textit{sc-элементов}) – \textit{sc-узел} может превратиться в \textit{sc-ребро}, \textit{sc-ребро} может превратиться в \textit{sc-дугу}, \textit{sc-дуга} может поменять направленность, \textit{sc-дуга} общего вида может превратиться в \textit{константную стационарную sc-дугу принадлежности}, и т.д.;
\end{scnitemize}
Подчеркнем, что временный характер самого \textit{sc-элемента} (т.к. он может появиться или исчезнуть) никак не связан с возможно временным характером сущности, обозначаемой этим \textit{sc-элементом}. Т.е. временный характер самого sc-элемента и временный характер сущности, которую он обозначает – абсолютно разные вещи.

Таким образом, следует четко отличать динамику внешнего мира, описываемого базой знаний, а динамику самой базы знаний (динамику внутреннего мира). При этом очень важно, чтобы описание динамики базы знаний также входило в состав каждой базы знаний.

К числу понятий, используемых для описания динамики базы знаний относятся:
\begin{scnitemize}
    \item логически удаленный sc-элемент;
    \item сформированное множество;
    \item вычисленное число;
    \item сформированное высказывание;
\end{scnitemize}}

\scnstartsubstruct

\scnheader{Предметная область темпоральных сущностей базы знаний ostis-системы}
\scnidtf{Предметная область, описывающая динамику базы знаний, хранимой в sc-памяти}
\scniselement{предметная область}
\scnsdmainclasssingle{ситуация}
\scnsdclass{sc-элемент;наcтоящий sc-элемент;логически удаленный sc-элемент;число;невычисленное число;вычисленное число;понятие;основное понятие;неосновное понятие;понятие, переходящее из основного в неосновное;понятие, переходящее из неосновного в основное;специфицированная сущность;недостаточно специфицированная сущность;достаточно специфицированная сущность;средне специфицированная сущность;структура;файл;событие в sc-памяти*;элементарное событие в sc-памяти*;событие добавления sc-дуги, выходящей из заданного sc-элемента*;событие добавления sc-дуги, входящей в заданный sc-элемент*;событие добавления sc-ребра, инцидентного заданному sc-элементу*;событие удаления sc-дуги, выходящей из заданного sc-элемента*;событие удаления sc-дуги, входящей в заданный sc-элемент*;событие удаления sc-ребра, инцидентного заданному sc-элементу*;событие удаления sc-элемента*;событие изменения содержимого файла*}

\scnheader{sc-элемент}
\scnreltoset{разбиение}{наcтоящий sc-элемент;логически удаленный sc-элемент}

\scnheader{наcтоящий sc-элемент}
\scniselement{ситуативное множество}

\scnheader{логически удаленный sc-элемент}
\scniselement{ситуативное множество}

\scnheader{число}
\scnsubdividing{невычисленное число;вычисленное число}

\scnheader{невычисленное число}
\scniselement{ситуативное множество}

\scnheader{вычисленное число}

\scnheader{понятие}
\scnsubdividing{основное понятие;неосновное понятие;понятие, переходящее из основного в неосновное;понятие, переходящее из неосновного в основное}

\scnheader{основное понятие}
\scnidtf{основное понятие для данной ostis-системы}
\scniselement{ситуативное множество}
\scnexplanation{К \textbf{\textit{основным понятиям}} относятся те понятия, которые активно используются в системе и могут быть ключевыми элементами sc-агентов. К \textbf{\textit{основным понятиям}} относятся также все неопределяемые понятия.}

\scnheader{неосновное понятие}
\scnidtf{дополнительное понятие}
\scnidtf{вспомогательное понятие}
\scnidtf{неосновное понятие для данной ostis-системы}
\scniselement{ситуативное множество}
\scnexplanation{Каждое \textbf{\textit{неосновное понятие}} должно быть строго определено на основе \textit{основных понятий}. Такие \textbf{\textit{неосновные понятия}} используются только для понимания и правильного восприятия вводимой информации, в том числе, для выравнивания онтологий. Ключевым элементом \textit{sc-агентов} \textbf{\textit{неосновные понятия}} быть не могут.}
\scntext{правило идентификации экземпляров}{В случае, когда некоторое понятие полностью перешло из \textit{основных понятий} в неосновные, то есть стало \textbf{\textit{неосновным понятием}}, и соответствующее ему \textit{основное понятие} (через которое оно определяется) в рамках некоторого внешнего языка имеет одинаковый с ним основной идентификатор, то к идентификатору \textbf{\textit{неосновного понятия}} спереди добавляется знак \#. Если при этом соответствуюшее \textit{основное понятие} имеет в идентификаторе знак \$, добавленный в процессе перехода, то этот знак удаляется. Если указанные понятия имеют разные основные идентификаторы в рамках этого внешнего языка, то никаких дополнительных средств идентификации не используется.

Например:\\
\textit{\#трансляция sc-текста}\\
\textit{\#scp-программа}}

\scnheader{понятие, переходящее из основного в неосновное}
\scniselement{ситуативное множество}

\scnheader{понятие, переходящее из неосновного в основное}
\scniselement{ситуативное множество}
\scntext{правило идентификации экземпляров}{В случае, когда текущее \textit{основное понятие} и соответствующее ему \textbf{\textit{понятие, переходящее из неосновного в основное}} в рамках некоторого внешнего языка имеют одинаковый основной идентификатор, то к идентификатору понятия, переходящего из неосновного в основное спереди добавляется знак \$. Если указанные понятия имеют разные основные идентификаторы в рамках этого внешнего языка, то никаких дополнительных средств идентификации не используется.

Например:\\
\textit{\$трансляция sc-текста}\\
\textit{\$scp-программа}}

\scnheader{специфицированная сущность}
\scnsubdividing{недостаточно специфицированная сущность;достаточно специфицированная сущность;средне специфицированная сущность}

\scnheader{недостаточно специфицированная сущность}

\scnheader{достаточно специфицированная сущность}
\scnexplanation{К \textbf{\textit{достаточно специфицированным сущностям}} предъявляются следующие требования:
\begin{scnitemize}
    \item если сущность не является понятием, то для нее должны быть указаны
    \begin{scnenumerate}
    \item различные варианты обозначающих ее внешних знаков;
    \item классы, которым она принадлежит;
    \item связки, которыми она связана с другими сущностями (с указанием соответствующего отношения);
    \item значения параметров, которыми она обладает;
    \item те разделы базы знаний, в которых указанная сущность является ключевой;
    \item предметные области, в которые данная сущность входит.
    \end{scnenumerate}
    \item если специфицированная сущность является понятием, то для нее должны быть указаны:
    \begin{scnenumerate}
    \item различные варианты внешних обозначений этого понятия;
    \item предметные области, в которых это понятие исследуется;
    \item определение понятия;
    \item пояснения
    \item разделы базы знаний, в которых это понятие является ключевым;
    \item типичная семантическая окрестность – пример экземпляра понятия.
    \end{scnenumerate}
\end{scnitemize}}

\scnheader{средне специфицированная сущность}

\scnheader{структура}
\scnsubdividing{сформированная структура;несформированная структура}
\scnsubdividing{недостаточно сформированная структура;достаточно сформированная структура;структура, имеющая средний уровень сформированности}

\scnheader{файл}
\scnsubdividing{недостаточно сформированный внутренний файл;достаточно сформированный внутренний файл;внутренний файл, имеющий средний уровень сформированности}

\scnheader{событие в sc-памяти*}
\scnrelto{включение}{событие*}

\scnheader{элементарное событие в sc-памяти*}
\scnrelto{включение}{событие в sc-памяти*}
\scnexplanation{Под \textbf{\textit{элементарным событием в sc-памяти*}} понимается \textit{событие*}, соответствующее некоторому действию в \textit{sc-памяти}, в результате выполнения которого изменяется состояние только одного \textit{sc-элемента}.}
\scnsubdividing{событие добавления sc-дуги, выходящей из заданного sc-элемента*
;событие добавления sc-дуги, входящей в заданный sc-элемент*;событие добавления sc-ребра, инцидентного заданному sc-элементу*;событие удаления sc-дуги, выходящей из заданного sc-элемента*;событие удаления sc-дуги, входящей в заданный sc-элемент*;событие удаления sc-ребра, инцидентного заданному sc-элементу*;событие удаления sc-элемента*;событие изменения содержимого файла*}

\scnheader{событие добавления sc-дуги, выходящей из заданного sc-элемента*}

\scnheader{событие добавления sc-дуги, входящей в заданный sc-элемент*}

\scnheader{событие добавления sc-ребра, инцидентного заданному sc-элементу*}

\scnheader{событие удаления sc-дуги, выходящей из заданного sc-элемента*}

\scnheader{событие удаления sc-дуги, входящей в заданный sc-элемент*}

\scnheader{событие удаления sc-ребра, инцидентного заданному sc-элементу*}

\scnheader{событие удаления sc-элемента*}

\scnheader{событие изменения содержимого файла*}

\scnendstruct

\end{SCn}