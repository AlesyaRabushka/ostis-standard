\begin{SCn}

\scnsectionheader{\currentname}

\scnstartsubstruct

\scnheader{Предметная область внешних информационных конструкций и файлов ostis-систем}
\scniselement{предметная область}
\scnsdmainclasssingle{файл}
\scnsdclass{внешний язык;естественный язык;Русский язык;Английский язык;изображение;класс синтаксически эквивалентных информационных конструкций;максимальный класс синтаксически эквивалентных информационных конструкций}
\scnsdrelation{трансляция sc-текста*}

\scnheader{файл}
\scnidtf{знак файла}
\scnidtf{sc-знак файла}
\scnidtf{знак информационной конструкции, внешней по отношению к sc-памяти}
\scnidtf{sc-ссылка}
\scnexplanation{Под \textbf{\textit{файлом}} понимается любая информационная конструкция, внешняя по отношению к \textit{sc-памяти}, т.е., не являющаяся \textit{sc-текстом}. При этом каждому \textbf{\textit{файлу}} может быть поставлен в соответствие семантически эквивалентный \textit{sc-текст}.}

\scnheader{внешний язык}
\scnidtf{язык, внешний по отношению к sc-памяти}
\scnrelto{семейство подмножеств}{файл}
\scnexplanation{Под \textbf{\textit{внешним языком}} понимается множество \textit{файлов}, имеющих общую синтаксическую структуру.}

\scnheader{естественный язык}
\scnidtf{язык диалога с пользователем}
\scnsubset{внешний язык}
\scnhaselement{Русский язык}
\scnhaselement{Английский язык}
\scnexplanation{Под конкретным \textbf{\textit{естественным языком}} понимается некоторое множество \textit{файлов} (например, идентификаторов, естественно-языковых пояснений и т.д.), которые используются при диалоге с тем или иным пользователем, режим ведения которого он может выбрать.

В этом смысле некоторые фрагменты, такие как, например обозначения \textbf{sin}, \textbf{cos}, \textbf{a.e.} и т.п., могут входить в несколько \textbf{\textit{естественных языков}}, поскольку используются при диалоге, но исторически являться фрагментами другого языка.}

\scnheader{Русский язык}

\scnheader{Английский язык}

\scnheader{трансляция sc-текста*}
\scnexplanation{Связки отношения \textbf{\textit{трансляция sc-текста*}} связывают некоторый \textit{sc-текст} и \textit{файл}, который является семантическим эквивалентом этого \textit{sc-текста} на некотором внешнем языке (в том числе, например, языке геометрических чертежей, математических формул и т.д.).}

\scnheader{изображение}
\scnidtf{графический файл}
\scnidtf{графическая несимвольная информационная конструкция}
\scnsubset{файл}

\scnheader{класс синтаксически эквивалентных информационных конструкций}
\scnsuperset{максимальный класс синтаксически эквивалентных информационных конструкций}
\scnexplanation{\textbf{\textit{класс синтаксически эквивалентных информационных конструкций}} - \textit{класс} информационных конструкций имеющих общие синтаксические свойства без учета различий в форматах их кодирования, в используемых шрифтах, в форматировании и размещении.}

\scnheader{максимальный класс синтаксически эквивалентных информационных конструкций}
\scnexplanation{\textbf{\textit{максимальный класс синтаксически эквивалентных информационных конструкций}} - \textit{класс} всевозможных конструкций, имеющих общие синтаксические свойства без учета различий в форматах их кодирования, в используемых шрифтах, в форматировании и размещении.}

\scnendstruct

\end{SCn}