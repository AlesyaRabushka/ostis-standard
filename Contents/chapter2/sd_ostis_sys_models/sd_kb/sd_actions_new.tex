\begin{SCn}

\scnsectionheader{Предметная область и онтология действий, задач, планов и методов}

\scnstartsubstruct

\scnrelfromvector{ключевые знаки}{действие;класс действий;метод;класс методов;деятельность;вид деятельности}

\bigskip
\scnfragmentcaptiontext{Понятие действия}

\scnheader{действие}
\scnidtf{целенаправленный процесс, выполняемый одним или несколькими субъектами (кибернетическими системами с возможным применением некоторых инструментов}
\scnidtf{воздействие}
\scnidtf{акция}
\scnidtf{акт}
\scnidtf{операция}
\scnidtf{\uline{процесс} воздействия некоторой (возможно, коллективной) сущности (субъекта воздействия) на некоторую одну или несколько сущностей (объектов воздействия -- исходных объектов (аргументов) или целевых (создаваемых или модифицируемых) объектов)}
\scnsubset{процесс}
\scnidtf{целенаправленный ("осознанный"{}) процесс выполняемый (управляемый, реализуемый) неким субъектом}
\scnidtf{акция реализации некоторого замысла}
\scnidtf{преднамеренная акция}

\bigskip
\scnfragmentcaptiontext{Типология действий}

\scnheader{действие}
\scnsuperset{элементарное действие}
	\scnaddlevel{1}
	\scnidtf{действие, выполнение которого не требует его декомпозиции на множество поддействий (частных действий, действий более низкого уровня)}
	\scnexplanation{Элементарное действие выполняется одним индивидуальным субъектом и является либо элементарным действием, выполняемым в памяти этого субъекта (элементарным действием его "процессора"{}), либо элементарным действием одного из его эффекторов.}
	\scnaddlevel{-1}
\scnsuperset{сложное действие}
	\scnaddlevel{1}
	\scnsuperset{легко выполнимое сложное действие}
	\scnsuperset{трудно выполнимое действие}
	\scnaddlevel{-1}
\scnsubdividing{индивидуальное действие\\
	\scnaddlevel{1}
	\scnidtf{действие, выполняемое индивидуальной кибернетической системой}
	\scnsuperset{индивидуальное действие, выполняемое человеком}
	\scnsuperset{индивидуальное действие, выполняемое компьютерной системой}
	\scnaddlevel{-1}
;коллективное действие\\
	\scnaddlevel{1}
	\scnidtf{действие, выполняемое коллективом кибернетических систем (коллективом субъектов)}
	\scnsuperset{действие, выполняемое коллективом людей}
	\scnsuperset{действие, выполняемое коллективом индивидуальных компьютерных систем}
	\scnsuperset{действие, выполняемое коллективом людей и индивидуальных компьютерных систем}
		\scnaddlevel{1}
		\scnsuperset{действие, выполняемое Экосистемой OSTIS}
		\scnaddhind{-1}
		\scnsuperset{действие, выполняемое одним человеком во взаимодействии с одной индивидуальной компьютерной системой}
		\scnaddlevel{-1}
	\scnaddlevel{-1}}
\scnsubdividing{действие, выполняемое кибернетической системой в собственной памяти
;действие, выполняемое кибернетической системой в своей внешней среде
;действие, выполняемое кибернетической системой над своей физической оболочкой}


\scnheader{действие, выполняемое кибернетической системой в собственной памяти}
\scnidtf{действие, выполняемое в памяти}
\scnidtf{действие кибернетической системы, направленное на обработку информации, хранимой в её памяти}
\scnsuperset{действие, выполняемое кибернетической системой в собственной памяти и направленное на организацию её деятельности во внешней среде}
	\scnaddlevel{1}
	\scnidtf{действие, выполняемое кибернетической системой в её памяти и направленное на организацию её деятельности во внешней среде и в конечном счете -- на сенсо-моторную координацию деятельности её эффекторов}
	\scnaddlevel{-1}

\scnheader{действие, выполняемое кибернетической системой в своей внешней среде}  
\scnidtf{действие, выполняемое кибернетической системой в её внешней среде и осуществляемое (на самом низком уровне) эффекторами этой кибернетической системы}
   
\scnheader{действие}
\scnsubset{процесс}
\scnrelfrom{разбиение}{Темпоральный признак классификации действий}
\scnaddlevel{1}
\scneqtoset{настоящее действие\\
    \scnaddlevel{1}
    \scnidtf{активное действие}
    \scnidtf{действие, выполняемое в текущий момент}
    \scnaddlevel{-1}
;прошлое действие\\
    \scnaddlevel{1}
    \scnidtf{выполненное, завершенное действие}
    \scnaddlevel{-1}
;инициированное действие\\
    \scnaddlevel{1}
    \scnidtf{действие, ожидающее начала своего выполнение}
    \scnaddlevel{-1}
;прерванное действие\\
    \scnaddlevel{1}
    \scnidtf{действие, ожидающее продолжения своего выполнения}
    \scnaddlevel{-1}
;планируемое действие\\
    \scnaddlevel{1}
    \scnidtf{будущее действие}
    \scnaddlevel{-1}
;возможное действие}
\scnaddlevel{-1}

\scnrelfrom{разбиение}{Признак классификаций действий по их длительности}
\scnaddlevel{1}
\scneqtoset{краткосрочное действие\\
    \scnaddlevel{1}
    \scnidtf{тактическое действие}
    \scnaddlevel{-1}
;долгосрочное действие\\
    \scnaddlevel{1}
    \scnidtf{стратегическое действие}
    \scnaddlevel{-1}
;перманентное действие\\
    \scnaddlevel{1}
    \scnidtf{действие, постоянно выполняемое соответствующим субъектом, пока этот субъект существует}
    \scnaddlevel{-1}}
\scnaddlevel{-1}

\scnsuperset{действие, у которого цель известна, но задана не совсем точно}
	\scnaddlevel{1}
	\scnsuperset{действие, направленное на выявление противоречий в базе знаний}
		\scnaddlevel{1}
		\scnnote{Это действие декомпозируется на несколько самостоятельных поддействий, каждое из которых выявляет (локализует) противоречия (ошибки) конкретного формализуемого вида, для которого в базе знаний существует точное определение.}
		\scnaddlevel{-1}
	\scnaddlevel{-1}
\scnsuperset{действие, для которого априори не известен метод, обеспечивающий его выполнение}
	\scnaddlevel{1}
	\scnnote{Соответствующий метод либо не найден, либо его вообще нет в памяти.}
	\scnaddlevel{-1}

   
\scnheader{сложное действие}
\scnidtf{неэлементарное действие}
\scnidtf{действие выполнение которого требует декомпозиции этого действия на множество его \uline{поддействий}, т.е. частных действий более низкого уровня}
\scnnote{Декомпозиция сложного действия на поддействия может иметь весьма сложный иерархический вид с большим числом уровней иерархии, т.е. поддействиями \textit{сложного действия} могут также \textit{сложные действия}. Уровень сложности действия можно определять (1) общим числом его поддействий и (2) числом уровней иерархии этих поддействий.}
\scnnote{Декомпозиция \textit{сложного действия} на поддействия в конечном счете должна завершаться элементарными действиями.}
\scnnote{Темпоральные соотношения между поддействиями сложного действия могут быть самые различные, но в простейшем случае сложное действие представляет собой строгую последователность действий более низкого уровня иерархии.}


\scnheader{легко выполнимое сложное действие} 
\scnidtf{сложное действие, для выполнения которого известен соответствующий \textbf{\textit{метод}} и соответствующие этому методу исходные данные, а также (для действий, выполняемых во внешней среде) имеются в наличии все необходимые исходные объекты (расходные материалы и комплектация), а также средства (инструменты)}

\scnheader{трудно выполнимое действие}
\scnidtf{сложное действие, для выполнения которого в текущий момент либо не известен соответствующий метод, либо возможные методы известны, но отсутствуют условия их применения}

\bigskip
\scnfragmentcaptiontext{Отношения, заданные на множестве действий}

\scnheader{отношение, заданное на множестве*(действие)}
\scnidtf{отношение, заданное на множестве действий*}
\scnhaselement{\scnkeyword{поддействие*}}
	\scnaddlevel{1}
	\scnrelboth{обратное отношение}{наддействие*}
	\scnidtf{быть действием, являющимся частью заданного действия более высокого уровня иерархии*}
	\scnidtf{быть действием, направленным на решение задачи, которая является подзадачей по отношению к задаче, решение которой осуществляется заданным действием*}
	\scnsuperset{\scnkeyword{непосредственное поддействие}*}
		\scnaddlevel{1}
		\scnidtf{быть таким поддействием заданного действия, для которого не существует наддействия, которое было бы также поддействием заданного действия*}
		\scnaddlevel{-1}
	\scnaddlevel{-1}
\scnhaselement{\scnkeyword{последовательность действий}*}
\scnaddlevel{1}
\scnidtf{порядок выполнения (инициирования) действий*}
\scnidtf{передача управления от действия к действию*}
\scnidtf{goto*}
\scnidtf{Бинарное ориентированное отношение, каждая связка которого связывает два действия, первое из которых является действием, событие завершения которого является необходимым (но не обязательно достаточным) условием инициирования (начала выполнения) второго действия}
\scnnote{Связки Отношения \textit{последовательность действий*} могут иметь совпадающие первые компоненты. Это означает, что завершение действия может быть условием инициирования (передачи управления) сразу нескольким действиям, т.е. означает распараллеливание выполняемого сложного действия, состоящего из нескольких поддействий. Связки указанного Отношения могут иметь также совпадающие вторые компоненты. Это означает, что \uline{достаточным} условием инициирования действия, являющегося вторым компонентом указанных связок, является событие завершения всех непосредственно предшествующих действий.}
\scnnote{Безуспешно выполненное действие считается невыполненным и, следовательно, не может передать управление последующим действиям. Примерами таких действий являются действия, проверки наличия в текущий момент тех или иных ситуаций (условий). Если указанные ситуации (условия) альтернативны, то речь идет об условной передаче управления.}
\scnnote{Возможность безуспешного выполнения некоторых действий можно и нужно предусматривать при построении \textit{планов} выполнения сложных действий. Но при представлении \textit{протоколов}, описывающих то, как эти действия были выполнены на самом деле, все можно упростить -- можно удалить альтернативные "ветки"{} (цепочки) действий, которые следуют после безуспешно выполненных действий.}
\scnnote{Своего рода "штатным"{} вариантом синхронизации параллельно (одновременно) выполняемых или альтернативных "веток"{} (цепочек) с помощью Отношения \textit{последовательность действий*} является то, что инициирование действия осуществляется тогда и только тогда, когда завершается выполнение \uline{всех} непосредственно предшествующих ему действий. Для обеспечения более сложных вариантов синхронизации действий необходимо ввести два класса специальных \textit{действий}:
\begin{scnitemize}
\item \textit{и-синхронизация действий}
\item \textit{или-синхронизация действий}
\end{scnitemize}	
\scnaddlevel{-1}}

\scnheader{и-синхронизация действий}
\scnidtf{действие, которое успешно выполняется сразу после завершения выполнения всех непосредственно предшествующих действий и при этом ничего другого при выполнении этого действия не происходит}

\scnheader{или-синхронизация действий}
\scnidtf{действие, которое успешно выполняется сразу после завершения выполнения хотя бы одного непосредственно предшествующего действия}
	
\scnheader{субъект действия'}
\scnidtf{сущность, воздействующая на некоторую другую сущность в процессе некоторого действия'}
\scnidtf{сущность, создающая \uline{причину} изменений другой сущности (объекта действия)'}
\scnidtf{быть субъектом данного действия'}
\scnsuperset{субъект неосознанного воздействия'}
\scnsuperset{субъект осознанного воздействия'}
\scnaddlevel{1}
\scnidtf{субъект целенаправленного, активного воздействия'}
\scnaddlevel{-1}


\scnheader{объект действия'}
\scnidtf{аргумент действия'}
\scnidtf{сущность, на которую осуществляется воздействие в рамках заданного действия'}
\scnidtf{сущность, являющаяся в рамках заданного действия исходным условием (аргументов), необходимым для выполнения этого действия'}
\scnnote{Для разных действий количество объектов действий может быть различным.}

\scnheader{инструмент воздействия'}
\scnidtf{то, с помощью чего субъект осуществляет воздействие*}


\scnheader{продукт'}
\scnidtf{быть продуктом заданного действия*}
\scnidtf{результат*}
\scnidtf{"сухой"{} остаток*}
\scnidtf{то, ради чего может быть выполнено, выполняется или будет выполняться заданное действие*}
\scnnote{Продуктом действия может быть некоторая материальная сущность, некоторое множество (тираж) одинаковых материальных сущностей, некоторая информационная конструкция}
\scnrelboth{следует отличать}{цель*}
	\scnaddlevel{1}
	\scnidtf{спецификация продукта*}
	\scnnote{Следует также отмечать то, что является непосредственно результатом (продуктом) некоторого действия и то, что является предварительной (исходной, стартовой) спецификацией этого продукта.}
	\scnaddlevel{-1}

\bigskip
\scnfragmentcaptiontext{Отношения, заданные на множестве действий и связывающие действия с различного вида их спецификациями}
\bigskip

\scnheader{отношение, заданное на множестве*(действие)}
\scnhaselement{\scnkeyword{задача}*}
	\scnaddlevel{1}
	\scnidtf{формулировка задачи*}
	\scnidtf{спецификация действия, уточняющая то, \uline{что} должно быть сделано*}
\scnsubdividing{декларативная формулировка задачи*;процедурная формулировка задачи*}
\scnrelfrom{второй домен}{\scnkeyword{задача}}
	\scnaddlevel{1}
	\scnsuperset{задача обработки базы знаний}
	\scnsuperset{задача обработки файлов}
	\scnsuperset{задача, решаемая кибернетической системой во внешней среде}
	\scnsuperset{задача, решаемая кибернетической системой в собственной физической оболочке}
	\scnaddlevel{-1}
		\scnaddlevel{-1}
	
\scnhaselement{\scnkeyword{декларативная формулировка задачи}*}
\scnaddlevel{1}
\scnidtf{описание исходной ситуации и целевой ситуации специфицируемого действия*}
\scnexplanation{декларативная формулировка задачи включает в себя:
	\begin{scnitemize}
	\item связку отношения \textit{цель}*, связывающую специфицируемое действие с описанием целевой ситуации;
	\item само описание целевой ситуации;
	\item связку отношения \textit{исходная ситуация*}, связывающую специфицируемое действие с описанием исходной ситуации;
	\item непосредственно описание исходной ситуации;
	\item указание контекста (области решения) задачи.
	\end{scnitemize}
При этом указание и описание исходной ситуации может отсутствовать.}
\scnaddlevel{-1}

\scnhaselement{\scnkeyword{процедурная формулировка задачи}*}
\scnidtfexp{указание
	\begin{scnitemize}
	\item \textit{класса действий}, которому принадлежит специфицируемое \textit{действие}, а также 
	\item \textit{субъекта} или субъектов, выполняющих это действие (с дополнительным указанием роли каждого участвующего субъекта);
	\item \textit{объекта} или объектов, над которыми осуществляется действие (с указанием "роли"{} каждого такого объекта);
	\item используемых материалов;
	\item используемых инструментов (инструментальных средств);
	\item дополнительных темпоральных характеристик специфицируемого действия (сроки, длительность);
	\item приоритета (важности) специфицируемого действия.
	\end{scnitemize}}

\scnhaselement{\scnkeyword{исходная ситуация}*}
	\scnaddlevel{1}
\scnidtf{исходная ситуация, соответствующая заданному действию*}
\scnsubset{спецификация*} 
\scnrelfrom{второй домен}{ситуация}
\scnidtf{начальная ситуация*}
\scnidtf{описание того, что дано (что имеется) перед началом выполнения заданного (специфицируемого) действия*}
	\scnaddlevel{-1}

\scnhaselement{\scnkeyword{цель}*}
	\scnaddlevel{1}
	\scnidtf{целевая ситуация*}
	\scnsubset{спецификация*}
	\scnrelfrom{второй домен}{ситуация}
	\scnidtf{описание того, что требуется получить (какая ситуация должна быть достигнута) в результате выполнения заданного (специфицируемого) действия*}
	\scnidtf{цель выполнения действия*}
	\scnidtf{интенция, стремление, намерение, замысел, желание, устремление, направленность действия*}
	\scnaddlevel{-1}

\scnhaselement{\scnkeyword{план}*}
	\scnaddlevel{1}
	\scnidtf{план выполнения действия*}
	\scnidtf{процедурная спецификация выполнения действия*}
	\scnidtf{декомпозиция выполняемого действия на систему последовательно/параллельно выполняемых \uline{под}действий*}
	\scnidtf{описание того, как может быть выполнено соответствующее сложное действие*}
	\scnidtf{спецификация соответствующего действия, уточняющая то, \uline{как} предполагается выполнять это действие*}
	\scnidtf{план решения задачи (выполнения сложного действия) путем описания последовательности выполнения поддействий с описанием того, как передается управление от одних поддействий к другим, как осуществляется распараллеливание, как организуется выполнение циклов*}
	\scnaddlevel{-1}

\scnhaselement{\scnkeyword{декларативная спецификация выполнения действия*}}
	\scnaddlevel{1}
	\scnsubset{спецификация*}
	\scnrelfrom{второй домен}{декларативная спецификация выполнения действия}
	\scnaddlevel{1}
	\scnexplanation{В состав такой спецификации действия входят:
		\begin{scnitemize}
		\item \scnkeyword{контекст* \textit{действия}}, содержащий информацию, достаточную для его выполнения;
		\item \scnkeyword{множество используемых методов*} и инструментов, достаточных для выполнения действия.
		\end{scnitemize}}
	\scnidtf{непроцедурное описание выполнения сложного (неэлементарного) действия}
	\scnsuperset{функциональная спецификация выполнения действия}
	\scnsuperset{логическая спецификация выполнения действия}
	\scnnote{\textit{декларативная спецификация выполнения действия} -- это такой выделенный фрагмент \textit{базы знаний} (такой \textit{контекст*} выполнения соответствующего конкретного \textit{действия}), которого \uline{достаточно} для выполнения этого \textit{действия} с помощью заданного множества \textit{методов}, используемых в рамках указанного \textit{контекста*}. При этом важна \uline{минимизация} и самого \textit{контекста*} и \textit{множества используемых методов*}.}
	\scnaddlevel{-1}
	\scnaddlevel{-1}
\scnhaselement{\scnkeyword{контекст*}}
\scnhaselement{\scnkeyword{множество используемых методов*}}

\scnhaselement{\scnkeyword{протокол}*}
\scnaddlevel{1}
\scnidtf{декомпозиция выполненного действия на систему последовательно-параллельно выполненных его \uline{под}действий*}
\scnidtf{описание того, как действительно было выполнено соответствующее действие и, в частности, описание последовательности соответствующих ситуаций и событий*}
\scnidtf{протокол выполнения сложного действия, включающий в себя протоколы выполнения всех поддействий этого действия*}
\scnidtf{протокол решения задачи*}
\scnidtf{история решения выполненной задачи*}
\scnaddlevel{-1}

\scnhaselement{\scnkeyword{результативная часть протокола}*}
\scnaddlevel{1}
\scnidtf{часть протокола соответствующего выполненного действия, которая включает в себя только те его поддействия, которые действительно внесли вклад в построение результата ("сухого остатка"{}) этого выполненного действия*}
\scnnote{протокол выполненного действия и результативная часть этого протокола могут сильно отличаться. Примером тому является, например, соотношение между протоколом доказательства некоторой конкретной теоремы и результативной частью этого протокола, которая является подтверждением корректности проведенного доказательства и, соответственно, обоснованием истинности доказанной теоремы.}
\scnaddlevel{-1}

\scnheader{задача}
\scnidtf{спецификация действия, которое выполнилось, выполняется или может быть выполнено соответствующей кибернетической системой}
\scnnote{Каждой задаче и, соответственно, каждому специфицируемому действию соответствует определенная кибернетическая система, являющаяся субъектом, выполняющим это действие.}
\scnsubset{знание}
\scnnote{Каждая \textit{задача} - это \textit{знание}, описывающее то какое действие возможно потребуется выполнить.}
\scnsuperset{инициированная задача}
\scnaddlevel{1}
\scnidtf{формулировка задачи, которая подлежит выполнению}
\scnaddlevel{-1}
\scnidtf{спецификация (описание) соответствующего действия}
\scnidtf{задачная ситуация}
\scnidtf{формулировка задачи}
\scnidtf{постановка задачи}

\scnsuperset{декларативная формулировка задачи}
\scnaddlevel{1}
\scnidtf{задача, в формулировке которой явно указывается (описывается) целевая ситуация, т.е. то, что является результатом выполнения (решения) данной задачи}
\scnidtf{декларативаная формулировка задачи}
\scnaddlevel{-1}
\scnsuperset{процедурная формулировка задачи}
\scnaddlevel{1}
\scnidtf{процедурная формулировка задачи}
\scnidtf{задача, в формулировке которой явно указывается характеристика действия, специфицируемого этой задачей, а именно, например, указывается:
\begin{scnitemize}
\item субъект или субъекты, выполняющие это действие,
\item объекты, над которыми действие выполняется, - аргументы действия,
\item инструменты, с помощью которых выполняется действие,
\item момент и, возможно, дополнительные условия начала и завершения выполнения действия
\end{scnitemize}}
\scnaddlevel{-1}
\scnsuperset{декларативно-процедурная формулировка задачи}
\scnaddlevel{1}
\scnidtf{задача, в формулировке которой присутствуют как декларативные (целевые), так и процедурные аспекты}
\scnaddlevel{-1}
\scnnote{От качества (корректности и полноты) формулировки задачи, т.е. спецификации соответствующего действия, во многом зависит качество (эффективность) выполнения этого действия, т.е. качество процесса решения указанной задачи.}
\scnsuperset{проблема}
\scnaddlevel{1}
\scnidtf{проблемная задача}
\scnidtf{сложная, трудно решаемая задача}
\scnsuperset{изобретательская задача}
\scnaddlevel{-1}

\scnheader{декларативная формулировка задачи}
\scnrelto{второй домен}{декларативная формулировка задачи*}
\scnidtf{описание исходной (начальной) ситуации, являющейся условием выполнения соответствующего действия и целевой (конечной) ситуации, являющейся результатом выполнения этого действия}
\scnidtf{семантическая спецификация действия}
\scnnote{Формулировка \textit{задачи} может не содержать указания контекста (области решения) \textit{задачи} (в этом случае областью решения \textit{задачи} считается либо вся \textit{база знаний}, либо ее согласованная часть), а также может не содержать либо описания исходной ситуации, либо описания целевой ситуации. Так, например, описания целевой ситуации для явно специфицированного противоречия, обнаруженного в \textit{базе знаний} не требуется.}
\scnidtf{формулировка (описание) задачной ситуации с явным или неявным описанием контекста (условий) выполнения специфицируемого действия, а также результата выполнения этого действия}
\scnidtf{явное или неявное описание 
\begin{scnitemize}
\item того, что \uline{дано} - исходные данные, условия выполнения специфируемого действия,
\item того, что \uline{требуется} - формулировка цели, результата выполнения указанного действия
\end{scnitemize}}

\scnhaselementrole{пример}{\scnfilescg{figures/sd_task/declarative_task_statement.png}}

\scnnote{Выполнение данного действия сведется к следующим \uline{событиям}:
\begin{scnitemize}
\item для числа \textit{с} будет сгенерирован уникальный идентификатор, являющийся его представлением в соответствующей системе счисления
\item будет сгенерирована константная настоящая позитивная пара принадлежности, соединяющая узел "\textit{вычислено}"{} с узлом "\textit{с}"{}
\item удалится константная будущая позитивная пара принадлежности, а также константная настоящая нечеткая пара принадлежности, выходящие из узла "\textit{вычислено}".
\end{scnitemize}
Таким образом, после выполнения действия \uline{все} \uline{будущие} сущности, входящие в целевую ситуацию, становятся \uline{настоящими} сущностями, а некоторые \uline{настоящие} сущности, входящие в исходную ситуацию, становятся \uline{прошлыми}.}

\scnheader{задача}
\scnsuperset{задача, решаемая в памяти кибернетической системы}
\scnaddlevel{1}
\scnsuperset{задача, решаемая в памяти индивидуальной кибернетической системы}
\scnsuperset{задача, решаемая в общей памяти многоагентной системы}
\scnidtf{информационная задача}
\scnidtf{задача, направленная либо на \uline{генерацию} или поиск информации, удовлетворяющей заданным требованиям, либо на некоторое \uline{преобразование} заданной информации}
\scnsuperset{математическая  задача}
\scnaddlevel{-1}
\scnsuperset{элементарная информационная задача}
\scnsuperset{простая информационная задача}
\scnsuperset{проблемная информационная задача}
\scnaddlevel{1}
\scnidtf{интеллектуальная информационная задача}
\scnsuperset{проблема Гильберта}
\scnaddlevel{-1}

\scnheader{вопрос}
\scnidtf{запрос}
\scnsubset{задача, решаемая в памяти кибернетической системы}
\scnidtf{непроцедурная формулировка задачи на поиск (в текущем состоянии базы знаний) или на генерацию знания, удовлетворяющего заданным требованиям}
\scnsuperset{вопрос - что это такое}
\scnsuperset{вопрос - почему}
\scnsuperset{вопрос - зачем}
\scnsuperset{вопрос - как}
\scnaddlevel{1}
\scnidtf{каким способом}
\scnidtf{запрос метода (способа) решения заданного (указываемого) вида задач или класса задач либо, плана решения конкретной указываемой задачи}
\scnaddlevel{-1}

\scnheader{спецификация*}
\scnsuperset{сужение отношения по первому домену*(спецификация*; действие)*}
\scnaddlevel{1}
\scnidtftext{часто используемый sc-идентификатор}{спецификация действия*}
\scnsubdividing{
задача*\\
\scnaddlevel{1}
\scnsubdividing{
декларативная формулировка задачи*\\
\scnaddlevel{1}
\scnrelfrom{второй домен}{декларативная формулировка задачи}
\scnaddlevel{-1}
;процедурная формулировка задачи*\\
\scnaddlevel{1}
\scnrelfrom{второй домен}{процедурная формулировка задачи}
\scnaddlevel{-1}}
\scnaddlevel{-1}
;исходная ситуация*
;цель*
;план*
;декларативная спецификация выполнения действия*
;контекст действия*
\scnaddlevel{1}
\scnidtf{информационный ресурс необходимый для выполнения заданного действия*}
\scnaddlevel{-1}
;множество используемых методов*
\scnaddlevel{1}
\scnidtf{множество методов, используемых для выполнения заданного действия*}
\scnidtf{операционный (функциональный) ресурс, необходимый для выполнения заданного действия*}
\scnaddlevel{-1}
;протокол*
;результативная часть протокола*}
\scnaddlevel{-1}

\scnnote{Таким образом, каждому действию может быть поставлен в соответствие целый ряд видов спецификации этого действия, которые описывают различные аспекты специфицируемого действия - и то, что является причиной (условием) инициирования этого действия, и то, что является результатом ("сухим остатком") его выполнения, и то, и то, с помощью таких ресурсов оно может быть выполнено, и то, как управлять этими ресурсами в процессе выполнения действия, и то, как на самом деле это действие было выполнено.}

\scnheader{трансформация отношения путем обобщения компонентов его связок*(спецификация*)}
\scnhaselementvector{действие;задача}
\scnhaselementvector{действие;ситуация}
\scnhaselementvector{действие;декларативная формулировка задачи}
\scnhaselementvector{действие;процедурная формулировка задачи}
\scnhaselementvector{действие;план}
\scnhaselementvector{действие;декларативная спецификация выполнения действия}
\scnhaselementvector{действие;протокол}
\scnhaselementvector{действие;результативная часть протокола}

\scnheader{сужение отношения по первому домену*(спецификация*; действие)*}
\scnnote{\textit{спецификацию действия} (базовое описание действия) условно можно разбить на следующие части:
\begin{scnitemize}
\item описание состояния действия в текущий момент времени -- действие может принадлежать:
\begin{scnitemizeii}
\item либо классу \textit{прогнозируемых сущностей} (в случае действий -- это планируемые действия, которые могут быть, но не обязательно выполняться в будущем);
\item либо классу \textit{настоящих сущностей}, т.е. сущностей, существующих в настоящий (текущий) момент времени;
\item либо классу \textit{прошлых сущностей}, завершивших свое существование (в случае действий - это действия, выполнение которых уже завершено);
\end{scnitemizeii}
\item формулировки \textit{задачи}, которая должна быть решена в результате выполнения специфицируемого действия. Такая формулировка представляет собой логико-семантическое описание \textit{задачной продукции}, включающей в себя:
\begin{scnitemizeii}
\item описание \textit{исходной ситуации} и/или события (исходных условий того, что должно быть дано, исходных данных, исходного контекста). Для \textit{действий во внешней среде} (действий/задач, выполняемых во внешней среде) в описании \textit{исходной ситуации} должно быть включено описание необходимых для решения задачи материальных ресурсов (сырья, комплектации) с указанием их количества;
\item описание \textit{целевой ситуации} и/или события (того, что требуется получить в результате решения данной задачи);
\item указание дополнительных \textit{инструментальных средств}, используемых для выполнения специфицируемого действия (такие средства могут быть использованы только при выполнении \textit{действий во внешней среде}).
\end{scnitemizeii}
\item указание субъектов-исполнителей специфицируемого действия:
\begin{scnitemizeii}
\item множество тех, кто может выполнить это действие;
\item тот, кто должен (которому поручено выполнить это действие);
\end{scnitemizeii}
\item указание метода, на основании (путем интерпретации) которого специфицируемое действие может быть выполнено - таких методов в общем случае может быть несколько;
\item спецификация выполненного действия, т.е. действия, отнесенного к классу \textit{прошлых сущностей}:
\begin{scnitemizeii}
\item указание отрезка времени выполнения действия (момента начала и момента завершения);
\item указание числа прерываний (ожиданий) процесса выполнения действия;
\item указание "чистой"{} длительности процесса выполнения действия;
\item указание успешности выполнения процесса (в случае неуспешности - указание "штатных"{} причин и сбоев).
\end{scnitemizeii}
\end{scnitemize}}

\scnheader{следует отличать*}
\scnhaselementset{действие\\
	\scnaddlevel{1}
\scnnote{Каждому действию становится в соответствие кибернетическая система, являющаяся субъектом этого действия. Указанный субъект может быть либо индивидуальной, либо коллективной кибернетической системой.}
	\scnaddlevel{-1}
;воздействие\\
	\scnaddlevel{1}
	\scnsuperset{действие}
	\scnsubset{процесс}
	\scnnote{Сущностью, осуществляющей воздействие на какой-либо объект, может быть не только кибернетическая система, но также, например, и пассивный инструмент, управляемый некоторой кибернетической системой.}
	\scnaddlevel{-1}
}

\bigskip
\scnfragmentcaptiontext{Понятие класса действий и метода}

\scnheader{класс действий}
\scnrelto{семейство подклассов}{действие}
\scnidtfexp{\uline{максимальное} множество аналогичных (похожих в определенном смысле) действий, для которого существует (но не обязательно известный в текущий момент) по крайней мере один \textbf{метод} (или средство), обеспечивающий выполнение \uline{любого} действия из указанного множества действий} 
\scnidtf{множество однотипных действий}
\scnsuperset{класс элементарных действий}
\scnsuperset{класс легко выполнимых сложных действий}
\scnnote{Тот факт, что каждому выделяемому \textit{классу действий} соответствует по крайней мере один общий для них \textit{метод} выполнения этих \textit{действий}, означает то, что речь идет о \uline{семантической} "кластеризации"{} множества \textit{действий}, т.е. о выделении \textit{классов действий} по признаку \uline{семантической близости} (сходства) \textit{действий}, входящих в состав выделяемого \textit{класса действий}. При этом прежде всего учитывается аналогичность (сходство) \textit{исходных ситуаций} и \textit{целевых ситуаций} рассматриваемых \textit{действий}, т.е. аналогичность \textit{задач}, решаемых в результате выполнения соответствующих \textit{действий}. Поскольку одна и та же \textit{задача} может быть решена в результате выполнения нескольких \uline{разных} \textit{действий}, принадлежащих \uline{разным} \textit{классам действий}, следует говорить не только о \textit{классах действий} (множествах аналогичных действий), но и о \textbf{\textit{классах задач}} (о множествах аналогичных задач), решаемых этими \textit{действиями}. Так, например, на множестве \textit{классов действий} заданы следующие \textit{отношения}:
	\begin{scnitemize}
	\item \textit{отношение}, каждая связка которого связывает два разных (непересекающихся) \textit{класса действий}, осуществляющих решение одного и того же \textit{класса задач};
	\item \textit{отношение}, каждая связка которого связывает два разных \textit{класса действий}, осуществляющих решение разных \textit{классов задач}, один из которых является \textit{надмножеством} другого.
	\end{scnitemize}}

\scnheader{класс элементарных действий}
\scnidtf{множество элементарных действий, указание принадлежности которому является \uline{необходимым} и достаточным условием для выполнения этого действия}
\scnnote{Множество всевозможных элементарных действий, выполняемых каждым субъектом, должно быть \uline{разбито} на классы элементарных действий.}

\scnheader{класс легковыполнимых сложных действий}
\scnidtf{множество сложных действий, для которого известен и доступен по крайней мере один \textbf{\textit{метод}}, интерпретация которого позволяет осуществить полную (окончательную, завершающуюся элементарными действиями) декомпозицию на поддействия \uline{каждого} сложного действия из указанного выше множества}
\scnidtf{множество всех сложных действий, выполнимых с помощью известного \textit{метода}, соответствующего этому множеству}

\scnheader{спецификация*} 
\scnsuperset{сужение отношения по первому домену*(спецификация*; класс действий)*}
	\scnaddlevel{1}
  	\scnidtftext{часто используемый sc-идентификатор}{спецификация класса действий*}
  	\scnsubdividing{\textbf{обобщенная формулировка задач соответствующего класса*}\\
  	\scnaddlevel{1}
  	\scnsubdividing{\textbf{обобщенная декларативная формулировка задач соответствующего класса*}
  	;\textbf{обобщенная процедурная формулировка задач соответствующего класса*}}
  	\scnaddlevel{-1}
  	;\textbf{метод*}\\
  	\scnaddlevel{1}
  	\scnidtf{метод решения задач заданного класса*}
  	\scnidtf{метод выполнения действий соответствующего (заданного) класса*}
  	\scnsubdividing{\textbf{процедурный метод выполнения действий соответствующего класса*}\\
  	\scnaddlevel{1}
  	\scnidtf{обобщенный план выполнения действий заданного класса*}
  	\scnaddlevel{-1}
  	;\textbf{декларативный метод выполнения действий соответствующего класса*}\\
  	\scnaddlevel{1}
  	\scnidtf{обобщенная декларативная спецификация выполнения действий заданного класса*}
  	\scnaddlevel{-1}}
  	\scnaddlevel{-1}}
  	\scnaddlevel{-1}

\scnheader{класс задач}
\scnidtf{множество аналогичных действий}
\scnidtf{множество задач, для которого можно построить обобщенную формулировку задач, соответствующую всему этому множеству задач}
\scnnote{Каждая \textit{обобщенная формулировка задач соответствующего класса} по сути есть не что иное, как строгое логическое определение указанного класса задач.}

\scnheader{класс действий}
\scnsubdividing{\textbf{класс действий, однозначно задаваемый решаемым классом задач}\\
	\scnaddlevel{1}
	\scnidtf{\textit{класс действий}, обеспечивающих решение соответствующего \textit{класса задач} и использующих при этом любые, самые разные \textit{методы} решения задач этого класса}
	\scnaddlevel{-1}
	;\textbf{класс действий, однозначно задаваемый используемым методом решения задач}}

\scnheader{метод}
\scnrelto{второй домен}{метод*}
\scnidtf{описание того, \uline{как} может быть выполнено любое или почти любое действие, принадлежащее соответствующему классу действий}
\scnidtf{метод решения соответствующего класса задач, обеспечивающий решение любой или большинства задач указанного класса}
\scnidtf{обобщенная спецификация выполнения действий соответствующего класса}
\scnidtf{обобщенная спецификация решения задач соответствующего класса}
\scnidtf{программа решения задач соответствующего класса, которая может быть как процедурной, так и декларативной (непроцедурной)}
\scnidtf{знание о том, как можно решать задачи соответствующего класса}
\scnsubset{знание}
\scniselement{вид знаний}

\scnidtf{способ}
\scnidtf{знание о том, как надо решать задачи соответствующего класса задач (множества эквивалентных (однотипных, похожих) задач)}
\scnidtf{метод (способ) решения некоторого (соответствующего) класса задач}
\scnsubset{процедурная программа}
	\scnaddlevel{1}
	\scnsubset{алгоритм}
	\scnaddlevel{-1}
\scnidtf{информация (знание), достаточная для того, чтобы решить любую \textit{задачу}, принадлежащую соответствующему \textit{классу задач} с помощью соответствующей \textit{модели решения задач}}
\scnnote{В состав спецификации каждого \textit{класса задач} входит описание способа "привязки"{} \textit{метода} к исходным данным конкретной \textit{задачи}, решаемой с помощью этого \textit{метода}. Описание такого способа "привязки"{} включает в себя:
	\begin{scnitemize}
	\item набор переменных, которые входят как в состав \textit{метода}, так и в состав \textit{обобщенной формулировки задач соответствующего класса} и значениями которых являются соответствующие элементы исходных данных каждой конкретной решаемой задачи;
	\item часть \textit{обобщенной формулировки задач} того класса, которому соответствует рассматриваемый \textit{метод}, являющихся описанием \uline{условия применения} этого \textit{метода}.
	\end{scnitemize}
\bigskip
Сама рассматриваемая "привязка"{} \textit{метода} к конкретной \textit{задаче}, решаемой с помощью этого \textit{метода} осуществляется путем \uline{поиска} в \textit{базе знаний} такого фрагмента, который удовлетворяет условиям применения указанного \textit{метода}. Одним из результатов такого поиска и является установление соответствия между указанными выше переменными используемого \textit{метода} и значениями этих переменных в рамках конкретной решаемой \textit{задачи}. 

Другим вариантом установления рассматриваемого соответствия является явное обращение (вызов, call) соответствующего \textit{метода} (программы) с явной передачей соответствующих параметров. Но такое не всегда возможно, т.к. при выполнении процесса решения конкретной \textit{задачи} на основе декларативной спецификации выполнения этого действия нет возможности установить:
	\begin{scnitemize}
	\item когда необходимо инициировать вызов (использование) требуемого \textit{метода};
	\item какой конкретно \textit{метод} необходимо использовать;
	\item какие параметры, соответствующие конкретной инициируемой \textit{задачи}, необходимо передать для "привязки"{} используемого \textit{метода} к этой \textit{задаче}.
	\end{scnitemize}
	

Процесс "привязки"{} \textit{метода} решения \textit{задач} к конкретной \textit{задаче}, решаемой с помощью этого \textit{метода}, можно также представить как процесс, состоящий из следующих этапов:
	\begin{scnitemize}
	\item построение копии используемого \textit{метода};
	\item склеивание основных (ключевых) переменных используемого \textit{метода} с основными параметрами конкретной решаемой \textit{задачи}.
	\end{scnitemize}

В результате этого на основе рассматриваемого \textit{метода} используемого в качестве образца (шаблона) строится спецификация процесса решения конкретной задачи -- процедурная спецификация (\textit{план}) или декларативная.}
\scnnote{Заметим, что \textit{методы} могут использоваться даже при построении \textit{планов} решения конкретных \textit{задач}, в случае, когда возникает необходимость многократного повторения неких цепочек \textit{действий} при априори неизвестном количестве таких повторений. Речь идет о различного вида \textbf{циклах}, которые являются простейшим видом процедурных \textit{методов} решения задач, многократно используемых (повторяемых) при реализации \textit{планов} решения некоторых \textit{задач}.}

\scnheader{эквивалентность задач*}
\scnidtf{быть эквивалентной задачей*}
\scniselement{отношение}
\scntext{определение}{Задачи являются эквивалентными в том и только в том случае, если они могут быть решены путем интерпретации одного и того же \textit{метода} (способа), хранимого в памяти кибернетической системы.}
\scnnote{Некоторые \textit{задачи} могут быть решены разными \textit{методами}, один из которых, например, является обобщением другого.}

\scnheader{отношение, заданное на множестве методов}
\scnhaselement{подметод*}
	\scnaddlevel{1}
	\scnidtf{подпрограмма*}
	\scnidtf{быть методом, использование которого (обращение к которому) предполагается при реализации заданного метода*}
	\scnrelboth{следует отличать}{частный метод*}
	\scnaddlevel{1}
	\scnidtf{быть методом, обеспечивающим решение класса задач, который является подклассом задач, решаемых с помощью заданного метода*}
	\scnaddlevel{-1}

\scnheader{стратегия решения задач}
\scnsubset{метод}
\scnidtf{метаметод решения задач, обеспечивающий либо поиск одного релевантного известного метода, либо синтез целенаправленной последовательности аций применения в общем случае различных известных методов}
\scnnote{Можно говорить об универсальном метаметоде (универсальной стратегии) решения задач, объясняющем всевозможные частные стратегии.}
\scnexplanation{Можно говорить о нескольких глобальных \textit{стратегиях решения информационных задач} в базах знаний. Пусть в базе знаний появился знак инициированного действия с формулировкой соответствующей информационной цели, т.е. цели, направленной только на изменение состояния базы знаний. И пусть текущее состояние базы знаний не содержит контекста (исходных данных), достаточного для достижения указанной выше цели, т.е такого контекста, для которого в доступном пакете (наборе) методов (программ) имеется метод (программа), использование которого позволяет достигнуть указанную выше цель. Для достижения такой цели, контекст( исходные данные) которой недостаточен, существует три подхода (три стратегии): 
	\begin{scnitemize}
	\item декомпозиция (сведение изначальной цели к иерархической системе и/или подцелей (и/или подзадач) на основе анализа текущего состояния базы знаний и анализа того, чего в базе знаний не хватает для использования того или иного метода.) 
	
	При этом наибольшее внимание уделяется методам, для создания условий использования которых требуется меньше усилий. В конечном счете мы должны дойти (на самом нижнем уровне иерархии) до подцелей, контекст которых достаточен для применения одного из имеющихся методов (программ) решения задач;
	\item генерация новых знаний в семантической окрестности формулировки изначальной цели с помощью \uline{любых} доступных методов в надежде получить такое состояние базы знаний, которое будет содержать нужный контекст (достаточные исходные данные) для достижения изначальной цели с помощью какого-либо имеющегося метода решения задач;
	\item комбинация первого и второго подхода.
	\end{scnitemize}
Аналогичные стратегии существуют и для поиска пути решения задач, решаемых во внешней среде.}

\bigskip
\scnfragmentcaptiontext{Спецификация метода и понятие навыка}

\scnheader{метод}
\scnnote{Каждый конкретный метод рассматривается нами не только как важный вид спецификации соответствующего класса задач, но также и как объект, который и сам нуждается в спецификации, обеспечивающей непосредственное применение этого метода. Другими словами, метод является не только спецификацией (спецификацией соответствующего класса задач), но и \uline{объектом} спецификации.}

\scnheader{спецификация*}
\scnsuperset{\textbf{операционная семантика метода*}}
	\scnaddlevel{1}
		\scnidtf{спецификация метода*}
		\scneq{сужение отношения по первому домену*(спецификация*; метод)*}
		\scnidtf{семейство методов, обеспечивающих интерпретацию заданного метода*}
		\scnidtf{формальное описание интерпретатора заданного метода*}
		\scnrelfrom{второй домен}{\textbf{операционная семантика метода}}
		\scnaddlevel{1}
			\scnsuperset{\textbf{полное представление операционной семантики метода}}
			\scnaddlevel{1}
				\scnidtf{представление \textit{операционной семантики метода}, доведенное (детализированное) до уровня всех \textit{спецификаций элементарных действий}, выполняемых в процессе интерпретации соответствующего \textit{метода}}
			\scnaddlevel{-1}
		\scnaddlevel{-1}
	\scnaddlevel{-1}
\scnheader{навык}
\scnidtf{умение}
\scnidtf{объединение \textit{метода} с его исчерпывающей спецификацией -- \textit{полным представлением операционной семантики метода}}
\scnidtf{метод, интерпретация (выполнение, использование) которого полностью может быть осуществлено данной кибернетической системой, в памяти которой указанный метод хранится}
\scnidtf{метод, который данная кибернетическая система умеет (может) применять}
\scnidtf{метод + метод его интерпретации}
\scnidtf{умение решать соответствующий класс эквивалентных задач}
\scnidtf{метод плюс его операционная семантика, описывающая то, как интерпретируется (выполняется, реализуется) этот метод, и являющаяся одновременно операционной семантикой соответствующей модели решения задач}

\bigskip
\scnfragmentcaptiontext{Понятие класса методов и понятие модели решения задач}

\scnheader{класс методов}
\scnrelto{семейство подклассов}{метод}
\scnidtf{множество методов, для которых можно \uline{унифицировать} представление (спецификацию) этих методов}
\scnidtf{множество всевозможных методов решения задач, имеющих общий язык представления этих методов}
\scnidtf{множество всевозможных методов, представленных на данном языке}
\scnidtf{множество методов, для которых задан язык представления этих методов}

\scnhaselement{процедурный метод решения задач}
	\scnaddlevel{1}
		\scnsuperset{алгоритмический метод решения задач}
	\scnaddlevel{-1}
\scnhaselement{логический метод решения задач}
	\scnaddlevel{1}
		\scnsuperset{продукционный метод решения задач}
		\scnsuperset{функциональный метод решения задач}
	\scnaddlevel{-1}
\scnhaselement{искусственная нейронная сеть}
	\scnaddlevel{1}
		\scnidtf{класс методов решения задач на основе искусственных нейронных сетей}
	\scnaddlevel{-1}
\scnhaselement{генетический "алгоритм"{}}
\scnidtf{множество методов основанных на общей онтологии}
\scnidtf{множество методов, представленных на одинаковом языке}
\scnidtf{язык методов}
	\scnaddlevel{1}
		\scnnote{Таких специализированных языков может быть выделено целое множество, каждому из которых будет соответствовать своя модель решения задач (т.е. свой интерпретатор)}
	\scnaddlevel{-1}

\scnidtf{язык (например sc-язык) представлений методов соответствующего класса методов}
\scnidtf{множество методов решений задач, которому соответствует специальный язык (например sc-язык), обеспечивающий представление методов из этого множества}
\scnidtf{множество методов, которому ставится в соответствие отдельная модель решения задач}

\scnheader{язык представления методов} 
\scnidtf{язык представления методов, соответствующих определенному классу методов}
\scnsubset{язык}
\scnidtf{язык программирования}
\scnsuperset{язык представления методов обработки информации}
	\scnaddlevel{1}
		\scnidtf{язык программирования внутренних действий кибернетической системы, выполняемых в их памяти}
		\scnidtf{язык представления методов решения задач в памяти кибернетических систем}
	\scnaddlevel{-1}
\scnsuperset{язык представления методов решения задач во внешней среде кибернетических систем}
	\scnaddlevel{1}
		\scnidtf{язык программирования внешних действий кибернетических систем}
	\scnaddlevel{-1}

\scnheader{модель решения задач}
\scnidtf{метаметод интерпретации соответствующего класса методов}
\scnsubset{метод}
\scnidtf{метаметод}
\scnidtf{абстрактная машина интерпретации соответствующего класса методов}
\scnidtf{иерархическая система "микропрограмм"{}, обеспечивающих интерпретацию соответствующего класса методов}
\scnsuperset{алгоритмическая модель решения задач}
\scnsuperset{процедурная параллельная синхронная модель решения задач}
\scnsuperset{процедурная параллельная асинхронная модель решения задач}
\scnsuperset{продукционная модель решения задач}
\scnsuperset{функциональная модель решения задач}
\scnsuperset{логическая модель решения задач}
\scnaddlevel{1}
	\scnsuperset{четкая логическая модель решения задач}
	\scnsuperset{нечеткая логическая модель решения задач}
\scnaddlevel{-1}
\scnsuperset{"нейросетевая"{} модель решения задач}
\scnsuperset{"генетическая"{} модель решения задач}
\scnnote{Для интерпретации \uline{всех} моделей решения задач может быть использован агентно-ориентированный подход}
\scnexplanation{Каждая \textit{модель решения задач} задается:
\begin{scnitemize}
	\item соответствующим классом методов решения задач, т.е. языком представления методов этого класса;
	\item предметной областью этого класса методов; 
	\item онтологией этого класса методов (т.е. денотационной семантикой языка представления этих методов);
	\item операционной семантикой указанного класса методов.
\end{scnitemize}
}

\scnheader{спецификация*}
\scnsuperset{\textbf{модель решения задач}*}
	\scnaddlevel{1}
		\scneq{сужение отношения по первому домену(спецификация*; класс методов)*}
		\scnidtf{спецификация \textit{класса методов}*}
		\scnidtf{спецификация \textit{языка представления методов}*}
		\scnsubdividing{\textbf{синтаксис языка представления методов соответствующего класса}*;
\textbf{денотационная семантика языка представления методов соответствующего класса}*;
\textbf{операционная семантика языка представления методов соответствующего класса}*}
		\scnnote{Каждому конкретному \textit{классу методов} взаимно однозначно соответствует \textit{язык представления методов}, принадлежащих этому (специфицируемому) \textit{классу методов}. Таким образом, спецификация каждого \textit{класса методов} сводится к спецификации соответствующего \textit{языка представления методов}, т.е. к описанию его синтаксической, денотационной семантики и операционной семантики.
Примерами \textit{языков представления методов} являются все \textit{языки программирования}, которые в основном относятся к подклассу \textit{языков представления методов} -- к \textit{языкам представления методов обработки информации}. Но сейчас все большую актуальность приобретает необходимость создания эффективных формальных языков представления методов выполнения действий во внешней среде кибернетических систем. Без этого комплексная автоматизация, в частности, в промышленной сфере невозможна.}
	\scnaddlevel{-1}
	
\scnheader{денотационная семантика языка представления методов соответствующего класса}
\scnrelto{второй домен}{денотационная семантика языка представления методов соответствующего класса*}
\scnidtf{онтология соответствующего класса методов}
\scnidtf{денотационная семантика соответствующего класса методов}
\scnidtf{денотационная семантика языка (sc-языка), обеспечивающего представление методов соответствующего класса}
\scnidtf{денотационная семантика соответствующей модели решения задач}
\scnnote{речь идет о языке, обеспечивающем внутреннее представление методов соответствующего класса в ostis-системе, то синтаксис этого языка совпадает с синтаксисом sc-кода}
\scnsubset{онтология}

\scnheader{операционная семантика языка представления методов соответствующего класса}
\scnrelto{второй домен}{операционная семантика языка представления методов соответствующего класса*}
\scnidtf{метаметод интерпретации соответствующего класса методов}
\scnidtf{семейство агентов, обеспечивающих интерпретацию (использования) любого метода, принадлежащего соответствующему классу методов}
\scnidtf{операционная семантика соответствующей модели решения задач}

\scnheader{язык представления обобщенных формулировок задач для различных классов задач}
\scnnote{Поскольку каждому \textit{методу} соответствует \textit{обобщенная формулировка задач}, решаемых с помощью этого \textit{метода}, то каждому \textit{классу методов} должен соответствовать не только определенный \textit{язык представления методов}, принадлежащих указанному \textit{классу методов}, но и определенный \textit{язык представления обобщенных формулировок задач для различных классов задач}, решаемых с помощью \textit{методов}, принадлежащих указанному \textit{классу методов}.}

\bigskip
\scnfragmentcaptiontext{Понятие деятельности}

\scnheader{деятельность} 
\scnidtf{целостный, целенаправленный процесс \uline{поведения} (функционирования) одного субъекта или сообщества субъектов, осуществляемый на основе хорошо или не очень хорошо продуманной и согласованной \textit{технологии} в последнем случае качество деятельности определяется уровнем интеллекта единоличного или коллективного субъекта, осуществляющего этот целенаправленный процесс.}
\scnidtf{система действий, являющаяся некоторым кластером семантически близких действий, обладающих семантической близостью, семантической связностью и семантической целостностью}
\scnidtf{трудно выполнимая семантически целостная система действий}
\scnidtf{кластер множества действий, определяемый семантической близостью этих действий}
\scnidtf{система связанных между собой действий, имеющих общий контекст, общую область выполнения этих действий}
\scnnote{В состав каждой конкретной \textit{деятельности} входят \textit{действия}, являющиеся \textit{поддействиями}* других \textit{действий}, входящих в состав этой же \textit{деятельности}. При этом для каждого \textit{действия}, входящего в состав \textit{деятельности}, все поддействия этого \textit{действия} также входят в состав этой \textit{деятельности}. 

В состав каждой конкретной \textit{деятельности} входят также \textit{действия}, не являющиеся \textit{поддействиями}* других \textit{действий}, входящих в состав этой же \textit{деятельности}. Такие "первичные"{} ("независимые"{}, "самостоятельные"{}, "автономные"{}) \textit{действия} для заданной \textit{деятельности} могут инициироваться \uline{извне} этой \textit{деятельности} с помощью соответствующих инициирующих эти \textit{действия ситуаций} или \textit{событий}. Примерами таких инициирующих ситуаций, "порождающих"{} соответствующие действия, являются:
\begin{scnitemize}
	\item появление в \textit{базе знаний} каких-либо противоречий, информационных дыр, информационного мусора;
	\item появление в \textit{базе знаний} описаний (информационных моделей) каких-либо нештатных ситуаций в сложном объекте управления, на которые необходимо реагировать;
	\item появление в \textit{базе знаний} формулировок различного рода задач с явным указанием инициирования соответствующих действий, направленных на решение этих задач.
\end{scnitemize}
	К числу указанных "первичных"{} ("независимых"{}) \textit{действий}, входящих в состав \textit{объединенной деятельности кибернетической системы}, также относятся:
\begin{scnitemize}
	\item сложное действие, целью которого является перманентное обеспечение комплексной \textit{безопасности кибернетической системы};
	\item сложное действие, целью которого является перманентное  повышение качества информации (базы знаний), хранимой в памяти \textit{кибернетической системы};
	\item сложное действие, целью которого является перманентное повышение \textit{качества решателя задач кибернетической системы};
	\item сложное действие, целью которого является перманентная поддержка высокого уровня \textit{семантической совместимости} кибернетической системы со своими партнерами.
\end{scnitemize}
}

\scnheader{отношение, заданное на множестве*(деятельность)}
\scnhaselement{субъект*}
	\scnaddlevel{1}
		\scnidtf{быть субъектом заданного действия или деятельности*}
 		\scnidtf{кибернетическая система, которая в рамках заданного действия или деятельности выполняет ту или иную роль, воздействует на некий объект действия, используя тот или иной инструмент*}
 		\scniselement{отношение, заданное на множестве*(действие)}
 	\scnaddlevel{-1}
\scnhaselement{контекст*}
	\scnaddlevel{1} 
		\scnidtf{информационный контекст, в рамках которого осуществляется выполнение заданного действия или деятельности*}
		\scnidtf{область исполнения действия или деятельности*}
		\scnidtf{область действия или деятельности*}
		\scnrelfrom{первый домен}{(действие $\cup$ деятельность)}
		\scnidtf{совокупность знаний, достаточных для информационного обеспечения заданного действия или заданной деятельности}
		\scniselement{отношение, заданное на множестве* (действие)}
		\scnnote{Локализация (минимизация) \textit{контекста} заданного действия или деятельности является важнейшим "подготовленным"{} этапом, обеспечивающим существенное снижение "накладных расходов"{} при непосредственном выполнении этого \textit{действия} или \textit{деятельности}.}
	\scnaddlevel{-1}
\scnnote{Чаще всего \textit{контекстом} заданного \textit{действия} или \textit{деятельности} является некоторая \textit{предметная область} вместе с соответствующей ей интегрированной (объединенной) \textit{онтологией}. Поэтому хорошо продуманная декомпозиция \textit{базы знаний} интеллектуальной компьютерной системы на иерархическую систему \textit{предметных областей} и соответствующих им \textit{онтологий} имеет важное "практическое"{} значение, существенно повышающее качество (в частности, быстродействие) \textit{решателя задач} интеллектуальной компьютерной системы благодаря априорному  разбиению множества выполняемых \textit{действий} (решаемых задач) по соответствующих им \textit{контекстам}.}

\scnheader{следует отличать*}
\scnsuperset{\scnmakesetlocal{действие\\
 	\scnaddlevel{1}
		\scnidtf{процесс достижения конкретной цели конкретных обстоятельствах}
		\scnidtf{процесс решения конкретной задачи в конкретных условиях}
		\scnidtf{процесс задуманный, инициированный и осуществленный некоторым (или некоторыми) субъектами (кибернетическими системами)}
		\scnnote{\textit{действие} (точнее, соответствующая форма участия в его выполнении) является частью (фрагментом) \textit{деятельности} всех участвующих в этом субъектов (кибернетических систем)}
	\scnaddlevel{-1}
;деятельность\\
	\scnaddlevel{1}
		\scnidtf{система действий выполняемых соответствующим субъектом (кибернетической системой) "скрепленное"{} общим контекстом и определенным набором используемых навыков и инструментов}
		\scnnote{В отличие от \textit{действия}, \textit{деятельность} носит чаще всего перманентный характер в рамках времени существования соответствующего субъекта}
	\scnaddlevel{-1}
}}

\scnheader{деятельность кибернетической системы}
\scnidtf{полная система действий, выполняемых соответствующей кибернетической системой}
\scnidtf{деятельность субъекта}
\scnidtf{система всех действий соответствующего субъекта}
\scnsubdividing{внутренняя деятельность субъекта\\
	\scnaddlevel{1}
		\scnidtf{внутренняя деятельность соответствующего субъекта}
		\scnidtf{деятельность некоторого субъекта по обработке информации}
		\scnidtf{информационная деятельность}
	\scnaddlevel{-1}
;поведение субъекта\\
	\scnaddlevel{1}
		\scnidtf{внешнее поведение соответствующего субъекта}
		\scnidtf{деятельность субъекта во внешней среде}
	\scnaddlevel{-1}
}

\scnheader{(действие $\cup$ деятельность)}
\scnrelfrom{смотрите}{Теория действий, воздействий, деятельности (В.В. Мартынов), субъект, объект, инструмент, метод, навык, технология!!!}

\bigskip
\scnfragmentcaptiontext{Понятие вида деятельности и технологии}

\scnheader{вид деятельности}
\scnrelto{семейство подклассов}{деятельность}
\scnidtf{класс семантически целостных систем действия, для которых можно унифицировать используемые методы, информационные ресурсы и инструменты}
\scnidtf{класс трудно выполнимых и семантически целостных систем сложных действий}
\scnidtf{класс кластеров систем действий}
\scnidtf{множество деятельностей, которые могут быть реализованы с помощью общей технологии}
\scnhaselement{устранение противоречий в базе знаний}
\scnhaselement{устранение информационных дыр в базе знаний}
\scnhaselement{ликвидация информационного мусора в базе знаний}
\scnhaselement{управление сложным внешним объектом}
\scnhaselement{поддержка семантической совместимости с партнерами}
\scnhaselement{проектирование}
	\scnaddlevel{1} 
		\scnidtf{проектная деятельность}
		\scnidtf{построение такого описания (в частности, описания структуры) некоторого материального объекта, которого достаточно для воспроизводства (реализации, материализации) этого объекта либо при одиночном (уникальном), либо при массовом (промышленном) воспроизводстве указанного объекта}
		\scnnote{Примерами проектирования являются:
\begin{scnitemize}
\item проектирование здания;
\item проектирование машиностроительной конструкции;
\item проектирование микросхемы;
\item проектирование ostis-системы;
\item разработка системы шунтирования сердца;
\item разработка такого описания сложной геометрической фигуры, которого было бы достаточно для построения изображения (рисунка) этой фигуры с помощью, например, циркуля и линейки.
\end{scnitemize}}
	\scnaddlevel{-1}
\scnhaselement{разработка плана производства материального объекта по заданному проекту этого объекта}
	\scnaddlevel{1}
		\scnsuperset{разработка плана единичной реализации материального объекта по заданному проекту этого объекта}
		\scnsuperset{разработка плана массовой реализации материальных объектов по заданному их типовому проекту}
		\scnnote{Примерами данного вида действий являются:
\begin{scnitemize}
\item разработка плана-графика строительства конкретного здания;
\item разработка типового плана строительства зданий по заданному их типовому проекту;
\item разработка типового плана операций шунтирования сердца;
\item разработка алгоритма построения \uline{изображения} заданной геометрической фигуры с помощью циркуля и линейки
\end{scnitemize}}
	\scnaddlevel{-1}
\scnhaselement{производство}
	\scnaddlevel{1}
		\scnidtf{воспроизводство материального объекта по заданному его проекту и плану реализации}
		\scnidtf{производственная деятельность}
		\scnnote{Примерами данного вида действий являются:
\begin{scnitemize}
\item непосредственно строительство конкретного здания;
\item проведение конкретной хирургической операции;
\item процесс построения \uline{изображения} (рисунка) геометрической фигуры с помощью циркуля и линейки.
\end{scnitemize}}
	\scnaddlevel{-1}
\scnhaselement{реинжиниринг}
\scnhaselement{анализ}
\scnhaselement{интеграция}
	\scnaddlevel{1}
		\scnidtf{синтез}
	\scnaddlevel{-1}
\scnhaselement{деятельность в области здравоохранения}
\scnhaselement{образовательная деятельность}
\scnhaselement{эксплуатация сложного объекта}
\scnhaselement{научно-исследовательская деятельность}
\scnhaselement{управление}
	\scnaddlevel{1}
		\scnsuperset{целенаправленная координация деятельности нескольких субъектов}
		\scnaddlevel{1}
			\scnidtf{управление целенаправленной коллективной деятельностью нескольких субъектов}
		\scnaddlevel{-1}
	\scnaddlevel{-1}

\scnheader{проектирование}
\scnidtf{действие, направленное на построение (разработку) такой \uline{информационной} модели (проекта) некоторой \uline{материальной} сущности, которой \uline{достаточно}, чтобы соответствующий индивидуальный или коллективный субъект по соответствующей технологии (т.е. с помощью соответствующих методов и средств (инструментов)) смог воспроизвести (изготовить) указанную материальную сущность либо в одном экземпляре, либо в достаточно большом количестве таких экземпляров (копий), т.е. воспроизвести в промышленном масштабе}

\scnheader{производство}
\scnidtf{воспроизводство}
\scnidtf{изготовление}
\scnidtf{реализация}
\scnidtf{материализация}
\scnidtf{построение, синтез материальной сущности (артефакта)}
\scnidtf{изготовление материальной сущности в одной или во множестве экземпляров (копий)}
\scnidtf{производство (как действие)}

\scnheader{реинжиниринг}
\scnidtf{модификация}
\scnidtf{внесение изменений в некую сущность}
\scnidtf{обновление}
\scnidtf{реинжиниринг}
\scnidtf{перепроектирование}
\scnidtf{реконфигурация}
\scnidtf{трансформирование}
\scnsuperset{совершенствование}
	\scnaddlevel{1}
		\scnidtf{модификация, направленной на повышение качества модифицируемой сущности}
		\scnidtf{повышение качества}
		\scnidtf{улучшение}
		\scnsuperset{самосовершенствование}
		\scnaddlevel{1}
			\scnidtf{совершенствование, выполняемое самой совершенствуемой сущностью}
		\scnaddlevel{-1}
		\scnsuperset{совершенствование, осуществляемое извне}
		\scnnote{Самосовершенствоваться и обучаться могут только достаточно развитые кибернетические системы. Но совершенствоваться усилиями внешних субъектов могут любые сущности.}
	\scnaddlevel{-1}

\scnheader{анализ}
\scnidtf{построение (разработка, создание) спецификации (описания) основных связей и/или структуры, свойств, закономерностей, соответствующих (описываемой) сущности}
\scnnote{Объектом анализа может быть не только материальная сущность, но и процесс, ситуация, статическая структура, внешняя информационная конструкция, знание, понятие и другие абстрактные сущности}

\scnheader{интеграция}
\scnidtf{синтез}
\scnidtf{соединение}
\scnidtf{объединение}
\scnidtf{сборка}
\scnsubdividing{эклектичная интеграция\\
	\scnaddlevel{1}
		\scnidtf{интеграция без разрушения целостности интегрируемых сущностей}
		\scnidtf{интеграция без взаимопроникновения}
		\scnidtf{соединение систем по их входам/выходам}
	\scnaddlevel{-1}
;глубокая интеграция\\
	\scnaddlevel{1}
		\scnidtf{интеграция, в результате которой получается гибридная сущность}
		\scnidtf{интеграция с разрушением целостности (взаимопроникновением "диффузий"{}) интегрируемых сущностей}
		\scnidtf{"бесшовная"{} интеграция}
	\scnaddlevel{-1}
}

\scnheader{вид деятельности}
\scnexplanation{Если классу легко выполнимых сложных действий ставится в соответствие чаще всего \uline{один} \textit{метод} и, возможно, некоторый набор инструментальных средств, используемых в этом методе, то каждому виду деятельности ставится в соответствие своя \textbf{\textit{технология}}, включающая в себя некоторый набор используемых \textit{методов}, а также набор \textit{инструментальных средств}, используемых в этих \textit{методах}. Сложность здесь заключается:
\begin{scnitemize}
	\item в нетривиальности организации использования всего арсенала имеющейся \textit{технологии} для реализации (выполнения) каждой соответствующей \textit{деятельности};
	\item в трудности, а часто и в принципиальной невозможности \uline{полностью} автоматизировать реализацию соответствующей \textit{деятельности}.
\end{scnitemize}}

\scnheader{следует отличать*}
\scnhaselementset{действие\\
	\scnaddlevel{1}	
		\scnhaselementrole{пример}{Процесс доказательства Теоремы Пифагора}
		\scnaddlevel{1}	
		\scniselement{действие направленное на построение доказательства теоремы Геометрии Евклида}
	\scnaddlevel{-2}
;класс действий\\
	\scnaddlevel{1}
		\scnhaselementrole{пример}{процесс доказательства теоремы}
		\scnaddlevel{1}
		\scnidtftext{имя нарицательное}{действие, направленное на построение доказательства (логического обоснования) теоремы}
		\scnidtftext{имя собственное}{Класс действий, направленных на построение доказательств (логических обоснований) всевозможных теорем в различных формальных теориях}
	\scnaddlevel{-2}
;деятельность\\
	\scnaddlevel{1}
		\scnhaselementrole{пример}{Процесс эволюции Геометрии Евклида}
			\scnaddlevel{1}
			\scnidtf{Процесс эволюции формальной теории, являющейся формальным представлением Геометрии Евклида}
			\scnexplanation{В данный процесс входит и генерация гипотез в рамках Геометрии Евклида, и доказательство теорем, и выявление противоречий между высказываниями, и разрешение этих противоречий, и минимизация числа используемых определяемых понятий, и многое другое}
			\scnaddlevel{-1}
		\scnnote{\textit{деятельность} -- это то, что "превращает"{} множество самостоятельных и в определенной степени независимых \textit{действий}, принадлежащих разным \textit{классам действий}, в целостную, целенаправленную, сбалансированную систему \textit{действий}, ориентированную, прежде всего на поддержание качества и эволюцию \textit{кибернетических систем}, а также на обеспечение их адаптации к новым, ранее не предусмотренным обстоятельствам.}
	\scnaddlevel{-1}
;вид деятельности\\
	\scnaddlevel{1}
		\scnhaselementrole{пример}{процесс эволюции формальной теории}
		\scnaddlevel{1}
			\scnidtftext{имя собственное}{Класс процессов, направленных на эволюцию всевозможных формальных теорий (логических онтологий), которая также включает в себя возможность коррекции этих теорий.}
	\scnaddlevel{-2}
}

\scnheader{спецификация*}
\scnsuperset{сужение отношения по первому домену(спецификация*; вид деятельности)*}
	\scnaddlevel{1}
		\scnidtftext{часто используемый sc-идентификатор}{
спецификация вида деятельности*}
		\scneq{технология*}
		\scnaddlevel{1}
			\scnidtf{технология реализации (выполнения) деятельности соответствующего (заданного) вида*}
			\scnrelfrom{второй домен}{\textbf{технология}\\
			\scnidtf{технология соответствующего вида деятельности}
			\scnrelboth{аналог}{декларативный метод выполнения действий соответствующего класса}
			\scnaddlevel{1}
				\scnrelboth{аналог}{декларативная спецификация выполнения действия}
			\scnaddlevel{-1}
			\scnexplanation{\textit{технология} (как спецификация соответствующего вида деятельности) включает в себя:
			\begin{scnitemize}
				\item указание \textit{контекста}* специфицируемого \textit{вида деятельности};
				\item указание \textit{множества используемых методов}*, множества используемых инструментов, а также используемых материалов.
			\end{scnitemize}}
			}
		\scnaddlevel{-1}
	\scnaddlevel{-1}	

\scnheader{технология}
\scnexplanation{Каждая \textit{технология} представляет собой комплекс \textit{методов} (методик) и средств, обеспечивающих выполнение некоторого множества \textit{действий}, входящих в состав соответствующего \textit{вида деятельности}. Каждая \textit{технология} задается:
\begin{scnitemize}
	\item множеством методов (методик), которое разбивается на классы методов, эквивалентных по своей операционной семантике (по набору агентов, осуществляющих интерпретацию соответствующего класса методов);
	\item множеством агентов, являющихся средством интерпретации методов из указанного выше множества.
\end{scnitemize}
Указанное множество агентов также разбивается на подмножества, каждое из которых  соответствует своему классу методов и обеспечивает интерпретацию методов только этого класса.}

\scnidtf{множество (комплекс) навыков, обеспечивающих выполнение такого множества действий (задач), для которых отсутствует общий метод их выполнения}
\scnidtf{методика, инструментарий и дополнительные ресурсы, которые обеспечивают выполнение каждой конкретной деятельности, принадлежащей соответствующему виду деятельности}
\scnexplanation{с формальной точки зрения каждая технология задается ориентированной связкой, компонентами которой являются
\begin{scnitemize} 
	\item знак множества используемых методов 
	\item знак множества используемых инструментов 
	\item знак множества дополнительных используемых ресурсов
\end{scnitemize}}
\scnidtf{комплекс методов и средств (инструментов), с помощью которого некий субъект (который может быть как индивидуальным, так и коллективным) осуществляет некоторую деятельность (некоторое целенаправленное множество действий, входящих в состав этой деятельности)}
\scnsuperset{технология научно-теоретической деятельности}
\scnsuperset{технология проектирования}
	\scnaddlevel{1}
		\scnidtf{технология проектной деятельности}
		\scnidtf{технология построения такой информационной модели соответствующей сущности (артефакта), которой достаточно для воспроизводства этой сущности}
	\scnaddlevel{-1}
\scnsuperset{технология производства}
	\scnaddlevel{1}
		\scnidtf{технология производственной деятельности}
		\scnidtf{технология воспроизводства некоторого вида сущностей по заданным проектам этих сущностей}
	\scnaddlevel{-1}
\scnsuperset{технология здравоохранения}
\scnsuperset{технология образования}
	\scnaddlevel{1}
		\scnidtf{технология подготовки молодых специалистов}
		\scnidtf{технология образовательной деятельности}
	\scnaddlevel{-1}

\scnheader{отношение, заданное на множестве* (технология*)}
\scnhaselement{методы*}
	\scnaddlevel{1}
		\scnidtf{семейство методов, используемых в специфицируемой технологии  с дополнительным указанием их иерархии (т.е. с указанием того, какие методы используются при реализации других методов)}
	\scnaddlevel{-1}
\scnhaselement{активный инструмент*}
	\scnaddlevel{1}
		\scnidtf{средство, которое само способно выполнять некоторые действия, но при этом им надо как-то управлять (например, транспортные средства, компьютеры, …)}
		\scnidtf{средства автоматизации}
	\scnaddlevel{-1}
\scnhaselement{пассивный инструмент*}
	\scnaddlevel{1}
		\scnidtf{средство, которое само ничего делать не может (например, молоток, лопата, ножницы, …)}
	\scnaddlevel{-1}
\scnhaselement{комплектация*}
\scnhaselement{расходные средства*}
\scnhaselement{сырье*}
\scnhaselement{продукты*}
\scnhaselement{общий продукт*}
	\scnaddlevel{1}
		\scnidtf{объединенный (интегрированный) продукт*}
	\scnaddlevel{-1}
\scnhaselement{реализация технологии*}
	\scnaddlevel{1}
		\scnidtf{вариант (форма) реализации технологии*}
	\scnaddlevel{-1}
\scnhaselement{частная технология*}
	\scnaddlevel{1}
		\scnidtf{быть частной технологией по отношению к заданной технологии*}
	\scnaddlevel{-1}

\scnheader{продукты*}
\scnidtf{производимые сущности*}
\scnidtf{изготавливаемые материальные сущности*}
\scnidtf{продукция*}
\scnidtf{результаты выполнения соответствующего множества действий, осуществляемых во внешней среде*}
\scnidtf{продукты технологии*}
\scnidtf{множество материальных сущностей, производимых (создаваемых, порождаемых, изготавливаемых) с помощью заданной технологии*}
\scnidtf{то, что является "сухим остатком"{} при использовании данной технологии*}

\scnheader{технология}
\scnnote{Поскольку разработка каждой конкретной \textit{технологии} требует больших затрат, очень важно, чтобы \textit{технологии} создавались не под конкретные \textit{деятельности}, а для целых классов деятельностей (\textit{видов деятельности}). При этом важно, чтобы разрабатываемые \textit{технологии} охватывали как можно большее количество деятельностей, входящих в состав указанных \textit{видов деятельности}. Из этого следует целесообразность конвергенции и унификации различных сфер \textit{деятельности} для того, чтобы повысить мощность применения (использования) каждой разрабатываемой \textit{технологии}. Кроме того важна \textit{совместимость технологий}, позволяющая решать \textit{задачи}, требующие одновременного использования нескольких \textit{технологий}, причем, в непредсказуемых сочетаниях. Очень важно также, кроме \textit{видов деятельности}, которым соответствуют конкретные \textit{технологии}, ввести \textit{обобщенные виды деятельности} и построить их иерархии явно фиксировать стандарты, которым должны соответствовать все виды соответствующего обобщенного \textit{вида деятельности}. Это необходимо для обеспечения совместимости \textit{технологий}. Все используемые технологии должны "пронизывать"{} друг друга и составлять стройную иерархическую систему совместимых технологий (сумму технологий).}

\scnheader{класс технологий}
\scnidtf{множество похожих технологий, использующих, например, одинаковые методики и/или одинаковые активные инструменты и/или одинаковые пассивные инструменты и/или похожие множества продуктов}
\scnhaselement{технология проектирования}
	\scnaddlevel{1}
		\scnsuperset{технология проектирования интеллектуальных компьютерных систем}
		\scnsuperset{технология проектирования программных систем}
		\scnsuperset{технология проектирования микросхем}
		\scnsuperset{технология машиностроительного проектирования}
	\scnaddlevel{-1}
\scnhaselement{технология рецептурного производства}
	\scnaddlevel{1}
		\scnsuperset{технология производства молочных продуктов}
		\scnsuperset{технология производства мясных продуктов}
		\scnsuperset{технология фармацевтического производства}
	\scnaddlevel{-1}


\bigskip
\scnfragmentcaptiontext{Резюме предметной области и онтологии действий, задач, планов и методов}

\scnheaderlocal{следует отличать*}
\scnhaselementset{
	\scnmakevectorlocal{действие;класс действий};
	\scnmakevectorlocal{метод;класс методов};
	\scnmakevectorlocal{деятельность;вид деятельности}
}
\scnaddlevel{1}
\scnsubset{семейство подклассов*}

\scnnote{Все сущности, принадлежащие рассмотренным \textit{понятиям}, требуют достаточно детальной \textit{спецификации}. При этом не следует путать сами сущности и их \textit{спецификации}. Так, например, не следует путать \textit{действие} и \textit{задачу}, которая специфицирует (уточняет) это \textit{действие}. Особое место среди указанных понятий занимает понятие \textit{метода}, т.к. каждый конкретный \textit{метод}, с одной стороны, является \textit{спецификацией} соответствующего \textit{класса действий}, а, с другой стороны, сам нуждается в \textit{спецификации}, которая уточняет \textit{операционную семантику} этого \textit{метода}, (т.е. множество \textit{методов}, обеспечивающих \textit{интерпретацию} данного специфицируемого \textit{метода}) и тем самым "преобразует"{} специфицируемый \textit{метод} в \textit{навык}.}
\scnaddlevel{-1}

\scnheader{следует отличать*}
\scnhaselementvector{первый домен*(спецификация*)\\
\scnaddlevel{1}
\scnidtf{специфицируемая сущность}
\scnidtf{сущность, использование которой требует вполне определенной ее спецификации}
\scnsuperset{действие}
\scnsuperset{класс действий}
\scnsuperset{метод}
\scnsuperset{класс методов}
\scnsuperset{деятельность}
\scnsuperset{вид деятельности}
\scnaddlevel{-1};
второй домен*(спецификация*)\\
\scnaddlevel{1}
\scnidtf{спецификация}
\scnsuperset{задача}
\scnaddlevel{1}
\scnsuperset{декларативная формулировка задачи}
\scnsuperset{процедурная формулировка задачи}
\scnaddlevel{-1}
\scnsuperset{план}
\scnsuperset{декларативная спецификация выполнения действий}
\scnsuperset{протокол}
\scnsuperset{результативная часть протокола}
\scnsuperset{обобщенная формулировка задач соответствующего класса}
\scnsuperset{метод}
\scnsuperset{операционная семантика метода}
\scnsuperset{модель решения задач}
\scnaddlevel{-1}
}

\scnnote{
При этом следует отличать:
\begin{scnitemize}
\item спецификацию конкретного \textit{действия} (задачу, план, декларативную спецификацию выполнения действия, протокол, результативную часть протокола);
\item спецификацию конкретной деятельности (контекст*, множество используемых методов*);
\item спецификацию класса действий (обобщенную формулировку задачи, метод);
\item спецификацию вида деятельности (технологию);
\item спецификацию метода (операционную семантику метода);
\item спецификацию класса методов (модель решения задач).
\end{scnitemize}
}
\scnendstruct

\end{SCn}