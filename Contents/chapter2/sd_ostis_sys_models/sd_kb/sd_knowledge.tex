\begin{SCn}

\scnsectionheader{\currentname}

\scnstartsubstruct

\scnheader{Предметная область знаний}
\scniselement{предметная область}
\scnsdmainclasssingle{знание}
\scnsdclass{раздел;раздел-обоснование;раздел-документация;раздел-описание предметной области;раздел-описание семантической окрестности;раздел-описание принципов;неатомарный раздел;атомарный раздел}
\scnsdrelation{метазнание*;декомпозиция раздела*;базовый порядок разделов*;ключевой sc-элемент\scnrolesign;главный ключевой sc-элемент\scnrolesign;порядок ключевых sc-элементов*;комментарий*;аннотация*;введение*;заключение*;предисловие*;послесловие*;эпиграф*;детализация*}

\scnheader{знание}
\scnidtf{sc-знание}
\scnidtf{Множество всевозможных знаний}
\scnidtf{sc-знание или целостный фрагмент sc-знания}
\scnsubset{структура}
\scnsuperset{семантическая окрестность}
\scnsuperset{фактографическое знание}
\scnsuperset{раздел}
\scnsuperset{предметная область}
\scnsuperset{онтология}
\scnexplanation{\textbf{\textit{знание}} — это стационарная знаковая конструкция, обладающая некоторой семантической целостностью.}

\scnheader{метазнание*}
\scnidtf{быть метазнанием*}
\scniselement{бинарное отношение}
\scniselement{ориентированное отношение}
\scnrelfrom{первый домен}{знание}
\scnrelfrom{второй домен}{знание}
\scnexplanation{\textbf{\textit{метазнание*}} - это бинарное \textit{ориентированное отношение}, связки которого связывают некоторое исходное знание со знанием, которое является метазнанием исходного знания, то есть его спецификацией, например, описанием его структуры.

Примером связи между знанием и соответствующим ему \textbf{\textit{метазнанием*}} является переход от некоторого исходного знания к описанию его декомпозиции (сегментации) на некоторые части с указанием связей между этими частями.}

\scnheader{раздел}
\scnidtf{раздел базы знаний}
\scnidtf{sc-модель раздела базы знаний}
\scnsubdividing{атомарный раздел;неатомарный раздел}
\scnsuperset{раздел-обоснование}
\scnsuperset{раздел-описание принципов}
\scnsuperset{раздел-документация}
\scnsuperset{раздел-описание предметной области}
\scnsuperset{раздел-описание семантической окрестности}
\scnsuperset{раздел-теория предметной области}
\scnexplanation{\textbf{\textit{раздел}} – это знак множества всевозможных разделов, входящих в состав различных баз знаний. Каждый раздел представляет собой условно дидактически выделяемый фрагмент базы знаний, обладающий логической целостностью и завершенностью. В пределе вся база знаний конкретной \textit{ostis-системы} также является одним большим неатомарным \textbf{\textit{разделом}}.

Для каждого раздела необходимо явно указать принадлежность к множеству атомарных или неатомарных разделов.}
\scntext{правило идентификации экземпляров класса}{Экземпляры класса \textbf{\textit{раздел}} в рамках \textit{Русского языка} именуются по следующим правилам:
\begin{scnitemize}
\item в начале идентификатора пишется слово \textbf{Раздел} и ставится точка;
\item далее с прописной буквы название раздела, отражающее его содержание.
\end{scnitemize}
Например:\\
\textit{Раздел. SC-код}\\
\textit{Раздел. Предметная область sc-элементов}}

\scnheader{раздел-обоснование}
\scnexplanation{Каждый \textbf{\textit{раздел-обоснование}} представляет собой формальный текст, обосновывающий разработку чего-либо, то есть постановку проблемы, указание недостатков существующих решений такого рода, достоинства предлагаемых решений, предполагаемый эффект от применения разработанных моделей, методов и средств и т.д.}
\scntext{правило идентификации экземпляров класса}{Экземпляры класса \textbf{\textit{раздел-обоснование}} в рамках \textit{Русского языка} именуются по следующим правилам:
\begin{scnitemize}
\item в начале идентификатора пишется слово \textbf{Обоснование разработки} и ставится точка;
\item далее с прописной буквы указывается та сущность, обоснование которой будет представлено в указанном разделе.
\end{scnitemize}
Например:\\
\textit{Обоснование разработки. SC-код}}

\scnheader{раздел-документация}
\scnexplanation{Каждый \textbf{\textit{раздел-документация}} содержит документацию к чему-либо, как правило – к ostis-системе или ее компонентам. Документация описывает назначение системы или подсистемы, принципы работы с той или иной системой или подсистемой, ее структуру и состав. В случае многократно используемого компонента документация также содержит руководство по доработке и использованию такого компонента.}
\scntext{правило идентификации экземпляров класса}{Экземпляры класса \textbf{\textit{раздел-документация }} в рамках \textit{Русского языка} именуются по следующим правилам:
\begin{scnitemize}
\item в начале идентификатора пишется слово \textbf{Документация} и ставится точка;
\item далее с прописной буквы указывается та сущность, документация которой будет представлена в указанном разделе.
\end{scnitemize}
Например:\\
\textit{Документация. Технология OSTIS}}

\scnheader{раздел-описание предметной области}
\scnexplanation{Каждый \textbf{\textit{раздел-описание предметной области}} содержит знак самой этой \textit{предметной области}, знаки всех ее элементов, а также всю спецификацию этой \textit{предметной области}, включая все ее онтологии, в том числе – логическую онтологию, содержащую описание формальной теории, соответствующей данной \textit{предметной области}.}

\scnheader{раздел-описание семантической окрестности}
\scnexplanation{Каждый \textbf{\textit{раздел-описание семантической окрестности}} содержит знак \textit{структуры}, которая является семантической окрестностью некоторого \textit{sc-элемента}, знаки всех ее элементов, а так же спецификацию этой \textit{структуры}, например, указание принадлежности более частному классу семантических окрестностей.}

\scnheader{раздел-описание принципов}
\scnexplanation{Каждый \textbf{\textit{раздел-описание принципов}} содержит описание основополагающих принципов устройства либо функционирования какой-либо технологии, подсистемы, компонента и т.д.}

\scnheader{декомпозиция раздела*}
\scniselement{отношение декомпозиции}
\scniselement{бинарное отношение}
\scniselement{ориентированное отношение}
\scnsubset{базовая декомпозиция*}
\scnsubset{метазнание*}
\scnexplanation{\textbf{\textit{декомпозиция раздела*}} – это \textit{квазибинарное отношение} между разделом и множеством его подразделов.

Данное отношение задает дидактическую структуру раздела. В отличие от отношения \textit{базовая декомпозиция*}, которое связывает некоторую сущность с другими сущностями, являющимися её частями, отношение \textbf{\textit{декомпозиция раздела*}} связывает раздел с его подразделами, т.е. сужается область определения отношения \textit{базовая декомпозиция*}.

Все конструкции, описывающие конкретный факт \textbf{\textit{декомпозиции раздела*}}, попадают в соответствующий \textit{раздел}, описывающий структуру базы знаний.}

\scnheader{базовый порядок разделов*}
\scnidtf{базовая последовательность разделов*}
\scniselement{бинарное отношение}
\scniselement{ориентированное отношение}
\scniselement{отношение порядка}
\scnsubset{базовая декомпозиция*}
\scnsubset{метазнание*}
\scnrelfrom{первый домен}{раздел}
\scnrelfrom{второй домен}{раздел}
\scnrelfrom{область определения}{раздел}
\scnexplanation{\textbf{\textit{базовый порядок разделов*}} – это бинарное отношение между разделами, определяющее порядок их следования в рамках декомпозиции раздела.

Данное отношение задает дидактический порядок следования подразделов в рамках декомпозиции более общего раздела.

Все конструкции, описывающие конкретный факт указания \textbf{\textit{базового порядка разделов*}}, попадают в соответствующий \textit{раздел}, описывающий структуру базы знаний.}

\scnheader{неатомарный раздел}
\scnidtf{раздел, имеющий подразделы}
\scnidtf{раздел, декомпозируемый на подразделы}
\scnsubset{раздел}
\scnexplanation{\textbf{\textit{неатомарный раздел}} – знак множества всевозможных неатомарных разделов, входящих в состав различных документаций, то есть разделов, которые декомпозируются на более частные разделы.}

\scnheader{атомарный раздел}
\scnidtf{раздел, не имеющий подразделов}
\scnsubset{раздел}
\scnexplanation{\textbf{\textit{атомарный раздел}} – знак множества всевозможных атомарных разделов, входящих в состав различных документаций, то есть разделов, не декомпозируемых на более частные разделы.}
\scntext{примечание}{Любое понятие, входящее в состав атомарного раздела и не являющееся ключевым в рамках этого раздела должно стать ключевым в рамках какого-либо другого атомарного раздела.}

\scnheader{ключевой sc-элемент\scnrolesign}
\scnidtf{быть ключевым sc-элементом заданного sc-знания’}
\scnidtf{ключевой знак\scnrolesign}
\scnidtf{ключевой элемент sc-знания\scnrolesign}
\scnidtf{быть ключевым элементом заданной sc-структуры }
\scnidtf{быть ключевым элементом заданной sc-знания}
\scnsuperset{основной sc-элемент\scnrolesign}
\scniselement{ролевое отношение}
\scnexplanation{\textbf{\textit{ключевой sc-элемент\scnrolesign}} – ролевое отношение, связывающее знак каждой структуры (текста, знания) со специально выделяемыми (ключевыми) элементами этой структуры.

Каждая структура может иметь либо несколько ключевых элементов, либо один ключевой элемент, либо ни одного.}

\scnheader{главный ключевой sc-элемент\scnrolesign}
\scnidtf{быть ключевым sc-элементом заданного sc-знания\scnrolesign}
\scnsubset{ключевой sc-элемент\scnrolesign}
\scniselement{ролевое отношение}
\scnexplanation{\textbf{\textit{главный ключевой sc-элемент\scnrolesign}} – \textit{ролевое отношение}, связывающее знак некоторой структуры  (текста, знания) со специально выделяемыми главными элементами этой структуры. Главный элемент некоторой структуры является ее \textit{ключевым sc-элементом'}, но имеющим особый статус по каким-либо причинам, зависящим в каждом случае от семантики данного элемента и типа структуры.}

\scnheader{порядок ключевых sc-элементов*}
\scnidtf{следующий ключевой sc-элемент*}
\scniselement{отношение строгого порядка}
\scnexplanation{\textbf{\textit{порядок ключевых sc-элементов*}} – \textit{бинарное отношение}, связывающее \textit{sc-элемент}, являющийся \textit{ключевым sc-элементом’} некоторой \textit{структуры} и другой \textit{sc-элемент}, являющийся \textit{ключевым sc-элементом’} этой же \textit{структуры} и логически следующий за первым в процессе рассмотрения, например при описании какой-либо предметной области в рамках \textit{раздела}. Указание порядка может быть необходимо для дидактических целей, например, при изучении некоторого материала во многих случаях логичнее будет вначале рассмотреть более общее понятие, а только затем его подклассы и т.п.

Отметим, что знак связки отношения \textbf{\textit{порядок ключевых sc-элементов*}}, связывающей два \textit{ключевых sc-элемента’} некоторой \textit{структуры} также должен входить в эту \textit{структуру}, поскольку в рамках разных \textit{структур} порядок одних и тех же \textit{sc-элементов} может быть разным.}

\scnheader{комментарий*}
\scnidtf{sc-комментарий*}
\scniselement{бинарное отношение}
\scnsubset{метазнание*}
\scnexplanation{\textbf{\textit{комментарий*}} – \textit{бинарное отношение}, связывающее некоторую сущность (\textit{sc-текст} или знак файла) со знаком \textit{sc-текста}, являющегося комментарием к этой сущности.}

\scnheader{аннотация*}
\scniselement{бинарное отношение}
\scnsubset{метазнание*}
\scnexplanation{\textbf{\textit{аннотация*}} – \textit{бинарное отношение}, связывающее некоторый \textit{раздел} и \textit{sc-текст}, являющийся аннотацией к данному разделу. Как правило, аннотация содержит краткое (недетализированное) описание того, чему посвящен данный \textit{раздел}.}

\scnheader{введение*}
\scniselement{бинарное отношение}
\scnsubset{метазнание*}
\scnexplanation{\textbf{\textit{введение*}} – \textit{бинарное отношение}, связывающее некоторый \textit{раздел} и \textit{sc-текст}, являющийся введением к данному \textit{разделу}. Как правило, введение содержит информацию о том, как сущности, рассматриваемые в данном \textit{разделе}, связаны с сущностями, описанными в других \textit{разделах}.}

\scnheader{заключение*}
\scniselement{бинарное отношение}
\scnsubset{метазнание*}
\scnexplanation{\textbf{\textit{заключение*}} – \textit{бинарное отношение}, связывающее некоторый \textit{раздел} и \textit{sc-текст}, являющийся заключением к данному \textit{разделу}. В заключении, как правило, подводятся итоги и указываются основные положения, описанные в разделе, а также рассматриваются направления развития самого раздела и связанных с ним.}

\scnheader{предисловие*}
\scniselement{бинарное отношение}
\scnsubset{метазнание*}
\scnexplanation{\textbf{\textit{предисловие*}} – \textit{бинарное отношение}, связывающее некоторый \textit{раздел} и \textit{sc-текст}, являющийся предисловием к данному \textit{разделу}.}

\scnheader{послесловие*}
\scniselement{бинарное отношение}
\scnsubset{метазнание*}
\scnexplanation{\textbf{\textit{послесловие*}} – \textit{бинарное отношение}, связывающее некоторый \textit{раздел} и \textit{sc-текст}, являющийся послесловием к данному \textit{разделу}.}

\scnheader{эпиграф*}
\scniselement{бинарное отношение}
\scnsubset{метазнание*}
\scnexplanation{\textbf{\textit{эпиграф*}} – \textit{бинарное отношение}, связывающее некоторый \textit{раздел} и \textit{sc-текст}, являющийся эпиграфом к данному \textit{разделу}.}

\scnheader{детализация*}
\scniselement{бинарное отношение}
\scnsubset{метазнание*}
\scnexplanation{\textbf{\textit{детализация*}} – \textit{бинарное отношение}, связывающее некоторый \textit{раздел} и другой \textit{раздел}, в котором более детально описываются понятия, упоминаемые в первом \textit{разделе}, т.е. более детально рассматривается информация, содержащаяся в первом \textit{разделе}.}

\scnendstruct

\end{SCn}