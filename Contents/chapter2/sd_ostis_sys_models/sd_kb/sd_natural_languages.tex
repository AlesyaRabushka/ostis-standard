\begin{SCn}

\scnsectionheader{\currentname}

\scnstartsubstruct

\scnheader{язык}
\scnsubdividing{естественный язык\\
	\scnaddlevel{1}
	\scnexplanation{Естественный язык представляет собой язык, который не был создан целенаправленно}
	\scnaddlevel{-1}
;искусственный язык\\
	\scnaddlevel{1}
	\scnexplanation{Искусственный язык представляет собой язык, специально разработанный для воплощения определённых целей}
	\scnhaselement{Эсперанто}
	\scnhaselement{Python}
	\scnsuperset{сконструированный язык}
	\scnaddlevel{1}
	\scnexplanation{Сконструированный язык представляет собой искусственный язык, предназначенный для общения людей}
	\scnhaselement{Эсперанто}
	\scnaddlevel{-1}
	\scnaddlevel{-1}
}
\scnsuperset{международный язык}
	\scnaddlevel{1}
	\scnexplanation{Международный язык представляет собой естественный или искусственный язык, использующийся для общения людей разных стран}
	\scnhaselement{Английски язык}
	\scnhaselement{Русский язык}
	\scnaddlevel{-1}

\scnheader{плановый язык}
\scnreltoset{пересечение}{сконструированный язык;международный язык}

\scnheader{язык общения}
\scnreltoset{объединение}{естественный язык;сконструированный язык}
\scnhaselement{Английски язык}
\scnhaselement{Русский язык}
\scnhaselement{Эсперанто}
\scnreltoset{объединение}{корневой язык\\
	\scnaddlevel{1}
	\scnexplanation{Корневой язык представляет собой язык, для которого характерно полное отсутствие словоизменения и наличие грамматической значимости порядка слов, представленных только корнями.}
	\scnhaselement{Английски язык}
	\scnaddlevel{-1}
;агглютинативный язык\\
	\scnaddlevel{1}
	\scnexplanation{Агглютинативный язык характеризуется развитой системой употребления суффиксов, приставок, добавляемых к неизменяемой основе слова, которые используются для выражения числа, падежа, рода и др.}
	\scnhaselement{Английски язык}
	\scnaddlevel{-1}
;флективный язык\\
	\scnaddlevel{1}
	\scnexplanation{Для флективного языка характерно развитое употребление окончаний для выражения рода, числа, падежа, сложная система склонения глаголов, чередование гласных в корне. Строгое различение частей речи.}
	\scnhaselement{Русский язык}
	\scnaddlevel{-1}
;профлективный язык\\
	\scnaddlevel{1}
	\scnexplanation{Для профлективного языка для именного словоизменения характерна агглютинация, а для глагольного – флексия и чередование гласных (аблаут).}
	\scnaddlevel{-1}
}

\scnheader{словоформа}
\scnsubset{файл}
\scnexplanation{Словоформа - это слово, представленное в определенной грамматической форме.}

\scnheader{лексема}
\scnsubset{файл}
\scnexplanation{Лексема -  слово, рассматриваемое как единица словарного состава языка в совокупности всех его конкретных грамматических форм.}

\scnheader{грамматическая категория*}
\scniselement{бинарное отношение}
\scniselement{ориентированное отношение}
\scnrelfrom{первый домен}{язык}
\scnrelfrom{второй домен}{грамматическая категория}
\scnexplanation{Грамматическая категория* это бинарное ориентированное отношение, связывающее язык со  множеством взаимоисключающих и противопоставленных друг другу грамматических значений, задающих разбиение множества словоформ.}

\scnheader{парадигма*}
\scniselement{квазибинарное отношение}
\scniselement{ориентированное отношение}
\scnsubset{покрытие*}
\scnrelfrom{первый домен}{лексема}
\scnrelfrom{второй домен}{словоформа}
\scnexplanation{Парадигма* это квазибинарное отношение, связывающее лексему со множеством словоформ, принадлежащих данной лексеме и имеющих разные грамматические значения.}

\scnheader{морфема}
\scnsubset{файл}
\scnexplanation{Морфема — наименьшая единица языка, имеющая некоторый смысл.}

\scnheader{суффикс}
\scnsubset{морфема}
\scnexplanation{Суффикс – это морфема, которая стоит после корня и обычно служит для образования новых слов, хотя может использоваться и при образовании формы одного слова.}

\scnheader{корень}
\scnsubset{морфема}
\scnexplanation{Корень — морфема, несущая лексическое значение слова (или основную часть этого значения).}

\scnheader{Английский язык}
\scnrelfromset{грамматическая категория}{
	продолжительная форма';совершенная форма';простая форма'
}
\scnrelfromset{грамматическая категория}{сравнительная степень'\\
	\scnaddlevel{1}
	\scnexplanation{Сравнительную степень используется в случае, когда необходимо отметить, что предмет или человек обладает каким-то качеством в большей степени, чем другие.}
	\scnaddlevel{-1}
;положительная степень сравнения'\\
	\scnaddlevel{1}
	\scnexplanation{Положительная степень используется, чтобы указать, что предмет или человек обладает каким-то признаком или качеством.}
	\scnaddlevel{-1}
;превосходная степень сравнения'\\
	\scnaddlevel{1}
	\scnexplanation{Превосходная степень используется для указания того факта, что предмет или человек обладает каким-то качеством в наибольшей степени.}
	\scnaddlevel{-1}
}
\scnrelfromset{грамматическая категория}{общий падеж';косвенный падеж';притяжательный падеж';именительный падеж'}
\scnrelfromset{грамматическая категория}{женский род';мужской род';средний род'}
\scnrelfromset{грамматическая категория}{множественное число';единственное число'}
\scnrelfromset{грамматическая категория}{первое лицо';второе лицо';третье лицо'}
\scnrelfromset{грамматическая категория}{будущее время';прошедшее время';настоящее время'}

\scnheader{часть речи\scnsupergroupsign}
\scnrelto{семейство подмножеств}{лексема}
\scnexplanation{Часть речи представляют определенные лексико-грамматические разряды, на которые в зависимости от лексического значения от характера морфологических признаков и синтаксической функции делятся все слова языка.}

\scnhaselement{наречие}
	\scnaddlevel{1}
	\scnreltoset{объединение}{наречие места;наречие времени;наречие меры и степени;наречие образа действия}
	\scnsubdividing{простое наречие\\
		\scnaddlevel{1}
		\scnexplanation{Простые наречия не делятся на составные части.}
		\scnaddlevel{-1}
	;производное наречие\\
		\scnaddlevel{1}
		\scnexplanation{Производные наречия образованы при помощи суффиксов.}
		\scnaddlevel{-1}
	;сложное наречие\\
		\scnaddlevel{1}
		\scnexplanation{Сложные наречия образуются из нескольких корней.}
		\scnaddlevel{-1}
	;составное наречие\\
		\scnaddlevel{1}
		\scnexplanation{Составные наречия представляют собой сочетание служебного и знаменательного слова.}
		\scnaddlevel{-1}
	}
	\scnaddlevel{-1}
\scnhaselement{существительное}
	\scnaddlevel{1}
	\scnsubdividing{простое существительное\\
		\scnaddlevel{1}
		\scnexplanation{Простые существительные состоят из одного корня.}
		\scnaddlevel{-1}	
	;производное существительное\\
		\scnaddlevel{1}
		\scnexplanation{Производные существительные состоят из корня и одной или нескольких морфем (приставок или суффиксов).}
		\scnaddlevel{-1}
	;составное существительное\\
		\scnaddlevel{1}
		\scnexplanation{Составные существительные состоят по крайней мере из двух корней.}
		\scnaddlevel{-1}
	}
	\scnsubdividing{имя собственное\\
		\scnaddlevel{1}
		\scnexplanation{Имена собственные обозначают единственные в своем роде предметы или предметы, выделяемые из общего класса.}
		\scnaddlevel{-1}
	;имя нарицательное\\
		\scnaddlevel{1}
		\scnreltoset{объединение}{исчисляемое существительное\\
			\scnaddlevel{1}
			\scnsubdividing{конкретное исчисляемое существительное\\
				\scnaddlevel{1}
				\scnexplanation{Конкретные исчисляемые существительные - названия отдельных предметов и живых существ.}
				\scnaddlevel{-1}	
			;абстрактное исчисляемое существительное существительное\\
				\scnaddlevel{1}
				\scnidtf{отвлеченное исчисляемое существительное}
				\scnexplanation{Абстрактные исчисляемые существительные - названия исчисляемых понятий.}
				\scnaddlevel{-1}
			}
			\scnexplanation{Исчисляемые существительные могут быть посчитаны и имеют форму множественного числа.}
			\scnaddlevel{-1}
		;неисчисляемое существительное\\
			\scnaddlevel{1}
			\scnsubdividing{абстрактное неисчисляемое существительное\\
				\scnaddlevel{1}
				\scnidtf{отвлеченное неисчисляемое существительное}
				\scnexplanation{Абстрактные неисчисляемые существительные - названия неисчисляемых понятий.}
				\scnaddlevel{-1}	
			;вещественное неисчисляемое существительное\\
				\scnaddlevel{1}
				\scnexplanation{Вещественные неисчисляемые существительные - названия различных веществ и материалов.}
				\scnaddlevel{-1}
			}
			\scnexplanation{Неисчисляемые существительные не могут быть посчитаны и не имеют формы множественного числа.}
			\scnaddlevel{-1}
		;собирательное существительное\\
			\scnaddlevel{1}
			\scnexplanation{Собирательные существительные имеют форму единственного числа, но обозначают при этом группы лиц или понятий, рассматриваемые как одно целое.}
			\scnaddlevel{-1}
		}
		\scnexplanation{Имена нарицательные – это общие названия для всех однородных предметов.}
		\scnaddlevel{-1}
	}
	\scnaddlevel{-1}
\scnhaselement{глагол}
	\scnaddlevel{1}
	\scnsubdividing{простой глагол\\
		\scnaddlevel{1}
		\scnexplanation{Простые глаголы состоят только из одного корня.}
		\scnaddlevel{-1}
	;производный глагол\\
		\scnaddlevel{1}
		\scnexplanation{В производных глаголах, кроме корня, есть приставка и/или суффикс.}
		\scnaddlevel{-1}
	;сложный глагол\\
		\scnaddlevel{1}
		\scnexplanation{Сложные глаголы состоят из двух основ.}
		\scnaddlevel{-1}
	;составной глагол\\
		\scnaddlevel{1}
		\scnexplanation{Составные глаголы состоят из глагола и наречия или предлога.}
		\scnaddlevel{-1}
	}
	\scnsubdividing{смысловой глагол\\
		\scnaddlevel{1}
		\scnidtf{самостоятельный глагол}
		\scnsubset{простой глагол}
		\scnexplanation{Смысловые глаголы обладают собственным лексическим значением, они обозначают определенное действие или состояние.}
		\scnaddlevel{-1}
	;служебный глагол\\
		\scnaddlevel{1}
		\scnexplanation{Служебные глаголы не имеют самостоятельного значения. Они используются только для построения сложных форм глагола или составных сказуемых.}
		\scnsubdividing{глагол-связка\\
			\scnaddlevel{1}
			\scnexplanation{Глаголы-связки служат для соединения в предложении подлежащего с определенным состоянием.}
			\scnaddlevel{-1}
		;вспомогательный глагол\\
			\scnaddlevel{1}
			\scnexplanation{Вспомогательные глаголы служат для образования сложных глагольных форм.}
			\scnaddlevel{-1}
		;модальный глагол\\
			\scnaddlevel{1}
			\scnexplanation{Модальные глаголы отражают отношение говорящего к данному действию.}
			\scnaddlevel{-1}
		}
		\scnaddlevel{-1}
	}
\scnaddlevel{-1}
\scnhaselement{прилагательное}
\scnaddlevel{1}
\scnsubdividing{простое прилагательное\\
	\scnaddlevel{1}
	\scnexplanation{Простые прилагательные не имеют в своем составе суффиксов и приставок.}
	\scnaddlevel{-1}
;производное прилагательное\\
	\scnaddlevel{1}
	\scnexplanation{В составе производных прилагательных есть суффикс и/или приставка.}
	\scnaddlevel{-1}
;сложное прилагательное английского языка\\
	\scnaddlevel{1}
	\scnexplanation{Сложные прилагательные состоят из двух или более основ.}
	\scnaddlevel{-1}
}
\scnsubdividing{качественное прилагательное\\
	\scnaddlevel{1}
	\scnexplanation{Качественные прилагательные обозначают качества предмета прямо.}
	\scnaddlevel{-1}
;относительное прилагательное\\
	\scnaddlevel{1}
	\scnexplanation{Относительные прилагательные описывают качества предмета через его отношение к материалам, месту, времени.}
	\scnaddlevel{-1}
}
\scnaddlevel{-1}
\scnhaselement{местоимение}
\scnaddlevel{1}
\scnsubdividing{личное местоимение;притяжательное местоимение;указательное местоимение;возвратное местоимение;взаимное местоимение;вопросительное местоимение;относительное местоимение;неопределенное местоимение;отрицательное местоимение;разделительное местоимение;универсальное местоимение}
\scnaddlevel{-1}
\scnhaselement{предлог}
\scnaddlevel{1}
\scnsubdividing{производный предлог\\
	\scnaddlevel{1}
	\scnexplanation{Производный предлог - предлог, связанный происхождением с другими частями речи.}
	\scnaddlevel{-1}
;непроизводный предлог\\
	\scnaddlevel{1}
	\scnexplanation{Непроизводный предлог - так называемый первообразный предлог, который не может быть соотнесен по образованию с какой-либо частью речи.}
	\scnaddlevel{-1}
;сложный предлог\\
	\scnaddlevel{1}
	\scnexplanation{Сложный предлог - предлог, включающий в себя несколько компонентов.}
	\scnaddlevel{-1}
;составной предлог\\
	\scnaddlevel{1}
	\scnexplanation{Составной предлог представляет собой словосочетание. Он включают в себя слово из другой части речи и один или два предлога.}
	\scnaddlevel{-1}
}
\scnaddlevel{-1}
\scnhaselement{союз}
\scnaddlevel{1}
\scnsubdividing{сочинительный союз\\
	\scnaddlevel{1}
	\scnexplanation{Сочинительные союзы соединяют одинаковые по значимости слова, фразы, однородные члены предложения или независимые предложения в одно сложносочиненное предложение.}
	\scnaddlevel{-1}
;подчинительный союз\\
	\scnaddlevel{1}
	\scnexplanation{Подчинительные союзы соединяют придаточное предложение с основным, от которого оно зависит по смыслу, образуя сложноподчиненное предложение.}
	\scnaddlevel{-1}
;парный союз\\
	\scnaddlevel{1}
	\scnexplanation{Парные союзы служат для соединения слов, фраз или однородных, одинаковых частей одного предложения.}
	\scnaddlevel{-1}
;союзное наречие\\
	\scnaddlevel{1}
	\scnsubset{наречие}
	\scnexplanation{Союзные наречия соединяют два независимых предложения в одно сложносочиненное, или ставятся в начало предложения для его логической связи с предыдущим предложением.}
	\scnaddlevel{-1}
}
\scnsubdividing{простой союз\\
	\scnaddlevel{1}
	\scnexplanation{Простые союзы состоят из одного корня без суффиксов или префиксов.}
	\scnaddlevel{-1}
;сложный союз\\
	\scnaddlevel{1}
	\scnexplanation{Сложные союзы образованы от других частей речи, других союзов.}
	\scnaddlevel{-1}
;составной союз\\
	\scnaddlevel{1}
	\scnexplanation{Составные союзы состоят из двух и более слов, служебных и самостоятельных частей речи. К ним также относятся парные союзы.}
	\scnaddlevel{-1}
}
\scnaddlevel{-1}
\scnhaselement{числительное}
\scnaddlevel{1}
\scnsubdividing{порядковое числительное\\
	\scnaddlevel{1}
	\scnexplanation{Порядковые числительные обозначают порядок предметов.}
	\scnaddlevel{-1}
;количественное числительное английского языка\\
	\scnaddlevel{1}
	\scnexplanation{Количественные числительные обозначают количество предметов.}
	\scnaddlevel{-1}
}
\scnaddlevel{-1}

\scnsuperset{часть речи английского языка\scnsupergroupsign}
	\scnaddlevel{1}
	\scnhaselement{наречие английского языка}
		\scnaddlevel{1}
		\scnreltoset{объединение}{наречие места английского языка\\
			\scnaddlevel{1}
			\scnsubset{наречие места}
			\scnaddlevel{-1}
		;наречие времени английского языка\\
			\scnaddlevel{1}
			\scnsubset{наречие времени}
			\scnaddlevel{-1}
		;наречие меры и степени английского языка\\
			\scnaddlevel{1}
			\scnsubset{наречие меры и степени}
			\scnaddlevel{-1}
		;наречие образа действия английского языка\\
			\scnaddlevel{1}
			\scnsubset{наречие образа действия}
			\scnaddlevel{-1}
		}
		\scnsubdividing{простое наречие английского языка\\
			\scnaddlevel{1}
			\scnsubset{простое наречие}
			\scnaddlevel{-1}
		;производное наречие английского языка\\
			\scnaddlevel{1}
			\scnsubset{производное наречие}
			\scnaddlevel{-1}
		;сложное наречие английского языка\\
			\scnaddlevel{1}
			\scnsubset{сложное наречие}
			\scnaddlevel{-1}
		;составное наречие английского языка\\
			\scnaddlevel{1}
			\scnsubset{составное наречие}
			\scnaddlevel{-1}
		}
		\scnaddlevel{-1}
	\scnhaselement{существительное английского языка}
		\scnaddlevel{1}
		\scnsubdividing{простое существительное английского языка\\
			\scnaddlevel{1}
			\scnsubset{простое существительное}
			\scnaddlevel{-1}
		;производное существительное английского языка\\
			\scnaddlevel{1}
			\scnsubset{производное существительное}
			\scnaddlevel{-1}
		;составное существительное английского языка\\
			\scnaddlevel{1}
			\scnsubset{составное существительное}
			\scnaddlevel{-1}
		}
		\scnsubdividing{имя собственное английского языка\\
			\scnaddlevel{1}
			\scnsubset{имя собственное}
			\scnaddlevel{-1}
		;имя нарицательное английского языка\\
			\scnaddlevel{1}
			\scnreltoset{объединение}{исчисляемое существительное английского языка\\
				\scnaddlevel{1}
				\scnsubdividing{конкретное исчисляемое существительное английского языка\\
					\scnaddlevel{1}
					\scnsubset{конкретное исчисляемое существительное}
					\scnaddlevel{-1}
				;абстрактное исчисляемое существительное английского языка\\
					\scnaddlevel{1}
					\scnidtf{отвлеченное исчисляемое существительное английского языка}
					\scnsubset{абстрактное исчисляемое существительное}
					\scnaddlevel{-1}
				}
				\scnsubset{исчисляемое существительное}
				\scnaddlevel{-1}
			;неисчисляемое существительное английского языка\\
				\scnaddlevel{1}
				\scnsubdividing{абстрактное неисчисляемое существительное английского языка\\
					\scnaddlevel{1}
					\scnsubset{абстрактное неисчисляемое существительное}
					\scnaddlevel{-1}
					;вещественное неисчисляемое существительное английского языка\\
					\scnaddlevel{1}
					\scnsubset{вещественное неисчисляемое существительное}
					\scnaddlevel{-1}
				}
				\scnsubset{неисчисляемое существительное}
				\scnaddlevel{-1}
			;собирательное существительное английского языка\\
				\scnaddlevel{1}
				\scnsubset{собирательное существительное}
				\scnaddlevel{-1}
			}
			\scnsubset{имя нарицательное}
			\scnaddlevel{-1}
		}
		\scnaddlevel{-1}
	\scnhaselement{глагол английского языка}
		\scnaddlevel{1}
		\scnsubdividing{простой глагол английского языка\\
			\scnaddlevel{1}
			\scnsubset{простой глагол}
			\scnaddlevel{-1}
		;производный глагол английского языка\\
			\scnaddlevel{1}
			\scnsubset{производный глагол}
			\scnaddlevel{-1}
		;сложный глагол английского языка\\
			\scnaddlevel{1}
			\scnsubset{сложный глагол}
			\scnaddlevel{-1}
		;составной глагол английского языка\\
			\scnaddlevel{1}
			\scnsubset{составной глагол}
			\scnaddlevel{-1}
		}
		\scnsubdividing{смысловой глагол английского языка\\
			\scnaddlevel{1}
			\scnidtf{самостоятельный глагол английского языка}
			\scnsubset{простой глагол}
			\scnaddlevel{-1}
		;служебный глагол английского языка\\
			\scnaddlevel{1}
			\scnsubset{служебный глагол}
			\scnaddlevel{1}
			\scnsubdividing{глагол-связка английского языка\\
				\scnaddlevel{1}
				\scnsubset{глагол-связка}
				\scnaddlevel{-1}
			;вспомогательный глагол английского языка\\
				\scnaddlevel{1}
				\scnsubset{вспомогательный глагол}
				\scnaddlevel{-1}
			;модальный глагол английского языка\\
				\scnaddlevel{1}
				\scnsubset{модальный глагол}
				\scnaddlevel{-1}
			}
			\scnaddlevel{-2}
		}
		\scnaddlevel{-1}
	\scnhaselement{прилагательное английского языка}
	\scnaddlevel{1}
	\scnsubdividing{простое прилагательное английского языка\\
		\scnaddlevel{1}
		\scnsubset{простое прилагательное}
		\scnaddlevel{-1}
	;производное прилагательное английского языка\\
		\scnaddlevel{1}
		\scnsubset{производное прилагательное}
		\scnaddlevel{-1}
	;сложное прилагательное английского языка\\
		\scnaddlevel{1}
		\scnsubset{сложное прилагательное}
		\scnaddlevel{-1}
	}
	\scnsubdividing{качественное прилагательное английского языка\\
		\scnaddlevel{1}
		\scnsubset{качественное прилагательное}
		\scnaddlevel{-1}
	;относительное прилагательное английского языка\\
		\scnaddlevel{1}
		\scnsubset{относительное прилагательное}
		\scnaddlevel{-1}
	}
	\scnaddlevel{-1}
	\scnhaselement{местоимение английского языка}
		\scnaddlevel{1}
		\scnsubdividing{личное местоимение английского языка\\
			\scnaddlevel{1}
			\scnsubset{личное местоимение}
			\scnaddlevel{-1}
		;притяжательное местоимение английского языка\\
			\scnaddlevel{1}
			\scnsubset{притяжательное местоимение}
			\scnaddlevel{-1}
		;указательное местоимение английского языка\\
			\scnaddlevel{1}
			\scnsubset{указательное местоимение}
			\scnaddlevel{-1}
		;возвратное местоимение английского языка\\
			\scnaddlevel{1}
			\scnsubset{возвратное местоимение}
			\scnaddlevel{-1}
		;взаимное местоимение английского языка\\
			\scnaddlevel{1}
			\scnsubset{взаимное местоимение}
			\scnaddlevel{-1}
		;вопросительное местоимение английского языка\\
			\scnaddlevel{1}
			\scnsubset{вопросительное местоимение}
			\scnaddlevel{-1}
		;относительное местоимение английского языка\\
			\scnaddlevel{1}
			\scnsubset{относительное местоимение}
			\scnaddlevel{-1}
		;неопределенное местоимение английского языка\\
			\scnaddlevel{1}
			\scnsubset{неопределенное местоимение}
			\scnaddlevel{-1}
		;отрицательное местоимение английского языка\\
			\scnaddlevel{1}
			\scnsubset{отрицательное местоимение}
			\scnaddlevel{-1}
		;разделительное местоимение английского языка\\
			\scnaddlevel{1}
			\scnsubset{разделительное местоимение}
			\scnaddlevel{-1}
		;универсальное местоимение английского языка\\
			\scnaddlevel{1}
			\scnsubset{универсальное местоимение}
			\scnaddlevel{-1}
		}
		\scnaddlevel{-1}
	\scnhaselement{предлог английского языка}
		\scnaddlevel{1}
		\scnsubdividing{производный предлог английского языка\\
			\scnaddlevel{1}
			\scnsubset{производный предлог}
			\scnaddlevel{-1}
		;непроизводный предлог английского языка\\
			\scnaddlevel{1}
			\scnsubset{непроизводный предлог}
			\scnaddlevel{-1}
		;сложный предлог английского языка\\
			\scnaddlevel{1}
			\scnsubset{сложный предлог}
			\scnaddlevel{-1}
		;составной предлог английского языка\\
			\scnaddlevel{1}
			\scnsubset{составной предлог}
			\scnaddlevel{-1}
		}
		\scnaddlevel{-1}
	\scnhaselement{союз английского языка}
		\scnaddlevel{1}
		\scnsubdividing{сочинительный союз английского языка\\
			\scnaddlevel{1}
			\scnsubset{сочинительный союз}
			\scnaddlevel{-1}
		;подчинительный союз английского языка\\
			\scnaddlevel{1}
			\scnsubset{подчинительный союз}
			\scnaddlevel{-1}
		;парный союз английского языка\\
			\scnaddlevel{1}
			\scnsubset{парный союз}
			\scnaddlevel{-1}
		;союзное наречие английского языка\\
			\scnaddlevel{1}
			\scnsubset{наречие английского языка}
			\scnsubset{союзное наречие}
			\scnaddlevel{-1}
		}
		\scnsubdividing{простой союз английского языка\\
			\scnaddlevel{1}
			\scnsubset{простой союз}
			\scnaddlevel{-1}
		;сложный союз английского языка\\
			\scnaddlevel{1}
			\scnsubset{сложный союз}
			\scnaddlevel{-1}
		;составной союз английского языка\\
			\scnaddlevel{1}
			\scnsubset{составной союз}
			\scnaddlevel{-1}
		}
		\scnaddlevel{-1}
	\scnhaselement{числительное английского языка}
	\scnaddlevel{1}
	\scnsubdividing{порядковое числительное английского языка\\
		\scnaddlevel{1}
		\scnsubset{порядковое числительное}
		\scnaddlevel{-1}
	;количественное числительное английского языка\\
		\scnaddlevel{1}
		\scnsubset{количественное числительное}
		\scnaddlevel{-1}
	}
	\scnaddlevel{-2}

\end{SCn}