\begin{SCn}

\scnsectionheader{\currentname}

\scnstartsubstruct

\scnheader{Предметная область предметных областей}
\scnidtf{Предметная область, объектами исследования которой являются предметные области}
\scnexplanation{В состав \textbf{\textit{Предметной области предметных областей}} входят структурные спецификации всех \textit{предметных областей}, входящих в состав базы знаний \textit{ostis-системы}, в том числе, самой \textbf{\textit{Предметной области предметных областей}}. Таким образом, \textbf{\textit{Предметная область предметных областей}} является, во-первых, \textit{рефлексивным множеством}, во-вторых, рефлексивной предметной областью, то есть \textit{предметной областью}, одним из объектов исследования которой является она сама.}
\scniselement{рефлексивное множество}
\scnsdmainclasssingle{предметная область}

\scnsdclass{стационарная предметная область;нестационарная предметная область;понятие;sc-язык}

\scnsdrelation{понятие предметной области’;исследуемое понятие’;максимальный класс объектов исследования’;немаксимальный класс объектов исследования’;исследуемый класс первичных элементов’;исследуемое отношение’;класс исследуемых структур';понятие, исследуемое в предметной подобласти’;понятие, исследуемое в более общей предметной области’;понятие, исследуемое в неродственной предметной области’;частная предметная область*;частная предметная область по классу первичных элементов*;частная предметная область по исследуемым отношениям*;неродственные предметные области*;sc-язык и соответствующая предметная область*}

\scnheader{предметная область}
\scnidtf{sc-модель предметной области}
\scnidtf{sc-текст предметной области}
\scnidtf{sc-граф предметной области}
\scnidtf{представление предметной области в SC-коде}
\scnsubset{знание}
\scnsubset{бесконечное множество}
\scnexplanation{\textbf{\textit{Предметная область}} – это результат интеграции (объединения) частичных семантических окрестностей, описывающих все исследуемые сущности заданного класса и имеющих одинаковый (общий) предмет исследования (то есть один и тот же набор отношений, которым должны принадлежать связки, входящие в состав интегрируемых семантических окрестностей).


\textbf{\textit{Предметная область}} – \textit{структура}, в состав которой входят:
\begin{scnenumerate}
\item \textnormal{основные исследуемые (описываемые) объекты – первичные и вторичные;}
\item \textnormal{различные классы исследуемых объектов;}
\item \textnormal{различные связки, компонентами которых являются исследуемые объекты (как первичные, так и вторичные), а также, возможно, другие такие связки – то есть связки (как и объекты исследования) могут иметь различный структурный уровень;}
\item \textnormal{различные классы указанных выше связок (то есть отношения);}
\item \textnormal{различные классы объектов, не являющихся ни объектами исследования, ни указанными выше связками, но являющихся компонентами этих связок.}
\end{scnenumerate}


При этом все классы, объявленные исследуемыми понятиями, должны быть полностью представлены в рамках данной предметной области вместе со своими элементами, элементами элементов и т.д. вплоть до терминальных элементов.


Можно говорить о типологии \textbf{\textit{предметных областей}} по разным структурным признакам:
\begin{scnitemize}
    \item наличие метасвязей;
    \item наличие исследуемых структур, входящих в состав предметной области;
    \item наличие исследуемых (смежных, дополнительных) объектов, которых исследуются в других предметных областях;
\end{scnitemize}


Понятие \textbf{\textit{предметной области}} является важнейшим методологическим приемом, позволяющим выделить из всего многообразия исследуемого Мира только определенный класс исследуемых сущностей и только определенное семейство отношений, заданных на указанном классе. То есть осуществляется локализация, фокусирование внимания только на этом, абстрагируясь от всего остального исследуемого Мира.


Во всем многообразии \textbf{\textit{предметных областей}} особое место занимают
\begin{scnenumerate}
    \item \textit{Предметная область предметных областей}, объектами исследования которой являются всевозможные \textbf{\textit{предметные области}}, а предметом исследования – всевозможные \textit{ролевые отношения}, связывающие предметные области с их элементами, отношения, связывающие предметные области между собой, отношение, связывающее предметные области с их онтологиями
    \item \textit{Предметная область сущностей}, являющаяся предметной областью самого высокого уровня и задающая базовую семантическую типологию \textit{sc-элементов}(знаков, входящих в тексты \textit{SC-кода})
    \item Семейство \textbf{\textit{предметных областей}}, каждая из которых задает семантику и синтаксис некоторого \textit{sc-языка}, обеспечивающего представление онтологий соответствующего вида (например, \textit{теоретико-множественных онтологий}, \textit{логических онтологий}, \textit{терминологических онтологий}, \textit{онтологий задач и способов их решения} и т.д.)
    \item Семейство \textbf{\textit{предметных областей}} верхнего уровня, в которых классами объектов исследования являются весьма <<крупные>> классы сущностей. К таким классам, в частности
    
    \begin{scnitemizeii}
        \item класс всевозможных \textit{материальных сущностей},
        \item класс всевозможных \textit{множеств},
        \item класс всевозможных \textit{связей},
        \item класс всевозможных \textit{отношений},
        \item класс всевозможных \textit{структур},
        \item класс всевозможных \textit{временных (нестационарных) сущностей},
        \item класс всевозможных \textit{действий} (акций),
        \item класс всевозможных \textit{параметров} (характеристик),
        \item класс \textit{знаний} всевозможного вида 
        \item и т.п.
    \end{scnitemizeii}
\end{scnenumerate}


Каждой \textbf{\textit{предметной области}} можно поставить в соответствие:
\begin{scnenumerate}
    \item семейство соответствующих ей \textit{онтологий} разного вида;
    \item некий язык (в нашем случае – язык, построенный на основе \textit{SC-кода}), тексты которого представляют различные фрагменты соответствующей предметной области
\end{scnenumerate}


Указанные языки будем называть \textit{sc-языками}. Их синтаксис и семантика полностью задается \textit{SС-кодом} и \textit{онтологией} соответствующей \textbf{\textit{предметной области}}. Очевидно, что в первую очередь нас должны интересовать те \textit{sc-языки}, которые соответствуют \textbf{\textit{предметным областям}}, имеющим общий (условно говоря, предметно независимый) характер. К таким предметным областям, в частности, относятся:
\begin{scnitemize}
    \item \textit{Предметная область множеств}, описывающая множества и различные связи между ними
    \item Предметная область графовых структур
    \item \textit{Предметная область чисел} и числовых структур
    \item и т.д
\end{scnitemize}


Каждому типу знаний можно поставить в соответствие предметную область, которая является результатом интеграции всех знаний данного типа. Эти знания и становятся объектами исследования в рамках указанной предметной области


Понятие \textbf{\textit{предметной области}} может рассматриваться как обобщение понятия алгебраической системы. При этом семантическая структура базы знаний может рассматриваться как иерархическая система различных \textbf{\textit{предметных областей}}.
}
\scnsubdividing{стационарная предметная область;нестационарная предметная область}

\scnheader{стационарная предметная область}
\scnidtf{статическая предметная область}
\scnexplanation{\textbf{\textit{стационарная предметная область}} - это \textit{предметная область}, в которой связи между сущностями, входящими в ее состав, не зависят от времени (не меняются во времени). При этом некоторые из указанных сущностей могут иметь конечное время "жизни" (конечное время существования).


Таким образом, элементами \textbf{\textit{стационарной предметной области}} не могут быть \textit{временные сущности}.}

\scnheader{нестационарная предметная область}
\scnidtf{динамическая предметная область}
\scnexplanation{\textbf{\textit{нестационарная предметная область}} - это \textit{предметная область}, в которой некоторые связи между сущностями, входящими в ее состав, меняются со временем (то есть носят ситуационный, нестационарный характер, другими словами, являются \textit{временными сущностями}).}

\scnheader{понятие предметной области’}
\scnsubdividing{исследуемое понятие’;понятие, исследуемое в частной предметной области’;понятие, исследуемое в более общей предметной области’;понятие, исследуемое в неродственной предметной области’}
\scniselement{неосновное понятие}
\scnexplanation{\textbf{\textit{понятие предметной области’}} – это \textit{ролевое отношение}, указывающее в рамках \textit{предметной области} на знак множества, являющегося классом некоторых объектов.}

\scnheader{понятие}
\scnrelto{строгое включение}{класс}
\scnrelto{второй домен}{понятие предметной области'}
\scnexplanation{\textbf{\textit{понятие}} – класс, являющийся исследуемым понятием хотя бы для одной \textit{предметной области}.}
\scnrelfrom{правило идентификации экземпляров}{
\scnfilelong{Все \textit{идентификаторы}, соответствующие экземплярам класса \textbf{\textit{понятие}}, должны начинаться со строчной буквы.

Например:

\textit{треугольник}
\textit{множество}}}

\scnheader{исследуемое понятие’}
\scnidtf{класс объектов исследования данной предметной области’}
\scnidtf{понятие, исследуемое в данной предметной области’}
\scniselement{ролевое отношение}
\scnsubdividing{максимальный класс объектов исследования’;немаксимальный класс объектов исследования’}
\scnsubdividing{исследуемый класс первичных элементов’;исследуемое отношение’;класс исследуемых структур'}
\scnrelto{включение}{полностью представленное множество’}
\scnexplanation{\textbf{\textit{исследуемое понятие’}} – это \textit{ролевое отношение}, указывающее в рамках \textit{предметной области} на знак множества, являющегося классом объектов исследования данной предметной области, то есть такого множества, все элементы которого являются элементами данной предметной области.


\textbf{\textit{исследуемое понятие’}} может быть:
\begin{scnenumerate}
    \item классом \textit{первичных элементов'} этой \textit{предметной области};
    \item отношением (классов связок), связывающих элементы этой \textit{предметной области} – уровень иерархии этих элементов может быть различным;
    \item классом \textit{структур}, все элементы которых являются элементами заданной \textit{предметной области}, т.е. классом подструктур заданной \textit{предметной области}.
\end{scnenumerate}
}

\scnheader{максимальный класс объектов исследования’}
\scniselement{ролевое отношение}
\scnexplanation{\textbf{\textit{максимальный класс объектов исследования’}} – это \textit{ролевое отношение}, указывающее в рамках \textit{предметной области} на множество, являющееся максимальным классом объектов исследования данной предметной области, то есть на такое \textit{исследуемое понятие’}, для которого в рамках данной предметной области не существует другого \textit{исследуемого понятия'}, которое бы являлось надмножеством для данного.}

\scnheader{немаксимальный класс объектов исследования’}
\scniselement{ролевое отношение}
\scnexplanation{\textbf{\textit{немаксимальный класс объектов исследования’}} – это ролевое отношение, указывающее в рамках \textit{предметной области} на такое \textit{исследуемое понятие’}, для которого в рамках данной предметной области существует другое \textit{исследуемое понятие'}, являющееся надмножеством первого.}

\scnheader{исследуемый класс первичных элементов’}
\scnexplanation{\textbf{\textit{исследуемый класс первичных элементов’}} – такое \textbf{\textit{исследуемое понятие’}} для данной \textit{предметной области}, что все его элементы являются ее \textit{первичными элементами'}.}

\scnheader{исследуемое отношение’}
\scniselement{ролевое отношение}
\scnexplanation{\textbf{\textit{исследуемое отношение’}} – это \textit{ролевое отношение}, указывающее в рамках \textit{предметной области} на множество связок, являющееся исследуемым отношением данной предметной области, то есть таким отношением, все связки которого являются элементами этой \textit{предметной области}.


При этом элементы таких связок также входят в данную \textit{предметную область}, но в общем случае могут не являться элементами \textit{исследуемых понятий'} данной \textit{предметной области}.
}

\scnheader{класс исследуемых структур'}
\scniselement{ролевое отношение}
\scnexplanation{\textbf{\textit{класс исследуемых структур’}} – это \textit{ролевое отношение}, указывающее в рамках \textit{предметной области} на множество \textit{структур}, знак каждой из которых принадлежит данной \textit{предметной области}.


В общем случае в данную \textit{предметную область}, могут входить не все элементы таких \textit{структур}, а только некоторые из них (хотя бы один для каждой \textit{структуры}).
}

\scnheader{понятие, исследуемое в предметной подобласти’}
\scnidtf{класс исследуемых объектов, который детально исследуется в предметной области, которая является частной по отношению к заданной предметной области'}
\scniselement{ролевое отношение}
\scnrelto{включение}{частично представленное множество’}
\scnexplanation{\textbf{\textit{понятие, исследуемое в частной предметной области’}} – это \textit{понятие предметной области'} данной \textit{предметной области}, которое является \textit{исследуемым понятием'} для какой-либо из \textit{частных предметных областей*} относительно данной.}

\scnheader{понятие, исследуемое в более общей предметной области’}
\scnrelto{включение}{частично представленное множество’}
\scnexplanation{\textbf{\textit{понятие, исследуемое в более общей предметной области’}} – это \textit{понятие предметной области'}, которое является \textit{исследуемым понятием'} для какой-либо из \textit{предметных областей}, для которых данная \textit{предметная область} является \textit{частной предметной областью*}. 
}

\scnheader{понятие, исследуемое в неродственной предметной области’}
\scnrelto{включение}{частично представленное множество’}
\scnexplanation{\textbf{\textit{понятие, исследуемое в неродственной предметной области’}} – это \textit{понятие предметной области'}, которое является \textit{исследуемым понятием'} для какой-либо из \textit{предметных областей}, являющихся \textit{неродственными предметными областями*} для данной.}

\scnheader{частная предметная область*}
\scnidtf{дочерняя предметная область*}
\scnidtf{быть частной предметной областью*}
\scnidtf{предметная область, детализирующая описание одного из классов объектов исследования другой (более общей) предметной области*}
\scniselement{бинарное отношение}
\scniselement{ориентированное отношение}
\scniselement{неролевое отношение}
\scnsuperset{частная предметная область по классу первичных элементов*}
\scnsuperset{частная предметная область по исследуемым отношениям*}
\scnexplanation{\textbf{\textit{частная предметная область*}} – бинарное ориентированное отношение, с помощью которого задается иерархия предметных областей путем перехода от менее детального к более детальному рассмотрению соответствующих исследуемых понятий.}

\scnheader{частная предметная область по классу первичных элементов*}
\scnidtf{сужение предметной области по классу первичных элементов*}

\scnheader{частная предметная область по исследуемым отношениям*}
\scnidtf{частная предметная область по предмету исследования*}
\scnidtf{сужение предметной области по предмету исследования*}

\scnheader{родственные предметные области*}
\scniselement{бинарное отношение}
\scniselement{неориентированное отношение}
\scnexplanation{Связки отношения \textbf{\textit{родственные предметные области*}} связывают две \textit{предметные области}, имеющие общие элементы, однако не связанные отношением \textit{частная предметная область*}.}

\scnheader{sc-язык}
\scnidtf{максимальное множество знаний, являющихся фрагментами соответствующей предметной области (точнее, ее sc-модели)}
\scnexplanation{\textbf{\textit{sc-язык}} -– это подъязык (подмножество) \textit{SC-кода}, ориентированный на представление \textit{sc-текстов}, являющихся фрагментами некоторой \textit{предметной области}. Таким образом, каждому \textbf{\textit{sc-языку}} взаимно однозначно соответствует некоторая \textit{предметная область} (точнее, sc-модель некоторой \textit{предметной области}).}

\scnheader{sc-язык и соответствующая предметная область*}
\scniselement{бинарное отношение}
\scnexplanation{\textbf{\textit{sc-язык и соответствующая предметная область*}} - это бинарное ориентированное отношение, каждая связка которого связывает знак некоторого \textit{sc-языка} (первый компонент связки данного отношения) и знак соответствующей этому \textit{sc-языку} \textit{предметной области}.
}
\scnrelfrom{первый домен}{sc-язык}
\scnrelfrom{второй домен}{предметная область}

\scnendstruct \scnendcurrentsectioncomment

\end{SCn}