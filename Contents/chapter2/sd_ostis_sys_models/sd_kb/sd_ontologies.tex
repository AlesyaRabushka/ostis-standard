\begin{SCn}

\scnsectionheader{\currentname}

\scnstartsubstruct

\scnheader{Предметная область онтологий}
\scnidtf{Предметная область теории онтологий}
\scnidtf{Предметная область, объектами исследования которой являются онтологии}
\scniselement{предметная область}
\scnsdmainclasssingle{онтология}
\scnsdclass{интегрированная онтология;структурная спецификация;теоретико-множественная онтология;логическая онтология;логическая иерархия понятий;логическая иерархия высказываний;терминологическая онтология;онтология задач и решений задач;онтология классов задач и способов решения задач}
\scnsdrelation{онтология*;используемые константы*;используемые утверждения*}

\scnheader{онтология}
\scnidtf{система понятий соответствующей предметной области}
\scnidtf{концептуальный каркас (скелет) описания некоторой предметной области}
\scnidtf{концептуальная (семантическая) основа различных языков, обеспечивающих описание объектов исследования, принадлежащих заданной предметной области}
\scnidtf{семантический интерфейс для интеграции знаний по заданной предметной области и для согласованного понимания различными субъектами этих знаний}
\scnidtf{онтология соответствующей предметной области}
\scnidtf{описание концептов и отношений заданной предметной области}
\scnrelto{включение}{знание}
\scnsubdividing{интегрированная онтология;структурная спецификация;теоретико-множественная онтология;логическая иерархия понятий;логическая онтология;логическая иерархия высказываний;терминологическая онтология;онтология задач и решений задач;онтология классов задач и способов решения задач}
\scnexplanation{\textbf{\textit{онтология}} — это вид знаний, каждое из которых является спецификацией (описанием свойств) соответствующей \textit{предметной области}, ориентированной на описание свойств и взаимосвязей понятий, входящих в состав указанной \textit{предметной области}.}

\scnheader{онтология*}
\scnidtf{sc-онтология*}
\scnidtf{быть онтологией предметной области*}
\scnidtf{sc-онтология, специфицирующая заданную предметную область*}
\scnrelfrom{первый домен}{предметная область}
\scnrelfrom{второй домен}{онтология}
\scnexplanation{\textbf{\textit{онтология*}} — это бинарное отношение, связывающее некоторую предметную область с ее онтологией (спецификацией).}

\scnheader{интегрированная онтология}
\scnexplanation{\textbf{\textit{интегрированная онтология}} — это \textit{онтология}, объединяющая все \textit{онтологии} различного вида некоторой \textit{предметной области}.}

\scnheader{структурная спецификация}
\scnexplanation{\textbf{\textit{структурная спецификация}} — это \textit{онтология}, в которой описываются роли понятий, входящих в состав \textit{предметной области}, а также связи специфицируемых \textit{предметных областей} с другими \textit{предметными областями}.}

\scnheader{теоретико-множественная онтология}
\scnexplanation{\textbf{\textit{теоретико-множественная онтология}} — это \textit{онтология}, описывающая теоретико-множественные связи между понятиями заданной \textit{предметной области} (включение, разбиение, объединение, пересечение, разность множеств, область определения, домен, функция).}

\scnheader{логическая онтология}
\scnexplanation{\textbf{\textit{логическая онтология}} — это \textit{онтология}, описание системы высказываний заданной \textit{предметной области}.}

\scnheader{логическая иерархия понятий}
\scnidtf{логическая иерархия понятий, основанная на их определениях}
\scnexplanation{\textbf{\textit{логическая иерархия понятий}} — это \textit{онтология}, являющаяся надстройкой над \textit{логической онтологией}, включающая описание системы определений понятий заданной \textit{предметной области} с указанием набора понятий, через которые определяется каждое определяемое понятие рассматриваемой \textit{предметной области}.}

\scnheader{используемые константы*}
\scniselement{квазибинарное отношение}
\scnrelfrom{второй домен}{понятие}
\scnexplanation{\textbf{\textit{используемые константы*}} — это \textit{отношение}, связывающее некоторое \textit{определение} со множеством понятий, на основании которых определяется соответствующее данному \textit{определению} понятие в рамках рассматриваемой \textit{предметной области}.}

\scnheader{логическая иерархия высказываний}
\scnidtf{логическая система доказательств}
\scnidtf{логическая иерархия утверждений}
\scnidtf{логическая иерархия высказываний, основанная их на базовых доказательствах}
\scnexplanation{\textbf{\textit{логическая иерархия высказываний}} — это \textit{онтология}, являющаяся надстройкой над \textit{логической онтологией} и включающая описание системы утверждений рассматриваемой \textit{предметной области} с указанием набора \textit{утверждений}, через которые доказывается каждое \textit{утверждение}.}

\scnheader{используемые утверждения*}
\scniselement{квазибинарное отношение}
\scnrelfrom{второй домен}{утверждение}
\scnexplanation{\textbf{\textit{используемые утверждения*}} — это \textit{отношение}, связывающее утверждение со множеством утверждений, на основании которых оно доказывается в рамках рассматриваемой \textit{предметной области}.}

\scnheader{терминологическая онтология}
\scnexplanation{\textbf{\textit{терминологическая онтология}} — это \textit{онтология}, описывающая систему основных и неосновных терминов (имен, внешних обозначений), соответствующих концептам и отношениям заданной \textit{предметной области}, а также описание правил построения терминов для сущностей, являющихся элементами (экземплярами) указанных концептов и \textit{отношений}.}

\scnheader{онтология задач и решений задач}
\scnexplanation{\textbf{\textit{онтология задач и решений задач}} — это \textit{онтология}, описывающая задачи и их классы, решаемые в рассматриваемой \textit{предметной области}.}

\scnheader{онтология классов задач и способов решения задач}
\scnexplanation{\textbf{\textit{онтология классов задач и способов решения задач}} — это \textit{онтология}, описывающая способы решения задач и их классов в рамках \textit{предметной области}. Является \textit{метазнанием*} по отношению к \textit{онтологии задач и классов задач}.}

\scnendstruct \scnendcurrentsectioncomment

\end{SCn}