\begin{SCn}

\scnsectionheader{Предметная область и онтология технологий разработки баз знаний}

\scnstartsubstruct

\scnheader{Предметная область технологий разработки баз знаний}
\scnsdmainclasssingle{технология разработки баз знаний}
\scnsdclass{***}
\scnsdrelation{***}

\scnheader{методология разработки баз знаний}
\scnexplanation{Методология разработки онтологий представляет собой набор инструкций и руководств, описывающих процесс выполнения сложных процедур разработки онтологий. Она детализирует различные задачи, как они должны быть выполнены, в каком порядке и каким образом осуществлять документирование работы по созданию онтологий.}
\scnsubdividing{методология, поддерживающая совместную коллективную разработку онтологии;методология, не поддерживающая совместную коллективную разработку онтологии}
\scnsubdividing{методология, зависимая от инструментария;методология, частично зависимая от инструментария;методология, независимая от инструментария}
\scnauthorcomment{Есть много классификаций методологий, можно добавить}
\scnhaselement{скелетная методология Ушолда и Кинга}
\scnhaselement{методология Грюнингера и Фокса (TOVE)}
\scnhaselement{METHONTOLOGY}
\scnhaselement{On-To-Knowledge (OTK)}
\scnhaselement{KACTUS}
\scnhaselement{DILIGENT}
\scnhaselement{SENSUS}
\scnhaselement{UPON}

\scnheader{средство разработки баз знаний}
\scnsuperset{среда разработки онтологий}
\scnaddlevel{1}
\scnsuperset{инструмент создания онтологий}
\scnaddlevel{1}
\scnhaselement{Protege}
\scnhaselement{NeON}
\scnhaselement{Co4}
\scnhaselement{Ontolingua}
\scnhaselement{OntoEdit}
\scnhaselement{OilEd}
\scnhaselement{WebOnto}
\scnaddlevel{-1}
\scnsuperset{инструмент отображения, выравнивания и объединения онтологий}
\scnaddlevel{1}
\scnhaselement{PROMPT}
\scnhaselement{Chimaera}
\scnhaselement{OntoMerge}
\scnhaselement{OntoMorph}
\scnhaselement{OBSERVER}
\scnhaselement{FCAMerge}
\scnhaselement{ONION}
\scnaddlevel{-1}
\scnsuperset{инструмент аннотирования на основе онтологий}
\scnaddlevel{1}
\scnhaselement{MnM}
\scnhaselement{SHOE}
\scnhaselement{Knowledge Annotator}
\scnaddlevel{-2}
\scnsuperset{библиотека многократно используемых компонентов баз знаний}
\scnaddlevel{1}
\scnhaselement{Protege ontology library}
\scnhaselement{Ontaria ontology directory}
\scnaddlevel{-1}
\scnsuperset{средство коллективной разработки баз знаний}

\scnheader{средство коллективной разработки баз знаний}
\scnhaselement{Collaborative Protege}
\scnhaselement{NeON}
\scnhaselement{Co4}
\scnexplanation{В онтологическом инжиниринге любая онтология рассматривается как результат согласованной деятельности группы специалистов о модели некоторой области знаний. Исходя из этого с развитием методов и средств в области инженерии знаний все большее внимание стало уделяться инструментальной поддержке процесса коллективной разработки баз знаний и онтологий.}
\scntext{назначение}{
\begin{scnitemize}
\item управление взаимодействием и коммуникацией между разработчиками;
\item контроль за доступом к текущим результатам совместного проектирования;
\item фиксация авторских прав на экспертные знания, переданные в общее пользование;
\item обнаружение ошибок проектирования и управление коррекцией ошибок;
\item конкурентное управление изменениями.
\end{scnitemize}
}
\scntext{проблемы}{
\begin{scnitemize}
\item отсутствие развитых средств автоматического редактирования и верификации баз знаний, в том числе оценки полноты и избыточности;
\item отсутствие единого механизма коллективного создания баз знаний, включающего в себя средства согласования вносимых изменений между разработчиками разного уровня ответственности, типологию ролей разработчиков;
\item недостаточный уровень расширяемости инструментов разработки.
\end{scnitemize}
}

\scnheader{технология разработки баз знаний}
\scnsuperset{Wiki-технология}
\scntext{проблемы}{Несмотря на достигнутые успехи в области создания баз знаний, остаются актуальными следующие проблемы:

\begin{scnitemize}
\item трудоемкость одновременного использования моделей представления различных видов знаний;
\item несовместимость уже разработанных компонентов баз знаний приводит к необходимости повторной разработки уже существующих решений;
\item изменения, вносимые в базу знаний, могут повлечь необходимость внесения существенных изменений в саму структуру базы знаний, особенно в случае динамических баз знаний;
\item несмотря на наличие достаточно развитых средств создания баз знаний, они не в полной мере обеспечивают комплексную поддержку (в том числе – информационную) коллектива разработчиков на всех стадиях проектирования базы знаний, а также не обладают достаточной гибкостью и расширяемостью;
\item существующие средства ориентированы, как правило, на какой-либо конкретный формат хранения знаний, что затрудняет перенос уже разработанной базы знаний на другую платформу интерпретации модели.
\end{scnitemize}

Основной причиной всех указанных проблем является отсутствие в рамках базы знаний интеллектуальной системы совместимости различных видов знаний, в том числе метазнаний. Совместимость различных видов знаний включает два аспекта: синтаксическую совместимость, что подразумевает унификацию формы представления знаний, и семантическую совместимость, что подразумевает однозначную и единую для всех фрагментов базы знаний трактовку используемых понятий. Кроме того, при модификации и расширении база знаний должна сохранять свою целостность и непротиворечивость.

Существующие подходы к разработке баз знаний, как правило, предполагают решение задачи обеспечения синтаксической совместимости знаний путем соединения разнородных моделей представления знаний, а также разработки новых интегрированных моделей и новых языков представления знаний. Разработка базы знаний таким способом приводит к дополнительным накладным расходам при интеграции и обработке разнородных знаний и, как следствие, к резкому увеличению трудозатрат при модификации таких баз знаний и добавлении новых видов знаний.

Попытки решения задачи обеспечения семантической совместимости раличных видов знаний в рамках разрабатываемых баз знаний связаны с построением онтологий верхнего уровня, однако, отсутствие единой формальной основы, обеспечивающей однозначную интерпретацию представляемых знаний и вводимых новых понятий, не привело к решению указанной задачи. Кроме того, существующие средства создания баз знаний предполагают, что процессы разработки и модификации базы знаний осуществляются отдельно от процесса ее использования, что приводит к дополнительному усложнению решения задачи обеспечения совместимости знаний различного вида.}

%\newpage
%\section[Состояние работ в области формального представления \\знаний]{Состояние работ в области формального представления знаний}

\scnheader{модель представления знаний}
\scnsubdividing{продукционная модель;логическая модель;фреймовая модель;семантическая сеть}
\scntext{примечание}{На сегодняшний день существуют десятки моделей представления знаний, однако большинство из них базируются на основных четырех моделях, представленных выше.}
\scntext{примечание}{Отдельное внимание следует уделить рассмотрению средств для представления знаний, предлагаемых в рамках направления \textit{Semantic Web}, по причине их проработанности и распространенности.}

%К настоящему времени теоретические исследования в области разработки формальных моделей представления знаний и формальных моделей решения задач привели к широкому многообразию таких моделей.

%На сегодняшний день существуют десятки моделей представления знаний, однако большинство из них базируются на основных четырех: семантические сети, фреймовые, продукционные и логические модели~\cite{Gavrilova2001}. 

\scnheader{продукционная модель}
\scnexplanation{\textit{продукционная модель} является системой продукций, представляющих собой конструкции типа "Если (условие), то (действие)"{}.}
\scntext{примечание}{Продукционные модели удобны для представления логических взаимосвязей между фактами. Чаще всего данные модели используются для представления знаний в экспертных системах.}

%\textit{Продукционные модели} являются системами продукций, представляющих собой конструкции типа «Если (условие), то (действие)». Продукционные модели удобны для представления логических взаимосвязей между фактами. Чаще всего данные модели используются для представления знаний в экспертных системах~\cite{Averkin1992}.

\scnheader{логическая модель}
\scnexplanation{\textit{логическая модель} основывается на классическом исчислении предикатов и его расширениях, позволяет описывать свойства предметной области в виде набора аксиом и правил вывода.}
\scntext{примечание}{\textit{Логические модели}, как и \textit{продукционные модели}, удобны для представления логических взаимосвязей между фактами. Их отличительной особенностью являются строгость и формализованность.}

%\textit{Логические модели}, основанные на классическом исчислении предикатов и его расширениях, позволяют описывать свойства предметной области в виде набора аксиом и правил вывода ~\cite{Averkin1992}. Логические модели, как и продукционные, удобны для представления логических взаимосвязей между фактами. Их отличительной особенностью являются строгость и формализованность. 

\scnheader{cемантическая сеть}
\scnexplanation{\textit{семантическая сеть} -- модель представления знаний в виде графовой структуры, вершинами которой являются информационные единицы, а дуги обозначают связи между ними.}
\scntext{достоинство}{Главной особенностью семантических сетей является соединение в одном представлении синтаксического и семантического аспектов описаний знаний предметной области, что значительно снижает вычислительную сложность обработки знаний}
\scntext{примечание}{\textit{Семантическая сеть} является весьма распространенным способом представления знаний в интеллектуальных системах.}

%\textit{Семантические сети} -- модель представления знаний в виде графовой структуры, вершинами которой являются информационные единицы, а дуги обозначают связи между ними. Семантическая сеть является весьма распространенным способом представления знаний в интеллектуальных системах. В зависимости от характера отношений, допустимых в них, семантические сети имеют различную природу~\cite{Averkin1992}. Главной особенностью семантических сетей является соединение в одном представлении синтаксического и семантического аспектов описаний знаний предметной области, что значительно снижает вычислительную сложность обработки знаний~\cite{Golenkov2017, Osipov2015}.

\scnheader{фреймовая модель}
\scnexplanation{\textit{Фреймовые модели} представляют собой системы взаимосвязанных фреймов.}
	\scnaddlevel{1}
	\scntext{примечание}{Под фреймом объекта или явления понимается его минимальное описание, содержащее всю существенную информацию об этом объекте или явлении и обладающее таким свойством, что удаление из описания любой его части приводит к потере существенной информации, без которой описание объекта или явления не может быть достаточным для их идентификации. Фрейм задается именем и набором слотов, описывающих свойства объекта или явления.}
		\scnaddlevel{1}
		\scnrelfrom{автор}{Минский М.}
		\scnaddlevel{-1}	
	\scnaddlevel{-1}
%\textit{Фреймовые модели} представляют собой системы взаимосвязанных фреймов. Понятие фрейма было введено М. Минским, который понимал под фреймом объекта или явления его минимальное описание, содержащее всю существенную информацию об этом объекте или явлении и обладающее таким свойством, что удаление из описания любой его части приводит к потере существенной информации, без которой описание объекта или явления не может быть достаточным для их идентификации. Фрейм задается именем и набором слотов, описывающих свойства объекта или явления~\cite{Averkin1992}. 

%Каждая из описанных моделей имеет свои достоинства и недостатки, и при разработке баз знаний инженеру по знаниям необходимо сделать выбор в пользу наиболее подходящей модели, так как различные виды знаний требуют использования различных моделей их представления. 

%Основной проблемой, которая решается в данной диссертационной работе, является проблема совместимости представления различных видов знаний. Существующие подходы к разработке баз знаний, как правило, предполагают решение задачи обеспечения синтаксической совместимости знаний путем соединения разнородных моделей представления знаний, а также разработки новых интегрированных моделей и новых языков представления знаний. Разработка базы знаний таким способом приводит к дополнительным накладным расходам при интеграции и обработке разнородных знаний, и, как следствие, к резкому увеличению трудозатрат при модификации таких баз знаний и добавлении новых видов знаний.

%Существующие инструментальные средства создания баз знаний, как правило, ориентированы на использование одной из упомянутых выше моделей. Однако при создании систем, основанных на знаниях, в особенности при решении комплексных задач, возникает необходимость представления различных видов знаний в рамках одной базы знаний, чего не может обеспечить ни одна из вышеперечисленных моделей, взятых в отдельности~\cite{Zagorulko2013, Klestiov1986}. 

%В связи с этим возникает необходимость в создании универсальной модели представления знаний, которая позволила бы представлять любые виды знаний в унифицированном, удобном для обработки виде (рисунок \ref{pic_1.4_1}). Подходы к решению данной проблемы рассмотрены в работах~\cite{Zagorulko2013Concept, Ivashenko2014, TuzovskiIntegration2011, Chan2006}. 

%{\begin{figure}[H]
%\begin{center}
%\includegraphics[width=0.5\textwidth]{man-source/images/pic_1_4_1.png}\\[2mm]
%\caption{Унификация представления знаний}
%\label{pic_1.4_1}
%\end{center}
%\end{figure}

\scnheader{средства Semantic Web}
\scnexplanation{\textit{cредства Semantic Web} представляют собой набор методов и технологий, предназначенных для представления информации в виде, пригодном для машинной обработки.}
\scntext{примечание}{Информация представляется в виде \textit{семантической сети}, специфицируемой посредством \textit{онтологий}. Стандартизация представления информации позволяет компьютерной системе получать различную фактографическую информацию и делать на ее основе логические заключения.}
\scnrelfromset{основные инструменты}{модель описания ресурсов RDF;языки, обеспечивающие представление RDF-данных;средства представления метаданных RDF Schema;принципы представления знаний в виде онтологий;языки описания онтологий\\
	\scnaddlevel{1}
	\scnrelfromlist{включение;~пример}{OWL Lite;OWL DL;OWL Full}
	\scnaddlevel{1}
	\scntext{примечание}{Онтологии, представленные с помощью OWL, включают описание классов, их свойств, обеспечивающих связи между классами, и экземпляров этих классов.}
	\scntext{примечание}{Практика позволила выявить ограниченность выразительных способностей OWL и недостатки технического характера:
	\begin{scnitemize}
	\item сложность синтаксического разбора;
	\item невозможность обнаружить опечатки в именах.
	\end{scnitemize} 
Это привело к созданию новой версии языка — \textit{OWL 2}, целью создания которого являлось устранение указанных недостатков}
	\scnaddlevel{1}
	\scntext{примечание}{В настоящий момент OWL 2 является общепризнанным стандартом для представления онтологий.}
	\scnaddlevel{-1}
	\scnaddlevel{-1}
	\scnaddlevel{-1}
;хранилища баз знаний на основе RDF\\
	\scnaddlevel{1}
	\scnrelfromlist{включение;~пример}{Sesame;HyperGraphDB;Neo4j;Virtuoso;AllegroGraph}
	\scnaddlevel{1}
	\scntext{примечание}{Данные хранилища обеспечивают хранение и доступ к данным средствами языка запросов SPARQL}
	\scnaddlevel{-1}
	\scnaddlevel{-1}
;язык запросов к хранилищам RDF-данных SPARQL}
\scntext{недостатки}{К недостаткам cредств представления знаний, предлагаемых в рамках подхода Semantic Web можно отнести:
	\begin{scnitemize}
	\item OWL запрещает наличие дуг, инцидентных другим дугам, что вынуждает вводить дополнительную вершину, обозначающую связку, и далее связывать с ней все необходимые вершины соответствующими отношениями.Такой подход имеет ряд недостатков, в частности, приводит к необходимости:
	\begin{scnitemizeii}
	\item вводить новые отношения, которые связывают такую дополнительную вершину с остальными;
	\item преобразовывать указанным образом все связки рассматриваемого отношения, в противном случае связки одного и того же отношения в разных конструкциях будут представлены по-разному, что сильно затруднит обработку представленных таким образом знаний.
	\end{scnitemizeii}	
	 \item отсутствие возможности описания метасвязей;
	 \item отсутствие средств описания нечетких знаний;
	 \item отсутствие возможности описания свойств целых классов сущностей;
	 \item невозможность описания исключений из некоторых правил;
	 \item возможность задания метасвязей только для отдельных связей;
	 \item и др.
	\end{scnitemize}}
\scntext{примечание}{Таким образом, подход к созданию баз знаний на основе Semantic Web предоставляет средства формального представления знаний и доступа к ним, которые, однако, не позволяют в унифицированном виде представлять все виды знаний, необходимые для функционирования современных интеллектуальных систем.}

\scnheader{язык представления знаний}
\scntext{примечание}{Каждой \textit{модели представления знаний} соответствует некоторое множество \textit{языков представления знаний}, реализующих эти модели.}
\scnexplanation{В языках представления знаний, как правило, разделяется синтаксическая и семантическая составляющие. Синтаксис задает правила, по которым строятся конструкции данного языка, а семантика определяет правила интерпретации указанных конструкций.}
\scnsubdividing{CycL\\
	\scnaddlevel{1}
	\scnexplanation{\textit{CycL} -- язык представления знаний, основанный на онтологиях и используемый в рамках проекта Cyc.}
	\scnaddlevel{-1}
;IDEF5\\
	\scnaddlevel{1}
	\scnidtf{Integrated  Definitions  for  Ontology  Description  Capture  Method}
	\scnexplanation{\textit{IDEF5} -- стандарт онтологического исследования для наглядного представления данных, полученных в результате обработки онтологических запросов в простой естественной графической форме.}
	\scnaddlevel{-1}
;Prolog\\
	\scnaddlevel{1}
	\scnexplanation{\textit{Prolog} -- язык, основанный на языке предикатов математической логики дизъюнктов Хорна, представляющей собой подмножество логики предикатов первого порядка}
	\scnaddlevel{-1}
;CLIPS\\
	\scnaddlevel{1}
	\scnidtf{C Language Integrated Production System}
	\scnexplanation{\textit{CLIPS} -- язык представления знаний, основанный на логических правилах, использующийся одноименной программной оболочкой для создания экспертных систем}
	\scnaddlevel{-1}}
%Каждой модели представления знаний соответствует некоторое множество языков представления знаний, реализующих эти модели.
%В языках представления знаний, как правило, разделяется синтаксическая и семантическая составляющие. Синтаксис задает правила, по которым строятся конструкции данного языка, а семантика определяет правила интерпретации указанных конструкций.

%В настоящий момент для представления знаний используются различные языки ~\cite{Kazekin2008}. Рассмотрим некоторые из них: CycL ~\cite{CycL2016} (язык представления знаний, основанный на онтологиях и используемый в рамках проекта Cyc), IDEF5~\cite{Benjamin1994} (Integrated  Definitions  for  Ontology  Description  Capture  Method – стандарт онтологического исследования для наглядного представления данных, полученных в результате обработки онтологических запросов в простой естественной графической форме), Prolog~\cite{Prolog} (основанный на языке предикатов математической логики дизъюнктов Хорна, представляющей собой подмножество логики предикатов первого порядка), CLIPS~\cite{CLIPS} (C Language Integrated Production System – язык представления знаний, основанный на логических правилах, использующийся одноименной программной оболочкой для создания экспертных систем) и др.

%Отдельное внимание следует уделить рассмотрению средств для представления знаний, предлагаемых в рамках направления Semantic Web, по причине их проработанности и распространенности. Данное направление активно развивается консорциумом W3C, основной задачей которого является разработка стандартов Semantic Web~\cite{Horoshevski2008, W3C}. 

%Средства Semantic Web представляют собой набор методов и технологий, предназначенных для представления информации в виде, пригодном для машинной обработки. Информация представляется в виде семантической сети, специфицируемой посредством онтологий. Стандартизация представления информации позволяет компьютерной системе получать различную фактографическую информацию и делать на ее основе логические заключения. Использование стандартов W3C при разработке интеллектуальных приложений в последние десятилетия стало весьма популярным.

%В рамках направления Semantic Web для представления знаний были разработаны:
%\begin{itemize}
%  \item модель описания ресурсов RDF  (Resource Description Framework) и языки, обеспечивающие представление RDF-данных ~\cite{RDF}; 
%  \item средства представления метаданных RDF Schema;
%  \item принципы представления знаний в виде онтологий и языки описания онтологий (OWL Lite, OWL DL, OWL Full) ~\cite{OWL2012};
%  \item язык запросов к хранилищам RDF-данных SPARQL~\cite{SPARQL2013}; 
%  \item ряд других стандартов.
%\end{itemize}

%Также разработаны эффективные хранилища баз знаний на основе RDF, такие как Sesame, HyperGraphDB, Neo4j, Virtuoso, AllegroGraph, обеспечивающих хранение и доступ к данным средствами языка запросов SPARQL. Для редактирования онтологий, созданных на основе стандартов W3C, создано большое число редакторов, обладающих довольно широким функционалом~\cite{OntologyTools2016, OWL2016}.

%Получив статус рекомендации W3C, язык OWL стал активно использоваться в основанных на знания программных продуктах и исследовательских проектах. Системы автоматического логического вывода и редакторы онтологий стали ориентироваться на OWL. Практика позволила выявить ограниченность выразительных способностей OWL и недостатки технического характера (сложность синтаксического разбора, невозможность обнаружить опечатки в именах). Это привело к созданию новой версии языка — OWL 2, целью создания которого являлось устранение указанных недостатков.

%Онтологии, представленные с помощью OWL, включают описание классов, их свойств, обеспечивающих связи между классами, и экземпляров этих классов. Для записи таких онтологий могут использоваться разные форматы: RDF/XML, OWL/XML, Манчестерский синтаксис.

%На сайте консорциума W3C~\cite{N-aryRelations2006, RepresOWL2005} представлены некоторые ситуации, сложные с точки зрения формализации на OWL, и варианты их решения. Отдельный интерес представляет ситуация, когда необходимо формализовать некоторую метаинформацию о связке какого-либо отношения. OWL запрещает наличие дуг, инцидентных другим дугам, поэтому в этом случае разработчики предлагают вводить дополнительную вершину, обозначающую связку, и далее связывать с ней все необходимые вершины соответствующими отношениями. Такой подход имеет ряд недостатков, в частности, приходится вводить новые отношения, которые связывают такую дополнительную вершину с остальными. Кроме того, такой подход вынуждает преобразовывать указанным образом все связки рассматриваемого отношения, в противном случае связки одного и того же отношения в разных конструкциях будут представлены по-разному, что сильно затруднит обработку представленных таким образом знаний.

%Средства представления знаний, предлагаемые в рамках подхода Semantic Web, обладают некоторыми недостатками, одним из которых является их ограниченность, в частности, отсутствие возможности описания метасвязей, средств описания нечетких знаний, отсутствие возможности описания свойств целых классов сущностей, невозможность описания исключений из некоторых правил и др.~\cite{Gorshkov2016}.

%Таким образом, подход к созданию баз знаний на основе Semantic Web предоставляет средства формального представления знаний и доступа к ним, которые, однако, не позволяют в унифицированном виде представлять все виды знаний, необходимые для функционирования современных интеллектуальных систем. Тем не менее в настоящий момент OWL 2 является общепризнанным стандартом для представления онтологий. Проблема задания связей над связями (метасвязей) частично решена в RDF при помощи механизма реификации~\cite{Semantics}. Однако данный механизм значительно увеличивает количество элементов, необходимых для описания одной связи, поскольку обязывает отдельными триплетами описать все три компонента отдельной связки. Кроме того, оба представленных механизма обеспечивают возможность задания метасвязей только для отдельных связей, в то время как на практике часто возникает необходимость специфицировать целый фрагмент базы знаний, например, указать его авторство, временные характеристики, связь с другими фрагментами и т. д.
\newpage
\section{Виды знаний и подходы к их структуризации}

Под знаниями в широком смысле понимается совокупность сведений, которые формируют целостное описание некоторого объекта, явления или проблемы~\cite{Averkin1992}. В работе~\cite{Gavrilova2001} понятие знаний определяется как хорошо структурированные данные, или данные о данных (т. е. метаданные). 

Понятие знаний тесно связано с понятием предметной области. Под \textit{предметной областью} в инженерии знаний понимают совокупность реальных или абстрактных объектов (сущностей), связей и отношений между этими объектами, а также процедур преобразования этих объектов для решения задач, возникающих в предметной области~\cite{Averkin1992}.

Знания о некоторой предметной области представляют собой совокупность сведений об объектах этой предметной области, их существенных свойствах и связывающих их отношениях, процессах, протекающих в данной предметной области, а также методах анализа возникающих в ней ситуаций и способах разрешения ассоциируемых с ними проблем \cite{Gavrilova2001}.

Согласно источникам~\cite{Averkin1992, Jong1996} знания разделяются на \textit{декларативные} и \textit{процедурные}. Под \textit{декларативными знаниями} понимаются знания, которые записаны в памяти интеллектуальной системы так, что они непосредственно доступны для использования после обращения к соответствующему полю памяти. В виде декларативных знаний обычно записывается информация о свойствах предметной области, фактах, имеющих в ней место и тому подобная информация. Декларативным знаниям противопоставляются \textit{процедурные} – это знания, которые хранятся в памяти интеллектуальной системы в виде описаний процедур, с помощью которых их можно получить. В виде процедурных знаний обычно описываются информация о предметной области, характеризующая способы решения задач в этой области, а также различные инструкции, методики и тому подобная информация.

В инженерии знаний также известны следующие признаки классификации знаний:
\begin{itemize}
  \item по глубине;
  \item по владельцу;
  \item по форме;
  \item по источнику получения;
  \item по сфере применения.
\end{itemize}

Полная классификация по указанным признакам приведена на \mbox{рисунке \ref{pic_1.2}~\cite{Gavrilova2016, Tarasov2003}.} 

\begin{figure}[H]
\begin{center}
\includegraphics[width=1\textwidth]{man-source/images/pic_1_2.pdf}\\[2mm]
\caption{Признаки классификации знаний}
\label{pic_1.2}
\end{center}
\end{figure}

Важным этапом создания базы знаний является ее структуризация. Данная стадия традиционно является (наряду со стадией извлечения) <<узким местом>> в жизненном цикле разработки интеллектуальных систем~\cite{Gavrilova2008}. Подходы к структурированию знаний основываются на современной теории сложных систем~\cite{Wasson2005}, где акцент ставится на стадии проектирования. 

Структуризация базы знаний, т. е. выделение в ней различных связанных между собой фрагментов, необходима по целому ряду причин~\cite{Gavrilova2016}:
\begin{itemize}
  \item для повышения эффективности обработки баз знаний путем указания областей решения задач;
  \item для выделения независимых фрагментов базы знаний с целью организации распределения работ по проектированию (когда разным исполнителям поручается разработка разных фрагментов базы знаний, имеющих достаточно четкие границы);
  \item для дидактических целей (человеку, усваивающему некоторые знания, желательно иметь своего рода оглавление этих знаний, что позволяет планировать их усвоение и рассматривать их с различной степенью детализации), что является немаловажным фактором при работе с системами, основанных на знаниях.
\end{itemize}

Целью структуризации базы знаний является ее декомпозиция на множество фрагментов, связанных друг с другом тем или иным набором отношений. В зависимости от типологии таких фрагментов и набора отношений могут выделяться различные подходы к структуризации знаний.

Существенный вклад в математическую теорию систем и основы структурирования внесли исследователи Т.~А.~Гаврилова~\cite{Gavrilova1992},  В.~М.~Глушков~\cite{Glushkov1964}, Н.~Н.~Моисеев~\cite{Moiseev1981}, Г.~С.~Осипов~\cite{Osipov1997, Osipov2015}, Д.~А.~Поспелов~\cite{Pospelov1986}.

База знаний чаще всего содержит три уровня знаний~\cite{Lapshin2010}:
\begin{itemize}
  \item общие, или абстрактные знания, которые описывают закономерности, общие для большого числа предметных областей (знания о теоретико-множественных связях, знания о терминах, знания о логических моделях предметных областей, знания о базовых математических отношениях и операциях и др.);
  \item знания о конкретной предметной области (domain-specific knowledge). Например, знания по геометрии, истории, медицине и др.;
  \item конкретные знания, добавляемые в базу знаний пользователями или программными агентами.
\end{itemize}

В работе~\cite{Gavrilova2008} предложена классификация различных методов структурирования знаний (рисунок \ref{pic_1.3}), в рамках которых выделены различные виды знаний.

В основе методов структурирования информации, как и методов разработки сложных систем, традиционно используется иерархический подход~\cite{Mesarovich1978} как методологический прием разделения формально описанной системы на уровни (блоки, или модули). На верхних уровнях иерархии представляются описания наименьшей степени детализации, которые отражают общие особенности предметной области (или системы), на следующих уровнях степень детализации описания увеличивается, при этом предметная область (или система) рассматривается не целиком, а отдельными частями. Преимуществом данного подхода является сведение исходной задачи к подзадачам, которые должны быть решены в рамках предметной области (или системы)~\cite{Gavrilova2016}.

В качестве одной из исторически первых моделей, применимых для структуризации знаний, можно также рассматривать модель клубных систем, предложенную в работе~\cite{Borschev1976}. Модель, предложенная в данной работе, позволяет выделять отдельные фрагменты заданного множества и рассматривать отношения между ними, и далее при необходимости осуществлять аналогичные действия уже с выделенными фрагментами, опускаясь таким образом на необходимый уровень детализации. Такая модель позволяет, с одной стороны, специфицировать целые фрагменты исследуемого множества, рассматривая их как отдельные сущности, с другой стороны – абстрагироваться от детального рассмотрения тех фрагментов, для которых в текущий момент этого не требуется.

\begin{figure}[H]
\begin{center}
\includegraphics[width=1.0\textwidth]{man-source/images/pic_1_3.pdf}\\[2mm]
\caption{Классификация методов структурирования знаний}\label{pic_1.3}
\end{center}
\end{figure}

В работе~\cite{Gavrilova2016} описан подход к структуризации знаний, основанный на методологии объектно-структурного анализа~\cite{Gavrilova1995}. Данный подход основывается на разбиении предметной области на восемь слоев (страт) в зависимости от вида знаний, рассматриваемого на том или ином слое (рисунок \ref{pic_1.4}).

В работе~\cite{Krivorutski1995} был предложен подход к структуризации знаний, в основе которого лежит концептуальная модель структурирования знаний, базирующаяся на представлении различных видов знаний как объектов расслоенного пространства. Такая модель получила название стратифицированной фрактальной модели (или ФС-модели). ФС-модель определяется как совокупность непересекающихся слоев (информационных миров) и их отображений в информационном пространстве. Каждому уровню соответствует свой слой этого пространства и, следовательно, свой информационный мир.


\begin{figure}[H]
\begin{center}
\includegraphics[width=1.0\textwidth]{man-source/images/pic_1_4.pdf}\\[2mm]
\caption{Виды знаний, выделяемые в рамках объектно-структурного анализа}
\label{pic_1.4}
\end{center}
\end{figure}


Графически ФС-модель представляется в виде совокупности вложенных сферических оболочек. Точка на одной из сфер, условно обозначающая информационный объект, в свою очередь может быть расслоена при необходимости более детального рассмотрения данного объекта. Каждый из исследователей на практике работает с определенным <<фракталом>> знаний – фрагментом информационного пространства, что соответствует, например, выделению дисциплин при изучении реального мира. Фрактальный подход может рассматриваться как обобщение объектного подхода к проектированию программных и информационных систем.

На сегодняшний день наиболее эффективным средством формализации и структуризации различных областей знаний являются онтологии~\cite{Guarino1995}. Данный подход используется в большинстве современных решений проблем разработки интеллектуальных систем и их компонентов~\cite{BorgestRole2014, BorgestVvedenie2014, Gavrilova2016, Gladun2013, Globa2014, Fillipov2016, Efimenko2011, Kleschev2001, Kudruavtsev2010, Lapshin2010, Rubashkin2012, Gruber1995, Guarino1995, Karray2012, Mizoguchi1995, Sowa1995}. 

Термин \textit{онтология} был введен Т. Грубером и трактуется как эксплицитная спецификация концептуализации~\cite{Gruber1995}. Применительно к интеллектуальным системам под онтологией понимается формальная спецификация предметной области, включающая описания классов объектов исследования и отношений, заданных на объектах исследования. 

В качестве признаков классификации онтологий используются такие признаки, как цель создания, степень формальности, содержимое~\cite{Dobrov2006}. 

Онтологии по цели создания делятся на:
\begin{itemize}
  \item \textit{онтологии представления} – целью такого вида онтологии является описание области представления знаний, создание языка для спецификации онтологий более низких уровней;
  \item \textit{онтологии верхнего уровня} – назначение данного вида онтологий заключается в создании онтологий для общих предметных областей, свойства которых исследуются онтологиями более низкого уровня. Онтологии верхнего уровня могут быть повторно используемы вместе с соответствующими им онтологиями более низкого уровня;
  \item \textit{онтологии предметных областей} – область охвата таких онтологий ограничена одной предметной областью. Такие онтологии обобщают понятия, использующиеся в некоторых задачах домена, абстрагируясь от самих задач. Эти онтологии повторно используемы внутри одной предметной области;
  \item прикладные онтологии – назначение такого вида онтологии в описании концептуальной модели конкретной задачи или приложения. Такую онтологию нет возможности использовать повторно.
\end{itemize}

По степени формальности онтологии делятся на неформальные (описываемые на естественном языке), более формализованные (основанные на отношениях таксономии) и сильно формализованные, задают формальную семантику понятий в разрешенных языком точных и непротиворечивых выражениях~\cite{Nikonenko2009}.

Онтологии по содержимому делятся на общие онтологии (описывающие сущности, события, пространство, время и др.), онтологии задач (описывающие типологию классов задач и их спецификацию) и предметные онтологии (описывающие множества предметов).

Кроме того, выделяют следующие виды онтологий: простые и многоуровневые, легкие и весомые, статические и динамические~\cite{Kleschev2001}. 

В свою очередь в зависимости от набора используемых отношений онтологии делятся на следующие классы~\cite{Gavrilova2001}: 
\begin{itemize}
  \item словарь понятий – явно определяется смысл терминов словаря с помощью соответствующей функции интерпретации;
  \item пассивный словарь – включает множество интерпретируемых понятий предметной области, множество интерпретирующих терминов и соответствующие им функции интерпретации;
  \item таксономия понятий – описывает отношения типа <<is a>> между понятиями предметной области;
  \item мерономия понятий – описывает отношения типа <<part of>> между понятиями предметной области;
  \item метасистема понятий – описывает отношения <<is a>> и <<part of>> между понятиями предметной области; 
  \item онтология с ограничениями – описывает отношения <<is a>>, <<part of>> и другие, дополнительно уточняемые отношения между понятиями предметной области;
  \item полная онтология – включает описания отношений <<is a>>, <<part of>> и других, дополнительно уточняемых отношений между понятиями предметной области, а также соответствующие им функции интерпретации.
\end{itemize}

Онтологии позволяют сформировать понятийный базис рассматриваемой предметной области, что является ключевым фактором в процессе структуризации знаний. Онтологии являются основой любой базы знаний и используются для интеграции различных баз знаний и их частей~\cite{Gavrilova2016}. Данный факт обуславливает целесообразность использования онтологического подхода к решению поставленных в данной диссертационной работе проблем.

В работах~\cite{Lapshin2010, Hazman2011} отмечается актуальность проблемы изменения онтологий с течением времени. В настоящий момент онтологии представляют собой статические спецификации предметной области, которые не отражают изменений в онтологиях, происходящих во времени. Между тем появляются новые понятия, некоторые понятия устаревают и становятся неактуальными, меняется терминология, изменяется точка зрения на трактовку тех или иных понятий. Автор отмечает необходимость учета таких возможностей при разработке инструментов создания онтологий. Решение одного из аспектов данной проблемы, связанного с формальной спецификацией динамики онтологий, предложено в рамках данной диссертационной работы.

Проведенный анализ показывает, что на сегодняшний день разработано большое количество подходов к структуризации баз знаний, в основе каждого из которых лежат различные варианты декомпозиции баз знаний, что позволяет рассматривать базу знаний в различных аспектах. Однако актуальной остается задача обеспечения возможности использования различных подходов к структуризации в рамках одной базы знаний одновременно.

\newpage
\section{Онтологии верхнего уровня}

Отдельный интерес с точки зрения решения проблемы совместимости различных видов знаний представляют онтологии верхнего уровня, которые ориентированы на описание фундаметальных понятий, таких, как <<сущность>>, <<явление>>, <<отношение>>, <<действие>> и др. В онтологии верхнего уровня представлена систематизация знаний о реальном мире безотносительно к какой-либо конкретной предметной области. Основной функцией, которая возлагалась на онтологии верхнего уровня, является поддержка семантической совместимости онтологий предметных областей и прикладных онтологий. Поддержка предполагает создание общей точки для формулирования определений. Термины предметно-ориентированных онтологий подчинены терминам онтологии более высокого уровня.

Рассмотрим некоторые наиболее проработанные проекты по разработке онтологий верхнего уровня.

OpenCyc \cite{OpenCyc} -- открытая для общего пользования часть коммерческого проекта Cyc, на текущий момент наиболее масштабной и детализированной онтологии в области общего знания. База знаний OpenCyc содержит информацию из различных предметных областей: философия, математика, химия, биология, психология, лингвистика и т. д. Структурно база знаний OpenCyc состоит из констант (терминов) и правил (формул), оперирующих этими константами. Правила делятся на два вида: аксиомы и выводимые утверждения. Под аксиомами в OpenCyc понимаются утверждения, которые были явно и вручную введены в базу знаний экспертами, а не появились там в результате работы машины вывода. Все утверждения или формулы в базе знаний OpenCyc фиксируются на языке CycL \cite{CycL2016}, выразительно эквивалентном исчислению предикатов первого порядка.

DOLCE (Descriptive Ontology for Linguistic and Cognitive Engineering) \cite{DOLCE} -- базовая онтология проекта WonderWeb. Ее предполагается применять в проекте создания семантического веба (SemanticWeb) для обеспечения согласования между интеллектуальными агентами, использующими разную терминологию. Эта онтология имеет когнитивный уклон, поскольку в основном фиксируются онтологические категории естественного языка и знания «здравого смысла». Онтология не претендует на звание универсальной, стандартной или общей. В основе данной онтологии лежит разделение всех сущностей на универсальные, которые могут иметь экземпляры, и индивидные (частные), которые не могут иметь экземпляры.

SUMO (Suggested Upper Merged Ontology) \cite{SUMO} -- онтология верхнего уровня, разработанная в рамках проекта рабочей группы IEEE SUO (IEEE Standard Upper Ontology Working Group) и Teknowledge. Проект претендует на статус формирования стандарта для онтологий верхнего уровня. Онтология SUMO содержит наиболее общие и самые абстрактные концепты, имеет исчерпывающую иерархию фундаментальных понятий (около 1000 понятий), а также набор аксиом (примерно 4000), определяющих эти понятия. Назначение SUMO -- содействовать улучшению интероперабельности данных, извлечения и поиска информации, автоматического вывода (доказательства), обработки естественного языка. Иерархия классов в SUMO менее запутана, чем в OpenCyc, и более удобна для практического применения, чем DOLCE. SUMO является онтологией в чистом виде и не имеет ни достаточно развитого онторедактора, ни машины логического вывода; создатели SUMO предоставляют лишь информацию, которая может обрабатываться программно и включаться в качестве составной части в различные приложения и обрабатываться средствами этих приложений.

Онтология Джона Совы определяет базовые онтологические категории, полученные автором из
источников по логике, лингвистике, философии и искусственному интеллекту \cite{Sowa1995}. Для того чтобы сохранить открытость, онтология, по мнению Джона Совы, должна быть основана не на фиксированной иерархии концептов, а на каркасе, описывающем различия, с помощью которого иерархия генерируется автоматически. В любом конкретном приложении концепты не определяются рисованием линий на диаграмме, а задаются выбором подходящего множества различий.

WordNet \cite{WordNet} -- один из наиболее полно разработанных тезаурусов общего назначения. Центральным объектом в WordNet является синсет, множество синонимов (или синонимический ряд). WordNet содержит около 70 тыс. синсетов, организованных в иерархию по отношению <<надкласс-подкласс>>. Существующая версия WordNet охватывает общеупотребительную лексику современного английского языка. В период с марта 1996 г. по сентябрь 1999 г. при финансировании Европейской комиссии был создан многоязычный вариант WordNet - EuroWordNet, который объединил в себе WordNet-словари английского, датского, испанского, итальянского, немецкого, французского, чешского и эстонского языков. Первоначально WordNet создавался как модель человеческой памяти. Появление WordNet и возможность его свободного использования вызвали большое число исследований по применению этого тезауруса в самых различных приложениях автоматической обработки текстов. Большое количество экспериментов привело к массовому выявлению и обсуждению проблем и недостатков WordNet, препятствующих его эффективному применению. К таким проблемам относятся: отсутствие отношений между частями речи; различия значений в WordNet слишком тонки для компьютерных приложений; проблема нехватки отношений между синсетами, относящимися к одной и той же тематической области, и др.

В целом попытки создать универсальную онтологию верхнего уровня пока не привели к ожидаемым результатам. Многие онтологии верхнего уровня содержат одни и те же понятия, однако их трактовка и принципы организации иерархии отличаются в разных онтологиях. Так, например, во всех онтологиях проводится разделение сущностей на абстрактные и реально существующие, на постоянные и временные сущности, деление на объект и процесс. В то же время даже на верхних уровнях наблюдаются существенные различия. В онтологии SUMO первично разделение на абстрактные и материальные сущности, а разделение на постоянные и временные – вторично. В DOLCE на верхнем уровне производится разделение на постоянные, временные, абстрактные и качественные сущности. В онтологии Джона Совы иерархии сущностей в явном виде нет: в ней описаны только категории, по которым понятия разделяются или группируются. В онтологии OpenCyc на верхнем уровне коллекция <<Нечто>> делится на <<Неосязаемые>> и <<Индивиды>>, но экземпляры и тех и других могут быть как абстрактными, так и материальными объектами. 

Возникновение нескольких онтологий верхнего уровня и конкуренция в этой области означает, что
создание онтологий верхнего уровня представляет собой трудоемкую задачу, для решения которой требуются модульная организация таких онтологий, понятное представление, специальные средства, поддерживающие процесс согласования онтологий. Очевидно, что разработка онтологий верхнего уровня -- это постоянный процесс, для которого необходимы инструменты, поддерживающие этот процесс и коллективную работу над онтологиями.

Онтологии верхнего уровня были призваны решить задачу обеспечения семантической совместимости представляемых знаний в базах знаний, однако отсутствие единой формальной основы, обеспечивающей однозначную интерпретацию представляемых знаний и вводимых новых понятий, не привело к решению указанной проблемы. Отсутствие удовлетворительного решения этой задачи приводит к несовместимости компонентов баз знаний, разрабатываемых для разных систем, и невозможности их повторного использования в других системах. Как следствие, имеет место многократная повторная разработка содержательно одних и тех же компонентов для разных баз знаний.

\newpage
\section{Анализ методов и средств создания баз знаний}

Так как в основе любой базы знаний лежат онтологии, то методы и средства разработки онтологий являются важнейшей частью технологий разработки баз знаний.

\subsection{Методологии разработки баз знаний}

В современной литературе при анализе методов разработки баз знаний основное внимание уделяется анализу методологий разработки онтологий, которые являются основной частью современных баз знаний.

Методология разработки онтологий представляет собой набор инструкций и руководств, описывающих процесс выполнения сложных процедур разработки онтологий. Она детализирует различные задачи, как они должны быть выполнены, в каком порядке и каким образом осуществлять документирование работы по созданию онтологий. 

Разработка приложения с использованием какой-либо методологии может потребовать больше времени, однако преимущества ее использования превышают время, необходимое для разработки. Методология имеет важное значение для качества, возможности повторного использования и поддержки приложения ~\cite{Sainter2000}.

Существует множество работ, посвященных обзору различных подходов и методологий проектирования онтологий~\cite{Iqbal2013, Sainter2000}.

В работе~\cite{Sloboduk2013} предложен вариант классификации существующих методологий разработки онтологий, основанный на наиболее существенных признаках. В соответствии с данной классификацией все существующие методологии можно условно разделить на представленные ниже группы. 

\begin{enumerate}
  \item {По поддержке коллективной разработки:
    \begin{itemize}
        \item методологии, поддерживающие совместную коллективную разработку онтологии; 
        \item методологии, не поддерживающие совместную коллективную разработку онтологии. 
    \end{itemize}
  }
  
  \item {По степени зависимости от инструментария: 
    \begin{itemize}
        \item зависимые;
        \item полузависимые;
        \item независимые.
    \end{itemize}
  }
  
  \item {По типу используемой модели жизненного цикла онтологии:
    \begin{itemize}
        \item без указания модели жизненного цикла онтологии;
        \item с итеративной моделью жизненного цикла онтологии;
        \item с моделью жизненного цикла онтологии на основе эволюционного прототипирования;
        \item с совпадающей с моделью жизненного цикла приложения.
    \end{itemize}
  }
  
  \item {По возможности формализации:
    \begin{itemize}
        \item предусматривающие способы формализации;
        \item не предусматривающие формализации. 
    \end{itemize}
  }
  
  \item {По возможности повторного использования разрабатываемых онтологий:
    \begin{itemize}
        \item поддерживающие повторное использование;
        \item не поддерживающие повторного использования.
    \end{itemize}
  }
  
  \item {По стратегии выделения концептов предметной области:
    \begin{itemize}
        \item снизу вверх (bottom-up);
        \item сверху вниз (top-down); 
        \item от середины (middle-out); 
        \item сочетающие различные стратегии.
    \end{itemize}
  }
  
  \item {По возможности поддержки совместимости разрабатываемых онтологий:
    \begin{itemize}
        \item поддерживающие совместимость;
        \item не поддерживающие совместимость.
    \end{itemize}
  }
\end{enumerate}

Среди основных методологий, наиболее сформировавшихся на сегодняшний день и ставшими базовыми в области создания онтологий предметных областей, можно выделить следующие: скелетную методологию Ушолда и Кинга~\cite{Uschold1996}, методологию Грюнингера и Фокса (TOVE)~\cite{Gruninger1995}, METHONTOLOGY~\cite{Gomez1996, Gomez2009}, On-To-Knowledge (OTK)~\cite{OntoEdit2002}, KACTUS~\cite{Bernaras1976}, DILIGENT~\cite{Pinto2004}, SENSUS~\cite{Swartout1997} и UPON~\cite{Nicola1997}. Их сравнительная характеристика в соответствии с приведенной классификацией приведена в \mbox{таблице \ref{table_prev_pic_1.5}.}

\setlength\LTleft{0pt}
\setlength\LTright{0pt}
\setlength\LTcapwidth{16.7cm}
\begin{longtable}[H]{@{\extracolsep{\fill}}|>{\footnotesize}c|>{\footnotesize}c|>{\footnotesize}c|>{\footnotesize}c|>{\footnotesize}c|>{\footnotesize}c|>{\footnotesize}c|}
% \begin{longtable}{|M{3cm}|>{\tiny}M{3cm}|>{\tiny}M{3cm}|>{\tiny}M{3cm}|>{\tiny}M{3cm}|>{\tiny}M{3cm}|>{\tiny}M{3cm}|}
% \centering
% \resizebox{\textwidth}{!}{

\caption{Сравнительная характеристика методологий построения \\онтологий}
\label{table_prev_pic_1.5}
\vspace{-5mm}
\\ \hline
\begin{tabular}[c]{@{}c@{}}Методоло-\\гия\end{tabular} & \begin{tabular}[c]{@{}c@{}}Совместное\\ конструи-\\рование\end{tabular} & 
\begin{tabular}[c]{@{}c@{}}Степень \\зависи-\\мости от \\прило-\\жения\end{tabular} & \begin{tabular}[c]{@{}c@{}}Используемая\\модель жизнен-\\ного цикла\\онтологии\end{tabular} & \begin{tabular}[c]{@{}c@{}}Возмож- \\ность \\ форма- \\лизации\end{tabular} & \begin{tabular}[c]{@{}c@{}}Поддержка \\повторного \\использова-\\ния\end{tabular} & \begin{tabular}[c]{@{}c@{}}Поддержка \\совмести-\\мости\end{tabular}  \\ \hline

\endfirsthead
\multicolumn{7}{@{\hskip0pt}l}%
{Продолжение таблицы \thetable{}} \\ \hline
\endhead

\endfoot
\hline  \endlastfoot

\begin{tabular}[c]{@{}l@{}} 1 \end{tabular} & 
\begin{tabular}[c]{@{}l@{}} 2 \end{tabular} & 
\begin{tabular}[c]{@{}l@{}} 3 \end{tabular}& 
\begin{tabular}[c]{@{}l@{}} 4 \end{tabular}& 
\begin{tabular}[c]{@{}l@{}} 5 \end{tabular}&
\begin{tabular}[c]{@{}l@{}} 6 \end{tabular}& 
\begin{tabular}[c]{@{}l@{}} 7 \end{tabular} \\ \hline

\begin{tabular}[l]{@{}l@{}}Uschold and \\ King\end{tabular} & 
\begin{tabular}[l]{@{}l@{}} - \end{tabular} & 
\begin{tabular}[l]{@{}l@{}}Незави-\\симая\end{tabular}& 
\begin{tabular}[l]{@{}l@{}}Не упоми-\\нается\end{tabular}& 
\begin{tabular}[l]{@{}l@{}}Не упоми-\\нается\end{tabular}&
\begin{tabular}[l]{@{}l@{}} - \end{tabular}& 
\begin{tabular}[l]{@{}l@{}} - \end{tabular} \\ \hline

\begin{tabular}[l]{@{}l@{}}TOVE\end{tabular} & 
\begin{tabular}[l]{@{}l@{}} - \end{tabular} & 
\begin{tabular}[l]{@{}l@{}}Полуза-\\висимая\end{tabular}& 
\begin{tabular}[l]{@{}l@{}}Не упоми-\\нается\end{tabular}& \begin{tabular}[l]{@{}l@{}}Логика\end{tabular}&
\begin{tabular}[l]{@{}l@{}} - \end{tabular}& 
\begin{tabular}[l]{@{}l@{}} - \end{tabular} \\

\pagebreak

\begin{tabular}[c]{@{}l@{}} 1 \end{tabular} & 
\begin{tabular}[c]{@{}l@{}} 2 \end{tabular} & 
\begin{tabular}[c]{@{}l@{}} 3 \end{tabular}& 
\begin{tabular}[c]{@{}l@{}} 4 \end{tabular}& 
\begin{tabular}[c]{@{}l@{}} 5 \end{tabular}&
\begin{tabular}[c]{@{}l@{}} 6 \end{tabular}& 
\begin{tabular}[c]{@{}l@{}} 7 \end{tabular} \\ \hline

\begin{tabular}[l]{@{}l@{}}SENSUS\end{tabular} & 
\begin{tabular}[l]{@{}l@{}} + \end{tabular} & \begin{tabular}[l]{@{}l@{}}Полуза-\\висимая\end{tabular}& 
\begin{tabular}[l]{@{}l@{}}Не упоми-\\нается\end{tabular}& 
\begin{tabular}[l]{@{}l@{}}Семанти-\\ческие сети\end{tabular}&
\begin{tabular}[l]{@{}l@{}} + \end{tabular}& 
\begin{tabular}[l]{@{}l@{}} + \end{tabular} \\ \hline

\begin{tabular}[l]{@{}l@{}}METHONTO-\\LOGY\end{tabular} & 
\begin{tabular}[l]{@{}l@{}} - \end{tabular} & 
\begin{tabular}[l]{@{}l@{}}Незави-\\симая\end{tabular}& \begin{tabular}[l]{@{}l@{}}Эволюционное\\прототипиро-\\вание\end{tabular}& 
\begin{tabular}[l]{@{}l@{}}Фреймы и\\дескрипци-\\онная\\логика\end{tabular}&
\begin{tabular}[l]{@{}l@{}} - \end{tabular}& 
\begin{tabular}[l]{@{}l@{}} - \end{tabular} \\ \hline

\begin{tabular}[l]{@{}l@{}}KACTUS\end{tabular} & 
\begin{tabular}[l]{@{}l@{}} - \end{tabular} & 
\begin{tabular}[l]{@{}l@{}}Зави-\\симая\end{tabular}& 
\begin{tabular}[l]{@{}l@{}}Совпадает с \\моделью жиз-\\ненного цикла\\приложения\end{tabular}& \begin{tabular}[l]{@{}l@{}}Не упоми-\\нается\end{tabular}&
\begin{tabular}[l]{@{}l@{}} - \end{tabular}& 
\begin{tabular}[l]{@{}l@{}} - \end{tabular} \\ \hline

\begin{tabular}[l]{@{}l@{}}OTK\end{tabular} & 
\begin{tabular}[l]{@{}l@{}} - \end{tabular} & 
\begin{tabular}[l]{@{}l@{}}Полуза-\\висимая\end{tabular}& 
\begin{tabular}[l]{@{}l@{}}Итерационная\end{tabular}& 
\begin{tabular}[l]{@{}l@{}}Не упоми-\\нается\end{tabular}&
\begin{tabular}[l]{@{}l@{}} - \end{tabular}& 
\begin{tabular}[l]{@{}l@{}} - \end{tabular} \\ \hline

\begin{tabular}[l]{@{}l@{}}UPON\end{tabular} & 
\begin{tabular}[l]{@{}l@{}} - \end{tabular} & 
\begin{tabular}[l]{@{}l@{}}Полуза-\\висимая\end{tabular}& 
\begin{tabular}[l]{@{}l@{}}Итерационная\end{tabular}& 
\begin{tabular}[l]{@{}l@{}}Семанти-\\ческие\\ сети\end{tabular}&
\begin{tabular}[l]{@{}l@{}} - \end{tabular}& 
\begin{tabular}[l]{@{}l@{}} - \end{tabular} \\ \hline

\begin{tabular}[l]{@{}l@{}}DILIGENT\end{tabular} & 
\begin{tabular}[l]{@{}l@{}} + \end{tabular} & 
\begin{tabular}[l]{@{}l@{}}Зави-\\симая\end{tabular}& \begin{tabular}[l]{@{}l@{}}Эволюционное\\прототипи-\\рование\end{tabular}& 
\begin{tabular}[l]{@{}l@{}}Не упоми-\\нается\end{tabular}&
\begin{tabular}[l]{@{}l@{}} - \end{tabular}& 
\begin{tabular}[l]{@{}l@{}} - \end{tabular} \\ \hline

\end{longtable}


Анализ рассмотренных методологий показывает, что ни одна из них не является полной, а все предлагаемые решения не унифицированы. Большинство методологий не поддерживают совместную разработку баз знаний, поддержку совместимости разрабатываемых баз знаний и, как следствие, поддержку повторного использования уже разработанных баз знаний и их компонентов. 

Кроме того, подавляющее большинство методологий разработки баз знаний описывают процесс разработки в общих чертах, не регламентируя действия участников на каждом этапе разработки онтологии, не уточняя принципы согласования новых понятий с уже существующими, высоким оказывается субъективное влияние разработчиков. Таким образом, проблема совместимости компонентов баз знаний остается актуальной при использовании даже наиболее развитых методологий.

\subsection{Инструментальные средства разработки баз знаний}

Рассмотрим подробнее некоторые наиболее распространенные на сегодняшний день средства разработки баз знаний.

\textbf{Wiki-Технология}

Wiki-Технология позволяет накапливать знания, которые представляются в интероперабельной форме, обеспечивая навигацию по знаниям. Использовать Wiki-Технологию возможно для проектов любого масштаба и тематической направленности (от открытых электронных энциклопедий, до справочных систем различных предприятий и учебных заведений)~\cite{Rogushina2016, Raman2006}. 

Wiki-Технология предоставляет своим пользователям средства хранения, структуризации текста, гипертекста, файлов и мультимедиа. Wiki-Технология использует в качестве инструмента платформу MediaWiki~\cite{MediaWiki2016}, которая позволяет осуществлять информационное взаимодействие, обеспечивать доступ к информационным ресурсам всем участникам процесса разработки системы, организовывать управление и наблюдение за разработкой~\cite{Gladun2008}. Среди достоинств данной технологии можно выделить простоту Wiki-разметки, коммуникативные возможности, которые реализуются через совместное редактирование страниц, а также посредством электронных обсуждений в Wiki или дополнительных средах, таких как чат или форум, проектный характер работы, сотрудничество, формирование единого продукта совместной деятельности обеспечивают содержательное взаимодействие, обмен знаниями, оценку и постоянное совершенствование работ~\cite{Rogushina2016}.

Влияние Semantic Web на подобные проекты постоянно возрастает, вследствие чего были созданы движки Wiki-сайтов, которые поддерживают онтологическое представление знаний и семантическую разметку ресурсов при помощи средств Semantic MediaWiki~\cite{SemMediaWiki2016}. Данные средства позволяют включать семантические аннотации в Wiki-разметку в виде OWL и RDF и явно разделять структурированную и неструктурированную информацию~\cite{Rogushina2016}.

Кроме указанных достоинств Wiki как технология имеет ряд недостатков:  дублирование информации на различных страницах, невозможность структурирования знаний ввиду отсутствия иерархии гиперссылок и отсутствия унификации представления информации, отсутствие возможности автоматической верификации. Кроме того, Wiki-Технология в настоящее время рассчитана на работу только со структурированными естественно-языковыми текстами, таким образом, на основе такой технологии не представляется возможным строить базы знаний интеллектуальных систем, поскольку неформальный текст непригоден для автоматической обработки в той степени, которая необходима для решения различных задач, например, задач логического вывода.

В данной диссертационной работе некоторые идеи Wiki-Технологии получили свое развитие и предложены подходы к устранению указанных проблем.

\textbf{Программные среды для построения онтологий}

Существующие программные средства построения баз знаний (онтологий) условно делятся на три группы~\cite{Chistiakova2014}:

1. \textit{Инструменты создания онтологий}. Данный класс средств обеспечивает процесс создания базы знаний <<с нуля>>. Помимо редактирования и просмотра средства обеспечивают поддержку документирования онтологий, импорт/экспорт онтологий в различные форматы и языки, управление библиотеками онтологий.

К ним относятся: Protégé~\cite{Protege2016}, NeON~\cite{Gomez2009}, Co4~\cite{Euzenat1996}, Ontolingua~\cite{Ontolingua2005}, OntoEdit~\cite{OntoEdit2002}, OilEd~\cite{OilEd2013}, WebOnto~\cite{Domigue2016} и т.д. Краткую характеристику этих и других инструментов можно найти в работах~\cite{BorgestRole2014, Ovdei2016, Alatrish2013}.

2. \textit{Инструменты для отображения, выравнивания и объединения онтологий}. Данный класс инструментов помогает пользователям найти сходство и различие между исходными онтологиями и создают результирующую онтологию, которая содержит элементы исходных онтологий. Автор~\cite{Ovdei2016} делит их на подгруппы по следующим признакам: 
\begin{itemize}
    \item для объединения двух онтологий с целью создания одной новой (PROMPT~\cite{PROMPT2016}, Chimaera~\cite{Chimaera2003}, OntoMerge~\cite{OntoMerge78});
    \item для определения функции преобразования из одной онтологии в другую (OntoMorph~\cite{Chalupsky2000}); 
    \item для определения отображения между концептами в двух онтологиях, находя пары соответствующих концептов (например, OBSERVER~\cite{OntologySystemEnhanced2003}, FCA-Merge ~\cite{FCA-Merge});
    \item для определения правил отображения для связи только релевантных частей исходных онтологий (ONION~\cite{ONION2006}).
\end{itemize}

3. \textit{Инструменты аннотирования на основе онтологий}. Важнейшим условием реализации целей семантического Web является возможность аннотировать Web-ресурсы метаинформацией. По этой причине в последнее время инструменты инженерии онтологий включают в свой состав инструменты аннотирования на основе онтологий. К ним относятся: MnM~\cite{Varges-Vera2002}, SHOE Knowledge Annotator~\cite{HEflin2009} и т. д. 

В контексте решения поставленных в рамках данной диссертационной работы задач имеет смысл рассмотреть подробно только первый класс инструментов. 

В онтологическом инжиниринге любая онтология рассматривается как результат согласованной деятельности группы специалистов о модели некоторой области знаний. Исходя из этого с развитием методов и средств в области инженерии знаний все большее внимание стало уделяться инструментальной поддержке процесса коллективной разработки онтологий, в рамках которого существует несколько основных задач~\cite{Efimenko2011,TuzovskiSystem2011}:
\begin{itemize}
    \item управления взаимодействием и коммуникацией между разработчиками;
    \item контроль за доступом к текущим результатам совместного проектирования; 
    \item фиксация авторских прав на экспертные знания, переданные в общее пользование;
    \item обнаружение ошибок проектирования и управление коррекцией ошибок;
    \item конкурентное управление изменениями.
\end{itemize}

В настоящее время для решения данных проблем уже существует несколько достаточно хорошо проработанных подходов и соответствующих инструментальных средств. Среди них можно выделить следующие:
\begin{itemize}
    \item Сollaborative Protege~\cite{Tudorache2008};
    \item проект NeOn~\cite{Gomez2009}; 
    \item инфраструктура совместной разработки согласованных баз знаний Co4~\cite{Euzenat1996}.
\end{itemize}

Рассмотрим данные средства подробнее.

\textbf{Проект Protege}

Protege~\cite{Efimenko2011, Protege2016, Tudorache2008} является наиболее популярным редактором онтологий в настоящее время. Данный редактор является свободно распространяемым с открытым кодом, который одновременно является фреймворком для построения баз знаний. Protégé позволяет экспортировать онтологии во множество форматов, включая RDF (RDF Schema), OWL, и др.

Архитектура Protege позволяет расширять его возможности за счет поддержки модулей расширения функциональности, обеспечивая возможность настройки редактора под нужды различных приложений. Популярность Protege объясняется, кроме прочего, тем, что в его разработке участвует целое сообщество, включающее разработчиков и ученых, правительственных и корпоративных пользователей.

Collaborative Protege является расширением существующей системы Protege, которая поддерживает совместную разработку онтологий. В дополнение к обычным операциям редактирования онтологии оно позволяет аннотировать как компоненты онтологии, так и изменения онтологии при помощи использования онтологии ChAO (Changes \& Annotation) для управления процессами совместного проектирования онтологии, фрагмент которой представлен на рисунке \ref{pic_1.6}. При этом система Protеgе сама становится системой, управляемой онтологией проектирования.

Collaborative Protege поддерживает поиск и фильтрацию пользовательских аннотаций, основанных на различных критериях. В Collaborative Protégé можно работать как в автономном, так и многопользовательском режиме, в котором несколько пользователей могут одновременно редактировать одну и ту же онтологию. Все изменения и аннотации, сделанные одним пользователем, сразу же видят другие пользователи.

\begin{figure}[H]
\begin{center}
\includegraphics[width=1.0\textwidth]{man-source/images/pic_1_6.png}\\[2mm]
\caption{Фрагмент онтологии ChAO кооперативной версии Protege}\label{pic_1.6}
\end{center}
\end{figure}

Основные поддерживаемые возможности для коллективной разработки онтологий Collaborative Protege:

\begin{itemize}
    \item \textit{аннотирование компонентов онтологии} (таких как классы, свойства, отдельные лица). Один пользователь может добавить комментарий к классу <<Person>>, указав, что его следует переименовать в <<Human>>;
    \item \textit{аннотация изменений} (создание классов, переименование и т. д.). Пользователи могут комментировать состояние переименования класса, если они не согласны с этой операцией;
    \item \textit{обсуждения}. Пользователь может ответить на комментарий другого пользователя об определенном компоненте онтологии. Таким образом, создаются потоки обсуждения, связанные с определенным компонентом онтологии. Также можно запустить более общие потоки обсуждения, не связанные с конкретным компонентом онтологии. Например, общая дискуссионная нить может иметь в качестве субъекта определенные дизайнерские решения, которые должны быть реализованы во всей онтологии;
    \item \textit{предложения и голосование}. Пользователь может начать предложение об изменении онтологии. Другие пользователи могут голосовать, согласны они или не согласны с этим предложением. В настоящее время поддерживаются два типа голосования: 5-звездочное голосование и согласие/несогласие;
    \item \textit{поиск и фильтрация аннотаций на основе различных критериев}. Пользователь может выполнять поиск аннотаций по имени автора, дате, когда были сделаны аннотации, типу аннотации (вопрос, комментарий и т. д.) и ключевым словам в теле аннотации. Эти критерии поиска могут использоваться по отдельности или путем комбинирования их в соединении или дизъюнкции;
    \item \textit{живое обсуждение (чат)}. Пользователи, подключенные одновременно к одному и тому же серверу Protege, могут общаться друг с другом. Сообщения чата транслируются всем пользователям.
\end{itemize}

Основными недостатками Collaborative Protégé являются отсутствие возможности явного задания роли разработчика в процессе создания базы знаний и отсутствие конкурентного управления изменениями в процессе разработки.

\textbf{Проект NeOn}

Европейский проект NeOn ~\cite{Gomez2009} ориентирован на достижение тех же целей, что и Сollaborative Protege.

Особенностями платформы NeOn являются:
\begin{itemize}
    \item поддержка <<жизненного цикла>>, включая взаимодействие активностей периода разработки и исполнения;
    \item ориентация на онтологический инжиниринг и использование онтологий;
    \item расширяемость архитектуры на всех уровнях.
\end{itemize}

Инструментарий NeOn поддерживает онтологический инжиниринг и управление, полный «жизненный цикл» и сетевую работу с онтологиями (модульность, отображение и т. д.). Разработан этот инструментарий на платформе Eclipse и расширяет базовую архитектуру за счет механизма плагинов Eclipse и веб-сервисов.

В отличие от Сollaborative Protege в NeON есть разделение пользователей по ролям. Права доступа пользователей назначаются системными администраторами. (Administrators) Авторизованному пользователю в различных модулях онтологии может быть назначена роль следующих типов:
\begin{itemize}
    \item инженер онтологии (Ontology engineer);
    \item предметный эксперт (Subject expert);
    \item валидатор (Validator);
    \item зритель (Viewer).
\end{itemize}

Роли назначаются в зависимости от типа прав, которыми пользователи будут обладать, и типа задач, которые они будут выполнять.

\textit{Редакторы онтологии} (Ontology editors) являются экспертами в предметной области, хотя они также могут быть специалистами по управлению информацией, терминологистами или переводчиками. Они отвечают за повседневную работу по редактированию и поддержке многоязычных онтологий, и они могут отвечать за разработку конкретных фрагментов онтологий, пересмотр работы других и разработку многоязычных версий онтологий.

\textit{Инженеры онтологии} специализируются на методах и проблемах моделирования онтологий. Они обладают различными уровнями знаний, от базовых до продвинутых, в области инженерных инструментов онтологии, но могут мало знать о моделируемой области. Как правило, они отвечают за определение исходного скелета онтологии и при этом учитывают её цель, возможные взаимодействия с унаследованными системами и другие соответствующие вопросы.

\textit{Предметные эксперты} являются редакторами онтологий, вставляющими или изменяющими содержимое онтологий.

\textit{Валидаторы} пересматривают, одобряют или отклоняют изменения, внесенные предметными экспертами, и они являются единственными, кто может копировать изменения в производственную среду для внешнего пользования.

\textit{Зрителям} разрешено входить в систему и консультироваться с одобренной информацией об онтологиях, но они не могут их редактировать.

Рабочий процесс в NeOn основан на присвоении статуса каждому элементу онтологии. Только если все элементы имеют статус <<Утверждено>> (Approved), онтология может быть опубликована или обновлена.

Возможные статусы для каждого элемента: 

\textit{Черновик (Draft)} – присваивается любому элементу, когда он впервые попадает в рабочий процесс редактирования, либо элементу, который был утвержден, а затем обновлен предметным экспертом.

\textit{Подлежит утверждению (To be approved)} – если предметный эксперт уверен в изменении элемента со статусом <<Черновик>> и хочет, чтобы он был проверен, элемент переводится в статус <<Подлежит утверждению>> и остается с ним до тех пор, пока валидатор не проверит его.

\textit{Утверждено (Approved)} – если валидатор принимает изменение элемента, которое находилось в статусе <<Подлежит утверждению>>, то оно переходит к утвержденному состоянию.

\textit{Опубликовано (Published)} – это статус всех элементов в онтологии, опубликованных в Интернете.

\textit{Будет удалено (To be deleted)} – если предметный эксперт считает, что элемент необходимо удалить, элемент будет помечен статусом <<Будет удалено>> и удален из онтологии, хотя окончательно удалить его сможет только валидатор.

В зависимости от роли пользователи по-разному влияют на рабочий процесс ~\cite{Gomez2009}.

Наиболее важные варианты использования проекта NeOn:

\textit{Поиск}. При редактировании онтологии редактор онтологии может выполнять поиск по всем редактируемым онтологиям.
\textit{Запрос ответа}. При редактировании онтологии редактор онтологии может выполнять запросы в редактируемых онтологиях. Запросы могут использовать и стандартный язык запросов (например, SPARQL), запрос на естественном языке или предопределенный запрос из шаблона.
\textit{Управление многоязычностью}. Редактор онтологии занимается добавлением языков к онтологии, выполнением проверки орфографии, управлением многоязычными метками, выбором рабочего языка, а также преодолением специфических особенностей перевода.
\textit{Экспорт} онтологии в другие форматы.
\textit{Преобразование} онтологий из других форматов.
\textit{Управление сопоставлениями}. Создание выравниваний между онтологиями вручную и полуавтоматическим способом. Создаются сопоставления между понятиями или модулями в различных онтологиях.
\textit{Визуализация}. Визуализация онтологий и их фрагментов определяется в зависимости от выполняемой задачи. Визуализируются отображения и отношения между концепциями и модулями в сетевых онтологиях.
\textit{Модульность}. Работа с модулями онтологии; создание модулей вручную, полуавтоматически и объединение модулей.
\textit{Управление статистикой}. Система фиксирует изменения онтологии. Пользователи могут видеть историю изменений, просматривать статистику использования и статистику онтологии (глубина, число дочерних узлов, число отношений и свойств, количество понятий на «ветке»).
\textit{Оценка и проверка онтологии}. Редактор онтологии может проверить качество разработки онтологии, проверить наличие дубликатов в онтологии, провести сравнение с другими онтологиями и оценить структурные свойства онтологии.
\textit{Документирование}. Автоматическое создание соответствующих метаданных, генерация UML-диаграмм и документации, касающейся используемых отношений и свойств.

К недостаткам проекта NeOn можно отнести средства взаимодействия пользователей, в отличие от Сollaborative Protege, в котором присутствуют и аннотации изменений, а также потоки обсуждения и обыкновенные чаты. Проект NeOn таких возможностей не предоставляет. 

\textbf{Совместная разработка согласованных баз знаний Co4}

Co4 – инфраструктура, обеспечивающая совместное конструирование базы знаний в интернет-среде. При этом разработка баз знаний трактуется как социальный процесс, в который вовлечено сообщество множества агентов, а система Co4 поддерживает разработку с помощью экспертов, которые являются равноправными участниками коллективной разработки ~\cite{Euzenat1996}. Одним из основных требований, предъявляемых к данной системе, является необходимость консенсуса. Какое-либо изменение базы знаний принимается только после согласования со всеми участниками процесса разработки. Для выполнения данного правила каждый из участников проекта может модифицировать только свое личное рабочее пространство, но не всю базу знаний.

В рамках Co4 каждый участник рассматривается системой как база знаний. Для построения согласованной базы знаний индивидуальные базы знаний организуются в дерево, чьи листья являются пользовательскими базами знаний, а промежуточные  -- групповыми базами знаний. При этом соблюдается условие, что каждая групповая база знаний представляет знания, согласованные между ее сыновьями. Когда пользователи достаточно уверены в своих индивидуальных базах знаний, то они предоставляют их своей групповой базе знаний. Это предложение приходит другим пользователям как запрос на комментарий, на который пользователи должны дать один из ответов:

\begin{itemize}
    \item \textit{принять}, когда пользователь считает, что предложенные знания должны быть интегрированы в согласованную базу знаний;
    \item \textit{отвергнуть}, когда пользователь считает, что предложенные знания не должны быть добавлены в согласованную базу знаний;
    \item \textit{запрос} (challenge), когда пользователь предлагает другое изменение.
\end{itemize}

Целью Co4 является создание баз знаний, которые отвечают требованиям:

\begin{itemize}
    \item \textit{согласованные} – любое добавление в базу знаний происходит после принятия всеми пользователями, участвующими в разработке;
    \item \textit{совместно построенные} – сотрудничество пользователей в целях создания базы знаний;
    \item \textit{последовательные} – поскольку они хранятся в официальном хранилище знаний и проверяются на согласованность.
\end{itemize}

Отсутствие ролей пользователей в Co4 является серьезной проблемой, в частности, для задач с неоднозначным решением. Кроме того, описание процессов взаимодействия пользователей при разработке не является частью базы знаний, как например в Сollaborative Protege.

К основным недостаткам рассмотренных инструментов можно отнести:

\begin{itemize}
    \item отсутствие развитых средств автоматического редактирования и верификации баз знаний, в том числе оценки полноты и избыточности;
    \item отсутствие единого механизма коллективного создания баз знаний, включающего в себя средства согласования вносимых изменений между разработчиками разного уровня ответственности, типологию ролей разработчиков;
    \item недостаточный уровень расширяемости инструментов разработки.
\end{itemize}

\subsection[Анализ средств компонентной разработки баз \\знаний]{Анализ средств компонентной разработки баз знаний}

Повторное использование готовых компонентов широко применяется во многих отраслях, связанных с проектирование различного рода систем, поскольку позволяет уменьшить трудоемкость разработки и ее стоимость (путем минимизации количества труда за счет отсутствия необходимости разрабатывать какой-либо компонент), повысить качество создаваемого контента. Использование готовых компонентов предполагает, что распространяемый компонент верифицирован и документирован, а возможные ошибки и ограничения устранены либо специфицированы и известны.

Вопросы компонентного проектирования интеллектуальны систем и, в частности, баз знаний, рассматриваются в работах~\cite{Gribova2015, Borisov2014, Globa2012, Gribova2011, Zarogulko2016}. При создании первых систем, основанных на знаниях, предполагалось, что именно эти системы будут идеально решать задачу повторно используемых компонентов, однако разработчики столкнулись с рядом проблем, актуальных до настоящего времени~\cite{Bolotova2012}. 

Многие исследователи и разработчики определяют наличие онтологических библиотек как важный компонент инфраструктуры Semantic Web \cite{Semantics}. Первые библиотеки и коллекции онтологий были разработаны в рамках таких проектов, как Ontolingua server~\cite{Ontolingua2005}, DAML~\cite{OilEd2013}, Protégé ontology library~\cite{Protege2016}, Ontaria ontology directory~\cite{Ontaria} и SchemaWeb~\cite{SchemaWeb}. Многие из этих проектов в настоящее время не поддерживаются, однако на смену им приходит новое поколение библиотек онтологий~\cite{Aquina2012}. 

Однако, как показано в работах~\cite{Debruyne2009, Leenheer2009}, большинство онтологий, разработанных на основе стандартов Semantic Web, не согласуются между собой и, соответственно, не могут быть повторно использованы в качестве компонентов баз знаний, как это предполагалось разработчиками стандартов Semantic Web.

К числу проблем в области компонентного проектирования баз знаний можно отнести следующие:

\begin{itemize}
    \item многие компоненты используют для идентификации язык разработчика (как правило, английский), и предполагается, что все пользователи будут использовать этот же язык. Однако для многих приложений это \mbox{недопустимо –} понятные только разработчику идентификаторы должны быть скрыты от конечных пользователей, которые должны быть в состоянии выбрать язык для идентификаторов, которые они видят~\cite{Bolotova2012};
    \item отсутствие унификации в принципах представления различных видов знаний в рамках одной базы знаний, и, как следствие, отсутствие унификации в принципах выделения и спецификации многократно используемых компонентов приводит к несовместимости компонентов, разработанных в рамках разных проектов~\cite{Gribova2016};
    \item отсутствие средств поиска компонентов, удовлетворяющих заданных критериям.
\end{itemize}

На сегодняшний день разработано большое число баз знаний по самым различным предметным областям ~\cite{Bergman2016}. Однако в большинстве случаев каждая база знаний разрабатывается отдельно и независимо от других, в отсутствие единой унифицированной формальной основы для представления знаний, а также единых принципов формирования систем понятий для описываемой предметной области. В связи с этим разработанные базы оказываются, как правило, несовместимы между собой и не пригодны для повторного использования.

Таким образом, актуальной остается проблема разработки общих унифицированных принципов выделения и спецификации многократно используемых компонентов баз знаний и формирование библиотеки таких совместимых компонентов. При этом могут быть выделены многократно используемые компоненты различного уровня сложности -- от спецификаций конкретных единичных понятий до многократно используемых онтологий и целых систем онтологий.

\scnendstruct

\end{SCn}