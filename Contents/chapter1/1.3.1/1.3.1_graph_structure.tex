\begin{SCn}

\bigskip
\scnfragmentcaption

\scnheader{графовая структура}

\scnidtfdef{абстрактная (математическая) структура, которая задается:
\begin{scnitemize}
	\item множеством ее элементов:
		\begin{scnitemizeii}
		 \item множеством ее вершин (узлов);
		 \item множеством ее связок:
		 \begin{scnitemizeiii}
		 \item множеством ее ребер (неориентированных пар 		элементов графовой структуры);
		 \item множеством ее дуг (ориентированных пар элементов графовой структуры);
		 \item множеством ее гиперребер, каждое из которых является конечным множеством элементов графовой структуры, имеющим мощность больше двух
		 \end{scnitemizeiii}
		 \end{scnitemizeii}
	\item бинарным ориентированным отношением инцидентности, связывающим каждую связку графовой структуры с каждым компонентом (элементом) этой связки.
\end{scnitemize}}

\scnheader{следует отличать*}
\scnhaselementset{\scnfileitem{\textit{графовую структуру} как абстрактный математический объект, в рамках которого не уточняется то, как выглядят (представляются, изображаются) элементы графовой структуры и связи их инцидентности}
;\scnfileitem{представление (изображение) \textit{графовой структуры} -- ее рисунок, ее представление в компьютерной памяти в виде матрицы инцидентности, матрицы смежности, списковой структуры}}


\scnheader{графовая структура}
\scnidtftext{часто используемый sc-идентификатор}{дискретная информационная конструкция}
\scnnote{Поскольку любая \textit{графовая структура} является дискретной математической моделью, которая может описывать любое множество \textit{сущностей}, связанных между собой заданным множеством \textit{связей}, все \textit{графовые структуры} с полным основанием можно считать дискретными \textit{информационными конструкциями}. Более того, любая дискретная \textit{информационная конструкция} (в том числе, и обычная цепочка символов) с формальной точки зрения является \textit{графовой структурой}. Тот факт, что теория графов рассматривает "синтаксические"{} свойства \textit{графовых структур} с точностью до их изоморфизма, не лишает \textit{графовые структуры} соответствующих "семантических"{} свойств.}
\scnexplanation{С семантической точки зрения графовая структура -- это нелинейная (в общем случае) знаковая конструкция, в состав которой могут входить знаки \uline{любых} сущностей. При этом указанные знаки \uline{синтаксически} разбиваются на два класса --
	\begin{scnitemize}
		\item на \textit{знаки} сущностей, которые не являются \uline{связями} между сущностями -- в теории графов такие знаки называются узлами (вершинами);
		\item на знаки \uline{связей} между \textit{сущностями} -- к таким \textit{знакам} относятся ребра неориентированных графов, гиперребра гиперграфов, дуги ориентированных графов.
	\end{scnitemize}	
Кроме того, на множестве знаков \textit{сущностей}, входящих в состав \textit{графовой структуры}, задаются \textit{отношения инцидентности}, которые связывают \textit{знаки} связей, входящих  в состав \textit{графовой структуры}, со знаками тех \textit{сущностей} которые являются компонентами указанных \textit{связей}.


Теория графов рассматривает только "синтаксические"{} аспекты \textit{графовых структур}.
Семантика \textit{графовой структуры} задается \textit{онтологией}, специфицирующей систему понятий, экземплярами которых являются элементы этой графовой структуры, т.е. \textit{знаки}, входящие в состав этой \textit{графовой структуры}.}


\scnheader{семантическая сеть}
\scnidtf{\textit{графовая структура}, являющаяся \uline{формальным уточнением} одного из видов \textit{смыслового представления информации}}
\scnsubset{графовая структура}
\scnsubset{смысловое представление информации}
	\scnaddlevel{1}
	\scnsubset{знаковая структура}
	\scnaddlevel{-1}
\scnidtf{графовая структура, \uline{вершины} (узлы) которой трактуются как знаки некоторых описываемых сущностей, а \uline{связки} (ребра, дуги, гиперребра, гипердуги) которой трактуются как знаки связей между описываемыми сущностями и/или знаками этих сущностей} 
\scnidtf{\uline{абстрактная} графовая и в общем случае нелинейная знаковая конструкция (знаковая структура), являющаяся вариантом \uline{смыслового} представления соответствующей информации}
\scnidtfexp{информационная конструкция, в которой \uline{явно} выделены знаки \uline{всех} описываемых сущностей, а также знаки связей, которые также считаются описываемыми сущностями и которые связывают либо сами описываемые сущности, либо описываемые сущности со знаками других описываемых сущностей, лиюо знаки описываемых сущностей}
\scnnote{Теоретико-графовая трактовка (уточнение) \textit{смыслового представления информации} является вполне естественной, т.к. любая описываемая сущность может иметь неограниченное количество связей с другими описываемыми сущностями, и очень часто анализ свойств какой-либо описываемой сущности предполагает анализ всех представленных (описанных) связей этой сущности с различными другими сущностями. Более того, для любых описываемых сущностей существует связывающая их связь (все в Мире взаимосвязано). Вопрос в том, какая это связь и нужно ли ее описывать. Далеко не все то, что можно описывать, целесообразно описывать.}
\scnrelfromvector{общие предпосылки}{
\scnfileitem{Информация в знаковой конструкции содержится не в самих знаках, а в конфигурации связей между знаками, обозначающими описываемые сущности}
;\scnfileitem{Конфигурация связей между описываемыми сущностями \uline{в общем случае} \uline{не} являются линейной} 
;\scnfileitem{Идеальным \textit{смысловым представлением информации} следует считать такую знаковую конструкцию, синтаксическая конфигурация связей между знаками которой \uline{изоморфна} конфигурации связей между описываемыми сущностями}}
\scnnote{Понятие семантической сети является основным понятием для \textit{Технологии OSTIS}. Ранее семантические сети рассматривались не как основа технологии разработки интеллектуальных компьютерных систем, а как наглядная иллюстрация представления знаний, не имеющая практической перспективы из-за сложности реализации, не обладающая универсализмом.


Для нас семантические сети -- это
	\begin{scnitemize}
	\item формальный подход к построению знаковых конструкций:
	\item формальный подход, позволяющий создавать целое \uline{семейство} языков и в том числе языков \uline{универсальных}:
	\item основа организации памяти нового типа -- структурно перестраиваемой (реконфигурируемой) памяти, обработка информации в которой сводится к реконфигурации связей между ее элементами.
	\end{scnitemize}}

\scnrelfromlist{достоинства}{\scnfileitem{\textbf{Семантическая сеть} наряду с системами правил является весьма распространенным способом представления знаний в интеллектуальных системах. Особое значение этот способ представления знаний приобретает в связи с развитием сети интернет. Кроме ряда особенностей, позволяющих применять семантические сети в тех случаях, когда системы правил не применимы, \textbf{семантические сети} обладают следующим важным свойством: они дают возможность \textbf{соединения в одном представлении синтаксиса и семантики} или \uline{синтаксического и семантического аспектов описаний} знаний предметной области. Происходит это благодаря тому, что в семантических сетях наряду с переменными для обозначения тех или иных объектов (элементов множеств, некоторых конструкций из них) присутствуют и сами эти элементы и конструкции; присутствуют и связи, сопоставляющие тем или иным переменным множества допустимых интерпретаций. Эти обстоятельства позволяют во многих случаях резко \textbf{уменьшить реальную вычислительную сложность решаемых задач}.
\newline
Помимо изобразительных возможностей, \textbf{семантические сети обладают более серьезными достоинствами}. То обстоятельство, что \textbf{вся информация об индивиде представлена в единственном месте} -- в одной вершине -- означает, что вся эта информация непосредственно доступна в этой вершине, что, в свою очередь, \textbf{сокращает время поиска}, в частности, при выполнении унификации и подставновки в задачах логического вывода.
Существует еще одна, более \textbf{тонкая особенность} расширенных семантических сетей -- они позволяют \textit{интегрировать в одном представлении \textbf{синтаксис и семантику}} (т.е. интерпретацию) клаузальных форм. Это позволяет в процессе вывода обеспечивать взаимодействие синтаксических и семантических, теоретико-модельных подходов, что, в свою очередь, также является фактором, зачастую делаютщим вывод практически более эффективным}\\
	\scnaddlevel{1}
	\scnrelto{цитата}{Осипов Г.С.-Метод ИИ-2015кн,с.43-54}
	\scnaddlevel{1}
	\scnrelto{часть}{Осипов Г.С.-Метод ИИ-2015кн}
	\scnaddlevel{-2}
;\scnfileitem{Все связи между \textit{знаками}, входящими в состав \textit{семантической сети} представляются с помощью специальных связующих элементов \textit{семантической сети} (дуг, ребер) и, следовательно, для описания указанных связей в \textit{семантической сети} нет необходимости использовать такие средства, как предлоги, союзы, падежи, склонения, спряжения, различные разделители и ограничители, что существенно упрощает обработку \textit{знаний}.}
;\scnfileitem{Соединение синтаксических и семантических аспектов в \textit{семантической сети} проявляется в том, что дуга или ребро, "синтаксически"{} соединяющая элементы \textit{семантической сети} описывает наличие соответствующей \textit{связи} между \textit{сущностями}, обозначаемыми указанными элементами \textit{семантической сети}.}}

\scnauthorcomment{Дооформить ссылки}

\end{SCn}