\begin{SCn}

\scnheader{формализация*}
\scniselementrole{ключевой знак}{Начало Предметной области и онтологии кибернетических систем}
\scnaddlevel{1}
\scnsourcecommentpar{Начало Раздела 1.1}
\scniselement{начало раздела Базы знаний IMS.ostis}
\scnaddlevel{-1}
\scniselement{бинарное ориентированное отношение}
\scnidtf{формализация информации*}
\scnidtf{пара, связывающая менее формализованное и более формализованное представление некоторой информации*}
\scnidtf{формализация информационной модели некоторой описываемой (моделируемой) системы взаимосвязанных сущностей*}
\scnidtf{Бинарное ориентированное отношение, каждая \textit{пара} которого, связывает два \textit{семантически эквивалентных} знания, второе из которых является более точным (более точно сформированным) знанием по сравнению с первым \textit{знанием}*.}
\scnexplanation{Повышение точности (строгости) формулировки знания -- минимизация (а в идеале -- исключение) \uline{неоднозначной} семантической интерпретации этой формулировки, т.е. несоответствия того, что хотел "сказать"{} автор формулировки, и того, как его поняли. Формализация знаний предполагает (1) точное (строгое) описание \textit{синтаксиса и денотационной семантики} того \textit{языка}, на котором формулируются \textit{знания} и (2) максимально возможное \uline{упрощение} синтаксических и семантических принципов, лежащих в основе указанного \textit{языка}. Очевидно, что \textit{естественные языки} указанным требованиям не удовлетворяют и, следовательно, не могут быть основой для точной формулировки \textit{научно-технических знаний} и, соответственно, для представления этих \textit{знаний} в \textit{памяти интеллектуальных компьютерных систем}. Очевидно также, что разработка \textit{\uline{универсального} языка} формального представления научно-технических знаний является \uline{основой} для глубокой конвергенции различных научно-технических дисциплин, для расширения областей применения современной математики и даже для появления новых разделов математики, которые, например, изучают общие свойства \textit{универсального смыслового пространства} и, в частности, свойство семантического расстояния(семантической близости) как между различными \textit{знаками}, так и между различными \textit{знаковыми конструкциями} (конфигурациями знаков).}
\scnaddlevel{1}
\scnnote{Слово "математика"{} означает "точное знание"{}.}
\scnaddlevel{1}
\scnrelto{цитата}{\textit{Арнольд В.И. Что TM--2012кн-c.4}}
\scnaddlevel{-2}
\scnauthorcomment{Дооформить}

\scnnote{Формализация информационной модели есть не что иное как "движение"{} в сторону семантического (смыслового) представления этой модель, т.е. переход к такому представлению этой модели, в котором мы избавляемся от всего, не имеющего отношения к сути моделируемой системы и касающегося только способа построения этой модели (т.е. её синтаксической структуры). }
\scnnote{Нет проблемы записать любое \textit{знание} в компьютерную \textit{память}. Для этого надо придумать соответствующий формат их кодирования. Но есть проблема представить это \textit{знание} так, чтобы с ним было легко работать, чтобы с использованием этого \textit{знания} можно было достаточно удобно (без лишних накладных расходов, обусловленных выбранным способом представления) решать самые различные информационные \textit{задачи} (задачи интеграции знаний, информационного поиска по базе знаний, верификации и оптимизации баз знаний, логического вывода, поиска способов решения задач, хранимых в базе знаний и т. д.).
Какими характеристиками должно обладать удобное представление знаний, удовлетворяющее указанным требованиям. Очевидно, что такое представление есть не что иное, как формальная (математическая) модель, семантически эквивалентная этим знаниям. Т.е. удобно представить знание -- это фактически построить соответствующую этому знанию \textit{математическую модель}.
Для интеллектуальных компьютерных систем важно не просто приобрести знания, но и представить их в такой форме, которая была бы удобна не только для человека (пользователя и разработчика), но и для различных компьютерных систем, т.е. не требовала бы переоформление (перезаписи) этих знаний для различных компьютерных систем. Очевидно, что такая форма записи (представления) знаний должна быть абсолютно не зависящий от различных компьютерных платформ.
Это и есть главная цель формализации знаний, обеспечивающей эффективную автоматизацию обработки этих знаний.}
\scnheader{формальное представление информации}\\
\scnsubset{информация}
\scnaddlevel{1}
\scnidtf{информационная конструкция}
\scnaddlevel{-1}
\scntext{вопрос}{Почему разработка и использование формальных моделей (математических моделей) представления \textit{информации} является важнейшим этапом развития любой научной и научно-технической дисциплины.
}\scnaddlevel{1}
\scnrelfromset{ответ}{\scnfileitem{Формализация любой \textit{предметной области} даёт возможность более конструктивно накапливать, интегрировать, понимать и систематизировать новые \textit{знания} об этой \textit{предметной области}};
\scnfileitem{Формализация \textit{предметной области} обеспечивает более строгую верификацию, обоснование (аргументацию, доказательство) и согласование различных точек зрения};
\scnfileitem{Формализация \textit{предметной области} создает условия для разработки строгих и легко воспроизводимых (реализуемых) \textit{методов} решения различных \textit{классов задач}}}
\scnaddlevel{1}
\scniselement{конъюнкция*}
\scnrelto{достоинства}{формальное представление информации}
\scnaddlevel{-2}
\scnidtf{формальное (формализованное) представление информационной конструкции}
\scnsubset{смысловое представления информации}
\scnnote{Высшим уровнем качества \textit{формального представления информации} является смысловое представление этой информации}
\scnidtf{формальная модель системы описываемых взаимосвязанных сущностей}
\scnidtf{математическая модель системы описываемых взаимосвязанных сущностей}
\scnidtf{формула}
\scnnote{Сам термин ``\textit{формальное представление информации}'' свидетельствует о том, что при таком представлении \textit{информации} сама \uline{форма} представляемой информационной конструкции (т.е. синтаксическая структура этой конструкции) имеет очевидную аналогию с описываемой конфигурацией связей между соответствующими соответствующими описываемыми \textit{сущностями}.
В предельном "идеальном"{} случае указанная аналогия между формой и смыслом информационной конструкции должна быть изоморфизмом.}
\scnnote{Формализация формализации рознь и, соответственно, степень приближения формы представления информации к "идеальному"{} смысловому представлению может быть различной. Разработка такого "идеального"{} \textit{языка смыслового представления информации} должна руководствоваться следующими основными критериями:
	\begin{scnitemize}		
		\item максимально возможное упрощения синтаксиса (как можно меньше синтаксических излишеств и синтаксического разнообразия).
		\item обеспечение \uline{универсальности} языка.
	\end{scnitemize}

Подчеркнем, что обеспечение универсальности \textit{языка смыслового представления информации} является весьма нетривиальной задачей, т.к. сложно одновременно достигнуть две противоречащие друг другу цели- обеспечить простоту синтаксиса языка и его неограниченную семантическую мощность. Косвенным подтверждением этого является большое количество созданных человечеством специализированных \textit{формальных языков}, \textit{языков смыслового представления информации} и даже \textit{языков семантических сетей}, что свидетельствует о востребованности \textit{смыслового представления информации}.}
\scnsubdividing{формальное представление информации, не являющееся смысловым
;смысловое представление информации, не являющееся семантической сетью
;нерафинированная семантическая сеть
\scnaddlevel{1}
\scnidtf{смысловое представления информации 2-го уровня}
\scnaddlevel{-1}
;рафинированная семантическая сеть
\scnaddlevel{1}
\scnidtf{смысловое представление информации 3-го уровня}
\scnaddlevel{-1}}

\end{SCn}