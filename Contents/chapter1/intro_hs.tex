\scsection{Предметная область и онтология кибернетических систем}
\label{intro_hs}

\begin{SCn}
	
\scnsectionheader{\currentname}

\scnstartsubstruct

\scsectionbeginningname{Начало Предметной области и онтологии кибернетических систем}

\scnstartsubstruct

\scnidtf{Иерархическая система свойств (характеристик) кибернетических систем, определяющих общий (интегральный) уровень их качества}
\scnidtf{Эволюционный подход к определению качества и, в частности, уровня интеллекта кибернетической системы}

\scntext{аннотация}{Рассмотрена иерархическая система свойств (в т.ч. способностей) кибернетических систем, определяющих их качество и позволяющих сформулировать требования, которым должна удовлетворять высокоинтеллектуальная система (идеальная интеллектуальная система).}

\scntext{предисловие}{Свойства (способности), которым должны удовлетворять \textit{интеллектуальные системы}, рассматриваются в целом ряде публикаций. Тем не менее, для \uline{практической} реализации \textit{компьютерных систем}, обладающих указанными свойствами (способностями), т.е. \textit{интеллектуальных компьютерных систем}, необходимо детализировать (уточнить) эти \textit{свойства}, пытаясь свести их к более конструктивным, прозрачным и понятным для реализации свойствам.}

\scnrelfromset{рассматриваемые вопросы}{
\scnfileitem{По каким свойствам (параметрам, характеристикам, способностям) кибернетических систем можно оценивать уровень их качества.};
\scnfileitem{Можно ли считать уровень развития какого-либо свойства (способности) кибернетической системы, т.е. значение какого-либо ее параметра (характеристики) оценкой уровня качества кибернетической системы по соответствующему аспекту.};
\scnfileitem{Может ли какое-либо свойство кибернетических систем определять (влиять на) значение сразу нескольких свойств более высокого уровня иерархии.};
\scnfileitem{Какими отношениями свойства кибернетических систем связаны со свойствами более низкого и, соответственно, более высокого уровня иерархии.};
\scnfileitem{Зачем нужна такая иерархия свойств, определяющих качество кибернетических систем и позволяющих детализировать (уточнять) то, какими свойствами определяется уровень (степень) развития каждого свойства (значение каждого свойства) за исключением свойств, которые условно можно считать элементарными, не требующими детализации (по крайнем мере, пока).};
\scnfileitem{Может ли иерархия свойств, определяющих качество кибернетических систем, быть критерием оценки и выбора того или иного подхода к построению интеллектуальных компьютерным систем.};
\scnfileitem{Какими свойствами (способностями) должна обладать кибернетическая система, имеющая высокий уровень интеллекта.};
\scnfileitem{Какими свойствами определяется уровень интеллекта многоагентной кибернетической системы.};
\scnfileitem{Как связан уровень интеллекта многоагентной системы с уровнем интеллекта агентов, входящих в ее состав.};
\scnfileitem{Почему, например, не каждый коллектив высокоинтеллектуальных людей демонстрирует высокий уровень интеллекта самого коллектива.};
\scnfileitem{Какими дополнительными свойствами кроме достаточно высокого уровня интеллекта должны обладать агенты многоагентных систем для обеспечения высокого уровня интеллекта самой многоагентной системы как самостоятельной целостной кибернетической системы.};
\scnfileitem{Как зависит уровень интеллекта многоагентной системы от организации взаимодействия между агентами, например, от использования централизованного или децентрализованного управления.}}

\scnrelfromvector{ключевые знаки}{
	кибернетическая система\\
	\scnaddlevel{1}	
	\scnsubdividing{
		естественная кибернетическая система;
		компьютерная система
		\scnaddlevel{1}	
		\scnidtf{искусственная кибернетическая система}
		\scnaddlevel{-1};
		естественно-искусственная кибернетическая система
		\scnaddlevel{1}
		\scnidtf{кибернетическая система, являющаяся симбиозом компонентов как естественного, так и искусственного происхождения}
		\scnaddlevel{-1}}
	\scnaddlevel{-1};
	качество кибернетической системы;
	физическая оболочка кибернетической системы;
	качество физической оболочки кибернетической системы;
	интеллект
	\scnaddlevel{1}
	\scnidtf{уровень интеллекта кибернетической системы}
	\scnidtf{интеллектуальность}
	\scnaddlevel{-1};
	интеллектуальная система
	\scnaddlevel{1}
	\scnidtf{интеллектуальная кибернетическая система}
	\scnsuperset{интеллектуальная компьютерная система}
	\scnaddlevel{-1};
	информация, хранимая в памяти кибернетической системы;
	качество информации, хранимой в памяти кибернетической системы;
	база знаний;
	смысловое представление информации в памяти кибернетической системы;
	решатель задач кибернетической системы;
	качество решателя задач кибернетической системы;
	память кибернетической системы;
	качество памяти кибернетической системы;
	обучаемость кибернетической системы;
	гибкость кибернетической системы;
	стратифицированность кибернетической системы;
	рефлексивность кибернетической системы
	\scnaddlevel{1}
	\scnidtf{уровень рефлексии кибернетической системы}
	\scnaddlevel{-1};
	многоагентная система;
	качество многоагентной системы;
	унифицированность агентов многоагентной системы;
	семантическая совместимость агентов многоагентной системы;
	социализация кибернетической системы
	\scnaddlevel{1}
	\scnidtf{способность кибернетической системы своей внутренней и внешней деятельностью обеспечивать высокий уровень интеллекта тех многоагентных систем, членом (агентом) которых она является}
	\scnaddlevel{-1}}

\scnauthorcomment{Поправить библиографию}

\scnrelfromvector{библиография}{
	Винер Н. Кибернетика;
	Поспелов Д.А, Гаазе-Рапопорт М. Г.  От амёбы до робота: Модели поведения;
	Финн В.К. [11] в статье Грибовской на OSTIS-2020;
	Кузнецов О.П. - 2009кн ТеореПИ-с.5-6;
	Ярушкина Н.Г. ред 2007-НечетГС-с.88-101;
	Редько В.Г.-2019кн-МоделКЭ}

\scnauthorcomment{проверить названия и порядок после всех правок}

\scnreltovector{конкатенация сегментов}{
	Иерархическая система свойств, определяющих (уточняющих) интегральный уровень качества кибернетической системы;
	Уточнение понятия кибернетической системы;
	Свойства, определяющие общий уровень качества кибернетической системы;
	Свойства, определяющие уровень интеллекта кибернетической системы;
	Свойства, определяющие качество физической оболочки кибернетической системы;
	Свойства, определяющие качество информации, хранимой в памяти кибернетической системы;
	Свойства, определяющие качество решателя задач кибернетической системы;
	Свойства, определяющие уровень обучаемости кибернетической системы;
	Свойства, определяющие уровень социализации кибернетической системы;
	Качество памяти;
	Качество многоагентной системы;
	Итоговый сегмент Начала Раздела}

\newpage


\bigskip
\scnsegmentheader{Уточнение понятия кибернетической системы}
\scnstartsubstruct

\scnheader{кибернетическая система}
\scnidtf{cистема, которая способна \uline{управлять} своими \uline{действиями}, адаптируясь к изменениям состояния внешней среды (среды своего "обитания") в целях самосохранения (сохранения своей целостности и "комфортности"{} существования путем удержания своих "жизненно"{} важных параметров в определенных рамках "комфортности") и/или в целях формирования определенных реакций (воздействий на внешнюю среду) в ответ на определенные стимулы (на определенные ситуации или события во внешней среде), а также которая способна (при соответствующем уровне развития) эволюционировать в направлении:
\begin{scnitemize}
    \item изучения своей внешней среды как минимум для предсказания последствий своих воздействий на внешнюю среду, а также для предсказания изменений внешней среды, которые не зависят от собственных воздействий;
    \item изучения самой себя и, в частности, своего взаимодействия с внешней средой;
    \item создания технологий (методов и средств), обеспечивающих изменение своей внешней среды (условий своего существования) в собственных интересах.
\end{scnitemize}
}
\scnidtf{адаптивная система}
\scnidtf{целенаправленная (целеустремленная) система}
\scnidtf{активный субъект самостоятельной деятельности}
\scnidtf{материальная сущность, способная целенаправленно (в своих интересах) воздействовать  на среду своего обитания  как минимум для сохранения своей целостности, жизнеспособности, безопасности}
\scnnote{Уровень (степень) адаптивности, целенаправленности, активности у систем, основанных на обработке информации может быть самым различным.}
\scnidtf{система, организация функционирования которой основано на обработке информации о той среде, в которой существует эта система}
\scnidtf{материальная сущность, способная к активной  целенаправленной деятельности, которая  на определенном уровне развития указанной сущности становится "осмысленной", планируемой, преднамеренной деятельностью}
\scnidtf{субъект, способный на самостоятельное выполнение некоторых "внутренних"{} и "внешних"{} действий либо порученных извне, либо инициированных самим субъектом}
\scnidtf{сущность, способная выполнять роль субъекта деятельности}
\scnidtf{естественная или искусственно созданная система, способная мониторить и анализировать свое состояние и состояние окружающей среды, а также способная достаточно активно воздействовать на собственное на собственное состояние и на состояние окружающей среды}
\scnidtf{система, способная в достаточной степени самостоятельно взаимодействовать со своей средой , решая различные задачи}
\scnidtf{система, основанная на обработке информации}
	
\scnrelto{ключевой знак}{Глушков В. М. Кибер. - 1979 ст}
\scnaddlevel{1}
	\scniselement{статья}
\scnaddlevel{-1}
\scnauthorcomment{дооформить библиографическую ссылку}

\bigskip
\scnfragmentcaption

\scnheader{Типология кибернетических систем}
\scnstartsubstruct

\scnheader{кибернетическая система}

\scnrelfrom{разбиение}{Признак естественности или искусственности кибернетических систем}
\scnaddlevel{1}
\scneqtoset{естественная кибернетическая система\\
    \scnaddlevel{1}
    \scnidtf{кибернетическая система естественного происхождения}
    \scnsuperset{человек}
    \scnaddlevel{-1}
;компьютерная система\\
    \scnaddlevel{1}
    \scnidtf{искусственная кибернетическая система}
    \scnidtf{кибернетическая система искусственного происхождения}
    \scnidtf{технически реализованная кибернетическая система}
    \scnaddlevel{-1}
;симбиоз естественных и искусственных кибернетических систем\\
    \scnaddlevel{1}
    \scnidtf{кибернетическая система, в состав которой входят компоненты как естественного, так и искусственного происхождения}
    \scnsuperset{сообщество компьютерных систем и людей}
    \scnaddlevel{-1}}
\scnaddlevel{-1}

\scnheader{искусственная сущность}
\scnidtf{артефакт}
\scnidtf{сущность, являющаяся либо результатом человеческой деятельности, либо частью самой этой деятельности}
\scnidtf{сущность искусственного происхождения}
\scnidtf{антропогенная сущность}
\scnsuperset{научно-техническое знание}
\scnaddlevel{1}
\scnidtf{знание, приобретенное в результате научно-технической деятельности человеческого общества}
\scnaddlevel{-1}
\scnsuperset{материальная искусственная сущность}
\scnaddlevel{1}
\scnsuperset{компьютерная система}
\scnaddlevel{-1}

\scnheader{компьютерная система}
\scnidtf{искусственная кибернетическая система}
\scnnote{Особенностью компьютерных систем является то, что они могут выполнять "роль"{} не только продуктов соответствующих действий по реализации этих систем, но и сами являются \textit{субъектами*}, способными выполнять (автоматизировать) широкий спектр действий. При этом интеллектуализация этих систем существенно расширяет этот спектр. \textit{См. интеллектуальная компьютерная система}.}
\scnidtf{технически реализованная кибернетическая система}
\scnidtf{искусственная кибернетическая система}
\scnsubset{кибернетическая система}
\scnsuperset{современная компьютерная система традиционного вида}
\scnsuperset{современная интеллектуальная компьютерная система}
\scnsuperset{интеллектуальная компьютерная система следующего поколения}
\scnaddlevel{1}
\scnsuperset{ostis-система}
\scnnote{Основной тенденцией эволюции компьютерных систем является повышение уровня их интеллектуальности.}
\scnrelfromset{особенность}{\scnfileitem{Ориентация на принципиально новые компьютеры};\scnfileitem{Cущественное повышение уровня интеллекта}}
\scnaddlevel{-1}
\scnrelfrom{разбиение}{Структурная классификация кибернетических систем}
\scnaddlevel{1}
\scneqtoset{простая кибернетическая система\\
;индивидуальная кибернетическая система\\
;многоагентая система\\
\scnaddlevel{1}
\scnsubdividing{
одноуровневый коллектив кибернетических систем
    \scnaddlevel{1}
    \scnidtf{многоагентная система, агентами которой не могут быть многоагентные системы}
    \scnaddlevel{-1}
;иерархический коллектив кибернетических систем
    \scnaddlevel{1}
    \scnidtf{многоагентная система, по крайней мере одним  агентом которой является многоагентная система}
    \scnaddlevel{-1}}
\scnsubdividing{коллектив из простых кибернетических систем\\
\scnaddlevel{1}
\scnnote{Такой коллектив может быть либо одноуровневым, либо иерархическим коллективом}
\scnaddlevel{-1};
коллектив из индивидуальных кибернетических систем;коллектив из индивидуальных и простых кибернетических систем}
\scnaddlevel{-1}}


\scnheader{кибернетическая система}
\scnrelfrom{разбиение}{Классификация кибернетических систем по признаку наличия надсистемы и роли в рамках этой надсистемы}
\scnaddlevel{1}
\scneqtoset{кибернетическая система, не являющаяся частью никакой другой кибернетической системы\\
\scnaddlevel{1}
\scnidtf{кибернетическая система, не имеющая надсистем}
\scnaddlevel{-1}
;кибернетическая система, встроенная в индивидуальную кибернетическую систему\\
;агент многоагентной системы\\
\scnaddlevel{1}
\scnidtf{кибернетическая система, являющаяся агентом одной или нескольких многоагентных систем}
\scnaddlevel{-1}
}
\scnaddlevel{-1}

\scnheader{простая кибернетическая система}
\scnidtf{\textit{кибернетическая система}, уровень развития которой находится ниже уровня \textit{индивидуальных кибернетических систем} и которая является специализированным средством обработки информации специализированным решателем задач, реализующим (интерпретирующим) чаще всего один \textit{метод} решения задач и, соответственно, решающим только заданный \textit{класс задач}}
\scnidtf{специализированный \textit{решатель задач}}
\scnnote{\textit{простая кибернетическая система} может быть \textit{компонентом*}, встроенным в \textit{индивидуальную кибернетическую систему}, а также может быть \textit{агентом*} \scnbigskip \textit{многоагентной системы}, являющейся коллективом из простых кибернетических систем}

\scnheader{индивидуальная кибернетическая система}
\scnidtf{условно выделенный уровень развития \textit{кибернетических систем}, в основе которого лежит переход от \textit{специализированного решателя задач к индивидуальному решателю}, обеспечивающему интерпретацию произвольного (нефиксированного) набора \textit{методов} (программ) решения задач при условии, если эти \textit{методы} введены (загружены, записаны) в \textit{память} \textit{кибернетической системы}}
\scnidtf{кибернетическая система, способная быть самостоятельной}
\scnexplanation{Признаками индивидуальных кибернетических систем являются:
\begin{scnitemize}
    \item наличие \textit{памяти}, предназначенной для хранения как минимум интерпретируемых \textit{методов} (программ)  и обеспечивающей корректировку (редактирование) хранимых \textit{методов}, а также их удаление  из \textit{памяти} и ввод (запись) в \textit{память} новых \textit{методов};
    \item легкая возможность "программировать"{} \textit{кибернетическую систему} на решение других задач, что обеспечивается наличием \textit{универсальной модели решения задач} и, соответственно, \textit{универсальным интерпретатором \uline{любых} моделей}, представленных (записанных) на соответствующем \textit{языке};
    \item наличие пусть даже простых средств коммуникации (обмена информацией) с другими \textit{кибернетическими системами} (например, с людьми);
    \item способность входить в различные \textit{коллективы кибернетических систем}.
\end{scnitemize}
}
\scnnote{класс \textit{индивидуальных кибернетических систем} — это определенный этап эволюции кибернетических систем, означающий переход к кибернетическим системам, которые способны самостоятельно "выживать"}
\scnidtf{самостоятельная автономная, целостная кибернетическая системам}
\scnidtf{субъект деятельности}
\scnnote{\textit{индивидуальная кибернетическая система} может быть агентом (членом) многоагентной системы (членом коллектива индивидуальных кибернетических систем), но некоторые многоагентные системы могут состоять из агентов , не являющихся  \textit{индивидуальными кибернетическими системами}, представляющих собой простые специализированные кибернетические системы, выполняющие достаточно простые действия (см. коллективное поведение автоматов Стефанюк теория самовоспроизводящихся автоматов Джон фон Нейман)}
\scnauthorcomment{Исправить библиографию}

\scnidtf{кибернетическая система, которая обладает достаточной самостоятельностью (целостностью), но не является коллективом таких самостоятельных  кибернетических систем}
\scnidtf{минимальная самостоятельная (самодостаточная, в известной степени автономная) кибернетическая система}
\scnidtf{индивидуальный субъект}

\scnheader{кибернетическая система, встроенная в индивидуальную кибернетическую систему}
\scnrelfrom{включение;пример}{sc-агент ostis-системы}
\scnrelfrom{включение;пример}
{решатель задач ostis-системы}
\scnaddlevel{1}
\scnidtf{коллектив всех sc-агентов ostis-системы}
\scnaddlevel{-1}

\scnheader{многоагентная система}
\scnidtf{коллектив взаимодействующих автономных кибернетических систем, имеющих общую среду обитания (жизнедеятельности)}
\scnsubdividing{одноуровневая многоагентная система;иерархическая многоагентная система}

\scnheader{одноуровневая многоагентная система}
\scnidtf{специализированное средство решения задач, реализующее либо \uline{одну} модель параллельного (распределенного) решения задач соответствующего класса, либо комбинацию \uline{фиксированного числа} разных и параллельно реализованных моделей решения задач}
\scnsubdividing{одноуровневая однородная многоагентная система;одноуровневая неоднородная многоагентная система}

\scnheader{коллектив индивидуальных кибернетических систем}
\scnsubset{многоагентная система}
\scnidtf{многоагентная система, агентами (членами) которой являются \uline{индивидуальные}(!) кибернетические системы}
\scnsubdividing{
коллектив людей\\
\scnaddlevel{1}
\scnidtf{человеческое сообщество}
\scnaddlevel{-1}
;сообщество компьютерных систем и людей
}

\scnheader{иерархический коллектив индивидуальных кибернетических систем}
\scnidtf{многоагентная система, агентами (членами) которой могут быть:
\begin{scnitemize}
    \item индивидуальные кибернетические системы;
    \item коллективы индивидуальных кибернетических систем;
    \item коллективы, состоящие из индивидуальных кибернетических систем и коллективов индивидуальных кибернетических систем и т.д.
\end{scnitemize}
}



\bigskip

\scnsegmentheader{Структура кибернетической системы}

\scnstartsubstruct

\scnexplanation{Кибернетическая систем состоит из:
	\begin{scnitemize}
		\item информация в памяти;
		\item решатель;
		\item интерфейс с  внешней средой и физической оболочкой;
		\item физическая оболочка;
		\item внешняя среда
	\end{scnitemize}
}

\scnheader{Кибернетическая система}
\scnrelfromset{обобщенная декомпозиция}{
информация, хранимая в памяти кибернетической системы;абстрактная память кибернетической системы;решатель задач кибернетической системы;физическая оболочка кибернетической системы
}

\scnheader{Информация, хранимая в памяти кибернетической системы}
\scnidtf{информация, хранимая в памяти \textit{кибернетической системы} и представляющая собой информационную модель среды, в которой действует (существует, функционирует) эта \textit{кибернетическая система}
}
\scnidtf{текущее состояние памяти кибернетической системы}
\scnidtf{текущее состояние внутренней (информационной) среды кибернетической системы}
\scnrelto{второй домен}{информация, хранимая в памяти оптической системы*}
\scnaddlevel{1}
\scniselement{бинарное отношение}
\scniselement{ориентированное отношение}
\scnaddlevel{-1}

\scnheader{абстрактная память кибернетической системы}
\scnidtf{внутренняя абстрактная информационная среда кибернетической системы, представляющая собой динамическую информационную  конструкцию, каждое состояние которой есть не что иное, как информация , хранимая в памяти кибернетической системы в соответствующий момент времени}
\scnidtf{абстрактная динамическая модель памяти кибернетической системы}
\scnsubset{динамическая информационная конструкция}
\scnaddlevel{1}
\scnidtf{процесс преобразования информационной конструкции}
\scnaddlevel{-1}

\scnheader{решатель задач кибернетической системы}
\scnidtf{совокупность всех навыков (умений), приобретенных кибернетической системой к рассматриваемому моменту}
\scnidtf{встроенный в кибернетическую систему субъект, способный выполнять целенаправленные ("осознанные") действия во внешней среде этой кибернетической системы, а также в её внутренней среде (в абстрактной памяти)}

\scnheader{действие кибернетической системы}
\scnsubset{действие}
\scnidtf{целенаправленное ("осознанное") действие, выполняемое кибернетической системой, а точнее, её решателем задач}
\scnsubdividing{внешнее действие кибернетической системы\\
	\scnaddlevel{1}
	\scnidtf{действие, выполняемое кибернетической системой в её внешней среде}
	\scnidtf{поведенческое действие}
	\scnaddlevel{-1}
;действие кибернетической системы, выполняемое в собственной физической оболочке
;действие кибернетической системы, исполняемое в собственной абстрактной памяти
\scnaddlevel{1}
	\scnidtf{речь идёт о действиях, направленных на преобразование информации,хранимой в памяти, но никак не на преобразование физической памяти (физической оболочки абстрактной памяти)}
\scnaddlevel{-1}	
}
\scnnote{Каждое \uline{сложное} действие,выполняемое кибернетической системой вне собственный абстрактной памяти, включает в себя поддействия, выполняемые в указанной абстрактной памяти. Это означает, что все внешние действия кибернетической системы \uline{управляются} внутренними её действиями (действиями в абстрактной памяти).}

\scnheader{задача}
\scnidtf{спецификация действия}
\scnidtf{формулировка задачи с различной степенью детализации (уточнения) специфицируемого (описываемого) действия, в состав которой может входить:
	\begin{scnitemize}
		\item описание цели (целевой ситуации);
		\item указание объектов (аргументов) действия;
		\item указание типа действия (класса действий, которому принадлежит данное действие);
		\item указание субъекта действия;
		\item указание инструмента (средств) выполненного действия;
		\item и др.
	\end{scnitemize}}

\scnnote{Процесс решения задачи и действие, специфицируемое этой задачей, (точнее, процесс выполнения этого действия) суть одно и то же.}


\scnheader{задача, решаемая кибернетической системой}
\scnidtf{задача, решаемая соответствующей кибернетической системой}
\scnidtf{Второй домен отношения "быть задачей, решаемой заданной кибернетической системой*"}
\scnrelboth{следует отличать}{задача, решаемая кибернетической системой*}
\scnaddlevel{1}
\scnidtf{быть задачей, решаемой заданной кибернетической системой*}
\scnaddlevel{-1}
\scnsubdividing{задача, решаемая кибернетической системой во внешней среде\\
	\scnaddlevel{1}
	\scnidtf{внешняя задача кибернетической системы}
	\scnidtf{задача, направленная на изменение состояния внешней среды соответствующей кибернетической системы ,но включающая в себя (в качестве подзадач) задачи, решаемые в памяти кибернетической системы, например: 
		\begin{scnitemize}
			\item интерфейсные задачи (анализ первичный информации о текущем состоянии внешней среды),
			\item cенсо-моторную координацию выполнения сложных действий во внешней среде, состоящих из большого количества частных (более простых) действий, находящихся на разных уровнях иерархии,
			\item задачи планирования целенаправленного поведения во внешней среде,
			\item задачи принятия решений.
		\end{scnitemize}}
	\scnaddlevel{-1}
;задача, решаемая кибернетической системой в собственной физической оболочке
;задача решаемая  кибернетической системой в абстрактной памяти
	\scnaddlevel{1}
	\scnidtf{задача, полностью решаемая в памяти кибернетической системы и направленная на изменение состояния информации, хранимой в памяти кибернетической системы}
	\scnidtf{внутренняя задача кибернетической системы}
	\scnaddlevel{-1}
}

\scnheader{навык}
\scnsubset{знание}
\scnexplanation{знание частного вида, содержащее (1) некоторые метод - знание о том, как можно решать задачи, принадлежащие соответствующему множеству задач, (2) полное знание о том, как указанный метод следует интерпретировать (реализовывать), декомпозируя исходные задачи на подзадачи и, в конечном счёте на элементарные действия, выполняемые \textit{процессором кибернетической системы}}
\scnidtf{умение}
\scnidtf{методы и средства обеспечивающие способность \textit{кибернетической системы} решать некоторое множество задач (выполнять некоторое множество действий)}

\scnheader{интерфейс кибернетической системы}
\scnidtf{условно выделяемый компонент \textit{решателя задач кибернетической системы}, обеспечивающий решение \textit{интерфейсных задач}, направленных на \uline{непосредственную} реализацию взаимодействия \textit{кибернетической системы} с её \textit{внешней средой}}
\scnidtf{решатель интерфейсных задач кибернетической системы}
\scnrelto{обобщенная часть}{решатель задач кибернетической системы}
\scnrelboth{следует отличать}{физическое обеспечение интерфейса кибернетической системы}
\scnaddlevel{1}
\scnrelto{обобщенная часть}{физическая оболочка кибернетической системы}
\scnaddlevel{-1}

\scnheader{физическая оболочка кибернетической системы}
\scnrelfromset{обобщенная декомпозиция}{память кибернетической системы\\
;процессор кибернетической системы
;физическое обеспечение интерфейса кибернетической системы
\scnaddlevel{1}
	\scnidtf{аппаратное обеспечение интерфейса кибернетической системы с её внешней средой}
	\scnrelfromset{обобщенная декомпозиция}{сенсорная подсистема физической оболочки кибернетической системы;
	эффекторная подсистема физической оболочки кибернетической системы}
\scnaddlevel{-1}
;корпус кибернетической системы
}

\scnheader{физическая оболочка кибернетической системы}
\scnidtf{часть кибернетической системы, являющаяся "посредником" между её внутренней средой (памятью, в которой хранится и обрабатывается информация кибернетической системы) и её внешней средой}
\scnrelto{второй домен}{физическая оболочка кибернетической системы* }
\scnaddlevel{1}
\scniselement{бинарное отношение}
\scniselement{ориентированное отношение}





\bigskip

\scnsegmentheader{Комплекс свойств, определяющих качество физической оболочки кибернетической системы}

\scnstartsubstruct

\scnheader{качество физической оболочки кибернетической системы}
\scnidtf{интегральное качество "аппаратной"{} (физической) основы кибернетической системы}
\scnidtf{"hardware"{} кибернетической системы}
\scnrelfromlist{свойство-предпосылка}{
качество памяти кибернетической системы;
качество процессора кибернетической системы;
качество сенсоров кибернетической системы;
качество эффекторов кибернетической системы;
приспособленность физической оболочки кибернетической системы к ее совершенствованию;
удобство транспортировки кибернетической системы;
надежность физической оболочки кибернетической системы
}
\scnheader{качество памяти кибернетической системы}
\scnreltolist{свойство-предпосылка}{
качество информации, хранимой в памяти кибернетической системы;
качество решателя задач кибернетической системы
}
\scnrelfromlist{свойство-предпосылка}{
способность памяти кибернетической системы обеспечить хранение высококачественной информации;
способность памяти кибернетической системы обеспечить функционирование высококачественного решателя задач;
объём памяти
}

\scnheader{память кибернетической системы}
\scnidtf{компонент \textit{кибернетической системы}, представляющий собой "внутреннюю"{} среду \textit{кибернетической системы}, в которой она хранит (запоминает) и преобразует \textit{информационную модель} своей \textit{внешней среды}. При этом важно, чтобы память обеспечивала высокий уровень \textit{гибкости} указанной \textit{информационной модели}. Важно также, чтобы эта \textit{информационная модель} была моделью не только \textit{внешней среды} \scnbigskip \textit{кибернетической системы}, но также и моделью самой этой \textit{информационной модели} -- описанием её \textit{текущей ситуации}, предыстории, закономерностей. Таким образом, \textit{кибернетическая система}, имеющая \textit{память}, функционирует в двух средах -- во внешней, в которой существуют и преобразуются внешние(материальные) сущности, и во внутренней, в которой существуют и преобразуются(обрабатываются) внутренние \textit{информационные конструкции}.}
\scnnote{\textit{Кибернетические системы}, находящиеся на низком уровне развития(качества) \textit{памяти} не имеют. Адаптационные механизмы такой кибернетической системы "жестко запаяны"{} в связях между блоками обработчика \textit{сигналов} при переходе от \textit{сигналов}, вырабатываемых \textit{сенсорами} к \textit{сигналам}, которые управляют \textit{эффекторами}.}
\scnidtf{внутренняя среда кибернетической системы, обеспечивающая хранение и преобразование(обработку) информационной модели внешней среды кибернетической системы}
\scnnote{Сам факт возникновения памяти в \textit{кибернетической системе} является важнейшим этапом её эволюции. Дальнейшее развитие \textit{памяти кибернетической системы}, обеспечивающее:
\begin{scnitemize}
	\item хранение все более качественной информации, хранимой в памяти
	\item все более качественную организацию обработки этой информации, т.е. переход на поддержку(обеспечение) все более качественных моделей обработки информации
\end{scnitemize}
является важнейшим фактором эволюции \textit{кибернетических систем}.}

\scnheader{способность памяти кибернетической системы обеспечить хранение высококачественной информации}
\scnrelfromlist{свойство-предпосылка}{
способность системы обеспечить компактное хранение сложноструктурированных баз знаний\\
	\scnaddlevel{1}
	\scnnote{Здесь имеется в виду необходимость перехода от линейной организации, памяти на физическом уровне (как последовательности ячеек памяти) к нелинейной, графодинамической памяти.}
	\scnaddlevel{-1}
;способность памяти кибернетической системы обеспечить хранение широкого многообразия знаний\\
	\scnaddlevel{1}
	\scnnote{имеется в виду хранение гибридных баз знаний}
	\scnaddlevel{-1}
}

\scnheader{способность памяти кибернетической системы обеспечить функционирование высококачественного решателя задач}
\scnrelfromlist{свойство-предпосылка}{качество доступа к информации, хранимой памяти кибернетической системы\\
	\scnaddlevel{1}
	\scnnote{Здесь имеется в виду необходимость перехода от адресного к ассоциативному доступу, причем, с расширением многообразия видов реализуемых запросов, в частности, к реализации запросов фрагментов баз знаний по заданному образцу произвольного размера и произвольной конфигурации.}
	\scnaddlevel{-1}
;логико-семантическая гибкость памяти кибернетической системы
;способность памяти кибернетической системы обеспечить интерпретацию широкого многообразия моделей решения задач
}

\scnheader{логико-семантическая гибкость памяти кибернетической системы}
\scnidtf{степень близости физической организации памяти кибернетической системы к реализуемым ею базовым семантически целостным действиям над информацией, хранимой в памяти}
\scnidtf{простота реализации базовых семантически целостных действий над информацией, хранимой в памяти кибернетической системы}
\scnnote{Важен переход от "мелких"{} действий, к элементарным действиям, имеющим логико-семантический смысл (целостность, законченность}

\scnheader{качество процессора кибернетической системы}
\scnrelto{свойство-предпосылка}{качество решателя задач кибернетической системы}
\scnrelfromlist{свойство-предпосылка}{способность процессора кибернетической системы обеспечить функционирования высококачественного решателя задач\\
	\scnaddlevel{1}
	\scnrelfromlist{свойство-предпосылка}{многообразие моделей решения задач, интерпретируемых процессором кибернетической системы
;простота и качество интерпретации процессором системы широкого многообразия моделей решения задач\\
		\scnaddlevel{1}	
		\scnnote{Указанная простота определяется степенью близости интерпретируемых моделей решения задач к “физическому” уровню организации процессора кибернетической системы}
		\scnaddlevel{-1}
;обеспечение процессором кибернетической системы качественного управления информационными процессами в памяти\\
		\scnaddlevel{1}
		\scnnote{Речь идет о грамотном сочетание таких аспектов управление процессами, как централизация и децентрализация, синхронность и асинхронность, последовательность и параллельность}
		\scnrelfrom{свойство-предпосылка}{уровень параллелизма обработки информации в памяти кибернетической системы}
		\scnidtf{максимальное количество одновременно выполняемых информационных процессов в памяти кибернетической системы}
		\scnaddlevel{-1}
;быстродействие процессора кибернетической системы}
	\scnaddlevel{-1}
}

\scnheader{многообразие моделей решения задач, интерпретируемых  процессором кибернетической системы}
\scnnote{Максимальным уровнем качества процессора кибернетической системы по данном параметру является его универсальность, т.е. его принципиальная возможность интерпретировать любую модель решения как интеллектуальных, так и неинтеллектуальных задач(алгоритмизацию, процедурную параллельную синхронную, процедруную параллельную асинхронную, продукционную, нейросетевую, генетическую, функциональную, целое семейство моделей).}

\scnheader{качество сенсоров кибернетической системы}
\scnrelfrom{свойство-предпосылка}{многообразие видов сенсоров кибернетической системы\\
	\scnidtf{многообразие средств восприятия (отображения) информации о текущем состоянии внешней среды кибернетической системы и её собственной физической оболочки}
}

\scnheader{качество эффекторов кибернетической системы}
\scnrelfrom{свойство-предпосылка}{многообразие видов эффекторов кибернетической системы\\
	\scnidtf{многообразие средств воздействия на собственную физическую оболочку кибернетической системы и через нее на внешнюю среду этой системы}
	\scnnote{Эффекторы кибернетической системы являются инструментами воздействия кибернетической системы на свою внешнюю среду}
}

\scnheader{приспособленность физической оболочки кибернетической системы к её совершенствованию}
\scnidtf{приспособленность кибернетической системы к повышению качества её физической оболочки}
\scnidtf{простота ремонта и совершенствования таких компонентов кибернетической системы как память, процессор, сенсоры, эффекторы}
\scnrelfrom{частное свойство}{ремонтопригодность физической оболочки кибернетической системы}
\scnrelfromset{группа свойств-предпосылок}{гибкость физической оболочки кибернетической системы
;стратифицированность физической оболочки кибернетической системы\\
	\scnaddlevel{1}
	\scnidtf{мобильность физической оболочки кибернетической системы}
	\scnidtf{легкость сохранения целостности физической оболочки кибернетической системы при внесении различных изменений (локализация области учета последствий внесения изменений, предсказуемость последствий)}
	\scnaddlevel{-1}
}

\bigskip

\scnendstruct \scninlinesourcecommentpar{Закончили Сегмент ``Комплекс свойств, определяющих качество физической оболочки кибернетической системы''}


\bigskip
\scnsegmentheader{Комплекс свойств, определяющих уровень интеллекта кибернетической системы}
\scnstartsubstruct

\scnheader{интеллект}
\scniselement{свойство}
\scniselement{упорядоченное свойство}
\scnidtf{уровень (степень, величина) интеллекта кибернетической системы}
\scnidtf{Семейство классов \textit{кибернетических систем}, обладающих эквивалентным (одинаковым) уровнем интеллекта -- от низкого до высокого уровня интеллекта}
\scnidtf{свойство кибернетических систем, характеризующее эффективность их взаимодействия со своей средой (средой их "жизнедеятельности"{})}
\scnrelfrom{область определения}{кибернетическая система}
\scnexplanation{С формальной точки зрения интеллектуальность -- это семейство классов кибернетических систем, в каждый из которых входят кибернетические системы, эквивалентные по уровню и характеру проявления интеллектуальных свойств (в том числе способностей).\\
Таким образом, характер (вид) интеллектуальных свойств кибернетических систем и уровень их развития для разных кибернетических систем может быть разным. В соответствии с этим кибернетические системы можно сравнивать между собой.}
\scnnote{Основным свойством (характеристикой, качеством, параметром) кибернетической системы является уровень (степень) ее интеллекта, который является \uline{интегральной} характеристикой, определяющей уровень эффективности взаимодействия кибернетической системы со средой своего существования.}
\scnidtf{комплексное свойство (качество) кибернетической системы, определяющее уровень ее "выживаемости"{} во внешней среде и предполагающее возможность воздействия на эту среду и даже возможность ее преобразования}
\scnidtf{интеллектуальный потенциал кибернетической системы}
\scnidtf{спектр знаний, навыков и способностей к обучению кибернетической системы}
\scnidtf{интеллектуальность кибернетической системы}
\scnnote{Процесс эволюции \textit{кибернетических систем} следует рассматривать как процесс повышения уровня их качества по целому ряду свойств (характеристик) и, в первую очередь, как процесс повышения уровня их \textit{интеллекта}. При этом можно говорить об эволюции каждой конкретной \textit{кибернетической системы} в процессе своей "жизнедеятельности"{}, а также об эволюции целого класса \textit{кибернетических систем}, когда новые экземпляры этого класса являются более интеллектуальными, чем их предшественники. В таком аспекте, в частности, можно рассматривать эволюцию \textit{компьютерных систем} (искусственных кибернетических систем).}
\scnnote{Очень важно уточнить, какими иными свойствами \textit{кибернетических систем} определяется уровень и характер их интеллектуальности. Подчеркнем, что \uline{любая} \textit{кибернетическая система} обладает соответствующим уровнем интеллектуальности. Пусть даже и достаточно низким. Существенным является уточнение того, за счет чего уровень интеллектуальности \textit{кибернетической системы} может быть повышен. Нет смысла проводить четкую границу между \textit{интеллектуальными кибернетическими системами} и неинтеллектуальными. Но есть смысл уточнять направления повышения уровня интеллектуальности \textit{кибернетических систем.}}
\scntext{эпиграф}{Никто не может провести линию, отделяющую атмосферу от космоса, или черту, за которой начинается жизнь, или границу электронного облака. Все дело в степени проявления свойства.}
	\scnaddlevel{1}
	\scnrelfrom{автор}{Барт Коско}
	\scnaddlevel{-1}
\scnnote{Прежде, чем говорить о требованиях, предъявляемых к \textit{технологии проектирования и производства интеллектуальных компьютерных систем (искусственных кибернетических систем}, обладающих высоким уровнем \textit{интеллекта)}, необходимо уточнить (детализировать) \textit{свойства}, присущие указанным системам и являющиеся предпосылками, обеспечивающими высокий уровень \textit{интеллекта}. Подчеркнем, что указанные \textit{свойства}, уточняющие (детализирующие, обеспечивающие, определяющие) \textit{свойства} \scnbigspace \textit{интеллектуальных систем}\scnbigspace (\textit{свойства}, определяющие уровень \textit{интеллекта} этих систем) должны быть общими как для искусственных кибернетических систем (\textit{компьютерных систем}), так и для \textit{естественных кибернетических систем.}}
\scnidtf{интегральное качество информационного обеспечения и информационных процессов в кибернетической системе}
\scnidtf{интегральное качество кибернетической системы, определяемое:
	\begin{scnitemize}
		\item уровнем ее образованности -- качеством накопленных к заданному моменту знаний и умений (навыков);
		\item уровнем ее обучаемости -- способностью \uline{самостоятельно} повышать уровень свой образованности.
	\end{scnitemize}}
\scnrelfromlist{свойство-предпосылка}{образованность кибернетической системы
;обучаемость кибернетической системы
;социализация кибернетической системы\\
	\scnaddlevel{1}
	\scnnote{интеллект \textit{кибернетической системы}, как и лежащий в его основе познавательный процесс, выполняемый кибернетической системой, имеет социальный характер, поскольку наиболее эффективно формируется и развивается в форме взаимодействия \textit{кибернетической} системы с другими \textit{кибернетическими системами}}
	\scnaddlevel{-1}}

\end{SCn}
