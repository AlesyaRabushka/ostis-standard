\begin{SCn}

\scnsectionheader{Предметная область и онтология решателей задач компьютерных систем}

\scnstartsubstruct

\scnheader{Предметная область решателей задач современных интеллектуальных компьютерных систем}
\scnsdmainclasssingle{решатель задач компьютерных систем}
\scnsdclass{***}
\scnsdrelation{***}

\scnheader{решатель задач компьютерных систем}
\scnexplanation{Одним из ключевых компонентов интеллектуальной системы, обеспечивающим возможность решать широкий круг задач, является решатель задач. Особенностью решателей задач интеллектуальных систем по сравнению с другими современными
программными системами является необходимость решать задачи в условиях, когда сведения, необходимые для решения задачи, не локализованы явно в базе знаний интеллектуальной системы и должны быть найдены в процессе решения задачи на основании каких-либо критериев}
\scnsuperset{объединенный решатель задач}
\scnaddlevel{1}
\scnexplanation{В общем случае решатель задач обеспечивает возможность решения задач, связанных как с непосредственно основной функциональностью системы, так и с обеспечением эффективности работы такой системы, а также с обеспечением автоматизации развития самой этой системы. Решатель задач, обеспечивающий выполнение всех перечисленных функций, будем называть \textbf{\textit{объединенным решателем задач}} указанной интеллектуальной системы}
\scnaddlevel{-1}
\scntext{проблемы разработки}{Несмотря на то что в настоящее время существует большое число моделей решения задач, многие из которых реализованы и успешно используются на практике в различных системах, остается актуальной проблема низкой согласованности принципов, лежащих в основе реализации таких моделей, и отсутствия единой унифицированной основы для реализации и интеграции различных моделей, что приводит к тому, что:
\begin{scnitemize}
\item затруднена возможность одновременного использования различных моделей решения задач в рамках одной системы при решении одной и той же комплексной задачи; практически невозможно комбинировать различные модели с целью решения задачи, для которой априори отсутствует алгоритм ее
решения;
\item практически невозможно использовать технические решения, реализованные в одной системе, в других системах, т. е. возможности использования компонентного подхода при построении решателей задач сильно ограничены. Как следствие, велико количество дублирований аналогичных решений в разных системах;
\item фактически отсутствуют комплексные методики и средства построения решателей задач, которые бы обеспечивали возможность проектирования, реализации и отладки решателей различного вида.
\end{scnitemize}

Следствиями указанных проблем являются:
\begin{scnitemize}
\item высокая трудоемкость разработки каждого решателя, увеличение сроков их разработки, а значит, и увеличение затрат на разработку и поддержку соответствующих интеллектуальных систем;
\item высокая трудоемкость внесения изменений в уже разработанные решатели, т. е. отсутствует или сильно затруднена возможность дополнения уже разработанного решателя новыми компонентами и внесения изменений в уже существующие компоненты в процессе эксплуатации системы. Таким образом, высока трудоемкость поддержки разработанных решателей;
\item высокий уровень профессиональных требований к разработчикам решателей задач, что обусловлено, в частности:
\begin{scnitemizeii}
\item высокой сложностью существующих формализмов в области решения задач, рассчитанных на их интерпретацию компьютерной системой, а не человеком;
\item отсутствием возможности рассматривать разрабатываемые решатели на разных уровнях детализации, выделения на каждом уровне достаточно независимых компонентов, что затрудняет процесс проектирования, тестирования и отладки таких решателей, а также снижает эффективность попыток объединения разработчиков решателей в коллективы по причине увеличения накладных расходов на согласование их деятельности;
\item низким уровнем информационной поддержки разработчиков и автоматизации их деятельности.
\end{scnitemizeii}
\end{scnitemize}
}

\scnheader{модель решения задач}
\scnsubdividing{модель решения задач на основе хранимых программ\\
\scnaddlevel{1}
\scnsuperset{модель решения задач на основе декларативных программ}
\scnsuperset{модель решения задач на основе императивных программ}
\scnsuperset{модель решения задач на основе генетических алгоритмов}
\scnauthorcomment{сложно сказать, генетический алгоритм это программа или нет, механизм логического вывода тоже наверное можно считать программой в таком случае}
\scnsuperset{модель решения задач на основе нейросетевых моделей}
\scnaddlevel{-1}
;модель решения задач в условиях, когда программа решения не известна\\
\scnaddlevel{1}
\scnsuperset{стратегия решения задач}
\scnaddlevel{1}
\scnhaselement{стратегия поиска пути решения задач в ширину}
\scnhaselement{стратегия поиска пути решения задач в глубину}
\scnaddlevel{-1}
\scnsuperset{модель логического вывода}
\scnaddlevel{1}
\scnsuperset{модель дедуктивного вывода}
\scnsuperset{модель индуктивного вывода}
\scnsuperset{модель трансдуктивного вывода}
\scnsuperset{модель нечеткого вывода}
\scnsuperset{модель на основе темпоральных логик}
\scnsuperset{модель на основе логик умолчаний}
\scnaddlevel{-2}
}

\scnheader{многоагентная система}
\scntext{достоинства}{
\begin{scnitemize}
\item автономность (независимость) агентов в рамках такой системы, что позволяет локализовать изменения, вносимые в решатель при его эволюции, и снизить соответствующие трудозатраты;
\item децентрализация обработки, т.е. отсутствие единого контролирующего центра, что также позволяет локализовать вносимые в решатель изменения.
\end{scnitemize}
}
\scntext{структура}{В общем случае для построения некоторой конкретной многоагентной системы необходимо уточнить следующие ее компоненты:
\begin{scnitemize}
\item модель собственно агента, входящего в состав такой системы, включая классификацию таких агентов и набор понятий, характеризующих каждый агент в рамках системы. В настоящее время наиболее популярной является модель BDI (belief-desire-intention), в рамках которой предполагается описывать на соответствующих языках <<убеждения>>, <<желания>> и <<намерения>> каждого агента системы. 
\item модель среды, в рамках которой находятся агенты, на события в которой они реагируют и в рамках которой могут осуществлять некоторые преобразования. Обзор разновидностей сред для многоагентных систем приводится в работе \cite{Weyns2007}.
\item модель коммуникации агентов, в рамках которой уточняется язык взаимодействия агентов (структура и классификация сообщений) и способ передачи сообщений между агентами. В настоящее время существует ряд стандартов, описывающих языки взаимодействия агентов, например, KQML \cite{Finin1994} и ACL \cite{ACL}. 
\item модель координации агентов, регламентирующая принципы их деятельности, в том числе, механизмы решения возможных конфликтов. В настоящее время основное число работ в области многоагентных систем направлено именно на разработку механизмов координации агентов, в числе которых выделение агентов более высокого уровня (метаагентов) \cite{Hartung2008}, различные социально-психологические модели \cite{Vasconcelos2009,Rumbell2012}, поведение на основе онтологий \cite{Gorodetsky2015} и другие.
\end{scnitemize}}
\scntext{проблемы разработки}{
\begin{scnitemize}
\item жесткая ориентация большинства средств построения многоагентных систем на модель BDI (Belief-Desire-Intention) приводит к существенным накладным расходам, связанным с необходимостью выражения конкретной практической задачи в системе понятий BDI. В то же время, ориентация на модель BDI неявно провоцирует искусственное разделение языков, описывающих собственно компоненты BDI и знания агента о внешней среде, что приводит к отсутствию унификации представления и, соответственно, дополнительным накладным расходам.
\item большинство современных средств построения многоагентных систем ориентированы на представление знаний агента при помощи узкоспециализированных языков, зачастую не предназначенных для представления знаний в широком смысле. Речь при этом идет как о знаниях агента о себе самом (например, в соответствии с моделью BDI), так и знаниях о внешней среде. В некоторых подходах вначале строится онтология, для создания которой, однако, часто используются средства с низкой выразительной способностью, не предназначенные для построения онтологий \cite{Evertsz2004,JADE2017}. В конечном итоге такой подход приводит к сильной ограниченности возможностей разработанных многоагентных систем и их несовместимости.
\item абсолютное большинство современных средств построения многоагентных систем предполагает, что взаимодействие агентов осуществляется путем обмена сообщениями непосредственно от агента к агенту. Такой подход обладает существенным недостатком, связанным с тем, что в этом случае каждый агент системы должен иметь достаточно полную информацию о других агентах в системе, что приводит к дополнительным затратам ресурсов, кроме того добавление или удаление одного или нескольких агентов приводит к необходимости оповещения об этом других агентов. Данная проблема решается путем организации общения агентов по принципу <<доски объявлений>> \cite{Jagannathan1989}, предполагающему, что сообщения помещаются в некоторую общую для всех агентов область, при этом каждый агент в общем случае может не знать, какому из агентов адресовано сообщение и от какого из агентов получено то или иное сообщение. Однако, данный подход не исключает проблему, связанную с необходимостью разработки специализированного языка взаимодействия агентов, который в общем случае не связан с языком, на котором описываются знания агента о решаемых задачах и окружающей среде.
\item многие средства построения многоагентных систем построены таким образом, что логический уровень взаимодействия агентов жестко привязан к физическому уровню реализации многоагентной системы. Например, при передаче сообщений от агента к агенту разработчику многоагентной системы необходимо помимо семантически значимой информации указывать ip-адрес компьютера, на котором расположен агент-получатель, кодировку, с помощью которой закодирован текст сообщения и другую техническую информацию, обусловленную исключительно особенностями текущей реализации средств.
\item в большинстве подходов среда, с которой взаимодействуют агенты, уточняется отдельно разработчиком для каждой многоагентной системы, что с одной стороны, расширяет возможности применения соответствующих средств, но с другой стороны приводит к существенным накладным расходам и несовместимости таких многоагентных систем. Кроме того, в ряде случаев разработчик также обязан учитывать особенности технической реализации средств разработки в плане их стыковки с предполагаемой средой, в роли которой может выступать, например, локальная или глобальная сеть.
\end{scnitemize}}

\scnendstruct

\end{SCn}