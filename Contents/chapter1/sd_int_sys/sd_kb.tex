\begin{SCn}

\scnsectionheader{Предметная область и онтология баз знаний}

\scnstartsubstruct

\scnheader{Предметная область баз знаний}
\scnsdmainclasssingle{база знаний}
\scnsdclass{***}
\scnsdrelation{***}

\scnheader{знание}
\scnsuperset{фактографическое знание}
\scnsuperset{закономерность}
\scnsuperset{программа}
\scnsuperset{алгоритм}
\scnsuperset{ситуация}
\scnsuperset{событие}
\scnsuperset{способ решения задачи}
\scnsuperset{онтология}

\scnheader{база знаний}

\scnheader{модель представления знаний}

\scnheader{язык представления знаний}
\scnsubdividing{универсальный язык представления знаний;специализированный язык представления знаний}
\scnhaselement{CycL}
\scnhaselement{IDEF5}
\scnaddlevel{1}
\scnidtf{Integrated Definitions for Ontology Description Capture Method}
\scnaddlevel{-1}
\scnhaselement{Prolog}
\scnhaselement{CLIPS}

\scnheader{онтология}
\scnidtf{эксплицитная спецификация концептуализации}
\scnidtf{формальная спецификация предметной области, включающая описания классов объектов исследования и отношений, заданных на объектах исследования}
\scnsuperset{онтология представления}
\scnsuperset{онтология верхнего уровня}
\scnsuperset{онтология предметной области}
\scnsuperset{прикладная онтология}
\scnsuperset{тезаурус}
\scnauthorcomment{есть много классификаций онтологий, не знаю, стоит ли приводить}

\scnheader{онтология верхнего уровня}
\scnexplanation{В онтологии верхнего уровня представлена систематизация знаний о реальном мире безотносительно к
какой-либо конкретной предметной области. Основной функцией, которая возлагалась на онтологии верхнего уровня, является поддержка семантической совместимости онтологий предметных областей и прикладных онтологий. Поддержка предполагает создание общей точки для формулирования определений. Термины предметно-ориентированных онтологий подчинены терминам онтологии более высокого уровня}
\scnhaselement{OpenCyc}
\scnhaselement{SUMO}
\scnhaselement{DOLCE}
\scnhaselement{Онтология Джона Совы}
\scnhaselement{WordNet}
\scnaddlevel{1}
\scniselement{тезаурус}
\scnaddlevel{-1}

\scnheader{язык представления знаний}
\scnsuperset{язык описания онтологий}
\scnaddlevel{1}
\scnhaselement{OWL}
\scnaddlevel{1}
\scnidtf{Ontology Web Language}
\scnaddlevel{-1}
\scnhaselement{OWL 2}
\scnaddlevel{-1}

\scnendstruct

\end{SCn}