\begin{SCn}

\scnsectionheader{Предметная область и онтология интерфейсов компьютерных систем}

\scnstartsubstruct

\scnheader{Предметная область интерфейсов компьютерных систем}
\scnsdmainclasssingle{интерфейс компьютерной системы}
\scnsdclass{***}
\scnsdrelation{***}

\scnheader{интерфейс компьютерной системы}
\scnsuperset{пользовательский интерфейс}
\scnaddlevel{1}
\scnsuperset{естественно-языковой интерфейс}
\scnaddlevel{1}
\scnsuperset{речевой интерфейс}
\scnaddlevel{-1}
\scnaddlevel{-1}


\scnheader{пользовательский интерфейс}
	\scnexplanation{Одним из наиболее важных компонентов компьютерной системы, обеспечивающим обмен информацией между пользователем и компьютерной системой, является пользовательский интерфейс. Он является совокупностью аппаратных и программных средств, обеспечивающих взаимодействия пользователя и компьютерной системы.}
}
\scnrelfromvector{пользовательский интерфейс}{
командный пользовательский интерфейс IMS;
графический пользовательский интерфейс\\
\scnaddlevel{1}
	\scntext{SILK-интерфейс}{\\
		\scntext{естественно-языковой интерфейс}{
			речевой интерфейс
		}
	}
	\scnaddlevel{-1}
}

\scnheader{командный пользовательский интерфейс IMS}
\scnexplanation{Пользовательский интерфейс, при котором обмен информацией между компьютером и пользователем осуществляется путем написания текстовых инструкций или команд, называется командным пользовательским интерфейсом.}

\scnheader{графический пользовательский интерфейс}
\scnexplanation{Если обмен информацией между копьютерной системой и пользователем осуществляеться в форме диалога, с помощью окон, меню, и других элементов управления, то такой пользовательский интерфейс называется графическим.}

\scnheader{SILK-интерфейс}
\scnexplanation{Для диалога между пользователем и системой может использоваться человеческая речь, при таком общении компьютер распознаёт и анализирует речь пользователя, так же результат выполниния команд преобразуется в понятную человеку форму,например в речь или изображения,тогда такой интерфейс называется SILK-интерфейс.}

\scnheader{естественно-языковой интерфейс}
			\scnexplanation{Если обмен информации происходит за счёт диалога, когда компьютер и пользователь общаются по средствам речи, то такой интерфейс называется естественно-языковым}



\scnendstruct

\end{SCn}