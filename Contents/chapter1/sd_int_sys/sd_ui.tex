\begin{SCn}

\scnsectionheader{Предметная область и онтология интерфейсов компьютерных систем}

\scnstartsubstruct

\scnheader{Предметная область интерфейсов компьютерных систем}
\scnsdmainclasssingle{интерфейс компьютерной системы}
\scnsdclass{***}
\scnsdrelation{***}

\scnheader{интерфейс компьютерной системы}
\scnsuperset{пользовательский интерфейс}
\scnaddlevel{1}
\scnsuperset{естественно-языковой интерфейс}
\scnaddlevel{1}
\scnsuperset{речевой интерфейс}
\scnaddlevel{-1}
\scnaddlevel{-1}


\scnheader{пользовательский интерфейс}
	\scntext{значение}{набор программных и аппаратных средств, обеспечивающих взаимодействие пользователя с компьютером.}
	\scntext{предназначение}{обмен информацией между пользователем и компьютерной системой}
	\scntext{база пользовательсокого интерфейса}{
\begin{scnitemize}
\item Описание процессов, относящихся к прошлому, настоящему и будущему пользовательского 				интерфейса.Под прошлым пользовательского интерфейса подразумеваются история его 					эксплуатации, а также эволюция интерфейса, под настоящим - текущее состояние 						пользовательского интерфейса, под будущим - планы развития пользовательского 						интерфейса.Анализ изложенных временных процессов позволяет делать оценки эффективности 				развития интерфейса и обеспечивает версионность при проектировании пользовательских 				интерфейсов.
\item Модели пользователей, содержащие информацию об их особенностях, возможностях и 						предпочтениях, что в свою очередь позволяет интерфейсу быть гибким и адаптироваться к 				пользователю, обеспечивая максимально эффективное взаимодействие.
\item Типология действий пользователей и ostis-систем, позволяющая
		описать принципы организации взаимодействия пользовательского интерфейса с пользователями 			на всех уровнях интерфейсного взаимодействия.
\item Типология объектов этих действий, позволяющая произвести уни-
		фикацию и согласование компонентов пользовательских интерфейсов, а
		также составить их иерархию.
\item Формальное описание внешних языков представления конструкций SC-кода, как универсальных, 				так и специализированных.
\end{scnitemize}
}
}
\scnheader{Предметная область и онтология пользовательских интерфейсов}
\scnexplanation{Под проектированием пользовательских интерфейсов будет подразумеваться пользовательский интерфейс ostis-системы, поэтому все приводимые далее принципы будут характеризовать именно данный вид интерфейсов. Таким образом, sc-модель такого интерфейса строится в соответствии с общими принципами построения ostis-систем:
}
\scnexplanation{
sc-модель базы знаний пользовательского интерфейса
ostis-системы;
sc-модель машины обработки знаний пользовательского
интерфейса ostis-системы
}
\scnrelfromvector{пользовательский интерфейс}{
командный пользовательский интерфейс IMS;\\
графический пользовательский интерфейс\\
\scnaddlevel{1}
	\scntext{SILK-интерфейс}{\\
	\scnaddlevel{1}
		\scntext{естественно-языковой интерфейс}{
		речевой интерфейс
		}
		\scnaddlevel{-1}
	}
	\scnaddlevel{-1}
}


\scnendstruct

\end{SCn}