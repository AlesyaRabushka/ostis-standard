\begin{SCn}

\scnsectionheader{Предметная область и онтология интерфейсов компьютерных систем}

\scnstartsubstruct

\scnheader{Предметная область интерфейсов компьютерных систем}
\scnsdmainclasssingle{интерфейс компьютерной системы}
\scnsdclass{***}
\scnsdrelation{***}

\scnheader{интерфейс компьютерной системы}
\scnsuperset{пользовательский интерфейс}
\scnaddlevel{1}
\scnsuperset{командный интерфейс}
\scnsuperset{графический интерфейс}
\scnaddlevel{1}
\scnsuperset{WIMP-интерфейс}
\scnaddlevel{-1}
\scnsuperset{SILK-интерфейс}
\scnaddlevel{1}
\scnsuperset{естественно-языковой интерфейс}
\scnaddlevel{1}
\scnsuperset{речевой интерфейс}
\scnaddlevel{-3}


\scnheader{пользовательский интерфейс}
	\scnexplanation{Одним из наиболее важных компонентов компьютерной системы, обеспечивающим обмен информацией между пользователем и компьютерной системой, является пользовательский интерфейс. Он является совокупностью аппаратных и программных средств, обеспечивающих взаимодействие пользователя и компьютерной системы.}
}

\scnheader{командный пользовательский интерфейс	}
\scnexplanation{Пользовательский интерфейс, при котором обмен информацией между компьютером и пользователем осуществляется путем написания текстовых инструкций или команд.}

\scnheader{графический пользовательский интерфейс}
\scnexplanation{Пользовательский интерфейс компьютерных систем, при котором обмен информацией между компьютерной системой и пользователем осуществляеться при помощи графических компонентов системы.}

\scnheader{WIMP-интерфейс}
\scnexplanation{Пользовательский интерфейс компьютерных систем, при котором обмен информацией между компьютерной системой и пользователем осуществляеться в форме диалога, с помощью окон, меню, и других элементов управления.}

\scnheader{SILK-интерфейс}
\scnidtf{(Speech – речь, Image – образ, Language – язык, Knowledge – знание)}
\scnexplanation{Этот вид интерфейса наиболее приближен к обычной, человеческой форме общения. При этом компьютер находит для себя команды, анализируя человеческую речь и находя в ней ключевые фразы. Результат выполнения команд он также преобразует в понятную человеку форму, например в языковую форму или изображение.}

\scnheader{естественно-языковой интерфейс}
\scnexplanation{Обмен информации происходит за счёт диалога между компьютерной системой и пользователем. Диалог ведётся на одном из естественных языков.}

\scnheader{речевой интерфейс}
\scnexplanation{Обмен информации происходит за счёт диалога, в процессе которого компьютер и пользователь общаются с помощью речи. Данный вид интерефейса наиболее приближен к естственному общению между людьми.}

\scnheader{пользовательский интерфейс компьютерных систем}
\scntext{проблемы разработки}{Несмотря на то, что в настоящее время существует большое число вариантов пользовательских интерфейсов, есть необходимость снижения накладных расходов и сроков их разработки, также необходимо устранить невозможность их адаптации под особенности конкретного пользователя. Поскольку пользователь любой системы общается с ней посредством интерфейса, то проблемы, связанные с интерфейсом, часто формируют негативное мнение о всей системе в целом и не позволяют в полной мере использовать ее функционал. Наиболее значимые недостатки современных пользовательских интерфейсов:
\begin{scnitemize}
\item сложность интерфейса компьютерных систем различного рода\\
	\scnaddlevel{1}
	\scntext{следствие}{Увеличенные затраты времени на обучение использованию таких интерфейсов и изучение дополнительных материалов.}
	\scnaddlevel{-1}
\item велики сроки разработки и затраты на проектирование и поддержку пользовательских интерфейсов\\
	\scnaddlevel{1}
	\scntext{следствие}{Усложнение процесса совершенствования пользовательских интерфейсов, что приводит к их быстрому моральному устареванию.}
	\scnaddlevel{-1}
\item отсутствие унификации в принципах построения пользовательских интерфейсов\\
	\scnaddlevel{1}
	\scntext{следствие}{Затрудняет возможность распараллеливания процесса проектирования пользовательских интерфейсов, а также ограничивает возможность повторного использования уже разработанных компонентов.}
	\scnaddlevel{-1}
\item затруднена возможность переноса пользовательских интерфейсов с одной платформы реализации на другую\\
	\scnaddlevel{1}
	\scntext{следствие}{Из-за больших отличий между различными пользовательскими интерфейсами увеличиваются сроки переобучения пользователей на работу с новым интерфейсом.}
	\scnaddlevel{-1}
\end{scnitemize}

}

\scnheader{пользовательский интерфейс компьютерных систем}
\scntext{принципы построения}{Для построения качетсвенного интерфейса при его построении необходимо придерживаться следующих принципов разработки интерфейсов:\\
\begin{scnitemize}
\item Принцип структуризации\\
	\scnaddlevel{1}
	\scnexplanation{Пользовательский интерфейс должен быть целесообразно структурирован. Родственные его части должны быть связаны, а независимые — разделены; похожие элементы должны выглядеть похоже, а непохожие — различаться.}
	\scnaddlevel{-1}
\item Принцип простоты\\
	\scnaddlevel{1}
	\scnexplanation{Наиболее распространенные операции должны выполняться максимально просто. При этом должны быть ясные ссылки на более сложные процедуры.}
	\scnaddlevel{-1}
\item Принцип видимости\\
	\scnaddlevel{1}
	\scnexplanation{Все функции и данные, необходимые для выполнения определенной задачи, должны быть видны, когда пользователь пытается ее выполнить.}
	\scnaddlevel{-1}
\item Принцип обратной связи\\
	\scnaddlevel{1}
	\scnexplanation{Пользователь должен получать сообщения о действиях системы и о важных событиях внутри нее. Сообщения должны быть краткими, однозначными и написанными на языке, понятном пользователю.}
	\scnaddlevel{-1}
\item Принцип толерантности\\
	\scnaddlevel{1}
	\scnexplanation{Интерфейс должен быть гибким и терпимым к ошибкам пользователя. Ущерб от ошибок должен снижаться за счет возможности отмены и повтора действий и за счет разумной интерпретации любых разумных действий и данных.}
	\scnaddlevel{-1}
\item Принцип повторного использования\\
	\scnaddlevel{1}
	\scnexplanation{Интерфейс должен многократно использовать внутренние и внешние компоненты, достигая тем самым унифицированности.}
	\scnaddlevel{-1}
\end{scnitemize}
}

\scnheader{пользовательский интерфейс компьютерных систем}
\scntext{критерии качества}{Для оценки качества разработанных интерфейсов применяются следующие критерии:\\
\begin{scnitemize}
\item Скорость работы пользователя\\
	\scnaddlevel{1}
	\scnexplanation{Пользовательский интерфейс должен быть максимально простым и понятным пользователю. Все компоненты пользовательского интерфейса должны помогать в принятии решений, а не усложнять их.}
	\scnaddlevel{-1}
\item Количество человеческих ошибок\\
	\scnaddlevel{1}
	\scnexplanation{Пользовательский интерфейс должен содержать элементы, которые позволят уменьшить количество допускаемых ошибок. Кроме того, пользовательский интерфейс должен содержать средства, позволяющие снизить чувствительность системы к ошибкам.}
	\scnaddlevel{-1}
\item Скорость обучения\\
	\scnaddlevel{1}
	\scnexplanation{Пользовательский интерфейс должен содержать средства, позволяющие пользователю в максимально короткие сроки научиться работать с программой или системой. К таким средствам относятся различного вида справочные системы, подсказки, информационные сообщения.}
	\scnaddlevel{-1}
\end{scnitemize}
}
\scnendstruct
\end{SCn}