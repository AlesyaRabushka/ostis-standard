\begin{SCn}

\scnsectionheader{\currentname}
\scnstartsubstruct

\scnrelfromlist{дочерний раздел}{Предметная область и онтология смыслового представления информации; Предметная область и онтология многоагентных моделей решения задач, основанных на смысловом представлении информации;Предметная область и онтология семантических моделей интерфейсов компьютерных систем, основанных на смысловом представлении информации}
\scntext{аннотация}{Данный раздел и дочерние ему разделы являются уточнением и обоснованием наших предложений, направленных на построение компьютерных систем следующего поколения, основанных на смысловом представлении обрабатываемой информации}
\scntext{основной тезис}{Для \uline{любой} \textit{компьютерной системы} можно построить эквивалентную ей логико-семантическую модель, основанную на смысловом представлении обрабатываемой информации}
\scnheader{логико-семантическая модель компьютерной системы}
\scnexplanation{Главным фактором обеспечения совместимости различных видов знаний, различных моделей решения задач и различных компьютерных систем в целом является 
\begin{scnitemize}
    \item унификация (стандартизация) представления информации в памяти компьютерных систем;
    \item унификация принципов организации обработки информации в памяти компьютерных систем.
\end{scnitemize}

Унификация представления информации, используемой в компьютерных системах, предполагает:
\begin{scnitemize}
    \item синтаксическую унификацию используемой информации – унификацию формы представления (кодирования) этой информации. При этом следует отличать:
    \begin{scnitemizeii}
    	\item кодирование информации в памяти компьютерной системы (внутреннее представление информации);
    	\item внешнее представление информации, обеспечивающее однозначность интерпретации (понимания, трактовки) этой информации разными пользователями и разными компьютерными системами;
    \end{scnitemizeii}
    \item семантическую унификацию используемой информации, в основе которой лежит согласование и точная спецификация всех (!) используемых понятий (концептов) с помощью иерархической системы формальных онтологий.
\end{scnitemize}}

\scnresetlevel
\scnheader{стандарт}
\scnhaselement{Стандарт OSTIS}
\scnaddlevel{1}
\scnidtf{Предлагаемый нами стандарт логико-семантических моделей компьютерных систем,  основанных на смысловом представлении информации, и технологии разработки таких моделей и соответствующих компьютерных систем}
\scnaddlevel{-1}
\scnidtf{знания о структуре и принципах функционирования искусственных систем соответствующего класса}
\scnidtf{онтология искусственных систем некоторого класса}
\scnidtf{теория искусственных систем некоторого класса}
\scnexplanation{Важно отметить, что грамотная унификация (стандартизация) должна не ограничивать творческую свободу разработчика, а гарантировать \textit{\uline{\textbf{совместимость}}} его результатов с результатами других разработчиков. Подчеркнем также, что текущая версия любого \textit{\textbf{стандарта}} -- это не догма, а только опора для дальнейшего его совершенствования.

Целью качественного \textit{стандарта} является не только обеспечения совместимости технических решений, но и минимизация дублирования (повторения) таких решений. Один из важных критериев качества \textit{стандарта} -- ничего лишнего.

\textit{Стандарты}, как и другие важные для человечества \textit{знания}, должны быть формализованы и должны постоянно совершенствоваться с помощью специальных \textit{интеллектуальных компьютерных систем}, поддерживающих процесс эволюции стандартов путем согласования различных точек зрения.}
\scnaddlevel{-1}

\end{SCn}