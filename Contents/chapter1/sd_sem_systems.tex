\begin{SCn}

\scnsectionheader{\currentname}
\scnrelfromlist{подраздел}{Предметная область и онтология семантических сетей, семантических языков и семантических моделей баз знаний; Предметная область и онтология агентно-ориентированных семантических моделей решателей задач;Предметная область и онтология семантических моделей интерфейсов компьютерных систем}

\scnstartsubstruct

\scnheader{Предметная область и онтология семантических моделей компьютерных систем}
\scnsdmainclasssingle{семантическая модель компьютерной системы}
\scnsdclass{стандарт}
\scnsdrelation{***}

\scnheader{семантическая модель компьютерной системы}
\scnexplanation{Главным фактором обеспечения совместимости различных видов знаний, различных моделей решения задач и различных компьютерных систем в целом является 
\begin{scnitemize}
    \item унификация (стандартизация) представления информации в памяти компьютерных систем;
    \item унификация принципов организации обработки информации в памяти компьютерных систем.
\end{scnitemize}

Унификация представления информации, используемой в компьютерных системах, предполагает:
\begin{scnitemize}
    \item синтаксическую унификацию используемой информации – унификацию формы представления (кодирования) этой информации. При этом следует отличать:
    \begin{scnitemizeii}
    	\item кодирование информации в памяти компьютерной системы (внутреннее представление информации);
    	\item внешнее представление информации, обеспечивающее однозначность интерпретации (понимания, трактовки) этой информации разными пользователями и разными компьютерными системами;
    \end{scnitemizeii}
    \item семантическую унификацию используемой информации в основе которой лежит согласование и точная спецификация всех (!) используемых понятий (концептов) с помощью иерархической системы формальных онтологий.
\end{scnitemize}}

\scnresetlevel
\scnheader{стандарт}
\scnidtf{знания о структуре и принципах функционирования искусственных систем соответствующего класса}
\scnidtf{онтология искусственных систем некоторого класса}
\scnidtf{теория искусственных систем некоторого класса}
\scnexplanation{Важно отметить, что грамотная унификация (стандартизация) должна не ограничивать творческую свободу разработчика, а гарантировать \textbf{совместимость} его результатов с результатами других разработчиков. Подчеркнем также, что текущая версия любого \textbf{стандарта} -- это не догма, а только опора для дальнейшего его совершенствования.

Целью качественного стандарта является не только обеспечения совместимости технических решений, но и минимализация дублирования (повторения) таких решений. Один из важных критериев качества стандарта -- ничего лишнего.

Стандарты, как и другие важные для человечества знания, должны быть формализованы и должны постоянно совершенствоваться с помощью специальных интеллектуальных компьютерных систем, поддерживающих процесс эволюции стандартов путем согласования различных точек зрения.}
\scnsuperset{стандарт семантических моделей компьютерных систем}
\scnaddlevel{1}
\scntext{примечание}{Предлагаемый нами \textit{стандарт семантических моделей компьютерных систем} представлен разделом \textit{Предметная область и онтология семантических моделей ostis-систем}.}
\scnaddlevel{-1}

\scnendstruct

\end{SCn}