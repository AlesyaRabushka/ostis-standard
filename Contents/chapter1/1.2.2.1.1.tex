\scnheader{модель представления знаний}
\scnsubdividing{продукционная модель;логическая модель;фреймовая модель;семантическая сеть}
\scntext{примечание}{На сегодняшний день существуют десятки моделей представления знаний, однако большинство из них базируются на основных четырех моделях, представленных выше.}
\scntext{примечание}{Отдельное внимание следует уделить рассмотрению средств для представления знаний, предлагаемых в рамках направления \textit{Semantic Web}, по причине их проработанности и распространенности.}


\scnheader{продукционная модель}
\scnexplanation{\textit{продукционная модель} является системой продукций, представляющих собой конструкции типа "Если (условие), то (действие)"{}.}
\scntext{примечание}{Продукционные модели удобны для представления логических взаимосвязей между фактами. Чаще всего данные модели используются для представления знаний в экспертных системах.}


\scnheader{логическая модель}
\scnexplanation{\textit{логическая модель} основывается на классическом исчислении предикатов и его расширениях, позволяет описывать свойства предметной области в виде набора аксиом и правил вывода.}
\scntext{примечание}{\textit{Логические модели}, как и \textit{продукционные модели}, удобны для представления логических взаимосвязей между фактами. Их отличительной особенностью являются строгость и формализованность.}


\scnheader{cемантическая сеть}
\scnexplanation{\textit{семантическая сеть} -- модель представления знаний в виде графовой структуры, вершинами которой являются информационные единицы, а дуги обозначают связи между ними.}
\scntext{достоинство}{Главной особенностью семантических сетей является соединение в одном представлении синтаксического и семантического аспектов описаний знаний предметной области, что значительно снижает вычислительную сложность обработки знаний}
\scntext{примечание}{\textit{Семантическая сеть} является весьма распространенным способом представления знаний в интеллектуальных системах.}


\scnheader{фреймовая модель}
\scnexplanation{\textit{Фреймовые модели} представляют собой системы взаимосвязанных фреймов.}
	\scnaddlevel{1}
	\scntext{примечание}{Под фреймом объекта или явления понимается его минимальное описание, содержащее всю существенную информацию об этом объекте или явлении и обладающее таким свойством, что удаление из описания любой его части приводит к потере существенной информации, без которой описание объекта или явления не может быть достаточным для их идентификации. Фрейм задается именем и набором слотов, описывающих свойства объекта или явления.}
		\scnaddlevel{1}
		\scnrelfrom{автор}{Минский М.}
		\scnaddlevel{-1}	
	\scnaddlevel{-1}


\scnheader{средства Semantic Web}
\scnexplanation{\textit{cредства Semantic Web} представляют собой набор методов и технологий, предназначенных для представления информации в виде, пригодном для машинной обработки.}
\scntext{примечание}{Информация представляется в виде \textit{семантической сети}, специфицируемой посредством \textit{онтологий}. Стандартизация представления информации позволяет компьютерной системе получать различную фактографическую информацию и делать на ее основе логические заключения.}
\scnrelfromset{основные инструменты}{модель описания ресурсов RDF;языки, обеспечивающие представление RDF-данных;средства представления метаданных RDF Schema;принципы представления знаний в виде онтологий;языки описания онтологий\\
	\scnaddlevel{1}
	\scnhaselement{OWL Lite}
	\scnhaselement{OWL DL}
	\scnhaselement{OWL Full}
	\scnaddlevel{1}
	\scntext{примечание}{Онтологии, представленные с помощью OWL, включают описание классов, их свойств, обеспечивающих связи между классами, и экземпляров этих классов.}
	\scntext{примечание}{Практика позволила выявить ограниченность выразительных способностей OWL и недостатки технического характера:
	\begin{scnitemize}
	\item сложность синтаксического разбора;
	\item невозможность обнаружить опечатки в именах.
	\end{scnitemize} 
Это привело к созданию новой версии языка — \textit{OWL 2}, целью создания которого являлось устранение указанных недостатков}
	\scnaddlevel{1}
	\scntext{примечание}{В настоящий момент OWL 2 является общепризнанным стандартом для представления онтологий.}
	\scnaddlevel{-1}
	\scnaddlevel{-1}
	\scnaddlevel{-1}
;хранилища баз знаний на основе RDF\\
	\scnaddlevel{1}		
	\scnhaselement{Sesame}
	\scnhaselement{HyperGraphDB}
	\scnhaselement{Neo4j}
	\scnhaselement{Virtuoso}
	\scnhaselement{AllegroGraph}
	\scnaddlevel{1}
	\scntext{примечание}{Данные хранилища обеспечивают хранение и доступ к данным средствами языка запросов SPARQL}
	\scnaddlevel{-1}
	\scnaddlevel{-1}
;язык запросов к хранилищам RDF-данных SPARQL}
\scntext{недостатки}{К недостаткам cредств представления знаний, предлагаемых в рамках подхода Semantic Web можно отнести:
	\begin{scnitemize}
	\item OWL запрещает наличие дуг, инцидентных другим дугам, что вынуждает вводить дополнительную вершину, обозначающую связку, и далее связывать с ней все необходимые вершины соответствующими отношениями.Такой подход имеет ряд недостатков, в частности, приводит к необходимости:
	\begin{scnitemizeii}
	\item вводить новые отношения, которые связывают такую дополнительную вершину с остальными;
	\item преобразовывать указанным образом все связки рассматриваемого отношения, в противном случае связки одного и того же отношения в разных конструкциях будут представлены по-разному, что сильно затруднит обработку представленных таким образом знаний.
	\end{scnitemizeii}	
	 \item отсутствие возможности описания метасвязей;
	 \item отсутствие средств описания нечетких знаний;
	 \item отсутствие возможности описания свойств целых классов сущностей;
	 \item невозможность описания исключений из некоторых правил;
	 \item возможность задания метасвязей только для отдельных связей;
	 \item и др.
	\end{scnitemize}}
\scntext{примечание}{Таким образом, подход к созданию баз знаний на основе Semantic Web предоставляет средства формального представления знаний и доступа к ним, которые, однако, не позволяют в унифицированном виде представлять все виды знаний, необходимые для функционирования современных интеллектуальных систем.}

\scnheader{язык представления знаний}
\scntext{примечание}{Каждой \textit{модели представления знаний} соответствует некоторое множество \textit{языков представления знаний}, реализующих эти модели.}
\scnexplanation{В языках представления знаний, как правило, разделяется синтаксическая и семантическая составляющие. Синтаксис задает правила, по которым строятся конструкции данного языка, а семантика определяет правила интерпретации указанных конструкций.}
\scnhaselement{CycL}
	\scnaddlevel{1}
	\scnexplanation{\textit{CycL} -- язык представления знаний, основанный на онтологиях и используемый в рамках проекта Cyc.}
	\scnaddlevel{-1}
\scnhaselement{IDEF5}
	\scnaddlevel{1}
	\scnidtf{Integrated  Definitions  for  Ontology  Description  Capture  Method}
	\scnexplanation{\textit{IDEF5} -- стандарт онтологического исследования для наглядного представления данных, полученных в результате обработки онтологических запросов в простой естественной графической форме.}
	\scnaddlevel{-1}
\scnhaselement{Prolog}
	\scnaddlevel{1}
	\scnexplanation{\textit{Prolog} -- язык, основанный на языке предикатов математической логики дизъюнктов Хорна, представляющей собой подмножество логики предикатов первого порядка}
	\scnaddlevel{-1}
\scnhaselement{CLIPS}
	\scnaddlevel{1}
	\scnidtf{C Language Integrated Production System}
	\scnexplanation{\textit{CLIPS} -- язык представления знаний, основанный на логических правилах, использующийся одноименной программной оболочкой для создания экспертных систем}
	\scnaddlevel{-1}}


\scnheader{знание}
\scnexplanation{Под \textit{знаниями} в широком смысле понимается совокупность сведений, которые формируют целостное описание некоторого объекта, явления или проблемы}
		\scnaddlevel{1}
		\scnrelfrom{автор}{Аверкин А.Н.}
		\scnaddlevel{-1}
\scnexplanation{Знания определяется как хорошо структурированные данные, или данные о данных (т. е. метаданные).}
		\scnaddlevel{1}
		\scnrelfrom{автор}{Гаврилова Т.А.}
		\scnaddlevel{-1}
\scnnote{Понятие \textit{знаний} тесно связано с понятием \textit{предметной области}.}


\scnheader{предметная область}
\scnexplanation{Под \textit{предметной областью} в инженерии знаний понимают совокупность реальных или абстрактных объектов (сущностей), связей и отношений между этими объектами, а также процедур преобразования этих объектов для решения задач, возникающих в предметной области.}
		\scnaddlevel{1}
		\scnrelfrom{автор}{Аверкин А.Н.}
		\scnaddlevel{-1}
		
\scnheader{знание}
\scnexplanation{\textit{Знания} о некоторой \textit{предметной области} представляют собой совокупность сведений об объектах этой \textit{предметной области}, их существенных свойствах и связывающих их отношениях, процессах, протекающих в данной предметной области, а также методах анализа возникающих в ней ситуаций и способах разрешения ассоциируемых с ними проблем.}
		\scnaddlevel{1}
		\scnrelfrom{автор}{Гаврилова Т.А.}
		\scnaddlevel{-1}
\scnsubdividing{декларативное знание\\
	\scnaddlevel{1}
	\scnexplanation{Под \textit{декларативными знаниями} понимаются знания, которые записаны в памяти интеллектуальной системы так, что они непосредственно доступны для использования после обращения к соответствующему полю памяти.}
	\scntext{примечание}{В виде \textit{декларативных знаний} обычно записывается информация о свойствах предметной области, фактах, имеющих в ней место и тому подобная информация.}
	\scnaddlevel{-1}
	;процедурное знание\\
	\scnaddlevel{1}
	\scnexplanation{\textit{процедурные знания} – это знания, которые хранятся в памяти интеллектуальной системы в виде описаний процедур, с помощью которых их можно получить.}
	\scntext{примечание}{В виде \textit{процедурных знаний} обычно описываются информация о предметной области, характеризующая способы решения задач в этой области, а также различные инструкции, методики и тому подобная информация.}
	\scnaddlevel{-1}}
\scntext{примечание}{В инженерии знаний известны следующие признаки классификации знаний:
\begin{scnitemize}
  \item по глубине;
  \item по владельцу;
  \item по форме;
  \item по источнику получения;
  \item по сфере применения.
\end{scnitemize}}
\scnnote{База знаний чаще всего содержит три уровня \textit{знаний}:
\begin{scnitemize}
  \item общие, или абстрактные \textit{знания}, которые описывают закономерности, общие для большого числа \textit{предметных областей} (знания о теоретико-множественных связях, знания о терминах, знания о логических моделях предметных областей, знания о базовых математических отношениях и операциях и др.);
  \item \textit{знания} о конкретной \textit{предметной области} (domain-specific knowledge). Например, знания по геометрии, истории, медицине и др.;
  \item конкретные \textit{знания}, добавляемые в \textit{базу знаний} пользователями или программными агентами.
\end{scnitemize}}

\scnheader{структуризация базы знаний}
\scnexplanation{\textbf{\textit{cтруктуризация базы знаний}} -- выделение в базе различных связанных между собой фрагментов.}
\scnnote{\textit{Структуризация базы знаний} необходима по следующим причинам:
	\begin{scnitemize}   
	\item для повышения эффективности обработки баз знаний путем указания областей решения задач;   
	\item для выделения независимых фрагментов базы знаний с целью организации распределения работ по проектированию (когда разным исполнителям поручается разработка разных фрагментов базы знаний, имеющих достаточно четкие границы);   
	\item для дидактических целей (человеку, усваивающему некоторые знания, желательно иметь своего рода оглавление этих знаний, что позволяет планировать их усвоение и рассматривать их с различной степенью детализации), что является немаловажным фактором при работе с системами, основанных на знаниях. 
	\end{itemize}}
\scnrelfrom{цель}{декомпозиция базы знаний на множество фрагментов, связанных друг с другом тем или иным набором отношений}
\scnrelfromset{принципы, лежащие в основе}{
\scnfileitem{В основе методов структурирования информации традиционно используется \textit{иерархический подход} как методологический прием разделения формально описанной системы на уровни (блоки, или модули).}
	\scnaddlevel{1}
	\scnnote{На верхних уровнях иерархии представляются описания наименьшей степени детализации, которые отражают общие особенности предметной области (или системы), на следующих уровнях степень детализации описания увеличивается, при этом предметная область (или система) рассматривается не целиком, а отдельными частями.}  
	\scntext{следовательно}{Преимуществом данного подхода является сведение исходной задачи к подзадачам, которые должны быть решены в рамках предметной области (или системы)}
	\scnaddlevel{-1}
;\scnfileitem{В качестве одной из исторически первых моделей, применимых для структуризации знаний, также рассматривается модель клубных систем}
	\scnaddlevel{1}
	\scnnote{Данная модель позволяет выделять отдельные фрагменты заданного множества и рассматривать отношения между ними, и далее при необходимости осуществлять аналогичные действия уже с выделенными фрагментами, опускаясь таким образом на необходимый уровень детализации.}
	\scnaddlevel{1}
	\scntext{следовательно}{Такая модель позволяет, с одной стороны, специфицировать целые фрагменты исследуемого множества, рассматривая их как отдельные сущности, с другой стороны – абстрагироваться от детального рассмотрения тех фрагментов, для которых в текущий момент этого не требуется.}
	\scnaddlevel{-2}
;\scnfileitem{Подход к структуризации знаний, основанный на методологии объектно-структурного анализа}
	\scnaddlevel{1}
	\scnnote{Данный подход основывается на разбиении предметной области на восемь слоев (страт) в зависимости от вида знаний, рассматриваемого на том или ином слое}
	\scnaddlevel{-1}
;\scnfileitem{Cтратифицированная фрактальная модель (или ФС-модели)}
	\scnaddlevel{1}
	\scnnote{В основе данного подхода лежит концептуальная модель структурирования знаний, базирующаяся на представлении различных видов знаний как объектов расслоенного пространства}
	\scnnote{Графически ФС-модель представляется в виде совокупности вложенных сферических оболочек. Точка на одной из сфер, условно обозначающая информационный объект, в свою очередь может быть расслоена при необходимости более детального рассмотрения данного объекта.}
	\scnaddlevel{1}
	\scntext{следовательно}{ФС-модель определяется как совокупность непересекающихся слоев (информационных миров) и их отображений в информационном пространстве. Каждому уровню соответствует свой слой этого пространства и, следовательно, свой информационный мир.}
	\scnaddlevel{-2}}
\scnnote{Актуальной остается задача обеспечения возможности использования различных подходов к структуризации в рамках одной базы знаний одновременно.}	
\scnnote{На сегодняшний день наиболее эффективным средством формализации и структуризации различных областей знаний являются онтологии.}

\scnheader{онтология}
\scnexplanation{\textit{\textbf{онтология}} трактуется как эксплицитная спецификация концептуализации}
		\scnaddlevel{1}
		\scnrelfrom{автор}{Грубер Т.}
		\scnaddlevel{-1}
\scnnote{Применительно к интеллектуальным системам под \textit{онтологией} понимается формальная спецификация \textit{предметной области}, включающая описания классов объектов исследования и отношений, заданных на объектах исследования}
\scnnote{В качестве признаков классификации онтологий используются такие признаки, как:
	\begin{scnitemize}
	\item цель создания;
	\item степень формальности;
	\item содержимое.
	\end{scnitemize}}
\scnrelfrom{разбиение}{Классификация онтологий по цели создания}
\scnaddlevel{1}
\scneqtoset{онтология представления\\
	\scnaddlevel{1}
	\scntext{назначение}{
	\begin{scnitemize}
	\item описание области представления знаний;
	\item создание языка для спецификации онтологий более низких уровней.
	\end{scnitemize}}
	\scnaddlevel{-1}
;онтология верхнего уровня\\
	\scnaddlevel{1}
	\scntext{назначение}{Создание онтологий для общих предметных областей, свойства которых исследуются онтологиями более низкого уровня.}
	\scnnote{Онтологии верхнего уровня могут быть повторно используемы вместе с соответствующими им онтологиями более низкого уровня.}
	\scnaddlevel{-1}
;онтология предметной области\\
	\scnaddlevel{1}
	\scnnote{Область охвата данных онтологий ограничена одной предметной областью.}
	\scntext{назначение}{Обобщение понятия, использующиеся в некоторых задачах домена, абстрагируясь от самих задач.}
	\scnnote{Данный вид онтологии повторно используемы внутри одной предметной области}
	\scnaddlevel{-1}
;прикладная онтология\\
	\scnaddlevel{1}
	\scntext{назначение}{Описание концептуальной модели конкретной задачи или приложения.}
	\scnnote{Данный вид онтологии нет возможности использовать повторно.}
	\scnaddlevel{-1}}
\scnaddlevel{-1}
\scnsubdividing{неформальная онтология\\
	\scnaddlevel{1}
	\scnidtf{онтология, описываемая на естественном языке}
	\scnaddlevel{-1}
;более формализованная онтология\\
	\scnaddlevel{1}
	\scnidtf{онтология, основанная на отношениях таксономии}
	\scnaddlevel{-1}
;сильно формализованная онтология\\
	\scnaddlevel{1}
	\scnidtf{онтология, которая  задает формальную семантику понятий в разрешенных языком точных и непротиворечивых выражениях}
	\scnaddlevel{-1}}
\scnsubdividing{общая онтология\\
	\scnaddlevel{1}
	\scnidtf{онтология, описывающая сущности, события, пространство, время}
	\scnaddlevel{-1}
;онтология задач\\
	\scnaddlevel{1}
	\scnidtf{онтология, описывающая типологию классов задач и их спецификацию}
	\scnaddlevel{-1}
;предметная онтология\\
	\scnaddlevel{1}
	\scnidtf{онтология, описывающая множества предметов}
	\scnaddlevel{-1}}	
\scnsubdividing{простая онтология;многоуровневая онтология}
\scnsubdividing{легкая онтология;весомая онтология}
\scnsubdividing{статическая онтология;динамическая онтология}
\scnrelfrom{разбиение}{Классификация онтологий в зависимости от набора используемых отношений}
\scnaddlevel{1}
\scneqtoset{словарь понятий\\
	\scnaddlevel{1}
	\scnexplanation{\textit{словарь понятий} -- явно определяется смысл терминов словаря с помощью соответствующей функции интерпретации.}
	\scnaddlevel{-1}
;пассивный словарь\\
	\scnaddlevel{1}
	\scnexplanation{\textit{пассивный словарь} -- включает множество интерпретируемых понятий предметной области, множество интерпретирующих терминов и соответствующие им функции интерпретации.}
	\scnaddlevel{-1}
;таксономия понятий\\
	\scnaddlevel{1}
	\scnexplanation{\textit{таксономия понятий} -- описывает отношения типа "is a"{} между понятиями предметной области.}
	\scnaddlevel{-1}
;мерономия понятий\\
	\scnaddlevel{1}
	\scnexplanation{\textit{мерономия понятий} -- описывает отношения типа "part of"{} между понятиями предметной области.}
	\scnaddlevel{-1}
;метасистема понятий\\
	\scnaddlevel{1}
	\scnexplanation{\textit{метасистема понятий} -- описывает отношения "is a"{} и "part of"{} между понятиями предметной области.}
	\scnaddlevel{-1}
;онтология с ограничениями\\
	\scnaddlevel{1}
	\scnexplanation{\textit{онтология с ограничениями} -- описывает отношения "is a"{}, "part of"{} и другие, дополнительно уточняемые отношения между понятиями предметной области.}
	\scnaddlevel{-1}
;полная онтология\\
	\scnaddlevel{1}
	\scnexplanation{\textit{полная онтология} -- включает описания отношений "is a"{}, "part of"{} и других, дополнительно уточняемых отношений между понятиями предметной области, а также соответствующие им функции интерпретации}
	\scnaddlevel{-1}}
\scnnote{Онтологии позволяют сформировать понятийный базис рассматриваемой предметной области, что является ключевым фактором в процессе структуризации знаний. Онтологии являются основой любой базы знаний и используются для интеграции различных баз знаний и их частей.}


\scnheader{онтология верхнего уровня}
\scnexplanation{\textit{онтология верхнего уровня} ориентирована на описание фундаметальных понятий, таких, как "сущность"{}, "явление"{}, "отношение"{}, "действие"{} и др.}
\scnnote{В \textit{онтологии верхнего уровня} представлена систематизация знаний о реальном мире безотносительно к какой-либо конкретной \textit{предметной области}.}
\scntext{назначение}{Поддержка семантической совместимости онтологий предметных областей и прикладных онтологий}
	\scnaddlevel{1}
	\scnnote{Поддержка предполагает создание общей точки для формулирования определений. Термины предметно-ориентированных онтологий подчинены терминам онтологии более высокого уровня.}
	\scnaddlevel{-1}
\scnhaselement{OpenCyc}
	\scnaddlevel{1}
	\scnexplanation{\textit{OpenCyc} -- открытая для общего пользования часть коммерческого проекта Cyc, на текущий момент наиболее масштабной и детализированной онтологии в области общего знания.}
	\scnnote{Структурно \textit{база знаний OpenCyc} состоит из констант (терминов) и правил (формул), оперирующих этими константами. 
	Правила делятся на два вида: 
	\begin{scnitemize}
	\item аксиомы -- утверждения, которые были явно и вручную введены в базу знаний экспертами, а не появились там в результате работы машины вывода;
	\item выводимые утверждения.	
	\end{scnitemize} 
  Все утверждения или формулы в базе знаний OpenCyc фиксируются на языке CycL, выразительно эквивалентном исчислению предикатов первого порядка.}
	\scnaddlevel{-1}
\scnhaselement{DOLCE}
	\scnaddlevel{1}
	\scnidtf{Descriptive Ontology for Linguistic and Cognitive Engineering}
	\scnexplanation{\textit{DOLCE} -- базовая онтология проекта WonderWeb.}
	\scntext{назначение}{Обеспечение согласования между интеллектуальными агентами, использующими разную терминологию.}
	\scnnote{Онтология имеет когнитивный уклон, поскольку в основном фиксируются онтологические категории естественного языка и знания "здравого смысла"{}. Онтология не претендует на звание универсальной, стандартной или общей. 
	В основе данной онтологии лежит разделение всех сущностей на универсальные, которые могут иметь экземпляры, и индивидные (частные), которые не могут иметь экземпляры.}
	\scnaddlevel{-1}
\scnhaselement{SUMO}
	\scnaddlevel{1}
	\scnidtf{Suggested Upper Merged Ontology}
	\scnexplanation{SUMO -- онтология верхнего уровня, разработанная в рамках проекта рабочей группы IEEE Standard Upper Ontology Working Group и Teknowledge.}
	\scntext{назначение}{Содействие улучшению интероперабельности данных, извлечения и поиска информации, автоматического вывода (доказательства), обработки естественного языка.}
	\scnnote{Онтология SUMO содержит наиболее общие и самые абстрактные концепты, имеет исчерпывающую иерархию фундаментальных понятий (около 1000 понятий), а также набор аксиом (примерно 4000), определяющих эти понятия. Cоздатели SUMO предоставляют лишь информацию, которая может обрабатываться программно и включаться в качестве составной части в различные приложения и обрабатываться средствами этих приложений.}
	\scnaddlevel{-1}
\scnhaselement{WordNet}
	\scnaddlevel{1}
	\scnexplanation{WordNet -- один из наиболее полно разработанных тезаурусов общего назначения.}
	\scnnote{Центральным объектом в WordNet является синсет, множество синонимов (или синонимический ряд). WordNet содержит около 70 тыс. синсетов, организованных в иерархию по отношению "надкласс-подкласс"{}.}
	\scnreltoset{недостатки текущего состояния}{
\scnfileitem{Отсутствие отношений между частями речи}
;\scnfileitem{Различия значений в WordNet слишком тонки для компьютерных приложений}
;\scnfileitem{Нехватка отношений между синсетами, относящимися к одной и той же тематической области}}
	\scnaddlevel{-1}}
\scnaddlevel{1}
\scnnote{В целом попытки создать универсальную онтологию верхнего уровня пока не привели к ожидаемым результатам. Многие онтологии верхнего уровня содержат одни и те же понятия, однако их трактовка и принципы организации иерархии отличаются в разных онтологиях. Так, например, во всех онтологиях проводится разделение сущностей на (1) абстрактные и реально существующие, (2) на постоянные и временные сущности, (3) деление на объект и процесс.
В то же время даже на верхних уровнях наблюдаются существенные различия. В онтологии SUMO первично разделение на абстрактные и материальные сущности, а разделение на постоянные и временные – вторично. В DOLCE на верхнем уровне производится разделение на постоянные, временные, абстрактные и качественные сущности.}
\scnaddlevel{-1}
\scnnote{Онтологии верхнего уровня были призваны решить задачу обеспечения семантической совместимости представляемых знаний в базах знаний, однако отсутствие единой формальной основы, обеспечивающей однозначную интерпретацию представляемых знаний и вводимых новых понятий, не привело к решению указанной проблемы. Отсутствие удовлетворительного решения этой задачи приводит к несовместимости компонентов баз знаний, разрабатываемых для разных систем, и невозможности их повторного использования в других системах. Как следствие, имеет место многократная повторная разработка содержательно одних и тех же компонентов для разных баз знаний.}