\scnheader{способность кибернетической системы к обеспечению семантической совместимости с партнерами}
\scnidtf{способность кибернетической системы к обеспечению взаимопонимания со своими партнерами.}
\scnrelfromset{комплекс частных свойств}{способность кибернетической системы к обеспечению семантической совместимости собственной базы знаний с базами знаний своих партнеров;способность кибернетической системы к обеспечению коммуникационной совместимости со своими партнерами
\newline
\scnaddlevel{1}
    \scnnote{Речь идет о согласовании внешних языков, используемых кибернетическими системами при их общении.}
\scnaddlevel{-1}
}

\scnrelfromset{комплекс частных свойств}{
уровень предварительной семантической совместимости кибернетической системы с партнерами
\scnaddlevel{1}
    \scnnote{Речь идет об обеспечении начальной (стартовой) семантической совместимости.}
\scnaddlevel{-1};
способность кибернетической системы к перманентной поддержке семантической совместимости с партнерами
\newline
\scnaddlevel{1}
    \scnnote{Речь идет о перманентном процессе поддержки необходимого уровня семантической совместимости(взаимопонимания) в условиях постоянной эволюции всех взаимодействующих кибернетических систем.}
\scnaddlevel{-1}
}

\scnheader{уровень предварительной семантической совместимости кибернетической системы с партнерами}
\scnidtf{унификация представления информации, хранимой в памяти всевозможных кибернетических систем}
\scnidtf{максимально возможная конвергенция, стандартизация, согласованность представления информации, хранимой в памяти всевозможных кибернетических систем}
\scnnote{речь идет об использовании всеми кибернетическими системами общего универсального языка внутреннего представления знаний и о согласовании используемых ими понятий}

\scnheader{способность кибернетической системы к перманентной поддержке семантической совместимости с партнерами}
\scnidtf{способность кибернетической системы к согласованию денотационной семантики знаков (и, в первую очередь, знаков понятий), используемых в собственной базе знаний с денотационной семантике тех знаков, которые входят в состав информации поступающей от других кибернетических систем-партнеров}
\scnidtf{способность кибернетической системы к повышению уровня семантической совместимости и взаимопонимания с другими системами (в том числе, с компьютерными системами, с людьми) в условиях перманентного процесса собственной эволюции (следствием которой является появление новых знаковых понятий и других описываемых сущностей, а также уточнение денотационной семантики используемых знаков), перманентной эволюции партнерских кибернетических систем и перманентной эволюции коллективно согласованной картины мира}
\scnnote{Рассматриваемое свойство (способность) кибернетической системы заключается в \uline{самостоятельной} реализацией перманентного (постоянного) процесса обеспечения поддержки своей семантической совместимости \uline{со всеми}(!) кибернетическими системами, с которыми данная кибернетическая система взаимодействует в текущий момент времени. Подчеркнем при этом, что условия поддержки семантической совместимости постоянно меняются -- меняется состав "партнеров"{}, меняются (эволюционируют) сами "партнеры"{}, эволюционирует и сама данная кибернетическая система}

\scnheader{следует отличать*}
\scnhaselementset{cпособность кибернетической системы к обеспечению семантической совместимости с партнерами\\
\scnaddlevel{1}
    \scniselement{свойство}
    \scnrelfrom{область определения}{кибернетическая система}
\scnaddlevel{-1};
cемантическая совместимость кибернетической системы с партнерами\\
\scnaddlevel{1}
    \scniselement{свойство}
    \scnrelfrom{область определения}{множество всевозможных неориентированных пар кибернетических систем*}
    \scnaddlevel{1}
    \scnidtf{множество всевозможных сочетаний кибернетических систем по две*}
    \scnidtf{множество всевозможных двухмощных множеств кибернетических систем*}
    \scnaddlevel{-1}
    \scnidtf{степень (уровень) семантической совместимости различных пар кибернетических систем}
\scnaddlevel{-1}
}


\scnheader{коммуникабельность кибернетической системы}
\scnidtftext{часто используемый sc-идентификатор*}{коммуникабельность}
\scnidtf{способность кибернетической системы к установлению взаимовыгодных контактов с другими кибернетическими системами (в том числе, с коллективами интеллектуальных систем) путем честного выявления взаимовыгодных общих целей (интересов).}
\scnidtf{способность кибернетической системы к формированию новых партнерских связей с другими кибернетическими системами}

\scnheader{способность кибернетической системы к обсуждению и согласованию целей и планов коллективной деятельности}
\scnidtf{способность активно участвовать в коллективном (в согласовании каких-либо предложений) -- т.е. в подтверждении (признании) этих предложений, либо в их отклонении с указанием причин или предлагаемых доработок}

\scnheader{способность кибернетической системы брать на себя выполнение актуальных задач в рамках согласованных планов коллективной деятельности}
\scnnote{Данная способность кибернетической системы предполагает:
\begin{scnitemize}
    \item учет приоритета актуальных задач;
    \item учет собственных возможностей;
    \item согласование распределения актуальных задач по исполнителям;
    \item публикацию момента начала и предполагаемого момента завершения выполнения указанной актуальной задачи
\end{scnitemize}
}

\scnheader{социальная ответственность кибернетической системы}
\scnrelfromset{свойство-предпосылка}{способность кибернетической системы выполнять качественно и в срок взятые на себя обязательства в рамках соответствующих коллективов; способность кибернетической системы адекватно оценивать свои возможности при распределении коллективной деятельности; альтруизм/эгоизм кибернетической системы; отсутствие/наличие действий, которые по безграмотности кибернетической системы снижают качество коллективов, в состав которых она входит; отсутствие/наличие "осознанных"{}, мотивированных действий снижающих качество коллективов, в состав которых кибернетическая система входит}

\scnheader{альтруизм/эгоизм кибернетической системы}
\scnnote{уровень мотивации к повышеннию качества коллективов, в состав которых кибернетическая система входит}
\scntext{эпиграф}{Надо любить науку, а не себя в науке.}
\scntext{эпиграф}{Ты играешь и всем своим видом показываешь: ``Смотрите, как я красиво играю'', а надо играть и показывать красоту самой музыки.}

\scnheader{социальная активность кибернетической системы}
\scnidtftext{часто используемый sc-идентификатор*}{социальная активность}
\scnidtf{пассионарность}
\scnrelfromset{свойство-предпосылка}{способность кибернетической системы к генерации предлагаемых целей и планов коллективной деятельности; активность кибернетической системы в экспертизе результатов других участников коллективной деятельности; способность кибернетической системы к анализу качества всех коллективов, в состав которых она входит, а также всех член этих коллективов; способность кибернетической системы к участию в формировании новых коллективов; количество и качество тех коллективов, в состав которых кибернетическая система входит или входила}

\scnheader{способность кибернетической системы к участию в формировании коллективов}
\scnidtf{уровень способности в создании таких коллективов кибернетических систем, в состав которых входит данная кибернетическая система и которые направлены на коллективное решение соответствующего актуального класса сложных комплексных задач, с каждой из которых не может справиться любая из имеющихся кибернетических систем.}
\scnnote{Формирование специализированного коллектива кибернетических систем сводится к тому, что в памяти каждой кибернетической системы, входящей в коллектив, генерируется спецификация этого коллектива, включающая в себя:
\begin{scnitemize}
    \item перечень весь членов коллектива;
    \item способности (возможности) каждого из них;
    \item обязанности в рамках коллектива;
    \item спецификацию всего множества задач (вида деятельности), для решения (выполнения) которых сформирован данный коллектив кибернетических систем
\end{scnitemize}
}
\scnnote{Каждая кибернетическая система может входить в состав большого количества коллективов, выполняя при этом в разных коллективах в общем случае разные "должностные обязанности"{}, разные "бизнес-процессы"...}
\scnnote{Рассмотренный принцип формирования специализированного коллектива, состоящего из компьютерных систем и людей, фактически означает автоматизацию системной интеграции компьютерных систем и децентрализованный ("горизонтальный") характер такой интеграции, это очевидно предполагает наличие достаточно высокого уровня интеллекта у интегрируемых компьютерных систем и людей.}

\scnheader{количество и качество тех коллективов, в состав которых кибернетическая система входит или входила}
\scnexplanation{Данная характеристика кибернетической системы уточняет спектр ее социальной активности}
\scnnote{Чем умнее (интеллектуальнее) многоагентные системы, членом которых является данная кибернетическая система, тем выше ее "социальный статус"{} и перспективы быть умнее -- есть у кого учиться}