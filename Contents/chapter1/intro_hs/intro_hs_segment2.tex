\scnsegmentheader{Комплекс свойств, определяющий общий уровень качества кибернетической системы}

\scnstartsubstruct

\scnheader{качество кибернетической системы}
\scnidtf{интегральный уровень качества кибернетической системы в заданный момент}
\scnidtf{комплексная оценка (характеристика) уровня качества кибернетической системы}
\scntext{пояснение}{Для того, чтобы уточнить (детализировать) понятие \textit{качества кибернетической системы}, необходимо
	\begin{scnitemize}
		\item задать метрику \textit{качества кибернетических систем} и
		\item построить иерархическую систему свойств (параметров, признаков), определяющих \textit{качество кибернетической системы}.
	\end{scnitemize}
}
\scniselement{упорядоченное свойство}
\scnidtf{эволюционный уровень кибернетической системы}
\scnidtf{интегральная (комплексная) оценка уровня развития (совершенства) кибернетической системы}
\scntext{пояснение}{\textit{Качество кибернетической системы} -- это такое свойство (характеристика) \textit{кибернетических систем}, такой признак их классификации, который позволяет разместить эти системы по "ступенькам"{} некоторой условной "эволюционной лестницы"{}. На каждую такую "ступеньку"{} попадают \textit{кибернетические системы}, имеющие одинаковый уровень развития, каждому их которых соответствует свой набор значений дополнительных свойств \textit{кибернетических систем}, которые уточняют (детализируют, специализируют) соответствующий уровень развития \textit{кибернетических систем}. Такой эволюционный подход к рассмотрению \textit{кибернетических систем} даёт возможность, во-первых, детализировать направления эволюции \textit{кибернетических систем} и, во-вторых, уточнить то место этой эволюции, где и благодаря чему осуществляется переход от неинтеллектуальных \textit{кибернетических систем} к интеллектуальным. Фактически речь идёт об эволюционной теории качества \textit{кибернетических систем}, рассматривающей эволюцию \textit{кибернетических систем} как в рамках жизненного цикла каждой из них, так и в рамках эволюции целой "популяции"{} при переходе от одного поколения \textit{кибернетических систем} к другому поколению (в частности, от одного поколения \textit{компьютерных систем} к другому).\\
В основе эволюционного подхода к рассмотрению многообразия \textit{кибернетических систем} лежит положение о том, что идеальных \textit{кибернетических систем} не существует, но существует постоянное стремление к идеалу, к большему совершенству. При этом важно уточнить, что конкретно в каждой \textit{кибернетической системе} следует изменить, чтобы привести эту систему к более совершенному виду.\\
Эволюционный подход к рассмотрению \textit{кибернетических систем} имеет важное практическое значение для развития (совершенствования) каждой конкретной \textit{компьютерной системы} (искуственной \textit{кибернетической системы}), а также для развития \textit{технологий} разработки \textit{компьютерных систем}. Так, например, развитие технологий разработки \textit{компьютерных систем} должно быть направлено на переход к таким новым архитектурным и функциональным принципам, лежащим в основе \textit{компьютерных систем}, которые
\begin{scnitemize}
	\item обеспечивают существенное снижение трудоемкости их разработки и сокращение сроков разработки, а также
	\item обеспечивают существенное повышение уровня \textit{интеллекта} и, в частности, уровня \textit{обучаемости} разрабатываемых \textit{компьютерных систем}, например, путём перехода от поддержки обучения с учителем к реализации эффективного самообучения (к автоматизации организации самостоятельного обучения).
	\end{scnitemize}
}}
\scnnote{В эволюции \textit{кибернетических систем} (и, в частности, \textit{компьютерных систем}) можно выделить целый ряд этапов:
\begin{scnitemize}
	\item переход от стимульно-реактивного поведения к поведению, предполагающему учёт постоянно накапливаемого собственного опыта, означает переход от протопамяти, которая просто фиксирует связи между стимулами и соответствующими реакциями и которая не предполагает изменения этих связей, к \textit{памяти}, которая становится средой "существования"{} информации, отражающей  собственный опыт \textit{кибернетической системы} (а в перспективе и многое другое) и которая обеспечивает высокую степень \textit{гибкости} хранимой \textit{информации}, т.е. широкие возможности изменения (корректировки) этой \textit{информации} в процессе функционирования \textit{кибернетической системы}. Таким образом, \textit{память кибернетической системы} вместе с хранимой в ней \textit{информацией} становится управляемым самой этой \textit{кибернетической системой} гибким "коммутатором"{} между её стимулами и реакциями, учитывающим не только накапливаемый собственный опыт, но и контекст (дополнительные обстоятельства) выполняемых \textit{действий} (реакций), рассматривающий выполняемые \textit{действия} с самых разных аспектов;
	\item включение в состав \textit{информации, хранимой в памяти компьютерной системы}, \textit{программ}, описывающих различные \textit{методы} обработки этой \textit{информации} и интерпретируемых \textit{процессором} указанной \textit{компьютерной системы};
	\item переход от указанной выше "коммутационной"{} трактовки \textit{информации, хранимой в памяти кибернетической системы} к её трактовке как мощной и постоянно совершенствуемой информационной модели внешней среды, в которой существует указанная \textit{кибернетическая система}. Это означает
	\begin{scnitemizeii}
		\item переход \textit{информации, хранимой в памяти кибернетической системы} на уровень \textit{базы знаний}, которой ставится в \textit{соответствие} достаточно чёткая \textit{денотационная семантика}, и
		\item переход \textit{программ}, хранимых в \textit{памяти кибернетической системы}, на уровень \textit{программ}, которые ориентированы на обработку \textit{базы знаний} и которые сами являются частью обрабатываемой \textit{базы знаний};
	\end{scnitemizeii}
	\item существенное расширение \textit{семантической мощности баз знаний} и многообразия используемых \textit{моделей решения задач}, в том числе, моделей, способных работать в условиях неполноты (недостаточности), нечеткости и недостоверности обрабатываемых \textit{знаний}.
\end{scnitemize}
}
\scnnote{Повышение качества искусственных\textit{ кибернетических систем} (\textit{компьютерных систем}) потребует формирования таких свойств (характеристик, способностей) \textit{компьютерных систем}, которые аналогичны психическим свойствам людей. Таким образом, дальнейшее развитие \textit{Искусственного интеллекта} (теории и практики создания \textit{интеллектуальных компьютерных систем} -- интеллектуальных искусственных \textit{кибернетических систем}) настоятельно потребует обобщения современной психологии (психологии биологических индивидов и их коллективов -- \textit{психологии естественных кибернетических систем}) и создания \textit{общей психологии кибернетических систем} (как естественных, так и искусственных) основанной на высоком уровне формализации.
}


\scnheader{качество кибернетической системы}
\scnrelfromlist{cвойство-предпосылка}{качество физической оболочки кибернетической системы мы качество решателя задач кибернетической системы; 
качество решателя задач кибернетической системы
\newline
\scnaddlevel{1}
\scnrelfrom{cвойство-предпосылка}{качество информации, хранимой в памяти кибернетической системы }
\scnaddlevel{-1};
качество информации, хранимой в памяти кибернетической системы;
гибридность кибернетической системы
\scnaddlevel{1}
\scnidtf{степень многообразия (1) видов знаний, хранимых в памяти кибернетической системы, (2) используемых моделей решение задач, (3) видов сенсоров и эффекторов}
\scnrelfromlist{частное свойство}{многообразие видов знаний, хранимых в памяти кибернетической системы;
многообразие моделей решения задач;
многообразие видов сенсоров и эффекторов}
\scnaddlevel{-1};
приспособленность кибернетической системы к её совершенствованию;
производительность кибернетической системы
\scnaddlevel{1}
\scnidtf{cкорость решения задач кибернетической системы}
\scnaddlevel{-1};
надежность кибернетической системы;
социализация кибернетической системы
}

\scnheader{гибридность кибернетической системы}
\scnrelfromlist{частное свойство}{многообразие видов знаний, хранимых в памяти кибернетической системы;
многообразие моделей решения задач;
многообразие видов сенсоров и эффекторов}

\scnheader{гибридная кибернетическая система}
\scnidtf{кибернетическая система, использующая многообразие рецепторных и/или эффекторных подсистем, и/или многообразие видов обрабатываемой информации, и/или многообразие способов решения задач}
\scnsuperset{гибридная компьютерная система}
\scnaddlevel{1}
\scnidtf{\textit{компьютерная система}, способная решать \textit{комплексные задачи}, требующие использования многообразия различных видов обрабатываемой информации и различных \textit{моделей решения задач}}
\scnaddlevel{-1}

\scnheader{приспособленность кибернетической системы к её совершенствованию}
\scnidtf{приспособленность кибернетической системы к эволюции, к повышению уровня своего качества}
\scnrelfromset{комплекс свойств-предпосылок}{обучаемость кибернетической системы
\scnaddlevel{1}
\scnidtf{способность кибернетической системы самостоятельно повышать уровень своего качества}
\scnidtf{способность кибернетической системы к самоэволюции, саморазвитию, устранению своих недостатков}
\scnaddlevel{-1};
приспособленность кибернетической системы к её совершенствованию, осуществляемому извне
\scnaddlevel{1}
\scnidtf{приспособленность кибернетической системы к её совершенствованию, осуществляемому внешними субъектами}
\scnidtf{удобство совершенствования кибернетической системы для её создателей}
\scnnote{Важнейшим фактором качества каждой \textit{технологии разработки кибернетических систем} является гибкость и стратифицированность разрабатываемых кибернетических систем при их совершенствовании, осуществляемом руками разработчиков}
\scnaddlevel{-1}
}
\scnrelfromset{комплекс свойств-предпосылок}{гибкость кибернетической системы;
стратифицированность кибернетической системы
\scnaddlevel{1}
\scnidtf{уровень стратифицированности кибернетической системы}
\scnidtf{качество разделения (декомпозиции) кибернетической системы на в достаточной степени независимые части (компоненты), определенные виды изменений которых не предполагают внесения изменений в другие части системы}
\scnaddlevel{-1}
}

\scnheader{гибкость кибернетической системы}
\scnidtf{реконфигурируемость кибернетической системы}
\scnidtf{модифицируемость кибернетической системы}
\scnidtf{реформируемость кибернетической системы }
\scnidtf{трансформируемость кибернетической системы}
\scnidtf{пластичность кибернетической системы}
\scnidtf{легкость реализации различного вида изменений в кибернетической системе}
\scnidtf{степень трансформенности кибернетической системы}
\scnidtf{простота внесения изменений в кибернетическую систему и многообразие видов возможных таких изменений}
\scnidtf{модифицируемость кибернетической системы}
\scnidtf{трансформируемость кибернетической системы} 
\scnidtf{реконфигурируемость кибернетической системы} 
\scnidtf{приспособленность к реинжинирингу кибернетической системы}
\scnidtf{мягкость}
\scnidtf{softness}
\scnidtf{приспособленность к внесению изменений}
\scnidtf{\uline{легкость} внесения изменений}
\scnnote{Чем легче вносить изменения в кибернетическую систему, тем выше скорость ее эволюции}
\scnnote{изменения могут вноситься (1) полностью самостоятельно (без учителя) (2) с помощью учителя-тренера ("терапевта"{}) путем создания определенных условий для совершенствования системы (3) "хирургически"{} -- путем непосредственного вмешательства извне (например, вмешательства разработчика)}
\scnnote{Чем выше \textit{гибкость кибернетической системы} -- тем ниже трудоемкость и меньше сроки внесения различных изменений в систему в направлении ее совершенствования (приближения к идеалу)}
\scnrelfromset{комплекс свойств-предпосылок}{простота внесения изменений в кибернетическую систему
\newline
\scnaddlevel{1}
\scnrelfrom{свойство-предпосылка}{стратифицированность кибернетической системы}
\scnaddlevel{-1};
многообразие, возможных изменений, вносимых в кибернетическую систему
}
\scnrelfromset{комплекс частных свойств}{гибкость информации, хранимой в памяти кибернетической системы;
гибкость решателя задач кибернетической системы;
гибкость физической оболочки кибернетической системы
\scnaddlevel{1}
\newline
\scnrelfrom{частное свойство}{гибкость памяти кибернетической системы}
\scnaddlevel{-1};
гибкость интерфейса кибернетической системы}
\scnrelfromset{комплекс частных свойств}{гибкость кибернетической системы при ее совершенствовании, осуществляемом извне;
гибкость возможных самоизменений кибернетической системы
\scnaddlevel{1}
\newline
\scnrelto{свойство-предпосылка}{обучаемость кибернетической системы}
\scnaddlevel{-1}}

\scnheader{приспособленность кибернетической системы к её совершенствованию, осуществляемому извне}
\scnidtf{приспособленность кибернетическиой системы к "хирургическим"{} методам её совершенствования, реализуемым разработчиками}
\scnidtf{насколько легко осуществлять обновление, перепроектирование, тестирование, ремонт (исправление ошибок) кибернетической системы}
\scnrelfromlist{свойство-предпосылка}{простота внесения изменений в кибернетическую систему, реализуемых извне
\newline
\scnaddlevel{1}
\scnrelfrom{свойство-предпосылка}{стратифицированность кибернетической системы}
\scnaddlevel{-1};
многообразие возможных изменений кибернетической системы, реализуемых извне}

\scnheader{производительность кибернетической системы}
\scnidtf{быстродействие кибернетической системы}
\scnidtf{интегральная оценка скорости решения задач, время реакции кибернетической системы на задачные ситуации}
\scnrelfromlist{частное свойство}{производительность базового интерпретатора логико-семантической модели кибернетической системы;
качество используемых кибернетической системой методов и моделей решения задач}

\scnheader{надежность кибернетической системы}
\scnidtf{способность кибернетической системы при соответствующих условиях ее функционирования сохранять (и, точнее, не снижать) уровень всех свойств и способностей, определяющих общее (комплексное) качество кибернетической системы}
\scnrelfromlist{свойство-предпосылка}{безотказность кибернетической системы; долговечность кибернетической системы; "ремонтопригодность"{} кибернетической системы
\newline
\scnaddlevel{1}
\scnrelfrom{основной sc-идентификатор}{"ремонтопригодность"{} кибернетических систем}
\scnaddlevel{1}
\scnnote{Здесь слово "ремонтопригодность"{} взято в кавычки, т.к. речь идет не только об искусственных (технических) кибернетических системах}
\scnaddlevel{-1}
\scnaddlevel{-1}
}

\bigskip

\scnendstruct \scnendsegmentcomment{{Комплекс свойств, определяющий общий уровень качества кибернетической системы}