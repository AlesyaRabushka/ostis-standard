\bigskip
\scnsegmentheader{Комплекс свойств, определяющих качество многоагентной системы}
\scnstartsubstruct

\scnheader{многоагентная система}
\scnexplanation{Переход от \textit{кибернетических систем} к коллективам взаимодействующих между собой \textit{кибернетических систем}, т.е. к социальной организации кибернетических систем, является важнейшим фактором эволюции \textit{кибернетических систем}.}

\scnsubset{кибернетическая система}
\scnsubdividing{моногенная многоагентная система\\
	\scnaddlevel{1}
	\scnidtf{однородная \textit{многоагентная система}, состоящая из однотипных \textit{агентов}}
	\scnaddlevel{-1}
;гетерогенная многоагентная система
	\scnaddlevel{1}
	\scnidtf{неоднородная \textit{многоагентная система}, состоящая из \textit{агентов} разного типа}
	\scnaddlevel{-1}
}

\scnsubdividing{простая многоагентная система\\
	\scnaddlevel{1}
	\scnidtf{многоагентная система, \textit{агенты} которой не являются \textit{многоагентными системами}}
	\scnaddlevel{-1}
;иерархическая многоагентная система
	\scnaddlevel{1}
	\scnidtf{многоагентная система, некоторые или все \textit{агенты} которой являются \textit{многоагентными	системами}}
	\scnaddlevel{-1}
}

\scnnote{Агенты \textit{многоагентной системы} могут, но вовсе не обязательно должны быть \textit{интеллектуальными системами}, как, например, агенты системы решателя задач, имеющего агентно-ориентированную архитектуру.}
\scnrelfrom{семантическая окрестность}{Понятие Технологии OSTIS}
\scnaddlevel{1}
\scnsourcecommentpar{Сегмент 3 Раздела 0.2}
\scnaddlevel{-1}

\scnheader{агент*}
\scnidtf{быть агентом данной многоагентной системы*}
\scnidtf{быть кибернетической системой, входящей в состав данной многоагентной системы*}
\scnnote{Агентом иерархической многоагентной системы может быть другая многоагентная система}
\scnsuperset{член многоагентной системы*}
	\scnaddlevel{1}
	\scnidtf{агент многоагентной системы, не являющийся агентом другого агента этой системы*}	
	\scnidtf{непосредственный (ближайший) агент многоагентной системы*}
\scnaddlevel{-1}	

\scnheader{кибернетическая система}
\scnsubdividing{индивидуальная кибернетическая система\\
	\scnaddlevel{1}
	\scnidtftext{пояснение}{минимальная целостная \textit{кибернетическая система} обладающая достаточно высоким уровнем самостоятельности и способности “выживать"{} в своей \textit{внешней среде}}
	\scnaddlevel{-1}
;кибернетическая система, являющаяся минимальным компонентом индивидуальной кибернетической системы\\
	\scnaddlevel{1}
	\scnexplanation{Это такой компонент, в состав которого не входят \textit{кибернетические системы}}
	\scnaddlevel{-1}
;кибернетическая система, являющаяся комплексом компонентов соответствующей индивидуальной кибернетической системы
;сообщество индивидуальных кибернетических систем\\
	\scnaddlevel{1}
	\scnsubdividing{простое сообщество индивидуальных кибернетических систем;иерархическое сообщество индивидуальных кибернетических систем}
	\scnaddlevel{-1}
}

\scnheader{многоагентная система}
\scnidtf{коллектив взаимодействующих  кибернетических систем}
\scnsubdividing{сообщество индивидуальных кибернетических систем\\
;индивидуальная кибернетическая система, реализованная в виде многоагентной системы\\
	\scnaddlevel{1}
	\scnsubset{кибернетическая система, являющаяся комплексом компонентов соответствующей индивидуальной кибернетической системы}
	\scnexplanation{Такая "внутренняя"{} \textit{многоагентная система} в индивидуальной кибернетической системе появляется, когда на определенном этапе её эволюции \textit{решатель задач} \scnbigspace \textit{индивидуальной кибернетической системы} “переходит"{} на \textit{агентно-ориентированную модель обработки информации} в памяти \textit{индивидуальной компьютерной системы}}
	\scnaddlevel{-1}
}
\scnidtf{кибернетическая система, представляющая собой коллектив взаимодействующих кибернетических систем, обладающих определенной степенью самостоятельности (самодостаточности, свободы выбора)}

\scnheader{многоагентная система с централизованным управлением}
\scnidtf{многоагентная система, в которой специально выделяются агенты, которые принимают решения в определенной области деятельности многоагентной системы и обеспечивают выполнение этих решений  путем управления деятельностью остальных агентов, входящих в состав этой системы}
\scnsubset{многоагентная система}

\scnheader{сообщество интеллектуальных систем с децентрализованным управлением}
\scnidtf{многоагентная система с децентрализованным управлением, агентами которой являются интеллектуальные системы}
\scnidtf{многоагентная система, в которой решения принимаются коллегиально и "автоматически"{} (\uline{решения} о признании новой кем-то предложенной информации -- в том числе, об инициировании некоторой задачи, \uline{решения} о коррекции (уточнении) уже ранее признанной (одобренной, согласованной) информации) \uline{на основе} четко продуманной и постоянно совершенствуемой методики, а также \uline{на основе} активного участия всех агентов в формировании новых предложений, подлежащих признанию (одобрению, согласованию)}
\scnsubset{многоагентная система}
\scnnote{В такой многоагентной системе все агенты участвуют в управлении этой системы}
\scnhaselement{Экосистема OSTIS}
\scnexplanation{В такой многоагентной системе отсутствуют специально "назначенные"{} агенты, которые "обязаны"{} принимать решения о том, какую коллективно решаемую задачу надо инициировать, и о том, как распределить между агентами подзадачи указанной инициированной задачи.}
\scnsubset{многоагентная система с децентрализованным управлением}
\scnsubset{сообщество интеллектуальных систем}
\scnnote{Примером такой системы является оркестр, способный играть без дирижера. При этом подчеркнем, что каждый музыкант такого оркестра:
	\begin{scnitemize}
		\item должен иметь квалификацию не только музыканта, но и дирижера и даже композитора
		\item должен быть договороспособным -- уметь согласовывать свои действия с действиями коллег
	\end{scnitemize}
Аналогичным примером децентрализованной многоагентной системы является строительная бригада, способная построить дом без бригадира, прораба, архитектора.}

\scnauthorcomment{Дооформить библиографию}

\scnheader{синергетическая кибернетическая система}
\scnidtf{эволюционная многоагентная система}
\scnidtf{многоагентная система, состоящая из когнитивных агентов}
\scnidtf{многоагентная система, обладающая высоким уровнем коллективного интеллекта, атомарными агентами которой являются индивидуальные интеллектуальные системы, имеющие высокий уровень социализации}
\scnrelfrom{пояснение}{Ярушкина Н.Г.ред.2007кн-нечетГС-стр.88-101}
\scnaddlevel{1}
\scnrelto{цитата}{Ярушкина Н.Г. ред. 2007кн-нечетГС}
\scnaddlevel{-1}
\scnrelfromlist{пояснение}{Тарасов В.Б. 1997ст-ЭволюС;Тарасов В.Б. 1998ст-ЭволюС}
\scnnote{Очевидным примером синергетической кибернетической системы является творческий коллектив, реализующий сложный наукоемкий проект. Огромная сложность создания таких коллективов является главной причиной медленного развития целого ряда весьма актуальных научно-технических проектов, таких как создание принципиально нового технологического уровня автоматизации человеческой деятельности на основе интеллектуальных семантически совместимых компьютерных систем, способных самостоятельно взаимодействовать друг с другом.}


\scnheader{многоагентная система}
\scnnote{Переход к \textit{многоагентным системам} является важнейшим фактором повышения \textit{качества} (и, в частности, уровня \textit{интеллекта}) \textit{кибернетических систем}, т.к. уровень интеллекта \textit{многоагентной системы} может быть значительно выше уровня интеллекта каждого входящего в неё агента. Но это бывает далеко не всегда, поскольку важнейшим фактором качества многоагентных систем является не только качество входящих в неё агентов, но и организация взаимодействия агентов и, в частности, переход от централизованного к децентрализованному управлению. Количество далеко не всегда переходит в новое качество.\\
Повышение уровня интеллекта многоагентной системы обеспечивается
\begin{scnitemize}
	\item не только повышением уровня интеллекта и, в первую очередь, уровня \textit{социализации} ее агентов;
	\item не только переходом от централизованного к децентрализованного управлению деятельности управлению деятельностью агентов;
	\item но и качеством общей базы знаний всей многоагентной системы.
\end{scnitemize}
}

\scnheader{социализация кибернетической системы}
\scnnote{Когда мы говорим о \textit{социализации кибернетических систем}, речь идет только об \textit{индивидуальных кибернетических системах}, т.е. о \textit{кибернетических  системах}, достигших некоторого уровня целостности и автономности и способных входить в состав различных коллективов. Соответственно этому, качество \textit{индивидуальных кибернетических систем} определяется, кроме всего прочего тем, насколько большой вклад \textit{индивидуальная кибернетическая система} вносит в повышение качества тех коллективов, в состав которых она входит. Указанное свойство \textit{индивидуальных кибернетических систем} будем называть уровнем их \textit{социализации}. Прежде, чем детализировать это свойство, целесообразно рассмотреть то, чем определяется качество коллектива кибернетических систем, например, качество творческого сообщества компьютерных систем и людей.}

\scnheader{качество сообщества компьютерных систем и людей}
\scnexplanation{Эффективность творческого коллектива (например в области научно-технической деятельности) определяется:
	\begin{scnitemize}
	\item согласованностью мотивации (целевой установки) всего коллектива и каждого его члена:
		\scnaddlevel{1}
		\begin{scnitemizeii}
			\item не должно быть синдрома "лебедя, рака и щуки";
			\item не должно быть противоречий между целью коллектива и творческой самореализацией каждого его члена;
		\end{scnitemizeii}
		\scnaddlevel{-1}	
	\item эффективной организацией децентрализованного управления деятельностью членов сообщества;
	\item четкой, оперативной и доступной всем фиксацией документации текущего состояния содеянного и направлений его дальнейшего развития;
	\item уровнем трудоемкости оперативности фиксации индивидуальных результатов в рамках коллективно создаваемого общего результата;
	\item уровнем структурированности и, прежде всего, стратифицированности обобщенной документации  (базы знаний);
	\item эффективностью ассоциативного доступа к фрагментам документации;
	\item гибкостью коллективно создаваемой базы;
	\item автоматизацией анализа содеянного и управления проектом.
\end{scnitemize}	
}


\scnheader{качество многоагентной системы}
\scnrelfromlist{свойство-предпосылка}{средний уровень интеллекта членов многоагентной системы;средний уровень социализации членов многоагентной системы;минимальный уровень социализации членов многоагентной системы\\
	\scnaddlevel{1}
	\scnnote{Члены многоагентной системы, имеющие низкий уровень социализации, существенно снижают качество системы.}
	\scnaddlevel{-1}
;качество организации взаимодействия членов многоагентной системы\\
\scnaddlevel{1}
	\scnnote{Высший уровень качества организации взаимодействия агентов многоагентной системы обеспечивается:
\begin{scnitemize}
	\item введением дополнительного специального (корпоративного) агента, выполняющего функцию хранителя интегратора общих (корпоративных) знаний многоагентной системы;
	\item реализацией децентрализованного взаимодействия агентов, управляемого текущим состоянием информации, хранимой в памяти корпоративного агента.
\end{scnitemize}}
\scnaddlevel{-1}
}

\scnheader{ostis-система}
\scnsubset{многоагентная система, управляемая общей базой знаний}
\scnaddlevel{1}
\scnnote{Агенты \textit{ostis-системы} (sc-системы) являются \uline{специализированными} \textit{кибернетическими системами}, \uline{действия} каждой из которых (кроме \textit{сенсорных sc-агентов}) инициализируются определенного вида ситуациями и/или событиями в памяти \textit{ostis-системы} и \uline{заключаются} (за исключением \textit{эффекторных sc-агентов}) в преобразовании текущего состояния информации, хранимой в этой памяти. Таким образом, sc-агенты не являются интеллектуальными системами.}
\scnaddlevel{-1}