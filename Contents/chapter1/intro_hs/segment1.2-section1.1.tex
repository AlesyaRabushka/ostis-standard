\bigskip

\scnfragmentcaption

\scnheader{Структура кибернетической системы}

\scnstartsubstruct

\scnheader{кибернетическая система}
\scnrelfromset{обобщенная декомпозиция}{
информация, хранимая в памяти кибернетической системы;абстрактная память кибернетической системы;решатель задач кибернетической системы;физическая оболочка кибернетической системы
}

\scnheader{информация, хранимая в памяти кибернетической системы}
\scnidtf{информация, хранимая в памяти \textit{кибернетической системы} и представляющая собой информационную модель среды, в которой действует (существует, функционирует) эта \textit{кибернетическая система}
}
\scnidtf{текущее состояние памяти кибернетической системы}
\scnidtf{текущее состояние внутренней (информационной) среды кибернетической системы}
\scnrelto{второй домен}{информация, хранимая в памяти кибернетической системы*}
\scnaddlevel{1}
\scniselement{бинарное отношение}
\scniselement{ориентированное отношение}
\scnaddlevel{-1}

\scnheader{абстрактная память кибернетической системы}
\scnidtf{внутренняя абстрактная информационная среда кибернетической системы, представляющая собой динамическую информационную  конструкцию, каждое состояние которой есть не что иное, как информация , хранимая в памяти кибернетической системы в соответствующий момент времени}
\scnidtf{абстрактная динамическая модель памяти кибернетической системы}
\scnsubset{динамическая информационная конструкция}
\scnaddlevel{1}
\scnidtf{процесс преобразования информационной конструкции}
\scnaddlevel{-1}

\scnheader{решатель задач кибернетической системы}
\scnidtf{совокупность всех навыков (умений), приобретенных кибернетической системой к рассматриваемому моменту}
\scnidtf{встроенный в кибернетическую систему субъект, способный выполнять целенаправленные ("осознанные") действия во внешней среде этой кибернетической системы, а также в её внутренней среде (в абстрактной памяти)}

\scnheader{действие кибернетической системы}
\scnsubset{действие}
\scnidtf{целенаправленное ("осознанное") действие, выполняемое кибернетической системой, а точнее, её решателем задач}
\scnsubdividing{внешнее действие кибернетической системы\\
	\scnaddlevel{1}
	\scnidtf{действие, выполняемое кибернетической системой в её внешней среде}
	\scnidtf{поведенческое действие}
	\scnaddlevel{-1}
;действие кибернетической системы, выполняемое в собственной физической оболочке
;действие кибернетической системы, выполняемое в собственной абстрактной памяти
\scnaddlevel{1}
	\scnidtf{речь идёт о действиях, направленных на преобразование информации, хранимой в памяти, но никак не на преобразование физической памяти (физической оболочки абстрактной памяти)}
\scnaddlevel{-1}	
}
\scnnote{Каждое \uline{сложное} действие,выполняемое кибернетической системой вне собственный абстрактной памяти, включает в себя поддействия, выполняемые в указанной абстрактной памяти. Это означает, что все внешние действия кибернетической системы \uline{управляются} внутренними её действиями (действиями в абстрактной памяти).}

\scnheader{задача}
\scnidtf{спецификация действия}
\scnidtf{формулировка задачи с различной степенью детализации (уточнения) специфицируемого (описываемого) действия, в состав которой может входить:
	\begin{scnitemize}
		\item описание цели (целевой ситуации);
		\item указание объектов (аргументов) действия;
		\item указание типа действия (класса действий, которому принадлежит данное действие);
		\item указание субъекта действия;
		\item указание инструмента (средств) выполненного действия;
		\item и др.
	\end{scnitemize}}

\scnnote{Процесс решения задачи и действие, специфицируемое этой задачей (точнее, процесс выполнения этого действия) суть одно и то же.}

\scnheader{задача, решаемая кибернетической системой}
\scnidtf{задача, решаемая соответствующей кибернетической системой}
\scnidtf{Второй домен отношения "быть задачей, решаемой заданной кибернетической системой*"}
\scnrelboth{следует отличать}{задача, решаемая кибернетической системой*}
\scnaddlevel{1}
\scnidtf{быть задачей, решаемой заданной кибернетической системой*}
\scnaddlevel{-1}
\scnsubdividing{задача, решаемая кибернетической системой во внешней среде\\
	\scnaddlevel{1}
	\scnidtf{внешняя задача кибернетической системы}
	\scnidtf{задача, направленная на изменение состояния внешней среды соответствующей кибернетической системы, но включающая в себя (в качестве подзадач) задачи, решаемые в памяти кибернетической системы, например: 
		\begin{scnitemize}
			\item интерфейсные задачи (анализ первичный информации о текущем состоянии внешней среды),
			\item cенсо-моторную координацию выполнения сложных действий во внешней среде, состоящих из большого количества частных (более простых) действий, находящихся на разных уровнях иерархии,
			\item задачи планирования целенаправленного поведения во внешней среде,
			\item задачи принятия решений.
		\end{scnitemize}}
	\scnaddlevel{-1}
;задача, решаемая кибернетической системой в собственной физической оболочке
;задача решаемая кибернетической системой в абстрактной памяти
	\scnaddlevel{1}
	\scnidtf{задача, полностью решаемая в памяти кибернетической системы и направленная на изменение состояния информации, хранимой в памяти кибернетической системы}
	\scnidtf{внутренняя задача кибернетической системы}
	\scnaddlevel{-1}
}

\scnheader{навык}
\scnsubset{знание}
\scnexplanation{знание частного вида, содержащее (1) некоторый метод -- знание о том, как можно решать задачи, принадлежащие соответствующему множеству задач, (2) полное знание о том, как указанный метод следует интерпретировать (реализовывать), декомпозируя исходные задачи на подзадачи и, в конечном счёте на элементарные действия, выполняемые \textit{процессором кибернетической системы}}
\scnidtf{умение}
\scnidtf{методы и средства, обеспечивающие способность \textit{кибернетической системы} решать некоторое множество задач (выполнять некоторое множество действий)}

\scnheader{интерфейс кибернетической системы}
\scnidtf{условно выделяемый компонент \textit{решателя задач кибернетической системы}, обеспечивающий решение \textit{интерфейсных задач}, направленных на \uline{непосредственную} реализацию взаимодействия \textit{кибернетической системы} с её \textit{внешней средой}}
\scnidtf{решатель интерфейсных задач кибернетической системы}
\scnrelto{обобщенная часть}{решатель задач кибернетической системы}
\scnrelboth{следует отличать}{физическое обеспечение интерфейса кибернетической системы}
\scnaddlevel{1}
\scnrelto{обобщенная часть}{физическая оболочка кибернетической системы}
\scnaddlevel{-1}

\scnheader{физическая оболочка кибернетической системы}
\scnrelfromset{обобщенная декомпозиция}{память кибернетической системы\\
;процессор кибернетической системы
;физическое обеспечение интерфейса кибернетической системы
\scnaddlevel{1}
	\scnidtf{аппаратное обеспечение интерфейса кибернетической системы с её внешней средой}
	\scnrelfromset{обобщенная декомпозиция}{сенсорная подсистема физической оболочки кибернетической системы;
	эффекторная подсистема физической оболочки кибернетической системы}
\scnaddlevel{-1}
;корпус кибернетической системы
}

\scnheader{физическая оболочка кибернетической системы}
\scnidtf{часть кибернетической системы, являющаяся "посредником"{} между её внутренней средой (памятью, в которой хранится и обрабатывается информация кибернетической системы) и её внешней средой}
\scnrelto{второй домен}{физическая оболочка кибернетической системы*}
\scnaddlevel{1}
\scniselement{бинарное отношение}
\scniselement{ориентированное отношение}



