\bigskip
\scnfragmentcaption

\scnheader{степень конвергенции и интеграции различного вида знаний, хранимых в памяти кибернетической системы} 
\scnidtf{уровень "бесшовной"{} интеграции различного вида знаний кибернетической системы}
\scnnote{Максимальный уровень конвергенции и интеграции знаний (в том числе,  и знаний различного вида) предполагает:
\begin{scnitemize} 
	\item использование универсального базового языка, по отношению к которому всем используемым видам знаний соответствуют специализированные языки, являющиеся подъязыками указанного базового языка
	\item построение четкой иерархии указанных специализированных языков по принципу "язык-подъязык"{}
	\item явное введение семейства отношений, заданных на множестве различных знаний и, в том числе, связывающих знания различного вида
\end{scnitemize}
}
\scnrelfromlist{свойства-предпосылка}{уровень формализованности информации, хранимой в памяти кибернетической системы}
\scnheader{уровень формализованности информации, хранимой в памяти кибернетической системы}
\scnidtf{степень приближения информации, хранимой в памяти кибернетической системы, к максимально простой и компактной форме представление информационной модели некоторого множества описываемых сущностей, которая является отражением определенной конфигурации связей между указанными сущностями}
\scnnote{Высшим уровнем формализации информации, хранимой в памяти кибернетической системы, является смысловое представление информации в форме семантических сетей. Смотрите раздел Предметная область и онтология семантических сетей, семантических языков и семантических моделей баз знаний}
\scnrelboth{следует отличать}{формализация*}
	\scnaddlevel{1}
		\scnidtf{Бинарное ориентированное отношение, каждая пара которого связывает некоторую информационную конструкцию с другой информационной конструкцией, которая семантически эквивалентна первой, но имеет более высокий уровень формализованности}
		\scnnote{Приобретение навыков формального представления информации не является простой проблемой даже для человека. По сути совокупность таких навыков -- это основа математической культуры, культуры точного изложения своих соображений. Некоторые примеры, иллюстрирующие нетривиальность проблемы смотрите в Арнольд В. И. 2012кн-ЧтоТМ-стр. 75-76}
	\scnaddlevel{-1}
\scnrelboth{следует отличать}{формализация}
	\scnaddlevel{1}
		\scnidtf{деятельность, направленная на повышение уровня формализованности представление информации}
		\scntext{метафора}{сближение синтаксиса с семантикой -- сближение синтаксической структуры информационной конструкции с её смысловой структурой}
	\scnaddlevel{-1}
\scnidtf{уровень способности кибернетической системы к формальному представлению знаний и используемых понятий, к рационализации идей}
\scnidtf{степень близости языка внутреннего представления (способа внутреннего кодирования) информации в памяти кибернетической системы к смысловому представлению информации}
\scnidtf{степень близости к изоморфизму соответствие между: (1) синтаксической структурой внутреннего представления информации в памяти кибернетической системы и (2) конфигурацией связей описываемых сущностей}
\scnrelfromlist{свойства-предпосылка}{многообразие форм дублирования информации, хранимой в памяти кибернетической системы
;относительный объём дублирования информации, хранимой в памяти кибернетической системы 
;многообразие фрагментов хранимой информации, не являющихся ни знаками, ни конфигурациями знаков 
;компактность представления представление информации, хранимой в памяти кибернетической системы}
\scnheader{смысловое представление информации}
\scnidtfexp{способ представления информации, в котором минимизируются "чисто синтаксические"{} аспекты представления информационных конструкций, не имеющие непосредственной семантической интерпретации
\scnnote{
Примерами "чисто синтаксических"{} аспектов представления информационных конструкций являются:
\begin{scnitemize}
	\item буквы, которые входят в состав слов и которые, следовательно, не являются знаками описываемых сущностей;
	\item алфавиты букв различны знаков;
	\item знаки препинания (разделители и ограничители);
	\item инцидентность (порядок, последовательность) букв и других символов, входящих в состав информационной конструкции
\end{scnitemize}
}
	\scnaddlevel{1}
		\scntext{следовательно}{Информационная конструкция, представленная на каком-либо привычном для нас языке, является достаточно громоздкой информационной конструкцией, смысл которой (т.е знаки описываемых сущностей и семантически интерпретируемых связи между знаками, отражающие соответствующие связи между обозначаемыми сущностями) сильно закамуфлирован. Это существенно усложняет обработку информации. если пытаться реализовывать “осмысленные” модели решения задач, для которых “смысловые” аспекты обрабатываемой информации являются ключевыми.}
	\scnaddlevel{-1}
}
\scnheader{смысловое представление информации}
\scnnote{Существенно подчеркнуть, что приближение внутреннего представления информации в памяти кибернетической системы к смысловому представлению информации является важнейшим фактором упрощения решателя задач кибернетической системы при реализации сложных моделей решения задач, требующих глубокого анализа смысла обрабатываемой информации. А это, в свою очередь, является важнейшим фактором качества решателя задач кибернетической системы.}
\scnheader{многообразие форм дублирования информации, хранимой в памяти кибернетической системы}
\scnidtf{многообразие видов семантической эквивалентности фрагментов информации, хранимой в памяти кибернетической системы}
\scnnote{Простейшим видом семантической эквивалентности является синонимия знаков, когда два разных фрагмента хранимой информации являются знаками, имеющими один и тот же денотат (т. е обозначающими одну и ту же сущность)}
\scnheader{относительный объем дублирования информации, хранимой в памяти кибернетической системы}
\scnidtf{частота присутствия в хранимой информации семантически эквивалентных информационных фрагментов и, в частности, синонимичных знаков}
\scnheader{многобразие фрагментов хранимой информации, не являющихся ни знаками, ни конфигурациями знаков}
\scnnote{Примерами фрагментов хранимой информации, не являющихся знаками или конфигурациями знаков являются:
\begin{scnitemize}
	\item буквы, входящие в состав слов
	\item слова, входящие в состав словосочетаний
	\item различного вида разделители, знаки препинания
	\item различного вида ограничители.
\end{scnitemize}
}
\scnheader{компактность представления информации, хранимой в памяти кибернетической системы}
\scnnote{Должно уменьшиться число элементов памяти, используемых для представления информации т.е. необходим переход к более компактным, но семантически эквивалентным информационным конструкциям}
\scnheader{стратифицированность информации,хранимой в памяти кибернетической системы}
\scnrelfrom{свойство-предпосылка}{структурированность информации, хранимой в памяти кибернетической системы}
\scnidtf{способность кибернетической системы выделять такие разделы информации, хранимой в памяти этой системы, которые бы ограничивали области действия агентов решателя задач кибернетической системы, являющиеся достаточными для решения заданных задач}
	\scnaddlevel{1}
		\scnnote{Существует правило, позволяющее каждой заданной задачи поставить в соответствие априори известный (выделенный) раздел хранимой информации, являющийся областью действия агентов решателя осуществляющих решение заданной задачи. Основными видами такого рода разделов хранимой информации являются \textit{предметные области} и \textit{онтологии}.}
	\scnaddlevel{-1}
\scnrelfrom{свойство-предпосылка}{рефлексивность информации, хранимой в памяти кибернетической системы}
\scnidtf{уровень систематизации знаний, хранимых в памяти кибернетической системы}
\scnidtf{уровень перехода от неструктурированных или слабоструктурированных данных к хорошо структурированным базам знаний}
\scnidtf{уровень перехода от первичной информации к метаинформации, метаметаинформации и т.д.}
\scnheader{рефлексивность информации,хранимой в памяти кибернетической системы}
\scnidtf{уровень применения средств самоописания (метаязыковых средств) в информации, хранимой в памяти кибернетической системы}
\scnidtf{относительный, объём и многообразие метаинформации, хранимой в памяти кибернетической системы}
\scnnote{рефлексивность информации, хранимой в памяти кибернетической системы, т.е. наличие метаязыковых средств является фактором, обеспечивающим не только структуризацию хранимой информации, но возможность описания синтаксиса и семантики самых различных языков, используемых кибернетической системой.}
\scnheader{простота и локальность выполнения семантически целостных операций над информацией, хранимой в памяти кибернетической системы}
\scnnote{Данное свойство касается не самой информации, хранимой в памяти, а язык кодирования (представления) информации в памяти кибернетической системы}
\scnidtf{гибкость выполнения семантически целостных операций над информацией, хранимой в памяти кибернетической системы}
\scnheader{база знаний}
\scnidtf{база знаний кибернетической системы}
\scnsubset{информация, хранимая в памяти кибернетической системы}
\scnidtfexp{информация, хранимая в памяти кибернетической системы и имеющая высокий уровень качества по всем показателям и, в частности, высокий уровень:
\begin{scnitemize}
	\item \textit{семантической мощности языка представления информации хранимой в памяти кибернетической системы} (в базе знаний указанный язык должен быть универсальным);
	\ite \textit{гибридности информации, хранимой в памяти кибернетической системы};
	\item \textit{многообразия видов знаний, хранимых в памяти кибернетической системы};
	\item формализованности информации, хранимой в памяти кибернетической системы;
	\item \textit{структурированности информации, хранимой в памяти кибернетической системы}
\end{scnitemize}
}
\scnnote{Переход информации, хранимой в памяти кибернетической системы на уровене качества, соответствующий базам знаний, является важнейшим этапом эволюции кибернетических систем. Подчеркнем при этом, что базы знаний по уровню своего качества могут сильно отличаться друг от друга}

\scnendstruct \scninlinesourcecommentpar{Завершили Сегмент 5 Начала Раздела \ref{intro_idtf}}


