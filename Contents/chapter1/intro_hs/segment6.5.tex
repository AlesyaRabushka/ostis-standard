\bigskip
\scnfragmentcaption

\scnheader{качество логико-семантической организации памяти кибернетической системы}
\scnidtf{качество базовых семантически целостных действий в памяти кибернетической системы}
\scnidtf{качество семантически элементарных (законченных, целостных) информационных процессов, выполняемых кибернетической системой в своей памяти}
\scnidtf{интегральная оценка того, насколько способствует (насколько близка) организация памяти кибернетической системы реализации "осмысленных"{} преобразований, хранимых в памяти знаний}
\scnidtf{степень приспособленности решателя задач кибернетической системы к обработке сложноструктурированных баз знаний}
\scnidtf{степень приспособленности решателя задач кибернетической системы к обработке хранимой в её памяти информации, имеющий высокий уровень качества как по форме представления информации, так и по её содержанию -- по многообразию представляемых знаний и по уровню их конвергенции и интеграции}
\scnrelfromlist{свойство-предпосылка}{семантический уровень доступа к информации, хранимой в памяти кибернетической системы
;семантическая гибкость информации, хранимой в памяти кибернетической системы
;степень конвергенции и интеграции представления навыков, хранимых в памяти кибернетической системы, с представлением обрабатываемой информации}


\scnheader{семантический уровень доступа к информации, хранимой в памяти кибернетической системы}
\scnidtf{степень ассоциативности доступа к информации, хранимой в памяти кибернетической системы}
\scnidtf{способность кибернетической системы локализовывать (находить) требуемый (запрашиваемый) фрагмент информации, хранимой в её памяти, не на основании известного адреса запрашиваемой информации (её местоположения в памяти), а на основании:
    \begin{scnitemize}
    \item известного типа запрашиваемой информации;
    \item известных сущностей, знаки которых входят в состав запрашиваемой информации;
    \item полностью или частично известной конфигурации запрашиваемой информации (т.е. конфигурации связей между известными и искомыми сущностями)
    \end{scnitemize}}
\scnexplanation{\textit{уровень доступа к информации, хранимой в памяти кибернетической системы} определяется тем, что нам достаточно знать об искомой в памяти кибернетической системы информации (в частности, об искомом знаке некоторой интересующей нас сущности). Мы можем знать место в памяти (ячейку памяти, область памяти), где находится интересующая нас информация. Такой доступ называется \uline{адресным}. Мы можем знать имя интересующей нас сущности, но не знать, где находится информация, описывающая эту сущность. Мы можем не знать имени интересующей нас сущности, но знать, как эта сущность связана с другими известными нам сущностями.}
\scnexplanation{Пусть нам необходимо локализовать (выделить) хранимую в памяти информацию, описывающую известные нам сущности, связанные известными нам отношениями, но местонахождение этой информации в памяти нам не известно. Если организация памяти нам представляет такую возможность, то такую память будем называть ассоциативной, т.е. памятью, обеспечивающей семантический доступ к хранимой в ней информации.}
\scnnote{Для того, чтобы построить информационную модель среды, в которой действует (функционирует) кибернетическая система, необходимо, с одной стороны, "разложить"{} эту информационную модель "по полочкам"{}, превратить её в некую систему из компонентов этой информационной модели, а, с другой стороны, обеспечить быстрый поиск нужного фрагмента указанной информационной модели, не зная, на каких "полочках"{}  находятся компоненты этого искомого фрагмента, который при этом может иметь произвольную конфигурацию и произвольный размер. Это и есть высший уровень \uline{ассоциативности} доступа к информации, хранимой в памяти кибернетической системы.}
\scnexplanation{Данное свойство, данная характеристика организации информации, хранимой в памяти кибернетической системы, является важнейшей характеристикой \uline{внутреннего} языка представления информации в памяти кибернетической системы. Указанная характеристика внутреннего языка определяется \uline{простотой процедур поиска} востребованных (запрашиваемых) фрагментов хранимой информации -- например, процедуры поиска знаков всех сущностей, каждая из которых связана с заданными (известными) сущностями связями заданных (известных) типов, процедуры поиска (выделения) знаков всех сущностей, которые связаны с заданной (известной) сущностью связью неважно какого типа, процедуры поиска информационного фрагмента заданному образцу (шаблону) произвольного размера и конфигурации, в котором выделены знаки известных сущностей и условные обозначения искомых сущностей.}
\scnaddhind{1}
\scnrelfrom{свойство-предпосылка}{степень близости языка внутреннего представления информации в памяти кибернетической системы к смысловому представлению информации}

\scnheader{семантическая гибкость информации, хранимой в памяти кибернетической системы}
\scnaddhind{1}
\scnrelfrom{свойство-предпосылка}{степень близости языка внутреннего представления информации в памяти кибернетической системы к смысловому представлению информации}
\scnidtf{простота реализации базовых (элементарных), но семантически целостных (семантически значимых, осмысленных) действий (операций) преобразования (обработки) информации, хранимой в памяти кибернетической системы}

\scnheader{базовое семантически целостное действие над информацией, хранимой в памяти кибернетической системы}
\scnidtf{элементарная семантически значимая (осмысленная) операция над информацией, хранимой в памяти кибернетической системы}
\scnnote{Здесь принципиальной является семантическая целостность (осмысленность) действия над хранимой информацией. Так, например, операция адресного доступа к требуемому фрагменту хранимой информации не является семантически целостной, так как смысл искомого (запрашиваемого) фрагмента хранимой информации не уточняется.}
\scnnote{Разные кибернетические системы могут использовать разные наборы классов базовых семантически целостных действий над информацией, хранимой в их памяти.}
\scnnote{Примерами \textit{базовых семантически целостных действий над информацией, хранимой в памяти кибернетической системы}, в частности, являются:
    \begin{scnitemize}
    \item операции поиска, генерации, удаления или замены связок между знаками известных сущностей;
    \item операции поиска, генерации, удаления или замены имен, приписываемых знакам известных сущностей.
    \end{scnitemize}
Существенно подчеркнуть, что простота реализации такого рода операций (т.е. гибкость хранимой в памяти информации) во многом обеспечивается стремлением к локальности выполнения этих операций. Такая локальность означает то, что при выполнении \uline{каждой} из указанных операций меняется только обрабатываемый фрагмент хранимой информации и не требуется никакого переразмещения в памяти остальной части хранимой информации.}

\scnheader{степень конвергенции и интеграции представления навыков, хранимых в памяти кибернетической системы, с представлением обрабатываемой информации}
\scnaddhind{1}
\scnrelto{частное свойство}{степень конвергенции и интеграции различного вида знаний, хранимых в памяти кибернетической системы}
    \scnaddlevel{1}
    \scnrelfrom{свойство-предпосылка}{степень близости языка внутреннего представления информации в памяти кибернетической системы к смысловому представлению информации}
    \scnaddlevel{-1}
\scnnote{Навыки кибернетической системы являются частным видом знаний, хранимых в её памяти, поэтому степень конвергенции навыков и обрабатываемых знаний определяется "глубиной"{} и "объемом"{} \uline{общих} (одинаковых) принципов, лежащих в основе как представления навыков, так представления обрабатываемых знаний.}

\bigskip
\scnfragmentcaption

\scnheader{качество решения интерфейсных задач в кибернетической системе}
\scnrelfromlist{частное свойство}{способность кибернетической системы к пониманию сенсорной информации
;способность кибернетической системы к пониманию принимаемых сообщений
;способность кибернетической системы к самостоятельной деятельности во внешней среде\\
    \scnaddlevel{1}
    \scnidtf{способность кибернетической системы к воздействию на внешнюю среду и к управлению своим поведением во внешней среде}
    \scnaddlevel{-1}}
    

\scnheader{интерфейсная задача}
\scnsuperset{задача анализа введенной информации}
    \scnaddlevel{1}
    \scnsuperset{задача анализа сенсорной информации}
        \scnaddlevel{1}
        \scnidtf{задача анализа информации, порождаемой (генерируемой) непосредственно сенсорами кибернетической системы}
        \scnsuperset{задача синтаксического анализа сенсорной информации}
        \scnsuperset{задача семантического анализа сенсорной информации}
            \scnaddlevel{1}
            \scnidtf{задача анализа сенсорной информации, направленного на \uline{понимание} этой информации -- на выявление (распознавание) в этой информации отображения (сенсорного описания) объектов, важных для кибернетической системы (т.е. объектов, описанных в базе знаний этой системы и, соответственно, представленных в этой базе знаний своими знаками либо знаками классов, которым эти объекты принадлежат), а также важных для кибернетических связей между указанными объектами}
            \scnidtf{задача генерации фрагмента базы знаний кибернетической системы, являющегося логическим следствием заданной сенсорной информации и представляющегося собой важную для кибернетической системы информацию}
            \scnidtf{задача извлечения из сенсорной информации (первичной информации) важной для кибернетической системы вторичной информации}
            \scnaddlevel{-1}
        \scnaddlevel{-1}
        \scnsuperset{задача анализа принимаемого вербального сообщения}
            \scnaddlevel{1}
            \scnidtf{задача анализа введенных знаковых конструкций}
            \scnidtf{задача анализа сообщений, введенных в кибернетическую систему}
            \scnidtf{задача анализа внешних знаковых конструкций}
            \scnsuperset{задача синтаксического анализа принимаемого вербального сообщения}
            \scnsuperset{задача трансляции принимаемого вербального сообщения на внутренний язык кибернетической системы}
            \scnsuperset{задача погружения нового фрагмента в состав согласованной части базы знаний}
                \scnaddlevel{1}
                \scnidtf{задача интеграции (встраивания) нового фрагмента базы знаний в состав базы знаний}
                \scnidtf{задача понимания нового фрагмента базы знаний в контексте её текущего состояния, что, прежде всего, требует обеспечения семантической совместимости (согласования понятий) между базой знаний и интегрируемым новым фрагментом}
                \scnaddlevel{-1}
            \scnaddlevel{-1}
    \scnaddlevel{-1}
\scnsuperset{задача управления эффекторами кибернетической системы при выполнении сложных воздействий на внешнюю среду и/или физическую оболочку этой кибернетической системы}
    \scnaddlevel{1}
    \scnidtf{задача целенаправленной сенсорно-эффекторной (в частности, сенсомоторной) координации}
    \scnaddlevel{-1}
    
\scnheader{сенсорная информация}
\scnidtf{информация, генерируемая непосредственно некоторой группой (конфигурацией) сенсоров (рецепторов) кибернетической системы}
\scnidtf{рецепторная информация}
\scnidtf{первичная информация, получаемая (приобретаемая) кибернетической системой}
\scnidtf{первичная знаковая конструкция, которая описывает те или иные свойства текущего состояния физической окружающей среды (внешней среды и физической оболочки) кибернетической системы}

\scnheader{сенсор кибернетической системы}
\scnidtf{рецептор кибернетической системы}
\scnexplanation{компонент кибернетической системы, генерирующий в памяти этой системы информацию о текущем значении соответствующего этому компоненту свойства (характеристики, параметра) того фрагмента физической окружающей среды кибернетической системы, который непосредственно смежен (пограничен) указанному компоненту}

\scnheader{эффектор кибернетической системы}
\scnidtf{компонент кибернетической системы, который способен менять своё состояние в целях непосредственного воздействия на свою физическую оболочку и на внешнюю среду}

\scnheader{способность кибернетической системы к пониманию сенсорной информации}
\scnidtf{способность к синтаксическому и семантическому анализу информации, формируемой сенсорами кибернетической системы, а также к "погружению"{} этой информации в состав общей информационной модели внешней среды кибернетической системы (в состав общей картины внешнего мира)}
\scnidtf{способность кибернетической системы к переходу от первичной (сенсорной) информации ко вторичной информации, которая описывает связи между вторичными объектами, каждый из которых представлен (описан) в первичной информации конфигурацией знаков своих частей с дополнительным описанием свойств каждой из этих частей}

\scnheader{способность кибернетической системы к самостоятельной деятельности во внешней среде}
\scnrelfromlist{свойство-предпосылка}{уровень развития эффекторов, обеспечивающих самостоятельное перемещение кибернетической системы\\
    \scnaddlevel{1}
    \scnrelfromlist{частное свойство}{уровень развития эффекторов, обеспечивающих локальное перемещение сенсоров кибернетической системы;уровень развития эффекторов, обеспечивающих функционирование манипуляторов кибернетической системы;уровень развития эффекторов, обеспечивающих перемещение всей физической оболочки кибернетической системы}
    \scnaddlevel{-1}
    ;качество управления поведением кибернетической системы во внешней среде\\
        \scnaddlevel{1}
        \scnidtf{качество сенсорно-эффекторной координации действий кибернетической системы при выполнении сложных действий во внешней среде}
        \scnaddlevel{-1}}