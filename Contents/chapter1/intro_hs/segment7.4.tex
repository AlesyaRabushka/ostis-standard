\scnheader{способность кибернетической системы к устранению информационных дыр, обнаруженных в информации, хранимой в ее памяти}
\scnrelfrom{свойство-предпосылка}{способность кибернетической системы генерировать ответы на вопросы различного вида в случае, если они целиком или частично отсутствуют в текущем состоянии информации, хранимой в памяти}
\scnnote{Формальным результатом обнаружения информационной дыры является формулировка запроса на недостающую информацию, которую необходимо сгенерировать.}


\scnheader{способность кибернетической системы к удалению информационного мусора, обнаруженного в информации, хранимой в ее памяти}
\scnidtf{способность кибернетической системы к забыванию (стиранию, удалению) ненужной (лишней, "отработанной"{}) информации, которая, например, играет роль информационных "лесов"{} при решении различных задач}
\scnnote{Критериями информационного мусора может быть:
		\begin{scnitemize}
		\item завершение решения задачи, для которой данная информация является вспомогательной и востребованной только в рамках решения соответствующей задачи;
		\item истечение срока давности хранения;
		\item легкая воспроизводимость (при необходимости).
		\end{scnitemize}}
		

\scnheader{способность кибернетической системы к семантическому погружению новых знаний в состав информации, хранимой в ее памяти}
\scnnote{Новая введенная в память информационная конструкция трактуется как конструкция, у которой входящие в нее знаки являются потенциальными синонимами тем знакам, которые уже присутствуют в хранимой информации. Поэтому для всех этих знаков надо проверить наличие их синонимов. После этого синонимичные знаки должны быть отождествлены. Отождествление знаков осуществляется либо путем приписывания им одинаковых идентификаторов (имен), либо путем "физического"{} склеивания этих знаков.}
\scnnote{Новой информацией, погружаемой (вводимой) в состав информации, хранимой в памяти кибернетической системы, может быть:
		\begin{scnitemize}
		\item либо принятое сообщение, поступившее от другой кибернетической системы и переведенное на внутренний язык данной системы;		
		\item либо информация, сгенерированная в результате решения какой-либо задачи.
		\end{scnitemize}}
		
	
\scnheader{способность кибернетической системы к обнаружению сходств в знаниях, хранимых в ее памяти}
\scnnote{Сходства в знаниях могут иметь самый разнообразный вид и далеко не всегда являются очевидными.}
\scnnote{Умение "видеть"{} сходство в различном и различие в сходном является важнейшим признаком интеллекта.}


\scnheader{способность кибернетической системы к конвергенции знаний, хранимых в ее памяти}
\scnidtf{способность кибернетической системы к увеличению сходств в знаниях хранимых в ее памяти}
\scnrelfrom{свойство-предпосылка}{способность к увеличению числа общих понятий для различных фрагментов информации, хранимой в памяти кибернетической системы, без ущерба качеству этих фрагментов}
\scnidtf{способность к "сближению"{} знаний путем:
		\begin{scnitemize}
		\item увеличения числа общих понятий, используемых в "сближаемых"{} знаниях;
		\item преобразования исходных знаний к их логически эквивалентным вариантам в целях получения фрагментов как можно большего размера и как можно в большем количестве, которые были бы:
				\begin{scnitemizeii}
				\item либо синтаксически изоморфными и содержащими как можно большее число общих понятий;	
				\item либо синтаксически изоморфными и одновременно семантически эквивалентными.
				\end{scnitemizeii}
		\end{scnitemize}}
		

\scnheader{способность кибернетической системы к интеграции знаний, хранимых в ее памяти}
\scnidtf{способность объединять имеющиеся знания и формировать целостную картину различных исследуемых объектов, систем, процессов, явлений}
\scnrelfrom{свойство-предпосылка}{способность кибернетической системы к конвергенции знаний, хранимых в ее памяти}
\scnnote{Качество (глубина) интеграции знаний определяется тем, насколько качественно до этого была проведена конвергенция интегрируемых знаний.}
\scnnote{Качественная ("бесшовная"{}, глубокая) интеграция различных знаний, хранимых в памяти кибернетической системы, дает возможность существенно снизить количество хранимых в памяти методов решения задач, так как позволяет некоторые ранее различные классы задач объединить в один класс задач. При этом очевидно, что такая интеграция знаний, хранимых в памяти кибернетической системы, требует разработки \uline{общих} (базовых) синтаксических и семантических принципов представления знаний различного вида.}


\scnheader{конвергенция и интеграция знаний}
\scnnote{Мы вынуждены смотреть на окружающую нас внешнюю среду (внешний мир) через "замочную скважину"{} своих сенсоров (рецепторов), своих персональных точек зрения, мировоззрения различных научных дисциплин. Но необходимо помнить, что целостную картину внешней среды (картину мира) можно построить только путем сближения (конвергенции) и соединения (интеграции) самых различных точек зрения, самых различных научных дисциплин и направлений. Мир не делится на различные дисциплины -- он един. Для эффективного решения задач конвергенции и интеграции знаний необходимо построить искусственную ("рукотворную") среду (память), в которой было бы удобно не только хранить самые различные знания, но и осуществлять конвергенцию и интеграцию этих знаний. При этом очень важно, чтобы формируемая таким образом информационная модель окружающей нас внешней среды (информационной картины мира) была общедоступна как для просмотра (ознакомления), причем, без каких бы то ни было "замочных скважин"{}, так и для ввода новых знаний, представляющих (отражающих) точку зрения их авторов.}
\scnrelfrom{эпиграф}{Древнеиндийская притча о слоне и слепцах}

		
\scnheader{следует отличать*}
\scnhaselementset{
конвергенция\scnsupergroupsign
\scnaddlevel{1}
\scnidtf{Свойство, определяющее степень близости (уровень конвергенции) между двумя заданными сущностями и, в частности, знаниями}
\scnaddlevel{-1};
конвергенция* 
\scnaddlevel{1}
\scnidtf{Множество пар близких (аналогичных, сходных) сущностей*}
\scnaddlevel{-1};
конвергенция
\scnaddlevel{1}
\scnidtf{Множество \uline{процессов} сближения различных пар сущностей}
\scnaddlevel{-1}}		
		
\scnhaselementset{
интеграция*
\scnaddlevel{1}
\scnidtf{Квазибинарное \uline{отношение}, каждая пара которого связывает множество интегрируемых сущностей с результатом интеграции*}
\scnaddlevel{-1};
интеграция
\scnaddlevel{1}
\scnidtf{Множество \uline{процессов} интеграции множества заданных сущностей}
\scnaddlevel{-1}}


\scnheader{способность кибернетической системы к обобщениям и формированию новых понятий}	
\scnnote{Важным примером обобщения является переход от задач к классам часто решаемых задач.}


\scnheader{cпособность кибернетической системы к генерации гипотез и обнаружению закономерностей в информации, хранимой в ее памяти}
\scnnote{Данная способность кибернетической системы является важнейшим фактором эволюции информации, хранимой в памяти кибернетической системы, в направлении перехода от данных (от фактографической информации) к знаниям.}


\scnheader{cпособность кибернетической системы к обоснованию или опровержению знаний, хранимых в ее памяти}
\scnnote{Примерами знаний, подлежащих обоснованию или опровержению, являются:
\begin{scnitemize}
	\item любое введенное в кибернетическую систему сообщение (любая новая информация, поступающая от любого субъекта);
	\item формулировка какой-либо задачи, предлагаемой для решения;
	\item формулировка какого-либо гипотетического утверждения (теоремы), подлежащего доказательству.
\end{scnitemize}}	
\scnidtf{способность к объяснению (обоснованию, аргументации) корректности, важности и целесообразности использовать (обратить внимание на) указываемое знание}
\scnidtf{способность либо находить в текущем состоянии базы знаний, либо генерировать (строить) ответы на \textit{почему-вопросы}}
	

\scnheader{способность кибернетической системы к экспериментальному подтверждению или опровержению гипотез о свойствах динамических систем с помощью имитационных моделей этих систем}
\scnnote{Создание динамических информационных моделей сложных динамических систем и проведение различного рода "мысленных"{} экспериментов с такими моделями является весьма перспективным и мощным методом исследования сложных динамических систем.}	


\scnheader{способность кибернетической системы к коррекции теорий, хранимых в ее памяти}
\scnidtf{способность к адаптации накопленных знаний к различным изменениям условий и жизненных ситуаций}
\scnnote{В основе данного свойства кибернетической системы лежит:
\begin{scnitemize}
	\item постоянная готовность кибернетической системы подвергнуть сомнению любое знание, хранимое в ее памяти;
	\item постоянное уточнение степени достоверности каждого знания, хранимого в памяти кибернетической системы.
\end{scnitemize}}	