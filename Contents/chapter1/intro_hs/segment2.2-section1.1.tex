
\scnheader{качество кибернетической системы}
\scnrelfromlist{cвойство-предпосылка}{качество физической оболочки кибернетической системы мы качество решателя задач кибернетической системы; 
качество решателя задач кибернетической системы
\newline
\scnaddlevel{1}
\scnrelfrom{cвойство-предпосылка}{качество информации, хранимой в памяти кибернетической системы }
\scnaddlevel{-1};
качество информации, хранимой в памяти кибернетической системы;
гибридность кибернетической системы
\scnaddlevel{1}
\scnidtf{степень многообразия (1) видов знаний, хранимых в памяти кибернетической системы, (2) используемых моделей решение задач, (3) видов сенсоров и эффекторов}
\scnrelfromlist{частное свойство}{многообразие видов знаний, хранимых в памяти кибернетической системы;
многообразие моделей решения задач;
многообразие видов сенсоров и эффекторов}
\scnaddlevel{-1};
приспособленность кибернетической системы к её совершенствованию;
производительность кибернетической системы
\scnaddlevel{1}
\scnidtf{cкорость решения задач кибернетической системы}
\scnaddlevel{-1};
надежность кибернетической системы;
социализация кибернетической системы
}

\scnheader{гибридность кибернетической системы}
\scnrelfromlist{частное свойство}{многообразие видов знаний, хранимых в памяти кибернетической системы;
многообразие моделей решения задач;
многообразие видов сенсоров и эффекторов}

\scnheader{гибридная кибернетическая система}
\scnidtf{кибернетическая система, использующая многообразие рецепторных и/или эффекторных подсистем, и/или многообразие видов обрабатываемой информации, и/или многообразие способов решения задач}
\scnsuperset{гибридная компьютерная система}
\scnaddlevel{1}
\scnidtf{\textit{компьютерная система}, способная решать \textit{комплексные задачи}, требующие использования многообразия различных видов обрабатываемой информации и различных \textit{моделей решения задач}}
\scnaddlevel{-1}

\scnheader{приспособленность кибернетической системы к её совершенствованию}
\scnidtf{приспособленность кибернетической системы к эволюции, к повышению уровня своего качества}
\scnrelfromset{комплекс свойств-предпосылок}{обучаемость кибернетической системы
\scnaddlevel{1}
\scnidtf{способность кибернетической системы самостоятельно повышать уровень своего качества}
\scnidtf{способность кибернетической системы к самоэволюции, саморазвитию, устранению своих недостатков}
\scnaddlevel{-1};
приспособленность кибернетической системы к её совершенствованию, осуществляемому извне
\scnaddlevel{1}
\scnidtf{приспособленность кибернетической системы к её совершенствованию, осуществляемому внешними субъектами}
\scnidtf{удобство совершенствования кибернетической системы для её создателей}
\scnnote{Важнейшим фактором качества каждой \textit{технологии разработки кибернетических систем} является гибкость и стратифицированность разрабатываемых кибернетических систем при их совершенствовании, осуществляемом руками разработчиков}
\scnaddlevel{-1}
}
\scnrelfromset{комплекс свойств-предпосылок}{гибкость кибернетической системы;
стратифицированность кибернетической системы
\scnaddlevel{1}
\scnidtf{уровень стратифицированности кибернетической системы}
\scnidtf{качество разделения (декомпозиции) кибернетической системы на в достаточной степени независимые части (компоненты), определенные виды изменений которых не предполагают внесения изменений в другие части системы}
\scnaddlevel{-1}
}

\scnheader{гибкость кибернетической системы}
\scnidtf{реконфигурируемость кибернетической системы}
\scnidtf{модифицируемость кибернетической системы}
\scnidtf{реформируемость кибернетической системы }
\scnidtf{трансформируемость кибернетической системы}
\scnidtf{пластичность кибернетической системы}
\scnidtf{легкость реализации различного вида изменений в кибернетической системе}
\scnidtf{степень трансформенности кибернетической системы}
\scnidtf{простота и многообразие внесения изменений в кибернетическую систему}
\scnidtf{модифицируемость кибернетической системы}
\scnidtf{трансформируемость кибернетической системы} 
\scnidtf{реконфигурируемость кибернетической системы} 
\scnidtf{приспособленность к реинжинирингу кибернетической системы}
\scnidtf{мягкость}
\scnidtf{softness}
\scnidtf{приспособленность к внесению изменений}
\scnidtf{\uline{легкость} внесения изменений}
\scnnote{Чем легче вносить изменения в кибернетическую систему, тем выше скорость ее эволюции}
\scnnote{изменения могут вноситься (1) полностью самостоятельно (без учителя) (2) с помощью учителя-тренера ("терапевта"{}) путем создания определенных условий для совершенствования системы (3) "хирургически"{} -- путем непосредственного вмешательства извне (например, вмешательства разработчика)}
\scnnote{Чем выше \textit{гибкость кибернетической системы} -- тем ниже трудоемкость и меньше сроки внесения различных изменений в систему в направлении ее совершенствования (приближения к идеалу)}
\scnidtf{простота внесения изменений в кибернетическую систему и многообразие видов возможных таких изменений}
\scnrelfromset{комплекс свойств-предпосылок}{простота внесения изменений в кибернетическую систему
\newline
\scnaddlevel{1}
\scnrelfrom{свойство-предпосылка}{стратифицированность кибернетической системы}
\scnaddlevel{-1};
многообразие, возможных изменений, вносимых в кибернетическую систему
}
\scnrelfromset{комплекс частных свойств}{гибкость информации, хранимой в памяти кибернетической системы;
гибкость решателя задач кибернетической системы;
гибкость физической оболочки кибернетической системы
\scnaddlevel{1}
\newline
\scnrelfrom{частное свойство}{гибкость памяти кибернетической системы}
\scnaddlevel{-1};
гибкость интерфейса кибернетической системы}
\scnrelfromset{комплекс частных свойств}{гибкость кибернетической системы при ее совершенствовании, осуществляемом извне;
гибкость возможных самоизменений кибернетической системы
\scnaddlevel{1}
\newline
\scnrelto{свойство-предпосылка}{обучаемость кибернетической системы}
\scnaddlevel{-1}}

\scnheader{приспособленность кибернетической системы к её совершенствованию, осуществляемому извне}
\scnidtf{приспособленность кибернетическиой системы к "хирургическим"{} методам её совершенствования, реализуемым разработчиками}
\scnidtf{насколько легко осуществлять обновление, перепроектирование, тестирование, ремонт (исправление ошибок) кибернетической системы}
\scnrelfromlist{свойство-предпосылка}{простота внесения изменений в кибернетическую систему, реализуемых извне
\newline
\scnaddlevel{1}
\scnrelfrom{свойство-предпосылка}{стратифицированность кибернетической системы}
\scnaddlevel{-1};
многообразие возможных изменений кибернетической системы, реализуемых извне}

\scnheader{производительность кибернетической системы}
\scnidtf{быстродействие кибернетической системы}
\scnidtf{интегральная оценка скорости решения задач, время реакции кибернетической системы на задачные ситуации}
\scnrelfromlist{частное свойство}{производительность базового интерпретатора логико-семантической модели компьютерной системы;
качество используемых кибернетической системой методов и моделей решения задач}

\scnheader{надежность кибернетической системы}
\scnidtf{способность кибернетической системы при соответствующих условиях ее функционирования сохранять (и, точнее, не снижать) уровень всех свойств и способностей, определяющих общее (комплексное) качество кибернетической системы}
\scnrelfromlist{свойство-предпосылка}{безотказность кибернетической системы; долговечность кибернетической системы; "ремонтопригодность"{} кибернетической системы
\newline
\scnaddlevel{1}
\scnrelfrom{основной sc-идентификатор}{"ремонтопригодность"{} кибернетических систем}
\scnaddlevel{1}
\scnnote{Здесь слово "ремонтопригодность"{} взято в кавычки, т.к. речь идет не только об искусственных (технических) кибернетических системах}
\scnaddlevel{-1}
\scnaddlevel{-1}
}
