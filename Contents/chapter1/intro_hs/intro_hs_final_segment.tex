\scnsegmentheader{Итоговый сегмент Раздела Предметная область и онтология кибернетических систем}


\scnstartsubstruct

\scnheader{качество кибернетической системы}
\scntext{резюме}{Уровень качества кибернетических систем определяется достаточно большим набором свойств (параметров, характеристик) кибернетических систем, каждое из которых определяет уровень качества кибернетической системы в соответствующем аспекте (ракурсе), указывая (задавая) уровень развития конкретных  способностей и возможностей кибернетической системы. При этом важно подчеркнуть следующее:
	\begin{scnitemize}
	\item существенное значение имеет не столько сам набор свойств, а иерархия этих свойств, позволяющая уточнять (детализировать) направления проявления (реализации) каждого свойства;
	\item существенное значение также имеет \uline{баланс} уровней развития различных свойств -- вклад разных свойств, обеспечивающих (определяющих) значение одного и того же свойства более высокого уровня иерархии, а значение этого свойства более высокого уровня может быть разным. Из этого следует, что не всегда следует акцентировать внимание на развитие некоторых свойств (характеристик). Нужен целостный, коллективный подход;
	\item рассмотренная иерархия свойств кибернетических систем является общей как для естественных, так и для искусственных кибернетических систем;
	\item приведенная иерархическая детализация свойств кибернетических систем (с помощью отношения ``\textit{частное свойство*}'' и отношения ``\textit{свойство-предпосылка*}''), определяющих качество таких систем, (1) дает возможность четко определить направления совершенствования (развития) кибернетических систем и (2) дает ориентир (систему критериев) для обоснования конкретных предложений по совершенствованию компьютерных систем, а также для сравнения различных альтернативных предположений;
	\item особое значение для развития кибернетических систем имеют такие их свойства, как стратифицированность, рефлексивность и социализация;
	\item важное значение имеет не только совершенствование кибернетических систем в соответствии с иерархической системой их свойств, но и совершенствование (в том числе, детализация) самой этой иерархической системы свойств. 
	\end{scnitemize}}


\bigskip

\scnendstruct \scninlinesourcecommentpar{Завершили Сегмент ``\textit{Итоговый сегмент Раздела Предметная область и онтология кибернетических систем}''}