\begin{SCn}

\textbf{\textit{Общий объем задач, решаемых кибернетической системой}}

	\scnidtf{общий объем задач, которые кибернетическая система способна решать}
	
	\scnidtf{общий объем (множество), задач (действий), которые кибернетическая система способна (может, умеет) решать (выполнять) в заданный (в том числе, в текущий) момент}
	
\scnrelfromlist{свойство-предпосылка}{мощность языка представления задач, решаемых кибернетической системой}
			
\textbf{\textit{мощность языка представления задач, решаемых кибернетической системой}}
	\scnidtf{мощность языка спецификации (описания) различного вида действий, выполняемых кибернетической системой}
		
\scnnote{мощность языка представления задач прежде всего определяется многообразием видов представляемых задач (многообразием видов описываемых действий)}
		
\scnreltolist{свойство-предпосылка}{многообразие видов задач, решаемых кибернетической системой}
	
\scnrelfromlist{частное свойство}{
	мощность языка представления задач, решаемых в памяти кибернетической системы\\
	\scnreltolist{свойство-предпосылка}{многообразие видов задач, решаемых в памяти кибернетической системы}
	;мощность языка представления задач, решаемых во внешней среде кибернетической системы\\
	\scnreltolist{свойство-предпосылка}{многообразие видов задач, решаемых во внешней среде кибернетической системы}
	;мощность языка представления задач, решаемых в рамках физической оболочки кибернетической системы\\
	\scnreltolist{свойство-предпосылка}{многообразие видов задач, решаемых в рамках физической оболочки кибернетической системы}}
			
\textbf{\textit{многообразие видов задач, решаемых кибернетической системой}}		
			
	\scnidtf{многообразие видов действий, которые кибернетическая система способна выполнять}
	
\scnnote{подчеркнем, что каждая задача есть спецификация соответствующего (описываемого) действия. Поэтому рассмотрение многообразия видов задач, решаемых кибернетической системой, полностью соответствует многообразию видов деятельности, осуществляемой этой системой. Важно заметить, что есть виды деятельности кибернетической системы, которые определяют качество и, в частности, уровень интеллекта кибернетической системы}
		
\scnrelfromlist{свойство-предпосылка}{мощность языка представления задач в памяти кибернетической системы}
		
\scnrelfromset{комплекс частных свойств}{
	многообразие видов задач, решаемых в памяти кибернетической системы
	;многообразие видов задач, решаемых во внешней среде кибернетической системы
	;многообразие видов задач, решаемых в рамках физической оболочки кибернетической системы}}
		
%\scnmakeset{sdfd; sdfsdf; sdf}
		
\textbf{\textit{способность кибернетической системы к анализу решаемых задач}}
\scnidtf{способность кибернетической системы осмысливать (ведать) то, что она творит}
\scnidtf{способность анализировать свои цели и, соответственно, решаемые задачи на предмет:
\begin{scnitemize}
	\item сложности достижения;
	\item целесообразности достижения (нужности, важности, приоритетности);
	\item соответствия цели существующим нормам (правилам) соответствующей деятельности
\end{scnitemize}}
			
\textbf{\textit{способность кибернетической системы к решению задач, методы решения которых ей в текущий момент известны
}}	
		
\scnnote{Указанными методами могут быть не только алгоритмы, но также и функциональные программы, продукционные системы, логические исчисления, генетические алгоритмы, искусственные нейронные сети различного вида}
			
\scnrelfromlist{свойство-предпосылка}{
	способность кибернетической системы к поиску хранимых в своей памяти методов решения инициированных задач
	;способность кибернетической системы к интерпретации хранимых в своей памяти методов решения задач}
			
\textbf{\textit{способность кибернетической системы к решению задач, методы решения которых ей в текущий момент не известны
}}	

\scnidtf{способность кибернетической системы к решению задач, для которых не найдены соответствующие (релевантные) им методы их решения}

\scnidtf{способность кибернетической системы строить цепочку "цель-план достижения цели-система действий"}
		
\scnnote{задачи, для которых не находятся соответствующие им методы, решаются с помощью метаметодов (стратегий) решения задач, направленных:
	\begin{scnitemize}
		\item на генерацию нужных исходных данных (нужного контекста), необходимых для решения каждой задачи;
		\item на генерацию плана решения задачи, описывающего сведение исходной задачи к подзадачам (до тех подзадач, методы решения которых системы известны);
		\item на сужение области решения задачи (на сужения контекста задачи, достаточного для ее решения).
	\end{scnitemize}}
				
\textbf{\textit{множество навыков, используемых кибернетической системой
}}			

\scnidtf{объем и многообразие навыков, приобретенных кибернетической системой к текущему моменту (с помощью учителей-разработчиков или полностью самостоятельно)}
\scnidtf{возможности, навыки, приобретенные кибернетической системой}
\scnidtf{опыт, приобретенный кибернетической системой}
			
\scnnote{новые навыки могут приобретаться кибернетической системой либо полностью самостоятельно, либо с помощью учителей, которые в простейшем случае просто сообщают обучаемой системе полностью сформулированные навыки. Для компьютерных систем учителями является их разработчики.}
				
\scnrelfromlist{частное свойство}{
	множество методов решения задач, используемых кибернетической системой
	;множество моделей решения задач, используемых кибернетической системой
	;мощность языка представления в памяти кибернетической системы методов и моделей решения задач}
				
\textbf{\textit{множество методов решения задач, используемых кибернетической системой}}			

\scnidtf{множество методов решения задач, используемых кибернетической системой и хранимых в ее памяти}
\scnreltolist{частное свойство}{многообразие видов знаний, хранимых в памяти кибернетической системы}
				
\textbf{\textit{метод решения задач}}

\scnexplanation{метод решения задачи это вид знаний, хранимых в памяти кибернетической системы и содержащих информацию, которой достаточно либо для сведения каждой задачи из соответствующего класса к полной системе подзадач, решение которых гарантирует решение исходной задачи, либо для окончательного решения этой задачи из указанного класса задач}
				
\textbf{\textit{множество моделей решений задач, используемых кибернетической системой}}
	\scnidtf{способность кибернетической системы к использованию различных видов методов решения задач, соответствующих различным моделям решения задач}
	\scnidtf{многообразие методов решения задач, используемых кибернетической системой}
	\scnrelfromlist{свойство-предпосылка}{
		мощность языка представления в памяти кибернетической системы методов и моделей решения задач}
			
\textbf{\textit{множество моделей решения задач, используемых кибернетической системой}}

\scnrelfromset{примечание}{\scnheaderlocal{следует отличать*}
\scnhaselementset{
	вид задач
	;модель решения задач\\
	\scnaddlevel{1}
	\scnexplanation{каждая модель решения задач задается 1 языком, обеспечивающим представление в памяти кибернетической системы некоторого класса методов решения задач, и 2 интерпретатором указанных методов, определяющим операционную семантику указанного языка}
	\scnaddlevel{-1}
	;метод решения задач
	;класс задач\\
	\scnaddlevel{1}
	\scnidtf{множество всех тех и только тех задач, которые решаются с помощью соответствующего метода}
	\scnaddlevel{-1}}}
\end{SCn}