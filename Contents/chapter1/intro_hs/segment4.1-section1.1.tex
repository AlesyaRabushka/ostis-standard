
\bigskip
\scnsegmentheader{Комплекс свойств, определяющих уровень интеллекта кибернетической системы}
\scnstartsubstruct

\scnheader{интеллект}
\scniselement{свойство}
\scniselement{упорядоченное свойство}
\scnidtf{уровень (степень, величина) интеллекта кибернетической системы}
\scnidtf{Семейство классов \textit{кибернетических систем}, обладающих эквивалентным (одинаковым) уровнем интеллекта -- от низкого до высокого уровня интеллекта}
\scnidtf{свойство кибернетических систем, характеризующее эффективность их взаимодействия со своей средой (средой их "жизнедеятельности"{})}
\scnrelfrom{область определения}{кибернетическая система}
\scnexplanation{С формальной точки зрения интеллектуальность -- это семейство классов кибернетических систем, в каждый из которых входят кибернетические системы, эквивалентные по уровню и характеру проявления интеллектуальных свойств (в том числе способностей).\\
Таким образом, характер (вид) интеллектуальных свойств кибернетических систем и уровень их развития для разных кибернетических систем может быть разным. В соответствии с этим кибернетические системы можно сравнивать между собой.}
\scnnote{Основным свойством (характеристикой, качеством, параметром) кибернетической системы является уровень (степень) ее интеллекта, который является \uline{интегральной} характеристикой, определяющей уровень эффективности взаимодействия кибернетической системы со средой своего существования.}
\scnidtf{комплексное свойство (качество) кибернетической системы, определяющее уровень ее "выживаемости"{} во внешней среде и предполагающее возможность воздействия на эту среду и даже возможность ее преобразования}
\scnidtf{интеллектуальный потенциал кибернетической системы}
\scnidtf{спектр знаний, навыков и способностей к обучению кибернетической системы}
\scnidtf{интеллектуальность кибернетической системы}
\scnnote{Процесс эволюции \textit{кибернетических систем} следует рассматривать как процесс повышения уровня их качества по целому ряду свойств (характеристик) и, в первую очередь, как процесс повышения уровня их \textit{интеллекта}. При этом можно говорить об эволюции каждой конкретной \textit{кибернетической системы} в процессе своей "жизнедеятельности"{}, а также об эволюции целого класса \textit{кибернетических систем}, когда новые экземпляры этого класса являются более интеллектуальными, чем их предшественники. В таком аспекте, в частности, можно рассматривать эволюцию \textit{компьютерных систем} (искусственных кибернетических систем).}
\scnnote{Очень важно уточнить, какими иными свойствами \textit{кибернетических систем} определяется уровень и характер их интеллектуальности. Подчеркнем, что \uline{любая} \textit{кибернетическая система} обладает соответствующим уровнем интеллектуальности. Пусть даже и достаточно низким. Существенным является уточнение того, за счет чего уровень интеллектуальности \textit{кибернетической системы} может быть повышен. Нет смысла проводить четкую границу между \textit{интеллектуальными кибернетическими системами} и неинтеллектуальными. Но есть смысл уточнять направления повышения уровня интеллектуальности \textit{кибернетических систем.}}
\scntext{эпиграф}{Никто не может провести линию, отделяющую атмосферу от космоса, или черту, за которой начинается жизнь, или границу электронного облака. Все дело в степени проявления свойства.}
	\scnaddlevel{1}
	\scnrelfrom{автор}{Барт Коско}
	\scnaddlevel{-1}
\scnnote{Прежде, чем говорить о требованиях, предъявляемых к \textit{технологии проектирования и производства интеллектуальных компьютерных систем (искусственных кибернетических систем}, обладающих высоким уровнем \textit{интеллекта)}, необходимо уточнить (детализировать) \textit{свойства}, присущие указанным системам и являющиеся предпосылками, обеспечивающими высокий уровень \textit{интеллекта}. Подчеркнем, что указанные \textit{свойства}, уточняющие (детализирующие, обеспечивающие, определяющие) \textit{свойства} \scnbigspace \textit{интеллектуальных систем}\scnbigspace (\textit{свойства}, определяющие уровень \textit{интеллекта} этих систем) должны быть общими как для искусственных кибернетических систем (\textit{компьютерных систем}), так и для \textit{естественных кибернетических систем.}}
\scnidtf{интегральное качество информационного обеспечения и информационных процессов в кибернетической системе}
\scnidtf{интегральное качество кибернетической системы, определяемое:
	\begin{scnitemize}
		\item уровнем ее образованности -- качеством накопленных к заданному моменту знаний и умений (навыков);
		\item уровнем ее обучаемости -- способностью \uline{самостоятельно} повышать уровень свой образованности.
	\end{scnitemize}}
\scnrelfromlist{свойство-предпосылка}{образованность кибернетической системы
;обучаемость кибернетической системы
;социализация кибернетической системы\\
	\scnaddlevel{1}
	\scnnote{интеллект \textit{кибернетической системы}, как и лежащий в его основе познавательный процесс, выполняемый кибернетической системой, имеет социальный характер, поскольку наиболее эффективно формируется и развивается в форме взаимодействия \textit{кибернетической} системы с другими \textit{кибернетическими системами}}
	\scnaddlevel{-1}}