
\bigskip
\scnsegmentheader{Комплекс свойств, определяющих качество информации, хранимой в памяти кибернетической системы}

\scnstartsubstruct

\scnheader{информация}
\scnidtf{информационная конструкция}
\scnidtf{информационная модель, состоящая из некоторого множества различных \textit{знаков}, обозначающих моделируемые (описываемые) \textit{сущности} любого вида и, в частности, \textit{знаков}, обозначающих различного вида \textit{связи} между \textit{знаками} описываемых \textit{сущностей} (такие \textit{связи} чаще всего являются отражениями (моделями) \textit{связей} между \textit{сущностями}, которые обозначаются связываемыми \textit{знаками})}
\scnaddlevel{1}
\scnnote{Подчеркнем, что \textit{связи} между \textit{знаками} описываемых \textit{сущностей} сами также могут быть описываемыми \textit{сущностями}, но для этого указанные \textit{связи} в рамках информационной модели должны быть представлены своими \textit{знаками}. Не все \textit{связи} между \textit{знаками} являются описываемыми \textit{сущностями}. Такими неописываемыми связями являются связи инцидентности знаков.} 
\scnaddlevel{-1}
\scnidtf{конфигурация знаков}
\scnidtf{знаковая конструкция} 
\scnidtf{текст}
\scnidtf{описание (отражение) некоторого множества (1) первичных сущностей, (2) понятий, (3) связей между ними, (4) связей между связями, (5) фрагментов данного описания, (6) связей между этими фрагментами}
\scnsuperset{дискретная информационная конструкция} 
\scnaddlevel{1}
\scnidtf{информационная конструкция, у которой все входящие в неё знаки имеют чёткие границы}

\scnsuperset{дискретная информационная конструкция, у которой входящие в неё знаки имеют \uline{условную} структуру}
\scnaddlevel{-1}  

\scnsubdividing {внутренняя информационная конструкция
\scnaddlevel{1}
\scnidtf{информационная конструкция, хранимая в памяти некоторой кибернетической \textit{системы}, и непосредственно интерпретируемая (понимаемая) решателем задач этой системы}
\scnaddlevel{-1}
;внешняя информационная конструкция   
\scnaddlevel{1}
\scnidtf{информационная конструкция, представленная на каком-либо внешнем носителе или в памяти другой кибернетической системы} 
\scnaddlevel{-1}
;файл   
\scnaddlevel{1}
\scnidtf{первичный электронный образ некоторой внешней информационной конструкции} 
\scnaddlevel{-1}}
\scnexplanation {Начало раздела \ref{intro_lang}}   
\scnidtf{информационная модель} 
\scnidtf{информационная модель (отражение, описание) некоторого множества связей между некоторым описываемыми (рассматриваемыми, исследуемыми, изучаемыми) сущностями}
\scntext{определение}{Множество всевозможных информационных конструкций (понятие информационной конструкции) представляет собой множество, на котором задано 
	\begin{scnitemize}
	\item Отношение \uline{синтаксической} эквивалентности и, соответственно, семейство классов синтаксической эквивалентности информационных конструкций
	\item Отношение \uline{семантической} эквивалентности и, ответственно, семейство классов семантической эквивалентности информационных конструкций 
	\item Отношение \uline{логической} эквивалентности и, соответственно, семейство классов логической эквивалентности информационных конструкций.
\end{scnitemize} 
При этом можно говорить об инварианте каждого класса синтаксически эквивалентных информационных конструкций, об инварианте каждого класса семантически эквивалентных информационных конструкций и об инварианте каждого класса логически эквивалентных информационных конструкций 
синтаксически эквивалентные информационные конструкции могут отличаться вариантами изображения букв (различным почерком, разными шрифтами), вариантами "разрезания"{} текста на страницы и на строчки.
Семантически эквивалентные информационные конструкции могут отличаться разными именами, обозначающими одни и те же сущности, разным порядком размещения этих имён.}
 
\scnheader{денотационная семантика информационной конструкции}\\
\scnexplanation{Каждая информационная конструкция имеет денотационную семантику, описывающую то, как связаны входящие в информационную конструкцию знаки с соответствующими им денотатами (т.е. сущностями, обозначаемыми этими знаками}

\scnheader{сенсорная информация}\\
\scnsubset{информация} \\
\scnidtf{первичная информация, приобретаемая кибернетической системы с помощью её сенсоров (рецепторов)}
\scnidtf{первичная информация}
\scnnote{Подчеркнем, что \textit{сенсорная информация} \scnbigspace \textit{кибернетической системы} с точки зрения её \textit{денотационной семантики} является простейшим видом \textit{знаковой конструкции}, в которой \textit{внешняя среда} \scnbigspace \textit{кибернетической системы} описывается
\begin{scnitemize}
	\item путём задания параметрического пространства (множество параметров, признаков, \textit{свойства}, характеристик), с помощью которого описываются состояние элементарных (атомарных) фрагментов \textit{внешней среды}, которые непосредственно являются смежными (соприкасаются с) чувствительными поверхностями \textit{сенсоров кибернетической системы}; 
	\item путём пространственной декомпозиции наблюдаемой \textit{внешней среды} с выделением указанных выше элементарных фрагментов этой среды (элементарных с "точки зрения"{} \textit{сенсоров кибернетической системы}) и с явным описанием пространственных связей между указанными элементарными фрагментами (эти связи соответствует пространственным связям между сенсорами);
	\item путём темпоральной декомпозиции наблюдаемой \textit{внешней среды}, которая предполагает фиксацию моментов времени для каждого события по изменению состояния измеряемого параметра каждого элементарного фрагмента наблюдаемой \textit{внешней среды} 
\end{scnitemize}}

\scnnote{Качество (в частности, информативность) \textit{сенсорной информации} обеспечивается:
\begin{scnitemize}
\item качеством используемого параметрического пространства 
\begin{scnitemizeii}
\item многообразием видов \textit{сенсоров}, т.е. многообразием параметров (свойств), с помощью которых описывается внешняя среда
\item информативностью каждого из указанных параметров 
\item целостностью (полнотой, достаточностью) всего набора рассматриваемых параметров 
\item отсутствием избыточности в наборе этих параметров 
\end{scnitemizeii}
\item общим количеством сенсоров и количеством сенсоров, соответствующих каждому параметру
\item способностью кибернетической системы перемещать сенсоры в пространстве 
\end{scnitemize}
}
\scnnote{\textit{сенсорная информация} обеспечивает формирование первичного описания состояния и динамики изменения не только \textit{внешней среды кибернетической системы}, но также и её физической оболочки, которую можно рассматривать как часть всей \textbf{\textit{физической среды кибернетической системы}}, противопоставляя такую \textit{физическую среду кибернетической системы} её внутренней (информационной, \uline{абстрактной}) среде, в которой хранится и обрабатывается \textit{информация}, используемая \textit{кибернетической системой}. Указанную абстрактную внутреннюю среду кибернетической системы будем называть \textbf{\textit{абстрактной памятью кибернетической системы}}.}

\scnheader{язык}  
\scnidtf{множество информационных конструкции, построенных по общим синтаксическим и семантическим правилам}
\scnsuperset{внутренний язык кибернетической системы} 
\scnaddlevel{1}
\scnidtf{язык, используемый кибернетической системой для представления информации, хранимой в её памяти}
\scnaddlevel{-1}

\scnheader{информация, хранимая в памяти кибернетической системы} 
\scnidtf{совокупность \uline{всей} информации, хранимой в памяти кибернетической системы}
\scnsubset{информация} 

\scnheader{качество информации, хранимой в памяти кибернетической системы} 
\scnidtf{качество знаний, приобретенных кибернетической системой к текущему моменту}
\scnidtf{уровень качества хранимой информации} 
\scnidtf{качество информационной модели среды кибернетической системы, хранимой в её памяти}
\scnidtf{уровень качества хранимых в памяти кибернетической системы внутренней информационной модели среды существования (жизнедеятельности) этой кибернетической системы} 
\scnidtf{интегральное качество знаний, накопленных кибернетической системой к текущему моменту} 
\scnidtf{степень приближения информации, хранимой в памяти кибернетической системы к качественной информационной модели той среды, в которой существует кибернетическая система, к систематизированной базе знаний, описывающей все свойства этой среды, необходимые для функционирования этой кибернетической системы}
\scnidtf{качество хранимой в памяти кибернетической системы информационной модели среды жизнедеятельности этой системы}
\scnnote{Качество информационной модели среды "обитания"{} кибернетической системы, в частности, определяется 
\begin{scnitemize}
	\item корректностью (адекватностью) этой модели, 
	\item полнотой -- достаточностью находящейся в ней информации для эффективного функционирования кибернетической системы;
	\item структурированностью, систематизированностью.
\end{scnitemize}
Важнейшим этапом эволюции информационной модели среды кибернетической системы является переход от недостаточно полной и несистематизированной информационные модели среды к \textit{базе знаний}. Именно поэтому важнейшим этапом повышения уровня интеллектуальности компьютерной систем является переход от традиционных компьютерных систем к компьютерным системам, основанным на знаниях.}
\scnrelfrom{комплекс свойств-предпосылок}{не-фактор} 
\scnrelfromlist{свойство-предпосылка}{семантическая мощность языка представления информации в памяти кибернетической системы; 
объём информации, в память кибернетической системы; 
степень конвергенции и интеграции различного вида знаний, хранимых в памяти кибернетической системы; 
стратифицированность информации, хранимой в памяти кибернетической системы;
простота и локальность выполнения семантически целостных операций над информацией, хранимой в памяти кибернетической системы}
