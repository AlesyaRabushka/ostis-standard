
\bigskip
\scnsegmentheader{Уточнение понятия Кибернетической системы}
\scnstartsubstruct

\scnheader{кибернетическая система}
\scnidtf{cистема, которая способна \uline{управлять} своими \uline{действиями}, адаптируясь к изменениям состояния внешней среды (среды своего "обитания") в целях самосохранения (сохранения своей целостности и "комфортности"\ существования путем удержания своих "жизненно"\ важных параметров в определенных рамках "комфортности") и/или в целях формирования определенных реакций (воздействий на внешнюю среду) в ответ на определенные стимулы (на определенные ситуации или события во внешней среде), а также которая способна (при соответствующем уровне развития) эволюционировать в направлении:
\begin{scnitemize}
    \item изучения своей внешней среды как минимум для предсказания последствий своих воздействий на внешнюю среду, а также для предсказания изменений внешней среды, которые не зависят от собственных воздействий;
    \item изучения самого себя и, в частности, своего взаимодействия с внешней средой;
    \item создания технологий (методов и средств), обеспечивающих изменение своей внешней среды (условий своего существования) в собственных интересах.
\end{scnitemize}
}

\scnheader{кибернетическая система}
\scnidtf{адаптивная система}
\scnidtf{целенаправленная (целеустремленная) система}
\scnidtf{активный субъект самостоятельной деятельности}
\scnidtf{материальная сущность, способная целенаправленно (в своих интересах) воздействовать  на среду своего обитания  как минимум для сохранения своей целостности, жизнеспособности, безопасности}

\scnnote{Уровень (степень) адаптивности, целенаправленности, активности у систем, основанных на обработке информации может быть самым различным.}

\scnheader{кибернетическая система}
\scnidtf{система, организация функционирования которой основано на обработке информации о той среде, в которой существует эта система}

\scnheader{кибернетическая система}
\scnidtf{материальная сущность, способная к активной  целенаправленной деятельности, которая  на определенном уровне развития указанной сущности становится "осмысленной", планируемой, преднамеренной деятельностью}
\scnidtf{субъект, способный на самостоятельное выполнение некоторых "внутренних"\ и "внешних"\ действий либо порученных извне, либо инициированных самим субъектом}
\scnidtf{сущность, способная выполнять роль субъекта деятельности}

\scnheader{кибернетическая система}
\scnidtf{естественная или искусственно созданная система, способная мониторить и анализировать свое состояние и состояние окружающей среды, а также способная достаточно активно воздействовать на собственное на собственное состояние и на состояние окружающей среды}
\scnidtf{система, способная в достаточной степени самостоятельно взаимодействовать со своей средой , решая различные задачи}
\scnidtf{система, основанная на обработке информации}
	
\scnrelto{ключевой знак}{Глушков В. М. Кибер. - 1979 ст}
статья библиографическая ссылка
\scnauthorcomment{не смогла разобраться с оформлением}

\bigskip
\scnsegmentheader{Типология кибернетических систем}
\scnstartsubstruct

\scnheader{кибернетическая система}

\scnrelfrom{разбиение}{Признак естественности или искусственности кибернетических систем}
\scnaddlevel{1}
\scneqtoset{естественная кибернетическая система\\
    \scnaddlevel{1}
    \scnidtf{кибернетическая система естественного происхождения}
    \scnsuperset{человек}
    \scnaddlevel{-1}
;компьютерная система\\
    \scnaddlevel{1}
    \scnidtf{искусственная кибернетическая система}
    \scnidtf{кибернетическая система искусственного происхождения}
    \scnidtf{технически реализованная кибернетическая система}
    \scnaddlevel{-1}
;симбиоз естественных и искусственных кибернетических систем\\
    \scnaddlevel{1}
    \scnidtf{кибернетическая система, в состав которой входят компоненты как естественного, так и искусственного происхождения}
    \scnsuperset{сообщество компьютерных систем и людей}
    \scnaddlevel{-1}}
\scnaddlevel{-1}

\scnheader{искусственная сущность}
\scnidtf{артефакт}
\scnidtf{сущность, являющаяся либо результатом человеческой деятельности, либо частью самой этой деятельности}
\scnidtf{сущность искусственного происхождения}
\scnidtf{антропогенная сущность}
\scnsuperset{научно-техническое знание}
\scnaddlevel{1}
\scnidtf{знание, приобретенное в результате научно-технической деятельности человеческого общества}
\scnaddlevel{-1}
\scnsuperset{материальная искусственная сущность}
\scnaddlevel{1}
\scnsuperset{компьютерная система}
\scnaddlevel{-1}

\scnheader{компьютерная система}
\scnidtf{искусственная кибернетическая система}
\scnnote{Особенностью компьютерных систем является то, что они могут выполнять "роль"\ не только продуктов соответствующих действий по реализации этих систем, но и сами являются \textit{субъектами*}, способными выполнять (автоматизировать) широкий спектр действий. При этом интеллектуализация этих систем существенно расширяет этот спектр. \textit{См. интеллектуальная компьютерная система}.}

\scnheader{компьютерная система}
\scnidtf{технически реализованная кибернетическая система}
\scnidtf{искусственная кибернетическая система}
\scnsubset{кибернетическая система}
\scnsuperset{современная компьютерная система традиционного вида}
\scnsuperset{современная интеллектуальная компьютерная система}
\scnsuperset{интеллектуальная компьютерная система следующего поколения}
\scnaddlevel{1}
\scnsuperset{ostis-система}
\scnnote{Основной тенденцией эволюции компьютерных систем является повышение уровня их интеллектуальности.}
\scnrelfromset{особенность}{Ориентация на принципиально новые компьютеры;Cущественное повышение уровня интеллекта.}
\scnaddlevel{-1}

\scnheader{компьютерная система}
\scnrelfrom{разбиение}{Структурная классификация кибернетических систем}
\scnaddlevel{1}
\scneqtoset{простая кибернетическая система\\
;индивидуальная кибернетическая система\\
;многоагентая система\\
\scnaddlevel{1}
\scnsubdividing{
одноуровневый коллектив кибернетических систем
    \scnaddlevel{1}
    \scnidtf{многоагентная система, агентами которой не могут быть многоагентные системы}
    \scnaddlevel{-1}
;иерархический коллектив кибернетических систем
    \scnaddlevel{1}
    \scnidtf{многоагентная система, по крайней мере одним  агентом которой является многоагентная система}
    \scnaddlevel{-2}}
\scnsubdividing{коллектив из простых кибернетических систем\\
\scnaddlevel{1}
\scnnote{Такой коллектив может быть либо одноуровневым, либо иерархическим коллективом}
\scnaddlevel{-1};
коллектив из индивидуальных кибернетических систем;коллектив из индивидуальных и простых кибернетических систем}
}
\scnaddlevel{-1}

\scnheader{кибернетическая система}
\scnrelfrom{разбиение}{Классификация кибернетических систем по признаку наличия надсистемы и роли в рамках этой надсистемы}
\scnaddlevel{1}
\scneqtoset{кибернетическая система, не являющаяся частью никакой другой кибернетической системы\\
\scnaddlevel{1}
\scnidtf{кибернетическая система, не имеющая надсистем}
\scnaddlevel{-1}
;кибернетическая система, встроенная в индивидуальную кибернетическую систему\\
;агент многоагентной системы\\
\scnaddlevel{1}
\scnidtf{кибернетическая система, являющаяся агентом одной или нескольких многоагентных систем}
\scnaddlevel{-1}
}
\scnaddlevel{-1}

\scnheader{простая кибернетическая система}
\scnidtf{\textit{кибернетическая система}, уровень развития которой находится ниже уровня \textit{индивидуальных кибернетических систем} и которая является специализированным средством обработки информации специализированным решателем задач, реализующим (интерпретирующим) чаще всего один \textit{метод} решения задач и, соответственно, решающим только заданный \textit{класс задач}}
\scnidtf{специализированный \textit{решатель задач}}
\scnnote{\textit{простая кибернетическая система} может быть \textit{компонентом*}, встроенным в \textit{индивидуальную кибернетическую систему}, а также может быть \textit{агентом*} \scnbigskip \textit{многоагентной системы}, являющейся коллективом из простых кибернетических систем}

\scnheader{индивидуальная кибернетическая система}
\scnidtf{условно выделенный уровень развития \textit{кибернетических систем}, в основе которого лежит переход от \textit{специализированного решателя задач к индивидуальному решателю}, обеспечивающему интерпретацию произвольного (нефиксированного) набора \textit{методов} (программ) решения задач при условии, если эти \textit{методы} введены (загружены, записаны) в \textit{память} \textit{кибернетической системы}}
\scnidtf{кибернетическая система, способная быть самостоятельной}
\scnexplanation{Признаками индивидуальных кибернетических систем являются:
\begin{scnitemize}
    \item наличие \textit{памяти}, предназначенной для хранения как минимум интерпретируемых \textit{методов} (программ)  и обеспечивающей корректировку (редактирование) хранимых \textit{методов}, а также их удаление  из \textit{памяти} и ввод (запись) в \textit{память} новых \textit{методов};
    \item легкая возможность "программировать"\ \textit{кибернетическую систему} на решение других задач, что обеспечивается наличием \textit{универсальной модели решения задач} и, соответственно, \textit{универсальным интерпретатором \uline{любых} моделей}, представленных (записанных) на соответствующем \textit{языке};
    \item наличие пусть даже простых средств коммуникации (обмена информацией) с другими \textit{кибернетическими системами} (например, с людьми);
    \item способность входить в различные \textit{коллективы кибернетических систем}.
\end{scnitemize}
}
\scnnote{класс \textit{индивидуальных кибернетических систем} — это определенный этап эволюции кибернетических систем, означающий переход к кибернетическим системам, которые способны самостоятельно "выживать"}

\scnheader{индивидуальная кибернетическая система}
\scnidtf{самостоятельная автономная, целостная кибернетическая системам}
\scnidtf{субъект деятельности}
\scnnote{\textit{индивидуальная кибернетическая система} может быть агентом (членом) многоагентной системы (членом коллектива индивидуальных кибернетических систем), но некоторые многоагентные системы могут состоять из агентов , не являющихся  \textit{индивидуальными кибернетическими системами}, представляющих собой простые специализированные кибернетические системы, выполняющие достаточно простые действия (см. коллективное поведение автоматов Стефанюк теория самовоспроизводящихся автоматов Джон фон Нейман)}
\scnauthorcomment{правильно ли записаны названия источников?}

\scnheader{индивидуальная кибернетическая система}
\scnidtf{кибернетическая система, которая обладает достаточной самостоятельностью (целостностью), но не является коллективом таких самостоятельных  кибернетических систем}
\scnidtf{минимальная самостоятельная (самодостаточная, в известной степени автономная) кибернетическая система}
\scnidtf{индивидуальный субъект}

\scnauthorcomment{не смогла разобраться с оформлением}
\scnheader{кибернетическая система, встроенная в индивидуальную кибернетическую систему}
\scnsuperset{пример*}\\
\scnaddlevel{1}
{sc-агент ostis-системы}\\
\scnaddlevel{-1}
\scnsuperset{пример*}\\
\scnaddlevel{1}
{решатель задач ostis-системы}\\
\scnidtf{коллектив всех sc-агентов ostis-системы}
\scnaddlevel{-1}

\scnheader{многоагентная система}
\scnidtf{коллектив взаимодействующих автономных кибернетических систем, имеющих общую среду обитания (жизнедеятельности)}
\scnsubdividing{одноуровневая многоагентная система;иерархическая многоагентная система}

\scnheader{одноуровневая многоагентная система}
\scnidtf{специализированное средство решения задач, реализующее либо \uline{одну} модель параллельного (распределенного) решения задач соответствующего класса, либо комбинацию \uline{фиксированного числа} разных и параллельно реализованных моделей решения задач}
\scnsubdividing{одноуровневая однородная многоагентная система;одноуровневая неоднородная многоагентная система}

\scnheader{коллектив индивидуальных кибернетических систем}
\scnsubset{многоагентная система}
\scnidtf{многоагентная система, агентами (членами) которой являются \uline{индивидуальные}(!) кибернетические системы}
\scnsubdividing{
коллектив людей\\
\scnidtf{человеческое сообщество}
;сообщество компьютерных систем и людей
}

\scnheader{иерархический коллектив индивидуальных кибернетических систем}
\scnidtf{многоагентная система, агентами (членами) которой могут быть:
\begin{scnitemize}
    \item индивидуальные кибернетические системы;
    \item коллективы индивидуальных кибернетических систем;
    \item коллективы, состоящие из индивидуальных кибернетических систем и коллективов индивидуальных кибернетических систем и т.д.
\end{scnitemize}
}

