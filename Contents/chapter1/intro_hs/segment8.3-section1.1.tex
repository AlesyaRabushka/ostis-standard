\bigskip
\scnfragmentcaption

\scnheader{договороспособность кибернетической системы}
\scnrelfromlist{свойство-предпосылка}{способность кибернетической системы к пониманию принимаемых сообщений
;особенности кибернетической системы к формированию передаваемых сообщений, понятных адресатам
;семантическая совместимость кибернетической системы с партнёрами 
;способность кибернетической системы к обеспечению семантической совместимости с партнёрами 
;коммуникабельность кибернетической системы 
;способность кибернетической системы к обсуждению и согласованию целей и планов коллективной деятельности 
;способность кибернетической системы брать на себя выполнение актуальных задач в рамках согласованных планов коллективной деятельности}
\scnheader{способность кибернетической системы к пониманию принимаемых сообщений}
\scnidtf{способность кибернетической системы к пониманию информации, поступающей извне от других кибернетических систем}
\scnidtf{способность кибернетической системы к отображению принимаемых сообщений в семантически эквивалентные фрагменты собственной базы знаний}
\scnrelfromset{комплекс частных свойств}{способность кибернетической системы к пониманию принимаемых вербальных сообщений 
;способность кибернетической системы к пониманию принимаемых невербальных сообщений} 
\scnrelfrom{свойство-предпосылка}{способность кибернетической системы к обеспечению семантической совместимости с партнёрами}

\scnheader{сообщение}
\scnidtf{информация, передаваемая (пересылаемая) от одной кибернетической системы к другой или к другим кибернетическим системам}
\scnnote{Каждому \textit{сообщению} ставится в соответствие одна \textit{кибернетическая система}, являющаяся \textbf{\textit{источником сообщения*}} и одна или несколько \textit{кибернетических систем}, являющихся \textbf{\textit{адресатами сообщения*}}. 
В соответствии с этим для каждой \textit{кибернетической системы} те сообщения, \textit{источником*} которых она является, будем называть \textbf{\textit{передаваемыми сообщениями}}, а те сообщения, \textit{адресатами*} которых она является, будем называть \textbf{\textit{принимаемыми сообщениями}}.}
\scnsubdividing{вербальное сообщение\\
	\scnaddlevel{1}
		\scnidtf{передаваемая словесная информация}
	\scnaddlevel{-1}
;невербальное сообщение\\
	\scnaddlevel{1}
		\scnnote{Примерами невербальных сообщений являются пересылаемые фото-документы, видео-материалы}
	\scnaddlevel{-1}
}
\scnrelfromset{обобщённая декомпозиция}{спецификация сообщения\\
	\scnaddlevel{1}
		\scnrelfromset{обобщённая декомпозиция}{указание источника специфицируемого сообщения;указание множества адресатов специфицируемого сообщения;отметка момента времени отправления специфицируемого сообщения;указание прагматического типа специфицируемого сообщения;указания запроса, ответом на который является специфицируемое сообщение\\
		\scnaddlevel{1}
			\scnnote{Если специфицируемое сообщение является ответом на некоторый запрос}
		\scnaddlevel{-1}
;указание раздела баз знаний адресатов, которому соответствует специфицируемое сообщение 
;указание способа представления тела сообщения\\
		\scnaddlevel{1}
			\scnnote{Для вербальных сообщений это указание используемого  внешнего языка}
		\scnaddlevel{-1}
	\scnaddlevel{-1}
	}
;тело сообщения\\
	\scnaddlevel{1}
		\scnidtf{собственно само сообщение}
	\scnaddlevel{-1}
}
\scnrelfrom{разбиение}{прагматический тип сообщения}
	\scnaddlevel{1}
	\scneqtoset{повествовательное сообщение\\
	\scnaddlevel{1}
		\scnsuperset{ответ на запрос}
	\scnaddlevel{-1}
;запрос
	\scnaddlevel{1}
		\scnidtf{вопросительное сообщение}
	\scnaddlevel{-1}
;команда редактирования баз знаний адресатов
;команда, инициирующая действие адресатов в их внешней среде}
     
\scnheader{следует отличать*}
\scnhaselementset{вербальная информация
;файл, содержащий вербальную информацию\\
	\scnaddlevel{1}
		\scnidtf{вербальная информация, представленная в виде файла}
	\scnaddlevel{-1}		
;вербальное сообщение}

\scnheader{вербальная информация}
\scnidtf{знаковая конструкция, которая имеет в общем случае произвольную денотационную семантику и которая может либо поступать на вход кибернетической системы через соответствующие ее сенсоры (рецепторы), либо через соответствующие эффекторы передаваться (пересылаться) в качестве сообщения другим кибернетическим системам}

\scnheader{следует отличать*}
\scnhaselementset{вербальная информация 
;сенсорная информация}
	\scnaddlevel{1}
		\scnnote{И \textit{вербальная информация} и \textit{сенсорная информация} являются \textit{знаковыми конструкциями}, но, во-первых, \textit{вербальная информация} может быть как внешней знаковой конструкцией, так и внутренней знаковой конструкцией, хранимой в памяти кибернетической системы, а \textit{сенсорная информация} всегда является внутренней \textit{знаковой конструкцией} кибернетической системы и, во-вторых, \textit{сенсорная информация} описывает только "пограничную"{} для \textit{кибернетической системы} физическую \textit{окружающую среду}, тогда, как \textit{вербальная информация} может описывать все, что угодно.}
 
\scnheader{следует отличать*}
\scnhaselementset{невербальная информация
;файл, содержащий невербальную информацию\\
	\scnaddlevel{1}
		\scnidtf{файл, содержимым которого является электронный образ некоторой невербальной информации}
	\scnaddlevel{-1}
;невербальное сообщение\\
	\scnaddlevel{1}
		\scnidtf{невербальная информация, представленная в виде файла и передаваемая (пересылаемая) от одной кибернетической системы к другой}
	\scnaddlevel{-1}
;сенсорная информация\\
	\scnaddlevel{1}
		\scnidtf{информация, формируемая сенсорами кибернетической системы}
	\scnaddlevel{-1}
}

\scnheader{невербальная информация}
\scnsuperset{музыкальное произведение}
\scnsuperset{танец}
\scnsuperset{произведение изобразительного искусства}
	\scnaddlevel{1}
		\scnsuperset{живопись}
		\scnsuperset{скульптура}
		\scnsuperset{графика}
	\scnaddlevel{-1}
\scnsuperset{статическое изображение}
\scnsuperset{динамическое изображение}

\scnheader{способность кибернетической системы к пониманию принимаемых вербальных сообщений}
\scnidtf{способность кибернетической системы к пониманию вербальной информации, поступающей извне из разных источников}
\scnnote{Понимание информации, поступающей извне, включает в себя:
\begin{scnitemize}
	\item перевод этой информации на внутренний язык кибернетической системы;
	\item локальную верификацию вводимой информации;
	\item погружение (конвергенцию, размещение) текста, являющегося результатом указанного перевода в состав хранимой информации (в частности, в состав базы знаний)
\end{scnitemize}}
\scnnote{Погружение вводимой информации в состав базы знаний кибернетической системы сводится к выявлению и устранению противоречий, возникающих между погружаемым текстом и текущего состояния базы знаний. Первым уровнем таких противоречий являются появляющиеся при интеграции погружаемого текста с текущим состоянием базы знаний \textit{омонимичные знаки} и пары \textit{синонимичных знаков}. Омонимичные знаки появляются в результате ошибочного отождествления знака, входящего в состав погружаемого текста, со знаком, входящим в состав погружаемого текста, со знаком, входящим в состав текущего состояния базы знаний. Появление пар синонимичных знаков, один из которых входит в погружаемый текст, а второй -- в текущее состояние базы знаний, при погружении вводимого текста является "штатным"{} противоречием, устранение которого осуществляется путем отождествления ("склеивания"{}) синонимичных знаков.}
\scnnote{Сложность проблемы понимания вводимой вербальной информации заключается не только в сложности непротиворечивого погружения вводимой информации в текущее состояние базы знаний, но и в сложности трансляции этой информации с внешнего языка на внутренний язык кибернетической системы, т. е. в сложности генерации текста внутреннего языка, семантически эквивалентного вводимому тексту внешнего языка. Очевидно, что для естественных языков указанная трансляция является сложной задачей, так как в настоящее время проблема формализации синтаксиса и семантики естественных языков не решена.}

\scnheader{семантическая совместимость кибернетической системы с партнерами}
\scnidtf{уровень взаимопонимания кибернетической системой со своими партнерами}
\scnidtf{степень конвергенции (близости) базы знаний кибернетической системы с базами знаний своих партнеров}
\scnheader{семантическая совместимость кибернетической системы с партнерами}
\scnrelto{частное свойство}{\textit{совместимость кибернетических систем}}
\scnexplanation{\textit{семантическая совместимость кибернетических систем} определяется 
\begin{scnitemize}
	\item количеством знаков, которые хранятся в памяти одной заданной кибернетической системы и денотационная семантика которых совпадает с денотационной семантикой знаков, хранимых в памяти другой заданной кибернетической системы (другими словами, это количество сущностей, которые описывают как в памяти первой кибернетической системы, так и в памяти второй кибернетической системы),
	\item тем, согласованы ли между двумя заданными кибернетическими системами факт совпадения денотационной семантики указанных выше знаков сущностей, описываемых в памяти как первой, так и второй кибернетической системы (такое согласование осуществляется путем согласования уникальных внешних идентификаторов (имен), которые приписываются указанным знакам сущностей и которые используются указанными кибернетическими системами при обмене сообщениями между ними).
\end{scnitemize}}
\scnnote{Прежде всего семантическая совместимость двух заданных кибернетических систем определяется согласованностью систем понятий, используемых обеими взаимодействующими кибернетическими системами, (т.е. совпадением семантической трактовки всех этих понятий) и включением в число таких общих понятий всех или почти всех неопределяемых понятий, а также тех определяемых понятий, которые обеими кибернетическими системами часто используются при определении остальных определяемых понятий.}
\scnnote{Высокий уровень семантической совместимости даже для кибернетических систем с высоким уровнем интеллекта (например, для людей) встречается значительно реже, чем хотелось бы. Очевидно, что проблема обеспечения перманентной поддержки семантической совместимости взаимодействующих кибернетических систем является необходимым условием обеспечения высокого уровня взаимопонимания кибернетических систем и, как следствие, эффективного их взаимодействия.}

