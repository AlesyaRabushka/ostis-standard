\bigskip

\scnsegmentheader{Комплекс свойств, определяющих качество физической оболочки кибернетической системы}

\scnstartsubstruct

\scnheader{качество физической оболочки кибернетической системы}
\scnidtf{интегральное качество "аппаратной"{} (физической) основы кибернетической системы}
\scnidtf{"hardware"{} кибернетической системы}
\scnrelfromlist{свойство-предпосылка}{
качество памяти кибернетической системы;
качество процессора кибернетической системы;
качество сенсоров кибернетической системы;
качество эффекторов кибернетической системы;
приспособленность физической оболочки кибернетической системы к ее совершенствованию;
удобство транспортировки кибернетической системы;
надежность физической оболочки кибернетической системы
}
\scnheader{качество памяти кибернетической системы}
\scnreltolist{свойство-предпосылка}{
качество информации, хранимой в памяти кибернетической системы;
качество решателя задач кибернетической системы
}
\scnrelfromlist{свойство-предпосылка}{
способность памяти кибернетической системы обеспечить хранение высококачественной информации;
способность памяти кибернетической системы обеспечить функционирование высококачественного решателя задач;
объём памяти
}

\scnheader{память кибернетической системы}
\scnidtf{компонент \textit{кибернетической системы}, представляющий собой "внутреннюю"{} среду \textit{кибернетической системы}, в которой она хранит (запоминает) и преобразует \textit{информационную модель} своей \textit{внешней среды}. При этом важно, чтобы память обеспечивала высокий уровень \textit{гибкости} указанной \textit{информационной модели}. Важно также, чтобы эта \textit{информационная модель} была моделью не только \textit{внешней среды} \scnbigskip \textit{кибернетической системы}, но также и моделью самой этой \textit{информационной модели} -- описанием её \textit{текущей ситуации}, предыстории, закономерностей. Таким образом, \textit{кибернетическая система}, имеющая \textit{память}, функционирует в двух средах -- во внешней, в которой существуют и преобразуются внешние(материальные) сущности, и во внутренней, в которой существуют и преобразуются(обрабатываются) внутренние \textit{информационные конструкции}.}
\scnnote{\textit{Кибернетические системы}, находящиеся на низком уровне развития(качества) \textit{памяти} не имеют. Адаптационные механизмы такой кибернетической системы "жестко запаяны"{} в связях между блоками обработчика \textit{сигналов} при переходе от \textit{сигналов}, вырабатываемых \textit{сенсорами} к \textit{сигналам}, которые управляют \textit{эффекторами}.}
\scnidtf{внутренняя среда кибернетической системы, обеспечивающая хранение и преобразование(обработку) информационной модели внешней среды кибернетической системы}
\scnnote{Сам факт возникновения памяти в \textit{кибернетической системе} является важнейшим этапом её эволюции. Дальнейшее развитие \textit{памяти кибернетической системы}, обеспечивающее:
\begin{scnitemize}
	\item хранение все более качественной информации, хранимой в памяти
	\item все более качественную организацию обработки этой информации, т.е. переход на поддержку(обеспечение) все более качественных моделей обработки информации
\end{scnitemize}
является важнейшим фактором эволюции \textit{кибернетических систем}.}

\scnheader{способность памяти кибернетической системы обеспечить хранение высококачественной информации}
\scnrelfromlist{свойство-предпосылка}{
способность системы обеспечить компактное хранение сложноструктурированных баз знаний\\
	\scnaddlevel{1}
	\scnnote{Здесь имеется в виду необходимость перехода от линейной организации, памяти на физическом уровне (как последовательности ячеек памяти) к нелинейной, графодинамической памяти.}
	\scnaddlevel{-1}
;способность памяти кибернетической системы обеспечить хранение широкого многообразия знаний\\
	\scnaddlevel{1}
	\scnnote{имеется в виду хранение гибридных баз знаний}
	\scnaddlevel{-1}
}

\scnheader{способность памяти кибернетической системы обеспечить функционирование высококачественного решателя задач}
\scnrelfromlist{свойство-предпосылка}{качество доступа к информации, хранимой памяти кибернетической системы\\
	\scnaddlevel{1}
	\scnnote{Здесь имеется в виду необходимость перехода от адресного к ассоциативному доступу, причем, с расширением многообразия видов реализуемых запросов, в частности, к реализации запросов фрагментов баз знаний по заданному образцу произвольного размера и произвольной конфигурации.}
	\scnaddlevel{-1}
;логико-семантическая гибкость памяти кибернетической системы
;способность памяти кибернетической системы обеспечить интерпретацию широкого многообразия моделей решения задач
}

\scnheader{логико-семантическая гибкость памяти кибернетической системы}
\scnidtf{степень близости физической организации памяти кибернетической системы к реализуемым ею базовым семантически целостным действиям над информацией, хранимой в памяти}
\scnidtf{простота реализации базовых семантически целостных действий над информацией, хранимой в памяти кибернетической системы}
\scnnote{Важен переход от "мелких"{} действий, к элементарным действиям, имеющим логико-семантический смысл (целостность, законченность}

\scnheader{качество процессора кибернетической системы}
\scnrelto{свойство-предпосылка}{качество решателя задач кибернетической системы}
\scnrelfromlist{свойство-предпосылка}{способность процессора кибернетической системы обеспечить функционирования высококачественного решателя задач\\
	\scnaddlevel{1}
	\scnrelfromlist{свойство-предпосылка}{многообразие моделей решения задач, интерпретируемых процессором кибернетической системы
;простота и качество интерпретации процессором системы широкого многообразия моделей решения задач\\
		\scnaddlevel{1}	
		\scnnote{Указанная простота определяется степенью близости интерпретируемых моделей решения задач к “физическому” уровню организации процессора кибернетической системы}
		\scnaddlevel{-1}
;обеспечение процессором кибернетической системы качественного управления информационными процессами в памяти\\
		\scnaddlevel{1}
		\scnnote{Речь идет о грамотном сочетание таких аспектов управление процессами, как централизация и децентрализация, синхронность и асинхронность, последовательность и параллельность}
		\scnrelfrom{свойство-предпосылка}{уровень параллелизма обработки информации в памяти кибернетической системы}
		\scnidtf{максимальное количество одновременно выполняемых информационных процессов в памяти кибернетической системы}
		\scnaddlevel{-1}
;быстродействие процессора кибернетической системы}
	\scnaddlevel{-1}
}

\scnheader{многообразие моделей решения задач, интерпретируемых  процессором кибернетической системы}
\scnnote{Максимальным уровнем качества процессора кибернетической системы по данном параметру является его универсальность, т.е. его принципиальная возможность интерпретировать любую модель решения как интеллектуальных, так и неинтеллектуальных задач(алгоритмизацию, процедурную параллельную синхронную, процедруную параллельную асинхронную, продукционную, нейросетевую, генетическую, функциональную, целое семейство моделей).}

\scnheader{качество сенсоров кибернетической системы}
\scnrelfrom{свойство-предпосылка}{многообразие видов сенсоров кибернетической системы\\
	\scnidtf{многообразие средств восприятия (отображения) информации о текущем состоянии внешней среды кибернетической системы и её собственной физической оболочки}
}

\scnheader{качество эффекторов кибернетической системы}
\scnrelfrom{свойство-предпосылка}{многообразие видов эффекторов кибернетической системы\\
	\scnidtf{многообразие средств воздействия на собственную физическую оболочку кибернетической системы и через нее на внешнюю среду этой системы}
	\scnnote{Эффекторы кибернетической системы являются инструментами воздействия кибернетической системы на свою внешнюю среду}
}

\scnheader{приспособленность физической оболочки кибернетической системы к её совершенствованию}
\scnidtf{приспособленность кибернетической системы к повышению качества её физической оболочки}
\scnidtf{простота ремонта и совершенствования таких компонентов кибернетической системы как память, процессор, сенсоры, эффекторы}
\scnrelfrom{частное свойство}{ремонтопригодность физической оболочки кибернетической системы}
\scnrelfromset{группа свойств-предпосылок}{гибкость физической оболочки кибернетической системы
;стратифицированность физической оболочки кибернетической системы\\
	\scnaddlevel{1}
	\scnidtf{мобильность физической оболочки кибернетической системы}
	\scnidtf{легкость сохранения целостности физической оболочки кибернетической системы при внесении различных изменений (локализация области учета последствий внесения изменений, предсказуемость последствий)}
	\scnaddlevel{-1}
}

\bigskip

\scnendstruct \scninlinesourcecommentpar{Закончили Сегмент ``Комплекс свойств, определяющих качество физической оболочки кибернетической системы''}