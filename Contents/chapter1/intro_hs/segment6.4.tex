\scnheader{качество целеполагания}
\scnidtf{качество реализации первого этапа решения сложных задач -- этапа генерации (построения) планов решения сложных задач}
\scnidtf{качество генерации планов выполнения сложных действий:
		\begin{scnitemize}
		\item как внутренних действий (в памяти кибернетической системы), так и внешних действий (во внешней среде)
		\item как собственных действий, так и действий других субъектов
		\end{scnitemize}}
\scnidtf{качество генерации планов действий кибернетической системы и, в частности, трудоемкость процесса генерации этих планов}
\scnidtf{качество организации целенаправленной деятельности кибернетической системы}
\scnidtf{качество построения цепочек "цель-план-действие"{}}
\scnidtf{качество генерации, анализа и инициирования собственных целей (собственных задач)}
\scnidtf{способность кибернетической системы к целеполаганию}
\scnrelfromlist{свойство-предпосылка}{самостоятельность целеполагания\\
	\scnaddlevel{1}
	\scnidtf{самостоятельность генерации и инициирования целей (задач), направленных на создание условий достижения соответствующих стратегических целей (сверхзадач)}
	\scnaddlevel{-1}
;целенаправленность целеполагания\\
	\scnaddlevel{1}
	\scnidtf{степень соответствия (степень полезности) генерируемых целей (задач) для достижения соответствующих стратегических целей (сверхзадач)}
	\scnaddlevel{-1}
;сбалансированность целеполагания\\
	\scnaddlevel{1}
	\scnidtf{качество расстановки приоритетов у сгенерированных и инициированных целей (задач) для обеспечения баланса между тактическими и стратегическими целями}
	\scnaddlevel{-1}}


\scnheader{самостоятельность целеполагания}
\scnidtf{способность кибернетической системы генерировать, инициировать и решать задачи, которые не являются подзадачами, инициированными внешними (другими) субъектами, а также способность на основе анализа своих возможностей отказаться от выполнения задачи, инициированной извне, переадресовав её другой кибернетической системе, либо на основе анализа самой этой задачи обосновать её нецелесообразность или некорректность}
\scnidtf{способность к самостоятельному целеполаганию (генерации идей) и к инициированию процессов их достижения (т.е. к принятию решений), способность свободно (в определенных рамках) выбирать (ставить перед собой цели)}
\scnidtf{уровень самостоятельности}
\scnidtf{способность решать задачи в комплексе, включая создание всех необходимых условий для их решения с учетом конкретных обстоятельств}
\scnidtf{умение решать задачи в условиях сильных помех (в осложненных обстоятельствах)}
\scnnote{Повышение уровня самостоятельности существенно расширяет возможности кибернетической системы, т.е. объем тех задач, которые она может решать не только в "идеальных"{} условиях, но и в реальных (осложненных) обстоятельствах.}	
\scnidtf{степень свободы выбора целей, подлежащих достижению, а также свободы генерации целей, не являющихся подцелями извне поставленных целей}


\scnheader{целенаправленность целеполагания}
\scnidtf{целеустремленность}
\scnidtf{целенаправленность}
\scnidtf{степень целостности деятельности}
\scnidtf{степень соответствия между тактическими и стратегическими уровнями деятельности}
\scnidtf{общее соотношение между временем, затраченным на "лишние"{} (ненужные, нецелесообразные, нецеленаправленные) действия и полезные действия}
\scnidtf{целесообразность деятельности}
\scnidtf{способность адекватно расставлять приоритеты своим целям и не "распыляться"{} на достижение неприоритетных (несущественных) целей}


\scnheader{качество реализации планов собственных действий}
\scnidtf{качество реализации целенаправленной деятельности на основе построенных планов}
\scnidtf{качество реализации построенных в памяти кибернетической системы планов выполнения сложных собственных действий, которые могут предполагать участие других субъектов}


\scnheader{способность кибернетической системы к локализации такой области информации, хранимой в ее памяти, которой достаточно для обеспечения решения заданной задачи}
\scnidtf{способность кибернетической системы к сужению области решения каждой решаемой ею задачи, что существенно минимизирует затраты кибернетической системы на учет и анализ факторов, априори незначимых (несущественных) для решения каждой решаемой задачи}
\scnnote{Для реализации данной способности важное значение имеет качественная стратификация базы знаний кибернетической системы на предметные области и соответствующие им онтологии.}


\scnheader{способность кибернетической системы к выявлению существенного в информации, хранимой в ее памяти}
\scnidtf{способность к выявлению (обнаружению, выделению) таких фрагментов информации, хранимой в памяти кибернетической системы, которые существенны (важны) для достижения соответствующих целей}
\scnnote{Понятие существенного (важного) фрагмента информации, хранимой в памяти кибернетической системы, относительно и определяется соответствующей задачей. Тем не менее, есть важные перманентно (постоянно) решаемые задачи, в частности задачи анализа качества информации, хранимой в памяти кибернетической системы. Существенные фрагменты хранимой информации, выделяемые в процессе решения этих задач, являются относительными не столько по отношению к решаемой задаче, сколько по отношению к текущему состоянию хранимой информации. Примерами таких фрагментов являются:
	\begin{scnitemize}
	\item обнаруженные противоречия (ошибки) с явным указанием того, что чему противоречит;
	\item обнаруженные информационные дыры, точнее точная спецификация этих дыр;
	\item обнаруженные мусорные фрагменты, которые либо носят вспомогательный характер, либо могут быть легко восстановлены (воспроизведены).
	\end{scnitemize}}
	
	
\scnheader{следует отличать*}
\scnhaselementset{
способность кибернетической системы к выявлению существенного в информации, хранимой в ее памяти\\
\scnaddlevel{1}
\scnnote{Здесь кибернетическая система выделяет информацию, которая необходима, но не обязательно достаточна для решения соответствующей задачи.}
\scnaddlevel{-1};
способность кибернетической системы к локализации такой области информации, хранимой в ее памяти, которой достаточно для обеспечения решения заданной задачи\\
\scnaddlevel{1}
\scnnote{Здесь кибернетическая система "отбрасывает"{} (исключает) информацию, которая априори несущественна для решения соответствующей (заданной) задачи.}
\scnaddlevel{-1}} 


\scnheader{активность кибернетической системы}
\scnidtf{уровень активности кибернетической системы}
\scnidtf{уровень мотивации к деятельности в различных направлениях}
\scnidtf{уровень "желания"{} действовать}
\scnidtf{активность/пассивность кибернетической системы}
\scnidtf{уровень инициативности, пассионарности, мотивированности}
\scnnote{Уровень активности кибернетической системы может быть разным для разных решаемых задач, для разных классов выполняемых действий, для разных видов деятельности.}
\scnnote{Следует отличать уровень активности (мотивации, желания) и направленность этой активности.}
\scnnote{Чем выше активность кибернетической системы, тем (при прочих равных условиях) она больше успевает сделать, следовательно, тем выше ее качество (эффективность).}
\scnrelboth{обратное свойство}{пассивность}
	\scnaddlevel{1}
	\scnidtf{уровень бездеятельности, медлительности, вялости, ленивости}
	\scnaddlevel{-1}
\scnrelfromlist{частное свойство}{познавательная активность; социальная активность}