\begin{SCn}

\scnsectionheader{\currentname}

\scnstartsubstruct

\scnrelto{частная предметная область и онтология}{Предметная область и онтология информационных конструкций}
\scnaddlevel{1}
\scnsourcecommentpar{Раздел 2.1.2.0}
\scnaddlevel{-1}

\scnsdmainclasssingle{смысловое представление информации}

\scnsdclass{семантическая сеть\\
	\scnaddlevel{1}
	\scnsubdividing{нерафинированная семантическая сеть;рафинированная семантическая сеть}
	\scnsubdividing{абстрактная семантическая сеть\\
		\scnaddlevel{1}
		\scnidtf{семантическая сеть, абстрагирующаяся от того, как физически представлены ее элементарные (атомарные) фрагменты, а также связи инцидентности между этими фрагментами}
		\scnaddlevel{-1}
	;графически представленная семантическая сеть\\
		\scnaddlevel{1}
		\scnidtf{нарисованная семантическая сеть}
		\scnaddlevel{-1}
	;семантическая сеть, хранимая в графодинамической памяти\\
		\scnaddlevel{1}
		\scnrelboth{следует отличать}{представление семантической сети в адресной памяти}
			\scnaddlevel{1}
			\scnnotsubset{семантическая сеть}
			\scnidtf{представление семантической сети в виде линейной информационной конструкции, которая хранится в адресной памяти и которая, строго говоря, уже не является семантической сетью, но является информационной конструкцией, семантически эквивалентной соответствующей (представляемой) семантической сети}			
			\scnaddlevel{-1}
		\scnaddlevel{-1}}
	\scnaddlevel{-1}
;язык семантических сетей\\
	\scnaddlevel{1}
	\scnidtf{язык, все тексты которого являются семантическими сетями}
	\scnsubdividing{специализированный язык семантических сетей;универсальный язык семантических сетей}
	\scnsuperset{язык рафинированных семантических сетей}
	\scnaddlevel{-1}}

\scnrelfromvector{рассматриваемые вопросы}{
\scnfileitem{Что такое семантические сети и в чем их принципиальное отличие от других вариантов представления информации}
;\scnfileitem{До какой степени можно минимизировать алфавит элементов семантических сетей}
;\scnfileitem{Можно ли все описываемые связи свести к бинарным связям и почему это целесообразно}
;\scnfileitem{Можно ли разработать \uline{универсальный} язык семантических сетей}
;\scnfileitem{До какой степени можно упростить синтаксические структуры семантических сетей до, условно говоря, рафинированного вида}
;\scnfileitem{Какими достоинствами обладает семантические сети}}

\scnrelfromlist{ссылка}{Понятие Технологии OSTIS\\
	\scnaddlevel{1}
	\scnsourcecommentpar{Сегмент 3 Раздела 0.2}
	\scntext{аннотация}{В данном сегменте \textit{Документации Технологии OSTIS} рассматриваются принципы, лежащие в основе \textit{Технологии OSTIS}, основным из которых является ориентация на использование \textit{\uline{универсального} языка рафинированных семантических сетей} в качестве внутреннего языка \textit{интеллектуальных компьютерных систем}}
	\scnaddlevel{-1}
;Описание внутреннего языка ostis-систем\\
	\scnaddlevel{1}
	\scnsourcecommentpar{Раздел 0.3.1}	
	\scntext{аннотация}{В данном разделе \textit{Документации Технологии OSTIS} рассматриваются принципы, лежащие в основе \textit{универсального языка рафинированных семантических сетей}, используемого в качестве внутреннего языка \textit{ostis-систем} -- \textit{интеллектуальных компьютерных систем} следующего поколения}
	\scnaddlevel{-1}
;Описание языка графического представления знаний ostis-систем\\
	\scnaddlevel{1}
	\scnsourcecommentpar{Раздел 0.3.3}
	\scntext{аннотация}{В данном разделе \textit{Документации Технологии OSTIS} рассматриваются принципы, лежащие в основе универсального языка графически представленных семантических сетей, используемого в \textit{пользовательском интерфейсе ostis-систем}}
	\scnaddlevel{-1}
;Бирюков Б.В. ТеориСГФ-1960ст;Гладун В.П.;Скороходько;Мартынов;Шенк;Мельчук-Жолковский Смысл-Текст;Кузнецов Игорь}
	
\scnauthorcomment{Дооформить библиографию}	

\bigskip
\scnfragmentcaption

\scnheader{знак}
\scnidtf{фрагмент информационной конструкции, обладающий свойством, \uline{обозначать} некоторую сущность (объект), которая наряду с другими сущностями описывается указанной информационной конструкцией}
\scnnote{\uline{Форма} представления знаков в известной степени условна и является результатом соглашения между носителями соответствующего языка. Знак может быть, например, представлен:
	\begin{scnitemize}
	\item  в виде фрагмента речевого сообщения (последовательностью фонем);
	\item в виде строки символов (последовательности букв) в заданном алфавите;
	\item в виде иероглифа, пиктограммы;
	\item в виде жеста.
	\end{scnitemize}}
\scniselementrole{ключевой знак}{Предметная область и онтология информационных конструкций}
	\scnaddlevel{1}
	\scnsourcecommentpar{Раздел 2.1.2.0}
	\scnhaselement{раздел Базы знаний IMS.ostis}
	\scnaddlevel{-1}
\scntext{характеристика элементов данного множества}{Знаки, используемые в различных языках, характеризуются:
	\begin{scnitemize}
	\item синтаксической структурой, по совпадению (изоморфизму) которых для разных знаокв предполагается их синонимия;
	\item денотационной семантикой, т.е. той сущностью, которая обозначается соответствующим знаком;
	\item типом (классом) обозначаемой сущности, которая может быть:
	 	\begin{scnitemizeii}
		\item материальным(физическим) элементом (точкой) абстрактного пространства, множеством, которое может быть:
			\begin{scnitemizeiii}
			\item связью;
			\item классом;
			\item структурой;
			\end{scnitemizeiii}
		\item реальной и вымышленной сущностью;
		\item константной (конкретной) и переменной (произвольной) сущностью;
		\item постоянно существующей и временно существующей сущностью (прошлой, настоящей, будущей);		
		\end{scnitemizeii}
	\item множеством тех связей, которые связывают сущность, обозначаемую данным знаком с другими сущностями, а также, если данный знак обозначает некоторую связь, множеством сущностей, которые связаны этой связью, т.е. сущностей, являющихся компонентом этой связи;
	\item текущим статусом самого знака в памяти кибернетической системы, который может быть:
		\begin{scnitemizeii}
			\item логически удаленным знаком;
			\item настоящим знаком;
			\item предлагаемым (возможно, будущим) знаком.
		\end{scnitemizeii}
	\end{scnitemize}}
	
\scnheader{денотат*}
\scnidtf{денотат заданного знака*}
\scnidtf{объект, обозначаемый заданным знаком*}
\scnidtf{денотационная семантика заданного знака*}
\scnidtf{смысл заданного знака*}
\scnidtf{Бинарное ориентированное отношение, каждая пара которого связывает:
	\begin{scnitemize}
			\item некоторый знак, представленный в той или иной форме в тексте исследуемого языка;
			\item \uline{со знаком} той сущности, которая обозначается указанным выше знаком в рамках используемого метаязыка.
		\end{scnitemize}}
\scnnote{Данное отношение используется, когда с помощью одного языка необходимо описать денотационную семантику другого языка. Фактически речь идет о переводе заданного знака, входящего в состав некоторого рассматриваемого текста, принадлежащего некоторому исследуемому языку (языку-объекту), на некоторый метаязык (в нашем случае на SC-код), денотационная семантика которого нам считается априори известной. Указанный перевод есть связь заданного знака с синонимичным ему знаком, входящим в состав текста, принадлежащего указанному метаязыку.}
\scnrelboth{обратное отношение}{внешний sc-идентификатор*}
\scnaddlevel{1}
\scnidtf{быть знаком, обозначающим заданную сущность*}
\scnaddlevel{-1}
\scniselementlist{ключевой знак}{Описание внешних идентификаторов знаков, входящих в тексты внутреннего языка ostis-систем\\
	\scnaddlevel{1}
	\scnsourcecommentpar{Раздел 0.3.2}
	\scniselement{раздел Базы знаний IMS.ostis}
	\scnaddlevel{-1}
;Предметная область и онтология знаков, входящих в тексты внутреннего языка ostis-систем\\
	\scnaddlevel{1}
	\scnsourcecommentpar{Раздел 2.1.1.2}	
	\scniselement{раздел Базы знаний IMS.ostis}
	\scnaddlevel{-1}}
	
\scnheader{информационная конструкция}
\scnidtf{информация}
\scnnote{В общем случае информационная конструкция представляет собой сложную иерархическую структуру, каждому уровню иерархии которой соответствует определенный класс информационных конструкций}
\scnsuperset{синтаксически элементарный фрагмент информационной конструкции}
	\scnaddlevel{1}
	\scnidtf{атомарный фрагмент информационной конструкции}
	\scnidtf{элемент информационной конструкции}
	\scnnote{Примерами таких элементарных фрагментов информационных конструкций являются буквы}
	\scnsuperset{буква}
	\scnaddlevel{-1}
\scnsuperset{простой знак}
	\scnaddlevel{1}
	\scnidtf{семантически элементарный фрагмент информационной конструкции}
	\scnsubset{знак}
	\scnaddlevel{-1}
\scnsuperset{выражение}
\scnaddlevel{1}
	\scnidtf{сложный (непростой) знак}
	\scnidtf{знак, являющийся одновременно некоторым знанием обозначаемой сущности (спецификацией этой сущности)}
	\scnidtf{знак, построенный как выражение вида "тот, который..."{}}
	\scnidtf{знак, в состав которого входят другие знаки}
	\scnsubset{знак}
	\scnaddlevel{-1}
\scnsuperset{простой текст}
	\scnaddlevel{1}
	\scnidtf{минимальная синтаксически целостная и корректная (правильная) информационная конструкция, включающая в себя:
	\begin{scnitemize}
	\item знак некоторой описываемой связи;
	\item минимальную спецификацию указанного знака связи (указание отношения, которому это связь принадлежит);
	\item указание \uline{всех} компонентов описываемой связи (знаков всех сущностей, связываемых этой связью, и/или всех знаков, связываемых этой связью -- описываемая связь может связывать не только "внешние"{} описываемые сущности, но и сами знаки);
	\item если описываемая связь не является бинарной, то связи с её компонентами могут потребовать явного представления знаков этих связей с дополнительным указанием роли этих компонентов.
	\end{scnitemize}}
	\scnsubset{текст}
	\scnaddlevel{-1}
\scnsuperset{сложный текст}
	\scnaddlevel{1}
	\scnidtf{информационная конструкция, являющаяся результатом соединения нескольких простых текстов}
	\scnsubset{текст}
	\scnaddlevel{-1}
\scnsuperset{простое знание}
	\scnaddlevel{1}
	\scnidtf{минимальная семантические целостная информационная конструкция}
	\scnidtf{знание, в состав которого не входят другие знания}
	\scnsubset{знание}
	\scnaddlevel{-1}	
\scnsuperset{сложное знание}
	\scnaddlevel{1}
	\scnidtf{информационная конструкция, являющаяся результатом соединения нескольких простых знаний}
	\scnidtf{знание, в состав которого не входят другие знания}
	\scnsubset{знание}
	\scnaddlevel{-1}	
\scniselementrole{ключевой знак}{Предметная область и онтология информационных конструкций}
\scnaddlevel{1}
	\scnsourcecommentpar{Раздел 2.1.2.0}
\scnaddlevel{-1}

\bigskip
\scnfragmentcaption

\scnheader{смысловое представление информации}
\scnidtf{смысловая форма представления информации}
\scnidtf{смысловое представление информационной конструкции}
\scnidtf{знаковая конструкция (текст), представленная в смысловой форме}
\scnidtf{смысловое представление информационной конструкции}
\scnidtftext{часто используемый sc-идентификатор}{смысл}
\scnidtf{смысловое представление}
\scnidtf{семантическое представление информации}

\scntext{основной принцип}{Как можно меньше лишнего, не имеющего отношения к смыслу представляемой информации.}
\scnidtf{такое представление информационной конструкции, которое существенно прощает соответствие между структурой самой этой информационной конструкции и описываемой (отображаемой) ею конфигурацией связей между рассматриваемыми (исследуемыми) сущностями}
\scnidtf{смысловое представление знаковой конструкции}
\scnidtf{абстрактная знаковая конструкция, являющаяся \uline{инвариантом} соответствующего максимального класса семантически эквивалентных знаковых конструкций}
\scnidtf{смысл информационной конструкции}
\scnidtf{денотационная семантика информационной конструкции}
\scnidtf{смысловое представление информационной конструкции}


\scnnote{Суть (смысл, денотационная семантика) любой информационной конструкции (информационной модели) сводится к описанию системы (конфигурации) связей между списываемыми (рассматриваемыми) сущностями. Важно, чтобы эта суть не была \uline{закамуфлирована} различными "синтаксическими"{} деталями, не имеющими никакого отношения к указанному смыслу (синтаксическая структура знаков, многократное повторение одного и того же знака, синонимия, омонимия, местоимения, предлоги, знаки препинания, разделители, ограничители, падежи и т.п.) а обусловленными \uline{формой} представления информационных конструкций, например, их линейностью.}


\scnexplanation{Смысловое представление любой информации в конечном счете сводится:
	\scnaddlevel{1}
	\begin{scnitemize}
		\item{к перечню знаков конкретных описываемых сущностей - как первичных сущностей, так и вторичных сущностей, которые сами являются информационными конструкциями (фрагментами данной конструкции)};
		\item{к явному описанию связи между знаками вторичных сущностей и самими этими сущностями (т.е. фрагментами информационной конструкции)};
		\item{к описанию других связей между описываемыми сущностями}
	\end{scnitemize}
}
\scnaddlevel{-1}

\scnauthorcomment{Дооформить ссылки}

\scnexplanation{Формализация смысла представляемой информации, т.е. строгое уточнение того, что такое \textit{смысловое представление информации}, является объективной основой для \uline{унификации} представления информации в \textit{памяти компьютерных систем} и \uline{ключом} к решению многих проблем семантической совместимости и эволюции компьютерных систем и технологий.

Согласно \textit{Мартынову В. В.} ~\scncite{Martynov}, <<фактически всякая мыслительная деятельность человека (не только научная), как полагают многие ученые, использует \uline{внутренний семантический код}, на который переводят с естественного языка и с которого переводят на естественный язык. Поразительная способность человека к идентификации огромного множества структурно различных фраз с одинаковым \textit{смыслом} и способность \uline{запомнить смысл вне этих фраз} убеждает нас в этом.>>

Приведем также слова \textit{Мельчука И. А.}~\scncite{MelchukST}:

<<Идея была следующая -- язык надо описывать следующим образом: надо уметь записывать смыслы фраз. \uline{Не фразы, а их \textit{смыслы}}, что отдельно. Плюс построить систему, которая по смыслу строит фразу. Это та область или тот поворот исследований, при котором интуиция способного лингвиста работает лучше всего: как выразить на данном языке данный смысл. Это -- то, для чего лингвистов учат..

Лингвистический \textit{смысл} научного текста -- это совсем не то, что ты, читая его, из него извлекаешь. Это, очень грубо говоря, инвариант синонимических перифраз. Ты можешь один и тот же смысл выразить очень многими способами. Когда ты говоришь, то можешь сказать по-разному: ``Сейчас я налью тебе вина'', или: ``Дай, я тебе предложу вина'', или: ``Не выпить ли нам по бокалу?'', -- все это имеет один и тот же смысл. И вот можно придумать, как записывать этот \textit{смысл}. Именно его. Не фразу, а \textit{смысл}. И работать надо от этого \textit{смысла} к реальным фразам. Синтаксис там по дороге тоже нужен, но он нужен именно по дороге, он не может быть ни конечной целью, ни начальной точкой. Это -- промежуточное дело.>>~\scncite{Melchuk}.
}

\scnnote{Грамотная унификация (стандартизация) \textit{смыслового представления информации} не должна привести к ограничению творческой свободы авторов различного вида публикуемых научно-технических знаний (и, в том числе, разработчиков \textit{баз знаний}), не должна гарантировать \textit{семантическую совместимость} различных \textit{знаний}, представленных различными авторами (разумеется, при условии соблюдения соответствующих правил построения этих \textit{знаний}). При этом любые \textit{стандарты} (в том числе и принятые стандарты \textit{смыслового представления информации}) должны постоянно эволюционировать. Текущая версия любого стандарта должна быть не догмой, а точкой опоры для дальнейшего совершенствования этого стандарта.}

\scnsuperset{УСК}
\scnaddlevel{1}
\scnidtf{Универсальный Семантический Код}
\scnrelfrom{автор}{Мартынов В. В.}
\scnnote{Разработанный Мартыновым В. В. Универсальный Семантический Код стал важнейшим этапом создания универсальных формальных средств смыслового представления знаний. Основная методологическая идея \textit{Мартынова В. В.}, касающаяся построения \textit{языка смыслового представления знаний}, заключается в том, чтобы выделить смысловые "кирпичики"{}, имеющие достаточно общий характер, а многообразие конкретных смыслов конструировать комбинаторно за счёт различных комбинаций (конфигураций) из этих "кирпичей"{}. Это можно назвать принципом минимизации типов атомарных смысловых фрагментов}

\scnauthorcomment{Дооформить библиографию}

\scnrelto{ключевой знак}{Книга УСК}


\scnheader{смысловое представление информации*}
\scnidtfexp{\textit{Бинарное ориентированное отношение}, каждая \textit{пара} которого связывает некоторую \textit{информационную конструкцию} со смысловым представлением этой \textit{информационной конструкции*}}

\scnsubset{формализация*}

\bigskip
\scnfragmentcaption

\scnheader{формализация*}
\scniselementrole{ключевой знак}{Начало Предметной области и онтологии кибернетических систем}
\scnaddlevel{1}
\scnsourcecommentpar{Начало Раздела 1.1}
\scniselement{начало раздела Базы знаний IMS.ostis}
\scnaddlevel{-1}
\scniselement{бинарное ориентированное отношение}
\scnidtf{формализация информации*}
\scnidtf{пара, связывающая менее формализованное и более формализованное представление некоторой информации*}
\scnidtf{формализация информационной модели некоторой описываемой (моделируемой) системы взаимосвязанных сущностей*}
\scnidtf{Бинарное ориентированное отношение, каждая \textit{пара} которого, связывает два \textit{семантически эквивалентных} знания, второе из которых является более точным (более точно сформированным) знанием по сравнению с первым \textit{знанием}*.}
\scnexplanation{Повышение точности (строгости) формулировки знания -- минимизация (а в идеале -- исключение) \uline{неоднозначной} семантической интерпретации этой формулировки, т.е. несоответствия того, что хотел "сказать"{} автор формулировки, и того, как его поняли. Формализация знаний предполагает (1) точное (строгое) описание \textit{синтаксиса и денотационной семантики} того \textit{языка}, на котором формулируются \textit{знания} и (2) максимально возможное \uline{упрощение} синтаксических и семантических принципов, лежащих в основе указанного \textit{языка}. Очевидно, что \textit{естественные языки} указанным требованиям не удовлетворяют и, следовательно, не могут быть основой для точной формулировки \textit{научно-технических знаний} и, соответственно, для представления этих \textit{знаний} в \textit{памяти интеллектуальных компьютерных систем}. Очевидно также, что разработка \textit{\uline{универсального} языка} формального представления научно-технических знаний является \uline{основой} для глубокой конвергенции различных научно-технических дисциплин, для расширения областей применения современной математики и даже для появления новых разделов математики, которые, например, изучают общие свойства \textit{универсального смыслового пространства} и, в частности, свойство семантического расстояния(семантической близости) как между различными \textit{знаками}, так и между различными \textit{знаковыми конструкциями} (конфигурациями знаков).}
\scnaddlevel{1}
\scnnote{Слово "математика"{} означает "точное знание"{}.}
\scnaddlevel{1}
\scnrelto{цитата}{\textit{Арнольд В.И. Что TM--2012кн-c.4}}
\scnaddlevel{-2}
\scnauthorcomment{Дооформить}

\scnnote{Формализация информационной модели есть не что иное как "движение"{} в сторону семантического (смыслового) представления этой модель, т.е. переход к такому представлению этой модели, в котором мы избавляемся от всего, не имеющего отношения к сути моделируемой системы и касающегося только способа построения этой модели (т.е. её синтаксической структуры). }
\scnnote{Нет проблемы записать любое \textit{знание} в компьютерную \textit{память}. Для этого надо придумать соответствующий формат их кодирования. Но есть проблема представить это \textit{знание} так, чтобы с ним было легко работать, чтобы с использованием этого \textit{знания} можно было достаточно удобно (без лишних накладных расходов, обусловленных выбранным способом представления) решать самые различные информационные \textit{задачи} (задачи интеграции знаний, информационного поиска по базе знаний, верификации и оптимизации баз знаний, логического вывода, поиска способов решения задач, хранимых в базе знаний и т. д.).
Какими характеристиками должно обладать удобное представление знаний, удовлетворяющее указанным требованиям. Очевидно, что такое представление есть не что иное, как формальная (математическая) модель, семантически эквивалентная этим знаниям. Т.е. удобно представить знание -- это фактически построить соответствующую этому знанию \textit{математическую модель}.
Для интеллектуальных компьютерных систем важно не просто приобрести знания, но и представить их в такой форме, которая была бы удобна не только для человека (пользователя и разработчика), но и для различных компьютерных систем, т.е. не требовала бы переоформление (перезаписи) этих знаний для различных компьютерных систем. Очевидно, что такая форма записи (представления) знаний должна быть абсолютно не зависящий от различных компьютерных платформ.
Это и есть главная цель формализации знаний, обеспечивающей эффективную автоматизацию обработки этих знаний.}
\scnheader{формальное представление информации}\\
\scnsubset{информация}
\scnaddlevel{1}
\scnidtf{информационная конструкция}
\scnaddlevel{-1}
\scntext{вопрос}{Почему разработка и использование формальных моделей (математических моделей) представления \textit{информации} является важнейшим этапом развития любой научной и научно-технической дисциплины.
}\scnaddlevel{1}
\scnrelfromset{ответ}{\scnfileitem{Формализация любой \textit{предметной области} даёт возможность более конструктивно накапливать, интегрировать, понимать и систематизировать новые \textit{знания} об этой \textit{предметной области}};
\scnfileitem{Формализация \textit{предметной области} обеспечивает более строгую верификацию, обоснование (аргументацию, доказательство) и согласование различных точек зрения};
\scnfileitem{Формализация \textit{предметной области} создает условия для разработки строгих и легко воспроизводимых (реализуемых) \textit{методов} решения различных \textit{классов задач}}}
\scnaddlevel{1}
\scniselement{конъюнкция*}
\scnrelto{достоинства}{формальное представление информации}
\scnaddlevel{-2}
\scnidtf{формальное (формализованное) представление информационной конструкции}
\scnsubset{смысловое представления информации}
\scnnote{Высшим уровнем качества \textit{формального представления информации} является смысловое представление этой информации}
\scnidtf{формальная модель системы описываемых взаимосвязанных сущностей}
\scnidtf{математическая модель системы описываемых взаимосвязанных сущностей}
\scnidtf{формула}
\scnnote{Сам термин ``\textit{формальное представление информации}'' свидетельствует о том, что при таком представлении \textit{информации} сама \uline{форма} представляемой информационной конструкции (т.е. синтаксическая структура этой конструкции) имеет очевидную аналогию с описываемой конфигурацией связей между соответствующими соответствующими описываемыми \textit{сущностями}.
В предельном "идеальном"{} случае указанная аналогия между формой и смыслом информационной конструкции должна быть изоморфизмом.}
\scnnote{Формализация формализации рознь и, соответственно, степень приближения формы представления информации к "идеальному"{} смысловому представлению может быть различной. Разработка такого "идеального"{} \textit{языка смыслового представления информации} должна руководствоваться следующими основными критериями:
	\begin{scnitemize}		
		\item максимально возможное упрощения синтаксиса (как можно меньше синтаксических излишеств и синтаксического разнообразия).
		\item обеспечение \uline{универсальности} языка.
	\end{scnitemize}

Подчеркнем, что обеспечение универсальности \textit{языка смыслового представления информации} является весьма нетривиальной задачей, т.к. сложно одновременно достигнуть две противоречащие друг другу цели- обеспечить простоту синтаксиса языка и его неограниченную семантическую мощность. Косвенным подтверждением этого является большое количество созданных человечеством специализированных \textit{формальных языков}, \textit{языков смыслового представления информации} и даже \textit{языков семантических сетей}, что свидетельствует о востребованности \textit{смыслового представления информации}.}
\scnsubdividing{формальное представление информации, не являющееся смысловым
;смысловое представление информации, не являющееся семантической сетью
;нерафинированная семантическая сеть
\scnaddlevel{1}
\scnidtf{смысловое представления информации 2-го уровня}
\scnaddlevel{-1}
;рафинированная семантическая сеть
\scnaddlevel{1}
\scnidtf{смысловое представление информации 3-го уровня}
\scnaddlevel{-1}}

\bigskip
\scnfragmentcaption

\scnheader{смысловое представление информации, не являющееся семантической сетью}
\scnnote{Данному уровню смыслового представления информации соответствуют предлагаемые нами универсальные формальные языки SCs-код и SCn-код}

\scnsuperset{SCs-код}
\scnaddlevel{1}
\scniselement{универсальный формальный язык}
\scniselementrole{ключевой знак}{Описание языка линейного представления знаний ostis-систем}
\scnaddlevel{-1}
\scnsuperset{SCn-код}
\scnaddlevel{1}
\scniselement{универсальный формальный язык}
\scniselementrole{ключевой знак}{Описание языка структурированного представления знаний ostis-систем}
\scnaddlevel{-1}

\scnreltovector{принципы, лежащие в основе}{\scnfileitem{В состав \textit{смыслового представления информации, не являющегося семантической сетью} могут входить все уровни иерархии представления информационной конструкции --
\begin{scnitemize}
		\item синтаксически элементарные фрагменты информационной конструкции, из которых строятся простые знаки описываемых сущностей, а также разделители и ограничители;
		\item простые знаки;
		\item выражения;
		\item простые тексты;
		\item сложные тексты;
		\item простые знания;
		\item сложные знания.
\end{scnitemize}};
\scnfileitem{Множество всех описываемых сущностей, \uline{не являющихся связями}, разбивается на два подмножества:
\begin{scnitemize}
	\item каждой сущности, принадлежащей первому подмножеству, \uline{взаимно однозначно} соответствует множество \uline{синтаксически эквивалентных} (синтаксически одинаковых) \textit{простых знаков}, каждый из которых обозначает указанную сущность;
	\item каждой сущности, принадлежащей второму подмножеству, соответствует в общем случае \uline{семейство} множеств, кажо из которых является максимальным множеством синтаксически эквивалентных выражений, обозначающих указанную сущность.
\end{scnitemize}
}
\scnaddlevel{1}
\scntext{следовательно}{Здесь синонимия \textit{простых знаков}, имеющих \uline{разную} синтаксическую структуру, отсутствует, а вот синонимия \textit{выражений}, имеющих разную синтаксическую структуру, вполне возможна. Подчеркнем при этом, что в рамках \textit{смыслового представления информации, не являющегося семантической сетью}, \scnbigspace \textit{знаки} (как \textit{простые знаки}, так и \textit{выражения}), имеющие одинаковую синтаксическую структуру, считаются также и семантически эквивалентными, т.е. обозначающими одну и ту же сущность. Это означает отсутствие омонимии синтаксически эквивалентных знаков}
\scntext{следовательно}{В рамках \textit{смыслового представления информации, не являющегося семантической сетью}, простые знаки, обозначающие \uline{разные} сущности, имеют легко устанавливаемое отличие своих синтаксических структур, а простые знаки, обозначающие одну и ту же сущность имеют легко устанавливаемое сходство своих синтаксических структур. Таким образом, в рамках \textit{смыслового представления информации, не являющегося семантической сетью}, \scnbigspace \uline{дублирование знаков}, т.е. многократное вхождение \textit{знаков} одной и той же сущности, \uline{допускается}}
\scnaddlevel{-1};
\scnfileitem{Связи как вид описываемых сущностей имеют очень важные особенности:
\begin{scnitemize}
	\item каждой описываемой \textit{связи} \uline{однозначно}, а в подавляющем числе случаев и \uline{взаимно однозначно} соответствует \textit{простой текст}, являющийся контекстом (спецификацией) этой \textit{связи};
	\item весьма редки \textit{кратные связи}, т.е. \textit{свзяи}, принадлежащие одному и тому же \textit{отношению} и связывающие одинаковым образом одни и те же \textit{сущности};
	\item довольно редко \textit{связи} являются компонентами других \textit{связей}.
\end{scnitemize}}
\scnaddlevel{1}
\scntext{следовательно}{Для подавляющего числа описываемых \textit{связей} нет никакой необходимости вводить обозначающие их \textit{знаки}, если эти \textit{связи} описываются соответствующими \textit{простыми текстами}. Вместо таких \textit{знаков} можно ввести условные представления этих \textit{связей}, отражающие их вид и направленность. Такие условные представления (изображения) описываемых \textit{связей} можно считать \textit{знаками}, но \textit{знаками}, семантические свойства которых принципиально отличаются от тех \textit{знаков} описываемых \textit{сущностей}, которые мы рассматривали выше. Любые данного вида разные \textit{знаки} описываемых \textit{связей} даже, если, они являются \textit{синтаксически эквивалентными}, т.е. имеют одинаковую структуру, считаются \textit{знаками} \uline{разных} описываемых \textit{связей}. Синонимия таких \textit{знаков} принципиально возможна, но только в том случае, если \textit{простые тексты}, описывающие соответствующие \textit{связи}, будут полностью \uline{продублированы}.}
\scnaddlevel{-1};
\scnfileitem{Для описания связей между описываемыми сущностями в смысловом представлении информации нет необходимости использовать такие приемы естественных языков, как склонения, спряжения, семантическая значимость последовательности знаков.};
\scnfileitem{В случае, если с помощью \textit{простых текстов} необходимо описать контекст (спецификацию) нескольких \uline{кратных} \textit{связей}, все эти \textit{связи} необходимо обозначить \textit{знаками} первого типа -- знаками, \textit{синтаксическая структура} каждого из которых \uline{уникальна.}
Кроме этого, необходимо ввести знак, который обозначает \textit{связь инцидентности} между описываемой \textit{связью} и компонентом этой \textit{связи}, и который относится к числу \textit{знаков} второго типа -- \textit{знаков}, разные экземпляры (разные вхождения) которого считаются обозначениями \uline{разных} \textit{связей}};
\scnfileitem{Для явного указания синонимии двух разных \textit{знаков} первого типа, имеющих разную \textit{синтаксическую структуру}, вводится фиктивная \textit{связь равенства}, которая сама не является описываемой \textit{связью}, а только указывает факт синонимии двух \textit{знаков}, по крайней мере один из которых должен быть \textit{выражением}.};
\scnfileitem{Каждая описываемая \textit{сущность} должна быть специфицирована путем указания типа этой \textit{сущности}. Описываемая \textit{сущность} может быть:
\begin{scnitemize}
	\item \textit{материальной сущностью};
		  \newline
		  \textit{точкой абстрактного пространства};
		  \newline
		  \textit{множеством}:
		  \begin{scnitemizeii}
		  	\item \textit{связью};
		  	\item \textit{классом};
		  	\item \textit{структурой};
		  \end{scnitemizeii}
	\item \textit{реальной сущностью};
		  \newline
		  \textit{вымышленной сущностью};
	\item \textit{константой};
		  \newline
		  \textit{переменной};
	\item \textit{постоянной сущностью};
		  \newline
		  \textit{временной сущностью}:
		  \begin{scnitemizeii}
		  	\item \textit{прошлой сущностью};
		  	\item \textit{настоящей сущностью};
		  	\item \textit{будующей сущностью}.
		  \end{scnitemizeii}
\end{scnitemize}
Кроме того, сам \textit{знак} описываемой сущности может иметь следующий статус:
\begin{scnitemize}
	\item \textit{логически удаленный знак};
	\item \textit{настоящий знак};
	\item \textit{будущий знак}.
\end{scnitemize}};
\scnfileitem{Возможно дублирование информации, т.е. могут присутствовать семантически эквивалентные информационные конструкции, входящие в остав одной информационной конструкции (например, в состав информации, хранимой в памяти одной компьютерной системы). Но при этом есть принципиальная возможность обнаружить такое дублирование информации}}

\bigskip
\scnfragmentcaption

\scnheader{графовая структура}

\scnidtfdef{абстрактная (математическая) структура, которая задается:
\begin{scnitemize}
	\item множеством ее элементов:
		\begin{scnitemizeii}
		 \item множеством ее вершин (узлов);
		 \item множеством ее связок:
		 \begin{scnitemizeiii}
		 \item множеством ее ребер (неориентированных пар 		элементов графовой структуры);
		 \item множеством ее дуг (ориентированных пар элементов графовой структуры);
		 \item множеством ее гиперребер, каждое из которых является конечным множеством элементов графовой структуры, имеющим мощность больше двух
		 \end{scnitemizeiii}
		 \end{scnitemizeii}
	\item бинарным ориентированным отношением инцидентности, связывающим каждую связку графовой структуры с каждым компонентом (элементом) этой связки.
\end{scnitemize}}

\scnheader{следует отличать*}
\scnhaselementset{\scnfileitem{\textit{графовую структуру} как абстрактный математический объект, в рамках которого не уточняется то, как выглядят (представляются, изображаются) элементы графовой структуры и связи их инцидентности}
;\scnfileitem{представление (изображение) \textit{графовой структуры} -- ее рисунок, ее представление в компьютерной памяти в виде матрицы инцидентности, матрицы смежности, списковой структуры}}


\scnheader{графовая структура}
\scnidtftext{часто используемый sc-идентификатор}{дискретная информационная конструкция}
\scnnote{Поскольку любая \textit{графовая структура} является дискретной математической моделью, которая может описывать любое множество \textit{сущностей}, связанных между собой заданным множеством \textit{связей}, все \textit{графовые структуры} с полным основанием можно считать дискретными \textit{информационными конструкциями}. Более того, любая дискретная \textit{информационная конструкция} (в том числе, и обычная цепочка символов) с формальной точки зрения является \textit{графовой структурой}. Тот факт, что теория графов рассматривает "синтаксические"{} свойства \textit{графовых структур} с точностью до их изоморфизма, не лишает \textit{графовые структуры} соответствующих "семантических"{} свойств.}
\scnexplanation{С семантической точки зрения графовая структура -- это нелинейная (в общем случае) знаковая конструкция, в состав которой могут входить знаки \uline{любых} сущностей. При этом указанные знаки \uline{синтаксически} разбиваются на два класса --
	\begin{scnitemize}
		\item на \textit{знаки} сущностей, которые не являются \uline{связями} между сущностями -- в теории графов такие знаки называются узлами (вершинами);
		\item на знаки \uline{связей} между \textit{сущностями} -- к таким \textit{знакам} относятся ребра неориентированных графов, гиперребра гиперграфов, дуги ориентированных графов.
	\end{scnitemize}	
Кроме того, на множестве знаков \textit{сущностей}, входящих в состав \textit{графовой структуры}, задаются \textit{отношения инцидентности}, которые связывают \textit{знаки} связей, входящих  в состав \textit{графовой структуры}, со знаками тех \textit{сущностей} которые являются компонентами указанных \textit{связей}.


Теория графов рассматривает только "синтаксические"{} аспекты \textit{графовых структур}.
Семантика \textit{графовой структуры} задается \textit{онтологией}, специфицирующей систему понятий, экземплярами которых являются элементы этой графовой структуры, т.е. \textit{знаки}, входящие в состав этой \textit{графовой структуры}.}


\scnheader{семантическая сеть}
\scnidtf{\textit{графовая структура}, являющаяся \uline{формальным уточнением} одного из видов \textit{смыслового представления информации}}
\scnsubset{графовая структура}
\scnsubset{смысловое представление информации}
	\scnaddlevel{1}
	\scnsubset{знаковая структура}
	\scnaddlevel{-1}
\scnidtf{графовая структура, \uline{вершины} (узлы) которой трактуются как знаки некоторых описываемых сущностей, а \uline{связки} (ребра, дуги, гиперребра, гипердуги) которой трактуются как знаки связей между описываемыми сущностями и/или знаками этих сущностей} 
\scnidtf{\uline{абстрактная} графовая и в общем случае нелинейная знаковая конструкция (знаковая структура), являющаяся вариантом \uline{смыслового} представления соответствующей информации}
\scnidtfexp{информационная конструкция, в которой \uline{явно} выделены знаки \uline{всех} описываемых сущностей, а также знаки связей, которые также считаются описываемыми сущностями и которые связывают либо сами описываемые сущности, либо описываемые сущности со знаками других описываемых сущностей, лиюо знаки описываемых сущностей}
\scnnote{Теоретико-графовая трактовка (уточнение) \textit{смыслового представления информации} является вполне естественной, т.к. любая описываемая сущность может иметь неограниченное количество связей с другими описываемыми сущностями, и очень часто анализ свойств какой-либо описываемой сущности предполагает анализ всех представленных (описанных) связей этой сущности с различными другими сущностями. Более того, для любых описываемых сущностей существует связывающая их связь (все в Мире взаимосвязано). Вопрос в том, какая это связь и нужно ли ее описывать. Далеко не все то, что можно описывать, целесообразно описывать.}
\scnrelfromvector{общие предпосылки}{
\scnfileitem{Информация в знаковой конструкции содержится не в самих знаках, а в конфигурации связей между знаками, обозначающими описываемые сущности}
;\scnfileitem{Конфигурация связей между описываемыми сущностями \uline{в общем случае} \uline{не} являются линейной} 
;\scnfileitem{Идеальным \textit{смысловым представлением информации} следует считать такую знаковую конструкцию, синтаксическая конфигурация связей между знаками которой \uline{изоморфна} конфигурации связей между описываемыми сущностями}}
\scnnote{Понятие семантической сети является основным понятием для \textit{Технологии OSTIS}. Ранее семантические сети рассматривались не как основа технологии разработки интеллектуальных компьютерных систем, а как наглядная иллюстрация представления знаний, не имеющая практической перспективы из-за сложности реализации, не обладающая универсализмом.


Для нас семантические сети -- это
	\begin{scnitemize}
	\item формальный подход к построению знаковых конструкций:
	\item формальный подход, позволяющий создавать целое \uline{семейство} языков и в том числе языков \uline{универсальных}:
	\item основа организации памяти нового типа -- структурно перестраиваемой (реконфигурируемой) памяти, обработка информации в которой сводится к реконфигурации связей между ее элементами.
	\end{scnitemize}}

\scnrelfromlist{достоинства}{\scnfileitem{\textbf{Семантическая сеть} наряду с системами правил является весьма распространенным способом представления знаний в интеллектуальных системах. Особое значение этот способ представления знаний приобретает в связи с развитием сети интернет. Кроме ряда особенностей, позволяющих применять семантические сети в тех случаях, когда системы правил не применимы, \textbf{семантические сети} обладают следующим важным свойством: они дают возможность \textbf{соединения в одном представлении синтаксиса и семантики} или \uline{синтаксического и семантического аспектов описаний} знаний предметной области. Происходит это благодаря тому, что в семантических сетях наряду с переменными для обозначения тех или иных объектов (элементов множеств, некоторых конструкций из них) присутствуют и сами эти элементы и конструкции; присутствуют и связи, сопоставляющие тем или иным переменным множества допустимых интерпретаций. Эти обстоятельства позволяют во многих случаях резко \textbf{уменьшить реальную вычислительную сложность решаемых задач}.
\newline
Помимо изобразительных возможностей, \textbf{семантические сети обладают более серьезными достоинствами}. То обстоятельство, что \textbf{вся информация об индивиде представлена в единственном месте} -- в одной вершине -- означает, что вся эта информация непосредственно доступна в этой вершине, что, в свою очередь, \textbf{сокращает время поиска}, в частности, при выполнении унификации и подставновки в задачах логического вывода.
Существует еще одна, более \textbf{тонкая особенность} расширенных семантических сетей -- они позволяют \textit{интегрировать в одном представлении \textbf{синтаксис и семантику}} (т.е. интерпретацию) клаузальных форм. Это позволяет в процессе вывода обеспечивать взаимодействие синтаксических и семантических, теоретико-модельных подходов, что, в свою очередь, также является фактором, зачастую делаютщим вывод практически более эффективным}\\
	\scnaddlevel{1}
	\scnrelto{цитата}{Осипов Г.С.-Метод ИИ-2015кн,с.43-54}
	\scnaddlevel{1}
	\scnrelto{часть}{Осипов Г.С.-Метод ИИ-2015кн}
	\scnaddlevel{-2}
;\scnfileitem{Все связи между \textit{знаками}, входящими в состав \textit{семантической сети} представляются с помощью специальных связующих элементов \textit{семантической сети} (дуг, ребер) и, следовательно, для описания указанных связей в \textit{семантической сети} нет необходимости использовать такие средства, как предлоги, союзы, падежи, склонения, спряжения, различные разделители и ограничители, что существенно упрощает обработку \textit{знаний}.}
;\scnfileitem{Соединение синтаксических и семантических аспектов в \textit{семантической сети} проявляется в том, что дуга или ребро, "синтаксически"{} соединяющая элементы \textit{семантической сети} описывает наличие соответствующей \textit{связи} между \textit{сущностями}, обозначаемыми указанными элементами \textit{семантической сети}.}}

\scnauthorcomment{Дооформить ссылки}


\bigskip
\scnfragmentcaption

\scnheader{нерафинированная семантическая сеть}
\scnnote{Переход от смыслового представления информации, не являющегося семантической сетью, к нерафинированным семантическим сетям представляет собой переход к информационным конструкциям, имеющим более простую синтаксическую структуру и денотационную семантику.\\
\newline
К нерафинированным семантическим сетям можно отнести тексты предлагаемого нами универсального формального SCg-кода, а также используемые в Semantic Web rdf-графы}

\scnsuperset{SCg-код}
\scnaddlevel{1}
\scnhaselement{универсальный формальный язык}
\scnhaselementrole{ключевой знак}{Описание языка графического представления знаний ostis-систем}
\scnaddlevel{-1}
\scnsuperset{rdf-граф}

\scnreltovector{принципы, лежащие в основе}{\scnfileitem{Поскольку в \textit{информационной конструкции} информация содержится не в самих \textit{знаках} (если не считать \textit{знаки}, являющиеся \textit{выражениями}), а в конфигурации связей между \textit{знаками}, очень важно \uline{явно} формально представить саму эту конфигурацию \textit{знаков}. И как нельзя лучше для этого подходит понятие \textit{графовой структуры} и, соответственно, понятие \textit{семантической сети}.\\
Что касается \textit{выражений}, то каждое из них легко трансформируется в \textit{семантически эквивалентную} информационную конструкцию, не являющиюся \textit{выражением}. Заметим, что \textit{выражения} используются исключительно для минимизации числа вводимых \textit{знаков} (имен) с уникальной синтаксической структурой.}
;\scnfileitem{\uline{Все} элементы, входящие в состав нерафинированной семантической сети и представленные узлами, ребрами или дугами, являются \textit{знаками}, обозначающими соответствующие описываемые \textit{сущности}, причём \textit{знаками} второго типа, которые, обозначая соответствующую \textit{сущность}, входят в \textit{информационную конструкцию} \uline{однократно} (отсутствует многократное вхождение \textit{знаков}, обозначающих одну и ту же \textit{сущность}). Также \textit{знаки} могут иметь синтаксическую структуру, которая не является уникальной для обозначаемой \textit{сущности}, а отражает только принадлежность этой сущности к соответствующих классам.
Таким образом, в \textit{нерафинированной семантической сети} в отличие от \textit{смыслового представления информации не являющегося семантической сетью}, доминируют не \textit{знаки} первого типа, а \textit{знаки} второго типа, которыми в \textit{нерафинированной семантической сети} представлены (обозначены) \uline{все} описываемые \textit{сущности}, а в \textit{смысловом представлении информации, не являющемся семанитеской сетью}, представлены \uline{только} \textit{бинарные связи} \uline{и то не все}.}
;\scnfileitem{\uline{Все} ребра \textit{нерафинированной семантической сети} являются знаками \textit{бинарных неориентированных связей} и формально трактуются как знаки \textit{двухмощных множеств}, каждым \textit{элементом} которых являются либо знак \textit{сущности}, соединяемой указанной \textit{бинарной связью}, либо \textit{знак}, который сам является \textit{сущностью}, соединяемой этой \textit{бинарной связью}. Более того, \uline{все} \textit{двухмощные множества}, не являющиеся \textit{кортежами} (ориентированными парами) в \textit{нерафинированной семантической сети} обозначаются \textit{ребрами} этой сети.}
;\scnfileitem{\uline{Все} дуги \textit{нерафинированной семантической сети} являются знаками \textit{бинарных ориентированных связей} и формально трактуются как знаки \textit{двухмощных кортежей} (ориентированных пар), каждым \textit{элементом} которых является либо знак \textit{сущности}, соединяемой указанной \textit{бинарной связи}, либо \textit{знак}, который сам является \textit{сущностью}, соединяемой этой \textit{бинарной связью}. Более того, \uline{все} \textit{ориентированные пары} в \textit{нерафинированной семантической сети} обозначаются \textit{дугами} этой сети.}
;\scnfileitem{\uline{Каждая} небинарная связь, описываемая в нерафинированной семантической сети, трактуется как множество, мощность которого не равна двум и обозначается соответствующим узлом этой сети, который соединяется дугами, принадлежащими отношению принадлежности со всеми знаками, которые либо обозначаются сущности, связывающие рассматриваемой небинарной связью, либо сами являются такими сущностями. Для описания ориентированных небинарных связей (в частности, небинарных кортежей) выделяется несколько подмножеств отношения принадлежности, соответствующих различным ролям элементов (компонентов) ориентированных небинарных связей.}
;\scnfileitem{В рамках нерафинированной семантической сети \uline{все} рассматриваемые связи между описываемыми сущностями представляются \uline{явно} в виде знаков, обозначающих эти связи.}
;\scnfileitem{В рамках нерафинированной семантической сети не используются такие средства, как разделители, ограничители и др.}
;\scnfileitem{Узлами \textit{нерафинированной семантической сети}, которые обозначают различного вида \uline{ключевые} описываемые \textit{сущности} (прежде всего, различные \textit{понятия}) приписываются уникальные \textit{знаки} (имена) этих \textit{ключевых сущностей}. Очевидно, что каждый такой \textit{узел} и приписываемое ему \textit{имя} -- это \textit{синонимичные знаки}, обозначающие одну и ту же \textit{сущность}, но являющиеся \textit{знаками} двух разных типов -- (1) \textit{знаком}, который \uline{однократно} представлен в рамках \textit{информационной конструкции}; (2) \textit{знаком}, синтаксическая структура которого \uline{взаимно однозначно} соответствует обозначаемой им \textit{сущности}.}
;\scnfileitem{Большинству узлов, обозначающих небинарные связи, большинству ребер и дуг, а также некоторым другим узлам нерафинированной семантической сети могут быть приписаны уникальные знаки (в частности, имена) понятий (чаще всего, отношений), которым принадлежат указанные узлы, ребра и дуги.}}

\bigskip
\scnfragmentcaption

\scnheader{рафинированная семантическая сеть}
\scntext{основной принцип}{Абсолютно ничего лишнего, не имеющего отношения к смыслу представляемой информации}
\scnidtf{\uline{предельно} компактная (сжатая) смысловая информационная модель соответствующей системы рассматриваемых (описываемых, исследуемых, моделируемых) сущностей}
\scnnote{Указанная система рассматриваемых сущностей представляет собой конфигурацию связей между этими сущностями. Подчеркнем при этом, что указанные связи между рассматриваемыми сущностями также входят в число рассматриваемых сущностей.}

\scnidtf{\textit{информационная конструкция}, являющаяся результатом максимально возможного упрощения ее \textit{синтаксической структуры} при обеспечении представления \uline{любой} \textit{информации}, что приводит к фактическому слиянию синтаксических и семантических аспектов представления \textit{информации}}
\scnidtf{\textit{семантическая сеть} "внутреннего"\ потребления, используемая для \textit{смыслового представления информации} в памяти \textit{компьютерных систем}}
\scnidtf{уточнение принципов \textit{смыслового представления информации}, которое основано, \uline{во-первых}, на четком противопоставлении \textit{внутреннего языка компьютерной системы}, используемого для хранения информации в памяти компьютера, и \textit{внешних языков компьютерной системы}, используемых для общения (обмена сообщений) \textit{компьютерной системы} с пользователями и другими \textit{компьютерными системами} (рафинированная семантическая сеть используется исключительно для \textit{внутреннего представления информации} в памяти \textit{компьютерной системы}), и, \uline{во-вторых} на максимально возможном упрощении \textit{синтаксиса внутреннего языка компьютерной системы} при обеспечении \uline{универсальности} путем исключения из такого внутреннего универсального языка средств, обеспечивающих коммуникационную функцию \textit{языка} (т.е. обмен сообщениями). 
\newline
Так, например, для \textit{внутреннего языка компьютерной системы} излишними являются такие коммуникационные средства \textit{языка}, как союзы, предлоги, разделители, ограничители, склонения, спряжения и другие.
\newline
\textit{Внешние языки компьютерной системы} могут быть как близки ее внутреннему языку, так и весьма далеки от него (как, например, \textit{естественные языки}).}
\scnidtf{\uline{абстрактная} знаковая конструкция, принадлежащая \uline{универсальному} внутреннему языку компьютерных систем и являющаяся \uline{инвариантом} соответствующего максимального множества семантически эквивалентных знаковых конструкций (текстов), принадлежащих самым различным языкам}

\scnrelfromvector{принципы лежащие в основе}{\scnfileitem{Каждый фрагмент \textit{рафинированной семантической сети} является либо \textit{знаком} (элементарным фрагментом, представленным либо \textit{узлом}, либо \textit{ребром}, либо \textit{дугой}), либо множеством \textit{знаков}, связанных между собой отношением \textit{инцидентности} элементов \textit{рафинированной семантической сети}. Указанное отношение \textit{инцидентности} является \textit{бинарным ориентированным отношением}, связывающим \textit{знаки} описываемых \textit{связей} (которые представляются \textit{ребрами}, \textit{дугами} и \textit{узлами}, если описываемая связь является небинарной) со \textit{знаками}, которые либо обозначают связываемые \textit{сущности}, либо сами являются такими сущностями}
\scnaddlevel{1}
\scntext{следовательно}{В состав \textit{рафинированной семантической сети} не входят такие средства синтаксической структуризации знаковых конструкций, как \textit{разделители} и \textit{ограничители}. Любая структуризация \textit{рафинированных семантических сетей} описывается явно с помощью метаязыковых средств путем:
\begin{scnitemize}
	\item введения узлов \textit{рафинированной семантической сети}, обозначающих различные \uline{не\-э\-ле\-мен\-тар\-ные} фрагменты этой семантической сети, являющиеся \textit{множествами} узлов, ребер и дуг, входящих в состав обозначаемого фрагмента;
	\item введения \textit{дуг принадлежности}, связывающих введенные \textit{узлы}, обозначающие неэлементарные фрагменты \textit{рафинированной семантической сети}, с элементами обозначаемых ими \textit{множеств};
	\item введения целого ряда \textit{отношений}, связывающих неэлементарные фрагменты \textit{рафинированной семантической сети} с другими фрагментами, а также с сущностями других видов;
	\item введения различных классов неэлементарных фрагментов \textit{рафинированной семантической сети}.
\end{scnitemize}}
\scnaddlevel{-1};
\scnfileitem{Абсолютно все \textit{знаки}, входящие в состав \textit{рафинированной семантической сети}, являются синтаксически элементарными (атомарными) фрагментами \textit{рафинированной семантической сети}, т.е. фрагментами, "внутренняя"\ структура которых не имеет никакого значения для семантического анализа и понимания \textit{рафинированной семантической сети}. Множеству \textit{знаков}, входящих в \textit{рафинированную семантическую сеть}, как и множеству \textit{букв}, входящих в обычный \textit{текст}, ставится в соответствие \textit{алфавит}, определяющий \uline{синтаксическую типологию} таких элементарных фрагментов \textit{рафинированной семантической сети}. При этом, если \textit{алфавит} букв обычного \textit{текста} не имеет никакой семантической интерпретации, то \textit{алфавит} элементарных фрагментов \textit{рафинированной семантической сети} имеет четкую семантическую интерпретацию -- каждый элемент этого \textit{алфавита} обозначает класс знаков \textit{сущности}, \uline{синтаксический тип} которых соответствует указанному элементу \textit{алфавита} (задается этим элементом \textit{алфавита знаков}, входящих в состав \textit{рафинированной семантической сети})}
\scnaddlevel{1}
\scntext{следовательно}{Таким образом, \textit{знаки}, входящие в \textit{рафинированную семантическую сеть}, не являются \textit{именами} (терминами) и, следовательно, не привязаны ни к какому \textit{естественному языку} и не зависят от субъективных терминотворческих пристрастий различных авторов. Это значит, что при коллективной разработке \textit{рафинированных семантических сетей}, соответствующих каким-либо информационным ресурсам, терминологические споры практически исключены.}
\scntext{следовательно}{В \textit{рафинированной семантической сети} нет необходимости использовать синтаксически элементарные фрагменты, \uline{не} являющиеся знаками описываемых \textit{сущностей}, т.е. фрагменты \textit{информационной конструкции}, из которых сторятся \textit{простые знаки}, \textit{выражения}, а также различные разделители и ограничители. Более того, в \textit{рафинированной семантической сети} нет необходимости противопоставлять \textit{простые знаки} и \textit{выражения}. Как \textit{простым знакам}, так и \textit{выражениям} в \textit{рафинированной семантической сети} соответствуют элементы этой сети, имеющие аналогичные \textit{денотаты}. Но при этом \textit{выражениям} дополнительно соответствуют семантически эквивалентные неэлементарные фрагменты \textit{рафинированной семантической сети}, которые специфицируют \textit{сущности}, обозначаемые этими \textit{выражениями}.}
\scnaddlevel{-1};
\scnfileitem{Абсолютно ве описываемые \textit{связи} между описываемыми сущностями в \textit{рафинированной семантической сети} представляются \uline{явно} в виде соответствующих \textit{знаков}, обозначающих эти \textit{связи} и инцидентных знакам связываемых \textit{сущностей}. Для бинарных связей, связывающих \uline{две} описываемые сущности, \textit{знаком} связей являются \textit{ребра} или \textit{дуги} \textit{рафинированной семантической сети}.}
\scnaddlevel{1}
\scntext{следовательно}{В \textit{рафинированных семантических сетях} нет необходимости использовать такие средства, как склонения, спряжения, род (мужской, женский, средний), семантически интерпретируемая последовательность слов.}
\scnaddlevel{-1};
\scnfileitem{Все \textit{знаки}, входящие в состав \textit{рафинированной семантической сети}, входят в нее \uline{однократно}. Т.е. в рамках \textit{рафинированной семантической сети} отсутствуют пары \textit{синонимичных знаков}, т.е. \textit{знаков}, имеющих один и тот же \textit{денотат}. Таким образом, разные элементы \textit{рафинированной семантической сети} априори считаются знаками \uline{разных} сущностей. При этом эти знаки могут принадлежать одному и тому же синтаксическому типу, т.е. одному и тому же элементу алфавита соответствующего языка \textit{рафинированных семантических сетей}. Таким образом, в \textit{рафинированных семантических сетях} отсутствует синонимия не только \textit{знаков}, имеющих одинаковую синтаксическую структуру, не только знаков, имеющих одинаковый синтаксический тип, но также и просто \uline{разных} знаков.}
\scnaddlevel{1}
\scntext{следовательно}{Появление в рафинированной семантической сети синонимичных знаков превращает эту семантическую сеть в некорректную и требует отождествления (склеивания) обнаруженных синонимичных знаков.}
\scnaddlevel{-1};
\scnfileitem{В рамках \textit{рафинированной семантической сети} отсутствуют \textit{синонимичные знаки}, т.е. \textit{знаки}, которые имеют не один, а несколько \textit{денотатов}, каждому из которых соответствует свой контекст (ракурс) семантической трактовки этого \textit{знака}.}
\scnaddlevel{1}
\scnnote{Когда речь идет об омонимии знаков в привычных нам языках, имеется в виду омонимия \uline{разных} знаков, имеющих одинаковую синтаксическую структуру, т.е. омонимия разных вхождений, разных экземпляров \uline{синтаксически эквивалентных}, но семантически различных знаков. Очевидным примером такого рода омонимии являются различного вида местоимения.}
\scnaddlevel{-1};
\scnfileitem{В рамках каждой \textit{рафинированной семантической сети} отсутствует дублирование информации не только в виде многократного вхождения \textit{синонимичных знаков}, т.е. \textit{знаков} с одинаковыми денотатами, но также и в виде многократного вхождения \textit{семантически эквивалентных} \textit{рафинированных семантических сетей}. Две \textit{рафинированные семантические сети} являются \textit{семантически эквивалентными} в том и только в том случае, если:
\begin{scnitemize}
	\item они \textit{изоморфны};
	\item пары соответствия указанного \textit{изоморфизма} связывают \textit{синонимичные знаки}. 
\end{scnitemize}
Таким образом, полное исключение \textit{омонимии знаков} является необходимым и достаточным условием исключения \textit{семантически эквивалентных рафинированных семантических сетей}. Подчеркнем при этом, что запрет \textit{семантической эквивалентности} в рамках \textit{рафинированной семантической сети} не означает запрета \textit{логической эквивалентности} фрагментов \textit{рафинированной семантической сети}. Логическая эквивалентность необходима для обеспечения компактности представления некоторых знаний. Тем не менее, логической эквивалентностью хранимых в памяти знаковых конструкций увлекаться не следует, т.к. \uline{\textit{логически эквивалентные}} знаковые конструкции -- это представление одного и того же \textit{знания}, но с помощью \uline{\textit{разных наборов понятий}}. В отличие от этого \uline{\textit{семантически эквивалентные}} \textit{знаковые конструкции} -- это представление одного и того же \textit{знания} с помощь одних и тех же \textit{понятий}. Очевидно, что многообразие возможных вариантов представления одних и тех же \textit{знаний} в памяти компьютерной системы существенно усложняет решение \textit{задач}. Поэтому, полностью исключив \textit{семантическую эквивалентность} в смысловой памяти, необходимо стремиться к минимизации \textit{логической эквивалентности}. Для этого необходимо грамотное построение системы используемых \textit{понятий} в виде иерархической системы формальных \textit{онтологий}.}
\scnaddlevel{1}
\scntext{следовательно}{Интеграция (соединение, объединение) двух \textit{рафинированных семантических сетей}, в результате чего могут появиться семантически эквивалентные фрагменты, сводится к тому, чтобы результат такого соединения был приведен в соответствие с требованием отсутствия синонимии элементов и семантической эквивалентности фрагментов \textit{рафинированной семантической сети}.}
\scnaddlevel{-1};
\scnfileitem{\textit{Рафинированные семантические сети} должны быть \uline{универсальными}, т.е. должны обеспечивать представление \uline{любой} информации, в том числе, и \textit{метаинформации}, обеспечивающей описание различных связей, свойств и закономерностей самих \textit{рафинированных семантических сетей}, на множестве которых, в частности, заданно \textit{отношение} "быть подструктурой*"\, которое связывает \textit{рафинированные семантические сети} с их фрагментами (частями), т.е. с теми \textit{рафинированными семантическими сетями}, которые входят в их состав.
\newline
Каждая \textit{рафинированная семантическая сеть} трактуется как множество \textit{знаков} \uline{взаимно однозначно} соответствующих обозначаемым ими \textit{сущностям} (денотатам этих \textit{знаков}) и множество \textit{связей} между этими \textit{знаками}.
\newline
Каждая \textit{связь} между \textit{знаками} трактуется, с одной стороны, как множество \textit{знаков}, связываемых этой \textit{связью}, а, с другой стороны, как описание (отражение, модель) соответствующей \textit{связи}, которая связывает денотаты указанных \textit{знаков} или денотаты одних \textit{знаков} непосредственно с другими \textit{знаками}, или сами эти \textit{знаки}. Примером первого вида \textit{связи} между \textit{знаками} является связь между \textit{знаками} \textit{материальных сущностей}, одна из которых является частью другой. Примером второго вида \textit{связи} между \textit{знаками} является \textit{связь} между знаком, входящим в состав внутреннего смыслового представления информации, и знаком файла, являющегося электронным отражением структуры представления указанного \textit{знака} во внешних \textit{знаковых конструкциях}. Примерами третьего вида \textit{связи} между \textit{знаками} является \textit{связь} между синонимичными знаками.
\newline
Денотатами \textit{знаков} могут быть \uline{любые} описываемые сущности, причем: (1) не только конкретные (константные, фиксированные), но и произвольные (переменные, нефиксированные)  сущности, "пробегающие"\ различные множества знаков (возможных значений), (2) не только реальные (материальные), но и абстрактные сущности (например, числа, точки различных абстрактных пространств), (3) не только "внешние"\, но и "внутренние"{} сущности, являющиеся множествами знаков, входящих в состав той же самой знаковой конструкции, хранимой в памяти компьютерной системы.};
\scnfileitem{Поскольку \textit{рафинированные семантические сети} ориентированы на \textit{смысловое представление информации} в памяти \textit{компьютеров нового поколения}, необходимо, с одной стороны, использовать накопленный полезный опыт представления информации в \textit{современных компьютерах}, а, с другой стороны, обеспечить взаимодействие \textit{компьютерных систем}, построенных на \textit{современных компьютерах}, с \textit{компьютерными системами}, построенными на \textit{компьютерах нового поколения}. Для этой цели в памяти \textit{компьютеров нового поколения} можно и нужно обеспечить обработку и хранение различного вида \textit{информационных конструкций}, представленных в различных широко используемых форматах. И ничто не препятствует такие \textit{информационные конструкции}, хранимые в памяти \textit{компьютера нового поколения} и не являющиеся \textit{рафинированными семантическими сетями}, рассматривать как \textit{сущности}, описываемые \textit{рафинированной семантической сетью}, хранимой в памяти этого \textit{компьютера нового поколения}. Такой вид \textit{сущностей}, описываемых \textit{рафинированной семантической сетью} и хранимых в той же \textit{памяти}, будем называть \textit{файлами}, описываемыми соответствующуей \textit{рафинированной семантической сетью}, т.е. "электронными"{} \ образами (копиями) соответствующих \textit{информационных конструкций}. Таким образом, среди \textit{узлов рафинированной семантической сети} появляются \textit{узлы}, являющиеся знаками \textit{файлов}, т.е. \textit{узлы}, денотаты (обозначаемые \textit{сущности}) которых находятся (хранятся) в той же памяти, что и обозначающие их \textit{узлы}.}
\scnaddlevel{1}
\scntext{следовательно}{Ничто не мешает в виде \textit{файла}, описываемого \textit{рафинированной семантической сетью}, хранить \textit{имя} (термин) какой-либо \textit{сущности}, описываемой этой же семантической сетью, а также связать это \textit{имя} (точнее, узел, обозначающий это \textit{имя}) с тем элементом \textit{рафинированной семантической сети}, который обозначает ту же описываемую \textit{сущность}.}
\scnaddlevel{-1};
\scnfileitem{Следствием указанных принципов \textit{рафинированных семантических сетей} является также то, что эти принципы приводят к нелинейным \textit{знаковым конструкциям} (к \textit{графовым структурам}), что усложняет реализацию \textit{памяти компьютерных систем}, но существенно упрощает ее логическую организацию (в частности, ассоциативный доступ).
\newline
Нелинейность \textit{рафинированных семантических сетей} обусловлена тем, что:
\begin{scnitemize}
	\item каждая описываемая \textit{сущность}, т.е. \textit{сущность}, имеющая соответствующий ей \textit{знак}, может иметь неограниченное число \textit{связей} с другими описываемыми \textit{сущностями};
	\item каждая описываемая \textit{сущность} в смысловом представлении имеет единственный \textit{знак}, т.к. синонимия \textit{знаков} здесь запрещена;
	\item все \textit{связи} между описываемыми \textit{сущностями} описываются (отражаются, моделируются) \textit{связями} между \textit{знаками} этих описываемых \textit{сущностей}.
\end{scnitemize}}
\scnaddlevel{1}
\scnnote{Напомним, что нелинейность информационных конструкций характерна не только для рафинированных, но и для нерафинированных семантических сетей.}
\scnaddlevel{-1}
}

\scnsuperset{SC-код}
\scnaddlevel{1}
\scnidtf{Semantic Computer Code}
\scniselement{универсальный формальный язык}
\scniselementrole{ключевой знак}{Описание внутреннего языка ostis-сиcтем}
\scnaddlevel{1}
\scnsourcecommentpar{Раздел 0.3.1}
\scnaddlevel{-1}
\scnexplanation{В качестве \textit{стандарта} \uline{универсального} \textit{смыслового представления информации} \textit{в памяти компьютерных систем} нами предложен SC-код (Semantic Computer Code). В отличие от УСК \textit{Мартынова В.В.}, он, во-первых, носит нелинейный характер и, во-вторых, специально ориентирован на кодирование информации в памяти компьютеров \uline{нового поколения}, ориентированных на разработку семантически совместимых \textit{интеллектуальных компьютерных систем} и названных нами \textit{семантическими ассоциативными компьютерами}. Более подробно это понятие (\textit{SC-код}) рассмотрено в разделе \textit{Предметная область и онтология внутреннего языка osts-систем}. Таким образом, основым лейтмотивом предлагаемого нами \textit{смыслового представления информации} является ориентация на формальную модель памяти \textit{компьютерных} \uline{не}фон-неймановского \textit{компьютера}, предназначенного для реализации \textit{интеллектуальных систем}, использующих \textit{смысловое представление информации}. Особенностями такого представления являются следующие:
\begin{scnitemize}
	\item ассоциативность памяти;
	\item поскольку при смысловом представлении информациия содержится в конфигурации связей между знаками, переработка информации сводится к реконфигурации этих связей (к графодинамическим процессам);
	\item прозрачная семантическая интерпретируемость и, как следствие, \textit{семантическая совместимость}.
\end{scnitemize}
Подчеркнем что, неявная привязка к фон-неймановским \textit{компьютерам} присутствует во всех известных \textit{моделях представления знаний}. Одним из примеров такой зависимости, является, например, обязательность именования описываемых объектов.} 
\scnaddlevel{-1}
\scnrelfromset{достоинства}{\scnfileitem{рафинированная семантическая сеть есть \uline{объективный}, не зависящий от субъективизма и многообразия синтаксических решений, способ представления информации};
\scnfileitem{в рамках \textit{рафинированной семантической сети} существенно упрощается процедура \textit{интеграции знаний} и погружения новых знаний в \textit{базу знаний}};
\scnfileitem{существенно упрощается процедура приведения различного вида \textit{знаний} к общему виду (к согласованной системе используемых \textit{понятий})};
\scnfileitem{существенно упрощается процедура интеграции различных \textit{решателей задач} и целых \textit{компьютерных систем}};
\scnfileitem{существенно упрощается автоматизация перманентного процесса \textit{поддержки семантической совместимости} (согласованности \textit{понятий} и \textit{онтологий}) для \textit{компьютерных систем} в условиях их постоянного совершенствования};
\scnfileitem{в рамках \textit{рафинированных семантических сетей} достаточно легко осуществляется переход от информационных конструкций к информационным \uline{мета}конструкциям путем введения узлов \textit{семантической сети}, обозначающих \textit{информационные конструкции}, а также дуг, связывающих эти узлы со всеми элементами обозначаемой ими \textit{информационной конструкции}};
\scnfileitem{на основе \textit{рафинированных семантических сетей} существенно упрощается интеграция различных дисциплин в области \textit{Искуственного интеллекта}, т.е. построение \textit{Общей формальной теории интеллектуальных компьютерных систем}, так как для построения общей формальной модели \textit{интеллектуальных компьютерных систем} необходим базовый \textit{язык}, в рамках которого можно было бы легко переходить от информации (от \textit{знаний}) к \textit{метаинформации} (к метазнаниям, к спецификациям исходных \textit{знаний}). Это потверждается тем, что:
\begin{scnitemize}
	\item подавляющее число \textit{понятий} \scnbigspace \textit{Искусственного интеллекта} носит метаязыковой характер;
	\item формальное смысловое уточнение почти каждого \textit{понятия} \scnbigspace \textit{Искусственного интеллекта} требует предшествующего формального уточнения соответсвующего языка-объекта. Так, например, как можно строго говорить о \textit{языке онтологий} (т.е. \textit{языке} спецификации \textit{предметных областей}), не уточнив \textit{язык} представления самих этих \textit{предметных областей}. как можно строго говорить о \textit{языке} описания способов обработки \textit{информации}, не уточнив \textit{язык }представления самой этой обрабатываемой \textit{информации}.
\end{scnitemize}}}

\bigskip
\scnfragmentcaption

\scnheader{язык смыслового представления информации}
\scnidtf{смысловой язык}
\scnidtf{семантический язык}
\scnsubdividing{язык смыслового представления информации, не являющийся языком семантических сетей;
язык семантических сетей}

\scnheader{язык семантических сетей}
\scnexplanation{Несмотря на то, что синтаксическая структура семантической сети во многом носит \uline{объективный} характер, поскольку определяется конфигурацией описываемых связей между описываемыми сущностями. Тем не менее, можно говорить о разных \textit{языках семантических сетей}, каждому из которых соответствует свой \textit{алфавит*} элементов (синтаксически атомарных фрагментов) \textit{семантических сетей}. При атом языки семантических сетей могут быть как специализированными, так и универсальными. Задача каждого из этих \textit{языков} -- обеспечить в рамках \textit{языка} полное отсутствие многообразия синтаксических форм представления одной и той же информации.}

\scnsubset{язык}
\scnaddlevel{1}
\scnidtf{множество информационных конструкций, для которого существуют, причем не обязательно в формализованном виде, (1) правила построения синтаксически корректных информационных конструкций, а также (2) правила, позволяющие установить семантическую корректность правильно построенных (синтаксически корректных) информационных конструкций}
\scnaddlevel{-1}
\scnidtf{язык, информационными конструкциями которого являются семантические сети и в рамках которого обеспечивается полное отсутствие многообразия форм представления одной и той же информации}
\scnidtf{графовый (нелинейный) язык смыслового представления информации}

\scnsubdividing{специализированный язык семантических сетей
\scnaddlevel{1}
\scnidtf{язык семантических сетей, семантическая мощность которого ограничена соответствующей предметной областью}
\scnaddlevel{-1};
универсальный язык емантических сетей
\newline
\scnaddlevel{1}
\scnnote{Человечество давно и широко использует различные специализированные языки семантических сетей -- язык принципиальных электрических схем, язык блок-схем программ, язык генеалогических деревьев и др. Но в настоящее время актуальным является создание такого \textit{универсального языка семантических сетей}
\begin{scnitemize}
	\item синтаксис и семантика которого были бы максимально просты;
	\item по отношению к которому все используемые специализированные языки были бы его подъязыками*;
	\item который был бы приспособлен к использованию в качестве внутреннего языка интеллектуальных компьютерных систем и компьютеров следующего поколения;
	\item который был бы удобной основой как для обмена информацией между интеллектуальными компьютерными системами, так и для общения интеллектуальных клмпьютерных систем с их пользователями
\end{scnitemize}
\scnaddlevel{-1}
}}
\scnsubdividing{язык нерафинированных семантических сетей;
язык рафинированных семантических сетей}

\scnheader{следует отличать*}
\scnhaselementset{язык семантических сетей
\scnaddlevel{1}
\scnidtf{язык семантических сетей, рассматриваемых как \uline{абстрактные} графовые структуры, в которых не уточняется способ их кодирования}
\scnhaselement{SC-код}
\scnaddlevel{-1};
графодинамический язык семантических сетей
\scnaddlevel{1}
\scnidtf{язык графического изображения (визуализации) семантических сетей}
\scnidtf{язык, текстами которого являются рисунки семантичеких сетей}
\scnhaselement{SCg-код}
\scnaddlevel{-1}}

\scnheader{универсальный язык семантических сетей}
\scnnote{Если ставить задачу разработки \uline{универсального}(!) языка, текстами которого являются графовые структуры, то классических графовых структур явно недостаточно. Так, например:
\begin{scnitemize}
	\item по аналогии с переходом от ребер к ребрам и гиперребрам необходим переход от дуг к ориентированным связкам, связывающим более чем два компонента и в рамках которых эти компоненты могут иметь разные роли, которые необходимо явно указывать (классическим видом таких связок являются кортежи);
	\item в семантических сетях, представляющих некоторые виды знаний, некоторые связки (ребра, дуги, гиперребра, ориентированные связки, связывающие более двух компонентов) могут быть компонентами других связок;
	\item в семантических сетях, представляющих различного вида метазнания необходимо вводить узлы, обозначающие целые фрагменты (подграфы) этих же семантических сетей, и, соответственно, вводить дуги, связывающие каждый из этих узлов со всеми элементами подграфа, обозначаемого этим узлом.
\end{scnitemize}}

\bigskip
\scnfragmentcaption

\scnheader{семантическая модель базы знаний}
\scnidtfexp{смысловое представление всей \textit{базы знаний} \textit{интеллектуальной компьютерной системы} в виде \textit{семантической сети}, принадлежащей \textit{универсальному языку семантических сетей}}
\scnexplanation{Для того, чтобы семантические сети могли быть использованы в качестве средства представления \textit{знаний} в памяти \textit{интеллектуальной компьютерной системы} необходимо:
\begin{scnitemize}
	\item рассмотреть \textit{семантические сети} как тексты, представляющие \uline{различного вида} \textit{знания};
	\item уточнить синтаксис и семантику \uline{универсального} (!) \textit{языка представления знаний}, текстами которого являются \textit{семантические сети} [Информатика: Энциклопедический словарь стр. 195].
	Считается, что \textit{семантические сети} являются теоретической моделью \textit{представления знаний}, не используемой на практике. [Информатика: Энциклопедический словарь стр. 207]
	Однако, если реализовать \textit{графодинамическую память} и разработать \textit{языки программирования}, ориентированные на обработку информации в такой памяти, то уникальные достоинства \textit{семантических сетей} будут практически использованы в полной мере.
\end{scnitemize}}

\scnauthorcomment{Дооформить библиографию}

\scnheader{семантическая модель базы знаний}
\scnrelfromvector{достоинства}{\scnfileitem{\textit{семантическая модель базы знаний}, построенная на основе \textit{универсального языка семантических сетей}, обеспечивает высокий уровень ассоциативности доступа к требуемым фрагментам \textit{базы знаний} благодаря широкому многообразию реализуемых видов запросов и существенному снижению реальной вычислительной сложности алгоритмов доступа (информационного поиска)};
\scnfileitem{\textit{семантическая модель базы знаний} позволяет реализовать эффективную семантическую навигацию по текущему состоянию \textit{базы знаний} (при просмотре \textit{базы знаний}) путем отображения различного вида \textit{семантических окрестностей} для указываемых \textit{элементов семантической сети}. При этом \textit{семантическая модель базы знаний} позволяет реализовать \uline{наглядную} двумерную или трехмерную визуализацию просматриваемого фрагмента \textit{базы знаний} (просматриваемой \textit{семантической сети}). Таким образом, \textit{семантическая сеть} является средством представления \textit{знаний}, удобным как для самой \textit{интеллектуальной компьютерной системы}, так и для её пользователей.};
\scnfileitem{сущностями, вписываемыми в \textit{семантической модели базы знаний} и, соответственно, обозначаемыми \textit{знаками} этих сущностей, могут быть не только \textit{части внешней среды} соответствующие \textit{интеллектуальной компьютерной системе}, но и \textit{части} (фрагменты) самой \textit{базы знаний}.  Это дает возможность \textit{базе знаний} включать в себя описание собственной структуры с любой степенью детализации и рассматривающее самые разные аспекты такой структуризации.};
\scnfileitem{\textit{семантическая модель базы знаний} позволяет \uline{явно} выделить фрагменты \textit{базы знаний}, представляющие различные \textit{предметные области} и соответствующие им \textit{онтологии}, а также \uline{явно} описать иерархию выделенных \textit{предметных областей} и иные связи мужду ними (например, различного рода морфизмы). Такая семантическая структуризация \textit{базы знаний} позволяет осуществлять локализацию области действия каждой конкретной операции обработки \textit{базы знаний}, что существенно упрощает реализацию этих операций. Каждая \textit{предметная область} выделенная в рамках \textit{семантической модели базы знаний}, описывающая соответствующий класс исследуемых (описываемых) \textit{сущностей} (объектов исследования) и соответствующих подклассов этого \textit{класса} с помощью соответствующего набора \textit{отношений} (в том числе, \textit{функций} и \textit{алгебраических операций}) и соответствующего набора \textit{свойств} (параметров), представляет собой результат интеграции (соединения) текстов соответствующего \textit{специализированного языка семантической сети}.};
\scnfileitem{размещение каждой информации, хранимой в составе \textit{семантической модели базы знаний}, и, соответственно, доступ к этой информации (поиск её в \textit{базе знаний}) определяются \uline{исключительно} семантическими характеристиками этой информации (т.е. её смыслом) и не зависят от особенностей реализации памяти \textit{интеллектуальной компьютерной системы}. Т.е. смысл информации \uline{однозначно} определяет её "местоположение" в \textit{семантической модели базы знаний}, а, точнее, её связи с остальной частью этой \textit{базы знаний}};
\scnfileitem{если объем представления \textit{информационной конструкции} определить как пару, состоящую (1) из количества синтаксически элементарных (атомарных) фрагментов этой конструкции и (2) из числа элементов \textit{алфавита} указанных элементарных фрагментов, и если рассмотреть множество всевозможных \uline{\textit{семантически эквивалентных*}} представлений каждой \textit{информационной конструкции}, то наиболее \uline{компактным} (сжатым) её представлением окажется представление в виде \textit{рафинированной семантической сети}. Более того, при расширении \textit{базы знаний} (при увеличении числа описываемых \textit{сущностей} и, в частности, числа описываемых \textit{связей}) компактность \textit{рафинированных семантических сетей} повышается, т.к. новые описываемые \textit{связи} далеко не всегда рассматривают связи между \uline{новыми} \textit{сущностями}, которые до этого не описывались.};
\scnfileitem{\textit{семантические сети} позволяют говорить о принципиально ином характере соединения ("конкатенации"\, интеграции) двух текстов в один интегрированный текст. \textit{Интеграция} двух \textit{семантических сетей} предполагает склеивание (отождествление) \uline{синонимичных} элементов интегрируемых \textit{семантических сетей}. Такая \textit{интеграция}, в частности, происходит при вводе (погружении) новой информации в состав \textit{семантической модели базы знаний}.};
\scnfileitem{семантические модели баз знаний дают возможность:
\begin{scnitemize}
\item конструктивно осуществлять анализ \textit{семантической связности базы знаний}, путем уточнения понятия семантической силы связи между различными элементами и фрагментами базы знаний;
\item конструктивно осуществлять кластеризацию баз знаний;
\item задавать метрику семантического расстояния между знаками, входящими в состав базы знаний;
\item осуществлять в рамках базы знаний описания различного вида соответствий (морфизмов) между различными фрагментами базы знаний (изоморфизмов, гомоморфизмов, аналогий и отличий различного вида). Так, например, большое значение имеет исследование таких соответствий между различными \textit{предметными областями};
\item широко использовать мощный арсенал теоретико-графовых \textit{алгоритмов} для выполнения различного рода операций обработки \textit{баз знаний}.
\end{scnitemize}}}

\bigskip
\scnfragmentcaption

\scnheader{следует отличать*}
\scnhaselementset{предельно омонимичный класс синтаксически эквивалентных знаков
\scnaddlevel{1}
\scnidtf{класс синтаксически эквивалентных знаков, все экземпляры (все вхождения) которого являются знаками \uline{разных} сущностей}
\scnaddlevel{1}
\scntext{следовательно}{В рамках рассматриваемого класса знаков синонимия знаков отсутствует}
\scnaddlevel{-1}
\scnidtf{максимальный класс синтаксически эквивалентных знаков, среди которых отсутствуют синонимичные знаки}
\scnaddlevel{-1};
частично омонимичный класс синтаксически эквивалентных знаков
\scnaddlevel{1}
\scnidtf{класс синтаксически эквивалентных знаков, среди экземпляров которого встречаются как синонимичные знаки, так и знаки \uline{разных} сущностей}
\scnaddlevel{-1};
неомонимичный класс синтаксически эквивалентных знаков
\scnaddlevel{1}
\scnidtf{класс синтаксически эквивалентных знаков, \uline{все} экземпляры которого являются знаками одной и той же сущности}
\scnidtf{класс синтаксически эквивалентных знаков, синтаксическая структура которых однозначно идентифицирует (соответствует) обозначаемую ими сущность}
\scnaddlevel{-1};
множество особенностей (характеристик), которыми обладает сущность, обозначаемая заданным знаком*
}
\scnhaselementset{смысловое представление информации*
\newline
\scnaddlevel{1}
\scnidtftext{часто используемый sc-идентификатор}{смысл*}
\scnaddlevel{-1};
смысловое представление информации
\newline
\scnaddlevel{1}
\scnrelto{второй домен}{смысловое представление информации*}
\scnaddlevel{-1};
синтаксическая структура информационной конструкции*;
синтаксическая структура информационной конструкции
\newline
\scnaddlevel{1}
\scnrelto{второй домен}{синтаксическая структура информационной конструкции*}
\scnaddlevel{-1};
денотационная семантика информационной конструкции*
\scnaddlevel{1}
\scnidtf{\textit{соответствие} (морфизм) между синтаксической структурой заданной информационной конструкции и ее \textit{смысловым представлением*}}
\scnnote{\textit{соответствие} между знаками входящими в состав \textit{рафинированной семантической сети} и их \textit{денотатами*} (обозначаемыми сущностями) являются \uline{взаимно однозначными}
\scnaddlevel{-1}}
}
\scnheader{следует отличать*}
\scnhaselementset{смысловое пространство
\newline
\scnaddlevel{-1}
\scnhaselement{SC-пространство}
\scnaddlevel{2}
\scnidtf{семантическое пространство}
\scnexplanation{объединение (соединение) всевозможных корректных абстрактных семантических сетей, принадлежащих некоторому языку абстрактных \textit{рафинированных семантических сетей}}
\scnidtf{глобальная (максимальная) абстрактная \textit{рафинированная семантическая сеть}, включающая в себя всевозможные абстрактные рафинированные семантические сети соответствующего языка}
\scnidtf{абстрактное смысловое пространство}
\scnaddlevel{-1};
абстрактная смысловая память
\scnaddlevel{1}
\scnidtf{абстрактная семантическая память}
\scnidtf{среда, обеспечивающая хранение абстрактных рафинированных семантических сетей, а также редактирование этих семантических сетей и при этом абстрагирующаяся от деталей этих процессов}
\scnidtf{абстрактная графодинамическая память, обеспечивающая хранение и редактирование абстрактных рафинированных семантических сетей}
\scnaddlevel{-1};
реальная смысловая память
\scnaddlevel{1}
\scnidtf{физическая реализация абстрактной смысловой памяти}
\scnaddhind{-1}
\scnrelboth{следует отличать}{программная реализация \uline{модели} абстрактной смысловой памяти на современных компьютерах}
\scnaddlevel{-1}}

\bigskip

\scnendstruct \scninlinesourcecommentpar{Завершили Раздел \ref{sem_mod_comp_sys} ``\nameref{sem_mod_comp_sys}''}

\end{SCn}