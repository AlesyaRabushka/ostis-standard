\begin{SCn}

\scnsectionheader{\currentname}

\scnstartsubstruct

\scnheader{Предметная область агентно-ориентированных семантических моделей решателей задач}
\scnsdmainclasssingle{***}
\scnsdclass{***}
\scnsdrelation{***}

\scnheader{совместимость моделей решения задач*}
\scnexplanation{Предлагаемый нами подход к существенному повышению уровня совместимости (интегрируемости) различных \textbf{моделей решения задач} заключается в следующем:

\begin{scnitemize}
    \item Вся информация, хранимая в памяти каждого \textit{\textbf{решателя задач}} (как собственно обрабатываемая информация, так и хранимые в памяти интерпретируемые навыки, например, различного вида программы), представляется в форме смыслового представления этой информации; 
    \item Собственно решение каждой задачи осуществляется коллективом агентов, работающих над общей для них смысловой (семантической) памятью и выполняющих интерпретацию хранимых в этой же памяти навыков;
    \item Интеграция двух разных моделей решения задач сводится:
    \begin{scnitemizeii}
        \item к объединению памяти первой модели с памятью второй модели;
        \item к интеграции всей информации, хранимой в памяти первой модели, с информацией, хранимой в памяти второй модели (эта интеграция осуществляется путем взаимного погружения соответствующих информационных конструкций друг в друга, т.е. путем склеивания синонимов, а также путем выравнивания используемых ими понятий);
        \item к объединению множества агентов, входящих в состав первой модели, со множеством агентов, входящих во вторую модель решения задач.
    \end{scnitemizeii}
\end{scnitemize}

Таким образом, унификация моделей решения задач путем приведения этих моделей к виду семантических моделей (т. е. моделей обработки информации, представленной в смысловой форме) повышает уровень совместимости этих моделей благодаря наличию прозрачной процедуры интеграции информации, представленной в смысловой форме, и тривиальной процедуры объединения множеств \textit{агентов}. Простота процедуры объединения множеств \textit{агентов}, соответствующих разным моделям решения задач, обусловлена тем, что непосредственного взаимодействия между этими агентами нет, а инициирование каждого из них определяется им самим, а также \uline{текущим состоянием} хранимой в памяти информации.

Таким образом, в качестве основы унификации принципов обработки информации в компьютерных системах предлагается использовать \textbf{многоагентный подход}. Ориентация на многоагентный подход обусловлена следующими основными преимуществами такого подхода \cite{Wooldridge2009}:

\begin{scnitemize}
    \item автономность (независимость) агентов, что позволяет локализовать изменения, вносимые в систему при ее эволюции, и снизить соответствующие трудозатраты;
    \item децентрализация обработки, т.е. отсутствие единого контролирующего центра, что также позволяет локализовать вносимые в систему изменения.
\end{scnitemize}

Но современные принципы построения \textit{\textbf{многоагентных систем}} при их применении для многоагентной обработки \textit{баз знаний} имеют ряд недостатков:
\begin{scnitemize}
    \item знания агента представляются при помощи узкоспециализированных языков, зачастую не предназначенных для представления знаний в широком смысле и онтологий в частности;
    \item большинство современных многоагентных систем предполагает, что взаимодействие агентов осуществляется путем обмена сообщениями непосредственно от агента к агенту;
    \item логический уровень взаимодействия агентов жестко привязан к физическому уровню реализации многоагентной системы;
    \item среда, с которой взаимодействуют агенты, уточняется отдельно разработчиком для каждой многоагентной системы, что приводит к существенным накладным расходам и несовместимости таких многоагентных систем.
\end{scnitemize}

Перечисленные недостатки предлагается устранять за счет использования следующих принципов:
\begin{scnitemize}
    \item коммуникацию агентов предлагается осуществлять путем спецификации (в общей памяти компьютерной системы) действий (процессов), выполняемых агентами и направленных на решение задач;
    \item в роли внешней среды для агентов выступает та же общая память;
    \item спецификация каждого агента описывается средствами языка представления знаний в той же памяти;
    \item синхронизацию деятельности агентов предлагается осуществлять на уровне выполняемых ими процессов;
    \item каждый информационный процесс в любой момент времени имеет ассоциативный доступ к необходимым фрагментам базы знаний, хранящейся в общей памяти.
\end{scnitemize}
}
\scnnote{\textbf{\textit{совместимость моделей решения задач}} -- это возможность использования разными моделями решения задач одних и тех же информационных ресурсов.}

\scnendstruct \scnendcurrentsectioncomment

\end{SCn}