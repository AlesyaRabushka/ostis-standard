\begin{SCn}

\scnsectionheader{Предметная область и онтология человеческой деятельности и соответствующих технологий}
\scnrelfromlist{подраздел}{Предметная область и онтология традиционных компьютерных
технологий;Предметная область и онтология технологий искусственного интеллекта}

\scnstartsubstruct

\scnheader{Предметная область человеческой деятельности и соответствующих технологий}
\scnsdmainclasssingle{***}
\scnsdclass{технология;информационная технология}
\scnsdrelation{***}

\scnheader{технология}
\scnexplanation{Система организации деятельности, направленной на решение сложных задач соответствующего класса и включающая в себя объекты деятельности, уточняемых до необходимого уровня детализации, а также программы (методы) выполнения соответствующих действий и инструменты (в т.ч. средства автоматизации), с помощью которых эти действия выполняются.}

\scnheader{компьютерная технология}

\scnsubdividing{технология проектирования компьютерных систем;технология сборки компьютерных систем;технология эксплуатации компьютерных систем;технология обновления компьютерных систем}
\scnsubdividing{традиционная компьютерная технология;технология искусственного интеллекта}

\scnheader{компьютеризация научной деятельности}
\scnidtf{автоматизация научной деятельности}
\scnproblems{Очевидно, что высшей формой информационной деятельности является \textbf{научная деятельность} и, следовательно, высшим уровнем развития компьютерных систем являются системы, непосредственно и активно участвующие в этой деятельности. Научная деятельность направлена на повышение качества наших знаний об окружающем нас Мире и, следовательно, связана с анализом, обработкой и систематизацией этих знаний. Совершенно очевидно что, если компьютерные системы, направленные на автоматизацию научной деятельности, будут \textbf{понимать} обрабатываемые ими научные знания и, следовательно, будут становиться не пассивными исполнителями, а партнерами научной деятельности, способными самостоятельно анализировать, систематизировать научные знания и использовать их в решении различных задач, то уровень автоматизации научной деятельности будет существенно повышен.

Важнейшими факторами сдерживания научно-технического прогресса в настоящее время являются:
\begin{scnitemize}
    \item многообразие ("вавилонское столпотворение") как естественных, так и формальных языков, используемых для оформления результатов научно-технических исследований;
    \item привязка научно-технических текстов к естественным языкам (монографии, отчеты, статьи);
    \item принципиальное противоречие между принципами эволюции естественных языков как основного средства коммуникации и требованиями, предъявляемыми к научно-техническим языкам.
\end{scnitemize}

Для решения указанных проблем необходимо:
\begin{scnitemize}
    \item построить строгую формальную систему научно-технических языков;
    \item построить четкую связь между научно-техническими и естественными языками; 
    \item обеспечить оформление научно-технических текстов на совместимых формальных языках, понятных и удобных как людям, так и компьютерным системам; 
    \item обеспечить поддержку эволюции всего этого мультиязыкового комплекса.
\end{scnitemize}

Важнейшим направлением повышения эффективности научно-технической деятельности (и, в частности, повышения темпов научно-технического развития) является переход от традиционного варианта оформления результатов этой деятельности (в форме отчетов, статей, монографий, справочников) к представлению научно-технической информации в виде энциклопедической системы взаимосвязанных баз знаний по различным научно-техническим дисциплинам. Формальным результатом любой научной дисциплины должна стать база знаний, отражающая текущее состояние этой дисциплины. 
Для прикладных научных дисциплин дополнительным результатом должна стать доступная для инженеров компьютерная система автоматизации проектирования искусственных систем соответствующего класса.

Представление о трудностях такого перехода сильно преувеличивается, т. к. современные средства инженерии знаний уже готовы к реализации таких проектов. Этому препятствует: 

\begin{scnitemize}
    \item боязнь нового, непривычного; 
    \item необходимость пересмотра организации научно-технической деятельности.
\end{scnitemize}

Но перспективой является переход на качественно новый уровень культуры научно-технического прогресса. Социальная значимость такого перехода заключается в следующем: 

\begin{scnitemize}
    \item Существенно повысятся темпы эволюции научных знаний благодаря тому, что добываемые научные знания представляются в форме, удобной как для людей, так и для компьютерных систем, а также благодаря автоматизации их интеграции, анализа, структуризации и согласования различных точек зрения. 
    \item Существенно повысится эффективность использования научных знаний в разрабатываемых компьютерных системах, благодаря тому, что отпадает необходимость этапа формализации этих знаний для включения их в состав баз знаний.
    \item Возможность непосредственного участия студентов в совершенствовании тех знаний, которые соответствуют изучаемым ими учебным дисциплинам, существенно повысит качество такого обучения, т.к. способствует индивидуальному, активному и системному усвоению учебного материала.
\end{scnitemize}

Основной проблемой развития научно-технической деятельности и, соответственно, ее информатизации является необходимость глубокой \textbf{конвергенции} различных научных дисциплин, о чем говорится в целом ряде работ~\cite{Palagin, Yankovskaya}.

Важной проблемой также является снижение времени и трудоемкости при организации информационного взаимодействия между научными работниками при \textbf{согласовании точек зрения}, при совместном выполнении каких-либо исследований, при совместной работе над статьями или монографиями, при рецензировании. 

При этом следует помнить, что любая точка зрения всегда имеет недостатки (неполноту, нечеткость и т.п.). Поэтому методологически необходимо переходить от практики противостояния точек зрения к практике интеграции точек зрения (в том числе и тех, которые кажутся альтернативными, противоречащими друг другу). Только так при разработке сложных систем можно добиться синергетического эффекта, в основе которого лежит компенсация недостатков одних точек зрения достоинствами других.

Так и должна быть устроена организация коллективного творческого процесса. Автоматизация такого процесса предполагает фиксацию множественности точек зрения и управление процессом согласования этих точек зрения.\\}
\scnaddlevel{1}
\scnsolutionapproach{Анализ проблем эволюции компьютерных систем разного уровня сложности, разного уровня обучаемости и интеллектуальности, разного назначения показывает, что проклятие "вавилонского столпотворения"\ и, как следствие, несовместимость, дублирование и субъективизм согласовываемых информационных ресурсов и моделей их обработки нас преследует везде:

\begin{scnitemize}
\item и в развитии традиционных компьютерных систем;
\item и в развитии технологий искусственного интеллекта;
\item и в развитии методов и средств компьютеризации научной и инженерной деятельности.
\end{scnitemize}

Рассматривая проблему обеспечения совместимости информационных ресурсов и моделей их обработки, следует говорить о разных аспектах решения этой проблемы:

\begin{scnitemize}
\item об обеспечении совместимости между различными компонентами компьютерных систем, а также между целостными компьютерными системами, входящими в коллективы компьютерных систем;
\item об обеспечении совместимости, т.е. высокого уровня взаимопонимания между различными компьютерными системами и их пользователями;
\item об обеспечении междисциплинарной совместимости, т.е. конвергенции различных областей знаний;
\item о методах и средствах постоянного мониторинга и восстановления совместимости в условиях интенсивной эволюции компьютерных систем и их пользователей, которая часто нарушает достигнутую совместимость (согласованность) и требует дополнительных усилий на ее восстановление.
\end{scnitemize}}
\scnaddlevel{-1}

\scnheader{компьютерная технология}
\scnexplanation{Ключевая на текущий момент проблема развития компьютерных технологий в целом и технологий искусственного интеллекта в частности - \textbf{проблема обеспечения информационной совместимости} компьютерных систем и в том числе интеллектуальных компьютерных систем.

Актуальность решения этой проблемы обусловлена тем, что:
\begin{scnitemize}
    \item информационная совместимость компьютерных систем существенно \textbf{повысит уровень их обучаемости} благодаря более эффективному восприятию опыта (знаний и навыков) от других  компьютерных систем;
    \item появится возможность существенно \textbf{расширять многообразие} используемых в компьютерной системе знаний и навыков без необходимости разработки специальных средств их согласования. Это также повышает уровень обучаемости компьютерных систем и позволяет переходить к \textbf{гибридным, синергетическим} компьютерным системам;
    \item появится возможность создания \textbf{коллективов компьютерных систем}, использующих универсальные принципы организации взаимодействия между компьютерными системами на содержательном уровне;
    \item появится возможность не только разрабатывать совместимые компьютерные системы, но и автоматизировать процесс постоянной \textbf{поддержки совместимости компьютерных систем}. Необходимость указанной поддержки вызвана тем, что совместимость компьютерных систем в ходе их эксплуатации и эволюции может нарушаться. Следовательно, должны существовать средства перманентного восстановления совместимости компьютерных систем в условиях их постоянного изменения;
    \item появится возможность автоматизации процесса постоянной поддержки (восстановления) информационной \textbf{совместимости} компьютерных систем не только с другими компьютерными системами, но и \textbf{с их пользователями};
    \item появится возможность существенно сократить сроки разработки новых компьютерных систем с помощью постоянно расширяемой \textbf{библиотеки многократно используемых компонентов компьютерных систем}, имеющих разный уровень сложности (вплоть до типовых встроенных подсистем) и различный вид (типовые встраиваемые знания, например, онтологии, широко используемые навыки, в частности, программы,  интерфейсные подсистемы, обеспечивающие обмен сообщениями с внешними субъектами на заданном внешнем языке).
\end{scnitemize}}

\scnheader{компьютерная система}
\scnevolutiondirections{В эволюции компьютерных систем можно выделить два общих направления.

\textbf{Первое направление} - это 

\begin{scnitemize}
    \item \textbf{расширение множества и многообразия задач}, решаемых компьютерной системой; 
    \item повышение \textbf{сложности этих задач} вплоть до трудно формализуемых (трудно решаемых) задач, интеллектуальных задач, решаемых в условиях неполноты, неточности, нечеткости и т.д.;
    \item повышение \textbf{качества решения задач} либо путем более эффективного использования известных моделей решения задач (например, путем разработки более качественных алгоритмов), либо путем использования принципиально новых моделей решения задач;
    \item расширение \textbf{многообразия используемых видов информации} (знаний);
    \item расширение \textbf{многообразия используемых моделей решения задач}.
\end{scnitemize}

Очевидно, что расширение множества решаемых задач в условиях пусть и большой, но всегда конечной памяти компьютерной системы делает все более и более актуальным переход от частных методов и моделей решения задач к их обобщениям (или, как отмечал Д.А. Поспелов, от связки "ключей"\ к набору "отмычек").

Очевидно также, что многообразие видов задач, решаемых компьютерными системами, многообразие используемых моделей решения задач приводит: 
\begin{scnitemize}
    \item к интегрированным информационным ресурсам;
    \item к интегрированным решателям задач;
    \item к интегрированным компьютерным системам;
    \item к коллективам компьютерных систем.
\end{scnitemize}

Проблема здесь заключается не в самой интеграции, а в ее качестве. Интеграция может быть \textbf{эклектичной}, если не обеспечить совместимость интегрируемых компонентов, а в случае такой совместимости интеграция может привести к новому качеству, к дополнительному расширению множества решаемых задач. Это будет означать переход от эклектичности к гибридности, синергетичности.

\textbf{Второе общее направление} эволюции компьютерных систем -- это повышение уровня их \textbf{обучаемости} и, как следствие, темпов их эволюции.

\textbf{Обучаемость} компьютерной системы определяется:

\begin{scnitemize}
    \item \textbf{трудоемкостью} и темпами приобретения (расширения) и совершенствования активно используемых знаний и навыков;
    \item \textbf{уровнем ограничений}, накладываемых на вид приобретаемых и используемых знаний и навыков (фактически, это ограничения на множество всех тех задач, которые принципиально могут быть решены данной компьютерной системой).
\end{scnitemize}

В свою очередь, \textbf{трудоемкость и темпы расширения и совершенствования} знаний и навыков компьютерной системы определяется:
\begin{scnitemize}
    \item \textbf{гибкостью} -- многообразием и трудоемкостью возможных изменений, вносимых в систему в процессе пополнения системы новыми знаниями и навыками и совершенствования уже приобретенных знаний и навыков;
    \item \textbf{стратифицированностью} -- четким разделением системы на достаточно независящие друг от друга уровни иерархии, т.е. возможностью локализации фрагментов компьютерной системы, не выходя за пределы которых, \uline{априори достаточно} проводить анализ последствий тех или иных вносимых в систему изменений;
    \item \textbf{рефлексивностью} –- способностью анализировать собственное состояние и свою деятельность;
    \item \textbf{гибридностью} -- способностью приобретать и использовать широкое (а в идеале -- неограниченное) многообразие знаний и навыков;
    \item \textbf{уровнем самообучаемости} -- уровнем активности, самостоятельности, целеустремленности в процессе своего обучения, т.е. уровнем способности к обучению \uline{без учителя}, уровнем автоматизации приобретения новых знаний и навыков, а также совершенствования уже приобретенных знаний и навыков;
    \item \textbf{совместимостью} -- трудоемкостью интеграции;
    \item \textbf{способностью к постоянному мониторингу и поддержке своей совместимости} как с другими компьютерными системами, так и со своими пользователями в условиях интенсивной эволюции этих компьютерных систем и их пользователей. 
\end{scnitemize}

\textbf{Совместимость} (трудоемкость интеграции) компьютерных систем может рассматриваться в двух аспектах:

\begin{scnitemize}
    \item в аспекте \textbf{глубокой интеграции} компьютерных систем, что предполагает преобразование нескольких компьютерных систем в одну целостную компьютерную систему путем объединения информационных и функциональных ресурсов интегрируемых компьютерных систем;
    \item в аспекте преобразования нескольких компьютерных систем в \textbf{коллектив взаимодействующих компьютерных систем}, способных к совместному корпоративному решению сложных задач.
\end{scnitemize}

Совместимость (трудоемкость интеграции) компьютерных систем определяется:
\begin{scnitemize}
    \item совместимостью различного вида информации (знаний), хранимой в памяти компьютерной системы;
    \item совместимостью различных моделей решения задач;
    \item совместимостью встроенных (в т.ч. типовых) подсистем, входящих в состав компьютерных систем;
    \item совместимостью внешней информации, поступающей на вход компьютерной системе, с информацией, хранимой в памяти компьютерной системы (трудоемкостью понимания внешней информации -- трансляции, погружения, выравнивания понятий);
    \item коммуникационной (в т.ч. семантической) совместимостью с пользователями и с другими компьютерными системами.
\end{scnitemize}

Важнейшая форма обучения компьютерной системы это приобретение новых знаний и навыков в "готовом"\ виде, т.е. в виде некоторых знаковых конструкций, вводимых в память компьютерной системы, поскольку приобретение знаний и навыков из внешних достоверных источников требует существенно меньшего времени по сравнению с их приобретением собственными силами, на основе собственного опыта и собственных ошибок. Но для того, чтобы указанная форма обучения была эффективной, необходимо максимально возможным образом упростить и формализовать механизм (процедуру) погружения новых знаний в память компьютерной системы.

Для решения этой задачи ключевое значение имеет создание удобного для этой цели способа кодирования различного вида информации в памяти компьютерной системы.

Поскольку основным каналом обучения компьютерных систем является приобретение ими знаний и навыков от других субъектов -- от других компьютерных систем и от пользователей (от разработчиков-учителей и от конечных пользователей), важнейшим фактором обучаемости компьютерной системы является превращение компьютерной системы в коммуникативную систему, способную эффективно общаться с внешними субъектами. Следовательно, уровень обучаемости компьютерных систем определяется также уровнем ее совместимости с самими этими внешними субъектами, с приобретаемыми ею знаниями и навыками, т.е. степенью того, как компьютерная система вместе с теми субъектами, с которыми она обменивается информацией, решает проблему "вавилонского столпотворения".\\}
\scnaddlevel{1}
\scnsolutionapproach{Суть предлагаемого нами подхода к решению проблем эволюции компьютерных систем заключается, во-первых, в объединении всех указанных выше направлений эволюции компьютерных систем (как общих направлений, так и частных) и, во-вторых, в трактовке проблемы обеспечения \textbf{совместимости} различных видов знаний, различных моделей решения задач, различных компьютерных систем как \textbf{ключевой проблемы} эволюции компьютерных систем, решение которой существенно упростит решение и многих других проблем.

Так, например, без обеспечения совместимости информационных ресурсов, используемых в разных компьютерных системах, а также информационных ресурсов, представляющих знания различного семантического вида невозможно:

\begin{scnitemize}
\item ни создавать \textbf{коллективы компьютерных систем}, способные координировать свои действия при кооперативном расширении сложных задач;
\item ни создавать \textbf{гибридные компьютерные системы}, которые способны при решении сложных комплексных задач использовать всевозможные сочетания разных видов знаний и разных моделей решения задач;
\item ни использовать \textbf{компонентную методику проектирования} компьютерных систем \textbf{на всех уровнях} иерархии проектируемых систем.
\end{scnitemize}

О какой информационной совместимости и взаимопонимании (в т.ч. между специалистами) можно говорить при наличии ужасающей понятийной и терминологической неряшливости, терминологического псевдотворчества, в том числе, в области информатики.

Говоря о \textbf{совместимости} компьютерных систем и их компонентов, а также совместимости компьютерных систем с пользователями, следует отметить неоднозначность трактовки термина ``совместимость''. В этой связи следует отличать:

\begin{scnitemize}
    \item совместимость как один из факторов обучаемости, как \textbf{способность} к быстрому повышению уровня согласованности (интеграции, взаимопонимания).
    Сравните обучаемость как \textbf{способность} к быстрому расширению знаний и навыков, но никак не характеристика объема и качества приобретенных знаний и навыков;
    \item совместимость как характеристика достигнутого уровня согласованности (интеграции, взаимопонимания).
\end{scnitemize}

Аналогичным образом интеллект компьютерной системы с одной стороны можно трактовать как \textbf{уровень} (объем и качество) приобретенных знаний и навыков, а с другой стороны как \textbf{способность} к быстрому расширению и совершенствованию знаний и навыков, т.е. как \textbf{скорость} повышения уровня знаний и навыков.

Кроме того, следует говорить не только о \textbf{способности} к быстрому повышению уровня согласованности и не только о достигнутом уровне согласованности, но и о самом \textbf{процессе} повышения уровня согласованности и, прежде всего, о перманентном процессе восстановления (поддержки, сохранения) достигнутого уровня согласованности, поскольку в ходе эволюции компьютерных систем и их пользователей (т. е. в ходе расширения и повышения качества их знаний и навыков) уровень их согласованности может понижаться.\\}
\scnaddlevel{-1}

\scnendstruct

\end{SCn}