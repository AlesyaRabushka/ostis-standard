\begin{SCn}

\scnheader{семантическая модель базы знаний}
\scnidtfexp{смысловое представление всей \textit{базы знаний} \textit{интеллектуальной компьютерной системы} в виде \textit{семантической сети}, принадлежащей \textit{универсальному языку семантических сетей}}
\scnexplanation{Для того, чтобы семантические сети могли быть использованы в качестве средства представления \textit{знаний} в памяти \textit{интеллектуальной компьютерной системы} необходимо:
\begin{scnitemize}
	\item рассмотреть \textit{семантические сети} как тексты, представляющие \uline{различного вида} \textit{знания};
	\item уточнить синтаксис и семантику \uline{универсального} (!) \textit{языка представления знаний}, текстами которого являются \textit{семантические сети} [Информатика: Энциклопедический словарь стр. 195].
	Считается, что \textit{семантические сети} являются теоретической моделью \textit{представления знаний}, не используемой на практике. [Информатика: Энциклопедический словарь стр. 207]
	Однако, если реализовать \textit{графодинамическую память} и разработать \textit{языки программирования}, ориентированные на обработку информации в такой памяти, то уникальные достоинства \textit{семантических сетей} будут практически использованы в полной мере.
\end{scnitemize}}

\scnauthorcomment{Дооформить библиографию}

\scnheader{семантическая модель базы знаний}
\scnrelfromvector{достоинства}{\scnfileitem{\textit{семантическая модель базы знаний}, построенная на основе \textit{универсального языка семантических сетей}, обеспечивает высокий уровень ассоциативности доступа к требуемым фрагментам \textit{базы знаний} благодаря широкому многообразию реализуемых видов запросов и существенному снижению реальной вычислительной сложности алгоритмов доступа (информационного поиска)};
\scnfileitem{\textit{семантическая модель базы знаний} позволяет реализовать эффективную семантическую навигацию по текущему состоянию \textit{базы знаний} (при просмотре \textit{базы знаний}) путем отображения различного вида \textit{семантических окрестностей} для указываемых \textit{элементов семантической сети}. При этом \textit{семантическая модель базы знаний} позволяет реализовать \uline{наглядную} двумерную или трехмерную визуализацию просматриваемого фрагмента \textit{базы знаний} (просматриваемой \textit{семантической сети}). Таким образом, \textit{семантическая сеть} является средством представления \textit{знаний}, удобным как для самой \textit{интеллектуальной компьютерной системы}, так и для её пользователей.};
\scnfileitem{сущностями, вписываемыми в \textit{семантической модели базы знаний} и, соответственно, обозначаемыми \textit{знаками} этих сущностей, могут быть не только \textit{части внешней среды} соответствующие \textit{интеллектуальной компьютерной системе}, но и \textit{части} (фрагменты) самой \textit{базы знаний}.  Это дает возможность \textit{базе знаний} включать в себя описание собственной структуры с любой степенью детализации и рассматривающее самые разные аспекты такой структуризации.};
\scnfileitem{\textit{семантическая модель базы знаний} позволяет \uline{явно} выделить фрагменты \textit{базы знаний}, представляющие различные \textit{предметные области} и соответствующие им \textit{онтологии}, а также \uline{явно} описать иерархию выделенных \textit{предметных областей} и иные связи мужду ними (например, различного рода морфизмы). Такая семантическая структуризация \textit{базы знаний} позволяет осуществлять локализацию области действия каждой конкретной операции обработки \textit{базы знаний}, что существенно упрощает реализацию этих операций. Каждая \textit{предметная область} выделенная в рамках \textit{семантической модели базы знаний}, описывающая соответствующий класс исследуемых (описываемых) \textit{сущностей} (объектов исследования) и соответствующих подклассов этого \textit{класса} с помощью соответствующего набора \textit{отношений} (в том числе, \textit{функций} и \textit{алгебраических операций}) и соответствующего набора \textit{свойств} (параметров), представляет собой результат интеграции (соединения) текстов соответствующего \textit{специализированного языка семантической сети}.};
\scnfileitem{размещение каждой информации, хранимой в составе \textit{семантической модели базы знаний}, и, соответственно, доступ к этой информации (поиск её в \textit{базе знаний}) определяются \uline{исключительно} семантическими характеристиками этой информации (т.е. её смыслом) и не зависят от особенностей реализации памяти \textit{интеллектуальной компьютерной системы}. Т.е. смысл информации \uline{однозначно} определяет её "местоположение" в \textit{семантической модели базы знаний}, а, точнее, её связи с остальной частью этой \textit{базы знаний}};
\scnfileitem{если объем представления \textit{информационной конструкции} определить как пару, состоящую (1) из количества синтаксически элементарных (атомарных) фрагментов этой конструкции и (2) из числа элементов \textit{алфавита} указанных элементарных фрагментов, и если рассмотреть множество всевозможных \uline{\textit{семантически эквивалентных*}} представлений каждой \textit{информационной конструкции}, то наиболее \uline{компактным} (сжатым) её представлением окажется представление в виде \textit{рафинированной семантической сети}. Более того, при расширении \textit{базы знаний} (при увеличении числа описываемых \textit{сущностей} и, в частности, числа описываемых \textit{связей}) компактность \textit{рафинированных семантических сетей} повышается, т.к. новые описываемые \textit{связи} далеко не всегда рассматривают связи между \uline{новыми} \textit{сущностями}, которые до этого не описывались.};
\scnfileitem{\textit{семантические сети} позволяют говорить о принципиально ином характере соединения ("конкатенации"\, интеграции) двух текстов в один интегрированный текст. \textit{Интеграция} двух \textit{семантических сетей} предполагает склеивание (отождествление) \uline{синонимичных} элементов интегрируемых \textit{семантических сетей}. Такая \textit{интеграция}, в частности, происходит при вводе (погружении) новой информации в состав \textit{семантической модели базы знаний}.};
\scnfileitem{семантические модели баз знаний дают возможность:
\begin{scnitemize}
\item конструктивно осуществлять анализ \textit{семантической связности базы знаний}, путем уточнения понятия семантической силы связи между различными элементами и фрагментами базы знаний;
\item конструктивно осуществлять кластеризацию баз знаний;
\item задавать метрику семантического расстояния между знаками, входящими в состав базы знаний;
\item осуществлять в рамках базы знаний описания различного вида соответствий (морфизмов) между различными фрагментами базы знаний (изоморфизмов, гомоморфизмов, аналогий и отличий различного вида). Так, например, большое значение имеет исследование таких соответствий между различными \textit{предметными областями};
\item широко использовать мощный арсенал теоретико-графовых \textit{алгоритмов} для выполнения различного рода операций обработки \textit{баз знаний}.
\end{scnitemize}}
}
\end{SCn}
