\begin{SCn}

\scnsectionheader{\currentname}

\scnstartsubstruct

\scnheader{Предметная область современных интеллектуальных компьютерных систем}
\scnsdmainclasssingle{интеллектуальная компьютерная система}
\scnsdclass{***}
\scnsdrelation{***}

\scnheader{интеллектуальная система}

\scnexplanation{В процессе эволюции \textit{систем, основанных на обработке информации}, в процессе развития свойств этих систем (увеличение \textit{объема памяти}, повышение \textit{скорости обработки информации}, повышение \textit{уровня гибкости}, уровня структуризации хранимой информации (уровня систематизации обрабатываемой информации, \textit{уровня рефлексивности}, \textit{уровня ассоциативности доступа к хранимой информации}, \textit{уровня стратифицированности}) появляется свойство \textit{интеллектуальности} (разумности) системы, основанной на обработке информации. Это свойство проявляется в целом ряде способностей системы:
\begin{scnitemize}
\item в способности строить \textit{систематизированную информационную модель среды}, в которой функционирует система, причем указанная среда включает в себя как \textit{внешнюю среду} (внешнее материальное окружение, внешние субъекты, с которыми необходимо взаимодействовать, собственное материальное воплощение), так и \textit{внутреннюю среду} (т.е. информацию, хранимую в памяти системы);
\item в способности к рассуждениям...

\scnauthorcomment{см. Финна и Литвинцева новую книгу!!}

\item в способности быстро обучаться

\item гибкость...

\item стратифицированность...

\item рефлексивность...

\end{scnitemize}
}

\scnendstruct \scnendcurrentsectioncomment

\end{SCn}