\begin{SCn}
\scnheader{Подготовка специалистов в области Искусственного интеллекта}
\scnnote{Массовая подготовка высококвалифицированных \textit{специалистов в области Искусственного интеллекта}, способных преодолеть современное кризисное состояние \textit{Искусственного интеллекта}, фактически и является самым главным фактом преодоления указанного кризиса.

Необходимым условием и эпицентром вывода \textit{Искусственного интеллекта} из кризисного состояния и повышения темпов эволюции технологий \textit{Искусственного интеллекта} является организация \textit{подготовки специалистов в области Искусственного интеллекта} на основе активного привлечения студентов, магистрантов и аспирантов к \uline{перманентному} процессу эволюции \textit{Технологии OSTIS}.

Очевидно, этому должно способствовать объединение соответствующей учебно-методической базы для разных кафедр, осуществляющих такую подготовку.

На современном этапе развития \textit{Искусственного интеллекта} требуется не просто подготовка специалистов в этой области -- а подготовка специалистов \uline{принципиально новой формации}, способных
\begin{scnitemize}
	\item рассматривать область \textit{Искусственного интеллекта} не просто как многообразие \textit{интеллектуальных компьютерных систем}, а как постоянно эволюционируемую \uline{\textit{Экосистему}} таких систем
	\item эффективно участвовать в решении как фундаментальных, системных, технологических проблем, так и практических, прикладных проблем эволюции указанной \textit{Экосистемы}
\end{scnitemize}
Все это требует существенного переосмысления организации учебного процесса и учебно-методического обеспечения и 

\scnfileshort{часто, например, совершенствование программных систем сводится к программным "заплаткам"{}. Через какое-то время мы имеем программу со множеством "заплаток"{}, как правило уже громоздкую и малоэффективную. В итоге -- иногда её проще выбросить и создать новую}

\scnaddlevel{-1}
\scnrelfrom{автор}{Курбацкий А. Н.}
\scnaddlevel{1}

Современная разработка каждой сложной программной системы требует построения \uline{качественной} формальной ("цифровой"{}) модели объекта управления, объекта автоматизации, причем \textit{семантически совместимой} с соответствующими моделями в смежных системах.

Здесь важна общая математическая культура и унификация такой формализации.

В настоящее время методологический подход к инженерной деятельности при разработке компьютерных систем часто выглядит следующим образом:

"Поставьте мне четкую инженерную задачу и я ее выполню. Но ответственность за ее постановку я с себя снимаю и не хочу учитывать критерий качества постановки задачи высшего уровня."{}

Для наукоемких проектов, реализуемых в рамках развивающихся технологий, это недопустимо.

Каждый инженер должен \uline{понимать}, что он делает и каковы истинные более глубокие критерии качества его результата.

Нужна принципиально новая психологическая установка.

Необходимо учитывать не только желание заказчика, но и общие принципы и стандарты разрабатываемых \textit{интеллектуальных компьютерных систем}.

В основе организации образовательной деятельности на современном этапе развития \textit{Искусственного интеллекта} лежит:
\begin{scnitemize}
	\item четкое формальное описание того, чему мы учим (каким знаниям и навыкам) -- в нашем случае это описание текущей версии \textit{Стандарта OSTIS} и направлений эволюции этого стандарта;
	\item уточнение того, что должен делать студент, магистрант и любой специалист для быстрого и качественного приобретения этих знаний и навыков.
\end{scnitemize}

Нужна \uline{комплексная} учебная программа по специальности \textit{Искусственный интеллект}, а не мозаика отдельных учебных дисциплин. И, соответственно этому необходимо \uline{комплексное} учебно-методическое пособие, достаточно полно отражающее текущее состояние теории и технологии проектирования \textit{интеллектуальных компьютерных систем}.

Использование проектного метода при подготовке специалистов в области Искусственного интеллекта предполагает составление систематизированного сборника упражнений и задач, в частности, направленных на эволюцию Технологии OSTIS и посильных для студентов специальности \textit{Искусственный интеллект}:
\begin{scnitemize}
	\item представление конкретных фрагментов различных предметных областей и онтологий;	
	\item представление конкретных специфицированных методов (пополнение библиотек используемых методов из разных предметных областей, например, из теории графов);
	\item спецификация библиографических источников (в контексте \textit{Базы знаний IMS.ostis});
	\item выявление синонимии, омонимии, противоречий;
	\item сравнительный анализ и обзор близких внешних публикаций.
\end{scnitemize}

Таким образом, фронт самостоятельных, весьма полезных и посильных для студентов работ весьма широк. Главное сформировать у студентов профессиональный интерес, познавательную активность, инициативность и самостоятельность.
}
\scntext{проектный метод}{Для того, чтобы научиться разрабатывать \textit{интеллектуальные компьютерные системы}, необходимо приобрести достаточно большой опыт участия и \uline{завершения} разработки реально востребованных \textit{интеллектуальных компьютерных систем}}
\scntext{проектный метод}{Для того, чтобы научиться разрабатывать и совершенствовать \textit{технологии искусственного интеллекта}, необходимо приобрести достаточно большой опыт успешного (!) участия в создании различных компонентов комплексной \textit{технологии Искусственного интеллекта}}
\end{SCn}
