\begin{SCn}
	\scnheader{формальная спецификация*}
	\scnidtf{Бинарное ориентированное \textit{отношение}, каждая \textit{пара} которого связывает
		\begin{scnitemize}
			\item некий формальный текст, являющийся формальной спецификацией (формальной моделью, формальным описанием) с
			\item некой сущностью, которая является \uline{объектом} указанной спецификации (моделирования, описания)
	\end{scnitemize}}
	
	\scnsuperset{sc-модель*}
	\scnaddlevel{1}
	\scnidtf{быть формальной спецификацией (формальной моделью, формальным описанием) заданного объекта, представленной на внутреннем смысловом языке интеллектуальных компьютерных систем (в SC-коде)}
	\scnaddlevel{-1}
	
	\scnheader{ostis-система}
	\scnidtf{\textit{интеллектуальная компьютерная система}, построенная по \textit{Технологии OSTIS} (по стандартам \textit{Технологии OSTIS}), что обеспечивает: 
		\begin{scnitemize}
			\item \textit{семантическую совместимость} (взаимопонимание) всех этих систем между собой;
			\item их \textit{способность к самостоятельному взаимодействию}, к координации своей деятельности при возникновении заранее непредсказуемых (нештатных) ситуаций.
	\end{scnitemize}}
	\scnaddlevel{1}
	\scntext{следовательно}{Если \textit{интеллектуальные компьютерные системы} не будут обладать указанными выше способностями, то ни о каких smart-предприятиях, smart-учреждениях, smart-городах, ни о каком smart-обществе и речи быть не может, т.к. все обстоятельства их деятельности заранее на этапе проектирования предусмотреть принципиально невозможно. Это означает, что \textit{интеллектуальные компьютерные системы} должны научиться самостоятельно "отрабатывать"{} все заранее непредусмотренные обстоятельства.}
	\scnaddlevel{-1}
	
	\scnheader{Стандарт OSTIS}
	\scnrelfromlist{основные цели}{
		\scnfileitem{Дать старт открытому проекту перманентной эволюции  \textit{Стандарта OSTIS}};
		\scnfileitem{Обеспечить подготовку молодых специалистов, квалификации которых достаточно для реализации быстрых темпов эволюции \textit{Стандарта OSTIS} путем: 
			\begin{scnitemize}
				\item интеграции (совмещения) \textit{Стандарта OSTIS} с \textit{комплексным учебно-методическим обеспечением} подготовки и повышениях квалификации соответствующих специалистов;
				\item активного подключения студентов, магистрантов и аспирантов к эволюции \textit{Стандарта OSTIS} в рамках учебного процесса.
		\end{scnitemize}};
		\scnfileitem{Существенно повысить качество и темпы эволюции \textit{Стандарта OSTIS} путем создания работоспособного \uline{творческого} коллектива авторов \textit{Стандарта OSTIS}};
		\scnfileitem{Существенно расширить число членов \textit{Авторского Коллектива Стандарта OSTIS} -- специалистов, непосредственно участвующих в реализации и эволюции \textit{Технологии OSTIS} в рамках соответствующего общедоступного открытого проекта}}
	\scnnote{Итак, наша цель -- сформировать \uline{открытый} творческий(!) коллектив, способный развивать \textit{Стандарт OSTIS}. К сожалению, соответствующих прецендентов, мягко говоря, немного. \uline{Коллективно}(!) творить человечество пока не научилось. Очень много откровенной имитации и неадекватности. Но научиться этому надо, ибо другого выхода из современного кризиса \textit{Искусственного интеллекта} нет. Что касается инструментальной поддержки коллективного процесса, то это, не являясь сутью этого процесса, обеспечивает существенное повышение его темпов, что весьма важно. Особо подчеркнем то, что непосредственно коллективный творческий процесс эволюции \textit{Стандарта OSTIS}, как и любой другой коллективный творческий процесс, принципиально(!) не может быть полностью автоматизирован. Автоматизировать согласование различных позиций (точек зрения) \uline{невозможно}(!). Основной проблемой здесь является \uline{иллюзия} человеческого взаимопонимания. Можно только фиксировать факт противоречивости точек зрения, фиксировать аргументы, фиксировать этапы процесса сближения этих точек зрения. Но не более - сам процесс сближения это "ручной"{} процесс, реализуемый исключительно авторами.}
	\scnrelfromlist{направления эволюции}{\scnfileitem{Конструктивное пополнение Документации Технологии OSTIS на уровне исходного текста базы знаний Метасистемы IMS.ostis (продумать при этом и технически поддерживать методики согласования авторских позиций
			\begin{scnitemize}
				\item При четкой фиксации авторских прав и разных точек зрения
				\item Через регулярные коллективные обсуждения - семинары.
		\end{scnitemize}};
		\scnfileitem{Совершенствование формы и \uline{стиля} представления материала};
		\scnfileitem{Совершенствование систематизации (структуризации, стратификации) материала, четкая фиксация наследования свойств};
		\scnfileitem{Четкий формальный сравнительный анализ со всеми ближкими подходами (Semantic Web, графовые базы данных и др.)};\scnfileitem{"Наращивание"{} \textit{Стандарта OSTIS} соответствующей учебно-методической информацией}; 
		\scnfileitem{Интеграция в состав \textit{Стандарта OSTIS} \uline{всех}(!) развиваемых в настоящее время в области \textit{Искусственного интеллекта} моделей представления знаний, моделей решения задач, моделей интерфейсов при обеспечении их \uline{семантической совместимости}. Весь современный арсенал результатов работ в области \textit{Искусственного интеллекта} в перспективе должен быть интегрирован в \textit{Стандарт OSTIS}, но не как мозаика ("зоопарк"{}) имеющихся трудносовместимых моделей, методов и средств построения \textit{интеллектуальных компьютерных систем}, имеющихся сервисов и информационных ресурсов};
		\scnfileitem{Загрузка исходного текста \textit{Стандарта OSTIS} в состав \textit{Базы знаний IMS.ostis} и организация дальнейшей работы по развитию \textit{Стандарта OSTIS} как коллективной разработки Базы знаний IMS.ostis};
		\scnfileitem{Все вопросы по подготовке специалистов в области \textit{Искусственного интеллекта}, а также всю научно-техническую деятельность магистрантов и аспирантов организовать через \textit{Базу знаний IMS.ostis}};
		\scnfileitem{Существенно расширить библиотеки многократно используемых компонентов \textit{ostis-систем}}}
	\scnnote{Фактически речь идет о поэтапном преобразовании современного представления всевозможных информационных ресурсов, различного вида стандартов, энциклопедий, википедий, толковых словарей, учебных пособий и др. в вид строгих формальных моделей, семантически совместимых друг с другом и полностью понятных, готовых к использованию интеллектуальными компьютерными системами. Одно дело -- систематизировать некоторый материал, а другое дело -- его постоянно совершенствовать и повышать эффективность его использования. Практическая ценность неразвиваемых текстов весьма низка из-за быстрого их морального старения.}
\end{SCn}