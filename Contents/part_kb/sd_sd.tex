\begin{SCn}

\scnsectionheader{\currentname}

\scnstartsubstruct

\scnreltovector{конкатенация сегментов}{Что такое предметная область;Роли знаков, входящих в состав предметных областей;Типология предметных областей и отношения, заданных на множестве предметных областей;Что такое sc-язык}

\scnheader{Предметная область предметных областей}
\scnidtf{Предметная область, объектами исследования которой являются предметные области}
\scnexplanation{В состав \textbf{\textit{Предметной области предметных областей}} входят структурные спецификации всех \textit{предметных областей}, входящих в состав базы знаний \textit{ostis-системы}, в том числе, самой \textbf{\textit{Предметной области предметных областей}}. Таким образом, \textbf{\textit{Предметная область предметных областей}} является, во-первых, \textit{рефлексивным множеством}, во-вторых, рефлексивной предметной областью, то есть \textit{предметной областью}, одним из объектов исследования которой является она сама.}
\scniselement{рефлексивное множество}
\scnsdmainclasssingle{предметная область}

\scnsdclass{статическая предметная область;динамическая предметная область;понятие;sc-язык}

\scnsdrelation{понятие предметной области\scnrolesign ;исследуемое понятие\scnrolesign ;максимальный класс объектов исследования\scnrolesign ;немаксимальный класс объектов исследования\scnrolesign ;исследуемый класс первичных элементов\scnrolesign ;исследуемое отношение\scnrolesign ;класс исследуемых структур\scnrolesign ;понятие, исследуемое в дочерней предметной области\scnrolesign ;понятие, исследуемое в материнской предметной области\scnrolesign ;вспомогательное понятие\scnrolesign ;дочерняя предметная область*;дочерняя предметная область по классу первичных элементов*;дочерняя предметная область по исследуемым отношениям*;предметная область sc-языка*}

\scnsegmentheader{Что такое предметная область}

\scnstartsubstruct

\scnheader{предметная область}
\scnidtf{sc-модель предметной области}
\scnidtf{sc-текст предметной области}
\scnidtf{sc-граф предметной области}
\scnidtf{представление предметной области в \textit{SC-коде}}
\scnsubset{знание}
\scnsubset{бесконечное множество}
\scnexplanation{\textbf{\textit{предметная область}} -- это результат интеграции (объединения) частичных семантических окрестностей, описывающих все исследуемые сущности заданного класса и имеющих одинаковый (общий) предмет исследования (то есть один и тот же набор отношений, которым должны принадлежать связки, входящие в состав интегрируемых семантических окрестностей).


\textbf{\textit{предметная область}} -- \textit{структура}, в состав которой входят:
\begin{scnitemize}
\item \textnormal{основные исследуемые (описываемые) объекты -- первичные и вторичные;}
\item \textnormal{различные классы исследуемых объектов;}
\item \textnormal{различные связки, компонентами которых являются исследуемые объекты (как первичные, так и вторичные), а также, возможно, другие такие связки -- то есть связки (как и объекты исследования) могут иметь различный структурный уровень;}
\item \textnormal{различные классы указанных выше связок (то есть отношения);}
\item \textnormal{различные классы объектов, не являющихся ни объектами исследования, ни указанными выше связками, но являющихся компонентами этих связок.}
\end{scnitemize}


При этом все классы, объявленные исследуемыми понятиями, должны быть полностью представлены в рамках данной предметной области вместе со своими элементами, элементами элементов и т.д. вплоть до терминальных элементов.


Можно говорить о типологии \textbf{\textit{предметных областей}} по разным структурным признакам:
\begin{scnitemize}
    \item наличие метасвязей;
    \item наличие исследуемых структур, входящих в состав предметной области;
    \item наличие исследуемых (смежных, дополнительных) объектов, которых исследуются в других предметных областях;
\end{scnitemize}


Понятие \textbf{\textit{предметной области}} является важнейшим методологическим приемом, позволяющим выделить из всего многообразия исследуемого Мира только определенный класс исследуемых сущностей и только определенное семейство отношений, заданных на указанном классе. То есть осуществляется локализация, фокусирование внимания только на этом, абстрагируясь от всего остального исследуемого Мира.


Во всем многообразии \textbf{\textit{предметных областей}} особое место занимают
\begin{scnitemize}
    \item \textit{Предметная область предметных областей}, объектами исследования которой являются всевозможные \textbf{\textit{предметные области}}, а предметом исследования -- всевозможные \textit{ролевые отношения}, связывающие предметные области с их элементами, отношения, связывающие предметные области между собой, отношение, связывающее предметные области с их онтологиями
    \item \textit{Предметная область сущностей}, являющаяся предметной областью самого высокого уровня и задающая базовую семантическую типологию \textit{sc-элементов}(знаков, входящих в тексты \textit{SC-кода})
    \item Семейство \textbf{\textit{предметных областей}}, каждая из которых задает семантику и синтаксис некоторого \textit{sc-языка}, обеспечивающего представление онтологий соответствующего вида (например, \textit{теоретико-множественных онтологий}, \textit{логических онтологий}, \textit{терминологических онтологий}, \textit{онтологий задач и способов их решения} и т.д.)
    \item Семейство \textbf{\textit{предметных областей}} верхнего уровня, в которых классами объектов исследования являются весьма "крупные"{} классы сущностей. К таким классам, в частности
    
    \begin{scnitemizeii}
        \item класс всевозможных \textit{материальных сущностей},
        \item класс всевозможных \textit{множеств},
        \item класс всевозможных \textit{связей},
        \item класс всевозможных \textit{отношений},
        \item класс всевозможных \textit{структур},
        \item класс всевозможных \textit{временных (временно существующих, непостоянных сущностей) сущностей},
        \item класс всевозможных \textit{действий} (акций),
        \item класс всевозможных \textit{параметров} (характеристик),
        \item класс \textit{знаний} всевозможного вида 
        \item и т.п.
    \end{scnitemizeii}
\end{scnitemize}


Каждой \textbf{\textit{предметной области}} можно поставить в соответствие:
\begin{scnitemize}
    \item семейство соответствующих ей \textit{онтологий} разного вида;
    \item некий язык (в нашем случае -- язык, построенный на основе \textit{SC-кода}), тексты которого представляют различные фрагменты соответствующей предметной области
\end{scnitemize}


Указанные языки будем называть \textit{sc-языками}. Их синтаксис и семантика полностью задается \textit{SС-кодом} и \textit{онтологией} соответствующей \textbf{\textit{предметной области}}. Очевидно, что в первую очередь нас должны интересовать те \textit{sc-языки}, которые соответствуют \textbf{\textit{предметным областям}}, имеющим общий (условно говоря, предметно независимый) характер. К таким предметным областям, в частности, относятся:
\begin{scnitemize}
    \item \textit{Предметная область множеств}, описывающая множества и различные связи между ними
    \item \textit{Предметная область отношений и соответствий}
    \item \textit{Предметная область структур} (в частности, графовых)
    \item \textit{Предметная область чисел и числовых структур}
    \item и т.д
\end{scnitemize}


Каждому типу знаний можно поставить в соответствие предметную область, которая является результатом интеграции всех знаний данного типа. Эти знания и становятся объектами исследования в рамках указанной предметной области.


Понятие \textbf{\textit{предметной области}} может рассматриваться как обобщение понятия алгебраической системы. При этом семантическая структура базы знаний может рассматриваться как иерархическая система различных \textbf{\textit{предметных областей}}.
}
\scnidtf{система связей некоторого множества объектов исследования, \uline{ключевыми} элементами которой являются:
	\begin{scnitemize}
	\item классы (точнее, знаки классов) объектов исследования (объектов, описываемых этой предметной областью);
	\item конкретные объекты исследования, обладающие особыми свойствами;
	\item классы связей, входящих в состав рассматриваемой системы -- отношения, заданные на множестве элементов рассматриваемой системы;
	\item параметры, заданные на множестве элементов рассматриваемой системы;
	\item классы структур, являющихся фрагментами рассматриваемой системы.
	\end{scnitemize}}
\scnidtf{структура, представляющая собой множество связей (точнее, знаков связей) и соответствующее множество компонентов этих связей, к числу которых относится:
	\begin{scnitemize}
	\item элементы (экземпляры) некоторых заданных классов \uline{объектов исследования} (первичных исследуемых сущностей);
	\item сами связи, входящие в состав указанной структуры;
	\item введенные классы объектов исследования;
	\item введенные отношения (классы связей);
	\item введенные параметры (классы классов эквивалентных сущностей);
	\item значения параметров (и, в частности, величины для измеряемых параметров);
	\item введенные структуры, являющиеся фрагментами (подструктурами) рассматриваемой структуры;
	\item введенные классы подструктур рассматтриваемой структуры.
	\end{scnitemize}}
\scnnote{Выделяемые в рамках \textit{базы знаний} интеллектуальной системы \textit{предметные области} и соответствующие им \textit{онтологии} -- это, своего рода, семантические страты, кластеры, позволяющие "разложить"{} все хранимые в памяти \textit{знания} по "семантическим полочкам"{} при наличии четких критериев, позволяющих \uline{однозначно} определить то, на какой "полочке"{} должны находиться те или иные \textit{знания}}
\scnnote{Существуют предметные области, в которых основным исследуемым понятием является множество всевозможных связей между экземплярами понятий, исследуемых в других предметных областях. Так, например, можно ввести Предметную область треугольников, Предметную область окружностей, а также Предметную область связей между треугольниками и окружностями.}

\bigskip
\scnendstruct \scnendsegmentcomment{Что такое предметная область}

\scnsegmentheader{Роли знаков, входящих в состав предметной области}

\scnstartsubstruct

\scnheader{роль элемента предметной области}
\scnidtf{ролевое отношения, связывающее предметные области с их ключевыми знаками}
\scnidtf{роль ключевого элемента (знака ключевой сущностей) предметной области}
\scnidtf{роль ключевого знака предметной области}
\scnhaselement{класс объектов исследования\scnrolesign}
\scnhaselement{максимальный класс объектов исследования\scnrolesign}
\scnhaselement{ключевой объект исследования\scnrolesign}
\scnhaselement{понятие, используемое в предметной области\scnrolesign}
\scnhaselement{первичный исследуемый элемент предметной области\scnrolesign}
\scnhaselement{вторичный исследуемый элемент предметной области\scnrolesign}
\scnhaselement{неисследуемый элемент предметной области\scnrolesign}


\scnheader{класс объектов исследования\scnrolesign}
\scnidtf{быть классом \uline{первичных} (для данной предметной области) объектов исследования\scnrolesign}
\scnnote{Понятие \uline{первичного} объекта исследования для предметной области является понятием \uline{относительным} и абсолютно не зависит от типа и уровня сложности этого объекта. Само исследование (спецификация) таких первичных исследуемых объектов осуществляется:
	\begin{scnitemize}
	\item путем введения различных классов объектов исследования, которым эти объекты принадлежат;
	\item путем введения различных связок из первичных объектов исследования и различных классов таких связок (отношений), которым принадлежат введенные связки;
	\item путем введения таких классов первичных объектов исследования, которые являются значениями вводимых параметров;
	\item путем введения различных структур, состоящих из первичных объектов исследования, из связок таких объектов, из введенных отношений и классов первичных объектов, из введенных параметров и значений этих параметров, и путем введения различных классов таких структур;
	\item путем введения различных связок из вторичных объектов исследования (т.е. из связок и структур) и путем введения различных классов таких связок;
	\item и далее можно переходить к объектам исследования более высокого уровня сложности, к параметрам, элементами значений которых являются такие объекты, а также к структурам, элементами которых являются объекты такого уровня и, соответственно, к классам таких структур.
	\end{scnitemize}}

\scnrelfrom{второй домен}{класс}
\scnaddlevel{1}
	\scnsuperset{\scnmakesetlocal{множество
			;отношение\\
			\scnaddlevel{1}
			\scnsubset{множество}
			\scnaddlevel{-1}
			;параметр\\
			\scnaddlevel{1}
			\scnsubset{класс классов}
			\scnaddlevel{-1}
			;значение параметра\\
			\scnaddlevel{1}
			\scnsubset{класс}
			\scnaddlevel{-1}
			;структура\\
			\scnaddlevel{1}
			\scnsubset{множество}
			\scnaddlevel{-1}
			;темпоральная сущность
			;темпоральная сущность базы знаний ostis-системы
			;семантическая окрестность
			;предметная область
			;онтология
			;логическая формула
			;действие
			;задача
			;информационная конструкция
			;язык
			;sc-конструкция
			;кибернетическая система
			;интеллектуальная компьютерная система
			;знание
			;база знаний
			;решатель задач интеллектуальной компьютерной системы
			;интерфейс интеллектуальной компьютерной системы
			;компьютерная система, основанная на смысловом представлении информации
			;смысловое представление информации
			;многоагентная модель решения задач, основанная на смысловом представлении информации
			;логико-семантическая модель интерфейсов компьютерных систем, основанных на смысловом представлении информации
			;решатель задач ostis-системы
			;действие, выполняемое ostis-системой
			;задача, решаемая ostis-системой
			:план решения задачи, реализуемый ostis-системой
			;протокол решения задачи, реализованный ostis-системой
			;метод решения класса задач, реализуемый ostis-системой
			;sc-агент\\
			\scnaddlevel{1}
			\scnidtf{внутренний агент ostis-системы, осуществляющий выполнение некоторого вида действий в памяти ostis-системы}
			\scnsuperset{sc-агент обработки информации в памяти ostis-системы}
			\scnsuperset{sc-агент управления внешними действиями ostis-системы}
			\scnaddlevel{-1}
			;Базовый язык программирования ostis-систем\\
			\scnaddlevel{1}
			\scnidtf{Язык SCP}	
			\scnaddlevel{-1}
			;искусственная нейронная сеть
			;интерфейс ostis-системы
			;интерфейсное действие пользователя ostis-системы
			;sc-агент интерфейса ostis-системы
			;естественный язык
			;базовый интерпретатор логико-семантических моделей ostis-систем
			;базовый интерпретатор логико-семантических моделей ostis-систем, реализованный программно на современных компьютерах
			;семантический ассоциативный компьютер
			;обучение пользователей ostis-систем
			;ostis-система персональной адаптивной поддержки всех видов деятельности пользователя
			;ostis-система управления рецептурным производством
			;ostis-система, реализующая интеллектуальный портал научно-технических знаний}}
		\scnaddlevel{1}
\scnnote{Здесь приведено семейство тех \textit{классов объектов исследования}, для которых в текущей версии \textit{Стандарта OSTIS} представлены соответствующие \textit{предметные области}. Очевидно, что это семейство должно быть существенно расширено и включить в себя, например, такие \textit{классы} сущностей, как:
	\begin{scnitemize}
	\item материальная сущность
	\item вещество
	\item физическое поле
	\item персона
	\item пространственная сущность
	\item юридическое лицо
	\item предприятие
	\item географический объект
	\item и многие другие
	\end{scnitemize}}
\scnaddlevel{-2}
\scnnote{Особого внимания требуют те \textit{классы объектов исследования}, которые носят наиболее общий характер  которым соответствуют \textit{предметные области и онтологии} \uline{высокого уровня}. Здесь важна продуманная система декомпозиции всего множества окружающих нас сущностей на иерархическую систему \textit{классов объектов исследования}, которой соответствует иерархическая система \textit{предметных областей и онтологий}, определяющая направления \uline{наследования свойств} исследуемых объектов.}

\scnheader{максимальный класс объектов исследования\scnrolesign}
\scnidtf{класс объектов исследования, для которого \uline{в заданной} (!) предметной области отсутствует другой класс объектов исследования, который был бы его надмножеством\scnrolesign}
\scnnote{В некоторых предметных областях может быть \uline{несколько} максимальных классов объектов исследования}


\scnheader{ключевой объект исследования\scnrolesign}
\scnidtf{особый объект исследования\scnrolesign}
\scnidtf{быть знаком особого исследуемого объекта в рамках заданной предметной области\scnrolesign}
\scnidtf{объект исследования, обладающий особыми свойствами\scnrolesign}
\scnhaselementrole{пример}{$\langle$Предметная область чисел; Нуль$\rangle$}
	\scnaddlevel{1}
	\scnnote{Особыми свойствами Числа \textit{Нуль} являются:
		\begin{scnitemize}
		\item Результатом сложения Числа \textbf{\textit{Нуль}} с любым числом \textbf{\textit{x}} является число \textbf{\textit{x}};
		\item Результатом умножения Числа \textbf{\textit{Нуль}} на любое число является Число \textbf{\textit{Нуль}}
		\end{scnitemize}}
	\scnaddlevel{-1}
\scnhaselement{$\langle$Предметная область чисел; Единица$\rangle$}
\scnhaselement{$\langle$Предметная область чисел; Число Пи$\rangle$}
\scnhaselement{$\langle$Предметная область чисел; Число Е$\rangle$}

\scnheader{ключевой элемент предметной области\scnrolesign}
\scnidtf{входящий в состав предметной области знак ключевой сущности\scnrolesign}
\scnsubdividing{понятие, используемое в предметной области\scnrolesign
;ключевой объект исследования\scnrolesign \\
	\scnaddlevel{1}
	\scnidtf{знак ключевого объекта исследования\scnrolesign}
	\scnaddlevel{-1}}


\scnheader{понятие, используемое в предметной области\scnrolesign}
\scnidtf{понятие, используемое в заданной предметной области не в качестве одного из объектов исследования, а в качестве \uline{ключевого} понятия\scnrolesign}
\scnsubset{используемое понятие\scnrolesign}
	\scnaddlevel{1}
	\scnidtf{понятие, используемое в sc-знании\scnrolesign}
	\scnsubset{используемое понятие*}
		\scnaddlevel{1}
		\scnidtf{понятие, используемое в знании, которое может быть представлено не только в SC-коде*}
		\scnaddlevel{-1}
	\scnaddlevel{-1}
\scnnote{Уточнение характера использования понятия в предментной области осуществляется по трем признакам:
	\begin{scnitemize}
	\item семантический тип используемого понятия;
	\item полнота вхождения элементов понятия в данную предметную область;
	\item наличие первого упоминания понятия;
	\item наличие определения понятия или объявления его неопределяемостис подробным пояснением и примерами;
	\item наличие исследования понятия.	
	\end{scnitemize}}
\scnrelfrom{разбиение}{семантический тип используемого понятия}
	\scnaddlevel{1}
	\scneqtoset{класс объектов исследования\scnrolesign
;отношение, используемое в предметной области\scnrolesign
;параметр, используемый в предметной области\scnrolesign
;класс структур, используемый в предметной области\scnrolesign}
	\scnaddlevel{-1}
\scnrelfrom{разбиение}{полнота вхождения элементов понятия в данную предметную область}
	\scnaddlevel{1}
	\scneqtoset{используемое понятие, все элементы которого входят в данную предметную область\scnrolesign \\
	\scnaddlevel{1}
	\scnnote{Для каждого используемого отношения в предметную область здесь должны входить не только знаки связок, но и все связки целиком с их компонентами}
	\scnaddlevel{-1}
;используемое понятие, не все элементы которого входят в данную предметную область\scnrolesign}
	\scnaddlevel{-1}
\scnrelfrom{разбиение}{наличие первого упоминания понятия}
	\scnaddlevel{1}
	\scneqtoset{понятие, вводимое в данной предметной области\scnrolesign
;понятие, которое в данной предметной области используется, но не вводится\scnrolesign}
	\scnaddlevel{1}
	\scnnote{Будем считать, что понятие вводится в данной предметной области в том и только в том случае, если ни в одной предметной области более высокого уровня это понятие не используется. Т.е. речь идет о первом упоминании этого понятия в рамках последовательности предметных областей от родительских к дочерним}
	\scnaddlevel{-1}
	\scnaddlevel{-1}
\scnrelfrom{разбиение}{наличие определения понятия или объявления его неопределяемости с подробным пояснением и примерами}
	\scnaddlevel{1}
	\scneqtoset{понятие, которое в данной предметной области определено или объявлено как неопределяемое
;понятие, которое в данной предметной области не имеет ни определения, ни указания факта его неопределяемости}
	\scnaddlevel{-1}
\scnrelfrom{разбиение}{наличие исследования понятия}
	\scnaddlevel{1}
	\scneqtoset{понятие, исследуемое в данной предметной области\scnrolesign
;понятие, которое в данной предметной области испольуется, но не исследуется\scnrolesign}
	\scnaddlevel{-1}
\scnnote{Понятие, используемое в базе знаний, может быть введено (впервые упомянуто) в одной предметной области, определено в другой, а исследоваться -- в третьей}


\scnheader{первичный исследуемый элемент предметной области\scnrolesign}
\scnidtf{знак первичного объекта исследования в рамках заданной предметной области\scnrolesign}


\scnheader{вторичный исследуемый элемент предметной области\scnrolesign}
\scnidtf{знак вторичного объекта исследования в рамках предметной области\scnrolesign}
\scnsuperset{связка элементов предметной области\scnrolesign}
	\scnaddlevel{1}
\scnsuperset{связка первичных элементов предметной области\scnrolesign}
\scnsuperset{метасвязка элементов предметной области\scnrolesign}
	\scnaddlevel{1}
\scnsuperset{метасвязка, в число компонентов которой входят связки элементов предметной области\scnrolesign}
\scnsuperset{метасвязка, в число компонентов которой входят классы элементов предметной области\scnrolesign}
\scnsuperset{метасвязка, в число компонентов которой входят структуры элементов предметной области\scnrolesign}
	\scnaddlevel{-1}
	\scnaddlevel{-1}
\scnsuperset{класс элементов предметной области\scnrolesign}
		\scnaddlevel{1}
\scnsuperset{класс первичных элементов предметной области\scnrolesign}
\scnsuperset{класс связок элементов предметной области\scnrolesign}
\scnsuperset{класс классов элементов предметной области\scnrolesign}
\scnsuperset{класс структур элементов предметной области\scnrolesign}
	\scnaddlevel{-1}
\scnsuperset{структура элементов предметной области\scnrolesign}
		\scnaddlevel{1}
\scnsuperset{тривиальная структура первичных элементов предметной области\scnrolesign}
\scnsuperset{структура, в число подмножеств которой входят связки элементов предметной области вместе со своими компонентами\scnrolesign}
\scnsuperset{структура, в число подмножеств которой входят классы элементов предметной области вместе со своими знаками\scnrolesign}
\scnsuperset{структура, в число подмножеств которой входят другие структуры вместе со своими знаками\scnrolesign}
	\scnaddlevel{-1}


\scnheader{неисследуемый элемент предметной области\scnrolesign}
\scnidtf{вспомогательный элемент предметной области, исследуемый в другой (смежной) предметной области\scnrolesign}
\scnnote{С помощью неисследуемых элементов предметной области описываются и исследуются различные вида связи между элементами, исследуемыми в данной \textit{предметной области} с элементами, исследуемыми в других \textit{предметных областях}. При этом \textit{связки}, компонентами которых являются как исследуемые, так и неисследуемые элементы данной \textit{предметной области} считаются \uline{исследуемыми} связками этой \textit{предметной области}. Примерами неисследуемых элементов, напримр, геометрической \textit{предметной области} являются \textit{числа}, являющиеся \textit{значениями величин} таких \textit{параметров}, как \textit{расстояние}\scnsupergroupsign, \textit{длина}\scnsupergroupsign, \textit{площадь}\scnsupergroupsign, \textit{объем}\scnsupergroupsign, а также различные числовые \textit{отношения} (\textit{сложение}*, \textit{умножение}*, \textit{возведение в степень}*), теоретико-множественные \textit{отношения} (\textit{включение}*, \textit{объединение}*, \textit{пересечение}*, \textit{принадлежность}*)}

\newpage
\scnheader{понятие}
\scnidtf{концепт}
\scnidtf{класс сущностей, который входит в состав по крайней мере одной предметной области в качестве (в роли) ключевого исследуемого понятия}
\scnnote{Семейство всех введенных понятий -- это, своего рода, семантическая система координат, позволяющая специфицировать всевозможные сущности в смысловом пространстве.}
\scnidtf{класс сущностей, который по крайней мере в одной \textit{предметной области} "объявлен"{} как \textit{понятие} (вводимое, исследуемое или вспомогательное)}
\scnnote{Каждому \textit{понятию} соответствует по крайней мере одна \textit{предметная область}, в которой это понятие является \textit{исследуемым понятием} и в которой рассматриваются основные характеристики этого \textit{понятия}. Если же в какой-либо \textit{предметной области} необходимо рассмотреть дополнительные связи этого \textit{понятия} с другими \textit{понятиями}, то оно объявляется как \textit{вспомогательное понятия}\scnrolesign .}
\scnidtf{Второй домен Отношения \textit{используемое понятие}*}
\scnrelto{второй домен}{используемое понятие*}
\scnidtf{класс сущностей (класс связок (в т.ч. отношение), класс классов (в т.ч. параметр), класс структур), который по крайней мере в одной \textit{предметной области} является \textit{используемым понятием}\scnrolesign}

\bigskip
\scnendstruct \scnendsegmentcomment{Роли знаков, входящих в состав предметной области}

\scnsegmentheader{Типология предметных областей и отношения, заданные на множестве предметных областей}
\scnstartsubstruct

\scnheader{предметная область}
\scnsubdividing{статическая предметная область\\
\scnaddlevel{1}
\scnidtf{стационарная предметная область}
\scnidtf{\textit{предметная область}, в которой связи между сущностями, входящими в ее состав, не зависят от времени (не меняются во времени), элементами \textbf{\textit{статической предметной области}} не могут быть \textit{временные сущности}}
\scnaddlevel{-1}
;квазистатическая предметная область\\
\scnaddlevel{1}
\scnidtf{\textit{предметная область}, решение задач в которой не требует учета темпоральных свойств объектов исследования} 
\scnaddlevel{-1}
;динамическая предметная область\\
\scnaddlevel{1}
\scnidtf{нестационарная предметная область} 
\scnidtf{\textit{предметная область}, которая описывает изменение состояния (в том числе внутренней структуры) объектов исследования и/или изменение конфигурации связей между объектами исследования} 
\scnidtf{\textit{предметная область}, в которой некоторые связи между сущностями, входящими в ее состав, меняются со временем (то есть носят ситуационный, нестационарный характер, другими словами, являются \textit{временными сущностями})} 
\scnaddlevel{-1}
}

\scnsubdividing{первичная предметная область\\
\scnaddlevel{1}
\scnidtf{\textit{предметная область}, объектами исследования которой являются \uline{внешние} сущности (обозначаемые первичными \textit{sc-элементами})}
\scnaddlevel{-1}
;вторичная предметная область\\
\scnaddlevel{1}
\scnidtf{метапредметная область} 
\scnidtf{\textit{предметная область}, объектами исследования которой являются \textit{sc-множества} (отношения, параметры, структуры, классы структур, знания, языки и др.)} 
\scnaddlevel{-1}
}

\scnnote{Во всем многообразии предметных областей \uline{особое} местро занимают:
\begin{scnitemize}
		\item \textbf{\textit{Предметная область предметных областей}}, объектами исследования которой являются всевозможные предметные области, а предметом исследования являются -- всевозможные ролевые отношения, связывающие предметные области с их элементами, отношения, связывающие предметные области между собой, отношение, связывающее предметные области с их онтологиями;
		\item \textbf{\textit{Предметная область сущностей}}, являющаяся предметной областью самого высокого уровня и задающая базовую семантическую типологию sc-элементов (знаков, входящих в тексты SC-кода);
		\item Семейство \textit{предметных областей}, каждая из которых задает семантику и синтаксис некоторого \textit{sc-языка}, обеспечивающего представление \textit{\uline{онтологий}} соответствующего вида (например, теоретико множественных онтологий терминологических онтологий);
		\item Семейство \textit{предметных областей} \uline{верхнего уровня}, в которых классами объектов исследования являются весьма "крупные"{} классы сущностей. К таким классам, в частности, относятся: 
		\begin{scnitemizeii}
			\item класс всевозможных материальных сущностей,
			\item класс всевозможных множеств,
			\item класс всевозможных связей,
			\item класс всевозможных отношений,
			\item класс всевозможных структур,
			\item класс всевозможных темпоральных (нестационарных) сущностей,
			\item класс всевозможных действий (воздествий, акций),
			\item класс всевозможных параметров (характеристик),
			\item класс знаний всевозможного вида и т.п.;
		\end{scnitemizeii}
		\item Предметные области абстрактных пространств (в том числе предметные области метрических пространств). Примерами абстрактного пространства являются Евклидово пространство геометрических точек и фигур, пространство всевозможных множеств, числовое пространство, SC-пространство (унифицированное смысловое пространство знаков всевозможных сущностей).
\end{scnitemize}
}

\scnheader{отношение, заданное на множестве предметных областей}
\scnhaselement{\scnkeyword{дочерняя предметная область*}}
\scnaddlevel{1}
\scnidtf{частная предметная область*}
\scnidtf{быть частной предметной областью*}
\scnidtf{близлежащий потомок предметной области*}
\scnidtf{сужение предметной области по классу объектов исследования*}
\scnidtf{предметная область, детализирующая описание одного из классов объектов исследования другой (более общей) предметной области*}
\scnidtf{предметная область, объединение классов объектов исследования которой является подмножеством объединения классов объектов исследования заданной предметной области*}
\scniselement{бинарное отношение}
\scniselement{ориентированное отношение}
\scniselement{неролевое отношение}
\scnsuperset{частная предметная область по классу первичных элементов*}
\scnsuperset{частная предметная область по исследуемым отношениям*}
\scnexplanation{\textit{дочерняя предметная область*} -- бинарное ориентированное отношение, с помощью которого задается иерархия предметных областей путем перехода от менее детального к более детальному рассмотрению соответствующих исследуемых понятий.}
\scnnote{Для любой \textit{предметной области} все свойства ее \textit{объектов исследования} \uline{наследуются} всеми ее \textit{дочерними предметными областями*}.}
\scnaddlevel{-1}
\scnhaselement{\scnkeyword{интеграция предметных областей*}}
\scnaddlevel{1}
\scnidtf{Отношение, связывающее заданное семейство предметных областей с предметной областью, которая является результатом их интеграции (это не только теоретико-множественное объединение заданных предметных областей, но и уточнение ролей ключевых понятий в интегрированной предметной области, поскольку одно и то же понятие в интегрируемых предметных областях может иметь разные роли).}
\scnaddlevel{-1}
\scnhaselement{\scnkeyword{изоморфность предметных областей*}}
\scnhaselement{\scnkeyword{гомоморфность предметных областей*}}

\scnheader{расширение семейства исследуемых отношений*}
\scnexplanation{Переход от одной предметной области к предметной области с тем же максимальным классомобъектов исследования, но с расширенным семейством отношений и, возможно, с расширенным семейством явно выделенных классов объектов исследования (подклассов максимального класса).}

\scnheader{переход к рассмотрению внутренней структуры объектов исследования*}
\scnexplanation{Переход от рассмотрения внешних связей объектов исследования к рассмотрению их "внутренней"{} структуры путем декомпозиции исследуемых объектов на части и путем включения в число исследуемых объектов тех, которые являются указанными частями.}

\scnheader{переход к рассмотрению структур из объектов исследования*}
\scnexplanation{Переход от описаниязаданного класса исследуемых объектов к описанию класса всевозможных множеств, элементами которых являются указанные объекты (например, переход от предметной области геометрических точек к предметной области геометрических фигур).}

\bigskip

\scnendstruct \scnendsegmentcomment{Типология предметных областей и отношения, заданные на множестве предметных областей}

\scnsegmentheader{Что такое sc-язык}
\scnstartsubstruct

\scnheader{sc-язык}
\scnidtf{максимальное множество текстов SC-кода, являющихся фрагментами соответствующей предметной области (точнее, ее sc-модели)}
\scnexplanation{\textbf{\textit{sc-язык}} --- это подъязык (подмножество) \textit{SC-кода}, ориентированный на представление \textit{sc-текстов}, являющихся фрагментами некоторой \textit{предметной области}. Таким образом, каждому \textbf{\textit{sc-языку}} взаимно однозначно соответствует некоторая \textit{предметная область} (точнее, sc-модель некоторой \textit{предметной области}).}

\scnheader{предметная область sc-языка*}
\scnidtf{предметная область заданного sc-языка*}
\scnidtf{быть предметной областью соответствующей заданному sc-языку*}
\scnidtf{sc-язык и соответствующая ему предметная область*}
\scniselement{бинарное отношение}
\scnexplanation{\textbf{\textit{предметная область sc-языка*}} - это бинарное ориентированное отношение, каждая связка которого связывает знак некоторого \textit{sc-языка} (первый компонент связки данного отношения) и знак соответствующей этому \textbf{\textit{sc-языку}} \textit{предметной области}.
}
\scnrelfrom{первый домен}{sc-язык}
\scnrelfrom{второй домен}{предметная область}

\scnheader{sc-язык}
\scnidtf{подъязык \textit{SC-кода}}
\scnexplanation{Множество языков, синтаксис каждого из которых полностью соответствует синтаксису \textit{SC-кода} (т.е. каждый из них является подъязыком SC-кода), а денотационная семантика каждого из них определяется интегральной (объединенной) онтологией  \textit{предметной области}, которая взаимно однозначно соответствует этому sc-языку.}
\scnnote{Каждой предметной области можно поставить в соответствие множество sc-текстов, которые являются фрагментами sc-моделей этой \textit{предметной области}, описывающими (специфицирующими) свойства объектов, исследуемых в указанной \textit{предметной области}. Указанное множество \textit{sc-текстов} будем называть \textit{sc-языком}, соответствующим указанной \textit{предметной области}. Очевидно, что \textit{предметных областей} и соответсвующих им \textit{sc-языков} существует неограниченное количество. Синтаксис и семантика \textit{sc-языков} полностью задается \textit{SC-кодом} и онтологией соответсвующей \textit{предметной области}. Очевидно, что в первую очередь нас должны интересовать те \textit{sc-языки}, которые соответствуют предметным областям, имеющим общий (условно говоря, предметно независимый) характер. К таким предметным областям, в частности, относятся:
\begin{scnitemize}
		\item \textit{Предметная область множеств}, описывающая множества и различные связи между ними
		\item \textit{Предметная область структур}
		\item \textit{Предметная область чисел и числовых структур}
		\item \textit{Предметная область отношений}
		\item \textit{Предметная область параметров, величин и шкал}
		\item \textit{Предметная область логических формул, высказываний и формальных теорий}
		\item и другие
\end{scnitemize}}

\scnnote{Особое место для всего семейства sc-языков занимают две предметные области и соответствующие им онтологии:
\begin{scnitemize}
		\item Предметная область SC-кода и онтология, определяющая его базовую денотационную семантику, задаваемую синтаксисом SC-кода
		\item Предметная область и онтология Базового универсального sc-языка
\end{scnitemize}}
\scnnote{Подчеркнем, что все sc-языки, кроме тех, которые задаются указанными двумя предметными областями и соответсвующими им онтологиями, являются \uline{специализированными} sc-языками.
}

\scnheader{Предметная область SC-кода и онтология, определяющая его базовую денотационную семантику, задаваемую синтаксисом SC-кода}
\scnexplanation{Указанная онтология SC-кода определяет денотационную семантику SC-кода путем семантической интерпретации
\uline{синтаксически} выделяемых классов sc-элементов. Т.е. данная денотационная семантика уточняет смысл \uline{только} тех классов sc-элементов, которые задаются \uline{синтаксически} путем "приписывания"{} sc-элементам соответствующей "синтаксической метки"{}.}

\scnheader{Предметная область и онтология Базового универсального sc-языка}
\scnexplanation{Указанная онтология Базового универсального sc-языка определяет денотационную семантику этого Базового универсального sc-языка и представляет собой онтологию \uline{верхнего уровня}, которая уточняет смысл понятий, лежащий в основе указанного sc-языка. В эту онтологию входит уточнение таких понятий, как материальная сущность, абстрактная сущность, число, пространство, множество, связь, класс, отношение, параметр, структура, класс структур, информационная конструкция, знание, статическая сущность, динамическая сущность, процесс, ситуация, действие, постоянная сущность, временная сущность, часть*, декомпозиция*, пространственная часть*, темпоральная часть* и т.д.}

\bigskip

\scnendstruct \scnendsegmentcomment{Что такое sc-язык}

\bigskip

\scnendstruct \scnendcurrentsectioncomment

\end{SCn}