\scseparatedfragment{Редакционная коллегия Cтандарта OSTIS}
\begin{SCn}

	\scnheader{Состав редакционной коллегии Стандарта OSTIS}
	\scneqtoset{Голенков В.В.;
		Головко В.А.;
		Гулякина Н.А.;
		Краснопрошин В.В.;
		Курбацкий А.Н.;
		Гордей А.Н.;
		Шункевич Д.В.;
		Азаров И.С.;
		Захарьев В.А.;
		Родченко В.Г.;
		Голубева О.В.;
		Кобринский Б.А.;
		Борисов В.В.;
		Аверкин А.Н.;
		Кузнецов О.П.;
		Козлова Е.И.;
		Гернявский А.Ф.;
		Таранчук В.Б.;
		Ростовцев В.Н.;
		Витязь С.П.
	}

	\scnidtf{Редколлегия Стандарта OSTIS}
	\scnrelfromset{направления и принципы организации деятельности}{
		\scnfileitem{Обеспечение целостности и повышения качества постоянно развиваемой (совершенствуемой) Технологии OSTIS, а также достаточно точное описание (документирование) каждой текущей версии этой технологии.};
		\scnfileitem{Обеспечение чёткого контроля совместимости версий Технологии OSTIS в целом, а также версий различных компонентов этой технологии.};
		\scnfileitem{Постоянное уточнение степени важности различных направлений развития Технологии OSTIS для каждого текущего момента.};
		\scnfileitem{Формирование и постоянное уточнение плана тактического и стратегического развития самой \textit{Технологии OSTIS}, а также полной документации этой Технологии в виде \textit{Стандарта OSTIS}. Подчеркнем при этом, что указанная документация является неотъемлемой частью \textit{Технологии OSTIS}.}
	}

	\scnrelfromset{направление деятельности}{
		\scnfileitem{Принципы организации деятельности \textit{Редколлегии Стандарта OSTIS}
		\begin{scnitemize}
			\item Продуктом является база знаний, а также ежегодно издаваемые коллективные монографии;
			\item Все рецензируют всё;
			\item Но по каждому разделу есть ответственный редактор;
			\item Согласование (рецензирование) дополнений/изменений осуществляется по следующим принципам:
				\begin{scnitemizeii}
					\item Либо формируется рецензия с замечаниями по обязательной доработке и пожеланиями;
					\item Либо фиксируется согласие с предлагаемыми изменениями;
					\item При этом каждый рецензент должен подтвердить устранение своих обязательных замечаний;
					\item Формируется и учитывается рейтинг рецензентов-экспертов;
					\item Предложение принимается автоматически по формуле, учитывающей рейтинг и количество согласий/рецензий.
				\end{scnitemizeii}
				\scnaddlevel{-2}
		}
	}

\end{SCn}
